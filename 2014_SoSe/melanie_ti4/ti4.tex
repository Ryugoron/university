\documentclass[11pt,a4paper,ngerman]{article}
\usepackage[bottom=2.5cm,top=2.5cm]{geometry}
\usepackage[ngerman]{babel}
\usepackage[utf8]{inputenc}
\usepackage[T1]{fontenc}
\usepackage{ae}
\usepackage{amssymb}
\usepackage{amsmath}
\usepackage{amsthm}
\usepackage{graphicx}
\usepackage{fancyhdr}
\usepackage{fancyref}
\usepackage{listings}
\usepackage{xcolor}
\usepackage{stmaryrd}
\usepackage{paralist}
\usepackage{tikz}
\usepackage{amsthm}
\usepackage{tabularx}

\usetikzlibrary{arrows,automata,shapes.geometric}

\newtheorem{propo}{Satz}
\newtheorem{lemmas}[propo]{Lemma}

\usepackage[pdftex, bookmarks=false, pdfstartview={FitH}, linkbordercolor=white]{hyperref}
\usepackage{fancyhdr}
\pagestyle{fancy}
\fancyhead[C]{Ti IV}
\fancyhead[L]{Protokoll}
\fancyhead[R]{SoSe 2014}
\fancyfoot{}
\fancyfoot[L]{}
\fancyfoot[C]{\thepage \hspace{1px} von \pageref{LastPage}}
\renewcommand{\footrulewidth}{0.5pt}
\renewcommand{\headrulewidth}{0.5pt}
\newcommand{\set}[1]{ \{ #1 \}}
\setlength{\parindent}{0pt}
\setlength{\headheight}{0pt}

\newcommand{\N}{\mathbb{N}}
\newcommand{\Q}{\mathbb{Q}}
\newcommand{\R}{\mathbb{R}}
\newcommand{\bigO}{\mathcal{O}}
\newcommand{\Rarr}{\Rightarrow}
\newcommand{\rarr}{\rightarrow}
\newcommand{\Pot}{\mathcal{P}}
\newcommand{\abs}[1]{\left |#1\right|}
\newcommand{\solved}{$\mbox{}$ \hfill $\square$}
\newcommand{\Epsilon}{\mathcal{E}}

\newcommand{\erw}[1]{\text{\bfseries E} \left[ #1 \right]}
\newcommand{\prob}[1]{\text{Pr}\left[ #1 \right]}

\date{}
\title{Protokoll 1}
\author{Melanie Skodzik}


%%
%% Enviroments for proofs and lemmas
%%
\newtheorem{prop}{\bfseries Behauptung}
\newtheorem{lemma}{\bfseries Lemma}

\begin{document}

\definecolor{darkblue}{rgb}{0,0,.6}
\definecolor{darkgreen}{rgb}{0,0.5,0}
\definecolor{darkred}{rgb}{0.5,0,0}
\definecolor{normblack}{rgb}{0,0,0}

\lstdefinestyle{moreC}{deletekeywords={return}, morekeywords={assert, define, subrange, subtype, int, bool, ::, check, set-evidence! true}}

\lstset{language=C, basicstyle=\ttfamily\fontsize{10pt}{10pt}\selectfont\upshape, commentstyle=\color{darkgreen}\sffamily, keywordstyle=\color{darkblue}\rmfamily\bfseries, breaklines=true, frame=single, xleftmargin=3mm, xrightmargin=3mm, tabsize=2, mathescape=true, showstringspaces=false, stringstyle=\color{darkred}, escapeinside={\%*}{*)}}

\lstset{literate=%
{\$}{\$}1
{\_}{\textunderscore}1
}
\lstset{extendedchars=\true}
\lstset{inputencoding=utf8}



\renewcommand{\figurename}{Grafik}

\maketitle
\thispagestyle{fancy}


\subsection*{Aufgabe 1}

Ein Beispiel Quellcode.


%%
%% Code Beispiel
%%
\begin{lstlisting}
#include <std.io>

int main(int argc, char *argv[]){
   fprintf("%s\n","Hello World!");
}
\end{lstlisting}


%%
%% Frame Beispiel
%%
\fbox{
   \begin{minipage}[l]{0.9\textwidth}
      at+pb 2
   
      030 12345678
   \end{minipage}
}

%%
%% Begriffe Erklären
%%

\begin{description}
   \item[at+pb:] Hier steht ein sinnvoller Text über das \emph{pb} Kommando.
   \item[at+ch:] Noch etwas tolles. Nur ist das Item fett gedruckt. Das ist ein wirklich langer Text, den ich gar nicht mehr aufhären will zu schreiben, weil er so schön ist.\\
      Hallo\\
      Du\\
      <3
   \item[at+x:] \mbox{}\\
      Hallo\\
      weiter\\
      ja
\end{description}

\label{LastPage}
\end{document}
