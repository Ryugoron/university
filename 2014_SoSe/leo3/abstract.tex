\documentclass[10pt]{article}

\usepackage[top=1cm, left=3cm, right=2cm]{geometry}

\begin{document}
\title{The Leo-III Project}
\author{Christoph Benzm\"{u}ller \and Max Wisniewski \and Alexander Steen}
\date{}
\maketitle


\noindent We present the Leo-III project, in which we design and implement
a state-of-the-art Higher-Order Logic Theorem Prover, the successor of the well known LEO-II prover~\cite{Benzmüller08leo-ii—} (Winner CASC 2010). Leo-III will be based on
ordered paramodulation/superposition.

In constract to LEO-II, we replace the internal term representation
(the commonly used simply typed lambda calculus) by a more expressive system supporting
type polymorphism. In the course of the project, we plan to further enhance the type
system with type classes and type constructors similar to System F$^\omega$.
This way, Leo-III will be one of few ATPs to support all TPTP input formats up to
TFF1~\cite{Sutcliffe:2009:TPL:1666192.1666217}.

In order to achieve a substantial performance speed-up, the architecture of Leo-III
will be based on massive parallelism (e.g. And/Or-Parallelism, Multisearch)~\cite{journals/amai/Bonacina00}.
The current design is a multi-agent blackboard
architecture~\cite{Weiss2013} that will allow to independently run agents with
our proof calculus as well as agents for external (specialized) provers.

\noindent The design of Leo-III will focus right from the start on compatibility to the widely used TPTP infrastructure
~\cite{Sutcliffe:2009:TPL:1666192.1666217} including its input language. Also, we plan to offer built-in support for specialized
external prover agents, as well as to provide external interfaces to be, itself, be used
from other provers (e.g. Isabelle/HOL~\cite{nipkow2002isabelle}).
The implementation will excessively use term sharing~\cite{Riazanov:2002:DIV:1218615.1218620}~\cite{Schulz:2002:EBT:1218615.1218621}
and several indexing techniques~\cite{Nieuwenhuis03onthe}~\cite{W35}.

\nocite{Sutcliffe10automatedreasoning}

\bibliographystyle{alpha}
\bibliography{cite}

\end{document}


Quellen?

%%
%% Points :
%%    - Leo2 nachfolger                         <- muss rein X
%%    - Higher Order Logic Theorem Prover
%%       -> Typeclasses
%%       -> Polymorphism                          <- muss rein X
%%       -> (also quasi richly typed formal system)
%%       -> Wie keiner vorher war.
%%       -> term/data sharing, labeling techniques                        <- muss rein X
%%       -> efficient reduction/substitution (z.b. durch ordering)                        <- muss rein X
%%    - Highly Concurrent
%%       -> Blackboard Architecture                        <- muss rein X
%%       -> Mutli-Agent Architecture                        <- muss rein X
%%       -> And/Or - Parallelismus                        <- muss reinX
%%       -> Distributed Search (Viele Kalkülagenten, die versuchen
%%             inferenz anzuwenden)
%%       -> Vorschlags oder Unterstützende Agenten
%%             ( Mögliche Unifikationspartner, Mögliche Unifier,
%%                Klauselnormalisierung, etc.)
%%    - State of the art                        <- muss rein X
%%    - Nah an TPTP entwickeln; kompatible Syntax                        <- muss reinX
%%    - Leicht anschließbar an ext. Systeme                        <- muss reinX
%%          ("Provision of proof objects and support for the integration
%%           of LEO-III with proof assistants or other AI systems.")
%%    - auch was mit special logics? also z.b. QML (modal logics)
%%    - Modularer Aufbau um es als Framework zu nutzen (relativ wichtig)                        <- muss rein
%%
