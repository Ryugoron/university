\begin{abstract}
The successor of Leo2, a higher order theorem prover with term-order (winner CASC 2010),
is a Prover based on ordered paramodulation/superposition. The underlying logic
uses an erniched higher-order calculus with type polymorphism and type classes. The whole system is designed to work fully concurrent/parallel and ...

wo sind wir gerade: evaluierung verschiedener agenten-modelle, und blackboard architekturen,
auswahl einer internen term sprache. um schnelle ziele zu erreichen wollen wir vorerst nur einige
features der sprache nutzen --- *aber* auch schon damit mehr haben als andere (rank-1 polymorphism).
Sketch: Grundablauf Architektur (independent agents greifen auf blackboard zu, z.b. spezialagenten
oder unifizierungsagenten...blabla).
\end{abstract}



Quellen?

%%
%% Points :
%%    - Leo2 nachfolger                         <- muss rein
%%    - Higher Order Logic Theorem Prover
%%       -> Typeclasses
%%       -> Polymorphism                          <- muss rein
%%       -> (also quasi richly typed formal system)
%%       -> Wie keiner vorher war.
%%       -> term/data sharing, labeling techniques                        <- muss rein
%%       -> efficient reduction/substitution (z.b. durch ordering)                        <- muss rein
%%    - Highly Concurrent
%%       -> Blackboard Architecture                        <- muss rein
%%       -> Mutli-Agent Architecture                        <- muss rein
%%       -> And/Or - Parallelismus                        <- muss rein
%%       -> Distributed Search (Viele Kalkülagenten, die versuchen
%%             inferenz anzuwenden)
%%       -> Vorschlags oder Unterstützende Agenten
%%             ( Mögliche Unifikationspartner, Mögliche Unifier,
%%                Klauselnormalisierung, etc.)
%%    - State of the art                        <- muss rein
%%    - Nah an TPTP entwickeln; kompatible Syntax                        <- muss rein
%%    - Leicht anschließbar an ext. Systeme                        <- muss rein
%%          ("Provision of proof objects and support for the integration
%%           of LEO-III with proof assistants or other AI systems.")
%%    - auch was mit special logics? also z.b. QML (modal logics)
%%    - Modularer Aufbau um es als Framework zu nutzen (relativ wichtig)                        <- muss rein
%%
