\documentclass[10pt]{article}

\usepackage[top=1cm, left=2cm, right=2cm]{geometry}

\begin{document}
\title{The Leo-III Project}
\author{Max Wisniewski \and Alexander Steen \and Christoph Benzm\"{u}ller}
\date{}
\maketitle
\noindent We report on the recently started Leo-III project, in which we design and implement
a state-of-the-art Higher-Order Logic Theorem Prover, the successor of the well known LEO-II prover~\cite{Benzmüller08leo-ii—}.
Leo-III will be based on ordered paramodulation/superposition.

In constrast to LEO-II, we replace the internal term representation
(the commonly used simply typed lambda calculus) by a more expressive system supporting
type polymorphism. In the course of the project, we plan to further enhance the type
system with type classes and type constructors similar to System F$^\omega$.

In order to achieve a substantial performance speed-up, the architecture of Leo-III
will be based on massive parallelism (e.g. And/Or-Parallelism, Multisearch)~\cite{journals/amai/Bonacina00}.
The current design is a multi-agent blackboard
architecture~\cite{Weiss2013} that will allow to independently run agents with
our proof calculus as well as agents for external (specialized) provers.

\noindent Leo-III will focus right from the start on compatibility to the widely used TPTP infrastructure
~\cite{Sutcliffe:2009:TPL:1666192.1666217}. Moreover, it will offer built-in support for specialized
external prover agents and provide external interfaces to  interactive provers such as Isabelle/HOL~\cite{nipkow2002isabelle}.
The implementation will excessively use term sharing~\cite{Riazanov:2002:DIV:1218615.1218620,Schulz:2002:EBT:1218615.1218621}
and several indexing techniques~\cite{Nieuwenhuis03onthe, W35}.
Leo-III will also offer special support for reasoning in various quantified non-classical
logics by exploiting a semantic embedding~\cite{C35} approach.

\bibliographystyle{alpha}
\bibliography{cite}

\end{document}
