\documentclass[twocolumn,a4paper,10pt]{article}

\usepackage{arw-dt14}
\usepackage{graphicx}

\begin{document}

%% Insert here the title of your paper
\title{The Leo-III Project}

%% Use \institute to create an institute
%% in case of more than one institute use the \and command
%% inside the institute command, like in the commented part
\institute{
   FU-Berlin, Arnimallee 7,
    \email{max.wisniewski@fu-berlin.de}
 \and
   FU-Berlin, Arnimallee 7,
   \email{a.steen@fu-berlin.de}
  \and
   FU-Berlin, Arnimallee 7,
   \email{c.benzmueller@fu-berlin.de}
}

%% Use \inst to reference a created institute and the \and
%% command to divide the authors, see the commented part
\author{
   Max Wisniewski\inst{1}
 \and
   Alexander Steen\inst{2}
 \and
   Christoph Benzm\"{u}ller\inst{3}
}

%% Write here your abstract and remember: \title, \institute, \author and
%% \abstract MUST be specified before \maketitle is called
\abstract{
We introduce the recently started Leo-III project --- a Higher-Order Logic Theorem Prover and successor to LEO-II.
}

%% 

\maketitle


\section{Summary}

\noindent We report on the recently started Leo-III project, in which we design and implement
a state-of-the-art Higher-Order Logic Theorem Prover, the successor of the well known LEO-II prover~\cite{Benzmüller08leo-ii—}.
Leo-III will be based on ordered paramodulation/superposition.

In constrast to LEO-II, we replace the internal term representation
(the commonly used simply typed lambda calculus) by a more expressive system supporting
type polymorphism. In the course of the project, we plan to further enhance the type
system with type classes and type constructors similar to System F$^\omega$.

In order to achieve a substantial performance speed-up, the architecture of Leo-III
will be based on massive parallelism (e.g. And/Or-Parallelism, Multisearch)~\cite{journals/amai/Bonacina00}.
The current design is a multi-agent blackboard
architecture~\cite{Weiss2013} that will allow to independently run agents with
our proof calculus as well as agents for external (specialized) provers.

\noindent Leo-III will focus right from the start on compatibility to the widely used TPTP infrastructure
~\cite{Sutcliffe:2009:TPL:1666192.1666217}. Moreover, it will offer built-in support for specialized
external prover agents and provide external interfaces to  interactive provers such as Isabelle/HOL~\cite{nipkow2002isabelle}.
The implementation will excessively use term sharing~\cite{Riazanov:2002:DIV:1218615.1218620,Schulz:2002:EBT:1218615.1218621}
and several indexing techniques~\cite{Nieuwenhuis03onthe, W35}.
Leo-III will also offer special support for reasoning in various quantified non-classical
logics by exploiting a semantic embedding~\cite{C35} approach.



% **** PUT YOUR OWN BIBLIOGRAPHY FILES IN HERE ****
\bibliography{cite}


\end{document}
