\documentclass[11pt,a4paper,ngerman]{article}
\usepackage[bottom=2.5cm,top=2.5cm]{geometry}
\usepackage[ngerman]{babel}
\usepackage[utf8]{inputenc}
\usepackage[T1]{fontenc}
\usepackage{ae}
\usepackage{amssymb}
\usepackage{amsmath}
\usepackage{amsthm}
\usepackage{graphicx}
\usepackage{fancyhdr}
\usepackage{fancyref}
\usepackage{listings}
\usepackage{xcolor}
\usepackage{stmaryrd}
\usepackage{paralist}
\usepackage{tikz}
\usepackage{amsthm}
\usepackage{tabularx}

\usetikzlibrary{arrows,automata,shapes.geometric}

\newtheorem{propo}{Satz}
\newtheorem{lemmas}[propo]{Lemma}

\usepackage[pdftex, bookmarks=false, pdfstartview={FitH}, linkbordercolor=white]{hyperref}
\usepackage{fancyhdr}
\pagestyle{fancy}
\fancyhead[C]{ADS}
\fancyhead[L]{Übung 4}
\fancyhead[R]{SoSe 2014}
\fancyfoot{}
\fancyfoot[L]{}
\fancyfoot[C]{\thepage \hspace{1px} von \pageref{LastPage}}
\renewcommand{\footrulewidth}{0.5pt}
\renewcommand{\headrulewidth}{0.5pt}
\newcommand{\set}[1]{ \{ #1 \}}
\setlength{\parindent}{0pt}
\setlength{\headheight}{0pt}

\newcommand{\N}{\mathbb{N}}
\newcommand{\Q}{\mathbb{Q}}
\newcommand{\R}{\mathbb{R}}
\newcommand{\bigO}{\mathcal{O}}
\newcommand{\Rarr}{\Rightarrow}
\newcommand{\rarr}{\rightarrow}
\newcommand{\Pot}{\mathcal{P}}
\newcommand{\abs}[1]{\left |#1\right|}
\newcommand{\solved}{$\mbox{}$ \hfill $\square$}
\newcommand{\Epsilon}{\mathcal{E}}

\newcommand{\erw}[1]{\text{\bfseries E} \left[ #1 \right]}
\newcommand{\prob}[1]{\text{Pr}\left[ #1 \right]}

\date{}
\title{Übung 4}
\author{Max Wisniewski, Melanie Skodzik}


%%
%% Enviroments for proofs and lemmas
%%
\newtheorem{prop}{\bfseries Behauptung}
\newtheorem{lemma}{\bfseries Lemma}

\begin{document}

\lstset{language=Pascal, basicstyle=\ttfamily\fontsize{10pt}{10pt}\selectfont\upshape, commentstyle=\rmfamily\slshape, keywordstyle=\rmfamily\bfseries, breaklines=true, frame=single, xleftmargin=3mm, xrightmargin=3mm, tabsize=2, mathescape=true}

\renewcommand{\figurename}{Grafik}

\maketitle
\thispagestyle{fancy}


\subsection*{Aufgabe 1}

\subsubsection*{(a)}

Geben Sei Pseudocode wie in der Vorlesung für die \lstinline|delete|-Operation vin van-Emde-Boas-Bäumen. Beschreiben Sie Ihren Algorithmus auch in WOrten und analysieren Sie die Laufzeit.\\

\noindent\textbf{Lösung:}\\

tbd

\subsubsection*{(b)}

Lösen Sie die folgenden Rekursionsgleichung und diskutieren Sie ihre Relevanz für van-Emde-Boas-Böume. Machen Sie geeignete Annahmen über den Rekursionsanker.

\begin{enumerate}[(i)]
	\item $T(N) = T(\sqrt{N})) + \mathcal{O}(1)$\\
	\noindent\textbf{Lösung:}\\
		tbd

	\item $T(N) = 2T(\sqrt{N}) + \mathcal{O}(1)$\\
	\noindent\textbf{Lösung:}\\
		tbd

	\item $S(N) = \left( \sqrt{N} + 1 \right) S( \sqrt{N}) + \mathcal{O}(\sqrt{N})$\\
	\noindent\textbf{Lösung:}\\
		tbd
\end{enumerate}

\subsection*{Aufgabe 2}

\subsubsection*{(a)}
Sei $T$ ein Feld der Länge $N$. Zu Beginn kann $T$ beliebige Daten enthalten. Wir nehmen an, dass $N$ riesig ist und wir keine Zeit haben um $T$ zu initialisieren. Beschreiben Sie, wie wir $T$ dennoch verwenden können, um sinnvoll Daten zu speichern. Ihre Mehtode soll jeden Zugriff auf $T$ in $\mathcal{O}(1)$ Zeit unterstützen.\\

\noindent\textbf{Lösung:}\\
tbd

\subsubsection*{(b)}

Bei van-Emde-Boas  Bäumen scheint es auf den ersten Blick notwendig zu sein $\mathcal{O}(N)$ Zeit in die Initialisierung des Baumes zu inverstieren. Beschreiben Sie, wie es besser geht.\\

\noindent\textbf{Lösung:}\\

tbd

\subsection*{Aufgabe 3}

Entwerfen und analysieren Sie eine Word-RAM Datenstruktur, die eine Menge von disjunkten Intervallen der Form $\left[a,b\right)$ speichert, mit $a,b \in \mathcal{U}$. Ihre Datenstruktur soll die folgenden Operationen in $\mathcal{O}(\log\log \, N)$ Zeit unterstützen: \lstinline|make(a,b)|, \lstinline|union(a,b,c)|, \lstinline|split(a,b,k| und \lstinline|find(k)|.\\

\noindent\textbf{Lösung:}\\

tbd

\label{LastPage}
\end{document}
