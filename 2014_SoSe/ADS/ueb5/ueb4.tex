\documentclass[11pt,a4paper,ngerman]{article}
\usepackage[bottom=2.5cm,top=2.5cm]{geometry}
\usepackage[ngerman]{babel}
\usepackage[utf8]{inputenc}
\usepackage[T1]{fontenc}
\usepackage{ae}
\usepackage{amssymb}
\usepackage{amsmath}
\usepackage{amsthm}
\usepackage{graphicx}
\usepackage{fancyhdr}
\usepackage{fancyref}
\usepackage{listings}
\usepackage{xcolor}
\usepackage{stmaryrd}
\usepackage{paralist}
\usepackage{tikz}
\usepackage{amsthm}
\usepackage{tabularx}
\usepackage{algorithmic}
\usepackage{algorithm}



\usetikzlibrary{arrows,automata,shapes.geometric}

\newtheorem{propo}{Satz}
\newtheorem{lemmas}[propo]{Lemma}

\usepackage[pdftex, bookmarks=false, pdfstartview={FitH}, linkbordercolor=white]{hyperref}
\usepackage{fancyhdr}
\pagestyle{fancy}
\fancyhead[C]{ADS}
\fancyhead[L]{Übung 4}
\fancyhead[R]{SoSe 2014}
\fancyfoot{}
\fancyfoot[L]{}
\fancyfoot[C]{\thepage \hspace{1px} von \pageref{LastPage}}
\renewcommand{\footrulewidth}{0.5pt}
\renewcommand{\headrulewidth}{0.5pt}
\newcommand{\set}[1]{ \{ #1 \}}
\setlength{\parindent}{0pt}
\setlength{\headheight}{0pt}

\newcommand{\N}{\mathbb{N}}
\newcommand{\Q}{\mathbb{Q}}
\newcommand{\R}{\mathbb{R}}
\newcommand{\bigO}{\mathcal{O}}
\newcommand{\Rarr}{\Rightarrow}
\newcommand{\rarr}{\rightarrow}
\newcommand{\Pot}{\mathcal{P}}
\newcommand{\abs}[1]{\left |#1\right|}
\newcommand{\solved}{$\mbox{}$ \hfill $\square$}
\newcommand{\Epsilon}{\mathcal{E}}

\newcommand{\erw}[1]{\text{\bfseries E} \left[ #1 \right]}
\newcommand{\prob}[1]{\text{Pr}\left[ #1 \right]}

\date{}
\title{Übung 4}
\author{Max Wisniewski, Melanie Skodzik, Paul Podlech}


%%
%% Enviroments for proofs and lemmas
%%
\newtheorem{prop}{\bfseries Behauptung}
\newtheorem{lemma}{\bfseries Lemma}

\begin{document}

\lstset{language=Pascal, basicstyle=\ttfamily\fontsize{10pt}{10pt}\selectfont\upshape, commentstyle=\rmfamily\slshape, keywordstyle=\rmfamily\bfseries, breaklines=true, frame=single, xleftmargin=3mm, xrightmargin=3mm, tabsize=2, mathescape=true}

\renewcommand{\figurename}{Grafik}

\maketitle
\thispagestyle{fancy}


\subsection*{Aufgabe 1}

\subsubsection*{(a)}


Geben Sie Pseudocode, wie in der Vorlesung, für die \lstinline|delete|-Operation von van-Emde-Boas-Bäumen an. Beschreiben Sie Ihren Algorithmus auch in Worten und analysieren Sie die Laufzeit.\\

\noindent\textbf{Lösung:}\\
Der in Algorithm \ref{algorithm1} angegebene Pseudocode tut folgendes:
\begin{itemize}
	\item Falls $T$ nur $x$ enthält wird $T$ leer gesetzt.
	\item Falls $T$ nur $x$ und einen weiteren Schlüssel $y$ enthält, wird $y$ als min und max eingetragen (erkennbar am Hilfsbaum).
	\item Falls $x$ min oder max ist, wird der Index des nächstgrößeren bzw. nächstkleineren Eintrags  aus dem Hilfsbaum geholt und der entsprechende Eintrag $y$ als min, bzw. max gesetzt. Dann wird $y$ aus dem Baum gelöscht. Falls der Baum, der $y$ enthielt dann leer ist, wird der Index aus dem Hilfsbaum gelöscht.
	\item Sonst wird der Index $i$ des Unterbaums der $x$ enthält berechnet und $x$ aus diesem gelöscht. Falls der Unterbaum dann leer ist, wird $i$ aus dem Hilfsbaum gelöscht.
\end{itemize}
\noindent\textbf{Analyse:}\\
Am Pseudo-Code sieht man sofort, dass es bis auf die zwei rekursiven Aufrufe nur konstant viele Operationen gibt. Der zweite rekursive Aufruf, das Löschen auf dem Hilfsbaum, wird aber nur dann ausgeführt, wenn der Unterbaum, der $x$ enthielt, nach dem Löschen leer ist. In diesem Fall, hat der erste rekursive Aufruf konstante Laufzeit.

Insgesamt ergibt sich eine Laufzeit von $T(N) = T(\sqrt{N}) + O(1)$, also $O(\log\log \, N)$.

\begin{algorithm}[H]
\caption{delete(T,x)}
\label{algorithm1}
\begin{algorithmic}
\REQUIRE van Emde Boas Tree $T$ for keys $0,\dots,N-1$, key $x$ 
\ENSURE $x$ will be deleted from $T$
\IF{$T.min == T.max == x$}
	\STATE $T.min \leftarrow T.max \leftarrow None$
	\RETURN
\ENDIF

\IF{$T.min == x$}
	\IF{$T.aux.min == T.aux.max == None$}
		\STATE $T.min \leftarrow T.max$
		\RETURN
	\ELSE
		\STATE $x \leftarrow T.child[T.aux.min].min$
		\STATE $T.min \leftarrow x$
	\ENDIF
\ENDIF

\IF{$T.max == x$}
	\IF{$T.aux.min == T.aux.max == None$}
		\STATE $T.max \leftarrow T.min$
		\RETURN
	\ELSE
		\STATE $x \leftarrow T.child[T.aux.max].max$
		\STATE $T.max \leftarrow x$
	\ENDIF
\ENDIF

\IF{$T.aux.min == T.aux.max == None$}
	\RETURN
\ENDIF

\STATE $i \leftarrow \lfloor \frac{x}{\sqrt{N}} \rfloor$
\STATE $delete(T.child[i], x \text{ mod } \sqrt{N})$
\IF{$T.child[i].min == T.child[i].max == None$}
	\STATE $delete(T.aux, i)$
\ENDIF
\RETURN
\end{algorithmic}
\end{algorithm}




\newpage

\subsubsection*{(b)}

Lösen Sie die folgenden Rekursionsgleichung und diskutieren Sie ihre Relevanz für van-Emde-Boas-Bäume. Machen Sie geeignete Annahmen über den Rekursionsanker.

\begin{enumerate}[(i)]
	\item $T(N) = T(\sqrt{N})) + \mathcal{O}(1)$\\
	\noindent\textbf{Lösung:}\\
		Hier liegt eine lineare Rekursion vor, bei der wir pro Schritt nur eine konstante Anzahl von Schritten benötigen
		und pro Schritt die Wurzel aus der Anzahl ziehen. Nach $k$ Schritten sieht die Gleichung also wie folgt aus.
		$$
			T(N) \leq T(N^\frac{1}{2^k}) + c \cdot k.
		$$
		Wir sind also daran interessiert, wie groß $k$ ist, bis der Anker erreicht ist. Wir setzen $N=2$ den Anker(mit
		einer anderen konstanten Laufzeit).
		$$
			N^\frac{1}{2^k} = 2 \Leftrightarrow \log \, N = 2^k \Leftrightarrow \log\log \, N = k.
		$$
		Die Laufzeit ist also beschränkt durch
		$$
			T(N) \leq T(2) + c \log\log \, N = \mathcal{O}(\log\log \, N).
		$$

		Diese Laufzeit trat bei der Analyse von van-Emde-Boas Bäumen auf. Wir haben pro Knoten eine Konstante Anzahl von
		Schritten getätigt und sind dann in die Rekursion gegangen, aber jeweils nur eine Rekursion pro Knoten. Damit haben wir die
		gezeigte Laufzeit erreicht.

	\item $T(N) = 2T(\sqrt{N}) + \mathcal{O}(1)$\\
	\noindent\textbf{Lösung:}\\
		Hier verzweigt sich unser Rekrusionsbaum in zwei Äste und das gleichmäßig.
		Wir haben in Tiefe $k$ also $2^k$ Knoten. Jeder Knoten hat die \emph{nicht} Rekursive Laufzeit,
		d.h. die gesamte Tiefe $k$ hat aufsummiert die Laufzeit $c \cdot 2^k$.

		Die Laufzeit bis zur Tiefe $k$ ist demnach
		$$
			T(N) \leq 2^{k+1}T(N^\frac{1}{2^k}) + \sum_{i=0}^kc \cdot 2^i.
		$$
		Wir haben im letzten Schritt den Anker schon bestimmt (ebenso bei $2$ konstant), 
		also haben wir wieder $k = \log \log \, N$.
		Wir erhalten als Lösung
		$$
		T(N) \leq 2^{\log\log \, N + 1} \cdot d + c \cdot \sum_{i=0}^k 2^i = 2\cdot \log \, N + c \cdot (\log \, N - 1) = \mathcal{O}(\log \, N).
		$$

		Diese Analyse ist der Grund, warum wir bei van-Emde-Boas Bäumen darauf achten mussten, dass wir nur eine
		Rekursion pro Knoten machen. Würden wir also z.B. die Minima und Maxima nicht in den Knoten speichern, müsste
		man in dem Fall, dass der Vorgänger nicht im zugeordneten Intervall steht, sowohl in der Hilfsdatenstruktur, als auch im
		Rekursiven van-Emde-Boas Baum an der gefundenen Stelle suchen. Damit würde die Laufzeit deutlich schlechter sein.

	\item $S(N) = \left( \sqrt{N} + 1 \right) S( \sqrt{N}) + \mathcal{O}(\sqrt{N})$\\
	\noindent\textbf{Lösung:}\\
		Wir wollen per Induktion auf die Lösung kommen, dass $S(N) = \mathcal{O}(N)$ liegt. Dazu sehen wir uns zunächst alle Konstante
		an.
		$$
		S(N) = (\sqrt{N} + 1 ) S( \sqrt{N}) + c \cdot \sqrt{N}.
		$$
		\begin{description}
			\item[Induktions Vermutung:] $S(N) \leq d \cdot N - e$, für $e > 0$. Also ist $S$ ist eine Funktion, die Konstant weg
			von einer linearen Funktion beschränkt ist.
			\item[I.A.] Wir haben hier noch keinen Anker gewählt, aber da in diesem Fall alles konstant ist, wählen wir uns
				$S(2) = b$ mit $b$ so, dass alles hin haut.
			\item[I.S.] $\sqrt{N} \rightarrow N$. (Oder für alle $N'$ kleiner $N$ gilt die Aussage).\\
				$$\begin{array}{rcl}
					S(N) &=& (\sqrt{N} + 1) S(\sqrt{N}) + c \cdot \sqrt{N}\\
						&\stackrel{\text{I.V.}}{\leq}& (\sqrt{N} + 1) (d \sqrt{N} - e) + c \cdot \sqrt{N}\\
						&=& d \cdot N - e + (d + c - e) \sqrt{N}\\
						&\leq& d \cdot N - e
				\end{array}$$
				Ist nun $e \geq d +c$ so können wir den letzten Schritt machen. Wir sollten uns im klaren sein,
				dass die Funktion für kleine $N$ negativ werden kann, weswegen wir den Rekursionsanker genügend groß
				wählen, damit wir solche Probleme nicht beachten müssen.
		\end{description}

		Die gefundene Funktion beschreibt den Speicherbedarf von van-Emde-Boas Bäumen. Wir speichern uns pro Knoten
		$\sqrt{N}$ Rekursive van-Emde-Boas Bäume plus einen Hilfsbaum alle mit der Größe $\sqrt{N}$. Darüber hinaus
		brauchen wir pro Knoten ein Array (von Pointern) der Größe $\sqrt{N}$ plus Konstant viel Speicher (für z.B. Minimum und Maximum).
		
		So sehen wir nun, dass wir linear viel Speicher benötigen, was sinnvoll erscheint, wenn wir alle Schlüssel speichern wollen und im Hinblick
		auf van-Emde-Boas Bäume, die immer unabhängig von der eingegebenen Schlüsselmenge sind.
\end{enumerate}

\subsection*{Aufgabe 2}

\subsubsection*{(a)}
Sei $T$ ein Feld der Länge $N$. Zu Beginn kann $T$ beliebige Daten enthalten. Wir nehmen an, dass $N$ riesig ist und wir keine Zeit haben um $T$ zu initialisieren. Beschreiben Sie, wie wir $T$ dennoch verwenden können, um sinnvoll Daten zu speichern. Ihre Methode soll jeden Zugriff auf $T$ in $\mathcal{O}(1)$ Zeit unterstützen.\\

\noindent\textbf{Lösung:}\\
Um zu wissen welche Stellen wir im Array schon initialisiert haben, benötigen wir eine Möglichkeit heraus zu finden, welche Felder wir schon beschrieben haben. Was wir dazu benutzen können ist ein zweites Arrays $T_2$. In diesem speichern wir der Reihe nach die ID des Feldes, wenn wir ein neues Beschreiben. Wir wissen zu jedem Zeitpunkt, wie weit wir schon eingefügt haben (speichern wir uns in einer Variable \lstinline|next|).\\

Ein Datum an Position $i$ ist also korrekt, wenn ein $j$ existiert mit $j <$ \lstinline|next| und $T_2[j] = i$. Das Problem wäre nun $j$ zu finden in $O(1)$. Dazu erstellen wir uns aber einfach ein zweites Feld $T_1$ indem wir uns eben dieses $j$ speichern.\\

\begin{lstlisting}[label="Einfügen eines neuen Datums"]
	$T_1$[i] = top
	$T_1$[top] = i
	top++
	$T$[i] = val
\end{lstlisting}
dazu müssen wir vorher nur bestimmen, ob ein Feld beschrieben ist über die Abfrage
\begin{lstlisting}[label="Darum existiert"]
	$T_2$[$T_1$[i]] = i && i < top
\end{lstlisting}
die beiden Arrays $T_1, T_2$ müssen dazu nur die selbe größe wie $T$ haben, wir bleiben also asymptotisch im sleben Speicherplatzbedarf. Die
Abfrage eins Datums ist dann:
\begin{lstlisting}[label="Datum ermitteln"]
	if $T_2$[$T_1$[i]] = i && i < top
		return $T$[i]
	else
		error "Nicht gefunden"
\end{lstlisting}

Wir sehen, dass alle diese Operationen die Laufzeit $O(1)$ haben.
\subsubsection*{(b)}

Bei van-Emde-Boas  Bäumen scheint es auf den ersten Blick notwendig zu sein $\mathcal{O}(N)$ Zeit in die Initialisierung des Baumes zu inverstieren. Beschreiben Sie, wie es besser geht.\\

\noindent\textbf{Lösung:}\\

Wir benutzen die Obenstehende Struktur um die Knoten des van-Emde-Boas Baumes zu initialisieren. Dabei initialisieren wir einen Knoten nur,
wenn er das erste mal angefragt wird. Wir legen zu beginn einen Knoten an, mit leerem Minimum, Maximum und einem leeren Array,
das nach \emph{(a)} in $O(1)$ zu initialisieren ist, sowie einen leeren Hilfsbaum für die Predecessorsuche in den Knoten.\\

Wir nun ein noch leerer Baum angefragt (entweder Hilfsbaum oder im Array) so wird dieser auch zunächst initialisiert.\\

Wir sehen, dass eine Initialisierung eines neuen leeren Baumes uns nur $\mathcal{O}(1)$ kostet, daher können wir in den Algorithmen für insert, search, predecessor genau die Bäume die wir neu besuchen mit initialisieren, da wir pro Knoten höchstens 2 Bäume initialisieren müssen.

\subsection*{Aufgabe 3}

Entwerfen und analysieren Sie eine Word-RAM Datenstruktur, die eine Menge von disjunkten Intervallen der Form $\left[a,b\right)$ speichert, mit $a,b \in \mathcal{U}$. Ihre Datenstruktur soll die folgenden Operationen in $\mathcal{O}(\log\log \, N)$ Zeit unterstützen: \lstinline|make(a,b)|, \lstinline|union(a,b,c)|, \lstinline|split(a,b,k| und \lstinline|find(k)|.\\

\noindent\textbf{Lösung:}\\

Wir nehmen als Grundlage einen van-Emde-Boas Baum. Zu jedem gespeicherten Wert ( der ja in min/max steht) speichern wir uns nun noch einen weiteren Wert, den wir \lstinline|right| nennen. Die Idee ist nun, dass wir nur die Linken Schranken im van-Emde-Boas Baum speichern, die jeweils ihre rechten Intervall Grenzen kennen.\\

Sie $T$ der so annotierte van-Emde-Boas Baum ich nehme hier \lstinline|find| und \lstinline|predecessor| synonym, weil beide eh das selbe tun. Falls
dies nicht gewünscht ist, so können wir in union und split find benutzen und find des neuen Baumes predecessor.

\begin{description}
	\item[make(a,b):]  Wir rufen $T$.insert(a) auf mit \emph{b} annotierten $a$.
	\item[union(a,b,c):] Wir rufen $T$.delete(b) auf. Nun fehlt $[b,c)$ im Baum. Daher rufen wir $T$.find(a) auf, und
		in dem schritt, in dem wir $a$ finden ändern wir die Annotation von $a$ zu $c$. Damit repräsentiert der Knoten nun $[a,c)$.
	\item[split(a,b,k):]\footnote{Wir nehmen auch hier an, dass $[a,b)$ ein Interval ist. Sonst liefert uns die Suche $T$.find(a) den Gegenbeweis.}
		 Umgekehrt zu \lstinline|union|. Wir nehmen an, dass wir $b$ kennen (sonst machen wir ein find auf $a$ und nehmen
		uns die Annotation). Daher ändern wir die Annotation von $a$ um zu $k$ (wir rufen find auf $a$ auf und schreiben die neue Annotation 	
		hinein) und danach $T$.insert(k) mit Annotation $b$. Damit repräsentiert nun $a$ das Intervall $[a,k)$ und der Knoten $k$ das Intervall
		$[k,b)$. 
	\item[find(k)] : Wir rufen $T$.find(k) auf. Nach der Struktur der Predecessorsuche wird uns der Wert $a$ ausgegeben, der der nächst kleinere
		im van-Emde-Boas Baum ist. Dies ist also ein Kandidat für das Intervall. ist die Annotation von $a$ nun $b$ und $k \in [a,b)$ so
		haben wir das Intervall gefunden. Falls $b < k$ ist, so ist es nicht das Intervall. Da aber nun keine andere Linke Grenze zwischen $k$ und 		
		$b$ liegt (da kein andere Wert zwischen $a$ und $k$ lag) gibt es kein Intervall das $k$ beinhalten und das können wir zurück geben.
\end{description}

Da wir pro Operation höchstens zwei mal eine Operation auf dem van-Emde-Boas Baum aufrufen und das hinzufügen der Annotation die Laufzeit nicht ändert, ist die Laufzeit der Operationen allesamt $\mathcal{O}(\log \log \, N)$.

\label{LastPage}
\end{document}
