\documentclass[11pt,a4paper,ngerman]{article}
\usepackage[bottom=2.5cm,top=2.5cm]{geometry}
\usepackage[ngerman]{babel}
\usepackage[utf8]{inputenc}
\usepackage[T1]{fontenc}
\usepackage{ae}
\usepackage{amssymb}
\usepackage{amsmath}
\usepackage{amsthm}
\usepackage{graphicx}
\usepackage{fancyhdr}
\usepackage{fancyref}
\usepackage{listings}
\usepackage{xcolor}
\usepackage{stmaryrd}
\usepackage{paralist}
\usepackage{tikz}
\usepackage{amsthm}
\usepackage{tabularx}

\usetikzlibrary{arrows,automata,shapes.geometric}

\newtheorem{propo}{Satz}
\newtheorem{lemmas}[propo]{Lemma}

\usepackage[pdftex, bookmarks=false, pdfstartview={FitH}, linkbordercolor=white]{hyperref}
\usepackage{fancyhdr}
\pagestyle{fancy}
\fancyhead[C]{ADS}
\fancyhead[L]{Übung 3}
\fancyhead[R]{SoSe 2014}
\fancyfoot{}
\fancyfoot[L]{}
\fancyfoot[C]{\thepage \hspace{1px} von \pageref{LastPage}}
\renewcommand{\footrulewidth}{0.5pt}
\renewcommand{\headrulewidth}{0.5pt}
\newcommand{\set}[1]{ \{ #1 \}}
\setlength{\parindent}{0pt}
\setlength{\headheight}{0pt}

\newcommand{\N}{\mathbb{N}}
\newcommand{\Q}{\mathbb{Q}}
\newcommand{\R}{\mathbb{R}}
\newcommand{\bigO}{\mathcal{O}}
\newcommand{\Rarr}{\Rightarrow}
\newcommand{\rarr}{\rightarrow}
\newcommand{\Pot}{\mathcal{P}}
\newcommand{\abs}[1]{\left |#1\right|}
\newcommand{\solved}{$\mbox{}$ \hfill $\square$}
\newcommand{\Epsilon}{\mathcal{E}}

\newcommand{\erw}[1]{\text{\bfseries E} \left[ #1 \right]}
\newcommand{\prob}[1]{\text{Pr}\left[ #1 \right]}

\date{}
\title{Übung 3}
\author{Max Wisniewski, Melanie Skodzik}


%%
%% Enviroments for proofs and lemmas
%%
\newtheorem{prop}{\bfseries Behauptung}
\newtheorem{lemma}{\bfseries Lemma}

\begin{document}

\lstset{language=Pascal, basicstyle=\ttfamily\fontsize{10pt}{10pt}\selectfont\upshape, commentstyle=\rmfamily\slshape, keywordstyle=\rmfamily\bfseries, breaklines=true, frame=single, xleftmargin=3mm, xrightmargin=3mm, tabsize=2, mathescape=true}

\renewcommand{\figurename}{Grafik}

\maketitle
\thispagestyle{fancy}


\subsection*{Aufgabe 1}

\subsubsection*{(a)}
Seien $T_1$ und $T_2$ zwei binäre Suchbäume mit Schlüsselmenge $\{1, \ldots, n\}$. Zeigen Sie, dass sich $T_1$ mit $O(n)$ Rotationen in $T_2$ konvertieren lässt. Bestimmen Sie die Anzahl der Rotationen für Ihre Strategie im schlimmsten Fall exakt.

Versuchen Sie, so effizient wie möglich zu sein.\\

\noindent\textbf{Beweis:}\\

Um den Baum $T_1$ in den Baum $T_2$ zu transformieren, können wir dies entweder direkt angeben oder beide über einen Referenzbaum führen. Dazu sehen wir zunächst, dass eine Links-/Rechtsrotationen hintereinander auf den selben Knoten (nicht Wert sondern Position im Baum) den Baum nicht verändern. Damit ist jede Rotation umkehrbar.

Dies können wir benutzen. Wenn $T'$ ein Referenzbaum ist und $R_1 = a_1 ... a_k$ eine Folge von Rotation um $T_1$ in $T'$ zu transformieren und
$R_2 = b_1 ... b_l$ ist um $T_2$ in $T'$ zu transformieren dann ist $R = a_1 .. a_k b_l^{-1} .. b_1^{-1}$ eine Folge von Rotationen um $T_1$ in $T_2$ zu transformieren,
wobei $b_i^{-1}$ genau die Inverse Rotation ist.

Nun brauchen wir noch einen Referenzbaum.\\

Wir haben hierzu den listenförmigen Baum gewählt, der nur linke Kinder hat. Sei $T$ ein Baum, bei dem noch ein rechtes Kind existiert und $x$ ein Knoten, der nur linkes Kind ist, mit einem rechten Kind (dies existiert, solange ein rechts Kind existiert, solange wir die Wurzel als linkes Kind betrachten). Machen wir nun eine Linksrotation um $x$,
so wird $x$ auf dem Linken Pfad bleiben so wie alle linken Kinder darunter. Dagegen wird das unmittelbar rechte Kind zum Vater von $x$ (dessen Linkes Kind es ist). Da $x$ vorher linkes Kind war ist der neue Vater nun auch linkes Kind.\\

Was wir schon festgestellt haben, ist das pro Rotation ein weiteres Kind links wird. Damit ist die maximale Anzahl von Rotationen $n-1$, da mindestens die Wurzel zu Beginn auf dem linken Pfad liegt. Würden wir die Taktik von beiden Bäumen $T_1$ und $T_2$ aus so benutzen, hätten wir eine Laufzeit von $2n-2$ was in $O(n)$ liegt.

Da in jedem Schritt nur ein Knoten auf den linken Pfad kommen kann, es der schlechteste Fall, wenn zu Beginn kein Knoten auf dem Linken Pfad ist. Das würde an sich aber
bedeuten, dass die Wurzel das Minimum des Baumes ist. Falls im anderen Baum auch das Minimum die Wurzel ist, so müssen wir die Wurzel nicht anfassen. Und lassen diese fix.

Um also die maximale Strecke zu gehen muss im Ziel (oder Quellbaum) nicht das Minimum Wurzel sein, damit wäre aber schon ein Knoten mehr auf dem linken Pfad,
da die Wurzel nun mindestens das Minimum links von sich haben muss.\\

Dies würde sich in einer Laufzeit von $2n-3$ niederschlagen. Wir können jetzt von der Wurzel her (und von den Blätter her) schon existierende Strukturen unangetastet lassen. Und diese als Nullpointer betrachten (wie in der Analyse der Tangobäume). Es ist uns aber nicht möglich zu sehen, ob die Grenze von $2n-3$ damit durchbrochen werden kann.

\subsubsection*{(b)}

Sei $n=5$. Betrachte Sie die Anfragefolge $\sigma = 2,1,5,1,2,4,4,2,3,5$. Bestimmen Sie $I(\sigma)$. Geben Sie eine möglichst gute Bearbeitungsstrategie für $\sigma$ an. Berechnen Sie die Kosten für Ihren Algorithmus und vergleichen Sie sie mit $I(\sigma)$.\\

\noindent\textbf{Lösung:}\\

Im vollständigen Baum befinden sich $4,2,1$ auf dem linken Pfad. Es wird also das erste mal bei Anfrage $3$ auf die 5 ein rechter Pfad Liebling, der danach sofort wieder auf links gesetzt wird. Nun bleiben wir bis zu den letzten beiden Anfragen auf links. Bei den letzten beiden gehen wir genau 2 mal nach Rechts, wobei wir vorher auf links waren.\\

Demnach ist $I(\sigma) = 3$.\\

Bei der optimalen Strategie für $\sigma$ sollten wir im Linksvollständigen Baum starten (1.). Um zu sehen, wie gut wir werden,
betrachten wir eine Strategie ohne Rotationen. Diese hat die Kosten $21$.\\

Zum anderen können wir eine unscharfe untere Abschätzung angeben. Wenn wir die Folge von Operationen durchgehen, so wissen wir, 
dass wenn wir bei zwei aufeinander folgenden Operationen auf $2$ verschiedene Elemente zugreifen, so werden wir nicht besser als mit einer  Laufzeit von $3$ auskommen,
da nur ein Element die Wurzel sein konnte. Der Zugriff auf das zweite Element besucht also mindestens 2 Knoten oder im Schritt vorher wurde das Element hoch rotiert.\\

In jedem Fall können wir für zwei aufeinander Folgende unterschiedliche Elemente immer eins an Kosten extra zählen. Zusammen mit den Kosten für einen Zugriff von mindestens
$1$ ergibt die schon einmal eine Laufzeit von $18$. Um die optimale Folge zu finden haben wir also schon einmal einen sehr kleinen Spielraum.

\begin{center}
   1.
   \begin{tikzpicture}[font=\small, minimum size=0.5cm]
      \node[circle, draw=black,name=a] {$4$};
      \node[circle, draw=black, name=b, below left of=a] {$2$};
      \node[circle, draw=black, name=c, below right of =a] {$5$};
      \node[circle, draw=black, name=d, below left of = b] {$1$};
      \node[circle, draw=black, name=e, below right of =b] {$3$};

      \path[-]
         (a) edge (b)
         (a) edge (c)
         (c) edge (d)
         (d) edge (e);
   \end{tikzpicture}
\end{center}

\subsection*{Aufgabe 2}

Geben Sie die Details für die Operationen \lstinline|join| und \lstinline|split| aus der Vorlesung.

\begin{description}

\item[\bfseries join:] Gegeben ein binärer Suchbaum $T$, dessen linker und rechter Teilbaum gültige AVL-Bäume sind, transformiere $T$ in einen gültigen AVL-Baum. Die Operationen soll $T$ nur durch Rotationen verändern und $\mathcal{O}(1 + | h_1 - h_2|)$ Schritte benötigen, wobei $h_1$ und $h_2$ die Höhen des linken und des rechten Teilbaums von $T$ sind.\\

\noindent\textbf{Lösung:}\\

Wir verfallen für die Analyse in eine große Fallunterscheidung. Sei o.B.d.A. der rechte Teilbaum der Größere.

\begin{enumerate}[1. {Fall:}]
   \item Im ersten Fall betrachten wir den Baum 
      \begin{center}
         \begin{tikzpicture}
            \node[circle, draw=black,name=a] {$x$};
            \node[circle, draw=black, name=b, below right of=a] {$y$};
            \node[draw=black, name=c, below right of =b] {$T_3$};
            \node[draw=black, name=d, below left of = a] {$T_1$};
            \node[draw=black, name=e, below left of =b] {$T_2$};

            \path[-]
               (a) edge (b)
               (a) edge (d)
               (b) edge (c)
               (b) edge (e);
         \end{tikzpicture}
      \end{center}
      wobei $h_1, h_2, h_3$ die Höhen von $T_1, T_2, T_3$ sind. Im ersten Fall nehmen wir dann, dass $h_2 \leq h_3$ gilt.
      
      Führen wir nun eine einfache Rotation um $x$ aus. Die Höhe des Baumes um das resultierende $x$ ist nun $h_2 + 1$, damit ist die Höhendifferenz richtig.
      Die gesamte Höhe das Baumes verändert sich nicht.\\

      Wir werden im folgenden sehen, dass die Höhe sich um maximal $1$ verringern kann, damit bleibt die Höhendifferenz immer noch richtig und der Baum verliert in diesem auch 
      höchstens nur eins an Höhe.\\

      Wir können nun Rekursiv in $x$ fortfahren, wobei die Höhendifferenz unter $x$ um mindestens $1$ und maximal $2$ kleiner geworden ist.
   \item In den nächsten beiden Fällen ist der innere Baum der größere. Wir haben also die folgende Situation.
      \begin{center}
         \begin{tikzpicture}
            \node[circle, draw=black,name=a] {$x$};
            \node[circle, draw=black, name=b, below right of=a] {$y$};
            \node[circle, draw=black, name=c, below left of=b] {$z$};
            \node[draw=black, name=d, below left of =a] {$T_1$};
            \node[draw=black, name=e, below left of = c] {$T_2$};
            \node[draw=black, name=f, below right of =c] {$T_3$};
            \node[draw=black, name=g, below right of=b] {$T_4$};

            \path[-]
               (a) edge (d)
               (a) edge (b)
               (b) edge (c)
               (b) edge (g)
               (c) edge (e)
               (c) edge (f);
         \end{tikzpicture}
      \end{center}
      Wir bezeichnen wieder mit $h_1$ bis $h_4$ die Höhe der Unterbäume und mit $h_x, h_y, h_z$ die Höhe der Bäume unter den Knoten $x,y,z$.\\

      Der Fall den wir nun betrachten ist, $h_2 < h_3$ ist. Wir wissen, dass $h_x = h_4 +1$ ist. Und damit ist $h_3 = h_4$ und $h_2 = h_3 - 1$.
      Was wir nun tun ist, dass wir einmal Links um $x$ rotieren und dann noch einmal. Der resultierende Baum ist\\
      \begin{center}
          \begin{tikzpicture}
            \node[circle, draw=black,name=a] {$y$};
            \node[circle, draw=black, name=b, below left of=a] {$z$};
            \node[circle, draw=black, name=c, below left of=b] {$x$};
            \node[draw=black, name=d, below left of =c] {$T_1$};
            \node[draw=black, name=e, below right of = c] {$T_2$};
            \node[draw=black, name=f, below right of =b] {$T_3$};
            \node[draw=black, name=g, below right of=a] {$T_4$};

            \path[-]
               (a) edge (b)
               (a) edge (g)
               (b) edge (c)
               (b) edge (f)
               (c) edge (d)
               (c) edge (e);
         \end{tikzpicture}
      \end{center}
      
      Die Höhe vom ursprünglichen Baum war $h_3 + 2$, nur nach diesen Rotationen ist die Höhe immer noch $h_3$. Die Höhendifferenz in $y$ danach ist noch
      korrekt, da $h_4 = h_z + 1$ gilt ($h_3$ ist nun unter $z$ eins Tiefer).
      In $z$ stimmt es auch, da $h_2 = h_3 - 1$ ist nun eins Tiefer steht. Damit ist die Höhendifferenz $0$. Fahren wir nun fort, reduziert sich die Höhe
      in $x$ um maximal $1$. Damit bleibt die AVL-Eigenschaft der Knoten erhalten.\\

      Wir fahren Rekursiv in $x$ fort und die Höhendifferenz ist auch um $2$ gefallen.

      \item Der letzte Fall wird aus dem Baum des letzten Falls gebildet, nur dass diesmal $h_2 \geq h_3$ angenommen wird. Wir führen die selben Rotationen,
         wie im letzten Fall aus und rotieren danach noch einmal um $y$ nach Links. Der resultierende Baum ist

      \begin{center}
         \begin{tikzpicture}
            \node[circle, draw=black,name=a] {$z$};
            \node[circle, draw=black, name=b, below left of=a] {$x$};
            \node[circle, draw=black, name=c, below right of=a] {$y$};
            \node[draw=black, name=d, below left of =b] {$T_1$};
            \node[draw=black, name=e, below right of = b] {$T_2$};
            \node[draw=black, name=f, below left of =c] {$T_3$};
            \node[draw=black, name=g, below right of=c] {$T_4$};

            \path[-]
               (a) edge (b)
               (a) edge (c)
               (b) edge (d)
               (b) edge (e)
               (c) edge (f)
               (c) edge (g);
         \end{tikzpicture}
      \end{center}

      TODO : Den Graph noch auseinander Rücken.\\

      Betrachten wir nur den linken Ergebnisbaum unter $y$, so stimmt hier die AVL Eigenschaft. Unsere Voraussetzung war, das $h_z < h_4$ vorher war. Demnach ist
      $h_4 = max\{ h_2, h_3 \}$. Im Unterbaum von $y$ ist dann entweder $h_3 = h_4$ oder $h_3 = h_4 - 1$, was die AVL-Eigenschaft erhält.\\
   
      Wir wissen, dass $h_2 \geq h_3$ ist (unser Fall), daher ist $h_x = h_y$ oder $h_x = h_y +1$ nach den Rotationen ($h_4$ und $h_1$ sind nach Voraussetzung höchstens so groß, spielen
      für die Höhe also keine Rolle). Die gesamte Höhe von des Baumes ist immer noch genau so groß.

      Fahren wir nun Rekursiv fort, so wird $h_x = h_y - 1$ oder $h_x = h_y$ (es ist zwar nicht mehr $x$, aber wir meinen den Knoten). Falls $h_2$ echt größer war als $h_3$,
      verringert sich die Höhe des gesamten Baumes um $1$, sonst bleibt sie gleich. Die Höhendifferenz wird auch um mindestens $1$ kleiner.
\end{enumerate}

Mit diesen Fällen fahren wir so lange fort, bis die Höhendifferenz unter $x$ kleiner als 2 ist. Da wir in jedem Schritt konstant viele Rotationen benötigen und
auch nur konstant viele Knoten mehr besuchen, ist die Laufzeit $O(|h_1 - h_2|)$, da wir in jedem Schritt die Höhendifferenz um mindestens $1$ reduzieren. Wir haben
auch gesehen, dass die Höhe eines Baumes um maximal $1$ abnehmen kann.

\item[\bfseries split:] Gegeben ein AVL-Baum $T$ und ein Knoten $x \in T$, transformiere $T$ in einen binären Suchbaum mit Wurzel $x$, so dass der linke und der rechte Teilbaum jeweils gültige AVL-Bäume sind. Die Operationen soll $T$ nur durch Rotationen verändern und $O(h)$ Schritte benötigen, wobei $H$ die Höhe von $T$ ist.\\

\mbox{}\hfill$\square$

\noindent\textbf{Lösung:}\\

Als generellen Ansatz bubbeln wir $x$ nach oben. Nach jeder Rotation von $x$ nach oben rufen wir \lstinline|join| auf dem nach unten rotierten Knoten auf um einen
AVL Baum daraus zu machen.

Sei o.B.d.A. der Baum wie folgt

\begin{center}
   \begin{tabularx}{0.6\textwidth}{ccc}
   \begin{tikzpicture}[font=\small, minimum size=0.5cm]
      \node[circle, draw=black,name=a] {$y$};
      \node[circle, draw=black, name=b, below left of=a] {$x$};
      \node[draw=black, name=c, below right of =a] {$T_3$};
      \node[draw=black, name=d, below left of = b] {$T_1$};
      \node[draw=black, name=e, below right of =b] {$T_2$};

      \path[-]
         (a) edge (b)
         (a) edge (c)
         (b) edge (d)
         (b) edge (e);
   \end{tikzpicture}
   &
   $\leadsto$
   
   \vspace{1cm}
   &
    \begin{tikzpicture}[font=\small, minimum size=0.5cm]
      \node[circle, draw=black,name=a] {$x$};
      \node[circle, draw=black, name=b, below right of=a] {$y$};
      \node[draw=black, name=c, below right of =b] {$T_3$};
      \node[draw=black, name=d, below left of =a] {$T_1$};
      \node[draw=black, name=e, below left of =b] {$T_2$};

      \path[-]
         (a) edge (b)
         (a) edge (d)
         (b) edge (c)
         (b) edge (e);
   \end{tikzpicture}
   \end{tabularx} 
\end{center} 

Und danach rufen wir \lstinline|join| auf $y$ auf. Seien $h_1, h_2, h_3$ die Höhen der Bäume $T_1, T_2, T_3$.

Was wir nun beobachten ist, dass der linke Teilbaum von $x$, nämlich $T_1$ in der Höhe gleich geblieben ist. 

Für den rechten Teilbaum gilt, dass der Baum nur größer wird. Falls $h_3$ größer oder gleich der Höhe vom Teilbaum von $x$ vor der Rotation ist,
so ist der Baum unter $y$ im resultierenden Baum um mindestens 2 größer. Führen wir nun \lstinline|join| auf, so verringert sich die Höhe um höchstens $1$.

Ist $h_3$ um 1 kleiner , so muss $h_3$ so groß sein, wie der größere der beiden Bäume. Falls $h_2 = h_3$ sind wir fertig und der rechte Teilbaum ist auf Größe $h_3 + 1$ gestiegen. Falls $h_2 < h_3$ ist, so steigt die Höhe des rechten Teilbaums nach dem \lstinline|join| genau $h_3$, was auch größer ist.\\

Wir sehen, dass in jedem Fall die Höhe vom gesamten Baum steigt. Damit können wir nun im nächsten Schritt weiter machen.\\

Was wir auf jeden Fall sehen ist, dass der rechte Baum immer nur größer wird. Und wir ziehen ihn bei einer Höhendifferenz $|h_1 - h_2|$ beim \lstinline|join| 
immer auf das Höhere Niveau. Wenn wir von unten her anfangen können wir so nur eine Partition von $h = h_1 + |h_2 - h_1| + |h_3 - h_2|$ von Höhendifferenzen bekommen,
so dass das \lstinline|join| auf den rechten Baum nur insgesamt $O(h)$ Operationen benötigt. Das selbe gilt für den linken Teilbaum. Nun kann es beides zwar nicht
beides gleichzeitig so groß werden, aber es reicht für die Abschätzung nach oben.\\

Zu guter Letzt benötigen wir noch $O(h)$ Operationen um $x$ nach oben zu bubbeln.

\mbox{}\hfill$\square$
\end{description}

\label{LastPage}
\end{document}
