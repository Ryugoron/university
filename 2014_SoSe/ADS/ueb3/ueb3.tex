\documentclass[11pt,a4paper,ngerman]{article}
\usepackage[bottom=2.5cm,top=2.5cm]{geometry}
\usepackage[ngerman]{babel}
\usepackage[utf8]{inputenc}
\usepackage[T1]{fontenc}
\usepackage{ae}
\usepackage{amssymb}
\usepackage{amsmath}
\usepackage{amsthm}
\usepackage{graphicx}
\usepackage{fancyhdr}
\usepackage{fancyref}
\usepackage{listings}
\usepackage{xcolor}
\usepackage{stmaryrd}
\usepackage{paralist}
\usepackage{tikz}
\usepackage{amsthm}

\usetikzlibrary{arrows,automata}

\newtheorem{propo}{Satz}
\newtheorem{lemmas}[propo]{Lemma}

\usepackage[pdftex, bookmarks=false, pdfstartview={FitH}, linkbordercolor=white]{hyperref}
\usepackage{fancyhdr}
\pagestyle{fancy}
\fancyhead[C]{ADS}
\fancyhead[L]{Übung 3}
\fancyhead[R]{SoSe 2014}
\fancyfoot{}
\fancyfoot[L]{}
\fancyfoot[C]{\thepage \hspace{1px} von \pageref{LastPage}}
\renewcommand{\footrulewidth}{0.5pt}
\renewcommand{\headrulewidth}{0.5pt}
\newcommand{\set}[1]{ \{ #1 \}}
\setlength{\parindent}{0pt}
\setlength{\headheight}{0pt}

\newcommand{\N}{\mathbb{N}}
\newcommand{\Q}{\mathbb{Q}}
\newcommand{\R}{\mathbb{R}}
\newcommand{\bigO}{\mathcal{O}}
\newcommand{\Rarr}{\Rightarrow}
\newcommand{\rarr}{\rightarrow}
\newcommand{\Pot}{\mathcal{P}}
\newcommand{\abs}[1]{\left |#1\right|}
\newcommand{\solved}{$\mbox{}$ \hfill $\square$}
\newcommand{\Epsilon}{\mathcal{E}}

\newcommand{\erw}[1]{\text{\bfseries E} \left[ #1 \right]}
\newcommand{\prob}[1]{\text{Pr}\left[ #1 \right]}

\date{}
\title{Übung 3}
\author{Max Wisniewski, Melanie Skodzik}


%%
%% Enviroments for proofs and lemmas
%%
\newtheorem{prop}{\bfseries Behauptung}
\newtheorem{lemma}{\bfseries Lemma}

\begin{document}

\lstset{language=Pascal, basicstyle=\ttfamily\fontsize{10pt}{10pt}\selectfont\upshape, commentstyle=\rmfamily\slshape, keywordstyle=\rmfamily\bfseries, breaklines=true, frame=single, xleftmargin=3mm, xrightmargin=3mm, tabsize=2, mathescape=true}

\renewcommand{\figurename}{Grafik}

\maketitle
\thispagestyle{fancy}


\subsection*{Aufgabe 1}

\subsubsection*{(a)}
Seien $T_1$ und $T_2$ zwei binäre Suchbäume mit Schlüsselmenge $\{1, \ldots, n\}$. Zeigen Sie, dass sich $T_1$ mit $O(n)$ Rotationen in $T_2$ konvertieren lässt. Bestimmen Sie die Anzahl der Rotationen für Ihre Strategie im schlimmsten Fall exakt.

Versuchen Sie, so effizient wie möglich zu sein.\\

\noindent\textbf{Beweis:}\\

tbd

\subsubsection*{(b)}

Sei $n=5$. Betrachte Sie die Anfragefolge $\sigma = 2,1,5,1,2,4,4,2,3,5$. Bestimmen Sie $I(\sigma)$. Geben Sie eine möglichst gute Bearbeitungsstrategie für $\sigma$ an. Berechnen Sie die Kosten für Ihren Algorithmus und vergleichen Sie sie mit $I(\sigma)$.\\

\noindent\textbf{Lösung:}\\

tbd


\subsection*{Aufgabe 2}

Geben Sie die Details für die Operationen \lstinline|join| und \lstinline|split| aus der Vorlesung.

\begin{description}

\item[\bfseries join:] Gegeben ein binärer Suchbaum $T$, dessen linker und rechter Teilbaum gültige AVL-Bäume sind, transformiere $T$ in einen gültigen AVL-Baum. Die Operationen soll $T$ nur durch Rotationen verändern und $\mathcal{O}(1 + | h_1 - h_2|)$ Schritte benötigen, wobei $h_1$ und $h_2$ die Höhen des linken und des rechten Teilbaums von $T$ sind.\\

\noindent\textbf{Lösung:}\\

tbd

\item[\bfseries split:] Gegeben ein AVL-Baum $T$ und ein Knoten $x \in T$, transformiere $T$ in einen binären Suchebaum mit Wurzel $x$, so dass der linke und der rechte Teilbaum jeweils gültige AVL-Bäume sind. Die Operationen soll $T$ nur durch Rotationen verändern und $O(h)$ Schritte benötigen, wobei $H$ die Höhe von $T$ ist.\\

\noindent\textbf{Lösung:}\\

tbd

\end{description}

\label{LastPage}
\end{document}
