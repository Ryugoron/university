\section*{Exercise 2}

In the knapsack problem, we are given $n$ items. Each item has a \emph{weight} $g_i$ and a \emph{value} $w_i$.
Furthermore , we have a maximum weight $G$. All inputs are positive integers.\\

We would like to find a set $I \subseteq \{ 1,...,n \}$ of items, such that the total value
$\underset{i \in I}{\sum} w_i$ is maximum, subject to the constraint that the total weight is at most
$G$, meaning $\underset{i \in I}{\sum} g_i \leq G$.

\subsection*{(a)} 

Define an appropriate decision version for the knapsack problem and show that it is NP-complete.\\

\textbf{Solution:}\\
We choose the canonical extension of an optimization problem to a decision problem. Let the defined variables
be as the the optimization problem. Than take an additional positive Integer $K$.\\

The decision question is now, does there exist a set $I \subseteq \{ 1, ... , n \}$, such that
$\underset{i \in I}{\sum} w_i \geq K$ and $\underset{ i \in I}{\sum} g_i \leq G$.

Let KNAP be this problem.

\begin{lemma}\label{cmp:ex1:knap:np}
The decision problem for the knapsack problem is NP-complete.
\end{lemma}
\textbf{Proof \ref{cmp:ex1:knap:np}.}\\
\begin{enumerate}[i)]
    \item KNAP $\in$ NP.\\
        Let $I \subset [n]$ be the optimal solution. Obviously $|I| < n^c$ for some constant $c>0$,
        because $I$ is itself part of the input.\\
        The Verifier $V$ as follows.\\
        Compute $\underset{i \in I}{\sum} w_i = w'$ and 
        $\underset{i \in I}{\sum} g_i = g'$.\\
        Accept if and only if $w' \geq K$ and $ g' \leq G$.\\

        Both sums can be computed in polynomial time and the comparison can be computed
        in logarithmic time in the length of the values.
        $V$ accepts $(w,I)$ in $T(n) \in O(n^c)$ for some constant $c>0$.

    \item SUBSET-SUM $\prec_p$ KNAP.\\
        To reduce subset-sum to knapsack we choose the following transformation. Let $S$ be the set of
        positive integers for subset-sum and $T$ a value. The question for knapsack is now, does there exist
        a subset $O \subseteq S$ such that $\underset{i \in O}{\sum} i = T$.

        The reduction function $f$ sets $w_i := g_i := i$ for all $i \in S$ and $G := K := T$. One can
        easily see, that the reduction can be done in linear time because we only copy values. \\

        $w \in$ SUBSET-SUM $\Rightarrow f(w) \in$ KNAP.\\
        There exists a subset $O \subset S$ such that $\underset{i \in O}{\sum} i = T$.
        After the application of $f$ we know that $O \subset I$ and
        $\underset{i \in O} w_i = K$ and $\underset{i \in O} g_i = G$. Therefore $O$ is a solution for
        KNAP on this input.\\

        $f(w) \in$ KNAP $\Rightarrow w \in$ SUBSET-SUM.\\
        There exists a subset $O \subset I$ such that $\underset{i \in O}{\sum} g_i \leq G$ and $\underset{i \in O}{\sum} w_i \geq K$.
        From the reduction $f$ we know that $\underset{i \in O}{\sum} g_i = \underset{i \in O}{\sum} w_i = \underset{i \in O}{\sum} i$ holds.\\
        $\Rightarrow \underset{i \in O}{\sum} i \leq G \land \underset{i \in O}{\sum} i \geq K$.\\
        Again from $f$ we know that $G = K = T$ holds.\\
        $\Rightarrow \underset{i \in O}{\sum} i = T$.\\
\mbox{} \hfill $\square$
\end{enumerate}


\subsection*{(b)}

Let $W := \underset{i=1}{\overset{n}{\sum}} w_i$. Show that the knapsack problem can be solved in $O(nW)$ time. Why does this not
contradict \emph{(a)}?\\

\textbf{Solution:}\\

To proof this, we show, that a dynamic program solves the knapsack problem and has the runtime of $O(nW)$.\\
Let $g[n,W]$ be a two dimensional array, that stores the in $g[i,w]$ the minimal weight, for the first $i$ objects,
with an value exactly $w$.

We can now give the recursive definition of to fill in $w$. Let the array be initialized with $\infty$.

$$\begin{array}{lclr}
    g[0,w]  &:=& 0          & \forall 0 \leq w \leq W\\
    g[i,0]  &:=& 0          & \forall 1 \leq i \leq n\\
    g[i,w]  &:=& g[i-1,w]   & \text{if } w - w_i \leq 0\\
    g[i,w]  &:=& \min \{ g[i-1,w] , g[i-1 , w - w_i] + g_i\}
\end{array}$$

After filling the array we can retrieve the maximal value by computing\\
$\underset{0 \leq w \leq W}{\text{argmax}} \{ g[n,w] \; | \; g[n,w] \leq G \}$.\\

This algorithm can be easily implemented through a double for loop, comparing
both cases. The result can be computed in a single for loop.

The algorithm computes the optimal solution, because it computes every possible solution and takes the greatest value.\\
The computation takes $O(nW)$ time, because the array has $n \times W$ cells and in each cell we have an constant
computation time $O(1)$.
To find the maximal value we iterate over $W$ cells and compute the maximum which can be done in time $O(W)$.\\
Therefore $T(n) = O(nW)$ holds.\\

This does not contradict with \emph{(a)} because $W$ can be exponential in the size of the input. The runtime of the algorithm
is pseudo polynomial.

\subsection*{(c)}

For the $(1 - \varepsilon)$ - approximation we use an algorithm based on the algorithm of $(b)$.

The trick is to round the values of $w_i$ such that we do not have to look at each possible value of $w$.\\

We assign new values by the following scheme

\begin{lstlisting}
M $\leftarrow$ $\max_{i\in I} w_i$
$\mu$ $\leftarrow$ $\varepsilon \frac{M}{n}$
w_i' $\leftarrow$ $\lfloor w_i / \mu \rfloor$ for all $i \in S$
\end{lstlisting}

and run the algorithm from $(b)$ on it.
With the modified values we can skip the array in $\mu$ steps.
$$
\frac{W}{\mu} = \frac{n}{\varepsilon \cdot M} \overset{n}{\underset{i=1}{\sum}} w_i \stackrel{M \geq w_i}{\leq} \frac{n}{\varepsilon} \overset{n}{\underset{i=1}{\sum}} 1 = O (\frac{n^2}{\varepsilon})
$$
The new grid has size $n \times O (\frac{n^2}{\varepsilon})$ therefore the runtime of the algorithm on the modified values is
in $poly(n, \frac{1}{\varepsilon})$.\\
Leaves us to proof, that the algorithm is a $(1-\varepsilon)$ - approximation.
Let $I$ be the set which gives us the optimal solution on $W$ and $O$ the set, that gives use the optimal solution on $W'$ the rounded values.\\

First observe, that $M \leq OPT$, because we can always can take the most valuable item. The rounding of the values gives us the estimation 
$ \mu w_i' \leq w_i \leq \mu (w_i' + 1)$ which leads us to $\mu w_i'  \geq w_i - \mu$.

$$\begin{array}{rcl}
    \underset{i \in O}{\sum} w_i &\stackrel{\text{Def. } w_i'}{\geq}& \mu \underset{i \in O}{\sum} w_i'\\
        &\stackrel{O \text{ opt. on } w_i'}{\geq}& \mu \underset{i \in I}{\sum} w_i'\\
        &\geq& (\underset{i \in I}{\sum} w_i) - |I|\mu\\
        &\stackrel{|I| < n}{\geq}& (\underset{i \in I}{\sum} w_i) - n \mu\\
        &\stackrel{\text{Def. }\mu}{=}& (\underset{i \in I}{\sum} w_i) - \varepsilon M\\
        &\stackrel{I \text{ opt. on }w_i}{\geq}& OPT - \varepsilon OPT = (1-\varepsilon) OPT
\end{array}$$ 

We can conclude that the algorithm is an FPTAS for the knapsack problem.
