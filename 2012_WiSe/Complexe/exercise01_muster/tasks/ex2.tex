\section*{Exercise 2}

In the knapsack problem, we are given $n$ items. Each item has a \emph{weight} $g_i$ and a \emph{value} $w_i$.
Furthermore , we have a maximum weight $G$. All inputs are positive integers.\\

We would like to find a set $I \subseteq \{ 1,...,n \}$ of items, such that the total value
$\underset{i \in I}{\sum} w_i$ is maximum, subject to the constraint that the total weight is at most
$G$, meaning $\underset{i \in I}{\sum} g_i \leq G$.

\subsection*{(a)} 

Define an appropriate decision version for the knapsack problem and show that it is NP-complete.\\

\textbf{Solution:}\\
We choose the canonical extension of an optimisation problem to a decision problem. Let the defined variables
be as the the optimisation problem. Than take an additional positive Integer $K$.\\

The decision question is now, does there exist a set $I \subseteq \{ 1, ... , n \}$, such that
$\underset{i \in I}{\sum} w_i \geq K$ and $\underset{ i \in I}{\sum} g_i \leq G$.

Let $KNAP$ be this problem.

\pagebreak

\begin{lemma}\label{cmp:ex1:knap:np}
The decision problem for the knapsack problem is NP-complete.
\end{lemma}
\textbf{Proof \ref{cmp:ex1:knap:np}.}\\
\begin{enumerate}[i)]
    \item $KNAP \in NP$.\\
        Let $I \subset [n]$ be the optimal solution. Obviously $|I| < n^c$ for some constant $c>0$,
        because $I$ is itsef part of the input.\\
        Now construct $TM V$ as follows.\\
        Compute $\underset{i \in I}{\sum} w_i = w'$ and\\
        $\underset{i \in I}{\sum} g_i = g'$.\\
        Accept if and only if $w' \geq K$ and $ g' \leq G$.\\

        Both sums can be computed in polynomial time and the final comparison can be computed
        in logarithmic time in the length of the values.
        So $V$ accepts $(w,I)$ in $T(n) \in O(n^c)$ for some constant $c>0$.

    \item SUBSET-SUM $\prec_p$ KNAP.\\
        To reduce subset-sum to knapsack we choose the following transformation. Let $S$ be the set of
        positive integers for subset-sum and $T$ a value. The question for knapsack is now, does there exist
        a subset $O \subseteq S$ such that $\underset{i \in O}{\sum} i = T$.\\

        The reduction now sets $w_i = g_i = i$ for all $i \in S$ and $G = K = T$ for the goal values. One can
        easily see, that the reduction can be done in linear time because we only copy values, thus it can be computed in polynomial time. \\

        Knapsack run on this input will now hold the following. If there exists a set $I$ that satisfies the constraints
        $\underset{ i \in I}{\sum} w_i = \underset{i \in I}{\sum} i \geq K$ and $\underset{i \in I}{\sum} g_i = \underset{i \in I}{\sum} i \leq K$.
        Thus if we set $O = I$ $\underset{i \in O}{\sum} i = T$ holds.

        If no solution existed in the original problem, than we cannot find a solution in the reduction, because both sums
        will sum up to exactly the same. So one of the constraints cannot be true.
\end{enumerate}


\subsection*{(b)}

Let $W := \underset{i=1}{\overset{n}{\sum}} w_i$. Show that the knapsack problem can be solved in $O(nW)$ time. Why does this not
contradict \emph{(a)}?\\

\textbf{Solution:}\\

To proof this, we show, that a dynamic program solves the knapsack problem and has the runtime of $O(nW)$.\\
Let $g[n,W]$ be a two dimensional array, that stores the in $g[i,w]$ the minimal weight, for the first $i$ objects,
with an value exactly $w$.

We can now give the recursive definition of to fill in $w$. Let the array be initialized with $\infty$.

$$\begin{array}{lclr}
    g[0,w]  &:=& 0          & \forall 0 \leq w \leq W\\
    g[i,0]  &:=& 0          & \forall 1 \leq i \leq n\\
    g[i,w]  &:=& g[i-1,w]   & \text{if } w - w_i \leq 0\\
    g[i,w]  &:=& \min \{ g[i-1,w] , g[i-1 , w - w_i] + g_i\}
\end{array}$$

After filling the array we can retrieve the maximal value by computing\\
$\underset{0 \leq w \leq W}{\text{argmax}} \{ g[n,w] \; | \; g[n,w] \leq G \}$.\\

This algorithm can be easily implemented through a double for loop, comparing
both cases. The result can be computed in a single for loop.

\begin{lemma}\label{ex1:t2:knapcorrect}
    The afore described algorithm computes the optimal solution.
\end{lemma}

\begin{lemma}\label{ex1:t2:knapruntime}
    The afore described algorithm can be computed in $O(n \cdot W)$.
\end{lemma}

\textbf{Proof \ref{ex1:t2:knapcorrect}.}\\

Let $I$ be the set of objects which leads to the solution $V$ of the algorithm.\\
By construction of the solution, we only accept fields in the array, that has a weight of maximal $G$.
Therefore the constraint is matched. And we summed all weights up correctly.\\
Now assume, there exists a solution $V'$ corresponding to a set of items $I'$, such that $V' > V$.\\
The solution $I$ was build, by taking the maximal value, of all possible.
Therefore the array has to be constructed wrong.\\

Therefore $g[n,V'] > G$ in our construction. But now we now, there exist the set $I'$ which gives us a way, through
the construction. (At each point we can take an item of the set and the resulting weight is smaller than $G$).
Therefore $g[n,V']$ cannot be greater than $G$.\\

\mbox{} \hfill $\square$

\textbf{Proof \ref{ex1:t2:knapruntime}.}\\

As mentioned afore, we can implement the algorithm through a for loop $i := 1 to n$ and
a internal loop $w := 1 to W$. In each step, we look at 2 places already computed,
add and subtract a constant number ($c > 0$) of values and take the minimal value of both.\\
Assuming that array access and addition is constant time, the algorithm runs in\\
first loop $\times$ second loop $\times c$ addition $\times 2 $ array access + third loop $\times 2\times $ compare\\
$T(n) \in O(n \cdot W \; c \cdot 1 \; 2 + W \cdot 2) = O(n \cdot W)$

\mbox{} \hfill $\square$

\subsection*{(c)}

For the $(1 - \varepsilon)$ - approximation we also use a trick learned in approximation algorithms, based on the algorithm of $(b)$.\\

The trick is to round the values of $w_i$, so we do not have to look at each single value.\\

\pagebreak

We assign new values by the following scheme

\begin{lstlisting}
M $\leftarrow$ $\max_{i\in I} w_i$
$\mu$ $\leftarrow$ $\varepsilon \frac{M}{n}$
w_i' $\leftarrow$ $\lfloor w_i / \mu \rfloor$ for all $i \in S$
\end{lstlisting}

and run the algorithm from $(b)$ on it.
With the modified Values we can skip the array in $\mu$ steps.
$$
    W' = \underset{i=1}{\overset{n}{\sum}} w_i' = \underset{i=1}{\overset{n}{\sum}} \left\lfloor \frac{w_i}{\varepsilon M/n} \right\rfloor = O(\frac{n^2}{\varepsilon}).
$$
Because now $W'$ is polynomial in $n$ and $\varepsilon$ it is strictly smaller than $B$ which is exponantial in the input size. So the runtime of our algorithm
is $O(n^3 \cdot \frac{1}{\varepsilon})$, which is in $poly(n,\frac{1}{\varepsilon})$.\\

Leaves us to proof, that the algorithm is a $(1-\varepsilon)$ - approximation.
Let $I$ be the set which gives us the optimal solution on $W$ and $O$ the set, that gives use the optimal solution on $W'$ the rounded values.\\

First observe, that $M \leq OPT$, because we can always can take the most valueable item. The rounding of the values gives us the estimation 
$ \mu w_i' \leq w_i \leq \mu (w_i' + 1)$ wich leads us to $\mu w_i'  \geq w_i - \mu$.

$$\begin{array}{rcl}
    \underset{i \in O}{\sum} w_i &\geq& \mu \underset{i \in O}{\sum} w_i'\\
        &\geq& \mu \underset{i \in I}{\sum} w_i'\\
        &\geq& (\underset{i \in I}{\sum} w_i) - |I|\mu\\
        &\geq& (\underset{i \in I}{\sum} w_i) - n \mu\\
        &=& (\underset{i \in I}{\sum} w_i) - \varepsilon M\\
        &\geq& OPT - \varepsilon OPT = (1-\varepsilon) OPT
\end{array}$$ 

We can conclude, that our algorithm is an FPTAS for the knapsack problem.
