\section*{Exercise 2}

In the knapsack problem, we are given $n$ items. Each item has a \emph{weight} $g_i$ and a \emph{value} $w_i$.
Furthermore , we have a maximum weight $G$. All inputs are positive integers.\\

We would like to find a set $I \subseteq \{ 1,...,n \}$ of items, such that the total value
$\underset{i \in I}{\sum} w_i$ is maximum, subject to the constraint that the total weight is at most
$G$, meaning $\underset{i \in I}{\sum} g_i \leq G$.

\subsection*{(a)} 

Define an appropriate decision version for the knapsack problem and show that it is NP-complete.\\

\textbf{Solution:}\\
We choose the canonical extension of an optimisation problem to a decision problem. Let the defined variables
be as the the optimisation problem. Than take an additional positive Integer $K$.\\

The decision question is now, does there exist a set $I \subseteq \{ 1, ... , n \}$, such that
$\underset{i \in I}{\sum} w_i \geq K$ and $\underset{ i \in I}{\sum} g_i \leq G$.

Let $KNAP$ be this problem.

\begin{lemma}\label{cmp:ex1:knap:np}
The decision problem for the knapsack problem is NP-complete.
\end{lemma}
\textbf{Proof \ref{cmp:ex1:knap:np}.}\\
\begin{enumerate}[i)]
    \item $KNAP \in NP$.\\
        Let $I \subset [n]$ be the optimal solution. Obviously $|I| < n^c$ for some constant $c>0$,
        because $I$ is itsef part of the input.\\
        Now construct $TM V$ as follows.\\
        Compute $\underset{i \in I}{\sum} w_i = w'$ and\\
        $\underset{i \in I}{\sum} g_i = g'$.\\
        Accept if and only if $w' \geq K$ and $ g' \leq G$.\\

        Both sums can be computed in polynomial time and the final comparison can be computed
        in logarithmic time in the length of the values.
        So $V$ accepts $(w,I)$ in $T(n) \in O(n^c)$ for some constant $c>0$.

    \item SUBSET-SUM $\prec_p$ KNAP.\\
        To reduce subset-sum to knapsack we choose the following transformation. Let $S$ be the set of
        positive integers for subset-sum and $T$ a value. The question for knapsack is now, does there exist
        a subset $O \subseteq S$ such that $\underset{i \in O}{\sum} i = T$.\\

        The reduction now sets $w_i = g_i = i$ for all $i \in S$ and $G = K = T$ for the goal values. One can
        easily see, that the reduction can be done in linear time because we only copy values, thus it can be computed in polynomial time. \\

        Knapsack run on this input will now hold the following. If there exists a set $I$ that satisfies the constraints
        $\underset{ i \in I}{\sum} w_i = \underset{i \in I} i \geq K$ and $\underset{i \in I}{\sum} g_i = \underset{i \in I}{\sum} i \leq K$.
        Thus if we set $O = I$ $\underset{i \in O}{\sum} i = T$ holds.

        If no solution existed in the original problem, than we cannot find a solution in the reduction, because both sums
        will sum up to exactly the same. So one of the constraints cannot be true.
\end{enumerate}


\subsection*{(b)}

Let $W := \underset{i=1}{\overset{n}{\sum}} w_i$. Show that the knapsack problem can be solved in $O(nW)$ time. Why does this not
contradict \emph{(a)}?\\

\textbf{Solution:}\\
We proof this by giving a dynamic program, we used in approximation algorithms.
We construct an Array $A$ of sets of tupel. A tupel $(t,w) \in A(j)$ means. There exists a 
subset $I \subseteq \{1,..,j \}$, such that $\underset{i \in I}{\sum} w_i = w$ and $\underset{i \in I}{\sum} g_i = t$.\\

Furthermore we introduce a \emph{domination} relation. A tupel $(t,w)$ dominates another tupel $(t',w')$, iff $t \leq t'$ and $w \geq w'$.
(Means we can get a bigger value with less weight).
Observe, that we can have at most $G+1$ and $W+1$ tupels in $A(j)$ for any $j\leq n$, because otherwise there is a tupel $(t',w') \in A(j)$
that is dominated by another tupel.
Next we assume, that we can put any item in the bag itself. If not, we can throw them away.

\begin{lstlisting}[float=tbh]
A(1) $\leftarrow$ {(0,0),($g_1$,$w_1$)}
for j $\leftarrow$ 2 to n do
    A(j) $\leftarrow$ A(j-1)
    for each (t,w) $\in$ A(j-1) do
        if t + $g_j$ $\leq$ G then
            A(j) $\leftarrow$ A(j) $\cup$ {(t + $g_j$, w + $w_j$)}
    Remove dominated pairs
return $\underset{(t,w)\in A(n)}{\max}$ w
\end{lstlisting}

In each iteration we make a constant number of steps for each element of the previous set of tupels. We showed, that in any set $A(j)$ we can bound
the number of tupels by $\min \{ G + 1, W + 1 \}$. We iterate in the outer loop for any element in our bag. \\

Therefore we can write, that the runtime of our algorithm lies in $O(nW)$. (If $G$ is smaller than $W$ our estimation is to big, but still holds).\\

This solution does not contradict with $(a)$ because $W$ is epxonantial in the inpu size (which is $log \, W$). The algorithm is pseudo-polynomial.

\subsection*{(c)}

For the $(1 - \varepsilon)$ - approximation we also use a trick learned in approximation algorithms, based on the algorithm of $(b)$.\\

The trick is to round the values of $w_i$, so we do not have to look at each single value.\\

We assign new values by the following scheme

\begin{lstlisting}
M $\leftarrow$ $\max_{i\in I} w_i$
$\mu$ $\leftarrow$ $\varepsilon \frac{M}{n}$
w_i' $\leftarrow$ $\lfloor w_i / \mu \rfloor$ for all $i \in S$
\end{lstlisting}

and run the algorithm from $(b)$ on it.
With the modified Values we now get
$$
W' = \underset{i=1}{\overset{n}{\sum}} w_i' = \underset{i=1}{\overset{n}{\sum}} \left\lfloor \frac{w_i}{\varepsilon M/n} \right\rfloor = O(\frac{n^2}{\varepsilon}).
$$
Because now $W'$ is polynomial in $n$ and $\varepsilon$ it is strictly smaller than $B$ which is exponantial in the input size. So the runtime of our algorithm
is $O(n^3 \cdot \frac{1}{\varepsilon})$, which is in $poly(n,\frac{1}{\varepsilon})$.\\

Leaves us to proof, that the algorithm is a $(1-\varepsilon)$ - approximation.
Let $I$ be the set which gives us the optimal solution on $W$ and $O$ the set, that gives use the optimal solution on $W'$ the rounded values.\\

First observe, that $M \leq OPT$, because we can always can take the most valueable item. The rounding of the values gives us the estimation 
$ \mu w_i' \leq w_i \leq \mu (w_i' + 1)$ wich leads us to $\mu w_i'  \geq w_i - \mu$.

$$\begin{array}{rcl}
    \underset{i \in O}{\sum} w_i &\geq& \mu \underset{i \in O}{\sum} w_i'\\
        &\geq& \mu \underset{i \in I}{\sum} w_i'\\
        &\geq& (\underset{i \in I}{\sum} w_i) - |I|\mu\\
        &\geq& (\underset{i \in I}{\sum} w_i) - n \mu\\
        &=& (\underset{i \in I}{\sum} w_i) - \varepsilon M\\
        &\geq& OPT - \varepsilon OPT = (1-\varepsilon) OPT
\end{array}$$ 

We can conclude, that our algorithm is an FPTAS for the knapsack problem.
