\section*{Exercise 1} 

Let $n$ be the amount of different Twilight stickers and $X$ be the random variable representing the the required number of bags to achieve all $n$ stickers.
Each bag contains a sticker that is chosen uniformly at random.\\
We would like to find the expected value $E[X]$.

%% ----------------------------------------------------
%%                          1 a)
%% ----------------------------------------------------

\subsection*{(a)}

Let $X_i$ be the random variable that represents the length of round $i$, where a round $i$ ends, if we achieve a sticker, different from the $i+1$ we
achieved thus far.

\begin{lemma}\label{ex1:t1:linearity}
    $E[X] = \overset{n}{\underset{i=1}{\sum}} E[X_i]$.
\end{lemma}

\textbf{Proof \ref{ex1:t1:linearity}.}\\
By construction $X = \overset{n}{\underset{i=1}{\sum}} X_i$ holds, because in each round we obtain a new sticker
and sum up all bags need in each round.\\
By linearity of the expected value it follows immediately that
$$
    E[X] = E \left[ \sum_{i=1}^{n} X_i \right] \stackrel{Lin.}{=} \sum_{i=1}^{n} E [X_i]
$$
holds.
\mbox{} \hfill $\square$

%% ----------------------------------------------------
%%                          1 b)
%% ----------------------------------------------------


\subsection*{(b)}

\begin{lemma}\label{ex1:t1:rounds}
    $E[X_i] = \frac{n}{n-i+1}$.
\end{lemma}

\textbf{Proof \ref{ex1:t1:rounds}.}\\
First we show, that $E[ X_i ]$ is a geometric distribution, meaning\\
$Pr (X_i = k) = (1 - p_i)^{k-1} \cdot p_i$ holds,
where $p_i$ is the probability to get a new card.\\

The event $X_i$ is described as the first time, we achieve a new card. If $X_i = k$ the first $k-1$ bags
may not contain a new card, which is the complementary event with probability $(1- p_i)$.\\

In terms of probability $Pr(X_i = k) = (1 - p_i)^{k-1} \cdot p_i$ holds.\\

Because $E [X_i ]$ is geometric distributed we now conclude
$E[ X_i ] = \frac{1}{p_i}$.\\

In the last step we now, that we already have $i-1$ stickers in the $i$-th round. So the probability
in a uniform distribution to get a new sticker is $p_i = \frac{n - i + 1}{n}$.\\
By the previous formula the claim $E[ X_i ] = \frac{n}{n - i + 1}$ follows.

\mbox{} \hfill $\square$
%% ----------------------------------------------------
%%                          1 c)
%% ----------------------------------------------------


\subsection*{(c)}

\begin{lemma}\label{ex1:t1:expected}
    $E[X]= O(n\, \log\, n)$.
\end{lemma}


\textbf{Proof \ref{ex1:t1:expected}.}\\
\[
E[X]\stackrel{(a)}{=}\sum_{i=1}^{n}E[X_{i}]\stackrel{(b)}{=}\sum_{i=1}^{n}\frac{n}{n-i+1}=n\sum_{i=1}^{n}\frac{1}{n-i+1}
\]


\[
=n\sum_{i=1}^{n}\frac{1}{i}=n\cdot O(\log\: n)=O(n\, \log\, n)
\]


\mbox{} \hfill $\square$
