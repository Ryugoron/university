\documentclass[11pt,a4paper,ngerman]{article}
\usepackage[bottom=2.5cm,top=2.5cm]{geometry} 
\usepackage{babel}
\usepackage[utf8]{inputenc} 
\usepackage[T1]{fontenc} 
\usepackage{ae} 
\usepackage{amssymb} 
\usepackage{amsmath}
\usepackage{amsthm} 
\usepackage{graphicx}
\usepackage{fancyhdr}
\usepackage{fancyref}
\usepackage{listings}
\usepackage{xcolor}
\usepackage{paralist}

\usepackage[pdftex, bookmarks=false, pdfstartview={FitH}, linkbordercolor=white]{hyperref}
\usepackage{fancyhdr}
\pagestyle{fancy}
\fancyhead[C]{Complexity Theory}
\fancyhead[L]{Exercise sheet 1}
\fancyhead[R]{WiSe 2012}
\fancyfoot{}
\fancyfoot[L]{}
\fancyfoot[C]{\thepage \hspace{1px} of \pageref{LastPage}}
\renewcommand{\footrulewidth}{0.5pt}
\renewcommand{\headrulewidth}{0.5pt}
\setlength{\parindent}{0pt} 
\setlength{\headheight}{0pt}

\date{}
\title{Exercise Sheet 1}
\author{}


%%
%% Enviroments for proofs and lemmas
%%
\newtheorem{lemma}{\bfseries Claim}

\begin{document}

\lstset{language=Pascal, basicstyle=\ttfamily\fontsize{10pt}{10pt}\selectfont\upshape, commentstyle=\rmfamily\slshape, keywordstyle=\rmfamily\bfseries, breaklines=true, frame=single, xleftmargin=3mm, xrightmargin=3mm, tabsize=2, mathescape=true}

\renewcommand{\figurename}{Figure}

\maketitle
\thispagestyle{fancy}

%%\section*{Exercise 1} 

Let $n$ be the amount of different Twilight stickers and $X$ be the random variable representing the the required number of bags to achieve all $n$ stickers.
Each bag contains a sticker that is chosen uniformly at random.\\
We would like to find the expected value $E[X]$.

%% ----------------------------------------------------
%%                          1 a)
%% ----------------------------------------------------

\subsection*{(a)}

Let $X_i$ be the random variable that represents the length of round $i$, where a round $i$ ends, if we achieve a sticker, different from the $i+1$ we
achieved thus far.

\begin{lemma}\label{ex1:t1:linearity}
    $E[X] = \overset{n}{\underset{i=1}{\sum}} E[X_i]$.
\end{lemma}

\textbf{Proof \ref{ex1:t1:linearity}.}\\
By construction $X = \overset{n}{\underset{i=1}{\sum}} X_i$ holds, because in each round we obtain a new sticker
and sum up all bags need in each round.\\
By linearity of the expected value it follows immediately that
$$
    E[X] = E \left[ \sum_{i=1}^{n} X_i \right] \stackrel{Lin.}{=} \sum_{i=1}^{n} E [X_i]
$$
holds.
\mbox{} \hfill $\square$

%% ----------------------------------------------------
%%                          1 b)
%% ----------------------------------------------------


\subsection*{(b)}

\begin{lemma}\label{ex1:t1:rounds}
    $E[X_i] = \frac{n}{n-i+1}$.
\end{lemma}

\textbf{Proof \ref{ex1:t1:rounds}.}\\
First we show, that $E[ X_i ]$ is a geometric distribution, meaning\\
$Pr (X_i = k) = (1 - p_i)^{k-1} \cdot p_i$ holds,
where $p_i$ is the probability to get a new card.\\

The event $X_i$ is described as the first time, we achieve a new card. If $X_i = k$ the first $k-1$ bags
may not contain a new card, which is the complementary event with probability $(1- p_i)$.\\

In terms of probability $Pr(X_i = k) = (1 - p_i)^{k-1} \cdot p_i$ holds.\\

Because $E [X_i ]$ is geometric distributed we now conclude
$E[ X_i ] = \frac{1}{p_i}$.\\

In the last step we now, that we already have $i-1$ stickers in the $i$-th round. So the probability
in a uniform distribution to get a new sticker is $p_i = \frac{n - i + 1}{n}$.\\
By the previous formula the claim $E[ X_i ] = \frac{n}{n - i + 1}$ follows.

\mbox{} \hfill $\square$
%% ----------------------------------------------------
%%                          1 c)
%% ----------------------------------------------------


\subsection*{(c)}

\begin{lemma}\label{ex1:t1:expected}
    $E[X]= O(n\, \log\, n)$.
\end{lemma}


\textbf{Proof \ref{ex1:t1:expected}.}\\
\[
E[X]\stackrel{(a)}{=}\sum_{i=1}^{n}E[X_{i}]\stackrel{(b)}{=}\sum_{i=1}^{n}\frac{n}{n-i+1}=n\sum_{i=1}^{n}\frac{1}{n-i+1}
\]


\[
=n\sum_{i=1}^{n}\frac{1}{i}=n\cdot O(\log\: n)=O(n\, \log\, n)
\]


\mbox{} \hfill $\square$


\section*{Exercise 1} 

Let $n$ be the amount of different Twilight stickers and $X$ be the random variable representing the the required number of bags to achieve all $n$ stickers.
Each bag contains a sticker that is chosen uniformly at random.\\
We would like to find the expected value $E[X]$.

%% ----------------------------------------------------
%%                          1 a)
%% ----------------------------------------------------

\subsection*{(a)}

Let $X_i$ be the random variable that represents the length of round $i$, where a round $i$ ends, if we achieve a sticker, different from the $i+1$ we
achieved thus far.

\begin{lemma}\label{ex1:t1:linearity}
    $E[X] = \overset{n}{\underset{i=1}{\sum}} E[X_i]$.
\end{lemma}

\textbf{Proof \ref{ex1:t1:linearity}.}\\
By construction $X = \overset{n}{\underset{i=1}{\sum}} X_i$ holds, because in each round we obtain a new sticker
and sum up all bags need in each round.\\
By linearity of the expected value it follows immediately that
$$
    E[X] = E \left[ \sum_{i=1}^{n} X_i \right] \stackrel{Lin.}{=} \sum_{i=1}^{n} E [X_i]
$$
holds.
\mbox{} \hfill $\square$

%% ----------------------------------------------------
%%                          1 b)
%% ----------------------------------------------------


\subsection*{(b)}

\begin{lemma}\label{ex1:t1:rounds}
    $E[X_i] = \frac{n}{n-i+1}$.
\end{lemma}

\textbf{Proof \ref{ex1:t1:rounds}.}\\
First we show, that $E[ X_i ]$ is a geometric distribution meaning\\
$Pr (X_i = k) = (1 - p_i)^{k-1} \cdot p_i$ holds,
where $p_i$ is the probability to get a new card.\\

The event $X_i$ is described as the first time, we achieve a new card. If $X_i = k$ the first $k-1$ bags
may not contain a new card, which is the complementary event with probability $(1- p_i)$.\\

In terms of probability $Pr(X_i = k) = (1 - p_i)^{k-1} \cdot p_i$ holds.\\

Because $E [X_i ]$ is geometric distributed we now conclude
$E[ X_i ] = \frac{1}{p_i}$.\\

In the last step we now, that we already have $i-1$ stickers in the $i$-th round. So the probability
in a uniform distribution to get a new sticker is $p_i = \frac{n - i + 1}{n}$.\\
By the previous formula the claim $E[ X_i ] = \frac{n}{n - i + 1}$ follows.

\mbox{} \hfill $\square$
%% ----------------------------------------------------
%%                          1 c)
%% ----------------------------------------------------


\subsection*{(c)}

\begin{lemma}\label{ex1:t1:expected}
    $E[X]= O(n\, \log\, n)$.
\end{lemma}


\textbf{Proof \ref{ex1:t1:expected}.}\\
\[
E[X]\stackrel{(a)}{=}\sum_{i=1}^{n}E[X_{i}]\stackrel{(b)}{=}\sum_{i=1}^{n}\frac{n}{n-i+1}=n\sum_{i=1}^{n}\frac{1}{n-i+1}
\]


\[
=n\sum_{i=1}^{n}\frac{1}{i}=n\cdot O(\log\: n)=O(n\, \log\, n)
\]


\mbox{} \hfill $\square$


%%\subsection*{\itshape 5. Hausdorff-Maß und Lebesque-Maß}

Seien $\mathcal{L}$ das Lebesque-Maß und $\mathcal{H}^1_\delta$ das approximative Hausdorff-Maß auf $\mathbb{R}$. Zeigen Sie,
dass für alle $A \subset \mathbb{R}$ gilt

$$
    \mathcal{L}^1(A) = H^1_\delta(A) \qquad \text{für alle }\delta > 0
$$

\textbf{Beweis:}\\

Um dies zu zeigen, beweise ich zunächst das folgende.

\begin{lemma}\label{ana3:ueb2:ex2:lebesque}
    Im $\mathbb{R}^1$ ist für einen Würfel $I \in \mathcal{K} = (a,b)$, sein Volumen $I = b - a$, das Lebesgue-Maß $\mathcal{L}^1 (I) = b - a$
    und das approximierte Hausdorff-Maß $\mathcal{H}^1_\delta (I) = b - a$ für alle $\delta > 0$.
\end{lemma}

\begin{lemma}\label{ana3:ueb3:ex2:offen}
    Sei $(a,b]$ ein Intervall. Dann ist ist $\mathcal{L}^1 ((a,b]) = \mathcal{L}^1 ((a,b))$. D.h. es ist egal, ob
    die Intervall geschlossen oder offen sind. (Ebenso $[a,b]$, $[a,b)$).
\end{lemma}

\textbf{Schlussfolgerung:} Für das Lebesgue-Maß ist es egal, welche Intervalle wir benutzten.\\

\textbf{Beweis \ref{ana3:ueb2:ex2:lebesque}.}\\

Das Volumen eines Würfels gerade als $| (a_1,b_1)\times ... \times (a_n,b_n)| = (b_1 - a_1) \cdot ... \cdot (b_n - a_n)$ definiert.
Nun muss für das Lebesque-Maß die Menge mit Würfeln übderdeckt werden. Sei also $(I_i)_{i \in \mathbb{N}}$ die Überdeckung, mit $I_1 = I$, sonst $\emptyset$.
Damit ist $\mathcal{L}^1 (I) = \overset{\infty}{\underset{i=1}{\sum}} |I_i| = |I_1| = b - a$.\\

Für das Hausdorff - Maß underscheiden wir zunächst. Ist $b - a < \delta$ können wir es direkt mit dem Würfel überdecken. Ist $\delta$ zu klein wählen wir als 
die Überdeckung $B_i = (a, a+ \frac{\delta}{b-a}], (a + \frac{\delta}{b-a}, a + 2 \frac{\delta}{b-a}], ... (b - \frac{\delta}{b-a}, b), \emptyset, ...$ und die restlichen
Mengen sind $\emptyset$. Wie man erkennt, ist nun $\mathcal{H}^m_\delta (I) = \overset{\infty}{\underset{i=1}{\sum}} \omega_1 \left(\frac{B_i}{2} \right) = b - a$, da
$\omega_1 = 2$ ist.\\
\mbox{} \hfill $\square$\\

\textbf{Beweis \ref{ana3:ueb3:ex2:offen}}\\

Sei $(a,b]$ das Intervall. Dann wird es durch $(a,b)$ und $(b - \varepsilon, b + \varepsilon)$ für jedes $\varepsilon > 0$ übderdeckt.
Seien $(B_i)_{i \in \mathbb{N}}$ offene Würfel, so dass $B_1 = (a,b)$, $B_2 = (b - \varepsilon, b - \varepsilon)$, $B_i = \emptyset$ $\forall i > 2$.\\
Nun ist $\overset{\infty}{\underset{i=1}{\sum}} B_i = b - a + 2 \varepsilon$. Lässt man nun $\varepsilon$ gegen 0 laufen, erhält man nur noch $b-a$.
Es kann auch nicht weniger werden, da man sonst einen Wert nicht mehr erreicht.\\
\mbox{} \hfill $\square$\\

Nun zeigen wir den Satz:\\
$\mathcal{L}^1(A) \geq \mathcal{H}^1_\delta (A)$.\\

Dies gilt trivialerweise. Sei $(I_i)_{i \in \mathbb{N}}$ eine überdeckung mit Würfeln sodass
$\overset{\infty}{\underset{i=1}{\sum}} |I_i| \leq \mathcal{L}^1 (A) + \varepsilon$, mit Seitenlängen, maximal $\delta$.
Dann ergibt sich, dass $\overset{\infty}{\underset{i=1}{\sum}} \omega_1 \frac{\diam I_i}{2} \leq \mathcal{L}^1 (a) + \varepsilon$.
Lässt man $\varepsilon$ nun gegen $0$ laufen, drehen sich die relationen nicht um. Da nun das approximative Hausdorff-Maß als das Infimum
aller Überdeckungen definiert ist, muss es kleiner gleich diesem sein.\\

$\mathcal{L}^1 (A) \leq \mathcal{H}^1_\delta (A)$.\\

Sei $(B_i)_{i \in \mathbb{N}}$ eine Überdeckung für $A$, mit $\diam B_i < \delta$ für alle $i \in \mathbb{N}$.\\
Wir können davon ausgehen, dass $B_i$ nur aus Intervallen besteht.
Nehmen wir an $B_j$ für ein $j \in \mathbb{N}$ wäre kein Intervall, dann gäbe es eine Menge von Intervallen
$(I_k)_{k \in \mathbb{N}}$, so dass $B_j = \bigcup I_k$. Nehmen wir ferner an, sie sind sortiert.
$I_1 = (a_1,b_1) , ... , I^\infty = (a_\infty, b_\infty)$. Dann ist\\
$\overset{\infty}{\underset{i=1}{\sum}} \omega_1 \frac{b_i - a_i}{2} \leq b_\infty - a_\infty$. Damit ist die Überdeckung für das
Hausdorff-Maß aus Intervallen gebildet, da diese kleiner sind.\\

Da wir nun nur noch Intervalle betrachten müssen für unsere Überdeckung von $A$, können wir die selbe
Überdeckung für das Lebesgue-Maß wählen (Dies ist nach 2 Möglich, da wir 
uns nicht um offene oder abgeschlossene Mengen kümmern müssen). Wie schon öfters gesehen, ist die Berechnugsformel der beiden im 
ein-dimensionalen gleich, falls die Überdeckung aus Würfeln mit Durchmesser kleiner als $\delta$ gebildet wird.\\

\mbox{} \hfill $\square$


\section*{Exercise 2}

In the knapsack problem, we are given $n$ items. Each item has a \emph{weight} $g_i$ and a \emph{value} $w_i$.
Furthermore , we have a maximum weight $G$. All inputs are positive integers.\\

We would like to find a set $I \subseteq \{ 1,...,n \}$ of items, such that the total value
$\underset{i \in I}{\sum} w_i$ is maximum, subject to the constraint that the total weight is at most
$G$, meaning $\underset{i \in I}{\sum} g_i \leq G$.

\subsection*{(a)} 

Define an appropriate decision version for the knapsack problem and show that it is NP-complete.\\

\textbf{Solution:}\\
We choose the canonical extension of an optimization problem to a decision problem. Let the defined variables
be as the the optimization problem. Than take an additional positive Integer $K$.\\

The decision question is "Does there exist a set $I \subseteq \{ 1, ... , n \}$ such that
$\underset{i \in I}{\sum} w_i \geq K$ and $\underset{ i \in I}{\sum} g_i \leq G$".

Let KNAP be this problem.

\begin{lemma}\label{cmp:ex1:knap:np}
The decision problem for the knapsack problem is NP-complete.
\end{lemma}
\textbf{Proof \ref{cmp:ex1:knap:np}.}\\
\begin{enumerate}[i)]
    \item KNAP $\in$ NP.\\
        Let $I \subset \{ 1, ... , n \}$ be the optimal solution. Obviously $|I| < n^c$ for some constant $c>0$,
        because $I$ is itself part of the input.\\
        Let the Verifier $V$ be defined as follows.\\
        Compute $\underset{i \in I}{\sum} w_i = w'$ and 
        $\underset{i \in I}{\sum} g_i = g'$.\\
        Accept if and only if $w' \geq K$ and $ g' \leq G$.\\

        Both sums can be computed in polynomial time and the comparison can be computed
        in logarithmic time in the length of the values.
        $V$ accepts $(w,I)$ in $T(n) \in O(n^c)$ for some constant $c>0$.

    \item SUBSET-SUM $\prec_p$ KNAP.\\
        To reduce subset-sum to knapsack we choose the following transformation. Let $S$ be the set of
        positive integers for subset-sum and $T$ a value. The question for knapsack is now, does there exist
        a subset $O \subseteq S$ such that $\underset{i \in O}{\sum} i = T$.

        The reduction function $f$ sets $w_i := g_i := i$ for all $i \in S$ and $G := K := T$. One can
        easily see, that the reduction can be done in linear time because we only copy values. \\

        $w \in$ SUBSET-SUM $\Rightarrow f(w) \in$ KNAP.\\
        There exists a subset $O \subset S$ such that $\underset{i \in O}{\sum} i = T$.
        After the application of $f$ we know that $O \subset I$ and
        $\underset{i \in O} w_i = K$ and $\underset{i \in O} g_i = G$. Therefore $O$ is a solution for
        KNAP on this input.\\

        $f(w) \in$ KNAP $\Rightarrow w \in$ SUBSET-SUM.\\
        There exists a subset $O \subset I$ such that $\underset{i \in O}{\sum} g_i \leq G$ and $\underset{i \in O}{\sum} w_i \geq K$.
        From the reduction $f$ we know that $\underset{i \in O}{\sum} g_i = \underset{i \in O}{\sum} w_i = \underset{i \in O}{\sum} i$ holds.\\
        $\Rightarrow \underset{i \in O}{\sum} i \leq G \land \underset{i \in O}{\sum} i \geq K$.\\
        Again from $f$ we know that $G = K = T$ holds.\\
        $\Rightarrow \underset{i \in O}{\sum} i = T$.\\
\mbox{} \hfill $\square$
\end{enumerate}


\subsection*{(b)}

Let $W := \underset{i=1}{\overset{n}{\sum}} w_i$. Show that the knapsack problem can be solved in $O(nW)$ time. Why does this not
contradict \emph{(a)}?\\

\textbf{Solution:}\\

We show that a dynamic program solves the knapsack problem and has the runtime of $O(nW)$.\\
Let $g[n,W]$ be a two dimensional array, that stores the in $g[i,w]$ the minimal weight, for the first $i$ objects
with an value exactly $w$.

We can now give the recursive definition of to fill in $w$. Let the array be initialized with $\infty$.

$$\begin{array}{lclr}
    g[0,w]  &:=& 0          & \forall 0 \leq w \leq W\\
    g[i,0]  &:=& 0          & \forall 1 \leq i \leq n\\
    g[i,w]  &:=& g[i-1,w]   & \text{if } w - w_i \leq 0\\
    g[i,w]  &:=& \min \{ g[i-1,w] , g[i-1 , w - w_i] + g_i\}
\end{array}$$

After filling the array we can retrieve the maximal value by computing\\
$\underset{0 \leq w \leq W}{\text{argmax}} \{ g[n,w] \; | \; g[n,w] \leq G \}$.\\

This algorithm can be easily implemented through a double for loop comparing
both cases. The result can be computed in a single for loop.

The algorithm computes the optimal solution, because it computes every possible solution and takes the greatest value.\\
The computation takes $O(nW)$ time because the array has $n \times W$ cells and in each cell we have an constant
computation time $O(1)$.
To find the maximal value we iterate over $W$ cells and compute the maximum which can be done in time $O(W)$.\\
Therefore $T(n) = O(nW)$ holds.\\

This does not contradict with \emph{(a)} because $W$ can be exponential in the size of the input. The runtime of the algorithm
is pseudo polynomial.

\subsection*{(c)}

For the $(1 - \varepsilon)$ - approximation we use an algorithm based on the algorithm of $(b)$.

The trick is to round the values of $w_i$ such that we do not have to look at each possible value of $w$.\\

We assign new values by the following scheme

\begin{lstlisting}
M $\leftarrow$ $\max_{i\in I} w_i$
$\mu$ $\leftarrow$ $\varepsilon \frac{M}{n}$
w_i' $\leftarrow$ $\lfloor w_i / \mu \rfloor$ for all $i \in S$
\end{lstlisting}

and run the algorithm from $(b)$ on it.
With the modified values we can skip the array in $\mu$ steps.
$$
\frac{W}{\mu} = \frac{n}{\varepsilon \cdot M} \overset{n}{\underset{i=1}{\sum}} w_i \stackrel{M \geq w_i}{\leq} \frac{n}{\varepsilon} \overset{n}{\underset{i=1}{\sum}} 1 = O (\frac{n^2}{\varepsilon})
$$
The new grid has size $n \times O (\frac{n^2}{\varepsilon})$ therefore the runtime of the algorithm on the modified values is
in $poly(n, \frac{1}{\varepsilon})$.\\
Leaves us to proof, that the algorithm is a $(1-\varepsilon)$ - approximation.
Let $I$ be the set which gives us the optimal solution on $W$ and $O$ the set, that gives use the optimal solution on $W'$ the rounded values.\\

First observe, that $M \leq OPT$, because we can always can take the most valuable item. The rounding of the values gives us the estimation 
$ \mu w_i' \leq w_i \leq \mu (w_i' + 1)$ which leads us to $\mu w_i'  \geq w_i - \mu$.

$$\begin{array}{rcl}
    \underset{i \in O}{\sum} w_i &\stackrel{\text{Def. } w_i'}{\geq}& \mu \underset{i \in O}{\sum} w_i'\\
        &\stackrel{O \text{ opt. on } w_i'}{\geq}& \mu \underset{i \in I}{\sum} w_i'\\
        &\geq& (\underset{i \in I}{\sum} w_i) - |I|\mu\\
        &\stackrel{|I| < n}{\geq}& (\underset{i \in I}{\sum} w_i) - n \mu\\
        &\stackrel{\text{Def. }\mu}{=}& (\underset{i \in I}{\sum} w_i) - \varepsilon M\\
        &\stackrel{I \text{ opt. on }w_i}{\geq}& OPT - \varepsilon OPT = (1-\varepsilon) OPT
\end{array}$$ 

We can conclude that the algorithm is an FPTAS for the knapsack problem.


%%\section*{Exercise 3}

Blabla. Subexp.


\section*{Exercise 3}

True or False? For every $\varepsilon > 0$, there exists an NP - complete problem that can be solved deterministically in
$O(2^{n^\varepsilon})$ steps. Explain your answer.

\textbf{Solution:}\\

We proof this part by construction a language for each $\varepsilon > 0$, that is $NP - complete$ and can be solved in
the given time.\\

Let PAD-SAT$_\varepsilon = \{ (\Psi , 1^{|\Psi|^{\left\lceil \frac{2}{\varepsilon} \right\rceil}}) \; | \; \Psi \text{ satisfiable in 3 cnf} \}$\\

\begin{lemma}\label{ex1:t3:npc}
	$\text{PAD-SAT}_\varepsilon$ is NP-complete.
\end{lemma}

\textbf{Proof \ref{ex1:t3:npc}.}\\
PAD-SAT$_\varepsilon \in$ NP.\\
Given an variable assignment we can check whether the formula $\Psi$ is satisfied in polynomial time. 
The number of variables has to be strictly smaller than the length of the formula. The assignment is polynomial bounded
in the input size hence the witness is polynomial in the size of the input.\\

3-SAT $\prec_p$ PAD-SAT$_\varepsilon$.\\
The reduction function copies the formula $\Psi$, computes $|\Psi|^{\left\lceil \frac{2}{\varepsilon} \right\rceil}$
and appends that many ones to $\Psi$. The computation can be done in polynomial time.
The reduction is valid because if $\Psi$ is satisfiable it remains satisfiable after the reduction and vis-versa.

\begin{lemma}\label{ex1:t3:time}
    There exists a DTM M such that $T_M(n) \in DTIME (2^{n^\varepsilon})$.
\end{lemma}

\textbf{Proof \ref{ex1:t3:time}.}\\
We construct a DTM M that checks the satisfiability by brute-force.
Given a word $w = (\Psi, 1^{|\Psi|^{\left\lceil \frac{2}{k} \right\rceil}})$ with $n = |w|$.\\
It holds that $|\Psi| \leq n^\frac{\varepsilon}{2}$. Otherwise 
$|w| = |\Psi| + |\Psi|^{\left\lceil \frac{2}{k} \right\rceil} > n^\frac{\varepsilon}{2} + (n^\frac{\varepsilon}{2})^{\frac{2}{k}} > n$
holds which is a contradiction.\\

$\Rightarrow$ There exists at most $n^\frac{\varepsilon}{2}$ possible assignments for $\Psi$. We can
compute the result for a given assignment in $n^c$ time for some $c > 0$.

Therefore $T_M (n) \leq n^c \cdot 2^{n^\frac{\varepsilon}{2}}$. At last we prove that $T_M(n) \in O(2^{n^\varepsilon})$.

$$
\lim_{n\rightarrow \infty} \frac{T_M (n)}{2^{n^\varepsilon}} = \lim_{n \rightarrow \infty} n^c \cdot 2^{n^\frac{\varepsilon}{2} - n^\varepsilon}
= n^c \cdot 2^{-n^\frac{\varepsilon}{2}} = 0
$$
By convergence criterion we know that there exist the DTM M that computes a NP complete problem in $O(2^{n^\varepsilon})$ time.\\
\mbox{} \hfill $\square$

\label{LastPage}

\end{document}
