\documentclass[11pt,a4paper,ngerman]{article}
\usepackage[bottom=2.5cm,top=2.5cm]{geometry} 
\usepackage{babel}
\usepackage[utf8]{inputenc} 
\usepackage[T1]{fontenc} 
\usepackage{ae} 
\usepackage{amssymb} 
\usepackage{amsmath}
\usepackage{amsthm} 
\usepackage{graphicx}
\usepackage{fancyhdr}
\usepackage{fancyref}
\usepackage{listings}
\usepackage{xcolor}
\usepackage{paralist}

\usepackage[pdftex, bookmarks=false, pdfstartview={FitH}, linkbordercolor=white]{hyperref}
\usepackage{fancyhdr}
\pagestyle{fancy}
\fancyhead[C]{Complexity Theory}
\fancyhead[L]{Exercise sheet 1}
\fancyhead[R]{WiSe 2012}
\fancyfoot{}
\fancyfoot[L]{}
\fancyfoot[C]{\thepage \hspace{1px} of \pageref{LastPage}}
\renewcommand{\footrulewidth}{0.5pt}
\renewcommand{\headrulewidth}{0.5pt}
\setlength{\parindent}{0pt} 
\setlength{\headheight}{0pt}

\date{}
\title{Exercise Sheet 1}
\author{}


%%
%% Enviroments for proofs and lemmas
%%
\newtheorem{lemma}{\bfseries Claim}

\begin{document}

\lstset{language=Pascal, basicstyle=\ttfamily\fontsize{10pt}{10pt}\selectfont\upshape, commentstyle=\rmfamily\slshape, keywordstyle=\rmfamily\bfseries, breaklines=true, frame=single, xleftmargin=3mm, xrightmargin=3mm, tabsize=2, mathescape=true}

\renewcommand{\figurename}{Figure}

\maketitle
\thispagestyle{fancy}

%%\section*{Exercise 1 \mdseries Collecting Stickers}

Let $n$ be the amount of different Twilight stickers and $X$ be the random variable representing the the required number of bags to achieve all $n$ stickers.
The stickers are distributed uniformly in the bags.\\
We would like to find the expected value $E[X]$.

%% ----------------------------------------------------
%%                          1 a)
%% ----------------------------------------------------

\subsection*{(a)}

Let $X_i$ be the random variable that represents the length of round $i$, where a round $i$ ends, if we achiev a sticker, different from the $i+1$ we
achieved thus far.

\begin{lemma}\label{ex1:t1:linearity}
    In the context $E[X] = \overset{n}{\underset{i=1}{\sum}} E[X_i]$ holds.
\end{lemma}

\textbf{Proof \ref{ex1:t1:linearity}.}\\
For $i \not= j$ the events $X_i, X_j$ are independend, because the events take place striklty after another. And a later round does not
depenend on the length of any previous round or the sticker taken.\\
We than now, that after round the first $i$ rounds we have $i$ different stickers. After $n$ rounds we then have all stickers.
The amount of bags needed is then the sum of the length of all rounds.\\
$$
    E[X] = E \left[ \sum_{i=1}^{n} X_i \right] \stackrel{Lin.}{=} \sum_{i=1}^{n} E [X_i]
$$
\mbox{} \hfill $\square$

%% ----------------------------------------------------
%%                          1 b)
%% ----------------------------------------------------


\subsection*{(b)}

\begin{lemma}\label{ex1:t1:rounds}
    In the context $E[X_i] = \frac{n}{n-i+1}$ holds.
\end{lemma}

\textbf{Proof \ref{ex1:t1:rounds}.}\\
First we show, that $E[ X_i ]$ is geometric distributed, meaning\\
$Pr (X_i = k) = (1 - p_i)^{k-1} \cdot p_i$ holds,
where $p_i$ is the possibility to get a new card.\\

The event $X_i$ is described as the first time, we achieve a new card. If $X_i = k$ the first $k-1$ bags
may not contain a new card, which is the complementary event with possibility $(1- p_i)$.\\

In terms of possibily $Pr(X_i = k) = (1 - p_i)^{k-1} \cdot p_i$ holds.\\

Because $E [X_i ]$ is geometric distributed we now conclude
$E[ X_i ] = \frac{1}{p_i}$.\\

In the last step we now, that we already have $i-1$ stickers in the $i$-th round. So the possibility
in a uniform distribution to get a new sticker is $p_i = \frac{n - i + 1}{n}$.\\
By the previous formula the claim $E[ X_i ] = \frac{n}{n - i + 1}$ follows.

\mbox{} \hfill $\square$
%% ----------------------------------------------------
%%                          1 c)
%% ----------------------------------------------------


\subsection*{(c)}

\begin{lemma}\label{ex1:t1:expected}
    For the task $E[X]=O(n\, log\, n)$ holds.
\end{lemma}


\textbf{Proof \ref{ex1:t1:expected}.}\\
\[
E[X]\stackrel{(a)}{=}\sum_{i=1}^{n}E[X_{i}]\stackrel{(b)}{=}\sum_{i=1}^{n}\frac{n}{n-i+1}=n\sum_{i=1}^{n}\frac{1}{n-i+1}
\]


\[
\stackrel{(*)}{=}n\sum_{i=1}^{n}\frac{1}{i}=n\cdot O(log\: n)=O(n\, log\, n)
\]


In the step $(*)$ we reordered the sum, because in the first one we summed $\frac{1}{n} + \frac{1}{n-1} + ... + 1$ and in the second one
we sum $1 + \frac{1}{2} + ... + \frac{1}{n}$. We can simply do this, because the sum is well defined (we only sum a finite amount
real numbers).

\mbox{} \hfill $\square$


\section*{Exercise 1} 

Let $n$ be the amount of different Twilight stickers and $X$ be the random variable representing the the required number of bags to achieve all $n$ stickers.
Each bag contains a sticker that is chosen uniformly at random.\\
We would like to find the expected value $E[X]$.

%% ----------------------------------------------------
%%                          1 a)
%% ----------------------------------------------------

\subsection*{(a)}

Let $X_i$ be the random variable that represents the length of round $i$, where a round $i$ ends, if we achieve a sticker, different from the $i+1$ we
achieved thus far.

\begin{lemma}\label{ex1:t1:linearity}
    $E[X] = \overset{n}{\underset{i=1}{\sum}} E[X_i]$.
\end{lemma}

\textbf{Proof \ref{ex1:t1:linearity}.}\\
By construction $X = \overset{n}{\underset{i=1}{\sum}} X_i$ holds, because in each round we obtain a new sticker
and sum up all bags need in each round.\\
By linearity of the expected value it follows immediately that
$$
    E[X] = E \left[ \sum_{i=1}^{n} X_i \right] \stackrel{Lin.}{=} \sum_{i=1}^{n} E [X_i]
$$
holds.
\mbox{} \hfill $\square$

%% ----------------------------------------------------
%%                          1 b)
%% ----------------------------------------------------


\subsection*{(b)}

\begin{lemma}\label{ex1:t1:rounds}
    $E[X_i] = \frac{n}{n-i+1}$.
\end{lemma}

\textbf{Proof \ref{ex1:t1:rounds}.}\\
First we show, that $E[ X_i ]$ is a geometric distribution meaning\\
$Pr (X_i = k) = (1 - p_i)^{k-1} \cdot p_i$ holds,
where $p_i$ is the probability to get a new card.\\

The event $X_i$ is described as the first time, we achieve a new card. If $X_i = k$ the first $k-1$ bags
may not contain a new card, which is the complementary event with probability $(1- p_i)$.\\

In terms of probability $Pr(X_i = k) = (1 - p_i)^{k-1} \cdot p_i$ holds.\\

Because $E [X_i ]$ is geometric distributed we now conclude
$E[ X_i ] = \frac{1}{p_i}$.\\

In the last step we now, that we already have $i-1$ stickers in the $i$-th round. So the probability
in a uniform distribution to get a new sticker is $p_i = \frac{n - i + 1}{n}$.\\
By the previous formula the claim $E[ X_i ] = \frac{n}{n - i + 1}$ follows.

\mbox{} \hfill $\square$
%% ----------------------------------------------------
%%                          1 c)
%% ----------------------------------------------------


\subsection*{(c)}

\begin{lemma}\label{ex1:t1:expected}
    $E[X]= O(n\, \log\, n)$.
\end{lemma}


\textbf{Proof \ref{ex1:t1:expected}.}\\
\[
E[X]\stackrel{(a)}{=}\sum_{i=1}^{n}E[X_{i}]\stackrel{(b)}{=}\sum_{i=1}^{n}\frac{n}{n-i+1}=n\sum_{i=1}^{n}\frac{1}{n-i+1}
\]


\[
=n\sum_{i=1}^{n}\frac{1}{i}=n\cdot O(\log\: n)=O(n\, \log\, n)
\]


\mbox{} \hfill $\square$


%%\subsection*{\itshape 2. Vollständigkeit von Funktionsräumen}

Für $E \subset \mathbb{R}^n, E \not= \emptyset$ setzen wir
$$
    B(E) := \{ f \, : \, E \rightarrow \mathbb{R} \; | \; f \text{ ist beschränkt} \}.
$$

Ferner definieren wir für zweit Funktionen $f,g \, : \, E \rightarrow \mathbb{R}$ ihren Abstand
$$
    d(f,g) := \underset{x \in E}{\sup} |f(x) - g(x)|.
$$

Zeigen Sie, dass $(B(E),d)$ vollständig ist.\\

\textbf{Lösung:}\\

\textbf{Metrik}:\\
Als erstes müssen wir zeigen, dass es sich bei $(B(E),d)$ um eine Metrik handelt.
\begin{enumerate}[1.]
    \item $\forall f,g \in B(E) \; : \; d(f,g) \geq 0$.\\
        Dies gilt trivialerweise, da $\forall x \in \mathbb{E} \; : \; |f(x) - g(x)| \geq 0$, da es sich um
        die Betragsfunktion handelt.\\
        $\forall f,g \in B(E) \; : \; d(f,g) = 0 \Leftrightarrow f = g$.\\
        $<=$:\\
            Wenn $f=g$ gilt, dass gilt insbesondere $\forall x \in E \; : \; f(x) = g(x)$.
            Daher ist $M = \{ |f(x) - g(x)| \; , \; x \in E\} = \{ 0 \}$ und $\sup M = 0$.\\
        $=>$:\\
            Da $d(f,g) \geq 0$ wissen wir, dass $\forall x \in E \; : \; f(x) - g(x) = 0$ gelten muss.
            Gäbe es nur einen Wert, der $\not= 0$ ist, so wäre das $\sup > 0$.\\
            Nun folgt daraus aber, dass $\forall x \in E \; : \; f(x) = g(x) \Rightarrow f = g$.
    \item $\forall f,g \in B(E) \; : \; d(f,g) = d(g,f)$.\\
        Dies folgt aus der Symmetrie von $|a - b| = |b - a| \forall a,b \in \mathbb{R}$.
    \item $\forall f,g,h \in B(E) \; : \; d(f,g) \leq d(f,h) + d(h,g)$\\
        Sei $(x_n)_{n\in\mathbb{N}}$ eine Folge, so dass $\underset{n \rightarrow \infty}{\lim} |f (x_n) - g(x_n)| = d(f,g)$,
        $(y_n)_{n\in\mathbb{N}}$ eine Folge, so dass $\underset{n \rightarrow \infty}{\lim} |f(y_n) - h (y_n)| = d(f,h)$,
        und $(z_n)_{n\in\mathbb{N}}$ eine Folge, so dass $\underset{n \rightarrow \infty}{\lim} |h (z_n) - g(z_n)| = d(h,g)$.\\

        Nun gilt $\underset{n \rightarrow \infty}{\lim} |f (x_n) - g (x_n) | \leq \underset{n \rightarrow \infty}{\lim} | f(x_n) - h (x_n) | + | h(x_n) - g(x_n)|$,
        da $\underset{n \rightarrow \infty}{\lim} |f(x_n) - h(x_n)| \leq |f(y_n) - h(y_n)|$ und für $z_n$ ebenso gilt, da die Folgen,
        als das supremum definiert waren und alle Funktionen in unserem Raum beschränkt sind. Es kann also kein größeren Wert geben.
\end{enumerate}


\textbf{Konvergenz}:\\
Sei nun $(f_n)_{n \in \mathbb{N}}$ Cauchy - Folge beliebig, aber fest.\\

Nun ist die Folge der Werte $f_n(x_0)$ für jedes $x_0 \in E$ eine Cauchy - Folge in $\mathbb{R}$, da gilt

$\forall \varepsilon > 0 \exists n_0 \in \mathbb{N} \forall n,m \geq n_0 \; : \; |f_n(x_0) - f_m(x_0)| \leq \underset{x \in E}{\sup} |f_n(x_0) -f_m (x_0) | < \varepsilon$

Wobei das $\varepsilon > 0$ aus der Definition der Cauchyfolge von $(f_n)_{n \in \mathbb{N}}$ genommen werden kann.\\

Da nun $f_n(x_0)$ eine Cauchyfolge in $\mathbb{R}$ ist, wissen wir, dass die Folge konvergiert. So können wir nun unsere mögliche
Grenzfunktion $f^\infty \; : \; E \rightarrow \mathbb{R}$ über \\
$x \mapsto \underset{n \rightarrow \infty}{\lim} f_n(x)$.\\

Nun zeige ich, dass $\underset{n \rightarrow \infty}{\lim} (f_n) = f^\infty$ gilt.\\

Dies zeigen wir auf folgende weise. Sei $\varepsilon > 0$ und sei $n_1 \in \mathbb{N}$, so dass für alle $m,n \geq n_1$
$d(f_n, f_m) < \frac{\varepsilon}{2}$. Diese Gleichung ist erfüllt, da $(f_n)_{n \in \mathbb{N}}$ Cauchy ist.

Nach der gezeigten konvergenz für die einzelnen Punkte, wissen wir, dass für all $x \in E$ ein $n_2 \in \mathbb{N}$ existiert,
so dass für alle $k \geq n_2 $ gilt $| f_k(x) - f^\infty (x) | < \frac{\varepsilon}{2}$.\\

Nun wählen wir $n_3 \geq \max \{ n_1 , n_2 \}$, denn dann gilt:
$$
    |f_n(x) - f^\infty (x)| \stackrel{Dreieck}{\leq} | f_n (x) - f_k (x)| + | f_k (x) - f(x) | < \frac{\varepsilon}{2} + \frac{\varepsilon}{2} = \varepsilon
$$

Als letzter Schritt müssen wir zeigen, dass $f^\infty \in B(E)$ liegt.\\
Da nun aber $f_n$ für alle $n \in \mathbb{N}$ beschränkt ist und $|f_n(x) - f^\infty (x)| < \varepsilon$ für alle $x \in E$ gilt, kann $f^\infty$ nicht divergieren,
da sonst die differenz nicht nach 0 gehen kann.\\

\mbox{} \hfill $\square$

\pagebreak


\section*{Exercise 2}

In the knapsack problem, we are given $n$ items. Each item has a \emph{weight} $g_i$ and a \emph{value} $w_i$.
Furthermore , we have a maximum weight $G$. All inputs are positive integers.\\

We would like to find a set $I \subseteq \{ 1,...,n \}$ of items, such that the total value
$\underset{i \in I}{\sum} w_i$ is maximum, subject to the constraint that the total weight is at most
$G$, meaning $\underset{i \in I}{\sum} g_i \leq G$.

\subsection*{(a)} 

Define an appropriate decision version for the knapsack problem and show that it is NP-complete.\\

\textbf{Solution:}\\
We choose the canonical extension of an optimization problem to a decision problem. Let the defined variables
be as the the optimization problem. Than take an additional positive Integer $K$.\\

The decision question is "Does there exist a set $I \subseteq \{ 1, ... , n \}$ such that
$\underset{i \in I}{\sum} w_i \geq K$ and $\underset{ i \in I}{\sum} g_i \leq G$".

Let KNAP be this problem.

\begin{lemma}\label{cmp:ex1:knap:np}
The decision problem for the knapsack problem is NP-complete.
\end{lemma}
\textbf{Proof \ref{cmp:ex1:knap:np}.}\\
\begin{enumerate}[i)]
    \item KNAP $\in$ NP.\\
        Let $I \subset \{ 1, ... , n \}$ be the optimal solution. Obviously $|I| < n^c$ for some constant $c>0$,
        because $I$ is itself part of the input.\\
        Let the Verifier $V$ be defined as follows.\\
        Compute $\underset{i \in I}{\sum} w_i = w'$ and 
        $\underset{i \in I}{\sum} g_i = g'$.\\
        Accept if and only if $w' \geq K$ and $ g' \leq G$.\\

        Both sums can be computed in polynomial time and the comparison can be computed
        in logarithmic time in the length of the values.
        $V$ accepts $(w,I)$ in $T(n) \in O(n^c)$ for some constant $c>0$.

    \item SUBSET-SUM $\prec_p$ KNAP.\\
        To reduce subset-sum to knapsack we choose the following transformation. Let $S$ be the set of
        positive integers for subset-sum and $T$ a value. The question for knapsack is now, does there exist
        a subset $O \subseteq S$ such that $\underset{i \in O}{\sum} i = T$.

        The reduction function $f$ sets $w_i := g_i := i$ for all $i \in S$ and $G := K := T$. One can
        easily see, that the reduction can be done in linear time because we only copy values. \\

        $w \in$ SUBSET-SUM $\Rightarrow f(w) \in$ KNAP.\\
        There exists a subset $O \subset S$ such that $\underset{i \in O}{\sum} i = T$.
        After the application of $f$ we know that $O \subset I$ and
        $\underset{i \in O} w_i = K$ and $\underset{i \in O} g_i = G$. Therefore $O$ is a solution for
        KNAP on this input.\\

        $f(w) \in$ KNAP $\Rightarrow w \in$ SUBSET-SUM.\\
        There exists a subset $O \subset I$ such that $\underset{i \in O}{\sum} g_i \leq G$ and $\underset{i \in O}{\sum} w_i \geq K$.
        From the reduction $f$ we know that $\underset{i \in O}{\sum} g_i = \underset{i \in O}{\sum} w_i = \underset{i \in O}{\sum} i$ holds.\\
        $\Rightarrow \underset{i \in O}{\sum} i \leq G \land \underset{i \in O}{\sum} i \geq K$.\\
        Again from $f$ we know that $G = K = T$ holds.\\
        $\Rightarrow \underset{i \in O}{\sum} i = T$.\\
\mbox{} \hfill $\square$
\end{enumerate}


\subsection*{(b)}

Let $W := \underset{i=1}{\overset{n}{\sum}} w_i$. Show that the knapsack problem can be solved in $O(nW)$ time. Why does this not
contradict \emph{(a)}?\\

\textbf{Solution:}\\

We show that a dynamic program solves the knapsack problem and has the runtime of $O(nW)$.\\
Let $g[n,W]$ be a two dimensional array, that stores the in $g[i,w]$ the minimal weight, for the first $i$ objects
with an value exactly $w$.

We can now give the recursive definition of to fill in $w$. Let the array be initialized with $\infty$.

$$\begin{array}{lclr}
    g[0,w]  &:=& 0          & \forall 0 \leq w \leq W\\
    g[i,0]  &:=& 0          & \forall 1 \leq i \leq n\\
    g[i,w]  &:=& g[i-1,w]   & \text{if } w - w_i \leq 0\\
    g[i,w]  &:=& \min \{ g[i-1,w] , g[i-1 , w - w_i] + g_i\}
\end{array}$$

After filling the array we can retrieve the maximal value by computing\\
$\underset{0 \leq w \leq W}{\text{argmax}} \{ g[n,w] \; | \; g[n,w] \leq G \}$.\\

This algorithm can be easily implemented through a double for loop comparing
both cases. The result can be computed in a single for loop.

The algorithm computes the optimal solution, because it computes every possible solution and takes the greatest value.\\
The computation takes $O(nW)$ time because the array has $n \times W$ cells and in each cell we have an constant
computation time $O(1)$.
To find the maximal value we iterate over $W$ cells and compute the maximum which can be done in time $O(W)$.\\
Therefore $T(n) = O(nW)$ holds.\\

This does not contradict with \emph{(a)} because $W$ can be exponential in the size of the input. The runtime of the algorithm
is pseudo polynomial.

\subsection*{(c)}

For the $(1 - \varepsilon)$ - approximation we use an algorithm based on the algorithm of $(b)$.

The trick is to round the values of $w_i$ such that we do not have to look at each possible value of $w$.\\

We assign new values by the following scheme

\begin{lstlisting}
M $\leftarrow$ $\max_{i\in I} w_i$
$\mu$ $\leftarrow$ $\varepsilon \frac{M}{n}$
w_i' $\leftarrow$ $\lfloor w_i / \mu \rfloor$ for all $i \in S$
\end{lstlisting}

and run the algorithm from $(b)$ on it.
With the modified values we can skip the array in $\mu$ steps.
$$
\frac{W}{\mu} = \frac{n}{\varepsilon \cdot M} \overset{n}{\underset{i=1}{\sum}} w_i \stackrel{M \geq w_i}{\leq} \frac{n}{\varepsilon} \overset{n}{\underset{i=1}{\sum}} 1 = O (\frac{n^2}{\varepsilon})
$$
The new grid has size $n \times O (\frac{n^2}{\varepsilon})$ therefore the runtime of the algorithm on the modified values is
in $poly(n, \frac{1}{\varepsilon})$.\\
Leaves us to proof, that the algorithm is a $(1-\varepsilon)$ - approximation.
Let $I$ be the set which gives us the optimal solution on $W$ and $O$ the set, that gives use the optimal solution on $W'$ the rounded values.\\

First observe, that $M \leq OPT$, because we can always can take the most valuable item. The rounding of the values gives us the estimation 
$ \mu w_i' \leq w_i \leq \mu (w_i' + 1)$ which leads us to $\mu w_i'  \geq w_i - \mu$.

$$\begin{array}{rcl}
    \underset{i \in O}{\sum} w_i &\stackrel{\text{Def. } w_i'}{\geq}& \mu \underset{i \in O}{\sum} w_i'\\
        &\stackrel{O \text{ opt. on } w_i'}{\geq}& \mu \underset{i \in I}{\sum} w_i'\\
        &\geq& (\underset{i \in I}{\sum} w_i) - |I|\mu\\
        &\stackrel{|I| < n}{\geq}& (\underset{i \in I}{\sum} w_i) - n \mu\\
        &\stackrel{\text{Def. }\mu}{=}& (\underset{i \in I}{\sum} w_i) - \varepsilon M\\
        &\stackrel{I \text{ opt. on }w_i}{\geq}& OPT - \varepsilon OPT = (1-\varepsilon) OPT
\end{array}$$ 

We can conclude that the algorithm is an FPTAS for the knapsack problem.


%%\section*{Exercise 3}

True or False? For every $\varepsilon > 0$, there exists an NP - complete problem that can be solved deterministicly in
$O(2^{n^\varepsilon})$ steps. Explain your answer.

\textbf{Solution:}\\

We proof this part by construction a language for each $\varepsilon > 0$, that is $NP - complete$ and can be solved in
the given time.\\

Let $PAD-SAT_\varepsilon = \{ (\Psi , 1^{|\Psi|^{\left\lceil \frac{k}{\varepsilon} \right\rceil}}) \; | \; \Psi \text{satisfyable in 3 cnf} \}$
where $k > 1$ is a arbitrary but fixed konstant.\\

\pagebreak

\begin{lemma}\label{ex1:t3:npc}
    $PAD-SAT_\varepsilon$ is NP-complete.
\end{lemma}

\textbf{Proof \ref{ex1:t3:npc}.}\\
This part is straight forward.\\
Given an variable assignment we can check, whether the formula $\Psi$ is satisfied in linear time, by simply iteration over
the clauses. Because the number of variables has to be strictly smaller than the formulas the assignment is polynomial bounded
in the inputsize.\\

We can next reduce $PAD-SAT_\varepsilon$ to $3-SAT$. The reduction copys $\Psi$ and throws away the rest. If 
the TM for $3 - SAT$ accepts, the word is in the language, because $\Psi$ is satisfiable. If it rejects,
$\Psi$ is not satisfiable, thus the word can't be in $PAD-SAT_\varepsilon$.

\begin{lemma}\label{ex1:t3:time}
    There exists a DTM M such that $T_M(n) \in DTIME (2^{n^\varepsilon})$.
\end{lemma}

\textbf{Proof \ref{ex1:t3:time}.}\\

From the definition we can conclude, that $|\Psi| \leq n^{\frac{\varepsilon}{k}}$ holds,
because the rest of the length of the word is filled with ones.\\

$\Psi$ is a boolean formula and this formula contains each occurence of its variables. Therefore there can't
be no more variables, than the length of $\Psi$.\\
So we can compute all Assignments, which are less than $2^{n^{\frac{\varepsilon}{k}}}$.
As said before we can iterate over all clauses and literals and check whether there occures at least one 
true in each clause. Therefore we have linear costs in the length of $\Psi$.\\

We now construct a DTM $M$.\\
Given a word $(\Psi, 1^{|\Psi|^{\left\lceil \frac{k}{\varepsilon} \right\rceil }})$, we check first
whether the number of ones is right in the end. This takes us at most $n$ time.
Next we test all assignments, if they satisfy the formula.\\

Testing takes $n^\frac{\varepsilon}{k}$ time and we have $2^{n^\frac{\varepsilon}{k}}$ possibility, so
the runtime of $M$ is $T_M (n) = n^\frac{\varepsilon}{k} 2^{n^\frac{\varepsilon}{k}}$. We ignore the
linear Faktor, because we know that we can choose a faktor so that this function is strictly greate after a given $n_0$.\\

Now we have to show, that $T_M(n) \in O(2^{n^\varepsilon})$. We do this by checking the limiting value
$$
    \underset{n \rightarrow \infty}{\lim} \frac{T_M (n)}{2^{n^\varepsilon}}
        = \underset{n \rightarrow \infty}{\lim} n^\frac{\varepsilon}{k} 2^{n^\frac{\varepsilon}{k} - n^\varepsilon}\\
        = 0
$$

Because $n^\frac{\varepsilon}{k} < n^\varepsilon$ holds for $k > 1$ and than we have a exponential function with a negative exponent
and a polynomial faktor. By knowledge of Analysis I this sequence converges to 0.\\
\mbox{} \hfill $\square$


\section*{Exercise 3}

True or False? For every $\varepsilon > 0$, there exists an NP - complete problem that can be solved deterministically in
$O(2^{n^\varepsilon})$ steps. Explain your answer.

\textbf{Solution:}\\

We proof this part by construction a language for each $\varepsilon > 0$, that is $NP - complete$ and can be solved in
the given time.\\

Let PAD-SAT$_\varepsilon = \{ (\Psi , 1^{|\Psi|^{\left\lceil \frac{2}{\varepsilon} \right\rceil}}) \; | \; \Psi \text{ satisfiable in 3 cnf} \}$\\

\begin{lemma}\label{ex1:t3:npc}
	$\text{PAD-SAT}_\varepsilon$ is NP-complete.
\end{lemma}

\textbf{Proof \ref{ex1:t3:npc}.}\\
PAD-SAT$_\varepsilon \in$ NP.\\
Given an variable assignment we can check whether the formula $\Psi$ is satisfied in polynomial time. 
The number of variables has to be strictly smaller than the length of the formula. The assignment is polynomial bounded
in the input size hence the witness is polynomial in the size of the input.\\

3-SAT $\prec_p$ PAD-SAT$_\varepsilon$.\\
The reduction function copies the formula $\Psi$, computes $|\Psi|^{\left\lceil \frac{2}{\varepsilon} \right\rceil}$
and appends that many ones to $\Psi$. The computation can be done in polynomial time.
The reduction is valid because if $\Psi$ is satisfiable it remains satisfiable after the reduction and vis-versa.

\begin{lemma}\label{ex1:t3:time}
    There exists a DTM M such that $T_M(n) \in DTIME (2^{n^\varepsilon})$.
\end{lemma}

\textbf{Proof \ref{ex1:t3:time}.}\\
We construct a DTM M that checks the satisfiability by brute-force.
Given a word $w = (\Psi, 1^{|\Psi|^{\left\lceil \frac{2}{k} \right\rceil}})$ with $n = |w|$.\\
It holds that $|\Psi| \leq n^\frac{\varepsilon}{2}$. Otherwise 
$|w| = |\Psi| + |\Psi|^{\left\lceil \frac{2}{k} \right\rceil} > n^\frac{\varepsilon}{2} + (n^\frac{\varepsilon}{2})^{\frac{2}{k}} > n$
holds which is a contradiction.\\

$\Rightarrow$ There exists at most $n^\frac{\varepsilon}{2}$ possible assignments for $\Psi$. We can
compute the result for a given assignment in $n^c$ time for some $c > 0$.

Therefore $T_M (n) \leq n^c \cdot 2^{n^\frac{\varepsilon}{2}}$. At last we prove that $T_M(n) \in O(2^{n^\varepsilon})$.

$$
\lim_{n\rightarrow \infty} \frac{T_M (n)}{2^{n^\varepsilon}} = \lim_{n \rightarrow \infty} n^c \cdot 2^{n^\frac{\varepsilon}{2} - n^\varepsilon}
= n^c \cdot 2^{-n^\frac{\varepsilon}{2}} = 0
$$
By convergence criterion we know that there exist the DTM M that computes a NP complete problem in $O(2^{n^\varepsilon})$ time.\\
\mbox{} \hfill $\square$

\label{LastPage}

\end{document}
