\subsection*{\itshape 2. Vollständigkeit von Funktionsräumen}

Für $E \subset \mathbb{R}^n, E \not= \emptyset$ setzen wir
$$
    B(E) := \{ f \, : \, E \rightarrow \mathbb{R} \; | \; f \text{ ist beschränkt} \}.
$$

Ferner definieren wir für zweit Funktionen $f,g \, : \, E \rightarrow \mathbb{R}$ ihren Abstand
$$
    d(f,g) := \underset{x \in E}{\sup} |f(x) - g(x)|.
$$

Zeigen Sie, dass $(B(E),d)$ vollständig ist.\\

\textbf{Lösung:}\\

\textbf{Metrik}:\\
Als erstes müssen wir zeigen, dass es sich bei $(B(E),d)$ um eine Metrik handelt.
\begin{enumerate}[1.]
    \item $\forall f,g \in B(E) \; : \; d(f,g) \geq 0$.\\
        Dies gilt trivialerweise, da $\forall x \in \mathbb{E} \; : \; |f(x) - g(x)| \geq 0$, da es sich um
        die Betragsfunktion handelt.\\
        $\forall f,g \in B(E) \; : \; d(f,g) = 0 \Leftrightarrow f = g$.\\
        $<=$:\\
            Wenn $f=g$ gilt, dass gilt insbesondere $\forall x \in E \; : \; f(x) = g(x)$.
            Daher ist $M = \{ |f(x) - g(x)| \; , \; x \in E\} = \{ 0 \}$ und $\sup M = 0$.\\
        $=>$:\\
            Da $d(f,g) \geq 0$ wissen wir, dass $\forall x \in E \; : \; f(x) - g(x) = 0$ gelten muss.
            Gäbe es nur einen Wert, der $\not= 0$ ist, so wäre das $\sup > 0$.\\
            Nun folgt daraus aber, dass $\forall x \in E \; : \; f(x) = g(x) \Rightarrow f = g$.
    \item $\forall f,g \in B(E) \; : \; d(f,g) = d(g,f)$.\\
        Dies folgt aus der Symmetrie von $|a - b| = |b - a| \forall a,b \in \mathbb{R}$.
    \item $\forall f,g,h \in B(E) \; : \; d(f,g) \leq d(f,h) + d(h,g)$\\
        Sei $(x_n)_{n\in\mathbb{N}}$ eine Folge, so dass $\underset{n \rightarrow \infty}{\lim} |f (x_n) - g(x_n)| = d(f,g)$,
        $(y_n)_{n\in\mathbb{N}}$ eine Folge, so dass $\underset{n \rightarrow \infty}{\lim} |f(y_n) - h (y_n)| = d(f,h)$,
        und $(z_n)_{n\in\mathbb{N}}$ eine Folge, so dass $\underset{n \rightarrow \infty}{\lim} |h (z_n) - g(z_n)| = d(h,g)$.\\

        Nun gilt $\underset{n \rightarrow \infty}{\lim} |f (x_n) - g (x_n) | \leq \underset{n \rightarrow \infty}{\lim} | f(x_n) - h (x_n) | + | h(x_n) - g(x_n)|$,
        da $\underset{n \rightarrow \infty}{\lim} |f(x_n) - h(x_n)| \leq |f(y_n) - h(y_n)|$ und für $z_n$ ebenso gilt, da die Folgen,
        als das supremum definiert waren und alle Funktionen in unserem Raum beschränkt sind. Es kann also kein größeren Wert geben.
\end{enumerate}


\textbf{Konvergenz}:\\
Sei nun $(f_n)_{n \in \mathbb{N}}$ Cauchy - Folge beliebig, aber fest.\\

Nun ist die Folge der Werte $f_n(x_0)$ für jedes $x_0 \in E$ eine Cauchy - Folge in $\mathbb{R}$, da gilt

$\forall \varepsilon > 0 \exists n_0 \in \mathbb{N} \forall n,m \geq n_0 \; : \; |f_n(x_0) - f_m(x_0)| \leq \underset{x \in E}{\sup} |f_n(x_0) -f_m (x_0) | < \varepsilon$

Wobei das $\varepsilon > 0$ aus der Definition der Cauchyfolge von $(f_n)_{n \in \mathbb{N}}$ genommen werden kann.\\

Da nun $f_n(x_0)$ eine Cauchyfolge in $\mathbb{R}$ ist, wissen wir, dass die Folge konvergiert. So können wir nun unsere mögliche
Grenzfunktion $f^\infty \; : \; E \rightarrow \mathbb{R}$ über \\
$x \mapsto \underset{n \rightarrow \infty}{\lim} f_n(x)$.\\

Nun zeige ich, dass $\underset{n \rightarrow \infty}{\lim} (f_n) = f^\infty$ gilt.\\

Dies zeigen wir auf folgende weise. Sei $\varepsilon > 0$ und sei $n_1 \in \mathbb{N}$, so dass für alle $m,n \geq n_1$
$d(f_n, f_m) < \frac{\varepsilon}{2}$. Diese Gleichung ist erfüllt, da $(f_n)_{n \in \mathbb{N}}$ Cauchy ist.

Nach der gezeigten konvergenz für die einzelnen Punkte, wissen wir, dass für all $x \in E$ ein $n_2 \in \mathbb{N}$ existiert,
so dass für alle $k \geq n_2 $ gilt $| f_k(x) - f^\infty (x) | < \frac{\varepsilon}{2}$.\\

Nun wählen wir $n_3 \geq \max \{ n_1 , n_2 \}$, denn dann gilt:
$$
    |f_n(x) - f^\infty (x)| \stackrel{Dreieck}{\leq} | f_n (x) - f_k (x)| + | f_k (x) - f(x) | < \frac{\varepsilon}{2} + \frac{\varepsilon}{2} = \varepsilon
$$

Als letzter Schritt müssen wir zeigen, dass $f^\infty \in B(E)$ liegt.\\
Da nun aber $f_n$ für alle $n \in \mathbb{N}$ beschränkt ist und $|f_n(x) - f^\infty (x)| < \varepsilon$ für alle $x \in E$ gilt, kann $f^\infty$ nicht divergieren,
da sonst die differenz nicht nach 0 gehen kann.\\

\mbox{} \hfill $\square$

\pagebreak
