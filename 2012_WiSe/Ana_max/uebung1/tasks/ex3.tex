\subsection*{\itshape 3. Norm und Skalarprodukt}

Sei $<\cdot,\cdot> \, : \, X \times X \rightarrow \mathbb{R}$ ein inneres Produkt auf einem rellen Vektorraum $X$. Wir
definieren eine Norm auf $X$ gemäß
$$
    \| x \| := \sqrt{<x,x>}.
$$

Zeigen Sie:

\subsubsection*{(i)}
$<x,y> \leq \| x \| \| y \|$\\
\textbf{Lösung:}\\

\subsubsection*{(ii)}
$\| x + y \| \leq \| x \| + \| y \|$\\
\textbf{Lösung:}\\

\subsubsection*{(iii)}
Betrachte nun den Folgenraum
$$
    l^2 := \{ x = (x_n)_{n \in \mathbb{N}} \; | \; x_n \in \mathbb{R}\ ; , \overset{\infty}{\underset{n=1}{\sum}} x_n^2 \leq \infty \}
$$

Zeigen Sie, dass durch
$$
    \| x \| := \sqrt{\sum^\infty_{n=1} x_n^2}
$$
eine Norm auf $l^2$ definiert ist.\\

\textbf{Lösung:}\\
