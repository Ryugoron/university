\subsection*{\itshape 3. Norm und Skalarprodukt}

Sei $<\cdot,\cdot> \, : \, X \times X \rightarrow \mathbb{R}$ ein inneres Produkt auf einem rellen Vektorraum $X$. Wir
definieren eine Norm auf $X$ gemäß
$$
    \| x \| := \sqrt{<x,x>}.
$$

Zeigen Sie:

\subsubsection*{(i)}
$<x,y> \leq \| x \| \| y \|$\\
\textbf{Lösung:}\\

Seien $x, y \in X$ und $t \in \mathbb{R}$.\\

Sei $g(t) = <x + ty, x + ty> = <x,x> + 2t <x,y> + t^2 <y,y>$.\\
Aus der ersten characterisierung mit $z = x + ty$ erhalten wir
$g(t) = <z,z> \geq 0$, da es sich um ein Skalarprodukt handelt.

Nun leiten wir $g$ nach $t$ ab, was
$g'(t) = 2t <y,y> + 2 <x,y>$ ergibt und bestimmen das minimum.
(Es ist ein Minimum, da $<y,y> \geq 0$.)\\
$$\begin{array}{lrcl}
    & 0 &=& g'(t_0)\\
\Leftrightarrow & 0 &=& 2t_0 <y,y> + 2 <x,y>\\
\Leftrightarrow & t &=& - \frac{<x,y>}{<y,y>}
\end{array}$$
Hier ist zu beachten, dass im Fall $<y,y> = 0$ gilt, dass $<x,y> = 0$ gelten muss, womit die
Gleichung trivialier Weise erfüllt ist. Wir können also $<y,y> > 0$ im folgenden Annehmen.\\

Dies setzten wir nun erneut in $g(t)$ ein:
$$\begin{array}{lrcl}
&   0       &\leq& g(t_0)\\
&           &=& <x,x> - 2 \frac{<x,y>}{<y,y>}\cdot <x,y> + \frac{<x,y>^2}{<y,y>}\\
&           &=& <x,x> - \frac{<x,y>^2}{<y,y>}\\
&           &=& \| x \|^2 - \frac{<x,y>^2}{\| y \|^2}\\
\Leftrightarrow& \| x \|^2 & \geq & \frac{<x,y>^2}{\| y \|^2}\\
\Leftrightarrow& \| x \|^2 \cdot \| y \|^2 &\geq& <x,y>^2\\
\Leftrightarrow& \| x \| \| y \| & \geq & <x,y>
\end{array}$$

\mbox{}\hfill$\square$

\subsubsection*{(ii)}
$\| x + y \| \leq \| x \| + \| y \|$\\
\textbf{Lösung:}\\


Der Beweis ist durch (i) straight-forward.
$$\begin{array}{rcl}
    \| x + y \| &=& \sqrt{ < x + y, x + y> }\\
                &=& \sqrt{ <x, x + y> + <y , x + y> }\\
                &\stackrel{(i)}{\leq}& \sqrt{\| x \|^2 + 2 \| x \| \| y \| + \| y \|^2 }\\
                &=& \sqrt{ (\| x \| + \| y \|)^2 }\\
                &=& \| x \| + \| y \|
\end{array}$$
\mbox{}\hfill$\square$

\subsubsection*{(iii)}
Betrachte nun den Folgenraum
$$
    l^2 := \{ x = (x_n)_{n \in \mathbb{N}} \; | \; x_n \in \mathbb{R}\ ; , \overset{\infty}{\underset{n=1}{\sum}} x_n^2 \leq \infty \}
$$

Zeigen Sie, dass durch
$$
    \| x \| := \sqrt{\sum^\infty_{n=1} x_n^2}
$$
eine Norm auf $l^2$ definiert ist.\\

\textbf{Lösung:}\\

Wir fassen den $l^2$ als unendlich dimensionalen Vektorraum auf, wobei ein Folgenglied $x_i$ einer Folge aus $l^2$ die $i$-te
Komponente ist.\\

Zunächst müssen zeige ich, dass durch
$$
<x,y> = \sum^\infty_{n=1} x_n y_n
$$
ein Skalarprodukt auf $l^2$ definiert ist.

Seien $x,y,z \in l^2$, $\alpha, \beta \in \mathbb{R}$.
\begin{enumerate}[a)]
    \item Bilinear\\
        $$\begin{array}{rcl}
            <\alpha x, y>   &=& \overset{\infty}{\underset{i=1}{\sum}} (\alpha x_i)y_i\\
                            &=& \alpha \overset{\infty}{\underset{i=1}{\sum}} x_i y_i\\
                            &=& \alpha < x , y>
        \end{array}$$
        $$\begin{array}{rcl}
            <x, \beta y>    &=& \overset{\infty}{\underset{i=1}{\sum}} x_i (\beta y_i)\\
                            &=& \beta \overset{\infty}{\underset{i=1}{\sum}} x_i y_i\\
                            &=& \beta < x , y>
        \end{array}$$
        $$\begin{array}{rcl}
            <x + z, y>      &=& \overset{\infty}{\underset{i=1}{\sum}} (x_i + z_i) y_i\\
                            &=& \overset{\infty}{\underset{i=1}{\sum}} x_i y_i + z_i y_i\\
                            &\stackrel{< \infty}{=} & \overset{\infty}{\underset{i=1}{\sum}} x_i y_i + \overset{\infty}{\underset{i=1}{\sum}} z_i y_i\\
                            &=& < x , y> + <z , y>
        \end{array}$$
        $$\begin{array}{rcl}
            <x , y + z>     &=& \overset{\infty}{\underset{i=1}{\sum}} x_i (y_i + z_i)\\
                            &=& \overset{\infty}{\underset{i=1}{\sum}} x_i y_i + x_i z_i\\
                            &\stackrel{< \infty}{=} & \overset{\infty}{\underset{i=1}{\sum}} x_i y_i + \overset{\infty}{\underset{i=1}{\sum}} x_i y_i\\
                            &=& < x , y> + <x , y>
        \end{array}$$
    \item Symmetrisch\\
        Trivial, da die Multiplikation auf $\mathbb{R}$ symmetrisch ist.
    \item positiv definit\\
        $$\begin{array}{rcl}
            <x,x>   &=& \overset{\infty}{\underset{i=0}{\sum}} x_i^2\\
                    &\stackrel{x_i^2 > 0}{\geq}& \overset{\infty}{\underset{i = 1}{\sum}} 0\\
                    &=& 0
        \end{array}$$
\end{enumerate}

Nun zeigen wir die Eigenschaften einer Norm:

\begin{enumerate}[I)]
    \item Positiv\\
        $x_i^2 > 0 \Rightarrow \overset{\infty}{\underset{i = 0}{\sum}} x_i^2 > 0 \Rightarrow \sqrt{\overset{\infty}{\underset{i = 0}{\sum}} x_i^2} > 0 \Rightarrow \| x \| > 0$\\
        Sei $x \in l^2$. z.z. $x = 0 \Leftrightarrow \| x \| = 0$.\\
        $\Rightarrow$:\\
        $x = 0 \Rightarrow \forall i \in \mathbb{N} : x_i = 0 \Rightarrow \sqrt{\overset{\infty}{\underset{i = 0}{\sum}} x_i^2} = 0 \Rightarrow \| x \| = 0$\\
        $\Leftarrow$:\\
        Da $\| x \| = 0$ gilt, können wir aus der ersten Eigenschaft folgern, dass $\forall i \in \mathbb{N} : x_i = 0$ gilt.
        $\Rightarrow x = 0$.
    \item Homogenität\\
        Sei $x \in l^2$ und $\alpha \in \mathbb{R}$.\\
        Dann gilt
        $$\begin{array}{rcl}
            \| d x \|   &=& \sqrt{\overset{\infty}{\underset{i = 0}{\sum}} (dx_i)^2}\\
                        &=& \sqrt{\overset{\infty}{\underset{i = 0}{\sum}} d^2 x_i^2}\\
                        &=& |d| \cdot \sqrt{\overset{\infty}{\underset{i = 0}{\sum}} x_i^2}\\
                        &=& |d| \| x \|
        \end{array}$$
    \item Dreiecksungleichung\\
        Seien $x, y \in l^2$.\\
        Dann gilt, da wir nun eine Norm durch ein Skalarprodukt definiert haben
        $$\begin{array}{rcl}
            \| x + y \| &\stackrel{(ii)}{\leq} \| x \| + \| y \|
        \end{array}$$
 
\end{enumerate}
\mbox{}\hfill$\square$
