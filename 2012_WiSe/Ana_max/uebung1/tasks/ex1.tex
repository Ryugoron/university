\subsection*{\itshape 1. Äquivalenz von Metriken}

Auf $\mathbb{R}^n$ seien drei verschiedene Metriken gegeben durch
$$
d(x,y) := |x - y| = \sqrt{\sum^n_{i=1} (x_i - y_i)^2}
$$
$$
\sigma (x,y) := \underset{1\leq i \leq n}{\max} |x_i - y_i| \qquad , \qquad \varrho (x,y) := \sum^n_{i=1} |x_i - y_i|
$$

\subsubsection*{(i)} 
Es bezeichen $B_r^d(x), B_r^\sigma (x)$ und $B_r^\varrho (x)$ offene Kugeln um $x\in\mathbb{R}^n$ mit dem Radius $r$.
Finden Sie nur von $n$ abhängige Konstanten $C_1, C_2, C_3$ und $C_4$, so dass
$$
    B^\varrho_{C_1r}(x) \subset B_r^d (x) \subset B_{C_2r}^\sigma (x) \text{ sowie } B^\sigma_{C_3r}(x) \subset B_r^d (x) \subset B_{C_2r}^\varrho (x). 
$$
\textbf{Lösung:}\\

\begin{enumerate}[\itshape a)]

    \item $B^\varrho_{C_1r} (x) \subset B^d_r (x)$.\\
        Wir wollen also zeigen, dass $\forall y \in B_{C_1r}^\varrho (x) \; : \; d(x,y) < r$ gilt.\\
        Sei $y \in B_{C_1r}^\varrho (x)$ beliebig. Dann wissen wir
        $$\begin{array}{rcl}
            \varrho (x,y)   &=& \overset{n}{\underset{i=1}{\sum}} |x_i - y_i| < C_1r\\
                            &\Rightarrow& |x_i - y_i| < C_1r, \; \forall 1 \leq i \leq n
        \end{array}$$
        Dies wissen wir, da alle Summanden größer als $0$ sind und sonst die Summe gesammt größer wäre.
        Nun setzen wir es in die Metrik von $d$ ein:
        $$\begin{array}{rcl}
            d(x,y)  &=& \sqrt{\overset{n}{\underset{i=1}{\sum}} (x_i - y_i)^2}\\
                    &<& \sqrt{\overset{n}{\underset{i=1}{\sum}} C_1^2 r^2}\\
                    &=& \sqrt{n} C_1 r
        \end{array}$$
        Wir wir sehen, ist für $C_1 = \frac{1}{\sqrt{n}}$ die Gleichung erfüllt und ist somit eine Grenze.

    \item $B_r^d (x) \subset B_{C_2r}^\sigma (x)$.\\
        Wir wollen zeigen, dass $\forall y \in B_r^d (x) \; : \; \sigma (x,y) < r$ gilt.\\
        Sei $y \in B_r^d (x)$ beliebig. Dann wissen wir
        $$\begin{array}{rcl}
            d(x,y)  &=& \sqrt{\overset{n}{\underset{i=1}{\sum}} (x_i - y_i)^2} < r\\
                    &\Leftrightarrow& \overset{n}{\underset{i=1}{\sum}} (x_i - y_i)^2 < r^2\\
                    &\Rightarrow& |x_i - y_i| < r \; , \forall 1 \leq i \leq n 
        \end{array}$$
        Nun setzen wir es in die Metrik von $\sigma$ ein:
        $$\begin{array}{rcl}
            \sigma (x,y)    &=& \underset{1 \leq i \leq n}{\max} |x_i - y_i|\\
                            &\stackrel{Vor.}{<}& r
        \end{array}$$
        Wir sehen, dass die Gleichung für $C_2 = 1$ gilt.

    \item $B_{C_3r}^\sigma (x) \subset B_r^d (x)$.\\
        Wir wollen zeigen, dass $\forall y \in B_{C_3r}^\sigma (x) \; : \; d(x,y) < r$ gilt.\\
        Sei $y \in B_{C_3r}^\sigma (x)$ beliebig. Dann wissen wir
        $$
            \sigma (x,y) = \underset{1\leq i \leq n}{\max} |x_i - y_i| < C_3r \Rightarrow \forall 1 \leq i \leq n \; : \; |x_i - y_i| < C_3r.
        $$
        Nun setzen wir es in die Metrik von $d$ ein:
        $$\begin{array}{rcl}
            d(x,y)  &=& \sqrt{\overset{n}{\underset{i=1}{\sum}} (x_i - y_i)^2}\\
                    &\stackrel{Vor.}{<}& \sqrt{\overset{n}{\underset{i=1}{\sum}} C_3^2 r^2}\\
                    &=& \sqrt{n} C_3 r
        \end{array}$$
        Wir sehen, dass die Gleichung für $C_3 = \frac{1}{\sqrt{n}}$ erüllt ist.

    \item $B_r^d (x) \subset B_{C_4r}^\varrho (x)$.\\
        Wir wollen zeigen, dass $\forall y \in B_r^d (x) \; : \; \varrho (x,y) < C_4r$ gilt.\\
        Sie $y \in B_r^d (x)$ beliebig. Dann wissen wir aus \emph{b)} $|x_i - y_i| < r$.
        Nun setzten wir es in die Metrik von $\varrho$ ein:\\
        $$\begin{array}{rcl}
            \varrho (x,y)   &=& \overset{n}{\underset{i=1}{\sum}} | x_i - y_i |\\
                            &\stackrel{Vor.}{<}& \overset{n}{\underset{i=1}{\sum}} r\\
                            & = & n \cdot r
        \end{array}$$
        Wir sehen, dass die Gleichung mit $C_4 = n$ gilt.
\end{enumerate}

\subsubsection*{(ii)} 
Sei $U \subset \mathbb{R}^n$ offen in $(\mathbb{R}^n, d)$. Zeigen Sie, dass dann $U$ auch offen ist 
in $(\mathbb{R}^n, \varrho)$ und $(\mathbb{R}^n, \sigma)$.\\

\textbf{Lösung:}\\

Für den ersten Teil $U$ offen in $(\mathbb{R}^n, \varrho)$, müssen wir zeigen, dass
$$
    \forall x \in U \exists r > 0 \; : \; B_r^\varrho (x) \subset U
$$
gilt. Sei $x \in U$ beliebig aber fest.\\
Nun wissen wir allerdings, dass ein $r' > 0$ existiert, so dass $B_r'^d (x) \subset U$ ist,
da $U$ offen bezüglich $d$ ist.\\
Nach (i) wissen wir, dass $B_{C_1r'}^\varrho (x) \subset B_{r'}^d (x)$ gilt und da $\subset$ transitiv ist,
folgt die Behauptung $B_{C_1r'}^\varrho (x) \subset U$, da $C_1r' > 0$ ist.\\

Der zweite Teil mit
$$
    \forall x \in U \exists r > 0 \; : \; B_r^\sigma (x) \subset U
$$
folgt analog mit $C_3r' > 0$.
