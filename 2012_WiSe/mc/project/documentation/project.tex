\documentclass{llncs}

\usepackage[utf8]{inputenc}

\usepackage{listings}
\usepackage{amssymb, amsmath}


\title{Five Message Handshake Project in Spin}
\titlerunning{Five Message Handshake}
\author{Alexander Steen \and Max Wisniewski}
\date{\today}
\institute{Institut f\"ur Informatik, FU Berlin}

\begin{document}

\maketitle

\begin{abstract}
The distributed algorithm for the mutual exclusion problem proposed by Suzuki and Kasami \cite{blub} is checked
with the modelchecker \emph{Spin}. We present a modeling for the algorithm in Promela, the properties we want
to check for this algorithm and a short error analysis, why the second algorithm of Suzuki and Kasami does not work.
\end{abstract}



%% --------------------------------------
%%           Section I
%% --------------------------------------
\section{Introduction}

Skip



%% --------------------------------------
%%           Section II
%% --------------------------------------
\section{Problem / Algorithm???}

No problem



%% --------------------------------------
%%           Section III
%% --------------------------------------
\section{Modeling in Spin}

In this part we look at some details of the modeling process.

\subsection{Why Spin}

Because FUCK YOU! Thats why.

\subsection{Channels}

The processes send \emph{REQUEST} and \emph{PRIVILIGE} messages which both
need to be handeld. For any atempt to enter a process sends \emph{REQUEST} to
each of its neighbors. We decided to use a mailbox kind of channel system because
this way a process does not have to iterate over many possible channels and check
each of them seperatly.

\begin{lstlisting}
chan mailbox[N] = [N] of {mtype, int, int, Queue, Array}
\end{lstlisting}

\subsection{Request Messages}

The methode \lstinline|p1| is excuted if a \emph{REQUEST} message is received.
In the modeling we decided to prioritize the receiving of \emph{REQUEST} at the
waiting points that are the \emph{remainder} and the waiting to enter the critical section.

This is neccessary to satisfy the deadlock freedom and fairness constraints.

\subsection{Global Variables}

We decided to model all local variables of the processes as a global array of variables.
We have done this out of debug reasons. This way we could check on receiving a message
if everything was the way we expected it.

\subsection{Send and Receive in Spin}

There occured an ambiguous error in our implementation when we used a wrong number of matching
variables in receiving a message. It happend some times that the received message differed from the
send message. This way some of the requesting processes where dropped from the queue and were not considered
for execution leading to a state where only one process was possible to enter the critical section.

\subsection{Queue}

We choose an implementation of a queue without a check for overflow.
\begin{lstlisting}
    insert 'insert' here
\end{lstlisting}
We could do this because in the program a number for a process is at most added once to the queue.
If we give the queue an array of size $N$ there could never occure an overflow.

\subsection{Next}

Write smth.

%% --------------------------------------
%%           Section IV
%% --------------------------------------
\section{LTL Properties}

A Mutual Exclusion Algorithm needs to satisfy the four properties
\begin{itemize}
    \item Mutual Exclusion,
    \item Absence of Starvation,
    \item Fairness,
    \item No Unneccessary Delay
\end{itemize}
to be considered as correct.

In Spin we have to model for each of these properties one or more
LTL - Properties.

\subsection{Mutual Exclusion}

We added a variable \lstinline|incs| that is incremented before the critical section
and decremented aftarwards. If initialized to zero mutual exclusion is expressed by the
property
\begin{equation}
    \square \left( \text{\lstinline|incs|} \leq 1 \right)
\end{equation}
which is used in both implementations.

\subsection{Absence of Starvation}

The algorithm already uses a flag \lstinline|requesting| to mark if a process wants to enter the critical section.
Using the counter \lstinline|incs| from the Mutual Exclusion property we can express deadlock freedom by
\begin{equation}
    \square \left( (\exists i \; : \; 0 \leq i \land i < N \Rightarrow \text{\lstinline|requesting|}[i]) \Rightarrow \diamond \text{\lstinline|incs|} = 1\right)
\end{equation}
again in both algorithms. For the checking in the bounded example we expanded the existential quantifier explicitly.
\begin{equation}
    \square \left( \text{\lstinline|requesting|}[0] \lor ... \lor \text{\lstinline|requesting|}[N-1] \Rightarrow \diamond \text{\lstinline|incs|} = 1 \right)
\end{equation}

\subsection{Fairness}

All processes do not differ except for their Identifier. Therefore we will check the fairness constraint for the first process and the second process.
The first one because it has initially the privilage and the second one as a representive for every other process.
This time we used a label at the critical section and an array for the process id's.

Fairness can be expressed by
\begin{equation}
    \square \left( \text{\lstinline|requesting|}[0] \Rightarrow \diamond \text{\lstinline|p[ids[0]]@critical|} 
    \land \text{\lstinline|requesting|}[1] \Rightarrow \diamond \text{\lstinline|p[ids[1]]@critical|}\right)
\end{equation}
in both algorithms.

\subsection{No Uneccessary Delay}

As in the last property we will check this one for one of the precesses.
\begin{equation}
    \square \left( \text{\lstinline|incs|}=0 \land \text{\lstinline|requesting|}[0] = 1 \land \text{requesting|}[1] = 0\land ... \text{\lstinline|requesting|}[N-1] = 0 
    \Rightarrow \diamond \text{\lstinline|incs|}=0\right)
\end{equation}

NO IDEAD HERE.

%% --------------------------------------
%%           Section V
%% --------------------------------------
\section{Checking the Properties}

We saw something

\subsection{Communication Diagrams}

Insert some.

%% --------------------------------------
%%           Section I
%% --------------------------------------
\section{Conclusion}

Yes.

%% --------------------------------------
%%           Section ??
%% --------------------------------------
\section{Bibliography}

Not today.

\end{document}
