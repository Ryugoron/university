\documentclass[11pt,a4paper,ngerman]{article}
\usepackage[bottom=2.5cm,top=2.5cm]{geometry} 
\usepackage{babel}
\usepackage[utf8]{inputenc} 
\usepackage[T1]{fontenc} 
\usepackage{ae} 
\usepackage{amssymb} 
\usepackage{amsmath} 
\usepackage{graphicx}
\usepackage{fancyhdr}
\usepackage{fancyref}
\usepackage{listings}
\usepackage{xcolor}
\usepackage{paralist}

%\usepackage[pdftex, bookmarks=false, pdfstartview={FitH}, linkbordercolor=white]{hyperref}
\usepackage{fancyhdr}
\pagestyle{fancy}
\fancyhead[C]{Semantik von Programmiersprachen}
\fancyhead[L]{Übung Nr. 3}
\fancyhead[R]{SoSe 2012}
\fancyfoot{}
\fancyfoot[L]{}
\fancyfoot[C]{\thepage / \pageref{LastPage}}
\renewcommand{\footrulewidth}{0.5pt}
\renewcommand{\headrulewidth}{0.5pt}
\setlength{\parindent}{0pt} 
\setlength{\headheight}{0pt}

\author{Tutor: Ansgar Schneider}
\date{}
\title{Max Wisniewski , Alexander Steen}

\begin{document}

\lstset{language=Pascal, basicstyle=\ttfamily\fontsize{10pt}{10pt}\selectfont\upshape, commentstyle=\rmfamily\slshape, keywordstyle=\rmfamily\bfseries, breaklines=true, frame=single, xleftmargin=3mm, xrightmargin=3mm, tabsize=2}

\maketitle
\thispagestyle{fancy}


%% ------------------------------------------------------
%%                     AUFGABE 1
%% ------------------------------------------------------

\section*{Aufgabe 1}
Definieren Sie die WSKEA-Maschine derart um, dass bei arithmetischen Ausdrücken rechte Unterausdrücke vor linken ausgewertet werden. Konstruieren Sie ein Beispiel, für das ein abweichendes Ergebnis erzielt wird. \\

Dafür müssen lediglich zwei Zustandübergänge geändert werden, der Rest kann gleich bleiben:

(i) Zuerst müssen die rechten Ausdrücke einer Operation zuerst auf den Keller gelegt werden, damit diese zuerst ausgewertet werden: \\

$ \Delta<W | S | T_1 \underline{OP} T_2.K | E | A> :=
        <W | S | T_2.T_1.\underline{OP}.K | E | A>$ \\

(ii) Dann müssen wir beim Herunternehmen der Ergebnisse die korrekte der Operationsanwendung wieder herstellen:\\

$ \Delta<n_1.n_2.W | S | +.K | E | A> :=
        <n_1+n_2.W | S | K | E | A>$, falls $n_1+n_2$ darstellbar.\\

Diese Regel kann analog auf alle anderen arithmetische Operationen angewendet werden.

ABWEICHENDES ERGEBNIS KONSTRUIEREN

%% ------------------------------------------------------
%%                     AUFGABE 2
%% ------------------------------------------------------

\section*{Aufgabe 2}
Erweitern Sie die WSKEA-Maschine um eine Komponente \texttt{N} für Nachrichten (Texte), in der kurze sinnvolle Meldungen eingetragen werden, wenn es keinen Folgezustand gibt oder wenn die Ausführung korrekt terminiert.

tba

% -----------------------------------------------------------
%			AUFGABE 3
% -----------------------------------------------------------
\section*{Aufgabe 3}
Die Syntax von WHILE sei um das repeat-until-Konstrukt erweitert, wie in der Aufgabe. Ergänzen Sie die operationelle Semantik.

tba

% -----------------------------------------------------------
%			AUFGABE 4
% -----------------------------------------------------------
\section*{Aufgabe 4}

tba

\label{LastPage}

\end{document}
