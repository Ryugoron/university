\documentclass[11pt,a4paper,ngerman]{article}
\usepackage[bottom=2.5cm,top=2.5cm,left=2cm,right=2cm]{geometry} 
\usepackage{babel}
\usepackage[utf8]{inputenc} 
\usepackage[T1]{fontenc} 
\usepackage{ae} 
\usepackage{amssymb} 
\usepackage{amsmath} 
\usepackage{graphicx}
\usepackage{fancyhdr}
\usepackage{fancyref}
\usepackage{listings}
\usepackage{xcolor}
\usepackage{paralist}
%\usepackage{bussproofs}
\usepackage{stmaryrd}

%\usepackage[pdftex, bookmarks=false, pdfstartview={FitH}, linkbordercolor=white]{hyperref}
\usepackage{fancyhdr}
\pagestyle{fancy}
\fancyhead[C]{Semantik von Programmiersprachen}
\fancyhead[L]{Übung Nr. 10}
\fancyhead[R]{SoSe 2012}
\fancyfoot{}
\fancyfoot[L]{}
\fancyfoot[C]{\thepage / \pageref{LastPage}}
\renewcommand{\footrulewidth}{0.5pt}
\renewcommand{\headrulewidth}{0.5pt}
\setlength{\parindent}{0pt} 
\setlength{\headheight}{0pt}

\author{Tutor: Ansgar Schneider}
\date{}
\title{Max Wisniewski , Alexander Steen}

\begin{document}

\lstset{language=Haskell, basicstyle=\ttfamily\fontsize{10pt}{10pt}\selectfont\upshape, commentstyle=\rmfamily\slshape, keywordstyle=\rmfamily\bfseries, breaklines=true, frame=single, xleftmargin=3mm, xrightmargin=3mm, tabsize=2, mathescape=true}
   
\maketitle
\thispagestyle{fancy}
\newcommand{\A}{\mathbb{A}}

\subsection*{Aufgabe 1 \mdseries\itshape Typüberprüfung}
Bestimmen Sie die Typen der folgenden Funktionen.
\begin{enumerate}[(i)]
   \item $\lambda f \, x.(f \, x) + 1$\\
      \textbf{Lösung:}\\
         Die ersten beiden Hinweise, die wir haben, ist, dass 
         wir eine Konstante $1 \; \in \; K^{\mathbb{N}_\bot}$ und
         eine Funktion $+ \; : \; [\mathbb{N}_\bot \rightarrow \mathbb{N}_\bot \rightarrow \mathbb{N}_\bot]$.
         Da wir $f$ in die $+$ Funktion stecken, muss der Rückgabetyp $\mathbb{N}_\bot$ sein.
         Über die Eingabe müssen wir nicht mehr wissen nur, dass $f$ eine Variable nimmt und
         das $x$ daher diesen Typ haben muss.\\
         
         $\lambda f \, x. (f \, x) + 1 \; : \; [ [D \rightarrow \mathbb{N}_\bot] 
         \rightarrow \mathbb{N}_\bot \rightarrow \mathbb{N}_\bot  ]$\\

         Nun setzten wir die Variablen ein und überprüfen.\\
         Sei $f \in X^{[D \rightarrow \mathbb{N}_\bot]}$ und $x \in X^{D}$.\\
         Dann ist das einsetzen Korrekt, da
         $(\lambda f \, x. (f \, x) + 1) f x = (f \, x) + 1 \; : \; \mathbb{N}_\bot \rightarrow \mathbb{N}_\bot$\\
         $1 \in K^{\mathbb{N}_\bot}$ das gilt also, nun überprüfen wir, ob
         $f( \, x) \; : \; \mathbb{N}$ erfüllt.\\
         $f \in X^{[D \rightarrow \mathbb{N}_\bot} x \in X^{D}$, daher ist $fx \; : \; \mathbb{N}_\bot$.\\
         
         Der Typ ist daher korrekt.

   \item $\lambda (x , y) f \; . \; f \, x \, y$\\
      \textbf{Lösung:}\\
         Sei $x \in X^{D_1}$, $y \in X^{D_2}$ und $f \in X^{D_3}$.\\
         Die Funktion $h = \lambda (x,y)f \; . \; f \, x \, y \; : \; D_1 
         \times D_2 \rightarrow D_3 \rightarrow D_4$. Wir müssen also überprüfen,
         was $D_1,D_2,D_3$ ist und welchen Rückgabetyp wir erhalten.\\

         Setzten wir $h \, (x,y) f$ ein erhalten wir:\\
         $f \, x \, y \; : \; D_4$.\\
         Damit wir nun am Ende ein Element von einem Typ erhalten (hier hätten auch 3 Atome stehen können).\\
         Daher muss $f$ eine Funktion sein, die beide Elemente $x,y$ aufnehmen kann.\\
         $\Rightarrow D_3 = D_1 \rightarrow D_2 \rightarrow D_5$. Und da wir nichts anderes tun ist auch
         $D_4 = D_5$.\\
         Weiter können wir nun nichts mehr sagen, also gilt:\\
         $h \; : \; D_1 \times D_2 \rightarrow [D_1 \rightarrow D_2 \rightarrow D_4] \rightarrow D_4$.

   \item $\lambda f. (f \lambda y.y)$\\
      \textbf{Lösung:}\\
         Sei $f \in X^{D_1}$. Dann hat die Funktion den folgenden Typ
         $h = \lambda f. (f \lambda y.y) \; : \; D_1 \rightarrow D_2$.\\

         Nun setzten wir unser $f$ ein und erhalten\\
         $(f \lambda y.y) \; : \; D_2$.\\
         Nun muss nach selben Überlegungen wie oben das $f$ die Funktion 
         $\lambda y.y \; : \; [D_3 \rightarrow D_3]$ schlucken können.\\
         
         Daher braucht ist der Typ $f \; : \; [ [D_3 \rightarrow D_3 ] \rightarrow D_4]$.\\
         Da dies nun der letzte Schritt ist muss $D_4 = D_2$ sein, 
         da $(f \lambda y.y) \; : \; D_2$ gelten muss.\\

         Die FUnktion hat also den folgenden Typ (umbenennung der Typklassen):\\
         $(h = \lambda f . (f \lambda y.y) \; : \; [[T \rightarrow T] \rightarrow S] \rightarrow S$

\end{enumerate}

\subsection*{Aufgabe 2 \mdseries\itshape Faltung}
Der Faltungsoperator \underline{lit} sei informall bestimmt durch:\\
\underline{lit} $= f(x_1,...x_n) x_{n+1} = f x_1 (f x_2 (... (f x_n x_{n+1}) ))$\\

\begin{enumerate}[(i)]
   \item Bestimmen Sie den Typ von \underline{lit}.\\
   \textbf{Lösung:}\\
      tbd

   \item Definieren Sie den Operator \underline{lit} im getypten 
      $\lambda$ - Kalkül unter Verwendung der Gleichungsschreibweise.\\

   \item Definieren Sie eine Funktion $f$ im getpyten $\lambda$ - Kalkül, so
      dass 
      $$
         f(x_1, ... , x_n) x = \left\{\begin{array}{lr}
            wahr &, \text{ falls }\exists i \leq n \; : \; x=x_i\\
            false &, \text{ sonst}
         \end{array}\right.
      $$

   \item Bearbeiten Sie (i)-(iii) für \\
      $\underline{lit'} f (x_1,...,x_{n+1}) = (f ((...(f x_1 x_2) ... x_{n+1})$

\end{enumerate}

\label{LastPage}

\end{document}
