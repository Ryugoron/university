\documentclass[11pt,a4paper,ngerman]{article}
\usepackage[bottom=2.5cm,top=2.5cm,left=2cm,right=2cm]{geometry} 
\usepackage{babel}
\usepackage[utf8]{inputenc} 
\usepackage[T1]{fontenc} 
\usepackage{ae} 
\usepackage{amssymb} 
\usepackage{amsmath} 
\usepackage{graphicx}
\usepackage{fancyhdr}
\usepackage{fancyref}
\usepackage{listings}
\usepackage{xcolor}
\usepackage{paralist}
\usepackage{bussproofs}
\usepackage{stmaryrd}

%\usepackage[pdftex, bookmarks=false, pdfstartview={FitH}, linkbordercolor=white]{hyperref}
\usepackage{fancyhdr}
\pagestyle{fancy}
\fancyhead[C]{Semantik von Programmiersprachen}
\fancyhead[L]{Übung Nr. 7}
\fancyhead[R]{SoSe 2012}
\fancyfoot{}
\fancyfoot[L]{}
\fancyfoot[C]{\thepage / \pageref{LastPage}}
\renewcommand{\footrulewidth}{0.5pt}
\renewcommand{\headrulewidth}{0.5pt}
\setlength{\parindent}{0pt} 
\setlength{\headheight}{0pt}

\author{Tutor: Ansgar Schneider}
\date{}
\title{Max Wisniewski , Alexander Steen}

\begin{document}

\lstset{language=Haskell, basicstyle=\ttfamily\fontsize{10pt}{10pt}\selectfont\upshape, commentstyle=\rmfamily\slshape, keywordstyle=\rmfamily\bfseries, breaklines=true, frame=single, xleftmargin=3mm, xrightmargin=3mm, tabsize=2, mathescape=true}

\maketitle
\thispagestyle{fancy}
\newcommand{\A}{\mathbb{A}}

%% ------------------------------------------------------
%%                     AUFGABE 1
%% ------------------------------------------------------

\section*{Aufgabe 1} Sei $A = (\A, \sqsubseteq)$ eine Struktur mit $\sqsubseteq \subseteq A \times A$ Halbordnung und minimalem Element $\perp \in \mathbb{A}$ bzgl. $\sqsubseteq$. \\
Zu zeigen: $\A$ endlich $\Rightarrow A$ ist cpo. \\

Da nach Voraussetzung bereits eine Halbordnung und ein minimales Element existiert, reicht es zu zeigen, dass für jede Kette $K \subseteq \A$ ein Supremum $\sup K \in A$ existiert. \\
\textbf{Beweis:}\\
Sei $A$ wie oben, $\A$ endlich und $K \subseteq \A$ Kette. Da $K$ ebenfalls endlich sein muss, gelte o.B.d.A. $K = \{ k_1, k_2, \ldots, k_n \}$, $n \in \mathbb{N}$, mit $k_1 \sqsubseteq k_2 \sqsubseteq \ldots \sqsubseteq k_n$. \\

(1) $k_n$ ist obere Schranke von $K$, da $\forall 1 \leq i \leq n: \, k_i \sqsubseteq k_n$, da $\sqsubseteq$ transitiv. \\
(2) $k_n$ ist kleinste obere Schranke von $K$: \\
Annahme: $\exists j < n \forall 1 \leq i \leq n: \, k_i \sqsubseteq k_j$ und $k_j \neq k_n$ (es gibt eine kleinere Schranke). \\
$\Rightarrow k_n \sqsubseteq k_j \; \lightning$ Widerspruch,\\
da bereits $k_j \sqsubseteq k_n$ nach Voraussetzung gilt und $\sqsubseteq$ antisymmetrisch ist.
\mbox{} \hfill $\square$
%% ------------------------------------------------------
%%                     AUFGABE 2
%% ------------------------------------------------------

\section*{Aufgabe 2}
O.B.d.A. sei $A = \{ \emptyset, \{ a \}, \{ b \}, \{ c \}, \{ a,b \}, \{ b,c \} \}$ mit Halbordnung $\subseteq$. Dabei ist $\emptyset$ der Knoten ganz unten, die einelementigen Mengen in der Mitte (v.l.n.r.) und die zweielementigen Mengen oben links bzw. oben rechts.

\begin{description}
\item[a)] Ist $A$ ein cpo? \\
Ja, $A$ ist ein cpo, da ein Minimum existiert (hier: $\emptyset$), eine Halbordnung existiert (hier: $\subseteq$) und $A$ endlich ist. Aus Aufgabe 1 folgt, dass $A$ ein cpo ist.
\item[b)] Ist $A$ eine Kette?\\
Nein, da z.B. weder $\{ a,b \} \subseteq \{ b,c \}$ noch $\{ b,c \} \subseteq \{ a,b \}$ gilt.
\item[c)] Existiert $\sup A$ in A?\\
Nein, da kein Element $k \in A$ existiert, für das $\{ a,b \} \subseteq k$ und $\{ b,c \} \subseteq k$ gilt.
\end{description}

%% ------------------------------------------------------
%%                     AUFGABE 3
%% ------------------------------------------------------

\section*{Aufgabe 3}
Gesucht: Zwei Halbordnungen mit minimalem Element, die kein cpo sind.

\begin{enumerate}
\item $(\mathbb{N}, \leq)$ hat kleinstes Element $0$ aber z.B. die Kette $K = \mathbb{N}$ hat kein Supremum in $\mathbb{N}$.
\item weiteres Beispiel
\end{enumerate}

%% ------------------------------------------------------
%%                     AUFGABE 4
%% ------------------------------------------------------

\section*{Aufgabe 4}
Die Funktion
$$ f(x) = \begin{cases}0 & \text{, falls } x = 1 \lor x = 3\\
                       f(x-2) & \text{, sonst} \end{cases}$$
hat die Lösungen $\{ x \mapsto n \cdot (-1)^x + n | \, n \geq 1 \}$ und damit unendlich viele.

\label{LastPage}

\end{document}
