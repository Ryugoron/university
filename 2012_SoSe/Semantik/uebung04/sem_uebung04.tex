\documentclass[11pt,a4paper,ngerman]{article}
\usepackage[bottom=2.5cm,top=2.5cm]{geometry} 
\usepackage{babel}
\usepackage[utf8]{inputenc} 
\usepackage[T1]{fontenc} 
\usepackage{ae} 
\usepackage{amssymb} 
\usepackage{amsmath} 
\usepackage{graphicx}
\usepackage{fancyhdr}
\usepackage{fancyref}
\usepackage{listings}
\usepackage{xcolor}
\usepackage{paralist}

%\usepackage[pdftex, bookmarks=false, pdfstartview={FitH}, linkbordercolor=white]{hyperref}
\usepackage{fancyhdr}
\pagestyle{fancy}
\fancyhead[C]{Semantik von Programmiersprachen}
\fancyhead[L]{Übung Nr. 3}
\fancyhead[R]{SoSe 2012}
\fancyfoot{}
\fancyfoot[L]{}
\fancyfoot[C]{\thepage / \pageref{LastPage}}
\renewcommand{\footrulewidth}{0.5pt}
\renewcommand{\headrulewidth}{0.5pt}
\setlength{\parindent}{0pt} 
\setlength{\headheight}{0pt}

\author{Tutor: Ansgar Schneider}
\date{}
\title{Max Wisniewski , Alexander Steen}

\begin{document}

\lstset{language=Haskell, basicstyle=\ttfamily\fontsize{10pt}{10pt}\selectfont\upshape, commentstyle=\rmfamily\slshape, keywordstyle=\rmfamily\bfseries, breaklines=true, frame=single, xleftmargin=3mm, xrightmargin=3mm, tabsize=2}

\maketitle
\thispagestyle{fancy}


%% ------------------------------------------------------
%%                     AUFGABE 1
%% ------------------------------------------------------

\section*{Aufgabe 1}

Angenommen, dass eine abstrakte Maschine und ihre Überführungsfunktion \emph{delta} gegeben sind.

\begin{enumerate}[a)]
    \item   Implementieren Sie die Semantikfunktion \emph{O}, die jeder Programm-Daten-Kombination
            die entsprechende Ausgabe zuordnet.

            \textbf{Lösung:}\\
            tbd

    \item   Testen Sie Ihre Semantikfunktion \emph{O} am Beispiel der WSKEA Maschine und $\Delta$.
    
            \textbf{Lösung:}\\
            tbd
\end{enumerate}
%% ------------------------------------------------------
%%                  AUFGABE 2
%% -----------------------------------------------------

\section*{Aufgabe 2}
%%
%%  Alles fertig. Muss nur gepastet werden und kommentiert.
%%
\begin{enumerate}[a)]
    \item   Implementieren Sie die Reduktionsfunktion von WHILE in einer Programmiersprache
            Ihrer Wahl.

            \textbf{Lösung:}\\
            \begin{lstlisting}
redT :: (T, (Map I Z, [K], [K])) -> (T, (Map I Z, [K], [K]))
redT ((Z z), t)           = ((Z z), t)
redT ((Id i), (s,e,a))     = 
    case Map.lookup i s of
        Just (z)  -> ((Z z), (s,e,a))
        Nothing     -> ((Id i), (s,e,a))
redT ((TApp v1 op v2), t) =
    let (r1, t')                = redT (v1, t)
        (r2, t'')               = redT (v2, t') in
            case (r1,r2) of
                (Z z1, Z z2)    -> ((Z (decodeOP op z1 z2)), t'')
                (lE, rE)        -> ((TApp lE op rE), t'')
redT (TRead, (s,((LiteralInteger z):e), a))  = ((Z z), (s,e,a))
redT (TRead, t)           = (TRead, t)
        

redB ((Literal b), t)           = ((Literal b), t)
redB ((Not b), t)               = 
    case redB (b, t) of
        ((Literal b'), t')   -> ((Literal (not b')), t')
        (b', t')             -> (b', t')
redB ((BApp b1 bop b2), t)      =
    let (r1, t')                = redT (b1, t)
        (r2, t'')               = redT (b2, t') in
            case (r1,r2) of
                (Z z1, Z z2)    -> ((Literal (decodeBOP bop z1 z2)), t'')
                (lE, rE)        -> ((BApp lE bop rE), t'')
redB (BRead, (s, ((LiteralBool b):e), a)) = ((Literal b), (s,e,a))
redB (BRead , t)                = (BRead, t)

red :: (C, (Map I Z, [K], [K])) -> (C, (Map I Z, [K], [K]))
red (Skip, t)                   = (Skip, t)
red ((Assign i t), (s,e,a))     = 
    let (v,(s',e',a')) = redT (t,(s,e,a)) in
        case v of
            (Z z)               -> (Skip, (Map.insert i z s', e', a'))
            _                   -> error ("Couldn't match " ++ (show t) ++" with Integer in "++(show i)++":="++(show t))
red ((Seq c1 c2), t)            =
    let (cr, t') = red (c1, t)      -- In our implementation could only be Skip
        (cr', t'') = red (c2, t')   -- Same again
    in
        case (cr,cr') of
            (Skip, Skip)         -> (Skip, t'')
            _                   -> error ("Couldn't match"++(show c1)++"; "++(show c2)++"to a real execution")
red ((If b c1 c2), t)            =
    case redB (b, t) of
        ((Literal True), t')    -> red (c1, t')
        ((Literal False), t')   -> red (c2, t')
        _                       -> error ((show b)++" in the expression "++(show (If b c1 c2))++"was no boolean")
red ((While b c), t)            =
    case redB (b,t) of
        ((Literal True), t')    -> 
            let (res, t'') = red (c, t') in
                case res of
                    (Skip)      -> red ((While b c), t'')
                    _           -> error ("Couldn't evaluate ("++(show c)++") in ("++(show (While b c))++")")
        ((Literal False), t'')   -> red (Skip, t'')
        _                       -> error ("Couldn't evaluate ("++(show b)++") in ("++(show (While b c))++")")
red ((BOut b), t)               =
    let (r, (s,e,a)) = redB (b,t) in
        case r of
            (Literal bool)  -> (Skip, (s,e,((LiteralBool bool):a)))
            _               -> error ("Type missmatch, expected bool in ("++(show b)++"), in ("++(show (BOut b))++")")
red ((TOut v), t)               =
    let (r, (s,e,a)) = redT (v,t) in
        case r of
            (Z z)           -> (Skip, (s,e,((LiteralInteger z):a)))
            _               -> error ("Type missmatch, expected bool in ("++(show v)++"), in ("++(show (TOut v))++")")
            \end{lstlisting}

            Wir haben uns dafür entschieden, da es nicht anders gebraucht wurde die transitive Hülle
            in die Reduktion einzubauen. Die Reihenfolge der Ausführung bliebt dadurch erhalten.
            Desweiteren geben wir in einem Fehlerfall halbwegs detailiert zurück, wo der Fehler
            aufgetreten ist. Wir könnten zur Sicherheit auch noch einmal den Halbausgewerteten Ausdruck
            zurückgeben.\\
            Bei redT und redB, werten wir die Terme aus, so wie wir es können. Treffen wir einmal
            als Rückgabe auf keinen Grundtyp, so bauen wir unser Statement wieder zusammen und geben
            es zurück. Im Hauptprogramm nun wird immer wenn so etwas passiert ein Fehler geworfen.

    \item   Implementieren Sie die Semantikfunktion \emph{eval}, die jeder Programm-Daten-Kombination
            die entsprechende Ausgabe zuordnet.

            \textbf{Lösung:}\\
            \begin{lstlisting}
eval :: C -> [K] -> [K]
eval prog eing  = 
    case red (prog, (Map.empty, eing, [])) of
        (Skip, (_,_,aus))   -> aus
        _                   -> error ("The programm wasn't evaluated complete")
            \end{lstlisting}

            Da wir in der Reduktion schon die transitive Hülle gebildet haben, müssen wir an dieser
            Stelle nur prüfen, ob das Programm wirklich fertig reduziert wurde und können dann
            die ausgabe zurückgeben.

    \item   Testen Sie Ihre Funktion \emph{eval} am Beispiel der ganzzahligen Division.

            \textbf{Ergebnis:}\\
            \begin{lstlisting}
RedSem> eval divTest [(LiteralInteger 10),(LiteralInteger 2)]
[5]
RedSem> eval divTest [(LiteralInteger 200),(LiteralInteger 13)]
[15]
            \end{lstlisting}
            Die beiden kurzen Testfälle ergben genau das, was wir erwarten, solange wir uns nicht
            selbst verrechnet haben sollten.
\end{enumerate}

%% ------------------------------------------------------------
%%                  AUFGABE 3
%% ------------------------------------------------------------

\section*{Aufgabe 3}

Gegeben sei folgende Syntax
\begin{tabular}{lcl}
W   & := &  True | False\\
LOP & := &  AND | OR\\
LA  & := &  W | $LA_1$ LOP $LA_2$ | NOT LA 
\end{tabular}

\textbf{Anmerkung:} Uns war es freigestellt uns eine Semantik zu überlegen.
                    Der einfachste Weg wäre es, jeden Ausdruck, den wir bekommen
                    auf True abzubilden. Damit wäre es semantisch völlig
                    spezifiziert. Wir haben uns aber an eine etwas sinnvollere
                    doch nicht zu komplizierte Semantik gesetzt. Da wir uns
                    wörtlich an die Befehle halten, haben wir daher keine Ausgabe.
                    Man könnte sich aber überlegen die Zwischenschritte in die
                    Ausgabe zu schreiben. Dies werden wir im Programmier part auch tun
                    um nachzuvollziehen, ob das richtige passiert.

\begin{enumerate}[a)]
    \item   Definieren Sie eine geeignete operationelle Semantik.\\
            \textbf{Lösung:}\\
            
            Wir führen eine weitere Menge von Kontroll-Symbolen für die abstrakte Maschiene ein
            $CON = \left\{ \underline{CAND}, \underline{COR}, \underline{CNOT} \right\}$. Damit
            ist der Zustandsraum unserer Abstrakten Maschiene $S = CON \cup W \cup LOP \cup LA$.
            Nun definieren wir die Semantikfunktion $\Delta$. Der Speicher ist die leere 
            Funktion $WK = \emptyset $, da wir keine Werte speichern können. Der Wertekeller $S$
            ist eine Liste von Weten $W$ und $E, A$ sind Listen von Werten $W$. Wobei diese nicht
            gebaucht werden.

            $$
            \begin{array}{lcl}
                \Delta \; : \; WK \times S \times K \times E \times A 
                    & \Longrightarrow & W \times S \times K \times E \times A\\
                \Delta \left< w | s | \varepsilon | e | a \right>
                    & \equiv & \left< w | s | \varepsilon | e | a \right>\\

                \Delta \left< w | s | True.k | e | a \right>
                    & \equiv & \left< w | True.s | k | e | a \right>\\

                \Delta \left< w | s | False.k | e | a \right>
                    & \equiv & \left< w | False.s | k | e | a \right>\\

                \Delta \left< w | s | (t1 AND t2).k | e | a \right>
                    & \equiv & \left< w | s | t1.t2.CAND.k | e | a \right>\\

                \Delta \left< w | s | (t1 OR t2).k | e | a \right>
                    & \equiv & \left< w | s | t1.t2.COR.k | e | a\right>\\

                \Delta \left< w | s | (NOT t1).k | e | a \right>
                    & \equiv & \left< w | s | t1.CNOT.k | e | a\right>\\

                \Delta \left< w | v1.v2.s | CAND.k | e | a \right>
                    & \equiv & \left< w | (v2 \land v1 ) . s | k | e | a \right>\\

                \Delta \left< w | v1.v2.s | COR.k | e | a \right>
                    & \equiv & \left< w | (v2 \lor v1).s | k | e | a \right>\\

                \Delta \left< w | v.s | CNOT.k | e | a \right>
                    & \equiv & \left< w | (\neg v).s | k | e | a \right>

            \end{array}$$

            Dies sind alle Überführung die wir benötigen. Da die Syntax und die Semantik keine Ausgabe 
            vorsieht , ist die Semantikfunktion einfach eine Funktion, die alles auf das leere Wort schickt.\\
            $O \; : \; LA \times W \Rightarrow W, \; (prog,eing) \mapsto \varepsilon$.

    \item   Definieren Sie eine geeignete Reduktionssemantik.\\
            \textbf{Lösung:}\\
            
            Wir übernehmen den Speicher, Ein- und Ausgabe aus Aufgabenteil a). Es bleibt zu erwähnen,
            dass wir uns, wie in Aufgabe 2 dafür entschieden haben, den transitiven Abschluss (transitive
            Hülle) gleich mit in die Definition aufzunehmen. Die Auswertung ist hier wieder links vor rechts.

            $$
            \begin{array}{rcl}
                (True, (s,e,a)) & \Longrightarrow & (True, (s,e,a))\\
                (False, (s,e,a)) & \Longrightarrow & (False, (s,e,a))\\
                (t1 AND t2, (s,e,a)) & \Longrightarrow & (v1 \land v2, (s'',e'',a''))\\
                                    & \text{where} & (t1, (s,e,a)) \stackrel{*}{\Longrightarrow} (v1, (s',e',a'))\\
                                    &&  (t2, (s',e',a')) \stackrel{*}{\Longrightarrow} (v2,(s'',e'',a''))\\
                
                (t1 OR t2, (s,e,a)) & \Longrightarrow & (v1 \lor v2, (s'',e'',a''))\\
                                    & \text{where} & (t1, (s,e,a)) \stackrel{*}{\Longrightarrow} (v1, (s',e',a'))\\
                                    &&  (t2, (s',e',a')) \Longrightarrow (v2,(s'',e'',a''))\\
                (NOT t, (s,e,a))    & \Longrightarrow & (\neg v, (s',e',a'))\\
                                    & \text{where} & (t, (s,e,a)) \stackrel{*}{\Longrightarrow} (v, (s',e',a'))
            \end{array}
            $$

            Wie in a) macht die Ausgabe bei uns nichts. Daher wird die Semantikfunktion \emph{eval} auch
            $\varepsilon$ ausgeben.
            
    \item   Beweisen Sie die Äquivalenz der Lösungen von 3 a) und b).\\
            \textbf{Lösung:}\\

            Wir zeigen Induktiv, dass der erste Fixpunkt von $\Delta$ und der transitive
            Abschluss von $\Longrightarrow$, die selbe Variablenbelegung, die selbe Ausgabe
            und die selbe restliche Eingabe hat. Die Spitze des Wertekellers muss dabei
            den selben Wert haben, wie der Zustand des Programms nach der Ausführung ist.\\
            \begin{description}
                \item{I.A.}   Auswertung von Konstanten $W$:
                        $$\begin{array}{rcl}
                            \Delta \left< w | s | True.\varepsilon | e | a \right> 
                                &=& \left< w | True. s | \varepsilon | e | a \right>\\
                            (True, (w,e,a)) & \Longrightarrow & (True, (w,e,a)) 
                        \end{array}$$
                        Wir sind im selben Zustand gestartet mit dem selben Programm. Zurück kam
                        beide male der selbe Wert.\\
                        $$\begin{array}{rcl}
                            \Delta \left< w | s | False.\varepsilon | e | a \right> 
                                &=& \left< w | False. s | \varepsilon | e | a \right>\\
                            (False, (w,e,a)) & \Longrightarrow & (False, (w,e,a)) 
                        \end{array}$$
                        Hier, wie im anderen Fall, bekommen wir den selben Rückgabewert. Eingabe, Ausgabe
                        und Wörterbuch unterscheiden sich nicht.
                \item{I.V.} Wir können Ausdrücke $LA$ der Tiefe (Abstrakte Syntax) $n-1$ auswerten.
                \item{I.S.} $n \leadsto n+1$ 
                        $\Delta \left< w | s | (t1 op t2).k | e | a \right> = \left< w | s | t1.t2.op'.k 
                        | e | a\right>$, da $t1$, nun ein Syntaxbaum maximal die Tiefe $n-1$ hat, 
                        wissen wir, dass es eine Funktionsfolge, der länge $l1$ gibt, so dass 
                        $\Delta^{(l1)} \left< w | s | t1.t2.op'.k | e | a \right> = 
                        \left<w' | v1.s | t2.op'.k | e' | a' \right>$ äquivalent zu einer Reduktion
                        $(t1, (w,e,a)) \Longrightarrow (v1, (w',e',a'))$ ist. Da auch $t2$ maximal $n-1$
                        tief ist, existiert ebenso eine Funktionfolge
                        $\Delta^{(l2)} \left< w' | v1.s | t2.op'.k | e' | a' \right> =
                        \left< w'' | v2.v1.s | op'.k | e'' | a'' \right>$, die äquivalent zu einer
                        Reduktion $(t2, (w',e',a')) \Longrightarrow (v2, (w'',e'',a''))$ ist.
                        Nun können wir auf der Seite von $\Delta$ noch einen Schritt ausführen und
                        wir erhalten $\Delta^{(l1+l2+2)} \left< w | s | (t1 op t2).k | e | a \right>
                        = \left< w | (t1 \underline{op} t2).s | k | e | a \right>$. Auf der Seite der Reduktion
                        haben wir aus der Reduktion der Teilschritte die nötige Vorraussetzung,
                        um $t1 op t2$ reduzieren zu können.
                        $(t1 op t2, (w,e,a)) \Longrightarrow (v1 \underline{op} v2, (w'',e'',a''))$. Aller
                        Vorraussetzungen sind erfüllt. Wir wissen nun, dass wir nach $l1+l2+2$ Schritten der
                        $\Delta$ Funktion den selben Wert, wie die Reduktion erreichen.\\

                        $\Delta \left< w | s | (NOT t).k | e | a \right> = \left<w | s | t.NOT.k | e 
                        | a \right>$. In diesem Ausdruck, ist $t1$ wiederum ein Syntaxbaum, mit maximaler
                        Tiefe $n-1$. Daher wissen wir, dass es eine Reduktion 
                        $(t, (w,e,a)) \Longrightarrow (v, (w',e',a')$ gibt, die äquivalent zu 
                        $\Delta^{(l)} \left< w | s | t.k | e | a \right> 
                        = \left< w' | v | NOT.k | e' | a' \right>$ ist mit $l \geq 0$. Nach $\Delta$ Funktion,
                        können wir es noch einen Schritt ableiten zu
                        $\Delta^{(l+1)} \left< w | s | (Not t).k | e | a \right> =
                        \left< w' | (Not v).s | k | e' | a' \right>$. Bei der Reduktion haben wir die
                        Vorraussetzungen erfüllt und können nun nach der Regel folgern, dass folgende
                        Reduktion gilt $(Not v, (w',e',a'))$.
            \end{description}
            Nach Struktureller Induktion kann das eine auf das andere Abgebildet werden. Wenn wir eine
            Semantikfunktion hätten, könnten wir an dieser Stelle auch sehen, dass die Ausgaben
            von beiden Spezifikationen identisch ist. Allderdings bilden beide ohnehin auf das leere
            Wort ab.
\end{enumerate}

%% ------------------------------------------------------------
%%                  AUFGABE 4
%% ------------------------------------------------------------

\begin{enumerate}[a)]
    \item   Vereinbaren Sie einen geeigneten Datentyp LA zur Darstellung 
            von logischen Ausrücken.\\
            \textbf{Lösung:}\\
            
            Wie immer ist die Definition der Datentypen in Haskell reine Abschreibe arbeit.
            \begin{lstlisting}
                type W      = Bool
                data LOP    = AND | OR | NOT deriving (Show,Eq)
                data LA     = Value W | App LA LOP LA | Not LA  deriving Eq
                data M      = MLA LA | FUN LOP deriving (Eq,Show)
            \end{lstlisting}
            Damit wir später nichts zu tun haben, ist NOT schon Teil von LOP. Bei der
            AUswertung in Termen wird es später einen Fehler werfen, aber so
            sparen wir uns extra Kontrollwörter für den Stack der WSKEA Maschine.


    \item   Implementieren Sie die operationelle und die Reduktionssemantik
            gemäß Ihrer Lösungen zu 3 a) und b).\\
            \textbf{Lösung:}\\
           
            Schauen wir uns zuerst die delta Funktion an:
            \begin{lstlisting}
delta :: ([W],[M],[String]) -> ([W],[M],[String])
delta (s,((MLA (Value b)):k), a)        = ((b:s),k,a)
delta (s,((MLA (App t1 op t2)):k),a)    = (s, ((MLA t1):(MLA t2):(FUN op):k),a)
delta (s,((MLA (Not t)):k),a)           = (s, ((MLA t):(FUN NOT):k),a)
delta ((b:s),((FUN NOT):k),a)       
    = (((not b):s),k,(("not ("++(show b)++") = "++(show.not $ b)):a))
delta ((b2:b1:s),((FUN AND):k),a)   
    = (((b1 && b2):s),k,(((show b1)++" && "++(show b2)++" = "++(show (b1&&b2))):a))
delta ((b2:b1:s),((FUN OR):k),a)   
    = (((b1 || b2):s),k,(((show b1)++" || "++(show b2)++" = "++(show (b1||b2))):a))
delta k                                 = k
            \end{lstlisting} 
            Prinzipiell ist es wie in Aufgabe 3a) gezeigt. Wir schreiben blos bei
            konkreten Operationen, die Operation samt Ergebniss zur Kontrolle
            für später in die Ausgabe.\\

            Die Reduktionsfunktion ist auch nicht viel spannender:
            \begin{lstlisting}
red :: (LA, [String]) -> (LA, [String])   -- Programm und Ausgabe
red (Value b, a)    = (Value b, a)
red (App t1 AND t2, a')  =
    let (v1,a'')    = red (t1,a')
        (v2,a)      = red (t2,a'')
    in
        case (v1,v2) of
            (Value w1, Value w2)    
                -> (Value (w1 && w2), (((show w1)++" && "++(show w2)++" = "++(show (w1&&w2))):a))
            _   -> error "no value returned"
red (App t1 OR t2, a')   =
    let (v1,a'')    = red (t1,a')
        (v2,a)      = red (t2,a'')
    in
        case (v1,v2) of
            (Value w1, Value w2)
                -> (Value (w1 || w2), (((show w1)++" || "++(show w2)++" = "++(show (w1||w2))):a))
            _   -> error "no value returned"
red (Not t1, a')    =
    let (v1,a)  = red (t1,a') in
        case v1 of
            (Value w)   -> (Value (not w),(("not ("++(show w)++") = "++(show.not $ w)):a))
            _           -> error "no value returned"
            \end{lstlisting}
            Die Fehler sollten nie auftreten, da wir wie man sich überzeugen kann, immer einen
            Value zurückgeben.\\

            Beide Implementierungen sind analog zur Aufgabe 3. Danach haben wir noch 2 einfache
            tests geschrieben, sowie die Semantikfunktionen eval und O.

            \begin{lstlisting}
eval :: LA -> [String]
eval z = snd.red $ (z,[])

anfang :: LA -> ([W],[M],[String])
anfang prog = ([],[(MLA prog)],[])

fix :: ([W],[M],[String]) -> ([W],[M],[String])
fix z   =
    case delta z of
        z   -> z
        z'  -> fix z'

o :: ([W],[M],[String]) -> [String]
o z = 
    let (_,_,aus) = fix z in
        aus

test1, test2 :: LA
test1   = App (Value True) AND (App (Value False) OR (Value True))
test2   = Not (App (Value True) AND (App (Value True) AND (App (Value False) OR (Value False))))
            \end{lstlisting}

            Und zu guter letzt die beiden Tests ausgeführt.
            \begin{lstlisting}
LA> o.anfang $ test1
["True && True = True","False || True = True"]
LA> eval test1
["True && True = True","False || True = True"]
LA> fix.anfang $ test1
([True],[],["True && True = True","False || True = True"])
LA> red (test1,[])
(True,["True && True = True","False || True = True"])
LA> o.anfang $ test2
["not (False) = True","True && False = False","True && False = False","False || False = False"]
LA> eval.anfang $ test2
["not (False) = True","True && False = False","True && False = False","False || False = False"]
LA> fix.anfang $ test2
([True],[],["not (False) = True","True && False = False","True && False = False","False || False = False"])
LA> red (test2,[])
(True,["not (False) = True","True && False = False","True && False = False","False || False = False"])
            \end{lstlisting}

            Wie wir in Aufgabe 3 bewiesen haben, bestätigt sich hier, dass zumindest diese einfachen
            Tests, das selbe Ergebnis liefern.
 
\end{enumerate}
\label{LastPage}

\end{document}
