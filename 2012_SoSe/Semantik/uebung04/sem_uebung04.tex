\documentclass[11pt,a4paper,ngerman]{article}
\usepackage[bottom=2.5cm,top=2.5cm]{geometry} 
\usepackage{babel}
\usepackage[utf8]{inputenc} 
\usepackage[T1]{fontenc} 
\usepackage{ae} 
\usepackage{amssymb} 
\usepackage{amsmath} 
\usepackage{graphicx}
\usepackage{fancyhdr}
\usepackage{fancyref}
\usepackage{listings}
\usepackage{xcolor}
\usepackage{paralist}

%\usepackage[pdftex, bookmarks=false, pdfstartview={FitH}, linkbordercolor=white]{hyperref}
\usepackage{fancyhdr}
\pagestyle{fancy}
\fancyhead[C]{Semantik von Programmiersprachen}
\fancyhead[L]{Übung Nr. 3}
\fancyhead[R]{SoSe 2012}
\fancyfoot{}
\fancyfoot[L]{}
\fancyfoot[C]{\thepage / \pageref{LastPage}}
\renewcommand{\footrulewidth}{0.5pt}
\renewcommand{\headrulewidth}{0.5pt}
\setlength{\parindent}{0pt} 
\setlength{\headheight}{0pt}

\author{Tutor: Ansgar Schneider}
\date{}
\title{Max Wisniewski , Alexander Steen}

\begin{document}

\lstset{language=Haskell, basicstyle=\ttfamily\fontsize{10pt}{10pt}\selectfont\upshape, commentstyle=\rmfamily\slshape, keywordstyle=\rmfamily\bfseries, breaklines=true, frame=single, xleftmargin=3mm, xrightmargin=3mm, tabsize=2}

\maketitle
\thispagestyle{fancy}


%% ------------------------------------------------------
%%                     AUFGABE 1
%% ------------------------------------------------------

\section*{Aufgabe 1}

Angenommen, dass eine abstrakte Maschine und ihre Überführungsfunktion \emph{delta} gegeben sind.

\begin{enumerate}[a)]
    \item   Implementieren Sie die Semantikfunktion \emph{O}, die jeder Programm-Daten-Kombination
            die entsprechende Ausgabe zuordnet.

            \textbf{Lösung:}\\
            tbd

    \item   Testen Sie Ihre Semantikfunktion \emph{O} am Beispiel der WSKEA Maschine und $\Delta$.

\end{enumerate}
%% ------------------------------------------------------
%%                  AUFGABE 2
%% -----------------------------------------------------

\section*{Aufgabe 2}
%%
%%  Alles fertig. Muss nur gepastet werden und kommentiert.
%%
\begin{enumerate}[a)]
    \item   Implementieren Sie die Reduktionsfunktion von WHILE in einer Programmiersprache
            Ihrer Wahl.

            \textbf{Lösung:}\\
            tbd

    \item   Implementieren Sie die Semantikfunktion \emph{eval}, die jeder Programm-Daten-Kombination
            die entsprechende Ausgabe zuordnet.

            \texbt{Lösung:}\\
            tbd

    \item   Testen Sie Ihre Funktion \emph{eval} am Beispiel der ganzzahligen Division.

            \textbf{Ergebnis:}\\
            tbd
\end{enumerate}

%% ------------------------------------------------------------
%%                  AUFGABE 3
%% ------------------------------------------------------------

\section*{Aufgabe 3}

Gegeben sei folgende Syntax
\begin{tabular}{lcl}
W   & := &  True | False\\
LOP & := &  AND | OR\\
LA  & := &  W | $LA_1$ LOP $LA_2$ | NOT LA 
\end{tabular}

\textbf{Anmerkung:} Uns war es freigestellt uns eine Semantik zu überlegen.
                    Der einfachste Weg wäre es, jeden Ausdruck, den wir bekommen
                    auf True abzubilden. Damit wäre es semantisch völlig
                    spezifiziert. Wir haben uns aber an eine etwas sinnvollere
                    doch nicht zu komplizierte Semantik gesetzt. Da wir uns
                    wörtlich an die Befehle halten, haben wir daher keine Ausgabe.
                    Man könnte sich aber überlegen die Zwischenschritte in die
                    Ausgabe zu schreiben. Dies werden wir im Programmier part auch tun
                    um nachzuvollziehen, ob das richtige passiert.

\begin{enumerate}[a)]
    \item   Definieren Sie eine geeignete operationelle Semantik.\\
            \textbf{Lösung:}\\
            
            Wir führen eine weitere Menge von Kontroll-Symbolen für die abstrakte Maschiene ein
            $CON = \left\{ \underline{CAND}, \underline{COR}, \underline{CNOT} \right\}$. Damit
            ist der Zustandsraum unserer Abstrakten Maschiene $S = CON \cup W \cup LOP \cup LA$.
            Nun definieren wir die Semantikfunktion $\Delta$. Der Speicher ist die leere 
            Funktion $WK = \emptyset $, da wir keine Werte speichern können. Der Wertekeller $S$
            ist eine Liste von Weten $W$ und $E, A$ sind Listen von Werten $W$. Wobei diese nicht
            gebaucht werden.

            $$
            \begin{array}{rcl}
                \Delta \; : \; WK \times S \times K \times E \times A & \Longrightarrow & W \times S \times K \times E \times A
            \end{array}
                \Delta \left< w | s | \varepsilon | e | a \right>
                    & \equiv & \left< w | s | \varepsilon | e | a \right>\\

                \Delta \left< w | s | True.k | e | a \right>
                    & \equiv & \left< w | True.s | k | e | a \right>\\

                \Delta \left< w | s | False.k | e | a \right>
                    & \equiv & \left< w | False.s | k | e | a \right>\\

                \Delta \left< w | s | (t1 AND t2).k | e | a \right>
                    & \equiv & \left< w | s | t1.t2.CAND.k | e | a \right>\\

                \Delta \left< w | s | (t1 OR t2).k | e | a \right>
                    & \equiv & \left< w | s | t1.t2.COR.k | e | a\right>\\

                \Delta \left< w | s | (NOT t1).k | e | a \right>
                    & \equiv & \left< w | s | t1.CNOT.k | e | a\right>\\

                \Delta \left< w | v1.v2.s | CAND.k | e | a \right>
                    & \equiv & \left< w | (v1 \land v2 ) . s | k | e | a \right>\\

                \Delta \left< w | v1.v2.s | COR.k | e | a \right>
                    & \equiv & \left< w | (v1 \lor v2).s | k | e | a \right>\\

                \Delta \left< w | v.s | CNOT.k | e | a \right>
                    & \equiv & \left< w | (\neg v).s | k | e | a \right>

            $$

            Dies sind alle Überführung die wir benötigen. Da die Syntax und die Semantik keine Ausgabe 
            vorsieht , ist die Semantikfunktion einfach eine Funktion, die alles auf das leere Wort schickt.\\
            $O \; : \; LA \times W \Rightarrow W, \; (prog,eing) \mapsto \varepsilon$.

    \item   Definieren Sie eine geeignete Reduktionssemantik.\\
            \textbf{Lösung:}\\
            
            Wir übernehmen den Speicher, Ein- und Ausgabe aus Aufgabenteil a).

            $$
            \begin{array}{rcl}
                (True, (s,e,a)) & \Longrightarrow & (True, (s,e,a))\\
                (False, (s,e,a)) & \Longrightarrow & (False, (s,e,a))\\
                (t1 AND t2, (s,e,a)) & \Longrightarrow & (v1 \land v2, (s'',e'',a''))\\
                                    & \text{where} & (t1, (s,e,a)) \stackrel{*}{\Longrightarrow} (v1, (s',e',a'))\\
                                    &&  (t2, (s',e',a')) \stackrel{*}{\Longrightarrow} (v2,(s'',e'',a''))\\
                
                (t1 OR t2, (s,e,a)) & \Longrightarrow & (v1 \lor v2, (s'',e'',a''))\\
                                    & \text{where} & (t1, (s,e,a)) \stackrel{*}{\Longrightarrow} (v1, (s',e',a'))\\
                                    &&  (t2, (s',e',a')) \Longrightarrow (v2,(s'',e'',a''))\\
                (NOT t, (s,e,a))    & \Longrightarrow & (\neg v, (s',e',a'))\\
                                    & \text{where} & (t, (s,e,a)) \stackrel{*}{\Longrightarrow} (v, (s',e',a'))
            \end{array}
            $$

            Wie in a) macht die Ausgabe bei uns nichts. Daher wird die Semantikfunktion \emph{eval} auch
            $\varepsilon$ ausgeben.
            
    \item   Beweisen Sie die Äquivalenz der Lösungen von 3 a) und b).\\
            \textbf{Lösung:}\\

            Wir haben hier Zwei Teile zu zeigen:
            \begin{enumerate}[I]
                \item   Auswertung von Konstanten $W$:
                        $$
                            
                        $$
            \end{enumerate}
\end{enumerate}

%% ------------------------------------------------------------
%%                  AUFGABE 4
%% ------------------------------------------------------------

\begin{enumerate}[a)]
    \item   Vereinbaren Sie einen geeigneten Datentyp LA zur Darstellung 
            von logischen Ausrücken.\\
            \textbf{Lösung:}\\
            tbd


    \item   Implementieren Sie die operationelle und die Reduktionssemantik
            gemäß Ihrer Lösungen zu 3 a) und b).\\
            \textbf{Lösung:}\\
            tbd


\end{enumerate}
\label{LastPage}

\end{document}
