\documentclass[11pt,a4paper,ngerman]{article}
\usepackage[bottom=2.5cm,top=2.5cm]{geometry} 
\usepackage{babel}
\usepackage[utf8]{inputenc} 
\usepackage[T1]{fontenc} 
\usepackage{ae} 
\usepackage{amssymb} 
\usepackage{amsmath} 
\usepackage{graphicx}
\usepackage{fancyhdr}
\usepackage{fancyref}
\usepackage{listings}
\usepackage{xcolor}
\usepackage{paralist}

%\usepackage[pdftex, bookmarks=false, pdfstartview={FitH}, linkbordercolor=white]{hyperref}
\usepackage{fancyhdr}
\pagestyle{fancy}
\fancyhead[C]{Semantik von Programmiersprachen}
\fancyhead[L]{Übung Nr. 3}
\fancyhead[R]{SoSe 2012}
\fancyfoot{}
\fancyfoot[L]{}
\fancyfoot[C]{\thepage / \pageref{LastPage}}
\renewcommand{\footrulewidth}{0.5pt}
\renewcommand{\headrulewidth}{0.5pt}
\setlength{\parindent}{0pt} 
\setlength{\headheight}{0pt}

\author{Tutor: Ansgar Schneider}
\date{}
\title{Max Wisniewski , Alexander Steen}

\begin{document}

\lstset{language=Haskell, basicstyle=\ttfamily\fontsize{10pt}{10pt}\selectfont\upshape, commentstyle=\rmfamily\slshape, keywordstyle=\rmfamily\bfseries, breaklines=true, frame=single, xleftmargin=3mm, xrightmargin=3mm, tabsize=2}

\maketitle
\thispagestyle{fancy}


%% ------------------------------------------------------
%%                     AUFGABE 1
%% ------------------------------------------------------

\section*{Aufgabe 1}

Angenommen, dass eine abstrakte Maschine und ihre Überführungsfunktion \emph{delta} gegeben sind.

\begin{enumerate}[a)]
    \item   Implementieren Sie die Semantikfunktion \emph{O}, die jeder Programm-Daten-Kombination
            die entsprechende Ausgabe zuordnet.

            \textbf{Lösung:}\\
            tbd

    \item   Testen Sie Ihre Semantikfunktion \emph{O} am Beispiel der WSKEA Maschine und $\Delta$.

\end{enumerate}
%% ------------------------------------------------------
%%                  AUFGABE 2
%% -----------------------------------------------------

\section*{Aufgabe 2}

\begin{enumerate}[a)]
    \item   Implementieren Sie die Reduktionsfunktion von WHILE in einer Programmiersprache
            Ihrer Wahl.

            \textbf{Lösung:}\\
            tbd

    \item   Implementieren Sie die Semantikfunktion \emph{eval}, die jeder Programm-Daten-Kombination
            die entsprechende Ausgabe zuordnet.

            \texbt{Lösung:}\\
            tbd

    \item   Testen Sie Ihre Funktion \emph{eval} am Beispiel der ganzzahligen Division.

            \textbf{Ergebnis:}\\
            tbd
\end{enumerate}

%% ------------------------------------------------------------
%%                  AUFGABE 3
%% ------------------------------------------------------------

\section*{Aufgabe 3}

Gegeben sei folgende Syntax
\begin{tabular}{lcl}
W   & := &  True | False\\
LOP & := &  AND | OR\\
LA  & := &  W | $LA_1$ LOP $LA_2$ | Not LA 
\end{tabular}

\begin{enumerate}[a)]
    \item   Definieren Sie eine geeignete operationelle Semantik.\\
            \textbf{Lösung:}\\
            tbd

    \item   Definieren Sie eine geeignete Reduktionssemantik.\\
            \textbf{Lösung:}\\
            tbd
    \item   Beweisen Sie die Äquivalenz der Lösungen von 3 a) und b).\\
            \textbf{Lösung:}\\
\end{enumerate}

%% ------------------------------------------------------------
%%                  AUFGABE 4
%% ------------------------------------------------------------

\begin{enumerate}[a)]
    \item   Vereinbaren Sie einen geeigneten Datentyp LA zur Darstellung 
            von logischen Ausrücken.\\
            \textbf{Lösung:}\\
            tbd

    \item   Implementieren Sie die operationelle und die Reduktionssemantik
            gemäß Ihrer Lösungen zu 3 a) und b).\\
            \textbf{Lösung:}\\
            tbd
\end{enumerate}
\label{LastPage}

\end{document}
