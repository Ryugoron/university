\documentclass[11pt,a4paper,ngerman]{article}
\usepackage[bottom=2.5cm,top=2.5cm,left=2cm,right=2cm]{geometry} 
\usepackage{babel}
\usepackage[utf8]{inputenc} 
\usepackage[T1]{fontenc} 
\usepackage{ae} 
\usepackage{amssymb} 
\usepackage{amsmath} 
\usepackage{graphicx}
\usepackage{fancyhdr}
\usepackage{fancyref}
\usepackage{listings}
\usepackage{xcolor}
\usepackage{paralist}
%\usepackage{bussproofs}
\usepackage{stmaryrd}

%\usepackage[pdftex, bookmarks=false, pdfstartview={FitH}, linkbordercolor=white]{hyperref}
\usepackage{fancyhdr}
\pagestyle{fancy}
\fancyhead[C]{Semantik von Programmiersprachen}
\fancyhead[L]{Übung Nr. 11}
\fancyhead[R]{SoSe 2012}
\fancyfoot{}
\fancyfoot[L]{}
\fancyfoot[C]{\thepage / \pageref{LastPage}}
\renewcommand{\footrulewidth}{0.5pt}
\renewcommand{\headrulewidth}{0.5pt}
\setlength{\parindent}{0pt} 
\setlength{\headheight}{0pt}

\author{Tutor: Ansgar Schneider}
\date{}
\title{Max Wisniewski , Alexander Steen}

\begin{document}

\lstset{language=Haskell, basicstyle=\ttfamily\fontsize{10pt}{10pt}\selectfont\upshape, commentstyle=\rmfamily\slshape, keywordstyle=\rmfamily\bfseries, breaklines=true, frame=single, xleftmargin=3mm, xrightmargin=3mm, tabsize=2, mathescape=true}
   
\maketitle
\thispagestyle{fancy}
\newcommand{\A}{\mathbb{A}}

\subsection*{Aufgabe 1 \mdseries\itshape Fortsetzungssemantik}

   Schreiben Sie ein WHILE-Programm C zur Berechnung des Quotienten zweier ganzer Zahlen. Beweisen
   Sie, dass die Fortsetzungssemantik $\mathcal{P} [C] <3,2> = <1>$ gilt.\\

\textbf{Beweis:}\\
   Unser Programm sieht wie folgt aus
   $C \equiv output (read \div read)$\\

   Nun beweisen wir, dass dieses Programm die Spezifikation erfüllt:\\
   $$\begin{array}{rcl}
      &&\mathcal{P}[output (read \div read)] <3,2> \\&=& \mathcal{C}[output (read \div read)] id \star \pi_3
         (s_0,<3,2>, \varepsilon)\\
      &=& \mathcal{T}[read \div read] (\lambda n(s,e,a). id (s,e,a.n)) \star \pi_3 (s_0,<3,2>, \varepsilon)\\
      &=& \mathcal{T}[read] (\lambda t_1. \mathcal{T}[read]
         (\lambda t_2. (\lambda n (s,e,a). (s,e,a.n)) (t_1 \div t_2)))\star \pi_3 (s_0,<3,2>, \varepsilon)\\
      &=& \mathcal{T}[read] (\lambda t_1. \mathcal{T}[read] 
         (\lambda t_2 (\lambda (s,e,a).(s,e,a.(t_1 \div t_2))))) (s_0,<3,2>, \varepsilon)\\
      &=& (\lambda t_1. \mathcal{T}[read](\lambda t_2. (\lambda (s,e,a).(s,e,a.(t_1 \div t_2))))) 
         \star \pi _3 3 (s_0,<2>,\varepsilon)\\
      &=& \mathcal{T}[read](\lambda t_2.(\lambda (s,e,a).(s,e,a.(3 \div t_2)))) \star \pi_3 
         (s_0,<2>,\varepsilon)\\
      &=& (\lambda t_2.(\lambda (s,e,a).(s,e,a.(3 \div t_2)))) \star \pi_3 2 (s_0,\varepsilon,\varepsilon)\\
      &=& (\lambda (s,e,a).(s,e,a.(3 \div 2))) \star \pi_3 (s_0, \varepsilon, \varepsilon)\\
      &=& \pi_3 (s_0, \varepsilon, \varepsilon.(3 \div 2))\\
      &=& \pi_3 (s_0, \varepsilon, <1>)\\
      &=& <1>
   \end{array}$$

   Damit haben wir die Spezifikation bewiesen und unser Programm stimmt daher.

\subsection*{Aufgabe 2 \mdseries\itshape FOR-Schleifen}
   Erweitern Sie die Sprache WHILE um FOR - Schleifen.
   Erklären Sie den deren Fortsetzungssemantik.\\

\textbf{Lösung:}\\
   Zunächst erweitern wir $C$ um das Kontrollkonstrukt FOR I:=$T_1$ TO $T_2$ DO C.\\

   Die Semantik sieht wie folgt aus:\\
   $\mathcal{C}[FOR \, I:=T_1 \, TO \, T_2 \, DO \, C] :=$\\ 
   $\mathcal{T}[T_1] (\lambda s.\mathcal{T}[T_2](\lambda e. \mathcal{C}[I:=s] \mathcal{B}[s<e] \text{cond}
      <\mathcal{C}[C]\circ \mathcal{C}[FOR \, I := s + 1 \, TO \, e \, DO C], \lambda z.z>))$

   Wir werten also zunächst $T_1$ aus und geben den Wert in unsere restliche Funktion. Nach dem
   ersten Auswerten ist dieser Wert $s$ nur noch inkrementiert und nicht mehr berechnet. 
   Als nächstes Werten wir das Ende aus, dass in jeder Iteration auch fix bleibt. 
   Danach prüfen wir jedes mal, ob $s<e$ gilt, wenn es gilt, führen wir das ganze noch einmal aus
   und wenn nicht, dann geben wir den Zustand einfach zurück.

\pagebreak

\subsection*{Aufgabe 3 \mdseries\itshape $\star$ Operator}
   Erläutern Sie, warum der $\star$ Operator in der Fortsetzungssemantik kaum
   noch vor kommt.\\

\textbf{Lösung:}\\
   Dies liegt daran, dass die Fortsetzungssemantik schon das ganze als erweiterte
   Konkatenation auswertet. Bei $\mathcal[C_1;C_2]$ wird zunächst $C_1$ ausgewertet
   und danach $C_2$ auf das Ergebnis angewandt. Das schöne daran ist,
   das die Fortsetzung $C_2$ gar nicht ausführen wird, wann in $C_1$ schon ein
   Fehler aufgetreten ist. Daher ist der Sternoperator schon durch die
   Fortsetzungssemantik nativ gegeben.

\subsection*{Aufgabe 4 \mdseries\itshape Fortsetzungssemantik II}
   Jemand hat bei der Definition der Fortsetzungssemantik einen Fehler gemacht:
   $$
      \mathcal{P}[C] e = \mathcal{C}[C]((\lambda z.z) \star \pi_3) <s_0, e , \varepsilon >
   $$
   Wo liegt der Fehler genau?\\

\textbf{Lösung:}\\
   Der Fehler besteht darin, dass der Ausdruck falsch geklammert ist.\\
   $((\lambda z.z) \star \pi_3)$ wird dafür sorgen, dass der Zustand aus der
   Eingabe einfach genommen wird und sofort danach die Ausgabe $\varepsilon$
   heraus getrennt wird. Die Fortsetzungssemantik bekommt also einen Falschen Typ
   geliefert, da $\mathcal{C}[C]$ einen Zustand erwartet, aber nur die Ausgabe bekommt.\\

   Klammert man es indes anders herum, ist alles korrekt. Da das $\lambda$ Kalkül
   ohnehin linksassoziativ geklammert ist, aber leider auch unnötig.

\label{LastPage}

\end{document}
