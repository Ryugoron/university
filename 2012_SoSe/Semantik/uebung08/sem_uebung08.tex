\documentclass[11pt,a4paper,ngerman]{article}
\usepackage[bottom=2.5cm,top=2.5cm,left=2cm,right=2cm]{geometry} 
\usepackage{babel}
\usepackage[utf8]{inputenc} 
\usepackage[T1]{fontenc} 
\usepackage{ae} 
\usepackage{amssymb} 
\usepackage{amsmath} 
\usepackage{graphicx}
\usepackage{fancyhdr}
\usepackage{fancyref}
\usepackage{listings}
\usepackage{xcolor}
\usepackage{paralist}
%\usepackage{bussproofs}
\usepackage{stmaryrd}

%\usepackage[pdftex, bookmarks=false, pdfstartview={FitH}, linkbordercolor=white]{hyperref}
\usepackage{fancyhdr}
\pagestyle{fancy}
\fancyhead[C]{Semantik von Programmiersprachen}
\fancyhead[L]{Übung Nr. 8}
\fancyhead[R]{SoSe 2012}
\fancyfoot{}
\fancyfoot[L]{}
\fancyfoot[C]{\thepage / \pageref{LastPage}}
\renewcommand{\footrulewidth}{0.5pt}
\renewcommand{\headrulewidth}{0.5pt}
\setlength{\parindent}{0pt} 
\setlength{\headheight}{0pt}

\author{Tutor: Ansgar Schneider}
\date{}
\title{Max Wisniewski , Alexander Steen}

\begin{document}

\lstset{language=Haskell, basicstyle=\ttfamily\fontsize{10pt}{10pt}\selectfont\upshape, commentstyle=\rmfamily\slshape, keywordstyle=\rmfamily\bfseries, breaklines=true, frame=single, xleftmargin=3mm, xrightmargin=3mm, tabsize=2, mathescape=true}

\maketitle
\thispagestyle{fancy}
\newcommand{\A}{\mathbb{A}}

%% ------------------------------------------------------
%%                     AUFGABE 1
%% ------------------------------------------------------

\section*{Aufgabe 1} 
\begin{description}
\item[a)] Gesucht: Funktion $f : A \to B$, mit $A,B$ cpos, $f$ nicht stetig. \\
Sei $|A| \geq 2, |B| \geq 2$, sei $\alpha \neq \bot_A \in A$ und $\beta \neq \bot_B \in B$.\\
Sei $f : A \to B$, mit $a \mapsto \begin{cases}
\bot_B & \text{, falls } a \neq \bot_A \\
\beta & \text{, sonst}
\end{cases}$\\
$f$ ist nicht stetig, da für die Kette $\{\bot_A, \alpha \}$ gilt: 

\begin{eqnarray*}
\sup f(\{\bot_A, \alpha \}) &=& \sup \{\beta, \bot_B\}\\
&=& \beta \\
&\neq & \bot_B \\
&=& f(\alpha ) \\
&=& f(\sup \{ \bot_A, \alpha\})
\end{eqnarray*}

\item[b)] Z.z. $f: B \to C, g: A \to B$ stetig $\Rightarrow f \circ g: A \to C$ stetig, mit $A,B,C$ cpo. 

\textbf{Beweis}:\\
Sei $f: B \to C, g: A \to B$ stetig, $A,B,C$ cpos und $K \subseteq A$ eine Kette. Dann ist $(f \circ g)(K) = f(g(K))$. \\
(1) $g$ stetig $\Rightarrow$ $G := g(K) \subseteq B$ Kette. Da $f$ stetig $\Rightarrow$ $f(G) = f(g(K)) = (f \circ g)(K) \subseteq C$ Kette.\\

(2) $f$ stetig $\Rightarrow \forall K' \subseteq  B \text{ Kette }: \, f(\sup K') = \sup f(K')$ und\\
$g$ stetig $\Rightarrow \forall K' \subseteq  A \text{ Kette }: \, g(\sup K') = \sup g(K')$.\\
Sei $G := g(K) \subseteq B $ Kette (nach (1)).\\
$\Rightarrow (f \circ g)(\sup K) = f(g(\sup K)) =f(\sup g(K)) = f(\sup G) = \sup f(G) = \sup f(g(K)) = \sup (f \circ g)(K)$.
\mbox{} \hfill $\square$
\end{description}

%% ------------------------------------------------------
%%                     AUFGABE 2
%% ------------------------------------------------------

\section*{Aufgabe 2}


%% ------------------------------------------------------
%%                     AUFGABE 3
%% ------------------------------------------------------

\section*{Aufgabe 3}


%% ------------------------------------------------------
%%                     AUFGABE 4
%% ------------------------------------------------------

\section*{Aufgabe 4}
Seien $D_1, D_2$ cpos und $f: D_1 \to D_2, g: D_2 \to D_1$ stetig.\\
Z.z. $fix_{f\circ g} = f(fix_{g\circ f})$ und $fix_{g\circ f} = g(fix_{f\circ g})$.\\

\textbf{Beweis}:\\
Nach Aufgabe 1 gilt $f \circ g : D_2 \to D_2$ stetig und $g \circ f : D_1 \to D_1$ stetig. Dann gilt nach dem Fixpunktsatz: $fix_{f \circ g}$ und $fix_{g \circ f}$ existieren und
$$fix_{f \circ g} = \sup_{n \in \mathbb{N}} \{(f \circ g)^n(\bot_{D_1}) \}, \quad fix_{g \circ f} = \sup_{n \in \mathbb{N}} \{(g \circ f)^n(\bot_{D_2}) \}$$

Da sowohl $\{(f \circ g)^n(\bot_{D_1}) |\, n\in \mathbb{N}\}$ als auch $\{(g \circ f)^n(\bot_{D_2}) |\, n\in \mathbb{N}\}$ Ketten sind (siehe Beweis zu Satz 3.7), gilt:\\

\begin{eqnarray*}
f (fix_{g \circ f}) &=& f(\sup_{n \in \mathbb{N}} \{(g \circ f)^n(\bot_{D_2}) \}) \\
&\stackrel{f \text{ stetig}}{=}& \sup_{n \in \mathbb{N}} \{f((g \circ f)^n(\bot_{D_2})) \}\\
&\stackrel{\text{Umordnung}}{=}& \sup_{n \in \mathbb{N}} \{(f \circ g)^n(f(\bot_{D_2})) \}\\
&\stackrel{(*)}{=}& \sup_{n \in \mathbb{N}} \{(f \circ g)^n(\bot_{D_1}) \}\\
&=& fix_{f\circ g}
\end{eqnarray*}

\begin{eqnarray*}
g (fix_{f \circ g}) &=& g(\sup_{n \in \mathbb{N}} \{(f \circ g)^n(\bot_{D_1}) \}) \\
&\stackrel{g \text{ stetig}}{=}& \sup_{n \in \mathbb{N}} \{g((f \circ g)^n(\bot_{D_1})) \}\\
&\stackrel{\text{Umordnung}}{=}& \sup_{n \in \mathbb{N}} \{(g \circ f)^n(g(\bot_{D_1})) \}\\
&\stackrel{(*)}{=}& \sup_{n \in \mathbb{N}} \{(g \circ f)^n(\bot_{D_2}) \}\\
&=& fix_{g\circ f}
\end{eqnarray*}

(*) gilt, weil $f,g$ strikt sind....blabla
\mbox{} \hfill $\square$
\label{LastPage}

\end{document}
