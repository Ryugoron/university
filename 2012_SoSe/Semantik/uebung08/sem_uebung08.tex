\documentclass[11pt,a4paper,ngerman]{article}
\usepackage[bottom=2.5cm,top=2.5cm,left=2cm,right=2cm]{geometry} 
\usepackage{babel}
\usepackage[utf8]{inputenc} 
\usepackage[T1]{fontenc} 
\usepackage{ae} 
\usepackage{amssymb} 
\usepackage{amsmath} 
\usepackage{graphicx}
\usepackage{fancyhdr}
\usepackage{fancyref}
\usepackage{listings}
\usepackage{xcolor}
\usepackage{paralist}
%\usepackage{bussproofs}
\usepackage{stmaryrd}

%\usepackage[pdftex, bookmarks=false, pdfstartview={FitH}, linkbordercolor=white]{hyperref}
\usepackage{fancyhdr}
\pagestyle{fancy}
\fancyhead[C]{Semantik von Programmiersprachen}
\fancyhead[L]{Übung Nr. 8}
\fancyhead[R]{SoSe 2012}
\fancyfoot{}
\fancyfoot[L]{}
\fancyfoot[C]{\thepage / \pageref{LastPage}}
\renewcommand{\footrulewidth}{0.5pt}
\renewcommand{\headrulewidth}{0.5pt}
\setlength{\parindent}{0pt} 
\setlength{\headheight}{0pt}

\author{Tutor: Ansgar Schneider}
\date{}
\title{Max Wisniewski , Alexander Steen}

\begin{document}

\lstset{language=Haskell, basicstyle=\ttfamily\fontsize{10pt}{10pt}\selectfont\upshape, commentstyle=\rmfamily\slshape, keywordstyle=\rmfamily\bfseries, breaklines=true, frame=single, xleftmargin=3mm, xrightmargin=3mm, tabsize=2, mathescape=true}

\maketitle
\thispagestyle{fancy}
\newcommand{\A}{\mathbb{A}}

%% ------------------------------------------------------
%%                     AUFGABE 1
%% ------------------------------------------------------

\section*{Aufgabe 1} 
\begin{description}
\item[a)] Gesucht: Funktion $f : A \to B$, mit $A,B$ cpos, $f$ nicht stetig. \\
Sei $|A| \geq 2, |B| \geq 2$, sei $\alpha \neq \bot_A \in A$ und $\beta \neq \bot_B \in B$.\\
Sei $f : A \to B$, mit $a \mapsto \begin{cases}
\bot_B & \text{, falls } a \neq \bot_A \\
\beta & \text{, sonst}
\end{cases}$\\
$f$ ist nicht stetig, da für die Kette $\{\bot_A, \alpha \}$ gilt: 

\begin{eqnarray*}
\bigsqcup f(\{\bot_A, \alpha \}) &=& \bigsqcup \{\beta, \bot_B\}\\
&=& \beta \\
&\neq & \bot_B \\
&=& f(\alpha ) \\
&=& f(\bigsqcup \{ \bot_A, \alpha\})
\end{eqnarray*}

\item[b)] Z.z. $f: B \to C, g: A \to B$ stetig $\Rightarrow f \circ g: A \to C$ stetig, mit $A,B,C$ cpo. 

\textbf{Beweis}:\\
Sei $f: B \to C, g: A \to B$ stetig, $A,B,C$ cpos und $K \subseteq A$ eine Kette. Dann ist $(f \circ g)(K) = f(g(K))$. \\
(1) $g$ stetig $\Rightarrow$ $G := g(K) \subseteq B$ Kette. Da $f$ stetig $\Rightarrow$ $f(G) = f(g(K)) = (f \circ g)(K) \subseteq C$ Kette.\\

(2) $f$ stetig $\Rightarrow \forall K' \subseteq  B \text{ Kette }: \, f(\bigsqcup K') = \bigsqcup f(K')$ und\\
$g$ stetig $\Rightarrow \forall K' \subseteq  A \text{ Kette }: \, g(\bigsqcup K') = \bigsqcup g(K')$.\\
Sei $G := g(K) \subseteq B $ Kette (nach (1)).\\
$\Rightarrow (f \circ g)(\bigsqcup K) = f(g(\bigsqcup K)) =f(\bigsqcup g(K)) = f(\bigsqcup G) = \bigsqcup f(G) = \bigsqcup f(g(K)) = \bigsqcup (f \circ g)(K)$.
\mbox{} \hfill $\square$
\end{description}

\pagebreak

%% ------------------------------------------------------
%%                     AUFGABE 2
%% ------------------------------------------------------

\section*{Aufgabe 2}

\begin{enumerate}[(a)]
	\item Zeigen Sie, wie Sie zu gegebene cpos $D_1,...,D_n$ mit $n \geq 2$ den Bereich der disjunkten Vereinigung 
		$D = (D_1+...+D_n)$ erklären können.\\
	\textbf{Lösung:}\\
		Als erstes stellen wir fest, dass es sich um die selbe Notation handelt, wie bei den Summenbereichen. Daher sollte
		die Lösung am Ende eine ähnliche Strucktur aufzeigen.\\

		Als erstes bilden wir das $\bot$ Elemente von $D$. Da die Mengen alle disjunkt sind, d.h. es existiert kein eindeutiges Minimum
		identifizieren wir wieder ale Minima mit einander. $\bot = \bot_i, \forall \bot_i \in D_i$.\\
		Danach ist die Disjunkte Vereinigung einfach $D = D_1 \overset{\cdot}{\cup} ... \overset{\cdot}{\cup} D_n$.\\

		Nun ist $(D, \sqsubseteq_D)$ ein cpo, mit $\sqsubseteq_D$:\\
		$(\bot,a) \in \subseteq_D$, für alle $a \in D$.\\
		$(a,b) \in \subseteq_D$, falls $\exists D_i \; : \; a,b \in D_i \, \land\, a \sqsubseteq_i b$.\\

		Die Struktur ist ein cpo, weil wir ein eindeutiges $\bot$ Element haben, das kleiner als alle anderen ist.
		Darüber hinaus, besitzt jede Kette ein Supremum, da nach Definition von $\sqsubseteq_D$ nur Ketten
		existieren können, die vorher in einem der cpos $D_i$ existeirt haben. Da dies aber ein cpo war,
		musste die Kette vorher ein supremum gehabt haben und damit auch in $D$.

	\item Definieren Sie folgende Injektions-, Projektions- und Testfunktion in kanonischer Weise:\\
		$$\begin{array}{rcl}
			in_i \; : \; D_i &\longrightarrow& (D_1 + ... + D_n)\\
				a & \longmapsto & \left\{ \begin{array}{lr} a &\text{, falls }a\not= \bot_{D_i} \\ \bot_D &, \text{sonst} \end{array} \right.\\
			out_i \; : \; (D_1 + ... + D_n) &\longrightarrow& D_i\\
				a & \longmapsto & \left\{ \begin{array}{lr} a &,\text{ falls } a \in D_i \\ \bot_{D_i} &, \text{ sonst }\end{array} \right.\\
			is_i \; : \; (D_1 + ... D_n) &\longrightarrow& BOOL_\bot\\
				a & \longmapsto& \left\{ \begin{array}{lr}
					wahr &, \text{ falls }a \in D_i \; \land \; a \not= \bot_D\\
					false &, \text{ falls }a \not\in D_i  \; \land \; a \not= \bot_D\\
					\bot_{BOOL} &, \text{ falls } d = \bot_D
				\end{array} \right.
		\end{array}$$

	Diese Definitionen sind analog zum Summenbereich gehalten. Da in unserem Fall aber die eindeutigekeit der Elemente gegeben ist,
	kann man, wie im Buch erwähnt, auch einfach die Funktionen weglassen und die ursprünglichen Elemente als Stellvertreter nehmen.
\end{enumerate}

\pagebreak

%% ------------------------------------------------------
%%                     AUFGABE 3
%% ------------------------------------------------------

\section*{Aufgabe 3}
Definieren Sie stetige Erweiterung der Addition und des Tests auf Gleichheit, so dass diese Operation total werden auf den cpo's $\mathbb{N}_\bot$ und $Bool_\bot$. Diskutieren Sie, ob es mehere solcher Erweiterungen gibt.\\

\textbf{Lösung:}\\
Bei den beiden cpo's, die vorgegeben wurden, handelt es sich um flache cpo's.\\

Die erste Definition die wir daher wählen, ist eine strikte Erweiterung.
$$\begin{array}{rcl}
	+ \; \mathbb{N}_\bot \times \mathbb{N}_\bot & \longrightarrow & \mathbb{N}_\bot\\
		(a,b) &\mapsto & \left\{\begin{array}{lr}
			a+b &, \text{ falls } a\not= \bot \; \land \; b \not=\bot\\
			\bot &, \text{ sonst }
		\end{array}\right.\\
	= \; \mathbb{N}_\bot \times \mathbb{N}_\bot &\longrightarrow& Bool_\bot\\
		(a,b) &\mapsto & \left\{\begin{array}{lr}
			wahr &, \text{ falls } a = b \; \land ;a \not= \bot \; \land \; b \not= \bot\\
			false &, \text{ falls } a \not= b \; \land ;a \not= \bot \; \land \; b \not= \bot\\
			\bot &, \text{ falls } a = \bot \; \lor \; b = \bot
		\end{array}\right. \\
	= \; Bool_\bot \times Bool_\bot & \longrightarrow & Bool_\bot\\
		(a,b) &\mapsto & \left\{ \begin{array}{lr}
			a \Leftrightarrow b &, \text{ fallse } a\not= \bot \; \land b \not= \bot\\
			\bot &, \text{ sonst}
		\end{array} \right.
\end{array}$$

In der Vorlesung haben wir gelernt, das eine strikte Erweiterung einer stetigen Funktion immer eine stetige Funktion ergibt. Daher sind diese Definierten Funktionen alle Erweiterungen, die die Aufgabe erfüllen.\\

Darüber hinaus, konnten wir keine weiteren Erweiterungen ausmachen, die sich als stetig erwiesen hätten.\\
Das Problem dabei ist immer, dass die Monotonie gewart bleiben muss.\\


Bilden wir beim test auf Gleichheit z.B. $\bot = \bot \mapsto wahr$ ab, so muss nach monotonie jedes Urbild - Element, das größer ist,
auf ein Bildelement abgebildet werden, das ebenfalls mindestens so groß ist. Damit ergebe die einzige alternative im Falle vom Test auf Gleichheit
eine Konstante $wahr$ oder $falsch$ Funktion.\\
Diese erschien uns aber nicht Sinnvoll\\

Beim $+$ ergibt sich ein ähnliches Problem mit der Monotonie. Sollte man $a + \bot$ nicht auf $\bot$ zuweisen, dann hat man eine Zahl $b$ erreicht.\\
Da nun aber $(a,\bot) \sqsubseteq (a,c)$ für ein beliebiges $c$ gilt, muss $b \sqsubseteq a+c$ gelten. Da in flachen cpo's die Zahlen aber nicht
untereinander vergleichabr sind, kann $a+c$ nur auf $b$ abgebildet werden, also ebenfalls konstant sein.

\pagebreak

%% ------------------------------------------------------
%%                     AUFGABE 4
%% ------------------------------------------------------

\section*{Aufgabe 4}
Seien $D_1, D_2$ cpos und $f: D_1 \to D_2, g: D_2 \to D_1$ stetig.\\
Z.z. $fix_{f\circ g} = f(fix_{g\circ f})$ und $fix_{g\circ f} = g(fix_{f\circ g})$.\\

\textbf{Beweis}:\\
Nach Aufgabe 1 gilt $f \circ g : D_2 \to D_2$ stetig und $g \circ f : D_1 \to D_1$ stetig. Dann gilt nach dem Fixpunktsatz: $fix_{f \circ g}$ und $fix_{g \circ f}$ existieren und
$$fix_{f \circ g} = \bigsqcup_{n \in \mathbb{N}} \{(f \circ g)^n(\bot_{D_1}) \}, \quad fix_{g \circ f} = \bigsqcup_{n \in \mathbb{N}} \{(g \circ f)^n(\bot_{D_2}) \}$$

Da sowohl $\{(f \circ g)^n(\bot_{D_1}) |\, n\in \mathbb{N}\}$ als auch $\{(g \circ f)^n(\bot_{D_2}) |\, n\in \mathbb{N}\}$ Ketten sind (siehe Beweis zu Satz 3.7), gilt:\\

\begin{eqnarray*}
f (fix_{g \circ f}) &=& f(\bigsqcup_{n \in \mathbb{N}} \{(g \circ f)^n(\bot) \}) \\
&\stackrel{f \text{ stetig}}{=}& \bigsqcup_{n \in \mathbb{N}} \{f((g \circ f)^n(\bot)) \}\\
&\stackrel{\text{Umordnung}}{=}& \bigsqcup_{n \in \mathbb{N}} \{(f \circ g)^n(f(\bot)) \}\\
&\stackrel{(*)}{=}& \bigsqcup_{n \in \mathbb{N}} \{(f \circ g)^n(\bot) \}\\
&=& fix_{f\circ g}
\end{eqnarray*}

\begin{eqnarray*}
g (fix_{f \circ g}) &=& g(\bigsqcup_{n \in \mathbb{N}} \{(f \circ g)^n(\bot) \}) \\
&\stackrel{g \text{ stetig}}{=}& \bigsqcup_{n \in \mathbb{N}} \{g((f \circ g)^n(\bot)) \}\\
&\stackrel{\text{Umordnung}}{=}& \bigsqcup_{n \in \mathbb{N}} \{(g \circ f)^n(g(\bot)) \}\\
&\stackrel{(*)}{=}& \bigsqcup_{n \in \mathbb{N}} \{(g \circ f)^n(\bot) \}\\
&=& fix_{g\circ f}
\end{eqnarray*}

(*) Da die Funktionen stetig sind, können wir die Resultierenden Ketten $(f\circ g)(a), \, (f \circ g)^2 (a), ...$ nach unten erweitern.
Es gilt, da die Funktion auch monoton ist, dass $\bot \sqsubseteq a \Rightarrow f(\bot) \sqsubseteq f(a)$. Da wir nur am supremum
interessiert sind, ersetzen wir also das kleinste Element der Kette durch $\bot$ und das Supremum der Folge bleibt immer noch gleich.
\mbox{} \hfill $\square$
\label{LastPage}

\end{document}
