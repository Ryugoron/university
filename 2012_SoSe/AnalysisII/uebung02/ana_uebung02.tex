\documentclass[11pt,a4paper,ngerman]{article}
\usepackage[bottom=2.5cm,top=2.5cm]{geometry} 
\usepackage{babel}
\usepackage[utf8]{inputenc} 
\usepackage[T1]{fontenc} 
\usepackage{ae} 
\usepackage{amssymb} 
\usepackage{amsmath} 
\usepackage{graphicx}
\usepackage{fancyhdr}
\usepackage{fancyref}
\usepackage{listings}
\usepackage{xcolor}
\usepackage{paralist}

\usepackage[pdftex, bookmarks=false, pdfstartview={FitH}, linkbordercolor=white]{hyperref}
\usepackage{fancyhdr}
\pagestyle{fancy}
\fancyhead[C]{Analysis II}
\fancyhead[L]{Aufgabenblatt 2}
\fancyhead[R]{SoSe 2012}
\fancyfoot{}
\fancyfoot[L]{}
\fancyfoot[C]{\thepage \hspace{1px} of \pageref{LastPage}}
\renewcommand{\footrulewidth}{0.5pt}
\renewcommand{\headrulewidth}{0.5pt}
\setlength{\parindent}{0pt} 
\setlength{\headheight}{0pt}

\date{}
\title{Max Wisniewski , Alexander Steen}
\author{Tutor : Adrian Steffens}

\begin{document}

\lstset{language=Pascal, basicstyle=\ttfamily\fontsize{10pt}{10pt}\selectfont\upshape, commentstyle=\rmfamily\slshape, keywordstyle=\rmfamily\bfseries, breaklines=true, frame=single, xleftmargin=3mm, xrightmargin=3mm, tabsize=2}

\maketitle
\thispagestyle{fancy}

%% ------------------------------------------------------
%%                     Aufgabe 5
%% ------------------------------------------------------

\subsection*{Aufgabe 5: \mdseries\itshape Unbestimmte Integrale}

Finden Sie die folgenden unbestimmten Integrale.

\begin{enumerate}[(i)]
	\item $\int (\log x)^2 \, dx$:\\
		Wir benutzen das Substitutionsverfahren und wählen $y = \log x$ als neue Basis. Damit ist $x = e^y$ und $dx = e^y \, dy$. 
		Eingesetzt in die Gleichung erhalten wir
		$$\begin{array}{rcl}
			\int (log x)^2 \, dx 	&\stackrel{sub. \, y}{=}& \int y^2 \cdot e^y \, dx\\
						&\stackrel{part.}{=}&  y^2\cdot e^y - \int 2ye^x \, dy\\
						&\stackrel{part.}{=}& y^2e^y  - 2ye^y + 2e^y\\
						&=& e^y \cdot (y^2 - 2y + 1) + e^y\\
						&=& e^y \cdot (y-1)^2 + e^y\\
						&\stackrel{resub. \, y}{=}& x \cdot ((\log x) - 1)^2 + x.
		\end{array}$$
		Dies können wir nun noch einmal ableiten um die Lösung zu verifizieren.
		$$\begin{array}{rcl}
			\frac{d}{dx} x \cdot ((\log x) - 1)^2 +x
							&=& (\frac{d}{dx} x \cdot (\log x - 1))^2 + 1\\
							&=& (\log x - 1)^2 + x \cdot 2 (\log x - 1) \frac {1}{x} + 1\\
							&=& ((\log x - 1) + 1)^2 = (\log x)^2.
		\end{array}$$
		\mbox{}\hfill $\square$

	\item $\int \frac{1}{x \log x} \, dx$:\\
		Wir substituiren wieder druch $y=\log x$ und bekommen $x=e^y$ und $dx = e^y \, dy$.
		$$\begin{array}{rcl}
			\int \frac{1}{x \log x} \, dx 
					&\stackrel{sub. \, y}{=}&	\int \frac{1}{e^y \cdot y} \cdot e^y \, dy\\
					&=& \int \frac{1}{y} \, dy\\
					&=& \log y \, dy\\
					&\stackrel{resub. \, y}{=}& \log (\log x).
		\end{array}$$
		Wir testen das Ergebnis nocheinmal, indem wir ableiten.
		$$\begin{array}{rcl}
			\frac{d}{dx} \log \log x &=& (\frac{d}{dx} \log x) \frac{1}{\log x}\\
				&=& \frac{1}{x} \cdot \frac{1}{\log x}\\
				&=& \frac{1}{x \cdot \log x}
		\end{array}$$
\pagebreak

	\item $\int \frac{1+e^x}{1-e^x} \, dx$:\\

		$$\begin{array}{rcl}
            \int \frac{1+e^x}{1-e^x} \, dx
                &=& \int \frac{1}{1-e^x} \, dx + \int \frac{e^x}{1-e^x} \, dx\\
            &&\text{Wir verfahren mit den einzelintegralen fort}\\
            \int \frac{1}{1-e^x} \, dx &\stackrel{y=e^x}{=}&
                \int \frac{1}{(1-y)y} \, dy\\
                &=& \int \frac{1}{y} - \frac{1}{1-y} \, dy\\
                &=& \int \frac{1}{y} \, ds - \frac{1}{1-y} \, dy\\
                &=& \ln y - \ln (1-y)\\
                &\stackrel{resub.}{=}& x - \ln (1-e^x)\\
           \int \frac{e^x}{1-e^x} \, dx &\stackrel{z=1-e^x}{=}&
                \int \frac{e^x}{z} \cdot -e^x \, dz\\
                &=& -\frac{1}{z} \, dz\\
                &=& \ln (1- e^x)\\
                &\stackrel{resub.}{=} x\\
            \int \frac{1+e^x}{1-e^x} \, dx &=& x - 2\ln (1-e^x)
        \end{array}$$

        Einmal Ableiten zum gegenchecken:\\
        $$\begin{array}{rcl}
            \frac{d}{dx} x-2\ln(1-e^x)
                &=& 1 - 2 \frac{1}{1-e^x} \cdot -e^x\\
                &=& 1 + 2 \frac{e^x}{1-e^x}\\
                &=& \frac{1-e^x +2e^x}{1-e^x}\\
                &=& \frac{1+e^x}{1-e^x}
        \end{array}$$


	\item $\int \sqrt{1-x^2} \, dx$:\\

		$$\begin{array}{rcl}
            \int \sqrt{1-x^2} \, dx &\stackrel{x=\sin y}{=}&
                \int \sqrt{1- \sin^2 y} \cdot \cos y \, dy\\
            &\stackrel{Geo. Pyth}{=}& \int \sqrt{\cos^2 y} \cdot \cos y \, dy\\
            &=& \int \cos^2 y \, dy\\
            &\stackrel{Tip.}{=}& \int \frac{\cos (2 y) + 1}{2} \, dy\\
            &=& \frac{1}{2} \int \cos (2y) \, dy + \int \frac{1}{2} \, dy\\
            &\stackrel{z=2y}{=}&
                \frac{1}{2} \int \frac{1}{2} \cos z + \int \frac{1}{2} \, dz\\
            &=& \frac{1}{4} \int \cos z + \int \frac{1}{2} \, dy\\
            &=& \frac{\sin z}{4} + \frac{y}{2}\\
            &\stackrel{resub. z}{=}&
                \frac{\sin 2y}{4} + \frac{y}{2}\\
            &=&
                \frac{1}{2} \sin y \cdot \cos y + \frac{y}{2}\\
            &\stackrel{y=\arcsin x}{=}&
                \frac{1}{2} \cdot \left( x \cdot \cos \arcsin x 
                    + \arcsin x\right)\\
            &=& \frac{1}{2} \left( \sqrt{1-x^2}x + \arcsin{x} \right)
        \end{array}$$

        Es bleibt der Tip zu zeigen:\\
        $$\begin{array}{rcl}
            \frac{1+\cos (2x)}{2} &=& \frac{1+\cos (x+x)}{2}\\
                &\stackrel{Add.}{=}& \frac{1+\cos^2 (x) - \sin^2 (x)}{2}\\
                &\stackrel{Geo.Pyt}{=}& \frac{2 \cos^2 (x)}{2}\\
                &=& \cos^2 (x)
        \end{array}$$
\end{enumerate}

%% ------------------------------------------------------
%%                     Aufgabe 6
%% ------------------------------------------------------

\pagebreak

\subsection*{Aufgabe 6: \mdseries\itshape Uneigentliche Integrale I}

Für eine Funktion $f \; . \; (0,b) \rightarrow \mathbb{R}$ definiert man das \emph{uneigentliche Integral} durch
$$
\int_0^b f(x)dx = \lim_{\varepsilon \rightarrow 0^+} \int _\varepsilon^b f(x)dx,
$$
falls dieser Grenzwert existiert.
\begin{enumerate}[(i)]
	\item Bestimmen Sie
		$$
			\int_0^1 \frac{1}{\sqrt{x}} dx.
		$$

		\textbf{Lösung:}\\
		Wir integrieren zunächst und untersuchen danach das Verhalten im Grenzwert:
		$$\begin{array}{rcl}
			\int_0^1 \frac{1}{\sqrt{x}} \, dx &=& \underset{\varepsilon \rightarrow 0^+}{\lim} \int_\varepsilon^1 x^{-\frac{1}{2}} \, dx\\
				&=&\underset{\varepsilon \rightarrow 0^+}{\lim} \left[ 2 \cdot x^{\frac{1}{2}} \right]_\varepsilon^1\\
				&=& \underset{\varepsilon \rightarrow 0^+}{\lim} (2\sqrt{1} - 2\sqrt{\varepsilon})\\
				&=& 2 \sqrt{1} - \underset{\varepsilon \rightarrow 0^+}{\lim} 2 \sqrt{\varepsilon}\\
				&=& 2 \sqrt{1} - 0\\
				&=& 2
		\end{array}$$

		Der Grenzwert existiert und $\int_0^1 \frac{1}{\sqrt{x}} \, dx= 2$ gilt.
	\item Für welche $p \in \mathbb{R}$ existiert
		$$
			\int_0^1 \frac{1}{x^p}dx \; ?
		$$

		\textbf{Lösung:}\\
		Wir formen das ganze wie gehabt um und untersuchen, wie sich der Grenzwert verhätl.:
		$$\begin{array}{rcl}
			\int_0^1 \frac{1}{x^p} \, dx &=& \underset{\varepsilon \rightarrow 0^+}{\lim} \int_\varepsilon^1 x^{-p} \, dx\\
				&\stackrel{p\not=1}{=}& \underset{\varepsilon \rightarrow 0^+}{\lim} \left[ \frac{1}{1-p} x^{1-p}\right]_\varepsilon^1\\
				&=& \frac{1}{1-p}\cdot 1^{1-p} - \underset{\varepsilon \rightarrow 0^+}{\lim} \frac{1}{1-p} \cdot \varepsilon^{1-p} \\
                &=& \frac{1}{1-p} - \frac{1}{1-p} \cdot \underset{\varepsilon \rightarrow 0^+}{\lim} \varepsilon^{1-p} 
		\end{array}$$
		Nun können wissen wir, dass für $q>0$ gilt, dass $\underset{\varepsilon \rightarrow 0^+}{\lim} \varepsilon^q = 0$.
		Daraus können wir zum einen schließen, dass für $p<1$ der Grenzwert definiert ist und $\frac{1}{1-p}$ ist.\\
		Für $p>1$ divergiert es bestimmt gegen unendlich.\\
		Für den vorhin ausgenommen Fall $p=1$ gilt $\int \frac{1}{x} \, dx = \log |x|$ und diese Funktion ist für $x \rightarrow 0$ nicht definiert.\\
		Es ist also nur für $p<1$ uneigentlich integrierbar.
\end{enumerate}

%% ------------------------------------------------------
%%                     Aufgabe 7
%% ------------------------------------------------------

\subsection*{Aufgabe 7: \mdseries\itshape Uneigentliche Integrale II}

Man definiert das \emph{uneigentliche Integral} einer Funktion $f \; . \; [a,\infty) \rightarrow \mathbb{R}$ als
$$
\int_a^\infty f(x)dx = \lim_{N\rightarrow \infty} \int_a^N f(x)dx,
$$
falls dieser Grenzwert existiert.
\begin{enumerate}[(i)]
	\item Bestimmen Sei
		$$
			\int_1^\infty \frac{1}{x^4} dx
		$$
	
	\textbf{Lösung:}\\
	
	$$\begin{array}{rcl}
		\int_1^\infty \frac{1}{x^4} dx &=& \underset{{N\rightarrow \infty}}{\lim} \int_1^N \frac{1}{x^4} \, dx\\
			&=& \underset{{N\rightarrow \infty}}{\lim} \left[ \frac{1}{-3} \cdot x^{-3} \right]_1^N\\
			&=& \underset{{N\rightarrow \infty}}{\lim} - \frac{1}{3} \cdot N^{-3} - (-\frac{1}{3} \cdot 1^{-3})\\
			&=& 0 - (-\frac{1}{3})  \\
            &=& \frac{1}{3}
	\end{array}$$

	Der Grenzwert von $\frac{1}{x^p}, p>1$ ist Null, wie in Ana I bewiesen. Damit ist der Grenzwert definiert und $\int_1^\infty \frac{1}{x^4} dx = \frac{1}{3}$

	\item Für welche $p\in\mathbb{R}$ existiert
		$$
			\int_1^\infty \frac{1}{x^p}dx \; ?
		$$

	\textbf{Lösung:}\\
		$$\begin{array}{rcl}
			\int_1^\infty \frac{1}{x^p}dx &=& \underset{{N\rightarrow \infty}}{\lim} \int_1^N \frac{1}{x^p}dx\\
				&\stackrel{p\not=1}{=}& \underset{{N\rightarrow \infty}}{\lim} \left[ \frac{1}{1-p} x^{1-p}\right]_1^N\\
				&=& \frac{1}{1-p} 1^{1-p} -\underset{{N\rightarrow \infty}}{\lim} \frac{1}{1-p} N^{1-p}\\
                &=& \frac{1}{1-p} - \frac{1}{1-p} \cdot \underset{{N\rightarrow \infty}}{\lim} N^{1-p}
		\end{array}$$
		Nach Ana I wissen wir nun, dass $\underset{{N\rightarrow \infty}}{\lim} N^q, \; q < 0$ gegen Null strebt. Damit existiert der Grenzwert für $p > 1$ und ist $\frac{1}{(1-p)}$. 

%%Für $p=1$ gilt wieder, dass das unbestimmte Integral $\log |x|$ ist und divergiert bestimmt gegen unendlich. Dies gilt  
%%insbesondere für alle Werte $q>0$, da sowohl Wurzeln als auch Polynome divergieren.\\
		Also ist das Integral nur für $p > 1$ definiert.
\end{enumerate}

%% ------------------------------------------------------
%%                     Aufgabe 8
%% ------------------------------------------------------

\subsection*{Aufgabe 8: \mdseries\itshape Uneigentliche Integrale III}

Die Gamma-Funktion ist wie folgt definiert:
$$
	\Gamma (x) = \int_0^\infty e^{-t}t^{x-1} \, dt.
$$

\begin{enumerate}[(i)]
	\item Zeigen Sie die folgende Version der partiellen Integration:
		$$
			\int_a^\infty u'(x)v(x) \; dx = \left[ u(x)v(x) \right]_a^\infty - \int_a^\infty u(x)v'(x) \; dx,
		$$
		wobei mit dem ersten Ausdruck auf der rechten Seite der Grenzwert \\$\underset{x \rightarrow \infty}{\lim} u(x)v(x) - u(a)v(a)$ gemeint ist. 
		Weiter sei vorausgesetzt, dass all diese Grenzwerte existieren.\\
		
		\textbf{Lösung:}\\
		
		
		$$\begin{array}{rcl}
			\int_a^\infty u'(x)v(x) \; dx &\stackrel{Def.}{=} \underset{N \rightarrow \infty}{\lim} \int_a^N u'(x)v(x) \, dx\\
				&\stackrel{Parts.}{=}& \underset{N \rightarrow \infty}{\lim} \left( \left[ u(x)v(x) \right]_a^N  \int_a^N u(x)v'(x) \, dx \right)\\
				&\stackrel{Def.}{=}& \left[ u(x)v(x) \right]_a^\infty - \int_a^\infty u(x)v'(x) \, dx
		\end{array}$$

		Das integral ist nun genau dann definiert, wenn die Grenzwerte der einzel Ausdrücke definiert sind. Dies ist aber nach Aufgabe vorrausgesetzt. 
		Damit gilt die Formel.
	
	\item Zeigen Sei, dass für alle $x > 0$ das uneigentliche Integral $\Gamma (x)$ wohldefiniert ist.\\
		\textbf{Lösung:}\\
		
		Sei $x>0$. Es ist 
        $$
          \Gamma(x) = \int_{0}^{\infty}{e^{-t} t^{x-1} \; dt} = \lim_{a \to 0^{+}}{\int_{a}^{1}{e^{-t} t^{x-1} \; dt}} + \lim_{b \to \infty}{\int_{1}^{b}{e^{-t} t^{x-1} \; dt}}
        $$
    Es muss also gezeigt werden, dass beide uneigentlichen Integrale  konvergieren.\\

    (i) Erster Term\\
    Wegen $t > 0 \Rightarrow e^{-t} t^{x-1} < t^{x-1}$. Also gilt:
    $$
      \lim_{a \to 0^{+}}{\int_{a}^{1}{e^{-t} t^{x-1} \; dt}}
    < \lim_{a \to 0^{+}}{\int_{a}^{1}{t^{x-1} \; dt}}
    = \frac{1}{x} 1^x -  \lim_{a \to 0^{+}}{\frac{1}{x} a^x}
    = \frac{1}{x} < \infty
    $$
    Also konvergiert $\lim_{a \to 0^{+}}{\int_{a}^{1}{e^{-t} t^{x-1} \; dt}}$.
    
    (ii) Zweiter Term\\
    Da $t \geq 1$ ex. für $a > 0$ ein $c > 0$, s.d. für $0 < x \leq a$ gilt: $t^{x-1} < c \cdot e^{\frac{t}{2}}$. Also gilt:\\
    $$
     \lim_{b \to \infty}{\int_{1}^{b}{e^{-t} t^{x-1} \; dt}} \leq \lim_{b \to \infty}{c \int_{1}^{b}{e^{-\frac{t}{2}} \; dt}} < \infty
    $$
    Also konvergiert $\lim_{b \to \infty}{\int_{1}^{b}{e^{-t} t^{x-1} \; dt}}$.
    
    
	\item Zeigen Sie, dass
		$$
			\Gamma (x+1) = x \cdot \Gamma (x)
		$$
		gilt und daraus folgt, dass
		$$
			\forall n \in \mathbb{N} \; : \;\Gamma (n+1) = n!
		$$
		gilt.\\

		\textbf{Lösung:}\\
        (i) Funktionalgleichung: \\ 

		Sei $x > 0$.
        \begin{eqnarray*}
        x \cdot \Gamma(x) & = & x \cdot \int_{0}^{\infty}{e^{-t} t^{x-1} \; dt} \\
        &\stackrel{a)}{=}& x \cdot \left( \left. \frac{t^x}{x} e^{-t} \right|_{0}^{\infty} 
                           - \int_{0}^{\infty}{\frac{1}{x} t^x -e^{-t} \; dt} \right) \\
        &=& x \cdot \left( \left. \frac{t^x}{x} e^{-t} \right|_{0}^{\infty} 
                           + \frac{1}{x} \int_{0}^{\infty}{e^{-t} t^x \; dt} \right) \\
        &\stackrel{(*)}{=}& x \cdot \left( \frac{1}{x} \int_{0}^{\infty}{e^{-t} t^x \; dt} \right)  \\
        &=& \int_{0}^{\infty}{e^{-t} t^x \; dt} \\
        &=& \Gamma(x+1)
        \end{eqnarray*}

        (*) gilt, da $ \left. \frac{t^x}{x} e^{-t} \right|_{0}^{\infty} = 0 $, denn:
        
        $$
          \left. \frac{t^x}{x} e^{-t} \right|_{0}^{\infty}
        = \lim_{t \to \infty}{\frac{t^x}{x} e^{-t}} - (\frac{0^x}{x} e^{-0}) 
        = \lim_{t \to \infty}{\frac{t^x}{x} \frac{1}{e^t}}
        = \frac{1}{x} \lim_{t \to \infty}{\frac{t^x}{e^t}}
        $$
        Da $x^t = o(e^t)$, also der Nenner echt schneller wächst als der Zähler (vgl. Regel von L'Hopital), gilt
        $$
          \frac{1}{x} \lim_{t \to \infty}{\frac{t^x}{e^t}}
        = \frac{1}{x} \cdot 0 = 0
        $$

        (ii) Beziehung zur Fakuktät: \\
        
        Sei $n \in \mathbb{N}$. Dann folgt aus der Funktionalgleichung:
        $$
          \Gamma(n+1) = n \cdot \Gamma(n) = n \cdot (n-1) \cdot \Gamma(n-1)
        =  n \cdot (n-1) \cdot (n-2) \cdot \ldots \cdot 1 \cdot \Gamma(1)
        $$
        Zusätzlich gilt
        \begin{eqnarray*}
        \Gamma(1) &=& \int_{0}^{\infty}{e^{-t} t^{1-1} \; dt} \\
        &=& \int_{0}^{\infty}{e^{-t} \; dt} \\
        &=& \left. -e^{-t} \right|_{0}^{\infty} \\
        &=& \lim_{t \to \infty}{-e^{-t}} - -e^0 \\
        &=& 0 - (-1) \\
        &=& 1
        \end{eqnarray*}
        Daraus folgt die Behauptung.
\end{enumerate}

\label{LastPage}
\end{document}


