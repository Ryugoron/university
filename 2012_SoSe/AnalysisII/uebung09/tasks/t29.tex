\subsection*{Aufgabe 29: \mdseries\itshape Parametrisierung des Torus}

\begin{enumerate}[(i)]
	\item Welche Art von Kurve $\gamma$ im $\mathbb{R}^3$ wird durch die Abbildung
	$$
		\gamma \; : \; \mathbb{R} \rightarrow \mathbb{R}^3, \; \varphi \mapsto (a + b\cos \varphi,0,b\sin\varphi)
	$$
	für $a>b>0$ beschrieben.?\\

	\textbf{Lösung:}\\
		Die Kurve die hier beschrieben wird ist ein Kreis. Dieser befindet sich in der $x_1,x_3$ Ebene. Der Kreis
		liegt vollständig rechts vom Ursprung $(0,0)$. Der Mittelpunkt des Kreises ist $(a,0,0)$ und der Radius ist
		$b$. Das vollständige Rechts entsteht dadurch, das der Mittelpunkt $a$ vom Mittelpunkt entfernt ist,
		aber der Radius nur $b<a$ beträgt.

	\item Rotieren Sie diese Kurve entgegen dem Uhrzeigersinn um die $x_3$ Achse. Bezeichnen Sie den Rotationswinkel
	mit $\vartheta$. Sei erhalten so eine Funktion $f \; : \; \mathbb{R}^2 \rightarrow \mathbb{R}^3$ mit
	$(\varphi,\vartheta) \mapsto f (\varphi, \vartheta)$.\\

	Die durch $f$ dargestellte Fläche $T$ im $\mathbb{R}^3$ nennt man einen \emph{Torus}.\\

	\textbf{Lösung:}\\
		Nachdem wir unser Vector auf den Kreis abgebildet haben mit dem Winkel $\varphi$ werden wir danach
		noch auf das Ergebnis die Drehmatrix $D_\vartheta$ anwenden, um den Vecotr noch um den Winkel $\vartheta$ um
		die $x_3$ - Achse zu drehen.\\

		$$
			D_\vartheta = \begin{pmatrix}
				\cos \, \vartheta & - \sin \, \vartheta & 0\\
				\sin \, \vartheta & \cos \, \vartheta  & 0\\
				0 & 0 & 1
			\end{pmatrix}
		$$

		Wir wenden für $f$ also erst den Ergebnisvektor auf die Drehung an:
		$$\begin{array}{rcl}
			f(\varphi,\vartheta) = D_\vartheta \circ \gamma(\varphi) &=&
				D_\vartheta = \begin{pmatrix}
				\cos \, \vartheta & - \sin \, \vartheta & 0\\
				\sin \, \vartheta & \cos \, \vartheta & 0\\
				0 & 0 & 1
			\end{pmatrix} \times
			(a+b \cos\, \varphi,0,b \sin \, \varphi)^t\\
			&=& \begin{pmatrix}
				(a + b \cos \, \varphi) \cdot \cos \, \vartheta\\
				(a + b \cos \, \varphi) \cdot \sin \, \vartheta\\
				b \sin \varphi
			\end{pmatrix}
		\end{array}$$

		Dies ist der Ergebnisvektor (den wir an dieser Stelle gedreht dargestellt haben), die Funktion $f$ ist also\\
		$f(\varphi,\vartheta) = \left( (a + b \cos \, \varphi) \cdot \cos \, \vartheta,   (a + b \cos \, \varphi) \cdot \sin \, \vartheta, b \sin \varphi \right)$

	Da die beiden Drehungen linear unabhängig sind (der Kreis vollzieht eine Drehung um die verschobene $x_2$ Achse und der Torus durch
	Rotation um die $x_3$ Achse). Damit ist es egal in welcher Reihenfolge die Drehungen ausgeführt werden.

	\item Skizzieren Sie ...\\
		Nicht bearbeitet.
\end{enumerate}
