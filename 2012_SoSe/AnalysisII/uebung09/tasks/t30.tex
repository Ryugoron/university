\subsection*{Aufgabe 30: \mdseries\itshape Niveauflächen}
Für $x_1 > 0$, sei
$$
	h(x_1,x_3) := (x_1 - a)^2 + x_3^2 -b^2
$$
die beschriebene Funktion eines Kreises in der $[x_1,x_3]$ - Ebene im $\mathbb{R}^3$ mit Radius $b > 0$
und dem Mittelpunkt $(a,0,0)$, wobei $b<a$.

\begin{enumerate}[(i)]
	\item Setzen Sie diese Funktions rotationssymmetrisch um die $x_3$ - Achse auf den ganzen $\mathbb{R}^3$ fort,
		und bezeichen Sie diese neue Funktion mit $g$.\\

	\textbf{Lösung:}\\
		Um die Punktmenge, die wir von $h$ bekommen, benutzen wir einfach um eine zweite Rotation zu vollführen,
		die diesmal um die $x_3$ Achse geschieht und noch genauer um den Ursprung $(0,0,0)$ mit dem Radius
		$a$, damit wir uns auf einem Radius um den Ursprünglichen Mittelpunkt bewegen.\\

		Da wir zu diesem Zeitpunkt aber keine Ahnung hatten, wie wir das genau anstellen, haben wir
		uns einfach die expandierte stadardt Darstelllung der algebraischen Form genommen um in der nächsten
		Aufgabe damit weiter zu rechnen:\\
		$g(x_1,x_2,x_3) = (x_1^2+x_2^2+x_3^2 + a^2 - b^2)^2 - 4a^2(x_1^2 + x_2^2)$\\

		Der erste Teil bezeichnet hier die Berechnung des neuen Mittelpunktes, um den sich unser Kreis drehen
		soll und der zweite Summand bezeichnet der Radius um den sich der Mittelpunkt drehen darf. Hierbei
		darf sich der Mittelpunkt, wie gefordert nur auf der $x_1-x_2$ Ebene bewegen, da wir ja um die $x_3$ Achse drehen.\\

		Was hier nicht gänzlich der Rotation des Kreieses genügt, ist dass wir eigentlich eine Kugel um den rotierten Mittelpunkt
		bestimmen, aber da wir durch die Rotation des Kreises die Kugel übderdecken, ist diese Darstellung korrekt.

	\item Sie $f$ die Funktion aus Aufgabe 29. Zeigen Sie, dass
		$$
			f(\mathbb{R}^2) = \{ x \in \mathbb{R}^3 \; | \; g(x) = 0\}.
		$$
		Dies besagt, dass $T$ die \emph{Niveaufläche der Funktion g für den Wert 0} ist.\\

	\textbf{Beweis:}\\
		Für den Beweis, setzten wir zunächst die Bildpunkt von $f$ in die Funktion $g$ ein, um so zu sehen, dass alle $0$ ergeben.\\
		$$\begin{array}{rcl}
			g(\cos \vartheta (a + b \cos \varphi), \sin \vartheta ( a + b \cos \varphi), b \sin \varphi)\\
				&=& 0
		\end{array}$$
		Sagt uns eine Auflösung mit WolframAlpha.\\

		In der anderen Richtung müssen wir nun zeigen, dass jeder Punkt, der $g(x) = 0$ erfüllt auch im Bild von $f$ liegt.\\
		Wir haben allerdings keine Lust mehr weiter zu machen :)
	
	\item Berechnen Sie den Gradienten $\nabla g(x)$ für alle $x \in \mathbb{R}^3 \setminus\{(0,0,0)\}$. Zeigen Sie
		$$
			\nabla g(x) \not= 0 \; \; \text{ für alle } x \in T
		$$
		sowie
		$$
			\nabla g(x) \cdot \frac{\delta f}{\delta \vartheta} (\varphi, \vartheta) = \nabla g(x) \cdot \frac{\delta f}{\delta \varphi} (\varphi, \vartheta
				) = 0
		$$
		für alle $(\varphi, \vartheta) \in \mathbb{R}^2$ und $x = f(\varphi,\vartheta)$. Dies besagt, dass der Gradient von g senkrecht
		zur Niveaufläche $T$ ist.\\
	\textbf{Lösung:}\\
		Wir berechnen zunächst den Gradienten:\\
		$$\begin{array}{rcl}
			\nabla g (x) &=& (\frac{\delta \, g}{\delta \, x_1} (x), \frac{\delta \, g}{\delta \, x_2} (x), \frac{\delta \, g}{\delta \, x_3} (x))\\
				&=& ( 4x_1(x_1^2+x_2^2+x_3^2+a^2-b^2) -6a^2x_1,   \\
			&& 4x_2(x_1^2+x_2^2+x_3^2+a^2-b^2) -6a^2x_2, 4x_1(x_1^2+x_2^2+x_3^2+a^2-b^2) -6a^2x_1,\\
			&& 4x_3(x_1^2+x_2^2+x_3^2+a^2-b^2))
		\end{array}$$

		Als nächstes Berechnen wir berechnen nun noch die Partiellen Ableitungen $\frac{\delta \, f}{\delta \, \vartheta} (\varphi,\vartheta)$ und
		$\frac{\delta \, f}{\delta \, \varphi} (\varphi,\vartheta)$.\\

		$$\begin{array}{rcl}
			\frac{\delta \, f}{\delta \, \vartheta} (\varphi,\vartheta) &=& (- \sin \vartheta, \cos \vartheta , 0)
		\end{array}$$

		und

		$$\begin{array}{rcl}
			\frac{\delta \, f}{\delta \, \varphi} (\varphi,\vartheta) &=&
				( - b (\cos \vartheta ) \sin \varphi, -b (\sin \, \vartheta) \sin \varphi, b \cos \varphi )
		\end{array}$$

		Nun müssten wir noch in die Gleichungen einsetzen und überprüfen, ob es gleich 0 ist, aber wir haben an dieser Stelle auch keine Lust
		mehr zu rechnen.
\end{enumerate}
