\documentclass[11pt,a4paper,ngerman]{article}
\usepackage[bottom=2.5cm,top=2.5cm]{geometry} 
\usepackage{babel}
\usepackage[utf8]{inputenc} 
\usepackage[T1]{fontenc} 
\usepackage{ae} 
\usepackage{amssymb} 
\usepackage{amsmath} 
\usepackage{graphicx}
\usepackage{fancyhdr}
\usepackage{fancyref}
\usepackage{listings}
\usepackage{xcolor}
\usepackage{paralist}

%%\usepackage[pdftex, bookmarks=false, pdfstartview={FitH}, linkbordercolor=white]{hyperref}
\usepackage{fancyhdr}
\pagestyle{fancy}
\fancyhead[C]{Analysis II}
\fancyhead[L]{Aufgabenblatt 6}
\fancyhead[R]{SoSe 2012}
\fancyfoot{}
\fancyfoot[L]{}
\fancyfoot[C]{\thepage \hspace{1px} of \pageref{LastPage}}
\renewcommand{\footrulewidth}{0.5pt}
\renewcommand{\headrulewidth}{0.5pt}
\setlength{\parindent}{0pt} 
\setlength{\headheight}{15pt}

\date{}
\title{Max Wisniewski , Alexander Steen}
\author{Tutor : Adrian Steffens}

\newcommand{\limes}[2][n]{\underset{ #1 \rightarrow #2}{\lim}}

\begin{document}

\lstset{language=Pascal, basicstyle=\ttfamily\fontsize{10pt}{10pt}\selectfont\upshape, commentstyle=\rmfamily\slshape, keywordstyle=\rmfamily\bfseries, breaklines=true, frame=single, xleftmargin=3mm, xrightmargin=3mm, tabsize=2}

\maketitle
\thispagestyle{fancy}

\subsection*{Aufgabe 16: \mdseries\itshape Die Ungleichungen von Hölder und Young}
Seien $x_1,...,x_n$ und $y_1,...,y_n$ reelle Zahlen.

\begin{enumerate}[(i)]
	\item Zeigen Sie, dass für
		$$
			\frac{1}{p} + \frac{1}{q} = 1, \; p,q>1,
		$$
		und für alle $x,y\geq 0$ die \emph{Youngsche Ungleichung}
		$$
			x^{\frac{1}{p}} y^{\frac{1}{q}} \leq \frac{1}{p} x + \frac{1}{q} y
		$$
		richtig ist.\\
	\textbf{Beweis:}\\
		tbd

	\item Zeigen Sie nun unter Verwendung der Youngschen Ungleichung die Höldersche Ungliechung
		$$
			\left| \overset{n}{\underset{i=1}{\sum}} x_i y_i \right| \leq \left( \overset{n}{\underset{i=1}{\sum}} |x_i|^p \right)^{\frac{1}{p}}
				\left( \overset{n}{\underset{i=1}{\sum}} | y_i |^q \right) ^{\frac{1}{q}}.
		$$
		Für welche $p$ und $q$ gewinnen Sie die Cauchy-Schwarz-Ungleichung?
	\textbf{Lösung:}\\
		tbd
\end{enumerate}

\subsection*{Aufgabe 17: \mdseries\itshape Durchschnitt und Vereinigung von Mengen}
	Sie $M \subset \mathbb{R}^n$. Zeigen Sie
	\begin{enumerate}[(i)]
		\item $\overline{M} = \bigcap \{ A \subset \mathbb{R}^n \; | \; A \text{ abgeschlossen, } M \subset A \}$\\
		\textbf{Beweis:}\\
            Sei $T = \bigcap \{ A \subset \mathbb{R}^n \; | \; A \text{ abgeschlossen, } M \subset A \}$\\
		    Wir müssen die beiden Fälle $\overline{M} \subset T$ und $T \subset \overline{M}$ zeigen.\\

            Fangen wir mit $T \subset \overline{M}$ an.\\
            Wir wissen, dass $A \cap B \subset A$ für alle Mengen $A$ und $B$ gilt. 
            Da nun $M \subset \overline{M}$ gilt, ist $\overline{M}$ eine abgeschlossene Menge, die
            wir in $T$ vereinigt haben. Also ist $T \subset \overline{M}$.\\

            Als nächstes zeigen wir $\overline{M} \subset T$.\\
            Nehmen wir an, dass es nicht so wäre, dann gebe es mindestens einen Punkt $x \in \overline{M}$,
            mit $x \not \in T$. Da alle Mengen, die in $T$ geschnitten werden $M$ enthalten, wissen wir,
            dass $x\in M$ liegen muss.\\
            Daraus ergibt sich, dass $x \in \delta M$ liegen muss.\\

            \textbf{
            (TODO FORMULIERUNG: Weil der Rand in jeder epsilon Umgebung in der Menge liegen muss,
            kann es keine kleinere abgeschlossene Menge als $\overline{M}$ geben, da sonst
            nicht jeder Punkt von $M$ in dieser Menge liegen kann)
            }\\
            \mbox{} \hfill $\square$

		\item $\overset{\circ}{M} = \bigcup \{ \Omega \subset \mathbb{R}^n \; | \; \Omega \text{ offen, } \Omega \subset M \}$\\
		\textbf{Beweis:}\\
		    Sei $T = \bigcup \{ \Omega \subset \mathbb{R}^n \; | 
                \; \Omega \text{ offen, } \Omega \subset M \}$.\\
            Wir müssen erneut zeigen, dass $\overset{\circ}{M} \subset T$ und 
            $T \subset \overset{\circ}{M}$ gilt. Fangen wir mit
            $\overset{\circ}{M} \subset T$ an.\\
            
            Nach der Definition von inneren Punkten, liegt jeder dieser Punkte auch in $M$.
            Da $\overset{\circ}{M}$ eine offene Menge ist, wird diese mit in der Vereinigung von
            allen offenen Mengen sein. Daher gilt auch $\overset{\circ}{M} \subset T$.\\

            Als nächstes zeigen wir $T \subset \overset{\circ}{M}$.\\
            Wir wissen, dass die Vereinigung in diesem Fall eine offene Menge sein muss. Für jedes
            $x \in T$ muss also ein $\varepsilon > 0$ exsistieren, so dass $B_\varepsilon(x) \subset T$
            liegen muss. Dies bedeutet aber, da alle Mengen, die wir vereinigt haben, die Bedingung
            $\Omega \subset M$ erfüllen. Daher muss insbesondere $B_\varepsilon(x) \subset M$ für jedes
            $x \in T$ erfüllt sein. Dies ist nun allerdings genau die Definition von $\overset{\circ}{M}$.\\
            \mbox{}\hfill $\square$
            
	\end{enumerate}

\subsection*{Aufgabe 18: \mdseries\itshape Durchmesser und Abstand von Mengen}
	Sei $M \subset \mathbb{R}^n$. Man definiert
	$$
		diam \; M := \sup \{ |x-y| \; | \; x,y \in M \}
	$$
	Für $x \in M$ und $M \subset \mathbb{R}^n$ definiert man den Abstand von x zu M durch
	$$
		dist (x,M) := \inf \{|x-y| \; | \; y \in M \}.
	$$
	\begin{enumerate}[(i)]
		\item Zeigen Sie, dass $diam \; M = diam \overline{M}$.\\
		\textbf{Beweis:}\\
			Wir unterscheiden zunächst 3 Fälle.\\
            Zunächst kann der Punkt, zum einen können die entferntesten Punkte
            beide auf dem Rand liegen, dann könnte einer auf dem Rand und der andere
            im inneren liegen und zuletzt können beide im innere liegen.\\

            Liegen beide entferntesten Punkte auf dem Rand $\delta M$, so liegen
            sie insbesondere im Abschluss auch auf dem Rand $\delta M$. Damit ist
            die Gleichheit erfüllt.

            Liegt einer der Punkte nicht auf dem Rand, so können wir uns in der Umgebung
            einen beliebigen Punkt inneren Suchen. Wir wissen, dass um einen inneren Punkt
            $x \in \overset{\circ}{M}$ eine offene Kugel gibt, mit $\varepsilon > 0$ und
            $B_\varepsilon ( x ) \subset M$. Wir können nun parallel zum anderen Punkt
            unseren Punkt nach außen schieben. Damit wird der Abstand um $\varepsilon > 0$
            größer. Nun wissen wir, dass durch Grenzwertbildung ein Punkt auf dem Rand
            erreicht wird (definition der Vorlesung). Damit konvergiert auch
            der Durchmesser des Objektes gegen den Durchmesser des Abschlusses.\\

            Das gleiche gilt auch, wenn beide Punkte im inneren liegen. Hier müssen
            nur beide Punkte im limes nach außen geschoben werden und erreichen so
            einen Punkt auf dem Rand. Der Durchmesser wird im Grenzwert dann auch gleich sein.\\

            Da der Durchmesser über das supremum definiert ist, ist der Grenzwert im
            falle der offenen Kugel der genommene Wert und damit gleich dem Abstand auf dem Rand.\\
            \mbox{}\hfill $\square$

		\item Beweisen Sie die folgenden Äquivalenzen
			\begin{enumerate}[a)]
				\item $x \in \overline{M} \Longleftrightarrow dist(x,M) = 0$\\
				\textbf{Beweis:}\\
				    $\Rightarrow$:\\
                        Sei $x \in \overline{M}$. Dann ist
                        $|x - x| = 0$, da $x$ insbesondere in $M$ lag.\\
                        Nach Dreiecksungleichung kann kein Wert kleiner als
                        0 erreicht werden, somit ist das Infimum auch gleich 0.\\

                    $\Leftarrow$:\\
                        $\exists y \in M \; : \; |x - y | = 0$. Da wir uns 
                        durch den Betrag in einem Metrischen Raum befinden müssen,
                        gilt in diesem per Definiton $|x - y| = 0 \Leftrightarrow x = y$.
                        Deshalb muss $x \in M$ gelten.\\
                    \mbox{} \hfill $\square$	

				\item $x \in \overset{\circ}{M} \Longleftrightarrow dist(x,M^c) > 0$\\
				\textbf{Beweis:}\\
				    $\Rightarrow$:\\
                        Da $x \in \overset{\circ}{M}$ gilt, muss nach Definition eine
                        muss nach Definition eine offene Kugel existieren mit $\varepsilon > 0$
                        $B_\varepsilon(x) \subset M$. Wir wissen nun also, dass
                        $\forall y \in M^c \; : \; |x - y | \geq \varepsilon > 0$, da in
                        der unmittelbaren Umgebung um $x$ nur Punkte aus $M$ liegen können.\\

                    $\Leftarrow$:\\
                        Nach Vorraussetzung existiert ein $\varepsilon > 0$, mit
                        $dist(x,M^c) \geq \varepsilon > 0$. Nun folgt daraus, dass
                        $B_\varepsilon(x) \subset M$ gilt. Wäre dies nicht so,
                        würde ein Punkt in der Kugel existieren, der im nicht in der Menge
                        liegt. Dies würde aber bedeuten, dass der Abstand von $x$ zu einem
                        Punkt aus dem Komplement kleiner als $\varepsilon$ sein muss.
                        Damit ist das Infimum von allen kleiner als $\varepsilon$ und damit
                        auch $dist(x,M^c)$. Dies ist nun allerdings ein Widerspruch zur Annahme.\\
                    \mbox{} \hfill $\square$

				\item $x$ Randpunkt von $M$ $\Longleftrightarrow dist(x,M) = dist(x,M^c) = 0$\\
				textbf{Beweis:}\\
					$\Rightarrow$:\\
                        Sei $x$ Randpunkt von $M$.\\
                        Das bedeutet, zum einen, dass $\forall \varepsilon > 0 \; : \; B_\varepsilon(x)
                        \cap M \not= \empty$ ist, aber $\forall \varepsilon > 0 \; : \; \neg (B_\varepsilon(x)
                        \subset M )$.\\
                        Die zweite Aussage bedeutet insbesondere, dass $\forall \varepsilon > 0 \; : \;
                        B_\varepsilon(x) \cap M^c$.\\
                        Nun gilt, dass $dist (x,M) < \varepsilon$ für alle $\varepsilon > 0$, da in
                        den Kugeln Punkte liegen. Nun ist $\inf \{ \varepsilon \; | \; \varepsilon > 0 \} = 0$
                        . Und dieses Spiel gilt im Umkehrschluss auch für den Abstand zum komlement.\\

                    $\Leftarrow$:\\ 
                        Wir wissen, dass in jeder $\varepsilon > 0$ Umgebung um $x$ ein Punkt $y \in M$ und
                        ein Punkt $z \in M^c$ existieren muss, mit $y \in B_\varepsilon(x)$ und
                        $z \in B_\varepsilon(x)$. Das Infimum ist, wie oben gezeigt, genau 0.\\
                        Nun sind aber 2 solche Punkte in $M$ und $M^c$ in jeder offenen Kugel vorhanden,
                        daher kann der Schnitt nie leer sein (zu $M$ und $M^c$).\\
                        Dies ist genau die Definition eines Randpunktes.\\
                        \mbox{} \hfill $\square$
			\end{enumerate}
		
		\item Für $\varepsilon > 0$ definiert die $\varepsilon$-Umgebung einer Menge M durch
			$$
				M_\varepsilon := \{ x \in \mathbb{R}^N \; | \; dist(x,M) < \varepsilon \}.
			$$
			Zeigen Sie, dass für alle $\varepsilon > 0$ die Menge $M_\varepsilon$ offen ist, und bestimmen Sie
			$$
				\underset{\varepsilon > 0}{\bigcap} M_\varepsilon
			$$
			\textbf{Lösung:}\\
			    Sei $\varepsilon > 0$ beliebig aber fest.\\
                Sei nun $x_0 \in M_\varepsilon$ ein beliebiger Punkt. Nach Definition
                von $M_\varepsilon$ ist $dist(x_0,M) = d_0 < \varepsilon$.
                Wir wählen uns nun einen Radius um diesen Punkt, um so zu zeigen, dass
                die offene Kugel mit diesem Radius in $M_\varepsilon$ liegt.\\
                Wir können zunächst beobachten, dass wenn $x \in M$ gilt, dass der
                Radius dann trivialerweise $\varepsilon$ sein kann, da alle Punkte
                in diesem Radius in $M_\varepsilon$ liegen. Wir betrachten also nur
                Punkte mit $dist(x_0,M) > 0$. Bei der Konstrution des Radius müssen wir nun
                zum einen Beachten, dass wir nicht über den $\varepsilon$ Abstand zu Menge lappen,
                aber auch nicht zur Seite herrausfallen, wenn wir uns zu nah in einer Spitze befinden.\\
                
                Wir wählen also den Radius $r_0 = \min \{\varepsilon - d_0, \varepsilon^2-d_0^2 \}$.
                Der erste ist der Abstand in Verlängerung das Abstandes zum Punkt in der Menge,
                von dem wir $d_0$ entfernt sind, das andere ist mit dem Satz des Pythagoras umgeformt
                und gibt den Abstand senkrecht zur Verlängerung an.\\
                Nun wissen wir, dass alle Punkte im Kreis $B_{r_0}(x_0)$ vom Ursprünglichen Punkt
                $x$ zu dem $|x - x_0 | < \varepsilon$ war auch weniger als $\varepsilon$ entfernt sind.
                Daher muss $B_{r_0}(x_0) \subset M_\varepsilon$ sein.\\
                \mbox{} \hfill $\square$\\
            
                Nun bestimmen wir $T = \underset{\varepsilon > 0}{\bigcap} M_\varepsilon$.\\
                Wir zeigen, dass $\overline{M} = T$ gilt.\\
                
                $\subset$:\\
                Wir können als erstes leicht sehen, dass $M \subset T$ gilt, da wir schon
                bewiesen haben, dass $dist(x,M) = 0 \Leftrightarrow  x \in M$ gilt.
                Also liegt $x$ insbesondere für jedes $\varepsilon$ in $M_\varepsilon$.\\

                Als nächstes müssen wir nur noch zeigen, dass der Rand $\delta M \subset T$
                enthalten ist.\\
                Für den Rand gilt nun die Definition $x \in \delta M \Leftrightarrow \forall \varepsilon
                > 0 \; : \; B_\varepsilon (x) \cap M \not = \emptyset$.\\
                Wir wissen nun aber, dass nach dem Schnitt für alle $\varepsilon$ $\exists y \in M \; : \;
                | x - y | < \varepsilon$ existieren muss. Dann muss allerdings nach dem Schnitt
                genau so ein $y$ existieren.\\

                $\supset$:\\
                Sei $x \in T$.
                Wir machen prinzipiell die selbe Unterscheidung, wie eben. Entweder ist
                $x \in M$ oder $x \in \delta M$.\\
            
                Wir eben schon erwähnt, gilt für jedes $\varepsilon > 0$ $M \subset M_\varepsilon$.
                Also auch insbesondere $M \in \underset{\varepsilon > 0}{\bigcap} M_\varepsilon$.\\

                Nehmen wir nun einmal an, dass $x \not \in \delta M$ liegt. Das bedeutet, dass es ein
                $\varepsilon ' > 0$ gibt, so dass $B_{\varepsilon '}(x) \cap M = \emptyset$. Dies
                bedeutet aber insbesondere, dass $x \not\in M_\varepsilon$ liegen, kann da es keinen
                Punkt in $M$ gibt, der einen Abstand hat, der Nah genug an $x$ liegt.\\
                Daraus folgt, dass im Schnitt von allen $M_\varepsilon$ der Punt nicht drin liegen kann.\\

                \mbox{} \hfill $\square$
	\end{enumerate}

\subsection*{Aufgabe 19: \mdseries\itshape Umfang von Mengen}
	\begin{enumerate}[(i)]
		\item Man zeige, dass es für jede beschränkte Menge $M \subset \mathbb{R}^n$, die aus mindestens zwei Punkten besteht,
		genau eine Kugel $K = \overline{B_R(a)}$ mit kleinstmöglichem Radius $R > 0$ gibt, die $M$ enthält. Man nennt diese Kugel 
		$K$ die Umkugel von M und den Radius R den Umkugelradius von M.\\
		\textbf{Beweis:}\\
			Sei $M$ eine beliebige Menge und $d = diam \, M$ ihr Durchmesser. Seien
            $x, y \in \mathbb{R}^n$ zwei Punkte, so dass $x,y$ Grenzwerte von zwei Folgen
            in der Menge $M$ sind und $|x - y| = d$. Wir wählen nun $x + (x-y)/2 = a$, den
            Punkt der genau in der Mitte der Menge liegt.\\

            Nach Definition von $diam \, M$ gibt es keinen größeren Abstand in der Menge
            $M$, die Kugel $B_d(a)$ hat nun genau diesen Durchmesser, beinhaltet also
            zunächst gesammt $\overset{\circ}{M}$. Das einzge, was nun noch nicht in
            der Kugel existieren muss, sind Punkte $x,y\in M$ mit $|x -y | = d$,
            also genau Punkte auf dem Rand. Nun wissen wir aber, dass diese
            Punkte im Abschluss der Kugel liegen müssen, da der Abstand der Kugel
            im Grenzwert gegen $d$ konvergiert, also für beide Punkte eine
            Folge von Punkten in der offenen Kugel existiert, so dass der Grenzwert
            $x$ bz. $y$ ist.

            Demnach ist $\overline{B_d(a)}$ eine Kreis, der gesammt $M$ enthält.\\

            Als nächstes können wir nun sagen, dass dies die kleinste Kugel ist,
            da wir, falls wir eine kleinere Kugel wählen, 2 Punkte $a,b$ exsitieren
            können, mit $|x - y| = d$. Da die Kugel aber nun einen kleineren Radius als
            $d$ hat, wird entweder $x$ oder $y$ nicht in der Kugel liegen können.\\

            \mbox{} \hfill $\square$

		\item Sie $M \subset \mathbb{R}^n$ symmetrisch um den Ursprung, dass heißt
			$$
				x \in M \Longleftrightarrow -x \in M.
			$$
			Zeigen Sie, dass $M \subset \overline{B_{(diam \, M )/ 2}(0)}$
		\textbf{Beweis:}\\
			Sei $d = diam \, M$ der Durchmesser von $M$. Nehmen wir nun an, dass
            der Durchmesser zwischen $x,y \in M$ mit $y \not= -x$ und $|x - (-x)| < d$
            besteht. Wir wissen
            dann aber, dass $2 |x| \geq |x - y|^2 - |x|$ gilt, nach Dreiecksungleichung.\\
        
            Dies ist ein Widerspruch, daher muss der Durchmesser zwischen 2 Punkten bestehen,
            die symmetrisch zum Ursprung sind.\\

            Nach Aufgabenteil (i), können wir nun $a= x + (x - (-x))/2 = 0$ als Mttelpunkt
            wählen mit Radius $d = diam \, M / 2$ und dieser Umkreis enthält
            ebenfalls nach (i) die gesammte Menge.\\
            \mbox{} \hfill $\square$
            

		\item Man zeige, dass zwischen dem UmkugelRadius $R$ und dem Durchmesser $\delta := diam \, M$ einer beschränkten Menge
		 $M \subset \mathbb{R}^2$ mit mindestens zwei Elementen die Beziehung
			$$
				R \leq \frac{\delta}{\sqrt{3}}
			$$
		besteht. Geben Sie ein Beispiel für eine dreipunktige Menge $M$, für die Gleichheit richtig ist an.\\
		textbf{Lösung:}\\
			
	\end{enumerate}

\label{LastPage}
\end{document}


