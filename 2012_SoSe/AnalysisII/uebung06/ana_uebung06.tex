\documentclass[11pt,a4paper,ngerman]{article}
\usepackage[bottom=2.5cm,top=2.5cm]{geometry} 
\usepackage{babel}
\usepackage[utf8]{inputenc} 
\usepackage[T1]{fontenc} 
\usepackage{ae} 
\usepackage{amssymb} 
\usepackage{amsmath} 
\usepackage{graphicx}
\usepackage{fancyhdr}
\usepackage{fancyref}
\usepackage{listings}
\usepackage{xcolor}
\usepackage{paralist}

%%\usepackage[pdftex, bookmarks=false, pdfstartview={FitH}, linkbordercolor=white]{hyperref}
\usepackage{fancyhdr}
\pagestyle{fancy}
\fancyhead[C]{Analysis II}
\fancyhead[L]{Aufgabenblatt 6}
\fancyhead[R]{SoSe 2012}
\fancyfoot{}
\fancyfoot[L]{}
\fancyfoot[C]{\thepage \hspace{1px} of \pageref{LastPage}}
\renewcommand{\footrulewidth}{0.5pt}
\renewcommand{\headrulewidth}{0.5pt}
\setlength{\parindent}{0pt} 
\setlength{\headheight}{15pt}

\date{}
\title{Max Wisniewski , Alexander Steen}
\author{Tutor : Adrian Steffens}

\newcommand{\limes}[2][n]{\underset{ #1 \rightarrow #2}{\lim}}

\begin{document}

\lstset{language=Pascal, basicstyle=\ttfamily\fontsize{10pt}{10pt}\selectfont\upshape, commentstyle=\rmfamily\slshape, keywordstyle=\rmfamily\bfseries, breaklines=true, frame=single, xleftmargin=3mm, xrightmargin=3mm, tabsize=2}

\maketitle
\thispagestyle{fancy}

\subsection*{Aufgabe 16: \mdseries\itshape Die Ungleichungen von Hölder und Young}
Seien $x_1,...,x_n$ und $y_1,...,y_n$ reelle Zahlen.

\begin{enumerate}[(i)]
	\item Zeigen Sie, dass für
		$$
			\frac{1}{p} + \frac{1}{q} = 1, \; p,q>1,
		$$
		und für alle $x,y\geq 0$ die \emph{Youngsche Ungleichung}
		$$
			x^{\frac{1}{p}} y^{\frac{1}{q}} \leq \frac{1}{p} x + \frac{1}{q} y
		$$
		richtig ist.\\
	\textbf{Beweis:}\\
		Wir formen, wie in der Aufgabe vorgeschlagen ersteinmal die Gleichung um ($x \geq 0$):
        $$\begin{array}{crcl}
            & x^{\frac{1}{p}} y^{\frac{1}{q}} &\leq& \frac{1}{p} x + \frac{1}{q} y\\
            \Leftrightarrow & \frac{x^{\frac{1}{p}}}{x} y^{\frac{1}{p}} & \leq & \frac{1}{p} + \frac{1}{q} \frac{y}{z}\\
            \Leftrightarrow & \frac{1}{x^{\frac{1}{q}}} & \leq & \frac{1}{p} + \frac{1}{q} \frac{1}{q} \frac{y}{z}\\
            \Leftrightarrow & 0 \leq \frac{1}{p} + \frac{1}{q} \frac{y}{x} - \frac{y}{x}^{\frac{1}{q}}\\
            \Leftrightarrow & 0 \leq \frac{1}{p} + \frac{1}{q} z - z^{\frac{1}{q}}
        \end{array}$$

        Nun zeigen wir, dass diese Funktion immer größer als 0 sein muss und zeigen damit die ursprüngliche
        Ungleichung (für $x>0$).
        Wir sehen, dass die Funktion in $z=0$ den Funktionwert $\frac{1}{q} > 0$ annimmt.\\
        Betrachten wir die Ableitung dieser Funktion $f'(z) = \frac{1}{q} \left( 1 - z^{\frac{1}{q} -1 } \right)$, dann sehen wir, dass diese Funktion auf $(0,1)$ negativ ist, dass heißt der Funktionswert fällt.\\
        Bei $z=1$ ist die Ableitung $0$ und auf $(1,\infty)$ ist der Wert der Ableitung positiv.\\
        
        Dies bedeutet für unsere Kurven diskusion, dass die stetige Funktion auf dem Intervall $(0,1)$ fällt
        und danach wieder steigt. Da die Funktion $f(1) = 0$ ist, sehen wir, dass $f(z) \geq 0$ gelten muss.\\

        Nun können wir noch den Fall $x=0$ betrachten. Hier fällt die urprüngliche Gleichung auf
        $0 \leq \frac{1}{q} y$ zusammen, die mit $q>1, y\geq 0$ immer erfüllt ist.\\

        \mbox{} \hfill $\square$

	\item Zeigen Sie nun unter Verwendung der Youngschen Ungleichung die Höldersche Ungliechung
		$$
			\left| \overset{n}{\underset{i=1}{\sum}} x_i y_i \right| \leq \left( \overset{n}{\underset{i=1}{\sum}} |x_i|^p \right)^{\frac{1}{p}}
				\left( \overset{n}{\underset{i=1}{\sum}} | y_i |^q \right) ^{\frac{1}{q}}.
		$$
		Für welche $p$ und $q$ gewinnen Sie die Cauchy-Schwarz-Ungleichung?
	\textbf{Lösung:}\\
		tbd
\end{enumerate}

\subsection*{Aufgabe 17: \mdseries\itshape Durchschnitt und Vereinigung von Mengen}
	Sie $M \subset \mathbb{R}^n$. Zeigen Sie
	\begin{enumerate}[(i)]
		\item $\overline{M} = \bigcap \{ A \subset \mathbb{R}^n \; | \; A \text{ abgeschlossen, } M \subset A \}$\\
		\textbf{Beweis:}\\
            Sei $T = \bigcap \{ A \subset \mathbb{R}^n \; | \; A \text{ abgeschlossen, } M \subset A \}$\\
		    Wir müssen die beiden Fälle $\overline{M} \subset T$ und $T \subset \overline{M}$ zeigen.\\

            Fangen wir mit $T \subset \overline{M}$ an.\\
            Wir wissen, dass $A \cap B \subset A$ für alle Mengen $A$ und $B$ gilt. 
            Da nun $M \subset \overline{M}$ gilt, ist $\overline{M}$ eine abgeschlossene Menge, die
            wir in $T$ vereinigt haben. Also ist $T \subset \overline{M}$.\\

            Als nächstes zeigen wir $\overline{M} \subset T$.\\
            Nehmen wir an, dass es nicht so wäre, dann gebe es mindestens einen Punkt $x \in \overline{M}$,
            mit $x \not \in T$. Da alle Mengen, die in $T$ geschnitten werden $M$ enthalten, wissen wir,
            dass $x\in M$ liegen muss.\\
            Daraus ergibt sich, dass $x \in \delta M$ liegen muss.\\

            \textbf{
            (TODO FORMULIERUNG: Weil der Rand in jeder epsilon Umgebung in der Menge liegen muss,
            kann es keine kleinere abgeschlossene Menge als $\overline{M}$ geben, da sonst
            nicht jeder Punkt von $M$ in dieser Menge liegen kann)
            }\\
            \mbox{} \hfill $\square$

		\item $\overset{\circ}{M} = \bigcup \{ \Omega \subset \mathbb{R}^n \; | \; \Omega \text{ offen, } \Omega \subset M \}$\\
		\textbf{Beweis:}\\
		    Sei $T = \bigcup \{ \Omega \subset \mathbb{R}^n \; | 
                \; \Omega \text{ offen, } \Omega \subset M \}$.\\
            Wir müssen erneut zeigen, dass $\overset{\circ}{M} \subset T$ und 
            $T \subset \overset{\circ}{M}$ gilt. Fangen wir mit
            $\overset{\circ}{M} \subset T$ an.\\
            
            Nach der Definition von inneren Punkten, liegt jeder dieser Punkte auch in $M$.
            Da $\overset{\circ}{M}$ eine offene Menge ist, wird diese mit in der Vereinigung von
            allen offenen Mengen sein. Daher gilt auch $\overset{\circ}{M} \subset T$.\\

            Als nächstes zeigen wir $T \subset \overset{\circ}{M}$.\\
            Wir wissen, dass die Vereinigung in diesem Fall eine offene Menge sein muss. Für jedes
            $x \in T$ muss also ein $\varepsilon > 0$ exsistieren, so dass $B_\varepsilon(x) \subset T$
            liegen muss. Dies bedeutet aber, da alle Mengen, die wir vereinigt haben, die Bedingung
            $\Omega \subset M$ erfüllen. Daher muss insbesondere $B_\varepsilon(x) \subset M$ für jedes
            $x \in T$ erfüllt sein. Dies ist nun allerdings genau die Definition von $\overset{\circ}{M}$.\\
            \mbox{}\hfill $\square$
            
	\end{enumerate}

\subsection*{Aufgabe 18: \mdseries\itshape Durchmesser und Abstand von Mengen}
	Sei $M \subset \mathbb{R}^n$. Man definiert
	$$
		diam \; M := \sup \{ |x-y| \; | \; x,y \in M \}
	$$
	Für $x \in M$ und $M \subset \mathbb{R}^n$ definiert man den Abstand von x zu M durch
	$$
		dist (x,M) := \inf \{|x-y| \; | \; y \in M \}.
	$$
	\begin{enumerate}[(i)]
		\item Zeigen Sie, dass $diam \; M = diam \overline{M}$.\\
		\textbf{Beweis:}\\
			Wir unterscheiden zunächst 3 Fälle.\\
            Zunächst kann der Punkt, zum einen können die entferntesten Punkte
            beide auf dem Rand liegen, dann könnte einer auf dem Rand und der andere
            im inneren liegen und zuletzt können beide im innere liegen.\\

            Liegen beide entferntesten Punkte auf dem Rand $\delta M$, so liegen
            sie insbesondere im Abschluss auch auf dem Rand $\delta M$. Damit ist
            die Gleichheit erfüllt.

            Liegt einer der Punkte nicht auf dem Rand, so können wir uns in der Umgebung
            einen beliebigen Punkt inneren Suchen. Wir wissen, dass um einen inneren Punkt
            $x \in \overset{\circ}{M}$ eine offene Kugel gibt, mit $\varepsilon > 0$ und
            $B_\varepsilon ( x ) \subset M$. Wir können nun parallel zum anderen Punkt
            unseren Punkt nach außen schieben. Damit wird der Abstand um $\varepsilon > 0$
            größer. Nun wissen wir, dass durch Grenzwertbildung ein Punkt auf dem Rand
            erreicht wird (definition der Vorlesung). Damit konvergiert auch
            der Durchmesser des Objektes gegen den Durchmesser des Abschlusses.\\

            Das gleiche gilt auch, wenn beide Punkte im inneren liegen. Hier müssen
            nur beide Punkte im limes nach außen geschoben werden und erreichen so
            einen Punkt auf dem Rand. Der Durchmesser wird im Grenzwert dann auch gleich sein.\\

            Da der Durchmesser über das supremum definiert ist, ist der Grenzwert im
            falle der offenen Kugel der genommene Wert und damit gleich dem Abstand auf dem Rand.\\
            \mbox{}\hfill $\square$

		\item Beweisen Sie die folgenden Äquivalenzen
			\begin{enumerate}[a)]
				\item $x \in \overline{M} \Longleftrightarrow dist(x,M) = 0$\\
				\textbf{Beweis:}\\
				    $\Rightarrow$:\\
                        Sei $x \in \overline{M}$. Dann ist
                        $|x - x| = 0$, da $x$ insbesondere in $M$ lag.\\
                        Nach Dreiecksungleichung kann kein Wert kleiner als
                        0 erreicht werden, somit ist das Infimum auch gleich 0.\\

                    $\Leftarrow$:\\
                        $\exists y \in M \; : \; |x - y | = 0$. Da wir uns 
                        durch den Betrag in einem Metrischen Raum befinden müssen,
                        gilt in diesem per Definiton $|x - y| = 0 \Leftrightarrow x = y$.
                        Deshalb muss $x \in M$ gelten.\\
                    \mbox{} \hfill $\square$	

				\item $x \in \overset{\circ}{M} \Longleftrightarrow dist(x,M^c) > 0$\\
				\textbf{Beweis:}\\
				    $\Rightarrow$:\\
                        Da $x \in \overset{\circ}{M}$ gilt, muss nach Definition eine
                        muss nach Definition eine offene Kugel existieren mit $\varepsilon > 0$
                        $B_\varepsilon(x) \subset M$. Wir wissen nun also, dass
                        $\forall y \in M^c \; : \; |x - y | \geq \varepsilon > 0$, da in
                        der unmittelbaren Umgebung um $x$ nur Punkte aus $M$ liegen können.\\

                    $\Leftarrow$:\\
                        Nach Vorraussetzung existiert ein $\varepsilon > 0$, mit
                        $dist(x,M^c) \geq \varepsilon > 0$. Nun folgt daraus, dass
                        $B_\varepsilon(x) \subset M$ gilt. Wäre dies nicht so,
                        würde ein Punkt in der Kugel existieren, der im nicht in der Menge
                        liegt. Dies würde aber bedeuten, dass der Abstand von $x$ zu einem
                        Punkt aus dem Komplement kleiner als $\varepsilon$ sein muss.
                        Damit ist das Infimum von allen kleiner als $\varepsilon$ und damit
                        auch $dist(x,M^c)$. Dies ist nun allerdings ein Widerspruch zur Annahme.\\
                    \mbox{} \hfill $\square$

				\item $x$ Randpunkt von $M$ $\Longleftrightarrow dist(x,M) = dist(x,M^c) = 0$\\
				textbf{Beweis:}\\
					$\Rightarrow$:\\
                        Sei $x$ Randpunkt von $M$.\\
                        Das bedeutet, zum einen, dass $\forall \varepsilon > 0 \; : \; B_\varepsilon(x)
                        \cap M \not= \empty$ ist, aber $\forall \varepsilon > 0 \; : \; \neg (B_\varepsilon(x)
                        \subset M )$.\\
                        Die zweite Aussage bedeutet insbesondere, dass $\forall \varepsilon > 0 \; : \;
                        B_\varepsilon(x) \cap M^c$.\\
                        Nun gilt, dass $dist (x,M) < \varepsilon$ für alle $\varepsilon > 0$, da in
                        den Kugeln Punkte liegen. Nun ist $\inf \{ \varepsilon \; | \; \varepsilon > 0 \} = 0$
                        . Und dieses Spiel gilt im Umkehrschluss auch für den Abstand zum komlement.\\

                    $\Leftarrow$:\\ 
                        Wir wissen, dass in jeder $\varepsilon > 0$ Umgebung um $x$ ein Punkt $y \in M$ und
                        ein Punkt $z \in M^c$ existieren muss, mit $y \in B_\varepsilon(x)$ und
                        $z \in B_\varepsilon(x)$. Das Infimum ist, wie oben gezeigt, genau 0.\\
                        Nun sind aber 2 solche Punkte in $M$ und $M^c$ in jeder offenen Kugel vorhanden,
                        daher kann der Schnitt nie leer sein (zu $M$ und $M^c$).\\
                        Dies ist genau die Definition eines Randpunktes.\\
                        \mbox{} \hfill $\square$
			\end{enumerate}
		
		\item Für $\varepsilon > 0$ definiert die $\varepsilon$-Umgebung einer Menge M durch
			$$
				M_\varepsilon := \{ x \in \mathbb{R}^N \; | \; dist(x,M) < \varepsilon \}.
			$$
			Zeigen Sie, dass für alle $\varepsilon > 0$ die Menge $M_\varepsilon$ offen ist, und bestimmen Sie
			$$
				\underset{\varepsilon > 0}{\bigcap} M_\varepsilon
			$$
			\textbf{Lösung:}\\
			    Sei $\varepsilon > 0$ beliebig aber fest.\\
                Sei nun $x_0 \in M_\varepsilon$ ein beliebiger Punkt. Nach Definition
                von $M_\varepsilon$ ist $dist(x_0,M) = d_0 < \varepsilon$.
                Wir wählen uns nun einen Radius um diesen Punkt, um so zu zeigen, dass
                die offene Kugel mit diesem Radius in $M_\varepsilon$ liegt.\\
                Wir können zunächst beobachten, dass wenn $x \in M$ gilt, dass der
                Radius dann trivialerweise $\varepsilon$ sein kann, da alle Punkte
                in diesem Radius in $M_\varepsilon$ liegen. Wir betrachten also nur
                Punkte mit $dist(x_0,M) > 0$. Bei der Konstrution des Radius müssen wir nun
                zum einen Beachten, dass wir nicht über den $\varepsilon$ Abstand zu Menge lappen,
                aber auch nicht zur Seite herrausfallen, wenn wir uns zu nah in einer Spitze befinden.\\
                
                Wir wählen also den Radius $r_0 = \min \{\varepsilon - d_0, \varepsilon^2-d_0^2 \}$.
                Der erste ist der Abstand in Verlängerung das Abstandes zum Punkt in der Menge,
                von dem wir $d_0$ entfernt sind, das andere ist mit dem Satz des Pythagoras umgeformt
                und gibt den Abstand senkrecht zur Verlängerung an.\\
                Nun wissen wir, dass alle Punkte im Kreis $B_{r_0}(x_0)$ vom Ursprünglichen Punkt
                $x$ zu dem $|x - x_0 | < \varepsilon$ war auch weniger als $\varepsilon$ entfernt sind.
                Daher muss $B_{r_0}(x_0) \subset M_\varepsilon$ sein.\\
                \mbox{} \hfill $\square$\\
            
                Nun bestimmen wir $T = \underset{\varepsilon > 0}{\bigcap} M_\varepsilon$.\\
                Wir zeigen, dass $\overline{M} = T$ gilt.\\
                
                $\subset$:\\
                Wir können als erstes leicht sehen, dass $M \subset T$ gilt, da wir schon
                bewiesen haben, dass $dist(x,M) = 0 \Leftrightarrow  x \in M$ gilt.
                Also liegt $x$ insbesondere für jedes $\varepsilon$ in $M_\varepsilon$.\\

                Als nächstes müssen wir nur noch zeigen, dass der Rand $\delta M \subset T$
                enthalten ist.\\
                Für den Rand gilt nun die Definition $x \in \delta M \Leftrightarrow \forall \varepsilon
                > 0 \; : \; B_\varepsilon (x) \cap M \not = \emptyset$.\\
                Wir wissen nun aber, dass nach dem Schnitt für alle $\varepsilon$ $\exists y \in M \; : \;
                | x - y | < \varepsilon$ existieren muss. Dann muss allerdings nach dem Schnitt
                genau so ein $y$ existieren.\\

                $\supset$:\\
                Sei $x \in T$.
                Wir machen prinzipiell die selbe Unterscheidung, wie eben. Entweder ist
                $x \in M$ oder $x \in \delta M$.\\
            
                Wir eben schon erwähnt, gilt für jedes $\varepsilon > 0$ $M \subset M_\varepsilon$.
                Also auch insbesondere $M \in \underset{\varepsilon > 0}{\bigcap} M_\varepsilon$.\\

                Nehmen wir nun einmal an, dass $x \not \in \delta M$ liegt. Das bedeutet, dass es ein
                $\varepsilon ' > 0$ gibt, so dass $B_{\varepsilon '}(x) \cap M = \emptyset$. Dies
                bedeutet aber insbesondere, dass $x \not\in M_\varepsilon$ liegen, kann da es keinen
                Punkt in $M$ gibt, der einen Abstand hat, der Nah genug an $x$ liegt.\\
                Daraus folgt, dass im Schnitt von allen $M_\varepsilon$ der Punt nicht drin liegen kann.\\

                \mbox{} \hfill $\square$
	\end{enumerate}

\subsection*{Aufgabe 19: \mdseries\itshape Umfang von Mengen}
	\begin{enumerate}[(i)]
		\item Man zeige, dass es für jede beschränkte Menge $M \subset \mathbb{R}^n$, die aus mindestens zwei Punkten besteht,
		genau eine Kugel $K = \overline{B_R(a)}$ mit kleinstmöglichem Radius $R > 0$ gibt, die $M$ enthält. Man nennt diese Kugel 
		$K$ die Umkugel von M und den Radius R den Umkugelradius von M.\\
		\textbf{Beweis:}\\
            Zunächst wissen wir, da die Menge beschränkt ist, dass wir auf jedenfall eine Kugel
            existiert, die alle Elemente enthält. Wähle $x \in M$ 
            und nimm den Kreis $B_{diam \, M}(x)$, dieser enhält alle Elemente der Menge.
            Wir müssen also nur zeigen, dass es einen kleinstmöglichen
            Radius gibt und das der Mittelpunkt eindeutig ist.\\

            Es existiert eine Kugel mit kleinst möglichem Radius.\\
            \textbf{Beweis:}\\
                Nehmen wir an, es gebe keine Kugel mit kleinst möglichem Radius.
                Das bedeutet, dass es um einen Mittelpunkt $a$ eine folge von Kugel
                gibt, mit $(r)_{n\in \mathbb{N}}$ eine konvergente Folge von Radien.
                Die Folge ist nach unten beschränkt (da die Menge mindestens 2 Punkte hat)
                und da es keinen kleinst möglichen Radius gibt, muss es eine 
                Folge geben, so dass die Radien immer kleiner werden.\\
                Sei $r = \underset{n \rightarrow \infty}{\lim} r_n$.\\

                Was diese Folge erfüllen muss, ist, dass um den Mittelpunkt
                für jedes $k \in \mathbb{N}$ $M \subset \overline{B_{r_k}(a)}$ gelten
                muss. Was wir nun aussagen können ist, dass $\overset{\circ}{M} \subset B_{r}$.
                Dies gilt, da alle inneren Punkte in einer $\varepsilon$ Umgebung noch Punkte
                in der Menge hat und daher kein Folgeglied in diese Umgebung eindringen kann.\\

                Nun müssen wir nur noch zeigen, dass $\delta M \subset \overline{B_{r}(a)}$ ist.\\
                Wie wir aber schon gesehen haben, ist $\overset{\circ}{M} \subset B_{r}(a)$.
                Wenn wir jetzt auf beiden seiten den Abschluss bilden, bleibt die Beziehung erhalten.\\
                Nun gilt, dass $M \subset \overline{M} \subset \overline{B_{r}(a)}$.\\

                Dies ist ein Widerspruch zur Annahme das keine solche Kugel existieren kann.\\

            Eindeutigkeit des Mittelpunktes.\\
            \textbf{Beweis:}\\
                Nehmen wir an, dass $r$ der minimale Umradius von $M$ ist
                und dass es Punte $x,y \in M$ gibt, mit $x \not= y$, $M \subset B_r(x)$
                und $M \subset B_r(c)$.\\

                Wissen wir, $|x - y | = d > 0$ gilt. Wir betrachten nun den Punkt
                $a = x + (x - y)/2$ der genau zwischen $x$ und $y$ liegt. Seien $E_y,E_x$ die
                Menge der Punkte, die in einer $\varepsilon >0$ Kugel um einen Randpunkt 
                um die Umkugel um $x$, bzw. $y$ liegen.\\
                Wir wissen, dass diese Punkte einen Abstand von maximal $r-\varepsilon$ von $x$, bzw. $y$
                haben. Im Grenzwert können wir sagen, dass es Punkte mit Abstand $r$ von $x,y$ gibt.\\

                Nun wissen wir nach Dreiecksungleichung, dass der kleinste Umkreis der Kugel um $a$
                bezeichnet mit $r_a$ die Gleichung $r_a \leq r + d/2$. Das $d>0$ ist
                ist dies ein Widerspruch zu unserer Annahme, dass $r$ der kleinste Radius ist.\\
            \mbox{} \hfill $\square$

		\item Sie $M \subset \mathbb{R}^n$ symmetrisch um den Ursprung, dass heißt
			$$
				x \in M \Longleftrightarrow -x \in M.
			$$
			Zeigen Sie, dass $M \subset \overline{B_{(diam \, M )/ 2}(0)}$
		\textbf{Beweis:}\\
            Was wir an dieser Stelle sehen können ist, dass der Radius das Kreises nicht
            kleiner sein kann, als $diam \, M / 2$. Da sowohl $x$ als auch $-x$ in der Menge
            $M$ sind, können wir annehmen, dass der Radius diese Größe hat. Wäre
            er nicht so, müsste der Mittelpunkt nicht im Nullpunkt liegen (dieser Punkt sichert
            uns $diam \, M$ zu wie in (i) gezeigt). Nun wissen wir aber das dieser auch nicht
            auf der Verbindungslinie von $x, -x$ liegen kann, da sonst der Abstand zu einem
            dieser Punkte größer werden würde.\\
            Liegt der Mittelpunkt aber nicht auf der Linie, so können wir nach Dreiecksungleichung
            schließen, dass es zu mindestens einem Punkt $x$ oder $-x$ einen Abstand gibt,
            der Größer als $diam \, M$ ist. Daher muss der Radius $diam \, M$ sein.\\
            Es dem selben Gedankengang, sehen wir, dass der Mittelpunkt auf der Verbindungslinie
            liegen muss und das er im Nullpunkt liegt.\\

            \mbox{} \hfill $\square$

		\item Man zeige, dass zwischen dem Umkugelradius $R$ und dem Durchmesser $\delta := diam \, M$ einer beschränkten Menge
		 $M \subset \mathbb{R}^2$ mit mindestens zwei Elementen die Beziehung
			$$
				R \leq \frac{\delta}{\sqrt{3}}
			$$
		besteht. Geben Sie ein Beispiel für eine dreipunktige Menge $M$, für die Gleichheit richtig ist an.\\
		\textbf{Lösung:}\\
			
	\end{enumerate}

\label{LastPage}
\end{document}


