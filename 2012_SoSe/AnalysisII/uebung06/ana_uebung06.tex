\documentclass[11pt,a4paper,ngerman]{article}
\usepackage[bottom=2.5cm,top=2.5cm]{geometry} 
\usepackage{babel}
\usepackage[utf8]{inputenc} 
\usepackage[T1]{fontenc} 
\usepackage{ae} 
\usepackage{amssymb} 
\usepackage{amsmath} 
\usepackage{graphicx}
\usepackage{fancyhdr}
\usepackage{fancyref}
\usepackage{listings}
\usepackage{xcolor}
\usepackage{paralist}

%%\usepackage[pdftex, bookmarks=false, pdfstartview={FitH}, linkbordercolor=white]{hyperref}
\usepackage{fancyhdr}
\pagestyle{fancy}
\fancyhead[C]{Analysis II}
\fancyhead[L]{Aufgabenblatt 6}
\fancyhead[R]{SoSe 2012}
\fancyfoot{}
\fancyfoot[L]{}
\fancyfoot[C]{\thepage \hspace{1px} of \pageref{LastPage}}
\renewcommand{\footrulewidth}{0.5pt}
\renewcommand{\headrulewidth}{0.5pt}
\setlength{\parindent}{0pt} 
\setlength{\headheight}{15pt}

\date{}
\title{Max Wisniewski , Alexander Steen}
\author{Tutor : Adrian Steffens}

\newcommand{\limes}[2][n]{\underset{ #1 \rightarrow #2}{\lim}}

\begin{document}

\lstset{language=Pascal, basicstyle=\ttfamily\fontsize{10pt}{10pt}\selectfont\upshape, commentstyle=\rmfamily\slshape, keywordstyle=\rmfamily\bfseries, breaklines=true, frame=single, xleftmargin=3mm, xrightmargin=3mm, tabsize=2}

\maketitle
\thispagestyle{fancy}

\subsection*{Aufgabe 16: \mdseries\itshape Die Ungleichungen von Hölder und Young}
Seien $x_1,...,x_n$ und $y_1,...,y_n$ reelle Zahlen.

\begin{enumerate}[(i)]
	\item Zeigen Sie, dass für
		$$
			\frac{1}{p} + \frac{1}{q} = 1, \; p,q>1,
		$$
		und für alle $x,y\geq 0$ die \emph{Youngsche Ungleichung}
		$$
			x^{\frac{1}{p}} y^{\frac{1}{q}} \leq \frac{1}{p} x + \frac{1}{q} y
		$$
		richtig ist.\\
	\textbf{Beweis:}\\
		tbd

	\item Zeigen Sie nun unter Verwendung der Youngschen Ungleichung die Höldersche Ungliechung
		$$
			\left| \overset{n}{\underset{i=1}{\sum}} x_i y_i \right| \leq \left( \overset{n}{\underset{i=1}{\sum}} |x_i|^p \right)^{\frac{1}{p}}
				\left( \overset{n}{\underset{i=1}{\sum}} | y_i |^q \right) ^{\frac{1}{q}}.
		$$
		Für welche $p$ und $q$ gewinnen Sie die Cauchy-Schwarz-Ungleichung?
	\textbf{Lösung:}\\
		tbd
\end{enumerate}

\subsection*{Aufgabe 17: \mdseries\itshape Durchschnitt und Vereinigung von Mengen}
	Sie $M \subset \mathbb{R}^n$. Zeigen Sie
	\begin{enumerate}[(i)]
		\item $\overline{M} = \bigcap \{ A \subset \mathbb{R^n} \; | \; A \text{ abgeschlossen, } M \subset A \}$\\
		\textbf{Beweis:}\\
			tbd

		\item \overset{\circ}{M} = \bigcup \{ \Omega \subset \mathbb{R}^n \; | \; \Omega \text{ offen, } \Omega \subset M \}$\\
		\textbf{Beweis:}\\
			tbd
	\end{enumerate}

\subsection*{Aufgabe 18: \mdseries\itshape Durchmesser und Abstand von Mengen}
	Sei $M \subset \mathbb{R}^n$. Man definiert
	$$
		\diam \; M := \sup \{ |x-y| \; | \; x,y \in M \}
	$$
	Für $x \in M$ und $M \subset \mathbb{R}^n$ definiert man den Abstand von x zu M durch
	$$
		dist (x,M) := \inf \{|x-y| \; | \; y \in M \}.
	$$
	\begin{enumerate}[(i)]
		\item Zeigen Sie, dass $\diam \; M = \diam \overline{M}$.\\
		\textbf{Beweis:}\\
			tbd

		\item Beweisen Sie die folgenden Äquivalenzen
			\begin{enumerate}[a)]
				\item $x \in \overline{M} \Longleftrightarrow dist(x,M) = 0$\\
				\textbf{Beweis:}\\
					tbd

				\item $x \in \overset{\circ}{M} \Longleftrightarrow dist(x,M^c) > 0$\\
				\textbf{Beweis:}\\
					tbd

				\item $x$ Randpunkt von $M$ $\Longleftrightarrow dist(x,M) = dist(x,M^x) = 0$\\
				textbf{Beweis:}\\
					tbd

			\end{enumerate}
		
		\item Für $\varepsilon > 0$ definieren die $\varepsilon$-Umgebung einer Menge M durch
			$$
				M_\varepsilon := \{ x \in \mathbb{R}^N \; | \; dist(x,M) < \varepsilon \}.
			$$
			Zeigen Sie, dass für alle $\varepsilon > 0$ die Menge $M_\varepsilon$ offen ist, und bestimmen Sie
			$$
				\underset{\varepsilon > 0}{\bigcap} M_\varepsilon
			$$
			\textbf{Lösung:}\\
				tbd
	\end{enumerate}

\subsection*{Aufgabe 19: \mdseries\itshape Umfang von Mengen}
	\begin{enumerate}[(i)]
		\item Man zeige, dass es für jede beschränkte Menge $M \subset \mathbb{R}^n$, die aus mindestens zwei Punkten besteht,
		genau eine Kugel $K = \overline{B_R(a)}$ mit kleinstmöglichem Radius $R > 0$ gibt, die $M$ enthält. Man nennt diese Kugel 
		$K$ die Umkugel von M und den Radius R den Umkugelradius von M.\\
		\textbf{Beweis:}\\
			tbd

		\item Sie $M \subset \mathbb{R}^n$ symmetrisch um den Ursprung, dass heißt
			$$
				x \in M \Longleftrightarrow -x \in M.
			$$
			Zeigen Sie, dass $M \subset \overline{B_{(\diam \, M )/ 2}(0)}
		\textbf{Beweis:}\\
			tbd

		\item Man zeige, dass zwischen dem UmkugelRadius $R$ und dem Durchmesser $\delta := \diam \, M$ einer beschränkten Menge
		 $M \subset \mathbb{R}^2$ mit mindestens zwei Elementen die Beziehung
			$$
				R \leq \frac{\delta}{\sqrt{3}}
			$$
		besteht. Geben Sie ein Beispiel für eine dreipunktige Menge $M$, für die Gleichheit richtig ist an.\\
		textbf{Lösung:}\\
			tbd
	\end{enumerate}

\label{LastPage}
\end{document}


