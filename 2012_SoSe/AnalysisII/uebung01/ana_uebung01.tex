\documentclass[11pt,a4paper,ngerman]{article}
\usepackage[bottom=2.5cm,top=2.5cm]{geometry} 
\usepackage{babel}
\usepackage[utf8]{inputenc} 
\usepackage[T1]{fontenc} 
\usepackage{ae} 
\usepackage{amssymb} 
\usepackage{amsmath} 
\usepackage{graphicx}
\usepackage{fancyhdr}
\usepackage{fancyref}
\usepackage{listings}
\usepackage{xcolor}
\usepackage{paralist}

\usepackage[pdftex, bookmarks=false, pdfstartview={FitH}, linkbordercolor=white]{hyperref}
\usepackage{fancyhdr}
\pagestyle{fancy}
\fancyhead[C]{Analysis II}
\fancyhead[L]{Aufgabenblatt 1}
\fancyhead[R]{SoSe 2012}
\fancyfoot{}
\fancyfoot[L]{}
\fancyfoot[C]{\thepage \hspace{1px} of \pageref{LastPage}}
\renewcommand{\footrulewidth}{0.5pt}
\renewcommand{\headrulewidth}{0.5pt}
\setlength{\parindent}{0pt} 
\setlength{\headheight}{0pt}

\date{}
\title{Max Wisniewski , Alexander Steen}
\author{Tutor : not known}

\begin{document}

\lstset{language=Pascal, basicstyle=\ttfamily\fontsize{10pt}{10pt}\selectfont\upshape, commentstyle=\rmfamily\slshape, keywordstyle=\rmfamily\bfseries, breaklines=true, frame=single, xleftmargin=3mm, xrightmargin=3mm, tabsize=2}

\maketitle
\thispagestyle{fancy}

%% ------------------------------------------------------
%%                     Aufgabe 1
%% ------------------------------------------------------

\section*{Aufgabe 1: \mdseries Spezielle gleichmäßige Funktionen}

Sei $A \subset \mathbb{R}$. Eine Funktion $f \; : \; A \rightarrow \mathbb{R}$ heißt \emph{Hölder stetig} mit Exponent $\alpha \in \left( 0,1 \right]$ wenn es eine Konstante $C > 0$ gibt, so dass für alle $x,y \in A$ die Ungleichung
$$
    \left| f(x) - f(y)\right| \leq C \left| x - y \right|^\alpha
$$

gilt. Ist $\alpha = 1$ so nennt man $f$ \emph{Lipschitzstetig}

\begin{enumerate}[\bfseries a)]
    \item Sei $A=\left\{ z \in \mathbb{R} \; | \; z \geq 0 \right\}$ und $f \; : \; A \rightarrow \mathbb {R}$ gegeben durch $f(z) = \sqrt{z}$. Zeigen Sie dass $f$ Hölderstetig mit $\alpha = \frac{1}{2}$ ist.\\

\textbf{Lösung:}\\

Sei $x,y \in [a,b]$, dann gilt
$$
\begin{array}{lrcl}
&| \sqrt{x} - \sqrt{y} | & \leq & | \sqrt{x} | - | \sqrt{y} | \leq C \cdot \sqrt{| x - y |}\\
\Leftrightarrow & \left( \sqrt{x} - \sqrt{y} \right)^2 & \leq & C^2 \cdot |x - y|\\
\Leftrightarrow & x - 2\sqrt{x}\sqrt{y} + y &\leq& C^2 \cdot | x - y | \leq C^2 \cdot \left( |x| + |-y| \right)\\
\Leftrightarrow & - 2 \sqrt{x}\sqrt{y} &\leq& (C^2-1) (x + y),
\end{array}
$$
Für $C>1$, da $\sqrt{x}$ und $\sqrt{y}$ beide größer Null sind, ist die linke Seite der Gleichung kleiner Null. Da wir rechts $x+y$ rechnen und beide größer null sind, gilt $x+y > 0$. Wenn nun $C>1$ belibt die rechte Seite positiv. Für ein $C>1$ ist die Gleichung erfüllt und damit ist $f$ Hölderstetig mit $\alpha = \frac{1}{2}$

    \item Sei $A=\mathbb{R}$ und $f=\arctan$ (eingeschränkt auf $\left( -\frac{\pi}{2},\frac{\pi}{2} \right)$. Zeigen Sie, dass $f$ Lipschitz stetig ist.\\

\textbf{Lösung:}

tbd

    \item Sei $f \; : \; A \rightarrow \mathbb{R}$ Hölderstetig. Zeigen Sie, dass $f$ gleichmäßig stetig ist.\\

\textbf{Lösung:}

Sei $f \; : \; A \rightarrow \mathbb{R}$ Hölderstetig \\
$\Rightarrow \exists \alpha \in (0,1] \exists C > 0 \forall x,y \in A: \left| f(x) - f(y)\right| \leq C \left| x - y \right|^\alpha$ \\
Es sei $\varepsilon > 0$ und $ \left| x - y\right| < \delta$ für ein $\delta > 0$ $\forall x,y \in A$.\\

Z.z. $\left| f(x) - f(y)\right| < \varepsilon$.\\
Wähle $\delta = \left( \frac{\varepsilon}{C} \right)^{\frac{1}{\alpha}}$.\\
Also gilt:
$$
  \left| f(x) - f(y)\right| \leq C \left| x - y \right|^\alpha
< C \delta^\alpha = C \left( \left( \frac{\varepsilon}{C} \right)^{\frac{1}{\alpha}}\right)^\alpha
= \varepsilon
$$

\end{enumerate}

%%-----------------------------------------------
%%              Aufgabe 2
%%-----------------------------------------------

\section*{Aufgabe 2 : \mdseries Hauptsatz der Differential- und Integralrechnung}

Finden Sie die Ableitung der Funktion $f \; : \; \mathbb{R} \rightarrow \mathbb{R}$ definiert durch die folgenden Ausdrücke.

\begin{enumerate}[i)]
    \item $F(x) = \int_{0}^{x^2} \sin \, t \; dt$.\\

\textbf{Lösung:}\\

Nach dem Hauptsatz der Differential- und Integralrechnung ist die Ableitung die Umkehrung des Integrals. Die Grenzen werden nach Substitutionsmethode abgeleitet und als Faktor übernommen.
$$
\begin{array}{rcl}
F'(x)	&=& \left( \int_0^{x^2} \sin \, t \; dt\right)'\\
	&=& (x^2)' \cdot \sin x^2\\
	&=& 2x \cdot \sin x^2
\end{array}
$$

    
    \item $F(x) = \exp \left( \int_{0}^{x} p(t)\; dt\right)$, wobei $p \, : \, \mathbb{R} \rightarrow \mathbb{R}$\\

\textbf{Lösung:}\\

ACHTUNG : KÖNNTE EINE FALLSE SEIN, DA $p$ NICHT STETIG ODER INTEGRIERBAR IST.

$$
\begin{array}{rcl}
F'(x)	&=& \left( \exp \left( \int_0^x p(t) \; dt \right) \right)'\\
	&\stackrel{\text{Subst.}}{=}& \left( \int_0^x p(t) \; dt\right)' \cdot \exp \left( \int_0^x p(t) \; dt\right)\\
	&=& p(x) \cdot F(x)
\end{array}
$$

    \item Es sei $h \, : \, \mathbb{R} \rightarrow \mathbb{R}$ stetig und $f$ und $g$ auf ganz $\mathbb{R}$ differenzierbar. Setzten Sie dann
$$
    F(x) = \int_{f(x)}^{g(x)} h(t) \; dt
$$
und berechnen die Ableitung von $F$.\\

\textbf{Lösung:}\\

$$
\begin{array}{rcl}
F'(x)	&=& \left( \int_{f(x)}^{g(x)} h(t) \; dt \right)'\\
	&\stackrel{\text{Haupt.}}{=}& g'(x) h(g(x)) - f'(x) h (f(x))
\end{array}
$$

Die Ableitung existiert, da $g$ und $f$ differenzierbar sind. Da $h$ stetig ist, kann der Hauptsatz der Differential- und Integralrechnung verwendet werden.

\end{enumerate}
%% ---------------------------------------------
%%              Aufgabe 3
%% ---------------------------------------------

\section*{Aufgabe 3 : \mdseries Mittelwertsatz der Integralrechnung}

\begin{enumerate}[i)]
    \item Es sei $f$ eine auf dem Interval $[a,b]$ integirebare Funktion mit $m \leq f(x) \leq M$ für alle $x\in[a,b]$. Dann gibt es ein $\mu \in [m,M]$ mit Eigenschaft
$$
    \int_{a}^{b} f(x) \, dx = (b - a) \mu.
$$

\textbf{Lösung:}\\

tbd

    \item Es sei $f$ stetig auf $[a,b]$. Zeigen Sie, dass gilt 
$$
    \int_{a}^{b} f(x)\, dx = (b-a) f(\xi)
$$
für ein $\xi\in [a,b]$. Begründen Sie anhand eins Gegenbeispiels, dass die Stetigkeit von $f$ notwendig ist.\\

\textbf{Lösung:}\\


Aus Aufgabe 2 a) wissen wir, dass ein $\mu$ existiert, dass den Flächeninhalt ergibt. Es bleibt zu zeigen, dass es innerhalb der maximums, minimums Grenze im Interval liegt.



    \item Sei nun $f$ stetig auf $[a,b]$, und $g$ integrierbar und positiv (bzw. negativ) auf $[a,b]$. Zeigen Sie dass 
$$
    \int_{a}^{b} g(x)f(x)\, dx = f(\xi) \int_{a}^{b} g(x) \, dx
$$ 
für ein $\xi \in [a,b]$ gilt. Man nennt dies den Mittelwertsatz der Intergralrechnung   . Begründen Sie anhand eines Gegenbeispiels, dass die Vorzeichenbedingung an $g$ notwendig ist. 
\end{enumerate}

%% ---------------------------------------------------
%%                  Aufgabe 4
%% ---------------------------------------------------

\section*{Aufgabe 4 : \mdseries Positivitätseigenschaft des Integrals}

\begin{enumerate}[i)]
    \item Sei $f$ integrierbar auf $[a,b]$ und $f\geq 0$ für alle $x \in [a,b]$. Zeigen Sie, dass dann gilt
$$
    \int_{a}^{b} f(x) \, dx \geq 0.
$$

\textbf{Lösung:}\\

Sei $P=a=x_0<...<x_n=b$ Partitionsfolge über n von $[a,b]$, sodass $U_n$ und $O_n$ Unter- und Obersummen konvergieren. Diese Partitonsfolge muss existieren, da die Funktion integrierbar ist (nach der bisher angenommenen definition von integrierbar). Dann gilt für alle $S_i = \sup \{ f(y) \; | \; x_i < y < x_{i+1} \}$, $S_i \geq 0$, da jeder Funktionswert des Supremums größer gleich $0$.\\
 Nun ist $\int_a^b f(x) \, dx = \underset{n \rightarrow \infty}{\lim} O_n = \underset{n \rightarrow \infty}{\lim} \underset{i=0}{\overset{n-1}{\sum}} S_i \geq 0$, da jeder Summand, wie gezeigt größer oder gleich 0 ist.

    \item Geben Sie ein Beispiel einer Funktion $f$ mit folgenden Eigenschaften:
$$
    f(x) \geq 0 \text{ für alle } x\in [a,b], \, f(x_0) > 0\text { für ein } x_0 \in [a,b], \int_{a}^{b} f(x) \, dx = 0.
$$

\textbf{Lösung}\\

tbd

    \item Sei $f(x) \geq 0$ für alle $x \in [a,b]$ und $f$ stetig mit $x_0 \in [a,b]$ mit $f(x_0) > 0$. Zeigen Sie, dass dann auch gilt
$$
    \int_{a}^{b} f(x) \, dx > 0.
$$

\textbf{Lösung:}\\

tbd

    \item Sei $f$ stetig auf $[a,b]$. Es gelte
$$
\int_{a}^{b} f(x)g(x) \, dx = 0
$$
für alle stetigen Funktionen $g$ auf $[a,b]$. Zeigen Sie, dass $f\equiv 0$
\end{enumerate}
\label{LastPage}
\end{document}


