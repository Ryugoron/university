\documentclass[11pt,a4paper,ngerman]{article}
\usepackage[bottom=2.5cm,top=2.5cm]{geometry} 
\usepackage{babel}
\usepackage[utf8]{inputenc} 
\usepackage[T1]{fontenc} 
\usepackage{ae} 
\usepackage{amssymb} 
\usepackage{amsmath} 
\usepackage{graphicx}
\usepackage{fancyhdr}
\usepackage{fancyref}
\usepackage{listings}
\usepackage{xcolor}
\usepackage{paralist}

%%\usepackage[pdftex, bookmarks=false, pdfstartview={FitH}, linkbordercolor=white]{hyperref}
\usepackage{fancyhdr}
\pagestyle{fancy}
\fancyhead[C]{Analysis II}
\fancyhead[L]{Aufgabenblatt 7}
\fancyhead[R]{SoSe 2012}
\fancyfoot{}
\fancyfoot[L]{}
\fancyfoot[C]{\thepage \hspace{1px} of \pageref{LastPage}}
\renewcommand{\footrulewidth}{0.5pt}
\renewcommand{\headrulewidth}{0.5pt}
\setlength{\parindent}{0pt} 
\setlength{\headheight}{15pt}

\date{}
\title{Max Wisniewski , Alexander Steen}
\author{Tutor : Adrian Steffens}

\newcommand{\limes}[2][n]{\underset{ #1 \rightarrow #2}{\lim}}

\begin{document}

\lstset{language=Pascal, basicstyle=\ttfamily\fontsize{10pt}{10pt}\selectfont\upshape, commentstyle=\rmfamily\slshape, keywordstyle=\rmfamily\bfseries, breaklines=true, frame=single, xleftmargin=3mm, xrightmargin=3mm, tabsize=2}

\maketitle
\thispagestyle{fancy}

\subsection*{Aufgabe 20: \mdseries\itshape Stetigkeit der Abstandsfunktion}
Für eine feste Menge $M \subset \mathbb{R}^n$ betrachten wir die  Abstandsfunktion $dist(x,M) \, : \, \mathbb{R}^n \rightarrow \mathbb{R}$, mit
$(x,M) \mapsto \inf \{ |x - y | \; | \; y \in M \}$.

\begin{enumerate}[(i)]
	\item Zeigen Sie, dass gilt
	$$
		\forall \, x,y \in \mathbb{R}^n \; : \; \left| dist(x,M) - dist(y,M) \right| \leq |x - y| 
	$$
	\textbf{Beweis:}\\
		Seien $x,y \in \mathbb{R}^n$. Und $z^*,z' \in M$, so dass $|x-z'| = dist(x,M), \, |y-z^*| = dist(y,M)$.
			$$\begin{array}{rcl}
				| dist(x,M) - dist(y,M) | & \stackrel{Def.}{=} & | \inf \{ |x - z| \; | \; z \in M\} - \inf \{ | y - z | \; | \; z \in M \} |\\
					&\stackrel{z',z^*}{=}& | \, | x - z'| \, - \, | z^* - y| \, |\\
					&\leq& |\, | x - z' + z^* - y| \, |\\
					&=& | x -y + z^* - z'|\\
					&\leq& |x - y| + |z^* - z'|\\
					&\leq& |x - y|
			\end{array}$$
		Gilt, da auf dem Metrischen Raum $\mathbb{R}^n$ die Dreicksungleichung gilt.
		\mbox{} \hfill $\square$

	\item Ist die Distanzfunktion gleichmäßig stetig?\\
	\textbf{Lösung:}\\
		Wir haben es hier prinzipiell mit eine Lipschitz stetigen Funktion zu tun, mit der Konstante 1. Da wir den Satz aber noch nicht
		für metrische Räume gezeigt haben, werden wir wohl das ganze noch per Hand beweisen.\\

		Sei $\varepsilon > 0$ beliebig, aber fest. Sei $\delta = \varepsilon$.\\
		Nun wissen wir, nach Vorraussetzung gleichmäßigen Stetigkeit, dass $|x - y| < \delta = \varepsilon$ gelten soll.
		Aus dem ersten Teil wissen wir aber auch, dass $| dist(x,M) - dist(y,M | \leq |x -y|| < \varepsilon$ gelten muss.\\

		Damit ist $dist$ gleichmäßig stetig.\\
		\mbox{} \hfill $\square$
\end{enumerate}

\subsection*{Aufgabe 21 \mdseries\itshape Stetige Funktionen auf dichten Teilmengen}

Seien $g,f \, : \, \mathbb{R}^n \rightarrow \mathbb{R}^m$ stetige Funktionen. Sei fernen $S \subset \mathbb{R}^n$ eine \emph{dichte} Teilmenge des $\mathbb{R}$, d.h. es gilt $\overline{S} = \mathbb{R}^n$. Schließlich gelte 
$$
	\forall x \in S \; : \; f(x) = g(x).
$$
Zeigen Sie, dass dann auch
$$
	\forall x \in \mathbb{R}^n \; : \; f(x) = g(x)	
$$	
richtig ist.\\

\textbf{Beweis:}\\
	Für stetige Funktionen gilt auch bei $\mathbb{R}^n\rightarrow \mathbb{R}^m$, dass
	$\underset{n \rightarrow a}{\lim} f(n) = f( \underset{n \rightarrow a}{\lim} n)$.\\

	Nun wissen wir über den Abschluss, dass für jeden Punkt aus $x \in \overline{S}$ eine Folge $(t)_{n\in\mathbb{N}}$ in $S$ existiert,
	so dass $\underset{n \rightarrow \infty}{\lim} t_n = x$.\\

	Nun sei $x \in \mathbb{R}^n = \overline{S}$. Wir wissen nun, dass eine Folge $(t)_{n \in \mathbb{N}}$ existiert, mit
	$\underset{n \rightarrow \infty}{\lim} t_n = x$.
	$$\begin{array}{rcl}
		f(x) &\stackrel{Def.: t}{=}& f(\underset{n \rightarrow \infty}{\lim} t_n)\\
			&\stackrel{stetig}{=}& \underset{n \rightarrow \infty}{\lim} f(t_n)\\
			&\stackrel{Vor.}{=}& \underset{n \rightarrow \infty}{\lim} g(t_n)\\
			&\stackrel{stetig}{=}& g( \underset{n \rightarrow \infty}{\lim} t_n)\\
			&\stackrel{Def.: t}{=}& g(x)
	\end{array}$$

\subsection*{Aufgabe 22 \mdseries\itshape Stetigkeit in höheren Dimensionen}

Überprüfen Sie folgende Funktionen auf Stetigkeit im Punkt $(0,0)$.

\begin{enumerate}[(i)]
	\item
		$$
			f(x,y) := \left\{ 
			\begin{array}{lr}
				\frac{x^2y^2}{x^2+y^2} & \text{für }(x,y)\not= (0,0)\\
				0& \text{für }(x,y) = (0,0)
			\end{array}\right.
		$$
	\textbf{Lösung:}\\
		Sei $\varepsilon > 0$ beliebig, aber fest. Sei nun $\delta > 0$ und es gelte $|x - x_0| = |x| < \delta$.\\
		Da $|x| < \delta$ gilt, wissen wir, dass insbesondere $x=(x_1,x_2)$ $|x_1| < \delta $ und $|x_2| < \delta$ gelten muss.\\
		Wenn wir nun annehmen, dass wir $\delta < 1 $ gilt, dann gilt auch $x_1^2 < |x_1| < \delta$ und $x_2$ ebenso.\\

		$$\begin{array}{rcl}
			|f(x,y) - f(x_0)| &=& | f(x,y) - f((0,0)) |\\
				&=& |f(x,y) - (0,0)| = |f(x,y)|\\
				&=& | \frac{x^2y^2}{x^2 + y^2} |\\
				&=& \frac{x^2 y^2}{x^2 + y^2}\\
				&\stackrel{\delta}{<}& \frac{\delta^2}{2\delta}\\
				&=& \frac{1}{2} \delta
		\end{array}$$

		Wählen wir nun ein $\delta ' = 2 \varepsilon$, dann ist $|f(x,y) - f(0,0)| < \varepsilon$.\\
		\mbox{} \hfill $\square$

	\item
		$$
			g(x,y) := \left\{
				\begin{array}{lr}
					\frac{x^2 - y^2}{x^2 + y^2} & \text{für }(x,y) \not= (0,0) \\
					0 & \text{für } (x,y) = (0,0)
				\end{array}
			\right.
		$$
	\textbf{Lösung:}\\
		Sei $(x_k)_{k \in \mathbb{N}} = \left((x_k^1,0)\right)_{k \in \mathbb{N}}$ eine Folge über $\mathbb{R}^2$ mit $x_k^1 \to 0$ für $k \to \infty$ also insbesondere mit $x_k \to (0,0)$ für $k \to \infty$. \\

    Nun gilt aber $f(x_k) = f(x_k^1,0) = \frac{\left(x_k^1\right)^2 - 0}{\left(x_k^1\right)^2 + 0} = \frac{\left(x_k^1\right)^2}{\left(x_k^1\right)^2} = 1$ und damit $\lim_{k \to \infty} f(x_k) = 1 \neq 0 = f(0,0)$. \\
  Damit ist die Funktion nicht in $(0,0)$ stetig.
\end{enumerate}

\subsection*{Aufgabe 23 \mdseries\itshape Halbstetige Funktionen}

Die Funktion $f \, : \, \mathbb{R}^n \rightarrow \mathbb{R}$ heißt \emph{unterhalbstetig} in $x_0 \in \mathbb{R}^n$, wenn für jedes $\varepsilon > 0$ ein $\delta > 0$ existiert, so dass gilt
	$$
		| x - x_0 | < \delta \Rightarrow f(x) > f(x_0) - \varepsilon.
	$$

\begin{enumerate}[(i)]
	\item Definieren Sie analog \emph{oberhalbstetig}.\\
	\textbf{Def.:}\\
		Eine Funktion $f \, : \, \mathbb{R}^n \rightarrow \mathbb{R}$ heißt \emph{oberhalbstetig} in $x_0 \in \mathbb{R}^n$, wenn für jedes
		$\varepsilon > 0$ ein $\delta > 0$ existiert, so dass gilt
		$$
			| x - x_0 | < \delta \Rightarrow f(x) < f(x_0) + \varepsilon.
		$$

	\item Geben Sie jeweils ein Beispiel für eine nicht stetige, aber oberhalbstetige Funktion $f$ und für eine nicht stetige, aber unterhablstetige 
		Funktion $g$ an. Skizzieren Sie!\\
	\textbf{Lösung:}\\
		Wir bilden die beiden Funktionen $f, g \; : \; \mathbb{R} \rightarrow \mathbb{R}$.\\
		$$\begin{array}{rcl}
			f(x) &:=& \left\{\begin{array}{lr}
				\sin(x) + 1 &, x \geq 2\\
				\sin(x) - 1 &, x < 2 
			\end{array}\right.\\
			g(x) &:=& \left\{\begin{array}{lr}
				\sin(x) + 1 &, x > 2\\
				\sin(x) - 1 &, x \leq 2 
			\end{array}\right.
		\end{array}$$
		
		$f$ ist nun oberhalbstetig und $g$ ist unterhalbstetig. Die Funktionen sind bekanntermaßen bis auf $x_0 = 2$ stetig.\\
		Bei $f$ und $x_0$ haben wir in unmittelbarer Umgebung einen Sprung nach unten. Von $\sin(2)+1$ auf $\sin(2) -1$. 
		Da dieser Sprung aber nach unten geht, ist das ganze nun in der Definition abgefangen. Nach Rechts ist die Funktion stetig,
		daher können wir für jedes $\varepsilon > 0$ ein $\delta > 0$ finden, da die rechtsseitige Folge konvergiert.\\
		In $x_0 = 2$ sind beide Funktionen nicht stetig, da zwischen den beiden Grenzwerten eine Differenz von 2 herrscht.\\

		$g$ ist nun prinzipiell analog, da wir nu den Wert von $\sin(2)+1$ auf $\sin(2) - 1$ geändert haben. Damit ist es
		oberhalbstetig, da wir den Sprung nur nach oben haben, und die linksseitige Folge konvergiert.\\
	\textbf{Skizze:}\\
		
\newpage
	\item Zeigen Sie, dass $f \, : \, \mathbb{R}^n \rightarrow \mathbb{R}$ genau dann unterhalbstetig ist, wenn die Menge
	$$
		\Omega_a := \left\{ x \in \mathbb{R}^n \; | \; f(x) > a \right\}
	$$
	offen ist für alle $a \in \mathbb{R}$.\\
	\textbf{Beweis:}\\
		$\Rightarrow$:\\
			Wir wollen zeigen, dass $f \; : \; \mathbb{R}^n \rightarrow$ unterhalbstetig impliziert, dass $\Omega_a$ offen ist.\\
			Dazu nehmen wir an, dass $\Omega_a$ nicht offen ist.\\

			Dies bedeutet, dass ein Punkt $x_0$ existiert, so dass für jedes $\varepsilon > 0$ die offene Kugel 
			$B_\varepsilon(x_0) $ nicht komplett in $\Omega_a$ enthalten ist. Also existiert ein Punkt $y \in B_\varepsilon(x_0)$, für
			den gilt $x_0 \not \in \Omega$.\\

			Wir wissen nun also zum einen, dass $|x - y | < \varepsilon$ für jedes $\varepsilon > 0$ gilt. Darüber hinaus wissen wir auch,
			dass $f(y) \leq \Omega_a$. Wir haben also ein $a$, so dass $f(x) \leq f(x_0) - a$ gilt.\\
			
			Dies bedeutet, dass $f$ nicht unterhalbstetig ist.\\

		$\Leftarrow$:\\
			Wir bilden die Kontraposition. Wenn $f$ nicht unterhalbstetig ist (z.B. nicht stetig aber oberhalbstetig), dann existiert
			eine Menge $\Omega_a$ die nicht offen ist.\\

			In der Teilaufgabe (ii), haben wir eine Funktion angegeben, die nicht unterhalbstetig war. In $x_0 = 2$ galt nun,
			dass es keinen Wert links von diesem Wert gab, so dass der Funktionwert in jeder $\varepsilon > 0$ lag. Damit gab
			es insbesondere keine $x \in \mathbb{R}$ werte, so dass $f(x) < f(x_0) + \varepsilon$ galt.\\

			Aus diesem $\varepsilon$ können wir nun eine offene Kugel $B_\varepsilon(x_0)$ bilden. In der unmittelbaren Umgebung
			lagen nun allerdings (nach links) keine Werte. Daher ist für alle $\varepsilon > 0$ nicht die ganze Kugel in der Menge.\\

		\mbox{} \hfill $\square$
\end{enumerate}
\label{LastPage}
\end{document}
