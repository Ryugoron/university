\documentclass[11pt,a4paper,ngerman]{article}
\usepackage[bottom=2.5cm,top=2.5cm]{geometry} 
\usepackage{babel}
\usepackage[utf8]{inputenc} 
\usepackage[T1]{fontenc} 
\usepackage{ae} 
\usepackage{amssymb} 
\usepackage{amsmath} 
\usepackage{graphicx}
\usepackage{fancyhdr}
\usepackage{fancyref}
\usepackage{listings}
\usepackage{xcolor}
\usepackage{paralist}

%%\usepackage[pdftex, bookmarks=false, pdfstartview={FitH}, linkbordercolor=white]{hyperref}
\usepackage{fancyhdr}
\pagestyle{fancy}
\fancyhead[C]{Analysis II}
\fancyhead[L]{Aufgabenblatt 7}
\fancyhead[R]{SoSe 2012}
\fancyfoot{}
\fancyfoot[L]{}
\fancyfoot[C]{\thepage \hspace{1px} of \pageref{LastPage}}
\renewcommand{\footrulewidth}{0.5pt}
\renewcommand{\headrulewidth}{0.5pt}
\setlength{\parindent}{0pt} 
\setlength{\headheight}{15pt}

\date{}
\title{Max Wisniewski , Alexander Steen}
\author{Tutor : Adrian Steffens}

\newcommand{\limes}[2][n]{\underset{ #1 \rightarrow #2}{\lim}}

\begin{document}

\lstset{language=Pascal, basicstyle=\ttfamily\fontsize{10pt}{10pt}\selectfont\upshape, commentstyle=\rmfamily\slshape, keywordstyle=\rmfamily\bfseries, breaklines=true, frame=single, xleftmargin=3mm, xrightmargin=3mm, tabsize=2}

\maketitle
\thispagestyle{fancy}

\subsection*{Aufgabe 20: \mdseries\itshape Stetigkeit der Abstandsfunktion}
Für eine feste Menge $M \subset \mathbb{R}^n$ betrachten wir die  Abstandsfunktion $dist(x,M) \, : \, \mathbb{R}^n \rightarrow \mathbb{R}$, mit
$(x,M) \mapsto \inf \{ |x - y | \; | \; y \in M \}$.

\begin{enumerate}[(i)]
	\item Zeigen Sie, dass gilt
	$$
		\forall \, x,y \in \mathbb{R}^n \; : \; \left| dist(x,M) - dist(y,M) \right| \leq |x - y| 
	$$
	\textbf{Beweis:}\\
		Seien $x,y \in \mathbb{R}^n$ und $z \in M$ beliebiger Punkt in $M$.
			$$\begin{array}{rcl}
				| dist(x,M) - dist(y,M) | & \stackrel{Def.}{=} & | \inf \{ |x - z| \; | \; z \in M\} - \inf \{ | y - z | \; | \; z \in M \} |\\
					&\stackrel{z',z^*}{=}& | \, | x - z'| \, - \, | y - z^* | \, |\\
					&\leq& | \, | x - z'| \, | - | \, | y - z^* | \, |\\
					&=& | x-z'| - | y - z^*|\\
					&=& |x| + |z'| - |y| - |z^*|\\
					&\leq& |x - y| + |z' - z^*|\\
					&\leq& |x - y|
			\end{array}$$
		Gilt, da auf dem Metrischen Raum $\mathbb{R}^n$ die Dreicksungleichung gilt.
		\mbox{} \hfill $\square$

	\item Ist die Distanzfunktion gleichmäßig stetig?\\
	\textbf{Lösung:}\\
		Wir haben es hier prinzipiell mit eine Lipschitz stetigen Funktion zu tun, mit der Konstante 1. Da wir den Satz aber noch nicht
		für metrische Räume gezeigt haben, werden wir wohl das ganze noch per Hand beweisen.\\

		Sei $\varepsilon > 0$ beliebig, aber fest. Sei $\delta = \varepsilon$.\\
		Nun wissen wir, nach Vorraussetzung gleichmäßigen Stetigkeit, dass $|x - y| < \delta = \varepsilon$ gelten soll.
		Aus dem ersten Teil wissen wir aber auch, dass $| dist(x,M) - dist(y,M | \leq |x -y|| < \varepsilon$ gelten muss.\\

		Damit ist $dist$ gleichmäßig stetig.\\
		\mbox{} \hfill $\square$
\end{enumerate}

\subsection*{Aufgabe 21 \mdseries\itshape Stetige Funktionen auf dichten Teilmengen}

Seien $g,f \, : \, \mathbb{R}^n \rightarrow \mathbb{R}^m$ stetige Funktionen. Sei fernen $S \subset \mathbb{R}^n$ eine \emph{dichte} Teilmenge des $\mathbb{R}$, d.h. es gilt $\overline{S} = \mathbb{R}^n$. Schließlich gelte 
$$
	\forall x \in S \; : \; f(x) = g(x).
$$
Zeigen Sie, dass dann auch
$$
	\forall x \in \mathbb{R}^n \; : \; f(x) = g(x)	
$$	
richtig ist.\\

\textbf{Beweis:}\\
	Für stetige Funktionen gilt auch bei $\mathbb{R}^n\rightarrow \mathbb{R}^m$, dass
	$\underset{n \rightarrow a}{\lim} f(n) = f( \underset{n \rightarrow a}{\lim} n)$.\\

	Nun wissen wir über den Abschluss, dass für jeden Punkt aus $x \in \overline{S}$ eine Folge $(t)_{n\in\mathbb{N}}$ in $S$ existiert,
	so dass $\underset{n \rightarrow \infty}{\lim} t_n = x$.\\

	Nun sei $x \in \mathbb{R}^n = \overline{S}$. Wir wissen nun, dass eine Folge $(t)_{n \in \mathbb{N}}$ existiert, mit
	$\underset{n \rightarrow \infty}{\lim} t_n = x$.
	$$\begin{array}{rcl}
		f(x) &\stackrel{Def.: t}{=}& f(\underset{n \rightarrow \infty}{\lim} t_n)\\
			&\stackrel{stetig}{=}& \underset{n \rightarrow \infty}{\lim} f(t_n)\\
			&\stackrel{Vor.}{=}& \underset{n \rightarrow \infty}{\lim} g(t_n)\\
			&\stackrel{stetig}{=}& g( \underset{n \rightarrow \infty}{\lim} t_n)\\
			&\stackrel{Def.: t}{=}& g(x)
	\end{array}$$

\subsection*{Aufgabe 22 \mdseries\itshape Stetigkeit in höheren Dimensionen}

Überprüfen Sie folgende Funktionen auf Stetigkeit im Punkt $(0,0)$.

\begin{enumerate}[(i)]
	\item
		$$
			f(x,y) := \left\{ 
			\begin{array}{lr}
				\frac{x^2y^2}{x^2+y^2} & \text{für }(x,y)\not= (0,0)\\
				0& \text{für }(x,y) = (0,0)
			\end{array}\right.
		$$
	\textbf{Lösung:}\\
		tbd

	\item
		$$
			g(x,y) := \left\{
				\begin{array}{lr}
					\frac{x^2 - y^2}{x^2 + y^2} & \text{für }(x,y) \not= (0,0) \\
					0 & \text{für } (x,y) = (0,0)
				\end{array}
			\right.
		$$
	\textbf{Lösung:}\\
		tbd
\end{enumerate}

\subsection*{Aufgabe 23 \mdseries\itshape Halbstetige Funktionen}

Die Funktion $f \, : \, \mathbb{R}^n \rightarrow \mathbb{R}$ heißt \emph{unterhalbstetig} in $x_0 \in \mathbb{R}^n$, wenn für jedes $\varepsilon > 0$ ein $\delta > 0$ existiert, so dass gilt
	$$
		| x - x_0 | < \delta \Rightarrow f(x) > f(x_0) - \varepsilon
	$$

\begin{enumerate}[(i)]
	\item Definieren Sie analog \emph{oberhalbstetig}.\\
	\textbf{Def.:}\\
		tbd

	\item Geben Sie jeweils ein Beispiel für eine nicht stetige, aber oberhalbstetige Funktion $f$ und für eine nicht stetige, aber unterhablstetige 
		Funktion $g$ an. Skizzieren Sie!\\
	\textbf{Lösung:}\\
		tbd

	\item Zeigen Sie, dass $f \, : \, \mathbb{R}^n \rightarrow \mathbb{R}$ genau dann unterhalbstetig ist, wenn die Menge
	$$
		\Omega_a := \left\{ x \in \mathbb{R}^n \; | \; f(x) > a \right\}
	$$
	offen ist für alle $a \in \mathbb{R}$.\\
	\textbf{Beweis:}\\
		tbd
\end{enumerate}
\label{LastPage}
\end{document}
