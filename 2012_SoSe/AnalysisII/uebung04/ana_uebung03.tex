\documentclass[11pt,a4paper,ngerman]{article}
\usepackage[bottom=2.5cm,top=2.5cm]{geometry} 
\usepackage{babel}
\usepackage[utf8]{inputenc} 
\usepackage[T1]{fontenc} 
\usepackage{ae} 
\usepackage{amssymb} 
\usepackage{amsmath} 
\usepackage{graphicx}
\usepackage{fancyhdr}
\usepackage{fancyref}
\usepackage{listings}
\usepackage{xcolor}
\usepackage{paralist}

\usepackage[pdftex, bookmarks=false, pdfstartview={FitH}, linkbordercolor=white]{hyperref}
\usepackage{fancyhdr}
\pagestyle{fancy}
\fancyhead[C]{Analysis II}
\fancyhead[L]{Aufgabenblatt 3}
\fancyhead[R]{SoSe 2012}
\fancyfoot{}
\fancyfoot[L]{}
\fancyfoot[C]{\thepage \hspace{1px} of \pageref{LastPage}}
\renewcommand{\footrulewidth}{0.5pt}
\renewcommand{\headrulewidth}{0.5pt}
\setlength{\parindent}{0pt} 
\setlength{\headheight}{15pt}

\date{}
\title{Max Wisniewski , Alexander Steen}
\author{Tutor : Adrian Steffens}

\newcommand{\limes}[2][n]{\underset{ #1 \rightarrow #2}{\lim}}

\begin{document}

\lstset{language=Pascal, basicstyle=\ttfamily\fontsize{10pt}{10pt}\selectfont\upshape, commentstyle=\rmfamily\slshape, keywordstyle=\rmfamily\bfseries, breaklines=true, frame=single, xleftmargin=3mm, xrightmargin=3mm, tabsize=2}

\maketitle
\thispagestyle{fancy}

%% ------------------------------------------------------
%%                     Aufgabe 5
%% ------------------------------------------------------

\subsection*{Aufgabe 10: \mdseries\itshape Berechnung von Taylorpolynomen}

Bestimmen Sie die Taylorpolynome vom Grad $n$ um den Punkt $x_0 = 0$.
Die Taylorformel um den Entwicklungspunkt $x_0$ sieht folgender Maßen aus
$$
T_n(x) = \sum_{k=0}^{n} \frac{f^{(k)}(a)}{k!}(x-a)^k.
$$

\begin{enumerate}[(i)]
    \item $f(x) = \frac{1}{1+x}$:\\
        \textbf{Beh.:} $\forall k \in \mathbb{N} : f^{(k)}(x) = (-1)^{k}\cdot k! \frac{1}{(1+x)^{k+1}}$\\
        \textbf{I.A.:} $k=1$\\
            $f^{(1)}(x) = f'(x) = (\frac{d}{dx} 1 + x) \cdot -1 \cdot \frac{1}{(1+x)^2}$\\
            $= (-1)^1 \cdot 1! \frac{1}{(1+x)^2}$\\
            $k=0$ ist die Funktion selber.\\
        \textbf{I.S.:} $k \rightarrow: k+1$\\
            $$\begin{array}{rcl}
                f^{(k+1)}(x)    &=& \frac{d}{dx} f^{(k)}\\
                        &\stackrel{I.V.}{=}&
                            \frac{d}{dx} \left( (-1)^{k}k! \frac{1}{(1+x)^{k+1}} \right)\\
                        &=& (-1)^k k! \left(\frac{d}{dx} \frac{1}{(1+x)^{k+1}}\right)\\
                        &=& (-1)^k k! \cdot 1 \cdot -(k+1) \frac{1}{(1+x)^{k+2}}\\
                        &=& (-1)^{k+1} \cdot (k+1)! \frac{1}{(1+x)^{k+2}}
            \end{array}$$
        \mbox{} \hfill $\square$\\

        Diese Ableitung benutzen wir nun, um das Taylorpolynom aufzuschreiben.\\
        $$\begin{array}{rcl}
            T_n(x) &\stackrel{Def.}{=}& \underset{k=0}{\overset{n}{\sum}} \frac{f^{(k)}(a)}{k!}(x-a)^k.\\
                &\stackrel{Abl.}{=}& \underset{k=0}{\overset{n}{\sum}} 
                    \frac{(-1)^k k! \frac{1}{(1+a)^{k+1}}}{k!}(x-a)^k.\\
                &\stackrel{a=0}{=}&\underset{k=0}{\overset{n}{\sum}} 
                    (-1)^k \frac{1}{(1+0)^{k+1}} \cdot (x-0)^k\\
                &=& \underset{k=0}{\overset{n}{\sum}}
                    (-1)^k x^k 
        \end{array}$$
        Wir haben nun die Monomdarstellung erreicht, wobei die Koeffizienten $a_k = (-1)^k$ sind.

    \item $g(x) = \frac{1}{\sqrt{1-x}}$:\\
        Sei $\xi (z) = \underset{k=0}{\overset{z-1}{\prod}} 2k+1$, das Produkt aller ungeraden Zahlen
        bis vor die $z$-te ungerade Zahl.
    
        \textbf{Beh.:} $g^{(k)} = \frac{\xi(k)}{2^k} (1-x)^{-\frac{1+2k}{2}}$ \\
        \textbf{I.A.:} \\
            $k=0$ $g^{(0)} = \frac{1}{2^0} (1-x)^{-\frac{1}{2}} = g(x)$\\
        \textbf{I.S.:} $k \rightarrow k+1$\\
            $$\begin{array}{rcl}
                g^{(k+1)} &=& \frac{d}{dx} g^{(k)}\\
                        &\stackrel{I.V.}{=}&
                            \frac{d}{dx} \left(\frac{\xi(k)}{2^{k}} (1-x)^{-\frac{1+2k}{2}} \right)\\
                        &=& \frac{\xi(k)}{2^{k}} \cdot -1 \cdot -\frac{2k+1}{2} (1-x)^{1+\frac{1+2k}{2}}\\
                        &=& \frac{\xi(k) \cdot (2k+1)}{2^{k}\cdot 2} (1-x)^{\frac{2(k+1)}{2}}\\
                        &=& \frac{\xi(k+1)}{2^{k+1}} (1-x)^{\frac{1+2(k+1)}{2}}
            \end{array}$$

        Mit dieser Ableitung können wir nun das Taylorpolynom aufschreiben:\\
        $$\begin{array}{rcl}
            T_n(x) &=& \overset{n}{\underset{k=0}{\sum}} 
                    \frac{f^{(k)}(a)}{k!} (x-a)^k\\
                &=& \overset{n}{\underset{k=0}{\sum}}
                    \frac{1}{k!} \cdot \frac{\xi(k)}{2^k} (1-a)^{-\frac{1+2k}{2}} (x-a)^k\\
                &=& \overset{n}{\underset{k=0}{\sum}}
                    \frac{1}{k!} \frac{\xi(k)}{2^k} 1^{-\frac{1+2k}{2}} x^k\\
                &=& \frac{\xi(k)}{2^k k!} x^k\\
        \end{array}$$
        Wir lassen es an dieser Stelle so stehen. Wir haben einen Faktor, abhängig von $k$
        und sind in der Monomdarstellung.\\
        Man kann sich gerne davon überzeugen, dass 
        $a_k = \frac{\xi(k)}{2^k k!} = \frac{\Gamma(k + \frac{1}{2})}{\sqrt{\pi} k!}$ gilt.
            
    \item $h(x) = xe^x$\\
        \textbf{Beh.:} $h^{(k)}(x) = (x + k)e^x$\\
        \textbf{I.A.:} $k=0$ $h^{(0)} = (x + 0)e^x = xe^x = h(x)$\\
        \textbf{I.S.:} $k \rightarrow k+1$\\
            $$\begin{array}{rcl}
                h^{(k+1)}(x) &=& \frac{d}{dx} h^{(k)}(x)\\
                    &=& \frac{d}{dx} (x + k)e^x\\
                    &=& e^x + (x + k)e^x\\
                    &=& (x+(k+1))e^x
            \end{array}$$
        Nun können wir das Taylorpolynom vom Grad $n$ am Entwicklungspunkt $a=0$ aufstellen.
        $$\begin{array}{rcl}
            T_n(x) &=& \overset{n}{\underset{k=0}{\sum}}
               \frac{h^{(k)}(a)}{k!} (x-a)^k\\
                    &=& \overset{n}{\underset{k=0}{\sum}} \frac{(a+k)e^a}{k!} (x-a)^k\\
                    &=& \overset{n}{\underset{k=0}{\sum}} \frac{k}{k!} x^k\\
                    &=& \overset{n}{\underset{k=0}{\sum}} \frac{1}{(k-1)!} x^k
        \end{array}$$
        Wir haben die Monomdarstellung mit den Koeffizienten $a_k = \frac{1}{(k-1)!}$.

 
\end{enumerate}

\subsection*{Aufgabe 11: \mdseries\itshape Gleichmäßige Konvergenz von Funktionsfolgen}

Bestimmen Sie für die folgenden Funktionsfolgen den punktweisen Limes
$$
    f(x) = \lim_{n \rightarrow \infty} f_n(x)
$$
(falls er existiert) und prüfen Sie, welche der Folgen gleichmäßig konvergiert.

\begin{enumerate}[(i)]
    \item $f_n(x) = e^{-nx^2}$ auf $[-1,1]$.\\ 
        Zunächst bestimmen wir den Limes punktweise:\\
        $$\begin{array}{rcl}
            f(x) &=& \limes{\infty} f_n(x)\\
                &=& \limes{\infty} e^{-nx^2}\\
                &=& \limes{\infty} \left( e^{x^2}\right)^{-n}
        \end{array}$$
        Für $a \geq 1$ gilt $\limes{\infty}\frac{1}{a^n} = 0$,
        Wir wissen, dass $e^x$ streng monoton steigt und bei $x=0$ eins
        erreicht. Daher ist $e^x$ für alle $x>0$ größer als 1.
        Daraus folgt, dass $f_n(x) = 0$ auf $[-1,0)$ und $(0,1]$.\\
        Im Fall $x=0$ ergibt sich $e^{0 \cdot -n} = 1$ für alle $n$.
        Daher ist $f(0) = 1$.\\

        Gleichmäßige Konvergenz TBD.

    \item $g_n(x) = \sqrt{x^2 + \frac{1}{n}}$ auf $[0,\infty).$\\
        Wir wissen, dass die Funktion $\sqrt{.}$ auf dem Interval stetig ist.
        Wir können den Limes also in die Funktion ziehen.\\
        $$\begin{array}{rcl}
            g(x) &=& \limes{\infty} g_n(x)\\
                &=& \limes{\infty} \sqrt{x^2 + \frac{1}{n}}\\
                &=& \sqrt{(\limes{\infty} x^2) + (\limes{\infty} \frac{1}{n})}\\
                &=& \sqrt{x^2 + 0}\\
                &=& x
        \end{array}$$

        Gleichmäßige Konvergenz: TBD.

    \item $h_n(x) = n\left( \sqrt{x + \frac{1}{n}} - \sqrt{x} \right)$\\
        Wir wissen nach Umformung, dass gilt $n = \left(\frac{1}{n} \right)^{-1}$ die
        als inverses Element bezüglich der Multiplikation.
        Nun können wir auf.
        $h_n(x) = \frac{\sqrt{x + \frac{1}{n}} - \sqrt{x}}{\frac{1}{n}}$ den Satz von l'Hopital
        anwenden.
        $$\begin{array}{rcl}
            h(x) &=& \limes{\infty} h_n(x)\\
                &=& \limes{\infty} \frac{\sqrt{x + \frac{1}{n}} - \sqrt{x}}{\frac{1}{n}}\\
                &\stackrel{l'Hopital}{=}& 
                    \limes{\infty} \frac{\frac{1}{2\sqrt{x+\frac{1}{n}}}\cdot-\frac{1}{n^2} 
                    + \frac{1}{2\sqrt{x}}}{-\frac{1}{n^2}}\\
                &=& \limes{\infty} -n^2 \left( - \frac{1}{2\sqrt{x}} 
                    + \frac{1}{2\sqrt{x+\frac{1}{n}}} \cdot \frac{1}{n^2}\right)\\
                &=& \limes{\infty} -n^2 \frac{2\sqrt{x} - 2\sqrt{x+\frac{1}{n} \cdot \frac{1}{n^2}}}
                    {n^2 \cdot 2\sqrt{x} \cdot 2\sqrt{x + \frac{1}{n}}}\\
                &=& \limes{\infty} \frac{2\sqrt{x} - 2\sqrt{x+\frac{1}{n}} \cdot \frac{1}{n^2}}
                    {2\sqrt{x} \cdot 2\sqrt{x + \frac{1}{n}}}\\
                && \text{Alles konergiert nun}\\
                &=& \frac{2\sqrt{x} - 2\sqrt{x+0} \cdot 0}{2\sqrt{x} \cdot 2\sqrt{x+0}}\\
                &=& \frac{1}{2\sqrt{x}}
        \end{array}$$ 
         
        Gleichmäßige Konvergenz: TBD.
    \item $k_n(x) = \arctan(nx)$ auf $[-\infty, \infty]$.\\
        tbd
\end{enumerate}

\subsection*{Aufgabe 12: \mdseries\itshape Gleichmäßige Konvergenz von Reihen}

Untersuchen Sie folgende Funktionsreihen auf gleichmäßige Konvergenz.

\begin{enumerate}[(i)]

    \item $\overset{\infty}{\underset{n=1}{\sum}} \frac{\sin(nx)}{n^\alpha}$ für $x\in\mathbb{R}$ und festes $\alpha > 1$.\\
        Wir schätzen zunächst die einzelnen Folgeglieder ab
        $$\begin{array}{rcl}
            \left| \overset{\infty}{\underset{k=0}{\sum}} \frac{\sin(kx)}{k^\alpha}\right| \\
                &\stackrel{\sin(a)\leq1}{\leq}&  
                    \left| \overset{\infty}{\underset{k=0}{\sum}} \frac{1}{k^\alpha}\right| \\
                &=& \overset{\infty}{\underset{k=0}{\sum}} \frac{1}{k^\alpha} 
        \end{array}$$
        Wir haben nun eine Folge von Zahlen gefunden $M_k = \frac{1}{k^\alpha}$ so dass
        du Funktionen der Reihe alle kleiner sind. Nach dem Weierstrass M-Test, gilt
        also, dass die urpsrüngliche Reihe gleichmäßig konvergiert, wenn 
        $\overset{\infty}{\underset{k=0}{\sum}} M_n$ konvergiert\footnote{Diese Reihe 
        konvergiert in der Tat gegen $\zeta(\alpha)$}.

    \item $\overset{\infty}{\underset{n=1}{\sum}} \frac{x}{n(1+nx^2)}$ für $x\in\mathbb{R}$.\\
        Wir zeigen über eine Fallunterscheidung, dass die Reihe bis auf $x=0$ punktweise
        divergiert. Damit ist klar, dass es nicht gleichmäßig konvergieren kann, da 
        diesen implizieren würde, dass die Reihe punktweise konvergieren würde.\\
        \textbf{Fall 1:} $x>0$\\
        Jeder Summand ist größer als Null. Nun gilt
        $$\begin{array}{crcl}
            & \frac{x}{n(1+nx^2)} &>& \frac{1}{n}\\
            \Leftrightarrow & \frac{x}{1+nx^2} &>& 1
        \end{array}$$
    \item $\overset{\infty}{\underset{n=1}{\sum}} (-1)^n \frac{x^2+n}{n^2}$ für $x\in\mathbb{R}$.\\
        Diese Reihe konvergiert Punktweise, daher kann der Trick von eben nicht angewandt werden.
        Wir zeigen aber, dass die Reihe trotzdem nicht gleichmäßig konvergiert.\\
        $$\begin{array}{rcl}
            \overset{\infty}{\underset{n=1}{\sum}} (-1)^n \frac{x^2 + n}{n^2}
                &=& \overset{\infty}{\underset{n=1}{\sum}} (-1)^n \left( \frac{1}{n} + \frac{x^2}{n^2} \right)\\
                &=& \overset{\infty}{\underset{n=1}{\sum}} (-1)^n \frac{1}{n} +
                    \overset{\infty}{\underset{n=1}{\sum}} (-1)^n \frac{x^2}{n}
        \end{array}$$
        Der erste Summand konvergiert und da er unabhängig von $x$ ist, passiert dies auch
        gleichmäßig. Im zweiten Teil steht nun $x^2$ als Faktor drin. Wir können den
        Wert jetzt also durch die Veränderung von $x$ beliebig groß oder klein machen.
        Wir können also ein $x$ so wählen, dass jede $\varepsilon$ Schranke durchbrochen werden
        kann.\\
        Damit ist die Reihe nicht gleichmäßig stetig.
\end{enumerate}

\label{LastPage}
\end{document}


