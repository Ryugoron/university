\documentclass[11pt,a4paper,ngerman]{article}
\usepackage[bottom=2.5cm,top=2.5cm]{geometry} 
\usepackage{babel}
\usepackage[utf8]{inputenc} 
\usepackage[T1]{fontenc} 
\usepackage{ae} 
\usepackage{amssymb} 
\usepackage{amsmath} 
\usepackage{graphicx}
\usepackage{fancyhdr}
\usepackage{fancyref}
\usepackage{listings}
\usepackage{xcolor}
\usepackage{paralist}

%%\usepackage[pdftex, bookmarks=false, pdfstartview={FitH}, linkbordercolor=white]{hyperref}
\usepackage{fancyhdr}
\pagestyle{fancy}
\fancyhead[C]{Analysis II}
\fancyhead[L]{Aufgabenblatt 4}
\fancyhead[R]{SoSe 2012}
\fancyfoot{}
\fancyfoot[L]{}
\fancyfoot[C]{\thepage \hspace{1px} of \pageref{LastPage}}
\renewcommand{\footrulewidth}{0.5pt}
\renewcommand{\headrulewidth}{0.5pt}
\setlength{\parindent}{0pt} 
\setlength{\headheight}{15pt}

\date{}
\title{Max Wisniewski , Alexander Steen}
\author{Tutor : Adrian Steffens}

\newcommand{\limes}[2][n]{\underset{ #1 \rightarrow #2}{\lim}}

\begin{document}

\lstset{language=Pascal, basicstyle=\ttfamily\fontsize{10pt}{10pt}\selectfont\upshape, commentstyle=\rmfamily\slshape, keywordstyle=\rmfamily\bfseries, breaklines=true, frame=single, xleftmargin=3mm, xrightmargin=3mm, tabsize=2}

\maketitle
\thispagestyle{fancy}

%% ------------------------------------------------------
%%                     Aufgabe 5 (epic fail, max!)
%% ------------------------------------------------------

\subsection*{Aufgabe 13: \mdseries\itshape Fehlerrechnung}
Berechnen Sie die folgenden Werte bis auf einen Fehler von $10^{-4}$.

\begin{enumerate}[(i)]
    \item $\sin (1)$:\\
        Wir berechnen die Taylorformel an dieser Stelle, wie in der Vorlesung von $f$ im Entwicklungspunkt
        $0$. Daher können wir gleichmäßige Konvergenz schon vorraus setzten.\\
        $f^{(4n)} = \sin$, $f^{(4n+1)} = \cos$, $f^{(4n+2)} = -\sin$, $f^{(4n+3)} = -\cos$. Diese
        setzen wir nun in die Taylorreihe ein:
        $$\begin{array}{rcl}
            p(x) &=& \overset{\infty}{\underset{n=0}{\sum}} \frac{f^{(n)}(a)}{n!}(x-a)\\
                &=&  \overset{\infty}{\underset{n=0}{\sum}} \frac{f^{(n)}(0)}{n!}x\\
                &=& \overset{\infty}{\underset{n=0}{\sum}} (-1)^{n}\frac{\cos(0)}{n!}x\\
                &=& \overset{\infty}{\underset{n=0}{\sum}} (-1)^{n}\frac{1}{n!}x\\
        \end{array}$$
        Aus der Vorlesung wissen wir nun, dass diese Reihe für $|x|\leq1$ gleichmäßig konvergiert,
        deshalb müssen wir nur noch das $n$ berechnen, so dass der Fehler klein genug wird,
        daher schätzen wir den Fehler durch das Restglied ab.\\
        $$\begin{array}{rcl}
            R_n(x) &=& \int_{a}^{x} \frac{(x-t)^n}{n!}f^{(n+1)}(t) \, dt\\
                &=&  \int_{a}^{x} \frac{(x-t)^n}{n!}f^{(n+1)}(t) \, dt\\
        \end{array}$$
    \item $e$::\\
        Wir wählen den Entwicklungspunkt $a=0$ und erhalten so die übliche Darstellung der $e$-Funktion.
        $e^{(n)}(x) = e^x$ und somit $e^0 = 1$.\\
        $$
            p(x) = \overset{\infty}{\underset{n=0}{\sum}} \frac{1}{n!}x^n
        $$
        Wie bereits bekannt ist, konvergiert diese Reihe gleichmäßig, wir müssen also nur den Fehler
        abschätzen.
        $$\begin{array}{rcl}
            R_n(x) &=& \int_a^x \frac{(x-t)^n}{n!} f^{(n+1)}(t) \, dt \\
                &=& \int_a^x \frac{(x-t)^n}{n!}e^t \, dt\\
                &=& \left[\frac{(x-t)^{n}}{n!}e^t\right]_a^x - \int_a^x \frac{-(x-t)^{n-1}}{n!}e^t \, dt\\
                &=& \left[\frac{(x-t)^{n}}{n!}e^t\right]_a^x + \left[\frac{(x+t)^{n-1}}{n!}e^t\right]_a^x
                    - \int_a^x \frac{-(x-t)^{n-1}}{n!}e^t \, dt\\
                && \text{Fortsetzen bis $(x-t)^0$ erreicht ist} \\
                &=& \left[ \underset{m=1}{\overset{n}{\sum}} \frac{(x-t)^m}{n!} e^t \right]_a^x 
                    - \int_a^x \frac{e^t}{n!} \, dt\\
                &=& \underset{m=0}{\overset{n}{\sum}} \frac{(x-x)^m}{n!}e^x - \underset{m=0}{\overset{n}{\sum}} \frac{(x)^m}{n!}e^0 - \frac{e^x}{n!} + \frac{e^0}{n!}\\
                &=& 0 - \frac{n+1}{n!} - \frac{e}{n!} + \frac{1}{n!}\\
                &=& - \frac{n+e}{n!}\\
        \end{array}$$
        In unserem Fall landen wir bei $n=9$ bei einer Zahl, die um $10^4$ nah an der Eingabe dran ist.
        Wir können $e = \frac{1}{1!} + \frac{1}{2!} + \frac{1}{3!} + \frac{1}{4!} + \frac{1}{5!}
        + \frac{1}{5!}+ \frac{1}{6!} + \frac{1}{7!} + \frac{1}{8!} + \frac{1}{9!} = 1.7182815$ abschätzen
        und sehen, dass der Fehler erst nach der 4. Nachkommastelle auftritt.
\end{enumerate}

\subsection*{Aufgabe 14: \mdseries\itshape Taylorentwicklung von arctan}

\begin{enumerate}[(i)]
    \item Begründen Sie zunächst, dass
        $$
            \frac{d}{dt} \arctan t = \frac{1}{1 + t^2} = \overset{\infty}{\underset{n=0}{\sum}} (-1)^nt^{2n}
        $$
        für $|t|<1$, für welche $t$ die Reihe gleichmäßig konvergiert.\\
    \textbf{Lösung:}\\
        Wir benutzen das (NAME FEHLT HIER) Kriterium, um zu zeigen, dass die Reihe auf $(-t,t)$ gleichmäßig
        konvergiert. Da hier die Grenzen leider nicht konvergieren, sondern die Reihe 2 Häufungspunkte
        für $t=1$ besitzt untersuchen wir eine $\varepsilon>0$ unter $t=1$.
        $\overset{\infty}{\underset{n=0}{\sum}} (-1)^n(1-\varepsilon)^{2n}$. Wir untersuchen die Reihe
        mit dem Leibnitz - Kriterium. $(1-\varepsilon)^{2n}$ ist ein streng monoton fallende Folge, die
        gegen 0 konvergiert (Die geometrische Reihe mit $(a)_{n\in\mathbb{N}} = aq^n$ mit $0\leq q < 1$
        konvergiert gegen 0).\\
        Wir wissen nun, dass die Reihe für alle $0<t<1$ gleichmäßig konvergiert. Für den negativen Fall
        setzten wir nun $-t$ ein. $\overset{\infty}{\underset{n=0}{\sum}} (-1)^n(-t)^{2n} =
        \overset{\infty}{\underset{n=0}{\sum}} (-1)^n(t)^{2n}$ gilt, da das ganze Symmetrisch in $t$ ist.
        Nun wissen wir, dass die Reihe auf $(-t,0)$ und $(0,t)$ gleichmäßig konvergiert.
        Der letzte Fall ist $t=0$. Aber für $t=0$ ist die Reihe für alle $n$ 0. Somit können wir
        das größte $N$ von allen 3 Bereichen nehmen und sagen, dass die Reihe auf $(-t,t)$ gleichmäßig
        konvergiert.

    \item Bestimmen Sie nun die Taylorreihe von $\arctan$ durch Integration. Zeigen Sie, dass die so
        gefundene Reihe für alle $|x|\leq 1$ konvergiert.\\
    \textbf{Lösung:}\\
        Nach dem Hauptsatz der Differential und Integralrechnung ist
        $$
            \int (\frac{d}{dt} \arctan (t) ) \, dt = \arctan.
        $$
        Nun haben wir für die Ableitung die Reihendarstellung
        $$
            \int (\frac{d}{dt} \arctan (t) ) \, dt 
                = \int \overset{\infty}{\underset{n=0}{\sum}} (-1)^nt^{2n} \, dt
        $$
        Wir wissen aus Aufgabenteil (i), dass die Reihe gleichmäßig konvergiert, also gilt:
        $$\begin{array}{rcl}
            \arctan (t) &=& \overset{\infty}{\underset{n=0}{\sum}} \int (-1)^nt^{2n} \, dt\\
                &=& \overset{\infty}{\underset{n=0}{\sum}} \frac{(-1)^n}{2n+1}t^{2n+1} \\
        \end{array}$$
        Da in diesem letzten Ausdruck kein Integral mehr auftaucht und wir eine gültige
        Reihendarstellung erreicht haben, befinden wir dies als fertige Auflösung des Tangenz.\\
        
        Im Gegensatz zur Ableitung konvergiert diese Darstellung nun auch für $t=1$ gleichmäßig.
        Die gleichmäßige Stetigkeit Argumentiert sich genau, wie bei (i), nur sehen wir diesmal,
        dass in $t=1$ immer noch die Reihe über die Folge $(-1)^n \frac{1}{2n+1}$ stehen bleibt,
        die Nullkonvergent ist und streng monoton fällt. Dammit konvergiert $\arctan$ auf $[-1,1]$ stetig.
        (Für -1 müssen wir dieses mal $(-1)^{n+1}$ bilden).

    \item Begründen Sie abschließend
        $$
            \frac{\pi}{4} = 1 - \frac{1}{3} + \frac{1}{5} - \frac{1}{7} + ...
        $$
        Hierfür sollte man zunächst betrachten, dass $\tan(\frac{\pi}{4}) = 1$ ist. Da $\arctan$ die
        Umkehrabbildung ist, setzten wir die 1 in unsere Reihe ein und erhalten
        $$\begin{array}{rcl}
            \arctan (1) &=& \frac{\pi}{4} \\
                &=& \overset{\infty}{\underset{n=0}{\sum}} \frac{(-1)^n}{2n+1}1^2n\\
                &=& \frac{1}{1} - \frac{1}{3} + \frac{1}{5} - \frac{1}{7} + ...
        \end{array}$$
\end{enumerate}

\subsection*{Aufgabe 15: \mdseries\itshape Schwarz-Ableitung}
Sei $f \; : \; (a - \varepsilon, a +\varepsilon) \longrightarrow \mathbb{R}$ differenzierbar.

\begin{enumerate}[(i)]
    \item Angenommen die zweite Ableitung von $f$ im Punkt $a$ existiert. Zeigen Sie, dass dann
        $$
            f''(a) = \underset{h \rightarrow \infty}{\lim} \frac{f(a+h) + f(a-h) - 2 f(a)}{h^2}
        $$
        gilt.\\
    \textbf{Lösung:}\\
        Um den Satz zu beweisen stellen wir die zweite Ableitung über den Differentialquotient
        der Funktion auf. Dieser existiert nach Vorraussetzung.
        $$
            f'(x) = \underset{h_1 \rightarrow 0}{\lim} \frac{f(x + h_1) - f(x)}{h_1}
        $$
        und
        $$
            f''(x) = \underset{h_2 \rightarrow 0}{\lim} \frac{f'(x + h_2) - f'(x)}{h_2}.
        $$
        Nun setzten wir die erste in die zweite Formel ein
        $$\begin{array}{rcl}
            f''(x) &=& \underset{h_2 \rightarrow 0}{\lim} \frac{f'(x + h_2) - f'(x)}{h_2}\\
                &=& \underset{h_2 \rightarrow 0}{\lim} \underset{h_1 \rightarrow 0}{\lim} 
                \frac{\frac{f(x + h_1 + h_2) - f(x + h_2)}{h_1} - \frac{f(x+h_1) - f(x)}{h_1}}{h_2}\\
                &=& \underset{\underset{h_2 \rightarrow 0}{h_1 \rightarrow 0}}{\lim}
                \frac{f(x + h_1 + h_2) - f(x + h_2) - f(x+h_1) + f(x)}{h_1 \cdot h_2}
        \end{array}$$
        Nun wenden wir unser Wissen über den Differentialquotient an, dass uns sagt, dass wenn die
        Folge konvergiert, dann muss es für jede Nullkonvergente Folge den selben Wert ergeben.
        Daher können wir für $h_1,h_2$ insbesondere die selbe Folge verwenden, eben die Folge,
        die wir in der Schwarzschen Ableitung verwenden werden.
        $$
            f''(x) = \underset{n \rightarrow \infty}{\lim} \frac{f(x + 2h_n) - 2f(x+h_n) + f(x)}{h_n^2}
        $$
        Nun sehen wir uns dagegen die schwarzsche Ableitung an.
        $$
            f^S(x) = \underset{n \rightarrow \infty}{\lim} \frac{f(x+h_n) + f(x-h_n) - 2f(x)}{h_n^2}
        $$
        Die beiden Terme sehen an dieser Stelle schon fast identisch aus. Wir verwenden nun unser Wissen
        dass eine differenzierbare Funktion stetig ist um zu Schlussfolgern, dass die Wert über den
        Bruchstrich den selben Wert ergeben, da wir den Grenzwert in die Funktion ziehen können und
        damit den selben Wert erhalten.
        

    \item Es sei
        $$
            f(x) := \left\{ \begin{array}{lr} x^2 & für \, x\geq 0 \\ -x^2 & für \, x<0\end{array}\right. .
        $$
        Zeigen Sie, dass
        $$
            \underset{h \rightarrow \infty}{\lim} \frac{f(0+h) + f(0-h) - 2f(0)}{h^2}
        $$
        existiert, obwohl $f''(0)$ nicht existiert.\\
    \textbf{Lösung:}\\
        Einfache Umformung aufgrund der Grenzwertrechenregeln. Der zweite Schritt ist unabhängig
        davon, ob $h<0$ ist oder nicht, da wir ohnehin beide Fälle aufsummieren werden.
        $$\begin{array}{rcl}
            f^S(x) &=& \underset{h \rightarrow \infty}{\lim} \frac{f(h) + f(-h) - 2f(0)}{h^2}\\
                &=& \underset{h \rightarrow \infty}{\lim} \frac{h^2 - h^2 -2\cdot 0^2}{h^2}\\
                &=& \underset{h \rightarrow \infty}{\lim} \frac{h^2}{h^2} - \frac{h^2}{h^2}\\
                &=& 1 - 1\\
                &=& 0
        \end{array}$$
        Die zweite schwarzsche Ableitung gibt uns also einen Grenzwert.

    \item Angenommen $f$ habe ein lokales Maximum in $a$ und die zweite Schwarz'sche Ableitung von $f$
        in $a$ exisitert. Zeigen Sie, dass diese dann kleiner oder gleich 0 sein muss.\\
    \textbf{Lösung:}\\
        tbd
\end{enumerate}

\label{LastPage}
\end{document}


