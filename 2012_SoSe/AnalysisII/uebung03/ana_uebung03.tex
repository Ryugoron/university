\documentclass[11pt,a4paper,ngerman]{article}
\usepackage[bottom=2.5cm,top=2.5cm]{geometry} 
\usepackage{babel}
\usepackage[utf8]{inputenc} 
\usepackage[T1]{fontenc} 
\usepackage{ae} 
\usepackage{amssymb} 
\usepackage{amsmath} 
\usepackage{graphicx}
\usepackage{fancyhdr}
\usepackage{fancyref}
\usepackage{listings}
\usepackage{xcolor}
\usepackage{paralist}

\usepackage[pdftex, bookmarks=false, pdfstartview={FitH}, linkbordercolor=white]{hyperref}
\usepackage{fancyhdr}
\pagestyle{fancy}
\fancyhead[C]{Analysis II}
\fancyhead[L]{Aufgabenblatt 3}
\fancyhead[R]{SoSe 2012}
\fancyfoot{}
\fancyfoot[L]{}
\fancyfoot[C]{\thepage \hspace{1px} of \pageref{LastPage}}
\renewcommand{\footrulewidth}{0.5pt}
\renewcommand{\headrulewidth}{0.5pt}
\setlength{\parindent}{0pt} 
\setlength{\headheight}{0pt}

\date{}
\title{Max Wisniewski , Alexander Steen}
\author{Tutor : Adrian Steffens}

\begin{document}

\lstset{language=Pascal, basicstyle=\ttfamily\fontsize{10pt}{10pt}\selectfont\upshape, commentstyle=\rmfamily\slshape, keywordstyle=\rmfamily\bfseries, breaklines=true, frame=single, xleftmargin=3mm, xrightmargin=3mm, tabsize=2}

\maketitle
\thispagestyle{fancy}

%% ------------------------------------------------------
%%                     Aufgabe 5
%% ------------------------------------------------------

\subsection*{Aufgabe 9: \mdseries\itshape Unbestimmte Integrale II}

Bestimmen Sie die folgenden Integrale:

\begin{enumerate}[a.]
    
    \item
        $$
            \int \frac{\log ( \log (x))}{x} \, dx
        $$

    \item
        $$
            \int \sin^3 (x) \, dx
        $$

    \item
        $$
           \int \left( \arcsin (x) \right)^2 \, dx
        $$

    \item
        $$
            \int \frac{2x^2 + x + 1}{(x+3)(x-1)} \, dx
        $$

\end{enumerate}

%% ------------------------------------------------------------
%%                      AUFGABE 2
%% ------------------------------------------------------------

\subsection*{Aufgabe 10 : \mdseries\itshape Identitäten}

Bei der Substituition von $t = \tan \left(\frac{x}{2} \right)$ gelten die folgenden
identitäten.

\begin{enumerate}[a.]
    \addtocounter{enumi}{4}
    \item
        $$
            \sin (x) = \frac{2t}{1+t^2}
        $$
    \item
        $$
            \cos (x) = \frac{1-t^2}{1+t^2}
        $$

\end{enumerate}
%% ------------------------------------------------------------
%%                  AUFGABE 3
%% ------------------------------------------------------------
        
\subsection*{Aufgabe 11 : \mdseries\itshape Trigonometrische Integrale}

\begin{enumerate}[a.]
    \addtocounter{enumi}{6}
    \item
        $$
            \int \frac{1}{1+\sin (x)} \, dx
        $$
    
    \item
        $$
            \int \frac{1}{3 + 5 \sin (x)} \, dx
        $$
\end{enumerate}


\label{LastPage}
\end{document}


