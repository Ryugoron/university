\documentclass[11pt,a4paper,ngerman]{article}
\usepackage[bottom=2.5cm,top=2.5cm]{geometry} 
\usepackage{babel}
\usepackage[utf8]{inputenc} 
\usepackage[T1]{fontenc} 
\usepackage{ae} 
\usepackage{amssymb} 
\usepackage{amsmath} 
\usepackage{graphicx}
\usepackage{fancyhdr}
\usepackage{fancyref}
\usepackage{listings}
\usepackage{xcolor}
\usepackage{paralist}

\usepackage[pdftex, bookmarks=false, pdfstartview={FitH}, linkbordercolor=white]{hyperref}
\usepackage{fancyhdr}
\pagestyle{fancy}
\fancyhead[C]{Analysis II}
\fancyhead[L]{Aufgabenblatt 3}
\fancyhead[R]{SoSe 2012}
\fancyfoot{}
\fancyfoot[L]{}
\fancyfoot[C]{\thepage \hspace{1px} of \pageref{LastPage}}
\renewcommand{\footrulewidth}{0.5pt}
\renewcommand{\headrulewidth}{0.5pt}
\setlength{\parindent}{0pt} 
\setlength{\headheight}{15pt}

\date{}
\title{Max Wisniewski , Alexander Steen}
\author{Tutor : Adrian Steffens}

\begin{document}

\lstset{language=Pascal, basicstyle=\ttfamily\fontsize{10pt}{10pt}\selectfont\upshape, commentstyle=\rmfamily\slshape, keywordstyle=\rmfamily\bfseries, breaklines=true, frame=single, xleftmargin=3mm, xrightmargin=3mm, tabsize=2}

\maketitle
\thispagestyle{fancy}

%% ------------------------------------------------------
%%                     Aufgabe 5
%% ------------------------------------------------------

\subsection*{Aufgabe 9: \mdseries\itshape Unbestimmte Integrale II}

Bestimmen Sie die folgenden Integrale:

\begin{enumerate}[a.]
    
    \item
        $$
            \int \frac{\log ( \log (x))}{x} \, dx.
        $$
        Wir substituieren $y=(\log x)$, weil die Ableitung $\frac{1}{x}$
        schon im Term steht.
        $$\begin{array}{rcl}
            \int \frac{\log ( \log \, x)}{x} \, dx
                &\stackrel{sub. y}{=}& \int \log y \, dy\\
                &=& y \cdot \log \, y - \int y \cdot \frac{1}{y} \, dy\\
                &=& y \log \, y - y\\
                &\stackrel{resub. x}{=}& (\log \, x)(\log (\log \, x)) - (\log \, x)\\
                &=& (\log \, x)(\log(\log \, x) - 1)
        \end{array}$$

        Zum testen noch einmal ableiten:
        $$\begin{array}{rcl}
            \frac{d}{dx} (\log \, x)(\log(\log \, x) - 1)
                &=& \frac{1}{x}\cdot(\log(\log \, x) - 1) + (\log \, x)(\frac{1}{\log \, x} \cdot \frac{1}{x})\\
                &=& \frac{\log(\log \, x)}{x} - \frac{1}{x} + \frac{1}{x}\\
                &=& \frac{\log(\log \, x)}{x}
        \end{array}$$
        \mbox{} \hfill $\square$
    \item
        $$
            \int \sin^3 (x) \, dx
        $$
        Wir berechnen zunächst das Integral von $\cos^2(x)\sin(x)$ um es in der nächsten
        Formel benutzen zu können.
        $$\begin{array}{rcl}
            \int \cos^2(x)\sin(x) \, dx &\stackrel{y=\cos(x)}{=}&
                \int -y^2 \, dy\\
                &=& -\frac{1}{3} y^3\\
                &\stackrel{resub.}{=}& -\frac{1}{3} \cos^3(x)
        \end{array}$$
        Nun wenden wir uns dem Integral zu lösen es durch partielle Integration auf.
        $$\begin{array}{rcl}
            \int \sin^3 x \, dx &=& \int \sin(x) \sin^2(x) \, dx\\
                &=& -\cos(x)\sin^2(x) - \int -\cos(x)\cdot 2 \sin(x) \cos(x) \, dx\\
                &=& - \cos(x)\sin^2(x) + 2 \int \cos^2(x)\sin(x) \, dx \\
                &=& - \cos(x)\sin^2(x) - \frac{2}{3} \cos^3(x)\\
        \end{array}$$
        Das könnten wir nun noch Umformen, aber wir haben nun schon einmal ein Integral.

        Dieses testen wir noch durch ableiten:
        $$\begin{array}{rcl}
               \frac{d}{dx} -\cos(x)\sin^2(x) - \frac{2}{3}\cos^3(x)
                    &=& \sin(x)\sin^2(x) - \cos(x)\cdot 2\sin(x)\cos(x) - \frac{2}{3} 3 \cos^2(x) (-\sin(x))\\
                    &=& \sin^3(x) - 2\cos^2(x)\sin(x) + 2 \cos^2(x)\sin(x)\\    
                    &=& \sin^3(x) 
        \end{array}$$
        \mbox{}\hfill $\square$
    \item
        $$
           \int \left( \arcsin (x) \right)^2 \, dx
        $$
        Partielle integration, indem eine 1 multipliziert wird.
        $$\begin{array}{rcl}
            \int \arcsin^2 (x) \, dx &=& \int \arcsin^2(x) \cdot 1 \, dx\\
                &\stackrel{parts}{=}& 
                    x \cdot \arcsin^2(x) + \int x \cdot 2\arcsin(x)\cdot \frac{1}{\sqrt{1-x^2}}\\
                &\stackrel{parts}{=}&
                    x\arcsin^2(x) + 2 \arcsin(x)\sqrt{1-x^2} - 2 \int 1 \, dx\\
                &=& x\arcsin^2(x) + 2\arcsin(x)\sqrt{1-x^2} - 2x 
        \end{array}$$
        Das Ergebnis testen wir nocheinmal durch ableiten:
        $$\begin{array}{rcl}
            \frac{d}{dx} \left( x\arcsin^2(x) + 2\arcsin(x)\sqrt{1-x^2} - 2x\right)
                &=& \arcsin^2(x) + x(\frac{d}{dx} \arcsin^2(x)) + 2\sqrt{1-x^2}\\
                &&    +(\frac{d}{dx} \arcsin(x)) + 2\arcsin(x) (\frac{d}{dx} \sqrt{1-x^2}) - 2\\
                &=& \arcsin^2(x) + x(2\arcsin(x)\frac{1}{\sqrt{1-x^2}} + 2\sqrt{1-x^2}\frac{1}{\sqrt{1-x^2}}\\
                &&    + 2 \arcsin(x) (-\frac{x}{\sqrt{1-x^2}}) - 2\\
                &=& \arcsin^2(x)
        \end{array}$$
        \mbox{} \hfill $\square$
    \item
        $$
            \int \frac{2x^2 + x + 1}{(x+3)(x-1)} \, dx
        $$
        Wir machen zunächst eine Generalbruchzerlegung und zeigen, dass diese stimmt.
        \begin{description}
            \item{\bfseries\rmfamily Behauptung:}
                $$
                    \frac{2x^2 + x + 1}{(x+3)(x-1)} = \frac{x-1}{x+3} + \frac{x}{x-1}
                $$
            \item{\bfseries\rmfamily Beweis:}
                $$\begin{array}{rcl}
                    \frac{x+4}{x-3} + \frac{x-1}{x-1}
                        &=& \frac{(x-1)(x-1) + (x)(x+3)}{(x+3)(x-1)}\\
                        &=& \frac{x^2 - 2x + 1 + x^2 + 3x}{(x+3)(x-1)}\\
                        &=& \frac{2x^2  + x + 1}{(x+3)(x-1)}
                \end{array}$$
        \end{description}
        Diese Zerlegung benutzen wir nun, um das Integral aufsummiert zu berechnen.\\
        $$\begin{array}{rcl}
            \int \frac{2x^2 + x + 1}{(x+3)(x-1)} \, dx &=& \int \frac{x-1}{x+3} + \frac{x}{x-1} \, dx\\
                &=& \int \frac{x-1}{x+3} \, dx + \int \frac{x}{x-1} \, dx \\
                &=& \int \frac{x+3-4}{x+3} \, dx + \int \frac{x-1+1}{x-1} \, dx\\
                &=& \int 1 \, dx - 4\int \frac{1}{x+3} \, dx + \int 1 \, dx + \int \frac{1}{x-1} \, dx\\
                &=& x - 4 \log (x+3) + x + \log(x-1)\\
                &=& 2x - 4\log(x+3) + \log(x-1)
        \end{array}$$
        Dieses Ergebniss leiten wir zum testen nocheinmal ab.\\
        $$\begin{array}{rcl}
            \frac{d}{dx} \left( 2x - 4\log(x+3) + \log(x-1) \right)
                &=& 2 - \frac{4}{x+3} - \frac{1}{x-1}\\
                &=& \frac{2(x+3)(x-1)}{(x+3)(x-1)} - \frac{4(x-1)}{(x+3)(x-1)} + \frac{x+3}{(x+3)(x-1)}\\
                &=& \frac{2x^2 + 4 x - 6 - 4x + 4 + x  + 3}{(x+3)(x-1)}\\
                &=& \frac{2x^2 + x + 1}{(x+3)(x-1)}
        \end{array}$$
        \mbox{}\hfill $\square$

\end{enumerate}

%% ------------------------------------------------------------
%%                      AUFGABE 2
%% ------------------------------------------------------------

\subsection*{Aufgabe 10 : \mdseries\itshape Identitäten}

Bei der Substituition von $t = \tan \left(\frac{x}{2} \right)$ gelten die folgenden
identitäten.

\begin{enumerate}[a.]
    \addtocounter{enumi}{4}
    \item
        $$
            \sin (x) = \frac{2t}{1+t^2}
        $$
        Wir setzen, dass für $t$ auf der linken Seite den substituierten Term ein.
        $$\begin{array}{rcl}
            \frac{2t}{1+t^2} &\stackrel{sub. t}{=}& \frac{2\tan\frac{x}{2}}{1+(\tan \frac{x}{2})^2}\\
                &\stackrel{y=\frac{x}{2}}{=}&
                    \frac{2\tan (y)}{1+\tan^2 (y)}\\
                &=& \frac{2 \frac{\sin(y)}{\cos(y)}}{1+ \frac{\sin^2(y)}{\cos^2 (y)}}\\
                &=& 2\frac{\sin (y) \cos^2 (y)}{\cos(y) \cdot (\cos^2(y) + \sin^2(y))}\\
                &=& 2 \sin(y)\cos(y)\\
                &\stackrel{resub.}{=}& 2 \sin(\frac{x}{2})\cos(\frac{x}{2})\\
                &=& \sin(\frac{x}{2})\cos(\frac{x}{2}) + \sin(\frac{x}{2})\cos(\frac{x}{2})\\
                &\stackrel{Add.Thm}{=}& \sin(\frac{x}{2} + \frac{x}{2})\\
                &=& \sin(x)
        \end{array}$$
        \mbox{}\hfill $\square$
    \item
        $$
            \cos (x) = \frac{1-t^2}{1+t^2}
        $$
        Wir setzen wiederum links den Term ein und ersetzten $y=\frac{x}{2}$ gleich.\\
        $$\begin{array}{rcl}
            \frac{1-t^2}{1+t^2} &=& \frac{1- \tan^2(y)}{1+\tan^2(y)}\\
                &=& \frac{1-\frac{\sin^2(y)}{\cos^2(y)}}{1+\frac{\sin^2(y)}{\cos^2(y)}}\\
                &=& \frac{(\cos^2(y)-\sin^2(y)) \cdot \cos^2(y)}{\cos^2(y) \cdot (\sin^2(y) + \cos^2(y))}\\
                &=& \cos^2(y) - \sin^2(y)\\
                &=& \cos(y)\cos(y) - \sin(y)\sin(y)\\
                &\stackrel{Add.Thm.}{=}& \cos( y + y)\\
                &\stackrel{resub.}{=}& \cos(x)
        \end{array}$$
        \mbox{} \hfill $\square$

\end{enumerate}
%% ------------------------------------------------------------
%%                  AUFGABE 3
%% ------------------------------------------------------------
        
\subsection*{Aufgabe 11 : \mdseries\itshape Trigonometrische Integrale}

\begin{enumerate}[a.]
    \addtocounter{enumi}{6}
    \item
        $$
            \int \frac{1}{1+\sin (x)} \, dx
        $$
        Wir benutzen zunächst die vorgeschlagene ersetzung und formen danach weiter um.
        $$\begin{array}{rcl}
            \int \frac{1}{1+\sin(x)} \, dx
                &\stackrel{sub.}{=}& \int \frac{1}{1+\frac{2t}{t^2 + 1}} \frac{2}{t^2 +1} \, dt\\
                &=& \int \frac{2\cdot (t^2 + 1)}{(t^2 + 1)(t^2 +2t + 1)} \, dt\\
                &=& 2 \int \frac{1}{(t+1)^2} \, dt\\
                &=& 2 (-\frac{1}{t+1})\\
                &=& -\frac{2}{\tan(\frac{x}{2}) + 1}\\
        \end{array}$$
        Dies ist nun auf jedenfall ein Integral von $\frac{1}{1+\sin(x)}$. Dies könnte man noch vereinfachen
        ist aber nicht gefragt. 
    \item
        $$
            \int \frac{1}{3 + 5 \sin (x)} \, dx
        $$
        
\end{enumerate}


\label{LastPage}
\end{document}


