\subsection*{Aufgabe 26: \mdseries\itshape Wegzusammenhang}
Eine Menge $A \subset \mathbb{R}^n$ heißt \emph{wegzusammenhängend}, wenn es für je zwei Punkte
$x,y \in A$ eine stetige Funktion $\gamma \; : \; [0,1] \rightarrow A$ gibt, mit $\gamma (0) = x$
und $\gamma (1) = y$. Man nennt $\gamma$ einen \emph{stetigen Weg von x nach y}.
\begin{enumerate}[(i)]
    \item Seien $f \; : \; A \subset \mathbb{R}^n \rightarrow \mathbb{R}^m$ stetig und $A$ 
        wegzusammenhängend. Zeigen Sie, dass dann auch $f(A)$ wegzusammenhängend ist.\\
    \textbf{Beweis:}\\
        tbd

    \item Zeige Sie, dass genau dann $A \subset \mathbb{R}$ wegzusammenhängend ist, wenn $A$ 
        ein Intervall ist, d.h. wenn für alle $x,y \in A, \; x\leq y, \;[x,y] \subset A$.\\
    \textbf{Beweis:}\\
        tbd

    \item Können Sie den bekannten Zwischenwertsatz aus der Analysis I auch auf Funktionen
        $f \; : \; A \subset \mathbb{R}^n \rightarrow \mathbb{R}$ verallgemeinern.\\
    \textbf{Beweis:}
        tbd
\end{enumerate}
