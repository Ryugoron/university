\subsection*{Aufgabe 24: \mdseries\itshape Stetige Abbildungen auf Punktmengen}
\begin{enumerate}[(i)]
    \item Sei $f \; : \; \mathbb{R}^n \rightarrow \mathbb{R}^m$ stetig. Zeigen Sie, dass für jede
        Menge $ A \subset \mathbb{R}^n$ gilt
        $$
            f(\overline{A}) \subset \overline{f(A)}.
        $$
    \textbf{Beweis:}\\
        Wir benutzten für den Beweis das Folgenkonvergenz kriterium für Abgeschlossene Menge.\\
        D.h. wenn $B$ eine abgeschlossene Menge ist muss
        für jede Folge $(x)_{k\in \mathbb{N}}$ aus $B$,
        $\left( \underset{n \rightarrow \infty}{\lim} x_k \right) \in B$ gelten.\\

        Nun gilt, aber, für jeden Punkt $x_0 \in \overline{A}$, dass es eine Folge
        $(x)_{n \in \mathbb{N}}$ gibt mit $\limes{n,\infty} x_k = x_0$.
        $$\begin{array}{rcl}
            f(x_0) &=& f(\underset{n \rightarrow \infty}{\lim} x_n)\\
                &=& \underset{n \rightarrow \infty}{\lim} f(x_n)\\
        \end{array}$$

        Wir haben eine Folge von Bildern der Funktion. Wir wissen, dass in einer abgeschlossenen
        Menge jede konvergente Folge gegen einen Punkt innerhalb der Menge konvergiert.
        Da $f(x_0)$ eine konvergente Folge $f(x_k)$ besitzt, da die $x_k$ konvergieren und $f$ stetig
        ist, muss das Bild des Abschluss des Quellbereiches  auch im Abschluss des Bild liegen.\\
        \mbox{} \hfill $\square$

    \item Ist das stetige Bild $f(M)$ einer beliebigen offenen bzw. abgeschlossenen Menge 
        $M \subset \mathbb{R}^n$ wieder offen bzw. abgeschlossen? Geben Sie ein Beispiel an.\\
    \textbf{Lösung:}\\
        (1) Sei $M \subset \mathbb{R}^n$ offen, und sei $f \equiv c \in \mathbb{R}^m$. \\
            Dann ist $f$ stetig und $f(M) = \{c\}$. Die Menge $\{c\}$ ist aber nicht offen, da für alle $\varepsilon > 0$ die Kugel $B_\varepsilon(c)$ nicht Teilmenge von $\{c\}$ ist.
\\
        (2) Sei $M \subset \mathbb{R}$ mit $M = (-\infty, 0]$ und $f: \mathbb{R} \to \mathbb{R}$, mit $x \mapsto e^x$ (also hier: $n = 1, m = 1$). \\
            Dann ist M abgeschlossen, da für alle konvergenten Folgen $(x_k)$, mit $x_k \in M$ gilt: $\lim_{k \to \infty} x_k = \alpha < \infty$ und $\alpha \geq 0$ und damit $\alpha \in M$.

            Es gilt weiterhin: $f(M) = f((-\infty, 0]) = (0,1]$ wobei $(0,1]$ nicht abgeschlossen ist in $\mathbb{R}$.\\

      $\Rightarrow$ stetiger Bilder von beliebigen offenen bzw. abgeschlossenen Mengen 
      müssen nicht wieder offen bzw. abgeschlossen sein.
    \newpage
    \item Sei $ f \; : \; \mathbb{R}^n \rightarrow \mathbb{R}^m$ stetig. Erfülle $M \subset \mathbb{R}^n$
        die Heine-Borell-Eigenschaft. Dann erfüllt $f(M)$ diese Eigenschaft auch.\\
    \textbf{Lösung:}\\
        Sei $\bigcup_{i \in I} U_i$ eine offene Überdeckungvon $f(M)$, mit $I$ Indexmenge. \\
        Aus der Vorlesung wissen wir, dass für eine stetige Funktion 
        $f: \mathbb{R}^n \to \mathbb{R}^m$ gilt: $f^{-1}(V)$ ist offen in $\mathbb{R}^n$,
        für alle offenen Mengen $V \subseteq \mathbb{R}^m$. \\

        Also folgt aus der Stetigkeit von $f$: 
        $V_i := f^{-1}(U_i)$ ist offen, $i \in I$.

        Da $M \subseteq \bigcup_{i \in I}V_i$ und $M$ die Heine-Borell-Eigenschaft erfüllt,
        existieren $i_1, i_2, ..., i_k \in I, k \in \mathbb{N}$ mit
        $M \subseteq \bigcup_{j = 1}^{k} V_{i_j}$.
        Daraus folgt $f(M) \subseteq f(\bigcup_{j = 1}^{k} V_{i_j})
        = \bigcup_{j = 1}^{k} f(V_{i_j})$. Durch Einsetzen erhalten wir dann
        $ f(M) \subseteq \bigcup_{j = 1}^{k} f(f^{-1}(U_{i_j})) = \bigcup_{j = 1}^{k} U_{i_j} $.

        Also existiert eine endliche offene Überdeckung von $f(M)$ \\
        $\Rightarrow$ $f(M)$ erfüllt die Heine-Borell-Eigenschaft.
        \mbox{} \hfill $\square$
\end{enumerate}

