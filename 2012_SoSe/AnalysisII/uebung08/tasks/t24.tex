\subsection*{Aufgabe 24: \mdseries\itshape Stetige Abbildungen auf Punktmengen}
\begin{enumerate}[(i)]
    \item Sei $f \; : \; \mathbb{R}^n \rightarrow \mathbb{R}^m$ stetig. Zeigen Sie, dass für jede
        Menge $ A \subset \mathbb{R}^n$ gilt
        $$
            f(\overline{A}) \subset \overline{f(A)}.
        $$
    \textbf{Beweis:}\\
        Wir benutzten für den Beweis das Folgenkonvergenz kriterium für Abgeschlossene Menge.\\
        D.h. wenn $B$ eine abgeschlossene Menge ist muss
        für jede Folge $(x)_{k\in \mathbb{N}}$ aus $B$,
        $\left( \underset{n \rightarrow \infty}{\lim} x_k \right) \in B$ gelten.\\

        Nun gilt, aber, für jeden Punkt $x_0 \in \overline{A}$, dass es eine Folge
        $(x)_{n \in \mathbb{N}}$ gibt mit $\limes{n,\infty} x_k = x_0$.
        $$\begin{array}{rcl}
            f(x_0) &=& f(\underset{n \rightarrow \infty}{\lim} x_n)\\
                &=& \underset{n \rightarrow \infty}{\lim} f(x_n)\\
        \end{array}$$

        Wir haben eine Folge von Bildern der Funktion. Wir wissen, dass in einer abgeschlossenen
        Menge jede konvergente Folge gegen einen Punkt innerhalb der Menge konvergiert.
        Da $f(x_0)$ eine konvergente Folge $f(x_k)$ besitzt, da die $x_k$ konvergieren und $f$ stetig
        ist, muss das Bild des Abschluss des Quellbereiches  auch im Abschluss des Bild liegen.\\
        \mbox{} \hfill $\square$

    \item Ist das stetige Bild $f(M)$ einer beliebigen offenen bzw. abgeschlossenen Menge 
        $M \subset \mathbb{R}^n$ wieder offen bzw. abgeschlossen? Geben Sie ein Beispiel an.\\
    \textbf{Lösung:}\\
        tbd

    \item Sei $ f \; : \; \mathbb{R}^n \rightarrow \mathbb{R}^m$ stetig. Erfülle $M \subset \mathbb{R}^n$
        die Heine - Borell - Eigenschaft. Dann erfüllt $f(M)$ diese Eigenschaft auch.\\
    \textbf{Lösung:}\\
        tbd
\end{enumerate}

