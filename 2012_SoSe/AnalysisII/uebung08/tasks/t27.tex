\subsection*{Aufgabe 27 \mdseries\itshape Stetigkeit der Umkehrfunktion}

\begin{enumerate}[(i)]
   \item Zeigen Sie, dass die Funktion 
        $$\begin{array}{rcl}
            f \; : \; (-1,1) &\longrightarrow& \mathbb{R}\\
                x & \mapsto & \frac{x}{1 - x^2}
        \end{array}$$
        einen Homöomorphismus von $(0,1)$ nach $\mathbb{R}^+$ definiert, d.h. 
        $f$ ist invertierbar zwischen den angegebenen Mengen und sowohl $f$ als auch
        $f^{-1}$ ist stetig.\\

    \textbf{Beweis.:}\\
        Sei $g : (0,1) \to \mathbb{R}^+$, mit $x \mapsto \frac{x}{1 - x^2}$.\\
        Z.z.: (1) $g$ bijektiv, (2) $g$ stetig, (3) $g^{-1}$ stetig. \\

        (1) $g$ bijektiv \\
            (1a) $g$ injektiv, also $g(a) = g(b) \Rightarrow a = b$, für alle $a,b \in (0,1)$. \\
                Seien $a,b \in (0,1)$. Dann gilt:
                \begin{eqnarray*}
                  g(a) &=& g(b) \\
                  \Leftrightarrow \frac{a}{1 - a^2} &=& \frac{b}{1 - b^2} \\
                  \Leftrightarrow a - ab^2 &=& b - ba^2 \\
                  \Leftrightarrow a &=& b
                \end{eqnarray*}
%                $\Rightarrow g$ injektiv. \\
            (1b) $g$ surjektiv, also $\forall c \in \mathbb{R}^+ \exists x \in (0,1):\, g(x) = c$.\\
                Sei $c \in \mathbb{R}^+$. Wähle $x := \frac{\sqrt{1+4c^2}-1}{2c}$. \\
                Dann gilt $x>0$, da $c>0$ ist steht im Zähler immer eine positive Zahl ($\sqrt{1 + x} \geq 1$ für $x > 0$),
		und der Nenner $2c > 0$.\\

		Es gilt darüber hinaus, dass $x < 1$, da
		$$\begin{array}{crcl}
			& \frac{\sqrt{1+4c^2} -1}{2c} &<& 1\\
			\Leftrightarrow & \sqrt{1+4c^2} -1 &<& 2c\\
			\Leftrightarrow & \sqrt{1+4c^2} &<& 2c +1\\
			\stackrel{>0 (siehe \, eben)}\Leftrightarrow &
				 1 + 4c^2 &<& 4c^2 + 4c + 1\\
			\Leftrightarrow & 0 & < & 2c
		\end{array}$$
		Und dies gilt, da $c > 0$ ist.
%                $\Rightarrow g$ surjektiv. \\
                $\Rightarrow g$ bijektiv. \\

\pagebreak

        (2) $g$ stetig \\
            Da die Funktion im Nenner $1-x^2$ stetig ist und auf dem Intervall $(0,1)$ keine Nullstellen hat, sowie $x$ stetig ist,
	ist nach Sätzen aus Ana I der Quotient aus den beiden wieder eine stetige Funktion.

        (3) $g^{-1}$ stetig\\
            Wie schon gesagt, ist die Umkehrfunktion auf diesem Intervall definiert durch $g^{-1}(c) = \frac{\sqrt{1+4c^2}-1}{2c}$.
	Nun ist sowohl der Zähler als auch der Nenner eine stetige Funktion. Die einzigen Stellen, an denen diese Funktion nun nicht per
	se stetig ist, sind die Nullstellen des Neners. Die Nullstelle des Nenners liegt aber nun bei $c = 0 \not\in \mathbb{R}^+$. Damit
	enthält die Funktion im Quellbereich keine Nullstelle und die Umkerhfunktion ist stetig.
    
    \item Sei die Funktion $f \; : \; [0,1) \cup [2,3] \rightarrow [0,2]$ gegeben durch
        $$
            f(x) = \left\{ \begin{array}{lr}
                x &, x\in [0,1)\\
                x-1 &, x \in [2,3]
            \end{array} \right. .
        $$
        Zeigen Sie, dass $f$ stetig und invertierbar ist, aber die Umkehrfunktion 
        $f^{-1} \; : \; [0,2] \rightarrow [0,1) \cup [2,3]$ nicht stetig ist.\\

    \textbf{Beweis:}\\
        \begin{enumerate}[a)]
		\item $f$ ist stetig:\\
			Die Funktion $f$ ist stetig. Nehmen wir an, es wäre nicht so, dann müsste es einen Punkt
			$x_0$ geben, gegen den eine Folge von Urbildern konvergiert, aber die Folge der Bilder konvergiert
			nicht gegen $f(x_0)$.\\
			Nun sehen wir, dass eine Folge von Urbildern, ab einer gewissen $\varepsilon >0$ Umgebung, entweder in
			$[0,1)$ oder in $[2,3]$ liegen müssen, da sonst das $\varepsilon > 1$ sein müsste. (Nach $\varepsilon-\delta$ unmöglich.)\\

			Wir können also du beiden Bereiche $f_1 = f|_{[0,1)}$ und $f_2 = f|_{[2,3]}$ getrennt untersuchen.\\
			Dabei handelt es sich nun aber um 2 stetige Funktionen, da $f_1(x) = x$ und $f_2(x) = x-1$ ist.\\

			Damit kann es keinen Punkt geben, bei der die Folge der Bilder nicht gegen das Bild des Grenzwertes der Urbilder
			konvergiert.

\pagebreak

		\item $f$ ist invertierbar:\\
			Sei $g : [0,2] \rightarrow [0,1) \cup [2,3]$ definiert durch
			$$
				g(x) = \left\{\begin{array}{lr}
					x &, x\in [0,1)\\
					x+1 &, x \in [1,2]
				\end{array} \right.
			$$
			Wir zeigen, dass $a = (f \circ g) = id_{[0,2]}$ ist und $b = (g \circ f) = id_{[0,1) \cup [2,3]}$.\\

			Betrachten wir zunächst $a$.\\
			Sei zunächst $x \in [0,1)$, dann ist $g(x) = x \in [0,1)$. Wenden wir nun $f$ an erhalten wir $f(g(x)) = x \in [0,1)$.
			Dieser Fall geht also schonmal auf.\\

			Sei als nächstes $x \in [1,2]$, dann ist $g(x) = x+1 \in [2,3]$. Wenden wir nun $f$ an erhalten wir $f(g(x)) = x + 1 - 1 = x \in [1,2]$.\\

			Also ist $(f \circ g) = id_{[0,2]}$.\\

			Betrachten wir nun $b$.\\
			Sei zunächst $x \in [0,1)$, dann ist $f(x) = x \in [0,1)$ und $g(f(x)) = x \in [0,1)$.\\
			Als nächstes ist $x \in [2,3]$, dann ist $f(x) = x-1 \in [1,2]$ und $g(f(x)) = x -1 + 1 = x \in [2,3]$.\\

			Also ist auch $(f \circ g) = id_{[0,1) \cup [2,3]}$.\\

			Diese beiden Bedingungen reichen aus, um zu zeigen, dass $g$ die Umkehrfunktion von $f$ ist. Also $f$ bijektiv
			also insbesondere auch invertierbar.

		\item $g$ ist nicht stetig:\\
			Betrachten wir den Punkt $x = 1$. Der linksseitige Grenzwert, gebildet aus $x < 1$ ist, da es sich um die Identität handelt
			$\underset{n \rightarrow \infty}{\lim} x_k = x \Rightarrow f ( \underset{n \rightarrow \infty}{\lim} ) = x$, da jedes $x \in [0,1)$ liegt.\\

			Der rechtseitige Grenzwert, mit einer Folge von $x_k > 1$ für alle $k$, ist aber nun 
			$f ( \underset{n \rightarrow \infty}{\lim} x_k ) = x +1$.\\

			Der linksseitige und rechtseitige Grenzwert in $x=1$ ist unterschiedlich, existiert im allgemeinen als einfacher Grenzwert also nicht
			daher ist die Funktion auch $[0,2]$ nicht stetig.
	\end{enumerate}
\end{enumerate}
