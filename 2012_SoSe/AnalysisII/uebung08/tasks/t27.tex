\subsection*{Aufgabe 27 \mdseries\itshape Stetigkeit der Umkehrfunktion}

\begin{enumerate}[(i)]
   \item Zeigen Sie, dass die Funktion 
        $$\begin{array}{rcl}
            f \; : \; (-1,1) &\longrightarrow& \mathbb{R}\\
                x & \mapsto & \frac{x}{1 - x^2}
        \end{array}$$
        einen Homöomorphismus von $(0,1)$ nach $\mathbb{R}^+$ definiert, d.h. 
        $f$ ist invertierbar zwischen den angegebenen Mengen und sowohl $f$ als auch
        $f^{-1}$ ist stetig.\\

    \textbf{Beweis.:}\\
        Sei $g : (0,1) \to \mathbb{R}^+$, mit $x \mapsto \frac{x}{1 - x^2}$.\\
        Z.z.: (1) $g$ bijektiv, (2) $g$ stetig, (3) $g^{-1}$ stetig. \\

        (1) $g$ bijektiv \\
            (1a) $g$ injektiv, also $g(a) = g(b) \Rightarrow a = b$, für alle $a,b \in (0,1)$. \\
                Seien $a,b \in (0,1)$. Dann gilt:
                \begin{eqnarray*}
                  g(a) &=& g(b) \\
                  \Leftrightarrow \frac{a}{1 - a^2} &=& \frac{b}{1 - b^2} \\
                  \Leftrightarrow a - ab^2 &=& b - ba^2 \\
                  \Leftrightarrow a &=& b
                \end{eqnarray*}
%                $\Rightarrow g$ injektiv. \\
            (1b) $g$ surjektiv, also $\forall c \in \mathbb{R}^+ \exists x \in (0,1):\, g(x) = c$.\\
                Sei $c \in \mathbb{R}^+$. Wähle $x := \frac{\sqrt{1+4c^2}-1}{2c}$. \\
                Dann gilt $x>0$, da $c>0$ ist steht im Zähler immer eine positive Zahl ($\sqrt{1 + x} \geq 1$ für $x > 0$),
		und der Nenner $2c > 0$.\\

		Es gilt darüber hinaus, dass $x < 1$, da
		$$\begin{array}{crcl}
			& \frac{\sqrt{1+4c^2} -1}{2c} &<& 1\\
			\Leftrightarrow & \sqrt{1+4c^2} -1 &<& 2c\\
			\Leftrightarrow & \sqrt{1+4c^2} &<& 2c +1\\
			\stackrel{>0 (siehe \, eben)}\Leftrightarrow &
				 1 + 4c^2 &<& 4c^2 + 4c + 1\\
			\Leftrightarrow & 0 & < & 2c
		\end{array}$$
		Und dies gilt, da $c > 0$ ist.
%                $\Rightarrow g$ surjektiv. \\
                $\Rightarrow g$ bijektiv. \\

        (2) $g$ stetig \\
            Da die Funktion im Nenner $1-x^2$ stetig ist und auf dem Intervall $(0,1)$ keine Nullstellen hat, sowie $x$ stetig ist,
	ist nach Sätzen aus Ana I der Quotient aus den beiden wieder eine stetige Funktion.

        (3) $g^{-1}$ stetig\\
            blablabla
    
    \item Sei die Funktion $f \; : \; [0,1) \cup [2,3] \rightarrow [0,2]$ gegeben durch
        $$
            f(x) = \left\{ \begin{array}{lr}
                x &, x\in [0,1)\\
                x-1 &, x \in [2,3]
            \end{array} \right. .
        $$
        Zeigen Sie, dass $f$ stetig und invertierbar ist, aber die Umkehrfunktion 
        $f^{-1} \; : \; [0,2] \rightarrow [0,1) \cup [2,3]$ nicht stetig ist.\\

    \textbf{Beweis:}\\
        tbd
\end{enumerate}
