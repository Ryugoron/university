\subsection*{Aufgabe 31 \mdseries\itshape Satz von euler über homogene Funktionen}

\begin{enumerate}[(i)]
   \item Man nennt eine Funktion $f \; : \; \mathbb{R}^n \rightarrow \mathbb{R}$ \emph{homogon vom Grade $m$},
      wenn
      $$
         f(tx)=t^m f(x) \text{ für alle }x\in\mathbb{R}^n\text{ und alle }t\in\mathbb{R}\text{ gilt.}
      $$
      Sei $f$ zusätzlich differenzierbar. Zeigen Sie, dass dann
      $$
         \nabla f(x) \cdot x = m f(x) \text{ für alle}x\in\mathbb{R}^n. 
      $$

   \textbf{Beweis:}\\
      tbd

   \item Berechnen Sie explizit $\nabla f(x) \cdot x$ für
      $$
         f(x) = \underset{i,j=1}{\overset{n}{\sum}} a_{ij}x_ix_j,
      $$
      und vergleichen Sie Ihr Ergebnis mit (i).\\

   \textbf{Lösung:}\\
      tbd
\end{enumerate}
