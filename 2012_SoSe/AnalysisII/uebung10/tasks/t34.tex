\subsection*{Aufgabe 34 \mdseries\itshape Extremalwertaufgabe}

   Ermitteln Sie, an welcher Stelle die Funktion
   $$
      z (x,y) = x^3 + 3xy + y^3, \qquad (x,y)\in \mathbb{R}^2
   $$
   Extrema besitzt.\\

\textbf{Lösung:}\\
   Für die Lösung bestimmen wir zunächst die kritischen Punkte $p_0 = (x_0,y_0)$, für die
   gilt $\nabla f(p_0) = 0$.

   $$
      \nabla f(p_0) = \begin{pmatrix} 3x_0^2 + 3y_0 \\ 3x_0 + 3y_0^2 \end{pmatrix}\\
   $$

   Nun haben wir 2 Bedingungen:
   $$\begin{array}{crcl}
      I:&0&=& 3x_0^2 + 3y_0\\
      II:&0&=& 3x_0 + 3y_0^2\\
      \hline
      I:&y_0&=& -x_0^2\\
      \text{in }II:&0&=& 3x_0 + 3x_0^4\\
         &&=&3 (x_0) (1+x_0^3)\\
      &\Rightarrow&& x_0=0 \; \lor \; x_0=-1\\
      \hline
      0\text{ in }I:&y-0&=& -0^2\\
         &&=&0\\
      -1\text{ in }I:&y_0&=& -(-1)^2\\
         &&=&-1
   \end{array}$$
   Die beiden kritischen Punkte sind also $p_0 = (0,0)$ und $p_1 = (-1,-1)$. Nun untersuchen wir die
   Hesse Matrix bezüglich dieser Werte.\\

   Hessematrix\\
   $$ H(x,y) = 
      \begin{pmatrix}
         6x & 3\\
         3 & 6y
      \end{pmatrix}
   $$

   Die Matrix bezüglich $p_0$ ist also
   $$H(x_0,y_0) = 
      \begin{pmatrix}
         0 & 3\\
         3 & 0
      \end{pmatrix}
   $$
   Diese Matrix ist nun leider nur semidefinit, daher handelt es sich hier um kein Extrempunkt,
   sondern aller Wahrscheinlichkeit nach um einen Sattelpunkt.\\

   Die Matrix bezüglich $p_1$ ist
   $$H(x_1,y_1)=
      \begin{pmatrix}
         -6 & 3\\
         3  & -6
      \end{pmatrix}
   $$

   Nach dem Hauptminoren Kriterium haben wir die Determinanten $-6$ und $27$ also können wir wiederum
   keine Aussage treffen.
