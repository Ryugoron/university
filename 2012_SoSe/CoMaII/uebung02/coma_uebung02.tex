\documentclass[11pt,a4paper,ngerman]{article}
\usepackage[bottom=2.5cm,top=2.5cm]{geometry} 
\usepackage{babel}
\usepackage[utf8]{inputenc} 
\usepackage[T1]{fontenc} 
\usepackage{ae} 
\usepackage{amssymb} 
\usepackage{amsmath} 
\usepackage{graphicx}
\usepackage{fancyhdr}
\usepackage{fancyref}
\usepackage{listings}
\usepackage{xcolor}
\usepackage{paralist}

%\usepackage[pdftex, bookmarks=false, pdfstartview={FitH}, linkbordercolor=white]{hyperref}
\usepackage{fancyhdr}
\pagestyle{fancy}
\fancyhead[C]{CoMa II}
\fancyhead[L]{Übung Nr. 2}
\fancyhead[R]{SoSe 2012}
\fancyfoot{}
\fancyfoot[L]{}
\fancyfoot[C]{\thepage / \pageref{LastPage}}
\renewcommand{\footrulewidth}{0.5pt}
\renewcommand{\headrulewidth}{0.5pt}
\setlength{\parindent}{0pt} 
\setlength{\headheight}{0pt}

\author{Tutor: Sebastian Scherer}
\date{}
\title{Max Wisniewski , Alexander Steen}

\begin{document}

\lstset{language=Pascal, basicstyle=\ttfamily\fontsize{10pt}{10pt}\selectfont\upshape, commentstyle=\rmfamily\slshape, keywordstyle=\rmfamily\bfseries, breaklines=true, frame=single, xleftmargin=3mm, xrightmargin=3mm, tabsize=2}

\maketitle
\thispagestyle{fancy}


%% ------------------------------------------------------
%%                     AUFGABE 1
%% ------------------------------------------------------

\section*{Aufgabe 1}
Die Funktion $f: \mathbb{R} \to \mathbb{R}$, mit $x \mapsto \frac{1}{x^2+1}$ soll nach der Methode von Lagrange interpoliert werden. \\

(i) Als quadratisches Polynom: \\ \\
Suche Knotenbasis $\mathfrak{P}$ des $P_2$ mit $\mathfrak{P} = \{\mathfrak{p}_0, \mathfrak{p}_1,\mathfrak{p}_2 \}$.\\
Es gilt:
\begin{eqnarray*}
\mathfrak{p}_k(x) & = & \prod_{i = 0, i \neq k}^{2}{\frac{x-x_i}{x_k-x_i}}\\
\end{eqnarray*}

Stützstellen sind $x_0 = -1$, $x_1 = 0$, $x_2 = 1$, also gilt für die Basis $\mathfrak{P}$:
\begin{eqnarray*}
\mathfrak{p}_0(x) = L_0(x) & = & \prod_{i = 1}^{2}{\frac{x-x_i}{x_k-x_i}} = \frac{x-x_1}{x_0-x_1} \cdot \frac{x-x_2}{x_0-x_2} \\
 & = & \frac{x-0}{(-1) - 0} \cdot \frac{x-1}{(-1) - 1} = -x \cdot \frac{x-1}{-2}
   = \frac{1}{2}(x^2 - x)\\
\mathfrak{p}_1(x) = L_1(x) & = & \prod_{i = 0,i \neq 1}^{2}{\frac{x-x_i}{x_k-x_i}} = \frac{x-x_0}{x_1-x_0} \cdot \frac{x-x_2}{x_1-x_2}\\
 & = & \frac{x-(-1)}{0 - (-1)} \cdot \frac{x-1}{0 - 1} = (x+1) \cdot -(x-1) = -x^2 + 1\\
\mathfrak{p}_2(x) = L_2(x)  & = & \prod_{i = 0}^{1}{\frac{x-x_i}{x_k-x_i}} = \frac{x-x_0}{x_2-x_0} \cdot \frac{x-x_1}{x_2-x_1} \\
 & = & \frac{x-(-1)}{1 - (-1)} \cdot \frac{x-0}{1 - 0} = \frac{x+1}{2} \cdot x
   = \frac{1}{2}(x^2 + x)\\
\end{eqnarray*}

Suche nun das Polynom $p \in P_2$:\\
Das Polynom $p$ wird durch $p = \sum_{k=0}^{2}{f(x_k) \mathfrak{p}_k(x)}$ bestimmt, also gilt:
$$
p = \sum_{k=0}^{2}{f(x_k) \mathfrak{p}_k(x)} = f(-1) \cdot \frac{1}{2}(x^2 - x)
  + f(0) \cdot (-x^2 + 1)
  + f(1) \cdot \frac{1}{2}(x^2 + x)
$$
$$
  = \frac{1}{4}(x^2 - x) + (-x^2 + 1) + \frac{1}{4}(x^2 + x) 
  = -\frac{1}{2}x^2 + 1
$$

(ii) Als kubisches Polynom mit zusätzlicher Stelle $x_3 = \frac{1}{2}$. \\ \\
Suche Knotenbasis $\mathfrak{Q}$ des $P_3$ mit $\mathfrak{Q} = \{\mathfrak{q}_0, \mathfrak{q}_1,\mathfrak{q}_2, \mathfrak{q}_3 \}$.\\

$\mathfrak{P}$ wie oben, zusätzliche Stützstelle $x_3 = 1/2$, also gilt für die Basis $\mathfrak{Q}$:
\begin{eqnarray*}
\mathfrak{q}_0(x) & = & \mathfrak{p}_0 \cdot \frac{x-x_3}{x_0-x_3} \\
 & = & \frac{1}{2}(x^2 - x) \cdot \frac{x-1/2}{-3/2}
   = \frac{1}{6}(-2x^3 + 3x^2 - x) \\
\mathfrak{q}_1(x) & = & \mathfrak{p}_1 \cdot \frac{x-x_3}{x_1-x_3} \\
 & = & (-x^2 + 1) \cdot \frac{x-1/2}{-1/2}
   = 2x^3 - x^2 - 2x + 1 \\
\mathfrak{q}_2(x) & = & \mathfrak{p}_2 \cdot \frac{x-x_3}{x_2-x_3} \\
 & = & \frac{1}{2}(x^2 + x) \cdot \frac{x-1/2}{1/2}
   = x^3 + \frac{1}{2}x^2 - \frac{1}{2}x\\
\mathfrak{q}_3(x) & = & \prod_{i = 0}^{2}{\frac{x-x_i}{x_k-x_i}} \\
 & = & \frac{x-x_0}{x_3-x_0} \cdot \frac{x-x_1}{x_3-x_1} \cdot \frac{x-x_2}{x_3-x_2}
   = \frac{x-(-1)}{1/2 - (-1)} \cdot \frac{x - 0}{1/2 - 0} \cdot \frac{x-1}{1/2 - 1}
   = \frac{8}{3} (x - x^3)
\end{eqnarray*}

Suche nun das Polynom $p' \in P_3$:\\
Das Polynom $p'$ wird durch $p' = \sum_{k=0}^{3}{f(x_k) \mathfrak{q}_k(x)}$ bestimmt, also gilt:
$$
p' = \sum_{k=0}^{3}{f(x_k) \mathfrak{q}_k(x)} =
    f(-1) \cdot \frac{1}{6}(-2x^3 + 3x^2 - x)
  + f(0) \cdot (2x^3 - x^2 - 2x + 1)
  + f(1) \cdot (x^3 + \frac{1}{2}x^2 - \frac{1}{2}x)
  + f(1/2) \cdot \frac{8}{3} (x - x^3)
$$
$$
  = \frac{1}{12}(-2x^3 + 3x^2 - x) + (2x^3 - x^2 - 2x + 1) + \frac{1}{2}(x^3 + \frac{1}{2}x^2 - \frac{1}{2}x) + \frac{4}{5} \cdot \frac{8}{3} (x - x^3)
  = \frac{1}{5}x^3 - \frac{1}{2}x^2 - \frac{1}{5}x + 1
$$

%% ------------------------------------------------------
%%                     AUFGABE 2
%% ------------------------------------------------------

\section*{Ausgabe 2}

\begin{description}
\item[a)] asd
\item[b)] asd
\begin{lstlisting}
function [p] = monomialcoefficients(xi, f)
%% grad bezeichnet den grad der inpolation, entspricht der länge der eingabeliste - 1
%% L enthält die lagrangepolynome
grad = size(xi,2) %% nur die längendimension interessiert uns
L = []
for k = 1:grad,
  temp = [1];
  range = 1:grad;
  indices = ones(1,grad);
  indices(k) = 0;
  for i = range(logical(indices)),
    temp = conv(temp, [1/(xi(k)-xi(i)), -xi(i)/(xi(k)-xi(i))]);
  end
  L(k,:) = temp;
end
  fks = eye(grad);
  for i = 1:grad,
    fks(i,i) = f(xi(i));
  end
  L = fks * L
  p = L' * ones(grad,1); 
  p = p'
\end{lstlisting}

\begin{lstlisting}
function [y] = monomialinterpolation(x, p)
grad = size(p,2);
temp = 0;
for i = 1:grad,
temp = temp + p(i)*x^(grad-i)
end
y = temp
\end{lstlisting}
\item[c)] asd
\item[d)] asd
\end{description}

\label{LastPage}

\end{document}
