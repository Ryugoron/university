\documentclass[11pt,a4paper,ngerman]{article}
\usepackage[bottom=2.5cm,top=2.5cm]{geometry} 
\usepackage{babel}
\usepackage[utf8]{inputenc} 
\usepackage[T1]{fontenc} 
\usepackage{ae} 
\usepackage{amssymb} 
\usepackage{amsmath} 
\usepackage{graphicx}
\usepackage{fancyhdr}
\usepackage{fancyref}
\usepackage{listings}
\usepackage{xcolor}
\usepackage{paralist}

%\usepackage[pdftex, bookmarks=false, pdfstartview={FitH}, linkbordercolor=white]{hyperref}
\usepackage{fancyhdr}
\pagestyle{fancy}
\fancyhead[C]{CoMa II}
\fancyhead[L]{Übung Nr. 1}
\fancyhead[R]{SoSe 2012}
\fancyfoot{}
\fancyfoot[L]{}
\fancyfoot[C]{\thepage / \pageref{LastPage}}
\renewcommand{\footrulewidth}{0.5pt}
\renewcommand{\headrulewidth}{0.5pt}
\setlength{\parindent}{0pt} 
\setlength{\headheight}{0pt}

\author{Tutor: Sebastian Scherer}
\date{}
\title{Max Wisniewski , Alexander Steen}

\begin{document}

\lstset{language=Pascal, basicstyle=\ttfamily\fontsize{10pt}{10pt}\selectfont\upshape, commentstyle=\rmfamily\slshape, keywordstyle=\rmfamily\bfseries, breaklines=true, frame=single, xleftmargin=3mm, xrightmargin=3mm, tabsize=2}

\maketitle
\thispagestyle{fancy}


%% ------------------------------------------------------
%%                     AUFGABE 1
%% ------------------------------------------------------

\section*{Aufgabe 1}

Seien $A = \left(
                \begin{array}{cc}
                    1 & 1 \\
                    0 & \varepsilon
                \end{array}
           \right),
      b_1 = \left(
                \begin{array}{c}
                    2 \\
                    \varepsilon
                \end{array}
           \right),
      b_2 = \left(
                \begin{array}{c}
                    0 \\
                    1 
                \end{array}
           \right)$
mit $0 < \varepsilon \ll 1$.

\begin{description}
\item[a)] Berechne die Kondition $\kappa(A) = \|A \|_{\infty} \|A^{-1} \|_{\infty} $ \\

Wegen $\|A \|_{\infty} = 2$ und
$\|A^ {-1} \|_{\infty} =
            \left|\left| \left(
                \begin{array}{cc}
                    1 & -\frac{1}{\varepsilon} \\
                    0 & \frac{1}{\varepsilon}
                \end{array}
           \right) \right|\right|_{\infty}
            = \frac{1}{\varepsilon}$
folgt
$$ \kappa(A) = 2 \cdot \frac{1}{\varepsilon} = \frac{2}{\varepsilon}$$.

\item[b)] Löse die Gleichungssysteme $Ax_1 = b_1$ und $Ax_2 = b_2$ und interpretiere die Ergebnisse im Bezug auf $\kappa(A)$. \\

Beide LGS sind bereits in Zeilenstufenform, sodass wir einfach rückwarts substituieren können. \\
Für das LGS $\left(
      \begin{array}{cc|c}
         1 & 1           & 2\\
         0 & \varepsilon & \varepsilon
      \end{array}
    \right)$: \\

(I) $\varepsilon x_{1_2} = \varepsilon \Leftrightarrow x_{1_2} = 1$ \\
(II) $x_{1_1} + x_{1_2} = 2 \Leftrightarrow x_{1_1} + 1 = 2 \Leftrightarrow x_{1_1} = 1$ \\
$ \Rightarrow x_1 = \left(\begin{array}{c} 1 \\ 1 \end{array} \right)$.


Für das LGS $\left(
      \begin{array}{cc|c}
         1 & 1           & 0\\
         0 & \varepsilon & 1
      \end{array}
    \right)$: \\

(I) $\varepsilon x_{2_2} = 1 \Leftrightarrow x_{2_2} = \frac{1}{\varepsilon}$ \\
(II) $x_{2_1} + x_{2_2} = 0 \Leftrightarrow x_{2_1} + \frac{1}{\varepsilon} = 0 \Leftrightarrow x_{2_1} = -\frac{1}{\varepsilon}$ \\
$ \Rightarrow x_2 = \left(\begin{array}{c} -\frac{1}{\varepsilon} \\ \frac{1}{\varepsilon} \end{array} \right)$.

Aus den Ergebnissen können wir sehen, dass der Fehler für kleine $\varepsilon$ bei $b_1$ absolut stark wächst. Bei $b_2$ streben allerdings der Fehler und der Ergebniswert gegen unendlich. Daher ist der Fehler für kleine $\varepsilon$ bei $b_2$ nicht so gravierenend, wie bei $b_1$. Für große $\varepsilon$ geht der relative Fehler beider Ergebnisse gegen 0.
\end{description}

\pagebreak

%% ------------------------------------------------------
%%                     AUFGABE 2
%% ------------------------------------------------------

\section*{Ausgabe 2}

Seien $A,B \in \mathbb{R}^{2,2}$, mit
$ A = \left(
        \begin{array}{cc}
          1         & \frac{1}{2} \\
          \sqrt(2)  & \frac{1}{\sqrt(2)}
        \end{array}
      \right),
 B = \left(
        \begin{array}{cc}
          1         & \frac{1}{2} \\
          \sqrt(2)  & \sqrt(\frac{1}{2})
        \end{array}
      \right)
$.

\begin{description}

\item[a)] Berechnen Sie in matlab mittels \textbackslash -Operator die Lösungen der LGS \\
          $Ax = \left( \begin{array}{c} 1 \\ 1\end{array}\right), 
           By = \left( \begin{array}{c} 1 \\ 1\end{array}\right)$. \\
            
          Die Matrizen werden mittels der folgenden Kommandos in matlab definiert:          
          \begin{lstlisting}
  >> A = [1, 1/2; sqrt(2), 1/sqrt(2)]
  >> B = [1, 1/2; sqrt(2), sqrt(1/2)]
          \end{lstlisting}
          
          Wir berechnen die Lösungen für $x$ und $y$ mittels \textbackslash -Operator wie folgt:

          \begin{lstlisting}
  >> x =  A \ [1;1]
  >> y =  B \ [1;1]
          \end{lstlisting}

          Dabei kommt bei der ersten Berechnung der Hinweis "Matrix is close to singular or badly scaled. Results may be inaccurate. RCOND = 7.850462e-17.", bei der zweiten Berechnung die Fehlermeldung "Matrix is singular to working precision.".
          Als Lösung erhalten wir
          $x = 10^{15} \cdot \left( \begin{array}{c} -1.3191 \\ 2.6381 \end{array} \right)$ und
          $y = \left( \begin{array}{c} -\infty \\ \infty \end{array} \right)$.

\item[b)] Erklären Sie ihre Beobachtungen. \\

          In matlab werden die Terme $1/sqrt(2)$ und $sqrt(1/2)$ verschieden gerundet.
          Dies kann man z.B. daran sehen, dass ihre Differenz von Null verschieden ist:

          \begin{lstlisting}          
  >> 1/sqrt(2) - sqrt(1/2)
  ans = -1.1102e-16
          \end{lstlisting}
          
          Aus diesem Grund entstehen bei der Berechnung verschiedene Rundungsfehler. \\
          Schauen wir uns die Matrizen $A$ und $B$ an, so sehen wir, dass diese singulär sind,
          da $det(A)=det(B)=0$ gilt. Aus diesem Grund können wir das LGS nicht mit 
          "\textbackslash " lösen, da hier die Matrizen invertiert werden. Bei $\sqrt{2}$ handelt es sich um sich um eine irrationale Zahl, die der PC  nur gerundet 
	darstellen kann. In Matrix B kann matlab den abstrakten Zwischenwert $\sqrt{\frac{1}{2}}$ zwischenspeichern und korrekt auflösen.
	Bei Matrix A wird allerdings zuerst geteilt. Bei dieser Operation verstärkt sich der Rundungsfehler der Darstellung der irrationalen Zahl.
	Dadurch wird die Matrix nicht mehr singulär und es kann ein Ergebniss berechnet werden.

\end{description}
%% Raw matlab output:
%%>> A = [1, 1/2; sqrt(2), 1/sqrt(2)]
%%
%%A =
%%
%%    1.0000    0.5000
%%    1.4142    0.7071
%%
%%>> B = [1, 1/2; sqrt(2), sqrt(1/2)]
%%
%%B =
%%
%%    1.0000    0.5000
%%    1.4142    0.7071
%%
%%>> A \ [1;1]
%%Warning: Matrix is close to singular or badly scaled.
%%         Results may be inaccurate. RCOND = 7.850462e-17. 
%%
%%ans =
%%
%%   1.0e+15 *
%%
%%   -1.3191
%%    2.6381
%%
%%>> B \ [1;1]
%%Warning: Matrix is singular to working precision. 
%%
%%ans =
%%
%%  -Inf
%%   Inf
%%


\pagebreak 

%% ------------------------------------------------------
%%                     AUFGABE 3
%% ------------------------------------------------------

\section*{Aufgabe 3}
Zeigen Sie, dass die von der $1$-Norm induzierte Matrixnorm der Spaltensummennorm entspricht. \\

Sei $A \in \mathbb{R}^{n \times n}, A \neq 0$, zu zeigen:
$$ \|A \|_1 := \sup_{x \neq 0}{\frac{\|Ax\|_{1}}{\|x\|_{1}}}
                  =\max_{1 \leq j \leq n}{\sum_{i=1}^{n}{|a_{ij}|}} $$

Wie in CoMa I gezeigt, gilt
$\sup_{x \neq 0}{\frac{\|Ax\|_{1}}{\|x\|_{1}}} = \sup_{\|x\|_{1} = 1}{\|Ax\|_{1}}$. \\ \\
Sei nun $x \in \mathbb{R}^{n}$, mit $\|x\|_{1} = 1$.\\

(1) z.z. $\|A \|_1 \leq \max_{1 \leq j \leq n}{\sum_{i=1}^{n}{|a_{ij}|}}$ 

$$
\|Ax\|_{1} \stackrel{\text{Def.}\|.\|_{1}}{=}
     \sum_{i=1}^{n}{\left| \sum_{j=1}^{n}{a_{ij}x_j} \right|}
\leq \sum_{i=1}^{n}{\sum_{j=1}^{n}{\left|a_{ij}x_j\right|} }
=    \sum_{i=1}^{n}{\sum_{j=1}^{n}{|a_{ij}||x_j|} }
$$
$$
=    \sum_{j=1}^{n}{\sum_{i=1}^{n}{|a_{ij}||x_j|} }
=    \sum_{j=1}^{n}{|x_j| \sum_{i=1}^{n}{|a_{ij}|} }
\leq \sum_{j=1}^{n}{|x_j| \max_{1\leq j \leq n}{\sum_{i=1}^{n}{|a_{ij}|}} }
$$
$$
\stackrel{(**)}{=} \max_{1\leq j \leq n}{\sum_{i=1}^{n}{|a_{ij}|}} \quad \text{(**) gilt, da} \sum_{j=1}^{n}{|x_j|} = 1
$$

$\Rightarrow \|A\|_{1} \leq \max_{1 \leq j \leq n}{\sum_{i=1}^{n}{|a_{ij}|}}$. \\ \\

(2) z.z. $\|A \|_1 \geq \max_{1 \leq j \leq n}{\sum_{i=1}^{n}{|a_{ij}|}}$ 


Sei $e_j$ der $j$-te Einheitsvektor.\\
$$    \max_{1 \leq j \leq n}{\| Ae_j \|_{1}} 
\stackrel{\text{Def.}}{=}     \max_{1 \leq j \leq n}{\sum_{i=1}^{n}{|a_{ij}|}}
\stackrel{(***)}{\leq}  \sup_{\|x\|_{1} = 1}{\|Ax\|_{1}}
=     \| A \|_1
$$
(***) gilt, da das Supremum natürlich nicht kleiner werden kann.\\
$\Rightarrow \|A\|_{1} \geq \max_{1 \leq j \leq n}{\sum_{i=1}^{n}{|a_{ij}|}}$.\\ \\
$\Longrightarrow \|A\|_{1} = \max_{1 \leq j \leq n}{\sum_{i=1}^{n}{|a_{ij}|}}$.

\mbox{}\hfill $\square$

\label{LastPage}

\end{document}
