\documentclass[11pt,a4paper,ngerman]{article}
\usepackage[bottom=2.5cm,top=2.5cm]{geometry} 
\usepackage{babel}
\usepackage[utf8]{inputenc} 
\usepackage[T1]{fontenc} 
\usepackage{ae} 
\usepackage{amssymb} 
\usepackage{amsmath} 
\usepackage{graphicx}
\usepackage{fancyhdr}
\usepackage{fancyref}
\usepackage{listings}
\usepackage{xcolor}
\usepackage{paralist}

%\usepackage[pdftex, bookmarks=false, pdfstartview={FitH}, linkbordercolor=white]{hyperref}
\usepackage{fancyhdr}
\pagestyle{fancy}
\fancyhead[C]{CoMa II}
\fancyhead[L]{Übung Nr. 1}
\fancyhead[R]{SoSe 2012}
\fancyfoot{}
\fancyfoot[L]{}
\fancyfoot[C]{\thepage / \pageref{LastPage}}
\renewcommand{\footrulewidth}{0.5pt}
\renewcommand{\headrulewidth}{0.5pt}
\setlength{\parindent}{0pt} 
\setlength{\headheight}{0pt}

\author{Tutor: XY}
\date{}
\title{Max Wisniewski , Alexander Steen}

\begin{document}

\lstset{language=Pascal, basicstyle=\ttfamily\fontsize{10pt}{10pt}\selectfont\upshape, commentstyle=\rmfamily\slshape, keywordstyle=\rmfamily\bfseries, breaklines=true, frame=single, xleftmargin=3mm, xrightmargin=3mm, tabsize=2}

\maketitle
\thispagestyle{fancy}


%% ------------------------------------------------------
%%                     AUFGABE 1
%% ------------------------------------------------------

\section*{Aufgabe 1}

Seien $A = \left(
                \begin{array}{cc}
                    1 & 1 \\
                    0 & \varepsilon
                \end{array}
           \right),
      b_1 = \left(
                \begin{array}{c}
                    2 \\
                    \varepsilon
                \end{array}
           \right),
      b_2 = \left(
                \begin{array}{c}
                    0 \\
                    1 
                \end{array}
           \right)$
mit $0 < \varepsilon \ll 1$.

\begin{description}
\item[a)] Berechne die Kondition $\kappa(A) = \|A \|_{\infty} \|A^{-1} \|_{\infty} $ \\

Wegen $\|A \|_{\infty} = 2$ und
$\|A^ {-1} \|_{\infty} =
            \left|\left| \left(
                \begin{array}{cc}
                    1 & -\frac{1}{\varepsilon} \\
                    0 & \frac{1}{\varepsilon}
                \end{array}
           \right) \right|\right|_{\infty}
            = \frac{1}{\varepsilon}$
folgt
$$ \kappa(A) = 2 \cdot \frac{1}{\varepsilon} = \frac{2}{\varepsilon}$$.

\item[b)] Löse die Gleichungssysteme $Ax_1 = b_1$ und $Ax_2 = b_2$ und interpretiere die Ergebnisse im Bezug auf $\kappa(A)$. \\

Beide LGS sind bereits in Zeilenstufenform, sodass wir einfach rückwarts substituieren können. \\
Für das LGS $\left(
      \begin{array}{cc|c}
         1 & 1           & 2\\
         0 & \varepsilon & \varepsilon
      \end{array}
    \right)$: \\

(I) $\varepsilon x_{1_2} = \varepsilon \Leftrightarrow x_{1_2} = 1$ \\
(II) $x_{1_1} + x_{1_2} = 2 \Leftrightarrow x_{1_1} + 1 = 2 \Leftrightarrow x_{1_1} = 1$ \\
$ \Rightarrow x_1 = \left(\begin{array}{c} 1 \\ 1 \end{array} \right)$.


Für das LGS $\left(
      \begin{array}{cc|c}
         1 & 1           & 0\\
         0 & \varepsilon & 1
      \end{array}
    \right)$: \\

(I) $\varepsilon x_{2_2} = 1 \Leftrightarrow x_{2_2} = \frac{1}{\varepsilon}$ \\
(II) $x_{2_1} + x_{2_2} = 0 \Leftrightarrow x_{2_1} + \frac{1}{\varepsilon} = 0 \Leftrightarrow x_{2_1} = -\frac{1}{\varepsilon}$ \\
$ \Rightarrow x_2 = \left(\begin{array}{c} -\frac{1}{\varepsilon} \\ \frac{1}{\varepsilon} \end{array} \right)$.

yada yada, kommentare!!
\end{description}


%% ------------------------------------------------------
%%                     AUFGABE 2
%% ------------------------------------------------------

\section*{Ausgabe 2}
>> A = [1, 1/2; sqrt(2), 1/sqrt(2)]

A =

    1.0000    0.5000
    1.4142    0.7071

>> B = [1, 1/2; sqrt(2), sqrt(1/2)]

B =

    1.0000    0.5000
    1.4142    0.7071

>> A \ [1;1]
Warning: Matrix is close to singular or badly scaled.
         Results may be inaccurate. RCOND = 7.850462e-17. 

ans =

   1.0e+15 *

   -1.3191
    2.6381

>> B \ [1;1]
Warning: Matrix is singular to working precision. 

ans =

  -Inf
   Inf


%% ------------------------------------------------------
%%                     AUFGABE 3
%% ------------------------------------------------------

\section*{Aufgabe 3}
Zeigen Sie, dass die von der $1$-Norm induzierte Matrixnorm der Spaltensummennorm entspricht. \\

Sei $A \in \mathbb{R}^{n \times n}$, zu zeigen:
$$ \|A \|_1 := \sup_{x \neq 0}{\frac{\|Ax\|_{1}}{\|x\|_{1}}}
                  =\max_{1 \leq j \leq n}{\sum_{i=1}^{n}{|a_{ij}|}} $$

Wie in CoMa I gezeigt, gilt
(*) $\sup_{x \neq 0}{\frac{\|Ax\|_{1}}{\|x\|_{1}}} = \sup_{\|x\|_{1} = 1}{\|Ax\|_{1}}$. \\ \\
Sei nun $x \in \mathbb{R}^{n}$, mit $\|x\|_{1} = 1$.\\

(1) $\|A \|_1 \leq \max_{1 \leq j \leq n}{\sum_{i=1}^{n}{|a_{ij}|}}$ 

$$
\|Ax\|_{1} \stackrel{\text{Def.}\|.\|_{1}}{=}
     \sum_{i=1}^{n}{\left| \sum_{j=1}^{n}{a_{ij}x_j} \right|}
\leq \sum_{i=1}^{n}{\sum_{j=1}^{n}{\left|a_{ij}x_j\right|} }
=    \sum_{i=1}^{n}{\sum_{j=1}^{n}{|a_{ij}||x_j|} }
$$
$$
=    \sum_{j=1}^{n}{\sum_{i=1}^{n}{|a_{ij}||x_j|} }
=    \sum_{j=1}^{n}{|x_j| \sum_{i=1}^{n}{|a_{ij}|} }
\leq \sum_{j=1}^{n}{|x_j| \max_{1\leq j \leq n}{\sum_{i=1}^{n}{|a_{ij}|}} }
$$
$$
\stackrel{(**)}{=} \max_{1\leq j \leq n}{\sum_{i=1}^{n}{|a_{ij}|}} \quad \text{(**) gilt, da} \sum_{j=1}^{n}{|x_j|} = 1
$$

Wegen (*) $\Rightarrow \|A\|_{1} \leq \sum_{i=1}^{n}{|a_{ij}|}$. \\ \\

(2) $\|A \|_1 \geq \max_{1 \leq j \leq n}{\sum_{i=1}^{n}{|a_{ij}|}}$ 

%\left( \sum_{i=1}^{m}{\sum_{j=1}^{n}{|a_{ij}|}} \right)
\label{LastPage}

\end{document}
