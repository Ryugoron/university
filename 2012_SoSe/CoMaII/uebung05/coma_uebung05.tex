\documentclass[11pt,a4paper,ngerman]{article}
\usepackage[bottom=2.5cm,top=2.5cm]{geometry} 
\usepackage{babel}
\usepackage[utf8]{inputenc} 
\usepackage[T1]{fontenc} 
\usepackage{ae} 
\usepackage{amssymb} 
\usepackage{amsmath} 
\usepackage{graphicx}
\usepackage{fancyhdr}
\usepackage{fancyref}
\usepackage{listings}
\usepackage{xcolor}
\usepackage{paralist}
\usepackage{subfigure}

%\usepackage[pdftex, bookmarks=false, pdfstartview={FitH}, linkbordercolor=white]{hyperref}
\usepackage{fancyhdr}
\pagestyle{fancy}
\fancyhead[C]{CoMa II}
\fancyhead[L]{Übung Nr. 5}
\fancyhead[R]{SoSe 2012}
\fancyfoot{}
\fancyfoot[L]{}
\fancyfoot[C]{\thepage / \pageref{LastPage}}
\renewcommand{\footrulewidth}{0.5pt}
\renewcommand{\headrulewidth}{0.5pt}
\setlength{\parindent}{0pt} 
\setlength{\headheight}{0pt}

\author{Tutor: Sebastian Scherer}
\date{}
\title{Max Wisniewski , Alexander Steen}

\begin{document}

\lstset{language=Pascal, basicstyle=\ttfamily\fontsize{10pt}{10pt}\selectfont\upshape, commentstyle=\rmfamily\itshape\small, keywordstyle=\rmfamily\bfseries, breaklines=true, frame=single, xleftmargin=3mm, xrightmargin=3mm, tabsize=2}

\maketitle
\thispagestyle{fancy}


%% ------------------------------------------------------
%%                     AUFGABE 1
%% ------------------------------------------------------

\section*{Aufgabe 1}
Das folgende matlab-Programm rechnet für $k= 1,2,6$ die jeweiligen Newton-Cotes-Formeln aus. Für andere Eingaben von $k$ wird das Program mit einem Fehler terminieren. Der Aufwand ist jeweils auf die Berechnung des Programms bezogen, welche Funktionsauswertungen nicht mehrfach benutzt und darum etwas ineffizienter ist, als es sein könnte. Allerdings wurde in der Aufgabe nichts von Effizienz gesagt, also ist das ja kein Problem.

\begin{lstlisting}[language=matlab]
function [S,A] = sumnc(ab,f,k,n)
% ab Integrationsintervall
% f zu intigrierende Funktion
% k regel der newton-cotes-formel
% n anzahl der teilintervalle

% Bezeichnungen ab hier wie im Skript
% h ist der Abstand zwischen den Stellen
h = (ab(2) - ab(1))/n;
% z sind die Stuetzstellen
z = ab(1) + (1:n).*h;

% Gewichte jeweils aus dem Skript entnommen,
% sowie Berechnungsvorgehen

if (k == 1) %% Trapez
   x = @(k) z(k);
   S = h/2 * (f(ab(1)) + f(ab(2)));
   A = 2; % Bereits 2 Auswertungen
   for k = 2:(n),
      S = S + h*f(x(k)); % Siehe Skript
      A = A+1; % Eine Berechnung mehr
   end
   return % Aufwand n+1
elseif (k == 2) %% Simpson
   % (Algorithmisches) Vorgehen analog zu Trapez
   x = @(k,j) z(k/2) + j*h/2;
   S = 0;
   A = 0;
   for k = 1:(n),
      S = S + h/6 * (f(x(2*k,0)) + 4*f(x(2*k,1)) + f(x(2*k,2)));
      A = A + 3;
   end
   return % Aufwand 3n
elseif (k == 6) %% Weddle-Regel
   x = @(k,j) z(k/6) + j*h/6;
   S = 0; A = 0;
   for k = 1:n,
      S = S + h/840 * (41*f(x(6*k,0)) + 216*f(x(6*k,1)) + 27*f(x(6*k,2)) + 272*f(x(6*k,3)) + 27*f(x(6*k,4)) + 216*f(x(6*k,5)) + 41*f(x(6*k,6)));
      A = A + 6;
   end
   return %Aufwand 6n
end
\end{lstlisting}

\begin{description}
\item[a)] 
Die Auswertungen wurden sowohl bei a) als auch bei b) mit folgendem Code gemacht (natürlich bei beiden Teilaufgaben mit verschiedenen Funktionen f und Grenzen ab)
\begin{lstlisting}[language=matlab]
>> for k = 1:50,                         
erg(k) = sumnc(ab,f,1,6*k);           
end
>> for k = 1:50,              
erg2(k) = sumnc(ab,f,2,6*k);
end
>> for k = 1:50,               
erg3(k) = sumnc(ab,f,6,6*k);
end
\end{lstlisting}

Und dann mit diesem Code geplottet (Achsenbenennung weggelassen):

\begin{lstlisting}[language=matlab]
h=loglog(6:6:6*50,abs(2-erg)) 
h2=loglog(6:6:6*50,abs(2-erg2)) 
h3=loglog(6:6:6*50,abs(2-erg3)) 
\end{lstlisting}

Dies kann man machen, da das exakte Integral $\int_0^{\pi} \sin(x) \, dx = 2$.

\begin{figure}[ht]
\centering
\subfigure[Fehler über Teilintervalle]{
	\includegraphics[scale=0.33]{diagrammeA/trapez.png}
}
\subfigure[Aufwand über Fehler]{
	\includegraphics[scale=0.33]{diagrammeA/trapezauf.png}
}
\caption{Newton-Cotes-Formel nach Trapez-Regel}
\end{figure}

\begin{figure}[ht]
\centering
\subfigure[Fehler über Teilintervalle]{
	\includegraphics[scale=0.33]{diagrammeA/simpson.png}
}
\subfigure[Aufwand über Fehler]{
	\includegraphics[scale=0.33]{diagrammeA/simpsonauf.png}
}
\caption{Newton-Cotes-Formel nach Simpson-Regel}
\end{figure}

\begin{figure}[ht]
\centering
\subfigure[Fehler über Teilintervalle]{
	\includegraphics[scale=0.33]{diagrammeA/weddle.png}
}
\subfigure[Aufwand über Fehler]{
	\includegraphics[scale=0.33]{diagrammeA/weddleauf.png}
}
\caption{Newton-Cotes-Formel nach Weddle-Regel}
\end{figure}

\newpage

\item[b)] Berechnung analog zu a). Hier ist der exakte Wert des Integrals $\int_0^1 \left( \sqrt(x) + \sin(21\pi x)\right) \, dx = \frac{2}{21} \left(7 + \frac{1}{\pi} \right)$. Dies haben wir so in matlab eingegeben und als Vergleichswert benutzt.

\begin{figure}[ht]
\centering
\subfigure[Fehler über Teilintervalle]{
	\includegraphics[scale=0.33]{diagrammeB/trapez.png}
}
\subfigure[Aufwand über Fehler]{
	\includegraphics[scale=0.33]{diagrammeB/trapezauf.png}
}
\caption{Newton-Cotes-Formel nach Trapez-Regel}
\end{figure}

\begin{figure}[ht]
\centering
\subfigure[Fehler über Teilintervalle]{
	\includegraphics[scale=0.33]{diagrammeB/simpson.png}
}
\subfigure[Aufwand über Fehler]{
	\includegraphics[scale=0.33]{diagrammeB/simpsonauf.png}
}
\caption{Newton-Cotes-Formel nach Simpson-Regel}
\end{figure}

\begin{figure}[ht]
\centering
\subfigure[Fehler über Teilintervalle]{
	\includegraphics[scale=0.33]{diagrammeB/weddle.png}
}
\subfigure[Aufwand über Fehler]{
	\includegraphics[scale=0.33]{diagrammeB/weddleauf.png}
}
\caption{Newton-Cotes-Formel nach Weddle-Regel}
\end{figure}

Schaut man sich die Integrale an, so stellt man fest, dass die erste Funktion eine glatten Bogen beschreibt und die zweite Funktion extrem springt.

\begin{figure}[ht]
\centering
\subfigure[$\int_0^{\pi} \sin(x) \, dx$]{
	\includegraphics[scale=0.5]{sin.png}
}
\subfigure[$\int_0^1 \left( \sqrt(x) + \sin(21\pi x)\right) \, dx$]{
	\includegraphics[scale=0.5]{snd.png}
}
\caption{Graphische Darstellung der beiden Integrale (von: Wolframalpha.com)}
\end{figure}

Hier fehlt noch eine Erklärung warum das irgendwie alles so ist, wie es ist. XXX

\end{description}


\newpage
%% ------------------------------------------------------
%%                     AUFGABE 2
%% ------------------------------------------------------

\section*{Aufgabe 2}

\label{LastPage}

\end{document}
