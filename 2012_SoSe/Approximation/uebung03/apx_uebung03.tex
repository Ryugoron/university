\documentclass[11pt,a4paper,ngerman]{article}
\usepackage[bottom=2.5cm,top=2.5cm]{geometry} 
\usepackage{babel}
\usepackage[utf8]{inputenc} 
\usepackage[T1]{fontenc} 
\usepackage{ae} 
\usepackage{amssymb} 
\usepackage{amsmath}
\usepackage{amsthm} 
\usepackage{graphicx}
\usepackage{fancyhdr}
\usepackage{fancyref}
\usepackage{listings}
\usepackage{xcolor}
\usepackage{paralist}

\usepackage[pdftex, bookmarks=false, pdfstartview={FitH}, linkbordercolor=white]{hyperref}
\usepackage{fancyhdr}
\pagestyle{fancy}
\fancyhead[C]{Approximation Algorithms}
\fancyhead[L]{Exercise 3}
\fancyhead[R]{SoSe 2012}
\fancyfoot{}
\fancyfoot[L]{}
\fancyfoot[C]{\thepage \hspace{1px} of \pageref{LastPage}}
\renewcommand{\footrulewidth}{0.5pt}
\renewcommand{\headrulewidth}{0.5pt}
\setlength{\parindent}{0pt} 
\setlength{\headheight}{0pt}

\date{}
\title{Max Wisniewski}
\author{Dozent : Panos Giannopoulos}


%%
%% Enviroments for proofs and lemmas
%%
\newtheorem{lemma}{\bfseries Lemma}

\begin{document}

\lstset{language=Pascal, basicstyle=\ttfamily\fontsize{10pt}{10pt}\selectfont\upshape, commentstyle=\rmfamily\slshape, keywordstyle=\rmfamily\bfseries, breaklines=true, frame=single, xleftmargin=3mm, xrightmargin=3mm, tabsize=2, mathescape=true}

\renewcommand{\figurename}{Figure}

\maketitle
\thispagestyle{fancy}

%% ------------------------------------------------------
%%                     Exercise 1
%% ------------------------------------------------------

\section*{Task 3: \mdseries (Makespan - Problem)}

In this exercise we should prove the following Lemma on the Makespan-Problem.

\begin{description}
    \item{\bfseries Lemma 2.8:} 
        {\rmfamily\itshape
            For any input to the problem of minimizing the makespan on
            identical parallel maschines for which the processing
            requirements of each job is more than one-third the 
            optimal makespan, the longest processing time rule
            computes an optimal schedule.
        }
    \item{\bfseries Proof:}\\
        Let $m$ be the number of machines, and 
        $J = \left\{ 1,... ,n\right\}$ the jobs, where $p_i, i \in J$
        is the processing-time. $c_i, \, i \in J$ is the time at which
        Job $i$ is finished.
        Let $C_{\max} = \underset{j\in J}{\max} \; c_j$ the value of the
        optimal solution and $C_{\max}^*$ the computationtime of the longest
        processing rule.
        Let $e \; : \; \{1..m\} \rightarrow 2^J $ be the function, that states, which
        machine executes which jobs.
 
    \vspace{2\lineskip}

        First obeserve, that in the optimal solution each machine can
        execute at most two jobs. Next we can assume, that the in the
        optimal solution the machines that executes two jobs, the
        longes one will allways be executed first.

    \vspace{2\lineskip}

        Now we show, that we can modify the optimal solution
        s.t. we have a solution, that satisfies the longest processing
        time rule.

    \vspace{2\lineskip}

        The Idea of the following algorithm is, that we look at two machines.
        If the second job of the first machine has a bigger executiontime
        than the first job of the seconde machine. (Assuming the $p_{first} > p_{seconde}$).
        Than we put the second job of the first machine as the first job of the second
        machine and take the second job of the second machine and make it to the second job
        of the first machine. After this change at least these two machines satisfy the
        longest processing time rule.

        \begin{lstlisting}
WHILE ($\exists$ i,j : $p_{e(i)[1]}$ > $p_{e(j)[0]}$ DO
    save := e(j)[1];
    e(j)[1] := e(j)[0];
    e(j)[0] := e(i)[1];
    e(i)[1] := save;
        \end{lstlisting} 
\pagebreak
        After this execution on the specific optimal schedule there hold two claims. The
        new schedule is $C_{\max}'$.
        \begin{lemma} \label{apx:ueb3:cmax}
            {\itshape
            $C_{\max}' \leq C_{\max}$
            }
        \end{lemma}
        \begin{lemma} \label{apx:ueb3:lastreordering}
            {\itshape
                If you take the reordered schedule from lemma \ref{apx:ueb3:cmax}
                you can exchange the following jobs on the machines and
                the maximal computation time is not worsen.\\
                If $i$ is the last job on one machine and $j$ is the last
                job on another machine, and $p_i>p_j \, \land \, c_i-p_i > c_j-p_j$
                holds, the two jobs are exchanged from one machine to
                another.
            }
        \end{lemma}
        \begin{lemma} \label{apx:ueb3:latest}
            {\itshape
            $C_{\max}'$ after lemma \ref{apx:ueb3:lastreordering} 
            satisfies the longest processing time rule.
            }
        \end{lemma}
        \begin{description}
            \item{\bfseries\rmfamily Proof \ref{apx:ueb3:cmax}. }
                We show, that in after each iteration $C_{\max}$ is
                monotonically decreasing. Let $(k_1,k_2)$ and $(t_1,t_2)$ be the
                jobs handeled by the two choosen processes. We assume, that the first
                one is greater, namely $k_1 \geq k_2 > t_1 \geq t_2$.
                The complete runtime of the first process is $k_1 + k_2$ and the seconde
                one runs in $t_1 + t_2$.
                After the exchange the first process handels $k_1+t_2$ and the second
                one handels are $k_2+t_1$. As we can see, both processes have with the
                underlaying mentioned relation, that $k_1+t_2 \leq k_1 + k_2$ and 
                $k_2+t_1 \leq k_1+t_2$.\\

                We conclude, that after each iteration the solution gets better.
            \item{\bfseries\rmfamily Proof \ref{apx:ueb3:lastreordering}. }
                For the computation time holds $C_{\max}\geq c_i$ and $C_{\max} \geq c_j$.
                Now we substitute the conditions for the change in the
                formula $c_{\max} \geq c_i-p_i + p_i > c_i - p_i + p_j$. The time
                on this machine is lessen. An for other machine
                $C_{\max} \geq c_j-p_j + p_j> c_j - p_j + p_i $ holds, because $c_j-p_j < c_i-p_i$.
            \item{\bfseries\rmfamily Proof \ref{apx:ueb3:latest}. }
                We assume there is a job $s$ that is longer than a job, that started
                earlier we name $t$. Because every job starts either at point zero or as the second
                the job $s$ has to be the second one and the job $t$ has to be the first one.
                
                But under these circumstances the algorithm would have exchanged the
                the jobs of these machines. Therefore there could not be two jobs as assumed
                and the lemma holds.
        \end{description}
        Considering these three lemmas, we conclude from lemma \ref{apx:ueb3:cmax} and 
        \ref{apx:ueb3:lastreordering} that the reordering will not change the schedule, because 
        otherwise there would be a smaller solution than the minimum.

        With lemma \ref{apx:ueb3:latest} we conclude, that the longest processing time rule
        will infact delifer an optimal solution. That is because the longest processing 
        time rule  algorithm will deliver always the same result in the value . 
        So this solution will always have the optimal computation time.
\end{description}

\label{LastPage}

\end{document}
