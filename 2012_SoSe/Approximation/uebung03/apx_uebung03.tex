\documentclass[11pt,a4paper,ngerman]{article}
\usepackage[bottom=2.5cm,top=2.5cm]{geometry} 
\usepackage{babel}
\usepackage[utf8]{inputenc} 
\usepackage[T1]{fontenc} 
\usepackage{ae} 
\usepackage{amssymb} 
\usepackage{amsmath} 
\usepackage{graphicx}
\usepackage{fancyhdr}
\usepackage{fancyref}
\usepackage{listings}
\usepackage{xcolor}
\usepackage{paralist}

\usepackage[pdftex, bookmarks=false, pdfstartview={FitH}, linkbordercolor=white]{hyperref}
\usepackage{fancyhdr}
\pagestyle{fancy}
\fancyhead[C]{Approximation Algorithms}
\fancyhead[L]{Exercise 3}
\fancyhead[R]{SoSe 2012}
\fancyfoot{}
\fancyfoot[L]{}
\fancyfoot[C]{\thepage \hspace{1px} of \pageref{LastPage}}
\renewcommand{\footrulewidth}{0.5pt}
\renewcommand{\headrulewidth}{0.5pt}
\setlength{\parindent}{0pt} 
\setlength{\headheight}{0pt}

\date{}
\title{Max Wisniewski}
\author{Dozent : Panos Giannopoulos}

\newcommand{\claim}{\addtocounter{claims}{1} \bfseries Lemma \arabic{claims}}
\newcommand{\proof}{\bfseries Proof}


\begin{document}

\lstset{language=Pascal, basicstyle=\ttfamily\fontsize{10pt}{10pt}\selectfont\upshape, commentstyle=\rmfamily\slshape, keywordstyle=\rmfamily\bfseries, breaklines=true, frame=single, xleftmargin=3mm, xrightmargin=3mm, tabsize=2}

\renewcommand{\figurename}{Figure}
\newcounter{claims}

\maketitle
\thispagestyle{fancy}

%% ------------------------------------------------------
%%                     Exercise 1
%% ------------------------------------------------------

\section*{Task 3: \mdseries (Makespan - Problem)}

In this exercise we should prove the following Lemma on the Makespan-Problem.

\begin{description}
    \item{\bfseries Lemma 2.8:} 
        {\rmfamily\itshape
            For any input to the problem of minimizing the makespan on
            identical parallel maschines for which the processing
            requirements of each job is more than one-third the 
            optimal makespan, the longest processing time rule
            computes an optimal schedule.
        }
    \item{\bfseries Proof:}\\
        Let $m$ be the number of maschines, and 
        $J = \left\{ 1,... ,n\right\}$ the jobs, where $p_i, i \in J$
        is the processing-time. $c_i, \, i \in J$ is the time at which
        Job $i$ is finished.
        Let $C_{\max} = \underset{j\in J}{\max} \; c_j$ the value of the
        optimal solution and $C_{\max}^*$ the computationtime of the longest
        processing rule. 

        First obeserve, that in the optimal solution each machine can
        execute at most two jobs. Next we can assume, that the in the
        optimal solution the machines that executes two jobs, the
        longes one will allways be executed first.

        Now we show, that we can modify the optimal solution
        s.t. we have a solution, that satisfies the longest processing
        time rule. 
\end{description}

\label{LastPage}

\end{document}
