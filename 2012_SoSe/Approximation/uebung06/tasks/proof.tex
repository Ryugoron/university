\subsection*{The Proof}

In this part we will show, that the follwoing holds\\

\begin{lemma}\label{multcut:runtime}
The algorithm $\mathcal{A}$ has a runtime in $P$.
\end{lemma}
\begin{lemma}\label{multcut:apx}
The algorithm $\mathcal{A}$ is a 2-approximation-algorithm
for the multicut problem in trees.
\end{lemma}

\begin{description}
   \item{\bfseries\itshape~Proof ~\ref{multcut:runtime}:}\\
      We see, that we have at most $k$ iterations of the while loop,
      because we cut every iteration at least one pair.\\
      In each iteration we first search the maximal depth, which we can
      obtain in $k \log \, k$ overall iterations (MaxHeap).\\
      We can optain the $e_t$ in P time. The path from $s_i$ to $t_i$
      is for every pair at most $n = |V|$ (-1). This is the amount
      of elements we have to search for the minimum next.\\
      For each of these $n$ elements we have to find all Path $P_j$
      that contains the observed edge. This can be done in ineffciently in
      $k \cdot n$ for every edge (k Paths maximal and n elements in each).
      So we find the $e_t$ in $O(n^2 k)$. $\Delta$ can be found in less
      time as easily seen. The rest of the actions needs constant time,
      or at least less than the given.\\
      So the first loop has runtime $O(n^2k^2)$. The next loop checks, for
      pair, whether we can delete one of the at most $k$ taken edges.\\
      We can loop at each path in $O(n)$ of $k$ paths and this at most $k$ times
      which leads us to $O(nk^2)$.\\
      So this algorithm runs in $O(n^2k^2)$ which obviosly lays in P.

   \item{\bfseries\itshape~Proof ~\ref{multcut:apx}:}\\
      At first we observe, that $F$ is feasable.\\
      In line 5 of the algorithm, the loop terminates, only
      if there exists no pair of vertices in the set $I$.\\
      In the loop body we remove an index only if we met one
      edge with equality in the dual. Due to complemantry slagness rule
      we know, that we have the edge in the Primal and so we removed an
      edge from the unique way from a the pair $i$.\\
      We conclude, that the set $F$ was feasable after the first loop.
      The seconde loop, will only remove an edge, if the set still remains feasable.\\
      So the resulting set $F$ must be feasable.\\

     TODO: Flow $\leq 2 \cdot$ Multicut,\\
          weg über Kanten, die man über einen $lca(s_i,t_i)$ laufen lässt.
\end{description}
