\documentclass[11pt,a4paper,ngerman]{article}
\usepackage[bottom=2.5cm,top=2.5cm]{geometry} 
\usepackage{babel}
\usepackage[utf8]{inputenc} 
\usepackage[T1]{fontenc} 
\usepackage{ae} 
\usepackage{amssymb} 
\usepackage{amsmath}
\usepackage{amsthm} 
\usepackage{graphicx}
\usepackage{fancyhdr}
\usepackage{caption}
\usepackage{subcaption}
\usepackage{fancyref}
\usepackage{enumerate}
\usepackage{listings}
\usepackage{xcolor}
\usepackage{paralist}
\usepackage{tabularx}

\usepackage[pdftex, bookmarks=false, pdfstartview={FitH}, linkbordercolor=white]{hyperref}
\usepackage{fancyhdr}
\pagestyle{fancy}
\fancyhead[C]{Numerik I}
\fancyhead[L]{Übung 8}
\fancyhead[R]{SoSe 2013}
\fancyfoot{}
\fancyfoot[L]{}
\fancyfoot[C]{\thepage \hspace{1px} of \pageref{LastPage}}
\renewcommand{\footrulewidth}{0.5pt}
\renewcommand{\headrulewidth}{0.5pt}
\setlength{\parindent}{0pt} 
\setlength{\headheight}{0pt}

\date{Tutor: Christina Schulz}
\title{Übung 8}
\author{Max Wisniewski, Alexander Steen}


%%
%% Enviroments for proofs and lemmas
%%
\newtheorem{lemma}{\bfseries Claim}

\begin{document}

\lstset{language=Pascal, basicstyle=\ttfamily\fontsize{10pt}{10pt}\selectfont\upshape, commentstyle=\rmfamily\slshape, keywordstyle=\rmfamily\bfseries, breaklines=true, frame=single, xleftmargin=3mm, xrightmargin=3mm, tabsize=2, mathescape=true}

\renewcommand{\figurename}{Figure}

\maketitle
\thispagestyle{fancy}

%%%%%%%%%%%%%%%%%%%%%%%%%%%%%%
%% Aufgabe 1     %%%%%%%%%%%%%%%%
%%%%%%%%%%%%%%%%%%%%%%%%%%%%%%
\subsection*{Aufgabe 1}

\begin{lstlisting}[language=matlab,numbers=left]
f1 = @(x) 1/(2*atan(1)) * 1/(1+x^2);
f2 = @(x) 1/(2*atan(500)) * 1/(1+(500*x)^2);
rombergQuad = zeros(12,2);
trapezQuad = zeros(12,2);
fehlerRomberg = zeros(12,2);
fehlerTrapez = zeros(12,2);
for i = 1:12
  trapezQuad(i,1) = trapQuad(-1:2^(-i):1,f1);
  trapezQuad(i,2) = trapQuad(-1:2^(-i):1,f2);
  rombergQuad(i,1) = romberg(-1,1,f1,2.^(-[0:i]));
  rombergQuad(i,2) = romberg(-1,1,f2,2.^(-[0:i]));
  fehlerTrapez(i,1) = abs(trapezQuad(i,1) - 1);
  fehlerTrapez(i,2) = abs(trapezQuad(i,2) - 1);
  fehlerRomberg(i,1) = abs(rombergQuad(i,1) - 1);
  fehlerRomberg(i,2) = abs(rombergQuad(i,2) - 1);
end
\end{lstlisting}

Es ergibt sich

\begin{lstlisting}
fehlerTrapez =

   0.013239352830249   0.840636419672265
   0.003315574025695   0.920306878383561
   0.000828929586313   0.960129548066769
   0.000207232961173   0.980015760024551
   0.000051808249116   0.989908833068686
   0.000012952062417   0.994757042698583
   0.000003238015606   0.996997544736434
   0.000000809503901   0.997832792863140
   0.000000202375976   0.997993557291557
   0.000000050593994   0.997999989669726
   0.000000012648499   0.998000000000075
   0.000000003162123   0.998000000000025

fehlerRomberg =

   0.002629023290789   0.893754213535242
   0.000167110611374   0.950404330248577
   0.000002186734143   0.975566264743692
   0.000000003720382   0.987770983969616
   0.000000000015422   0.993770542341976
   0.000000000000023   0.996645736088292
   0.000000000000001   0.997862028038352
   0.000000000000001   0.998141905293127
   0.000000000000001   0.998040882362203
   0.000000000000000   0.997998214562390
   0.000000000000001   0.997999880717347
   0.000000000000002   0.998000002633203

\end{lstlisting}

%%%%%%%%%%%%%%%%%%%%%%%%%%%%%%
%% Aufgabe 2     %%%%%%%%%%%%%%%%
%%%%%%%%%%%%%%%%%%%%%%%%%%%%%%
\subsection*{Aufgabe 2}


\label{LastPage}
\end{document}
