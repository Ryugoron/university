\documentclass[11pt,a4paper,ngerman]{article}
\usepackage[bottom=2.5cm,top=2.5cm]{geometry} 
\usepackage{babel}
\usepackage[utf8]{inputenc} 
\usepackage[T1]{fontenc} 
\usepackage{ae} 
\usepackage{amssymb} 
\usepackage{amsmath}
\usepackage{amsthm} 
\usepackage{graphicx}
\usepackage{fancyhdr}
\usepackage{fancyref}
\usepackage{enumerate}
\usepackage{listings}
\usepackage{xcolor}
\usepackage{paralist}

\usepackage[pdftex, bookmarks=false, pdfstartview={FitH}, linkbordercolor=white]{hyperref}
\usepackage{fancyhdr}
\pagestyle{fancy}
\fancyhead[C]{Numerik I}
\fancyhead[L]{Übung 2}
\fancyhead[R]{SoSe 2013}
\fancyfoot{}
\fancyfoot[L]{}
\fancyfoot[C]{\thepage \hspace{1px} of \pageref{LastPage}}
\renewcommand{\footrulewidth}{0.5pt}
\renewcommand{\headrulewidth}{0.5pt}
\setlength{\parindent}{0pt} 
\setlength{\headheight}{0pt}

\date{}
\title{Übung 2}
\author{Max Wisniewski, Alexander Steen}


%%
%% Enviroments for proofs and lemmas
%%
\newtheorem{lemma}{\bfseries Claim}

\begin{document}

\lstset{language=Pascal, basicstyle=\ttfamily\fontsize{10pt}{10pt}\selectfont\upshape, commentstyle=\rmfamily\slshape, keywordstyle=\rmfamily\bfseries, breaklines=true, frame=single, xleftmargin=3mm, xrightmargin=3mm, tabsize=2, mathescape=true}

\renewcommand{\figurename}{Figure}

\maketitle
\thispagestyle{fancy}


\subsection*{Aufgabe 1}
Seien $V = (\mathbb{R}^2,\| . \|_2)$ ein normierter Vektorraum und
$U = \{(u_1,u_2) \in V| u_1 - u_2 = 0 \} \subseteq V$ ein Unterraum. Sei weiterhin $f = (2,4) \in V$.

\begin{enumerate}[a)]
\item Bestapproximationsaufgabe: \\
      Finde $u \in U$, sodass $\forall v \in U: \, \|f-u\|_2 \leq \|f-v\|_2$. \\
      Existenz der Lösung: \\
      asd \\
      Eindeutigkeit der Lösung: \\
      asdasd
      
\item Gesucht: Zur Bestapproximationsaufgabe äquivalente Normalengleichung 
\item bla
\end{enumerate}
\subsection*{Aufgabe 2}
\subsection*{Aufgabe 3}

\label{LastPage}

\end{document}
