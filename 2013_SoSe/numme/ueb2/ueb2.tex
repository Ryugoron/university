\documentclass[11pt,a4paper,ngerman]{article}
\usepackage[bottom=2.5cm,top=2.5cm]{geometry} 
\usepackage{babel}
\usepackage[utf8]{inputenc} 
\usepackage[T1]{fontenc} 
\usepackage{ae} 
\usepackage{amssymb} 
\usepackage{amsmath}
\usepackage{amsthm} 
\usepackage{graphicx}
\usepackage{fancyhdr}
\usepackage{fancyref}
\usepackage{enumerate}
\usepackage{listings}
\usepackage{xcolor}
\usepackage{paralist}

\usepackage[pdftex, bookmarks=false, pdfstartview={FitH}, linkbordercolor=white]{hyperref}
\usepackage{fancyhdr}
\pagestyle{fancy}
\fancyhead[C]{Numerik I}
\fancyhead[L]{Übung 2}
\fancyhead[R]{SoSe 2013}
\fancyfoot{}
\fancyfoot[L]{}
\fancyfoot[C]{\thepage \hspace{1px} of \pageref{LastPage}}
\renewcommand{\footrulewidth}{0.5pt}
\renewcommand{\headrulewidth}{0.5pt}
\setlength{\parindent}{0pt} 
\setlength{\headheight}{0pt}

\date{}
\title{Übung 2}
\author{Max Wisniewski, Alexander Steen}


%%
%% Enviroments for proofs and lemmas
%%
\newtheorem{lemma}{\bfseries Claim}

\begin{document}

\lstset{language=Pascal, basicstyle=\ttfamily\fontsize{10pt}{10pt}\selectfont\upshape, commentstyle=\rmfamily\slshape, keywordstyle=\rmfamily\bfseries, breaklines=true, frame=single, xleftmargin=3mm, xrightmargin=3mm, tabsize=2, mathescape=true}

\renewcommand{\figurename}{Figure}

\maketitle
\thispagestyle{fancy}


\subsection*{Aufgabe 1}
Seien $V = (\mathbb{R}^2,\| . \|_2)$ ein normierter Vektorraum und
$U = \{(u_1,u_2) \in V| u_1 - u_2 = 0 \} \subseteq V$ ein Unterraum. Sei weiterhin $f = (2,4) \in V$.

\begin{enumerate}[a)]
\item Bestapproximationsaufgabe: \\
      Finde $u \in U$, sodass $\forall v \in U: \, \|f-u\|_2 \leq \|f-v\|_2$. \\
      Existenz der Lösung: \\
      Nach Satz 2.1 ist die Bestapproximationsaufgabe in einem normierten, linearen Raum immer lösbar. \\
      Eindeutigkeit der Lösung: \\
      Nach Satz 2.2 ist die Lösung eindeutig, falls $V$ strikt konvex ist. Nun ist $V$ strikt konvex, da
      $V$ ein Prähilbertraum ist (die Norm kommt von einem Skalarprodukt). Also ist die Lösung eindeutig.
      
\item Gesucht: Zur Bestapproximationsaufgabe äquivalente Normalengleichung 
    \begin{equation*}\begin{split}
      (u-f,v) = 0 
      & \Leftrightarrow  v_1(u_1-f_1) + v_2(u_2-f_2) = 0 \\
      &\Leftrightarrow  v_1 \left( (u_1-f_1) + (u_1-f_2) \right) = 0 \\
      &\Leftrightarrow  u_1 = u_2 = \frac{f_1+f_2}{2}
    \end{split}\end{equation*}
    Also ergibt sich für die Bestapproximation von $f$ der Vektor, der komponentenweise aus dem Durchschnitt
    des Komponenten von $f$ besteht.
\item Projektion: \\
    Wähle als Projektion $P : V \to U$ mit $P(x,y) = \left(\frac{x+y}{2},\frac{x+y}{2} \right)$. Dann ist $P$ Orthogonalprojektion, denn
    $(v,(I - P) f) = \left(\frac{x+y}{2},\frac{x+y}{2} \right),\left(\frac{1}{2}(f_1-f_2),\frac{1}{2}(f_2-f_1)\right) = 0$, mit $v = P(x,y) = \left(\frac{x+y}{2},\frac{x+y}{2} \right) \in R(P), f = (f_1,f_2) \in V$. \\
    Nach Satz 2.8 ist dann die Lösung der Bestapproximationsaufgabe $u = Pf = P(2,4) = (3,3)$.

\end{enumerate}
\subsection*{Aufgabe 2}
Zu zeigen: $(C[0,1],(\cdot,\cdot))$ ist mit $(v,w) = \int_0^1 v(x)w(x)\,dx$ kein Hilbertraum. \\
\textbf{Beweis}: Wir zeigen, dass eine Cauchyfolge $\left( f_n \right)$ mit $f_i \in C[0,1]$
                 nicht in $C[0,1]$ konvergiert. \\

  Sei $f_n (x) = \begin{cases}
                nx & \text{, für $x \in [0,\frac{1}{n}]$} \\
                1 & \text{, für $x \in (\frac{1}{n},1]$}
             \end{cases}$ für alle $n \in \mathbb{N}$. Dann ist jedes $f_n \in C[0,1]$ und es gilt
  \begin{equation*}
    \left( f_n \right) \stackrel{n \to \infty}{\longrightarrow} sgn_{{\left. \right|}_{[0,1]}} \text{in $L^2$}
  \end{equation*}
  wobei $sgn_{{\left. \right|}_{[0,1]}}$ die auf das Intervall $[0,1]$ eingeschränkte signum-Funktion ist.
  Da $\left(f_n\right)$ eine Cauchyfolge in $L^2$ und $C[0,1]$ ein Unterraum von $L^2$,
  ist $\left(f_n\right)$ ebenfalls Cauchyfolge in $C[0,1]$ (siehe Analysis III).
  Allerdings ist gilt $ sgn_{{\left. \right|}_{[0,1]}} \notin C[0,1]$.
  Damit ist $C[0,1]$ nicht vollständig, also kein Hilbertraum.
  
\subsection*{Aufgabe 3}
\begin{enumerate}[a)]
\item Listing~\ref{lst:approx} zeigt ein Matlab-Programm zur numerischen Bestapproximation von $e^x$ durch ein Polynom aus $p_n$. Dabei wird (wie im Skript gezeigt) ein LGS der Form $Ac = b$ mit
$$ A = (a_{ij}), a_{ij} = (x^i,x^j) \quad b = (b_i), b_i = (f,x^i) $$
wobei als Basis des $p_n$ die Monombasis ${1,x,x^2,\ldots,x^n}$ gewählt wurde.
Da das Skalarprodukt $(u,v)$ hier definiert ist als 
$$ (u,v) = \int_0^2 u(x)v(x)\,dx $$
können wir vereinfachen:
$$ a_{ij} = (x^i,x^j) = \int_0^2 u(x)v(x)\,dx = \frac{2^{i+j+1}}{i+j+1} $$
Die Einträge von $b$ müssen je nach Eingabe von $f$ vom Programm ausgerechnet werden (siehe Ziele 21).
Anschließend wird das LGS durch die solve-Funktion gelöst.

\lstinputlisting[label=lst:approx,numbers=left,caption=Bestapproximation,captionpos=b]{bestapprox.m}

\item Leider konnten wir das Programm nicht testen, da (1) die Matlab-Installation die erste Hälfte der Woche nicht erreichbar war über die Poolrechner und (2) leider die Lizenzen für das symbolic math-Paket abgelaufen waren und darum nicht ausgeführt werden konnten.
Das Programm sollte (hoffentlich) mit steigendem $n$ immer genauere Approximationen von $e^x$ zeigen.\\
\end{enumerate}
\label{LastPage}

\end{document}
