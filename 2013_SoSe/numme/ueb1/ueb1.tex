\documentclass[11pt,a4paper,ngerman]{article}
\usepackage[bottom=2.5cm,top=2.5cm]{geometry} 
\usepackage{babel}
\usepackage[utf8]{inputenc} 
\usepackage[T1]{fontenc} 
\usepackage{ae} 
\usepackage{amssymb} 
\usepackage{amsmath}
\usepackage{amsthm} 
\usepackage{graphicx}
\usepackage{fancyhdr}
\usepackage{fancyref}
\usepackage{listings}
\usepackage{xcolor}
\usepackage{paralist}

\usepackage[pdftex, bookmarks=false, pdfstartview={FitH}, linkbordercolor=white]{hyperref}
\usepackage{fancyhdr}
\pagestyle{fancy}
\fancyhead[C]{Numerik I}
\fancyhead[L]{Übung 1}
\fancyhead[R]{SoSe 2013}
\fancyfoot{}
\fancyfoot[L]{}
\fancyfoot[C]{\thepage \hspace{1px} of \pageref{LastPage}}
\renewcommand{\footrulewidth}{0.5pt}
\renewcommand{\headrulewidth}{0.5pt}
\setlength{\parindent}{0pt} 
\setlength{\headheight}{0pt}

\date{}
\title{Übung 1}
\author{Max Wisniewski, Alexander Steen}


%%
%% Enviroments for proofs and lemmas
%%
\newtheorem{lemma}{\bfseries Claim}

\begin{document}

\lstset{language=Pascal, basicstyle=\ttfamily\fontsize{10pt}{10pt}\selectfont\upshape, commentstyle=\rmfamily\slshape, keywordstyle=\rmfamily\bfseries, breaklines=true, frame=single, xleftmargin=3mm, xrightmargin=3mm, tabsize=2, mathescape=true}

\renewcommand{\figurename}{Figure}

\maketitle
\thispagestyle{fancy}


\subsection*{Aufgabe 1}
Es sei $g(x) = x + \frac{1}{1+x}$ und $M = \{ x \geq 0 \}$.

\begin{enumerate}
\item $g(M) \subseteq M$\\
  Sei $x \in M$, dann gilt 
  \begin{equation*}
    g(x) = \underbrace{x}_{\geq 0} + \underbrace{\frac{1}{1+x}}_{\geq 0} \geq 0
  \end{equation*}
  Also ist $g(x) \in M \Rightarrow g(M) \subseteq M$.
\item $|g(x) - g(y)| < |x-y|$ für $x \neq y$\\
  Seien $x,y \in M$, $x \neq y$. Sei weiterhin o.B.d.A. $x > y$. Dann gilt
  \begin{equation*}\begin{split}
    |g(x) - g(y)| &= |x + \frac{1}{1+x} - y - \frac{1}{1+y}|\\
                  &= |\underbrace{x - y}_{> 0} + \underbrace{\frac{1}{1+x} - \frac{1}{1+y}}_{< 0}| \\
                  &< |x - y|
  \end{split}\end{equation*}
\item $g$ besitzt keinen Fixpunkt in $M$\\
  Beweis durch Widerspruch: Sei $x^* \in M$ Fixpunkt von $g$. Dann gilt
  \begin{equation*}\begin{split}
    g(x^*) &= x^* = x^* + \frac{1}{1+ x^*}\\
    &\Leftrightarrow 0 = \frac{1}{1 + x^*}
  \end{split}\end{equation*}
  Das ist aber ein Widerspruch, da es keine Zahl $x$ gibt, für die $\frac{1}{1+x} = 0$ gilt.
  \mbox{} \hfill $\square$
\end{enumerate}
Dies ist kein Widerspruch zum Banachschen Fixpunktsatz, da es sich bei $g$ nicht um eine
Kontraktion handelt: Da $\frac{1}{x+1} \stackrel{x \to \infty}{\to} 0$ und damit
\begin{equation*}\begin{split}
|g(x) - g(y)| &= |\underbrace{\frac{1}{x+1} - \frac{1}{y+1}}_{\stackrel{x,y \to \infty}{\to} 0}  + x- y| \\
\end{split}\end{equation*} 
Für jedes feste $\alpha \in [0,1)$ ist $|g(x) - g(y)| \to |x-y| > \alpha|x-y|$, für $x,y$ groß genug.

\subsection*{Aufgabe 2}
Sei $F: \mathbb{R}^2 \to \mathbb{R}^2$ gegeben durch 
$$ F(x) := F(x_1,x_2) := \left(\begin{array}{c}
\frac{1}{3}x_2^2+\frac{1}{8}\\
\frac{1}{4}x_1^2-\frac{1}{6}\end{array} \right)$$

\subsubsection*{a)}
Zu zeigen: Es existiert ein eindeutiger Fixpunkt von $F$ in 
$D := \{x\in \mathbb{R}^2 | \left|x\right|_\infty \leq 1 \}$. \\

\textbf{Beweis}: (1) $F(D) \subseteq D$ \\
Sei $x \in D$. Dann gilt
\begin{equation*}\begin{split}
\left|F(x)\right|_\infty &=  \left|\left(\begin{array}{c}
\frac{1}{3}x_2^2+\frac{1}{8}\\
\frac{1}{4}x_1^2-\frac{1}{6}\end{array} \right)\right|_\infty \\
     &= \max \{ \underbrace{|\frac{1}{3}\overbrace{x_2^2}^{\leq 1}+\frac{1}{8}|}_{\leq 1}, \quad 
                \underbrace{|\frac{1}{4}\overbrace{x_1^2}^{\leq 1}-\frac{1}{6} |}_{\leq 1} \} \\
     &\Rightarrow x \in D
\end{split}\end{equation*}
(2) $F$ ist Kontraktion\\
asd
\subsubsection*{b)}
\begin{lstlisting}[language=matlab]
f = @(x) (1/3 * x(2)^2 + 1/8, 1/4 * x(1)^2 + 1/6);
lambda = 0.99 //lipschitz-konstante

lastx = (1,1); // x0: Startwert mit |x0| <= 1
x = f(1,1); // F(x0)

while lambda/(1-lambda) * norm(lastx - x,inf) > 1.0e-8 
  x = f(x); // x(n+1) = F(x(n))
end;

return x;
\end{lstlisting}
\subsection*{Aufgabe 3}

\label{LastPage}

\end{document}
