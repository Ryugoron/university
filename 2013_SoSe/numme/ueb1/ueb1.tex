\documentclass[11pt,a4paper,ngerman]{article}
\usepackage[bottom=2.5cm,top=2.5cm]{geometry} 
\usepackage{babel}
\usepackage[utf8]{inputenc} 
\usepackage[T1]{fontenc} 
\usepackage{ae} 
\usepackage{amssymb} 
\usepackage{amsmath}
\usepackage{amsthm} 
\usepackage{graphicx}
\usepackage{fancyhdr}
\usepackage{fancyref}
\usepackage{listings}
\usepackage{xcolor}
\usepackage{paralist}

\usepackage[pdftex, bookmarks=false, pdfstartview={FitH}, linkbordercolor=white]{hyperref}
\usepackage{fancyhdr}
\pagestyle{fancy}
\fancyhead[C]{Numerik I}
\fancyhead[L]{Übung 1}
\fancyhead[R]{SoSe 2013}
\fancyfoot{}
\fancyfoot[L]{}
\fancyfoot[C]{\thepage \hspace{1px} of \pageref{LastPage}}
\renewcommand{\footrulewidth}{0.5pt}
\renewcommand{\headrulewidth}{0.5pt}
\setlength{\parindent}{0pt} 
\setlength{\headheight}{0pt}

\date{}
\title{Übung 1}
\author{Max Wisniewski, Alexander Steen}


%%
%% Enviroments for proofs and lemmas
%%
\newtheorem{lemma}{\bfseries Claim}

\begin{document}

\lstset{language=Pascal, basicstyle=\ttfamily\fontsize{10pt}{10pt}\selectfont\upshape, commentstyle=\rmfamily\slshape, keywordstyle=\rmfamily\bfseries, breaklines=true, frame=single, xleftmargin=3mm, xrightmargin=3mm, tabsize=2, mathescape=true}

\renewcommand{\figurename}{Figure}

\maketitle
\thispagestyle{fancy}


\subsection*{Aufgabe 1}
Es sei $g(x) = x + \frac{1}{1+x}$ und $M = \{ x \geq 0 \}$.

\begin{enumerate}
\item $g(M) \subseteq M$\\
  Sei $x \in M$, dann gilt 
  \begin{equation*}
    g(x) = \underbrace{x}_{\geq 0} + \underbrace{\frac{1}{1+x}}_{\geq 0} \geq 0
  \end{equation*}
  Also ist $g(x) \in M \Rightarrow g(M) \subseteq M$.
\item $|g(x) - g(y)| < |x-y|$ für $x \neq y$\\
  Seien $x,y \in M$, $x \neq y$. Sei weiterhin o.B.d.A. $x > y$. Dann gilt
  \begin{equation*}\begin{split}
    |g(x) - g(y)| &= |x + \frac{1}{1+x} - y - \frac{1}{1+y}|\\
                  &= |\underbrace{x - y}_{> 0} + \underbrace{\frac{1}{1+x} - \frac{1}{1+y}}_{< 0}| \\
               %   &\leq |x-y| + |\frac{1}{1+x} - \frac{1}{1+y}|\\
                  &= \left| x - y +\frac{x-y}{(1+x)(1+y)} \right|\\
                  &\overset{(1+x)(1+y) > 1}{<} |x - y|
  \end{split}\end{equation*}
    The last line holds, because we do not remove more than $2|x-y|$ such that we cannot remove to much.
\item $g$ besitzt keinen Fixpunkt in $M$\\
  Beweis durch Widerspruch: Sei $x^* \in M$ Fixpunkt von $g$. Dann gilt
  \begin{equation*}\begin{split}
    g(x^*) &= x^* = x^* + \frac{1}{1+ x^*}\\
    &\Leftrightarrow 0 = \frac{1}{1 + x^*}
  \end{split}\end{equation*}
  Das ist aber ein Widerspruch, da es keine Zahl $x$ gibt, für die $\frac{1}{1+x} = 0$ gilt.
  \mbox{} \hfill $\square$
\end{enumerate}
Dies ist kein Widerspruch zum Banachschen Fixpunktsatz, da es sich bei $g$ nicht um eine
Kontraktion handelt: Da $\frac{1}{x+1} \stackrel{x \to \infty}{\to} 0$ und damit
\begin{equation*}\begin{split}
|g(x) - g(y)| &= |\underbrace{\frac{1}{x+1} - \frac{1}{y+1}}_{\stackrel{x,y \to \infty}{\to} 0}  + x- y| \\
\end{split}\end{equation*} 
Für jedes feste $\alpha \in [0,1)$ ist $|g(x) - g(y)| \to |x-y| > \alpha|x-y|$, für $x,y$ groß genug.

\subsection*{Aufgabe 2}
Sei $F: \mathbb{R}^2 \to \mathbb{R}^2$ gegeben durch 
$$ F(x) := F(x_1,x_2) := \left(\begin{array}{c}
\frac{1}{3}x_2^2+\frac{1}{8}\\
\frac{1}{4}x_1^2-\frac{1}{6}\end{array} \right)$$

\subsubsection*{a)}
Zu zeigen: Es existiert ein eindeutiger Fixpunkt von $F$ in 
$D := \{x\in \mathbb{R}^2 | \left|x\right|_\infty \leq 1 \}$. \\

\textbf{Beweis}: (1) $F$ ist kontraktion. \\
(i) $F(D) \subseteq D$ \\
Sei $x \in D$. Dann gilt
\begin{equation*}\begin{split}
\left|F(x)\right|_\infty &=  \left|\left(\begin{array}{c}
\frac{1}{3}x_2^2+\frac{1}{8}\\
\frac{1}{4}x_1^2-\frac{1}{6}\end{array} \right)\right|_\infty \\
     &= \max \{ \underbrace{|\frac{1}{3}\overbrace{x_2^2}^{\leq 1}+\frac{1}{8}|}_{\leq 1}, \quad 
                \underbrace{|\frac{1}{4}\overbrace{x_1^2}^{\leq 1}-\frac{1}{6} |}_{\leq 1} \} \\
     &\Rightarrow x \in D
\end{split}\end{equation*}
(i) $\exists \lambda \in [0,1) \forall x,y \in D: \left|F(x) - F(y)\right|_\infty < \lambda \left|x-y\right|_\infty$. \\
Mit $\lambda = \frac{1}{3}$ gilt die Behauptung, denn
% Dann gilt $\left|F(x) - F(y)\right|_\infty < \lambda \left|x-y\right|_\infty$, für alle $x,y \in D$. \\
%\textbf{Beweis}: 
\begin{equation*}\begin{split}
  \left|F(x) - F(y)\right|_\infty &= \left|
\left(\begin{array}{c}
\frac{1}{3}x_2^2+\frac{1}{8}\\
\frac{1}{4}x_1^2-\frac{1}{6}\end{array} \right)
-
\left(\begin{array}{c}
\frac{1}{3}y_2^2+\frac{1}{8}\\
\frac{1}{4}y_1^2-\frac{1}{6}\end{array} \right)
\right|_\infty \\
&= 
\left|\left(\begin{array}{c}
\frac{1}{3}(x_2^2- y_2^2)\\
\frac{1}{4}(x_1^2-y_1^2)\end{array} \right)
\right|_\infty
\leq \frac{1}{3}
\left| x - y \right|_\infty
\end{split}\end{equation*}

(2) Da der $\mathbb{R}^2$ eine Banachraum und $F$ eine Kontraktion auf $D$ ist, gilt nach dem Banachschen Fixpunktsatz, dass ein eindeutiger Fixpunkt in $D$ existiert. 
\mbox{} \hfill $\square$

\subsubsection*{b)}
Listing~\ref{lst:fix} zeigt die Matlab-Funktion \texttt{myfixpoint}, die bei Eingabe von \\
(1) \texttt{f}: der zu betrachtenden Funktion $f$\\
(2) \texttt{lambda}: der Lipschitzkonstanten $\lambda$ von $f$ \\
(3) \texttt{start}: einem Startpunkt der Iteration, und
(4) \texttt{error}: dem Fehler, bei dem die Iteration abgerochen werden soll \\
näherungsweise den Fixpunkt von $f$ berechnet und als \texttt{x} zurückgibt.

\begin{lstlisting}[language=matlab,caption=Funktion zur näherungsweisen Bestimmung eines Fixpunkts, label=lst:fix]
function [x] = myfixpoint (f, lambda, start, error)
%% Fixpunktiteration fuer
%% Funktion f mit Lipschitzkonstante lambda
%% vom Startwert start und Abbruchfehler error.

%% Initialisierung der ersten beiden Folgenelementee
lastx = start;
x = f(start);

%% Iteration  mit Abbruchbedingung der a posteori-Abschaetzung
while lambda / (1 - lambda) * norm(lastx - x, inf) > error
    lastx = x;
    x = f(x);
end;
\end{lstlisting}

Wie in Listing~\ref{lst:test} zu sehen ist, wurde mit Hilfe dieser Funktion der Fixpunkt der Funktion $F$ bis auf einen Fehler von $10^{-8}$ auf $x_{fix} = (0.1338,-0,1622)$ bestimmt.

\begin{lstlisting}[language=matlab,label=lst:test,caption=Testaufruf der Funktion myfixpoint]
>> f = @(x) [1/3*x(2)^2 + 1/8, 1/4*x(1)^2 - 1/6]
>> myfixpoint(f,1/3,[1,1],10e-8)

ans =

    0.1338   -0.1622
\end{lstlisting}

\subsection*{Aufgabe 3}

Sei $f : [a,b] \rightarrow \mathbb{R}$ eine zweifach stetig differenzierbare konvexe Funktion mit den Eigenschaften
$$\begin{array}{l}
    f(a) > 0 \text{ und } f(b) > 0\\
    f'(x) > 0 \text{ und } f''(x) > 0 \text{ für }a \leq x \leq b.
\end{array}$$
Zeigen Sie, dass das Newtonverfahren mit $x_0 = b$ gegen die einzige Nullstelle konvergiert.\\

\textbf{Lösung:}\\
Da die Funktion monoton wächst ($f'(x) > 0$) wissen wir, dass die Funktion in $a$ ihr Minimum annimmt und in $b$ ihr Maximum.
Wir haben eine Nullstelle in diesem Intervall nach dem Zwischenwertsatz, da $f(a)<0$ und $f(b) > 0$ und $f$ stetig ist.
Es ist die einzige Nullstelle, da die Funktion monoton ist.\\

$f$ ist auf $[a,b]$ Lipschitz-stetig mit $L = f(b) - f(a)$, da es die maximale Differenz von Werten auf diesem Intervall ist.\\

Nun wissen wir, dass nach dem Mittelwertsatz ein $x_M \in [a,b]$ existiert mit $f'(x_M) = \frac{f(b) - f(a)}{b - a} = \tau$.
Nach Konvexität wissen wir, dass für $x > x_M$ $f'(x) > \frac{f(b) - f(a)}{b-a}$, desshalb haben wir auf dem Interval
$[x_M,b]$ eine Untereschranke für die erste Ableitung.

Nun gilt $x_M < x*$, da wir sonst and der Position $b$ $f(b)$ überschreiten würden. Damit wissen wir, wenn wir in $b$ starten,
dass $f'$ immer durch $\frac{f(b) - f(a)}{b-a}$ nach unten beschränkt ist.

Da $f'(x) > 0$ für alle $x \in [a,b]$ wissen wir, dass wir $f'$ invertieren können mit $f'(x)^{-1} = \frac{1}{f'(x)}$.\\

Nach dem Satz über die Konvergenz vom Newtonverfahren aus der VL wissen wir nun, dass eben dieses in einem Interval mit dem Radius
$$
    \frac{2\pi}{L \cdot \tau^{-1}} = \frac{2\pi}{(f(b)-f(a))\cdot\frac{b-a}{f(b) - f(a)}} = \frac{2\pi}{a-b}.
$$

Und nun haben wir komplett den Faden verloren.

\label{LastPage}

\end{document}
