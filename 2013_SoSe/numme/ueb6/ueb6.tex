\documentclass[11pt,a4paper,ngerman]{article}
\usepackage[bottom=2.5cm,top=2.5cm]{geometry} 
\usepackage{babel}
\usepackage[utf8]{inputenc} 
\usepackage[T1]{fontenc} 
\usepackage{ae} 
\usepackage{amssymb} 
\usepackage{amsmath}
\usepackage{amsthm} 
\usepackage{graphicx}
\usepackage{fancyhdr}
\usepackage{fancyref}
\usepackage{enumerate}
\usepackage{listings}
\usepackage{xcolor}
\usepackage{paralist}
\usepackage{tabularx}

\usepackage[pdftex, bookmarks=false, pdfstartview={FitH}, linkbordercolor=white]{hyperref}
\usepackage{fancyhdr}
\pagestyle{fancy}
\fancyhead[C]{Numerik I}
\fancyhead[L]{Übung 6}
\fancyhead[R]{SoSe 2013}
\fancyfoot{}
\fancyfoot[L]{}
\fancyfoot[C]{\thepage \hspace{1px} of \pageref{LastPage}}
\renewcommand{\footrulewidth}{0.5pt}
\renewcommand{\headrulewidth}{0.5pt}
\setlength{\parindent}{0pt} 
\setlength{\headheight}{0pt}

\date{Tutor : Christina Schulz}
\title{Übung 6}
\author{Max Wisniewski, Alexander Steen}


%%
%% Enviroments for proofs and lemmas
%%
\newtheorem{lemma}{\bfseries Claim}

\begin{document}

\lstset{language=Pascal, basicstyle=\ttfamily\fontsize{10pt}{10pt}\selectfont\upshape, commentstyle=\rmfamily\slshape, keywordstyle=\rmfamily\bfseries, breaklines=true, frame=single, xleftmargin=3mm, xrightmargin=3mm, tabsize=2, mathescape=true}

\renewcommand{\figurename}{Figure}

\maketitle
\thispagestyle{fancy}

%%%%%%%%%%%%%%%%%%%%%%%%%%%%%%
%% Aufgabe 1 %%%%%%%%%%%%%%%%
%%%%%%%%%%%%%%%%%%%%%%%%%%%%%%
\subsection*{Aufgabe 1}

Sei $S_\Delta^m$ der Raum der Splines $m$-ter Ordnung zum Gitter $\Delta = \{ t_0 , ..., t_n \}$.

Zeigen Sie
\begin{equation*}
    \dim S_\Delta^m = n + m.
\end{equation*}

\textbf{Beweis:}\\

Der Beweis steht schon fast vollständig im Skript.

Eine Funktion
\begin{equation*}
    f \in S_\Delta^m = \{ f \in C^{m-1}[a,b] \, | \, v|_{I_k} = p_k \in \mathcal{P}_m\}
\end{equation*}
wird wie man sieht durch $n$ Polynome beschrieben, die alle Grad $k$ haben.
Wir können $f$ also durch $n \cdot (m+1)$ Koeffizienten beschreiben.\\

Da die Funktion im Übergang $m-1$ fach stetig differenzierbar sein muss, gilt nach übergangsbedingung,
dass $p_k^j(x_k) = p_{k+1}^(j)(x_k)$ für alle $0 \leq j < m$ und $0 \leq k < n$.\\

Wir werden also mindestens einen Wert (den Funktionwert $p_k(x_k)$ für jede Funktion fixieren müssen,
da hier die Funktion über das Gitter definiert ist. Wir wissen, dass wir ein Polynom vom Grad $m$ durch
$m+1$ Koeffizienten oder $m+1$ Stützstellen beschreiben können.\\

Setzen wir also $2$ der Werte fest, so erhalten wir einen Freiheitsgrad von $m-1$, für die restlichen
Koeffizienten. Nun ist der Ring der Polynome ein Vektorraum und somit sind bei fixierung von $2$ Werten
die restlichen $m-1$ noch linear unabhängig, da sonst $\dim P_m < m$ sein müsste.\\

Folglich erhalten wir
\begin{equation*}
    \dim S_\Delta^m = \underbrace{n(m+1)}_\text{\#Mögliche Koeffizienten} - \underbrace{m(n-1)}_\text{linear unabhängige Koeffizienten} = n + m
\end{equation*}
\mbox{}\hfill$\square$
\label{LastPage}
\end{document}
