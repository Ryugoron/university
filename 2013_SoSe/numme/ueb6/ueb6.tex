\documentclass[11pt,a4paper,ngerman]{article}
\usepackage[bottom=2.5cm,top=2.5cm]{geometry} 
\usepackage{babel}
\usepackage[utf8]{inputenc} 
\usepackage[T1]{fontenc} 
\usepackage{ae} 
\usepackage{amssymb} 
\usepackage{amsmath}
\usepackage{amsthm} 
\usepackage{graphicx}
\usepackage{fancyhdr}
\usepackage{fancyref}
\usepackage{enumerate}
\usepackage{listings}
\usepackage{xcolor}
\usepackage{paralist}
\usepackage{tabularx}

\usepackage[pdftex, bookmarks=false, pdfstartview={FitH}, linkbordercolor=white]{hyperref}
\usepackage{fancyhdr}
\pagestyle{fancy}
\fancyhead[C]{Numerik I}
\fancyhead[L]{Übung 6}
\fancyhead[R]{SoSe 2013}
\fancyfoot{}
\fancyfoot[L]{}
\fancyfoot[C]{\thepage \hspace{1px} of \pageref{LastPage}}
\renewcommand{\footrulewidth}{0.5pt}
\renewcommand{\headrulewidth}{0.5pt}
\setlength{\parindent}{0pt} 
\setlength{\headheight}{0pt}

\date{Tutor : Christina Schulz}
\title{Übung 6}
\author{Max Wisniewski, Alexander Steen}


%%
%% Enviroments for proofs and lemmas
%%
\newtheorem{lemma}{\bfseries Claim}

\begin{document}

\lstset{language=Pascal, basicstyle=\ttfamily\fontsize{10pt}{10pt}\selectfont\upshape, commentstyle=\rmfamily\slshape, keywordstyle=\rmfamily\bfseries, breaklines=true, frame=single, xleftmargin=3mm, xrightmargin=3mm, tabsize=2, mathescape=true}

\renewcommand{\figurename}{Figure}

\maketitle
\thispagestyle{fancy}

%%%%%%%%%%%%%%%%%%%%%%%%%%%%%%
%% Aufgabe 1 %%%%%%%%%%%%%%%%
%%%%%%%%%%%%%%%%%%%%%%%%%%%%%%
\subsection*{Aufgabe 1}

Sei $S_\Delta^m$ der Raum der Splines $m$-ter Ordnung zum Gitter $\Delta = \{ t_0 , ..., t_n \}$.

Zeigen Sie
\begin{equation*}
    \dim S_\Delta^m = n + m.
\end{equation*}

\textbf{Beweis:}\\

Der Beweis steht schon fast vollständig im Skript.

Eine Funktion
\begin{equation*}
    f \in S_\Delta^m = \{ f \in C^{m-1}[a,b] \, | \, v|_{I_k} = p_k \in \mathcal{P}_m\}
\end{equation*}
wird wie man sieht durch $n$ Polynome beschrieben, die alle Grad $k$ haben.
Wir können $f$ also durch $n \cdot (m+1)$ Koeffizienten beschreiben.\\

Da die Funktion im Übergang $m-1$ fach stetig differenzierbar sein muss, gilt nach übergangsbedingung,
dass $p_k^j(x_k) = p_{k+1}^(j)(x_k)$ für alle $0 \leq j < m$ und $0 \leq k < n$.\\

Wir verlieren nun für jede der Bedingungen einen Freiheitsgrad, da in den Punkten $t_k$ immer
gelten muss, dass die $i$-te Ableitung auf beiden angrezenden Funktionen je eine Variable fest setzen
wird, wie wir bei der Interpolation vorher gesehen haben.

Pro Polynom verlieren wir so also $m$ Koeffizienten, die wir frei wählen können. Dies gilt nicht für
$t_0$ und $t_n$ da wir nicht auf die Ramenbedingung zum Nachbern achten müssen. Die restlichen
Koeffizienten können immer noch frei gewählt werden, da es sich beim Polynomring um einen Vektorraum handelt.

Folglich erhalten wir
\begin{equation*}
    \dim S_\Delta^m = \underbrace{n(m+1)}_\text{\#Mögliche Koeffizienten} - \underbrace{m(n-1)}_\text{stetig im Gitter} = n + m
\end{equation*}
\mbox{}\hfill$\square$

\subsection*{Aufgabe 2}

\subsubsection*{(a)}

Implementieren Sie eine Matlab Function für das Interpolieren/Approxmieren einer Funktion mit Kubischen Splines.\\

\textbf{Lösung:}\\

tbd

\subsubsection*{(b)}

Testen Sie ihr Programm an der Funktion $f = \sqrt{5 + x}$ (WRONG).\\

\textbf{Lösung:}\\

tbd

\subsection*{Aufgabe 3}

Zeichnen Sie ihre Hand nach. Interpolieren Sie die gewählten Punkte mittels Kubischen Splines und natürlichen Randbedingungen
und interpolieren Sie es mittels \emph{......}. Berechnen Sie es einmal über $x$ und einmal über $y$. Welchen unterschied
haben die beiden Verfahren.\\

\textbf{Lösung:}\\

tbd


\label{LastPage}
\end{document}
