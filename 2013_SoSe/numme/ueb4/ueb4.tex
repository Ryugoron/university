\documentclass[11pt,a4paper,ngerman]{article}
\usepackage[bottom=2.5cm,top=2.5cm]{geometry} 
\usepackage{babel}
\usepackage[utf8]{inputenc} 
\usepackage[T1]{fontenc} 
\usepackage{ae} 
\usepackage{amssymb} 
\usepackage{amsmath}
\usepackage{amsthm} 
\usepackage{graphicx}
\usepackage{fancyhdr}
\usepackage{fancyref}
\usepackage{enumerate}
\usepackage{listings}
\usepackage{xcolor}
\usepackage{paralist}

\usepackage[pdftex, bookmarks=false, pdfstartview={FitH}, linkbordercolor=white]{hyperref}
\usepackage{fancyhdr}
\pagestyle{fancy}
\fancyhead[C]{Numerik I}
\fancyhead[L]{Übung 4}
\fancyhead[R]{SoSe 2013}
\fancyfoot{}
\fancyfoot[L]{}
\fancyfoot[C]{\thepage \hspace{1px} of \pageref{LastPage}}
\renewcommand{\footrulewidth}{0.5pt}
\renewcommand{\headrulewidth}{0.5pt}
\setlength{\parindent}{0pt} 
\setlength{\headheight}{0pt}

\date{Tutor : Christina Schulz}
\title{Übung 4}
\author{Max Wisniewski, Alexander Steen}


%%
%% Enviroments for proofs and lemmas
%%
\newtheorem{lemma}{\bfseries Claim}

\begin{document}

\lstset{language=Pascal, basicstyle=\ttfamily\fontsize{10pt}{10pt}\selectfont\upshape, commentstyle=\rmfamily\slshape, keywordstyle=\rmfamily\bfseries, breaklines=true, frame=single, xleftmargin=3mm, xrightmargin=3mm, tabsize=2, mathescape=true}

\renewcommand{\figurename}{Figure}

\maketitle
\thispagestyle{fancy}

%%%%%%%%%%%%%%%%%%%%%%%%%%%%%%
%% Aufgabe 1 %%%%%%%%%%%%%%%%
%%%%%%%%%%%%%%%%%%%%%%%%%%%%%%
\subsection*{Aufgabe 1}
Gegeben ein Gitter $0 = x_0 < x_1 < \ldots < x_m = 2 \pi$ und folgende Bestapproximationsaufgabe:

\begin{equation} \label{eq:1}
 p \in \mathcal{P}_n: \quad \sum_{i=0}^m {\left(p(x_i) - \sin(x_i) \right))^2} \leq \sum_{i=0}^m {\left(q(x_i) - \sin(x_i) \right))^2} \; \forall q \in \mathcal{P}_n
\end{equation}

\begin{enumerate}[a)]
\item Für die Bestapproximationsaufgabe \ref{eq:1} ergibt sich als lineares Ausgleichsproblem:
\begin{equation}
\|b - Ax \|_2 \leq \|b - Ay \|_2
\end{equation}
mit $b = \left(\sin(x_0),\ldots,\sin(x_m) \right)$ und $A$ der van-der-Monde-Matrix zu den Stützstellen, also
$$ A = \left( \begin{array}{ccccc} 1 & x_0 & x_0^2 & \ldots & x_0^n \\
  1 & x_1 & x_1^2 & \ldots & x_1^n \\
  \vdots & \vdots & \vdots &\vdots & \vdots \\
  1 & x_m & x_m^2 & \ldots & x_m^n \end{array} \right)$$
\item 
\end{enumerate}
%%%%%%%%%%%%%%%%%%%%%%%%%%%%%%
%% Aufgabe 2 %%%%%%%%%%%%%%%%
%%%%%%%%%%%%%%%%%%%%%%%%%%%%%%
\subsection*{Aufgabe 2}

%%%%%%%%%%%%%%%%%%%%%%%%%%%%%%
%% Aufgabe 3 %%%%%%%%%%%%%%%%
%%%%%%%%%%%%%%%%%%%%%%%%%%%%%%
\subsection*{Aufgabe 3}

%%%%%%%%%%%%%%%%%%%%%%%%%%%%%%
%% Aufgabe 4 %%%%%%%%%%%%%%%%
%%%%%%%%%%%%%%%%%%%%%%%%%%%%%%
\subsection*{Aufgabe 4}



\label{LastPage}
\end{document}
