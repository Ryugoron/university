\documentclass[11pt,a4paper,ngerman]{article}
\usepackage[bottom=2.5cm,top=2.5cm]{geometry} 
\usepackage{babel}
\usepackage[utf8]{inputenc} 
\usepackage[T1]{fontenc} 
\usepackage{ae} 
\usepackage{amssymb} 
\usepackage{amsmath}
\usepackage{amsthm} 
\usepackage{graphicx}
\usepackage{fancyhdr}
\usepackage{fancyref}
\usepackage{enumerate}
\usepackage{listings}
\usepackage{xcolor}
\usepackage{paralist}
\usepackage{tabularx}

\usepackage[pdftex, bookmarks=false, pdfstartview={FitH}, linkbordercolor=white]{hyperref}
\usepackage{fancyhdr}
\pagestyle{fancy}
\fancyhead[C]{Numerik I}
\fancyhead[L]{Übung 4}
\fancyhead[R]{SoSe 2013}
\fancyfoot{}
\fancyfoot[L]{}
\fancyfoot[C]{\thepage \hspace{1px} of \pageref{LastPage}}
\renewcommand{\footrulewidth}{0.5pt}
\renewcommand{\headrulewidth}{0.5pt}
\setlength{\parindent}{0pt} 
\setlength{\headheight}{0pt}

\date{Tutor : Christina Schulz}
\title{Übung 4}
\author{Max Wisniewski, Alexander Steen}


%%
%% Enviroments for proofs and lemmas
%%
\newtheorem{lemma}{\bfseries Claim}

\begin{document}

\lstset{language=Pascal, basicstyle=\ttfamily\fontsize{10pt}{10pt}\selectfont\upshape, commentstyle=\rmfamily\slshape, keywordstyle=\rmfamily\bfseries, breaklines=true, frame=single, xleftmargin=3mm, xrightmargin=3mm, tabsize=2, mathescape=true}

\renewcommand{\figurename}{Figure}

\maketitle
\thispagestyle{fancy}

%%%%%%%%%%%%%%%%%%%%%%%%%%%%%%
%% Aufgabe 1 %%%%%%%%%%%%%%%%
%%%%%%%%%%%%%%%%%%%%%%%%%%%%%%
\subsection*{Aufgabe 1}
Gegeben ein Gitter $0 = x_0 < x_1 < \ldots < x_m = 2 \pi$ und folgende Bestapproximationsaufgabe:

\begin{equation} \label{eq:1}
 p \in \mathcal{P}_n: \quad \sum_{i=0}^m {\left(p(x_i) - \sin(x_i) \right))^2} \leq \sum_{i=0}^m {\left(q(x_i) - \sin(x_i) \right))^2} \; \forall q \in \mathcal{P}_n
\end{equation}

\begin{enumerate}[a)]
\item Für die Bestapproximationsaufgabe \ref{eq:1} ergibt sich als lineares Ausgleichsproblem:
\begin{equation}
\|b - Ax \|_2 \leq \|b - Ay \|_2
\end{equation}
mit $b = \left(\sin(x_0),\ldots,\sin(x_m) \right)$ und $A$ der van-der-Monde-Matrix zu den Stützstellen, also
$$ A = \left( \begin{array}{ccccc} 1 & x_0 & x_0^2 & \ldots & x_0^n \\
  1 & x_1 & x_1^2 & \ldots & x_1^n \\
  \vdots & \vdots & \vdots &\vdots & \vdots \\
  1 & x_m & x_m^2 & \ldots & x_m^n \end{array} \right)$$
\item Schreiben Sie ein MATLAB Programm, dass das lineare Ausgleichsproblem mit Hilfe von Givens-Ratation oder Housholder-Reflexion löst.
Testen Sie Ihr Programm für mehrere Gitter mit verschiedenen Werten für $m$ und $n \leq m$. Welches Resultat erhält man für $n = m$?\\

\textbf{Lösung:}\\

Wir haben uns für eine Lösung mit Housholder-Reflexion entschieden.

\begin{lstlisting}
function [x] = ausgleich ( A,b )

dim = size(A);
%% Anzahl der iteration bis wir eine obere Dreiecksmatrix haben
t = min([dim(1)-1,dim(2)]);

for k = 1 : t
  e = [1;zeros(dim(1)-k,1)];
  v = A(k:dim(1),k);
  v = v + (-1*sign(v(1)))*norm(v,2) * e;
  v = v/norm(v,2);
  %% Baut die HousholdrMatrix zur k ten Spalte ab dem k ten Eintrag
  q = eye(dim(1)-k+1) - 2*v*v';
  q = [eye(k-1) zeros(k-1,dim(1)-k+1);zeros(dim(1)-k+1,k-1) q];
  %% Iteriert Q_i...Q_1A, was am Ende Q^TA ergibt
  A = q*A;
  %% Iteriert Q_i ... Q_1b, was am Ende Q^Tb ergibt.
  b = q*b;
end;
  %% Trennt die rechte obere Dreiecksmatrix vom Nullteil.
  R = A(1:dim(2),:)
  %% Berechnet die Loesung des linearen Ausgleichsproblems
  x = inv(R)*b(1:dim(2));
\end{lstlisting}

Für einfache Beispiele, haben wir das richtige $R$ bekommen. Leider ergibt
der Algorithmus auf die Matrix aus a) angewandt kein sinnvolles Ergebnis (siehe Anhang E-Mail).

\end{enumerate}
%%%%%%%%%%%%%%%%%%%%%%%%%%%%%%
%% Aufgabe 2 %%%%%%%%%%%%%%%%
%%%%%%%%%%%%%%%%%%%%%%%%%%%%%%
\subsection*{Aufgabe 2}

Sei $H = I - 2w \cdot w^T (w \in \mathbb{R}^n \| w \|_2 = 1$ und $I$ ist die Einheitsmatrix), eine Housholder Matrix.

\subsubsection*{(a)}

Zeigen Sie, dass der Vektor $w$ ein Eigenvektor der Matrix $H$ ist. Beweisen Sie, dass ein Vektor $x$, der orthogonal zu Vektor $w$
ist, auch ein Eigenvektor von $H$ ist. Welche Eigenwerte gehören zu den Eigenvektoren?\\

\textbf{Lösung:}\\

\begin{enumerate}[1.]
    \item z.z. $w$ ist Eigenvektor für $H$.\\
        \begin{equation*}
            H \cdot w = (I - 2 w \cdot w^T) \cdot w = v
        \end{equation*}
        Ein Eintrag der Matrix sieht wie folgt aus:
        \begin{equation}
            h_{ij} = \delta_{i}(j) - 2\cdot w_i \cdot w_j
        \end{equation}
        Wie wir sehen ist die Matrix $h$ symmetrisch.\\
        Multiplizieren wir nun $w$ an diese Matrix sieht jeder Eintrag
        folgt aus:
        \begin{equation*}\begin{split}
            v_i &= \overset{n}{\underset{j=1}{\sum}} h_{ij} \cdot w_i\\
                &= \overset{n}{\underset{j=1}{\sum}} (\delta_{i}(j) - 2\cdot w_i \cdot w_j) \cdot w_j\\
                &= \overset{n}{\underset{j=1}{\sum}} \delta_{i}(j) \cdot w_j - 2 \cdot w_i \cdot w_j^2\\
                &= w_i - 2 w_i \overset{n}{\underset{j=1}{\sum}} w_j^2\\
                &\stackrel{\|w\|_2 = 1}{=} w_i - 2 w_i \cdot 1\\
                &= - w_i
        \end{split}\end{equation*}
        Wir sehen, dass $w_i$ ein Eigenvektor von $H$ ist, mit dem Eigenwert $-1$.
    \item z.z. $x$ mit $\left\langle x , w \right\rangle = 0$.
        Sei $x$ ein beliebiger orthogonaler Vektor zu $w$ und sei $H x =x'$.
        Dan können wir wie eben die Komponenten von $x'$ berechnen.
        \begin{equation*}\begin{split}
            x'_i &= \overset{n}{\underset{j=1}{\sum}} (\delta_{i}(j) - 2 \cdot w_i \cdot w_j) \cdot x_j\\
                 &= w_i - 2 w_i \overset{n}{\underset{j=1}{\sum}} w_j \cdot x_j\\
                 &\stackrel{\left\langle w, x \right\rangle = 0}{=} w_i - 2 w_i \cdot 0\\
                 &= w_i
        \end{split}\end{equation*}
        Wir sehen, dass alle orthogonalen Vektoren zu $w$ Eigenvektoren von $H$ sind mit dem Eigenwert $1$.
\end{enumerate}


\subsubsection*{(b)}

Berechnen Sie alle andere Eigenwerte der Matrix $H$.\\

\textbf{Lösung:}\\

Wir wissen, dass $H$ $n$ Eigenwerte besitzt.
Der Raum $\mathbb{R}^n$ wird von einer Basis mit $n$ elementen aufgespannt.
Nach Basisaustauschlemma (Lina I) wissen wir, dass wir $w$ in die Basis aufnehmen können.
Fixieren wir $w$ und orthogonalisieren wir die Basis, so erhalten wir $w$ und $n-1$ orthonormale
Vektoren. Wie wir gezeigt haben ist der Eigenwert von $w$ $-1$ und von jedem der orhtogonalen Vektoren ist der
Eigenwert $1$.\\

Da die Größe der Eigenräume ($-11$ => $1$, $1$ => $n-1$) in der Summe $n$ ergibt, kann es keine weiteren Eigenwerte geben.

\subsubsection*{(c)}

Beweisen Sie, dass $H$ eine orthogonale Matrix ist.\\

\textbf{Beweis:}\\

Wie schon gezeigt ist $H$ symmetrisch. Daher müssen wir nur noch zeigen, dass
\begin{equation*}
    H \cdot H   = I
\end{equation*}
gilt.

Da die Dimension der Eigenräume $n$ ist, können wir $H$ diagonalisieren mit den Eigenwerten auf der Diagonalen und nennen
diese Matrix $D$.
Sei $S$ die orthogonale Transformationsmatrix, dann ist $H = S^{-1}DS$ und es folgt
\begin{equation*}\begin{split}
    H \cdot H &= (S^{-1} D S) (S^{-1} D S)\\
            &= S^{-1} (D D) S\\
            &\stackrel{(*)}{=} S^{-1} S\\
            &= I
\end{split}\end{equation*}
und $(*)$ gilt, da bei der Multiplikation von Diagonalmatrizen nur die Eigenwerte auf der Diagonalen quadritert werden.
Diese sind bei uns $-1$ und $1$, welche im quadrat alle $1$ ergeben und so eine Einheitsmatrix bilden.

%%%%%%%%%%%%%%%%%%%%%%%%%%%%%%
%% Aufgabe 3 %%%%%%%%%%%%%%%%
%%%%%%%%%%%%%%%%%%%%%%%%%%%%%%
\subsection*{Aufgabe 3}

Bestimmen Sie die Hermite-Interpoliterte $p \in \mathcal{P}_4$ zu den Hermite-Interpolationsbedingungen
\begin{equation*}\begin{split}
    p(0) &= 1\\
    p(1) &= 0, \quad p'(1) = 1, \quad p''(1) = 0\\
    p(2) &= 1
\end{split}\end{equation*}

\textbf{Lösung:}\\

Wir setzen zunächst:
$$
    x_0 = 0, x_1 = x_2 = x_3 = 1, x_4 = 1
$$
und
\begin{equation*}\begin{split}
    f[x_0] &= 1\\
    f[x_1] &= f[x_2] = f[x_3] = 0\\
    f[x_1,x_2] &= f[x_2, x_3] = 1\\
    f[x_1x_2x_3] &= 0\\
    f[x_4] &= 1
\end{split}\end{equation*}
damit berechnen wir nun das Neville-Tableau


\begin{tabularx}{\textwidth}{c|cccccc}
$k$ & $x_k$ & $f[x_k]$ & $f[x_{k-1},f_k]$ & $f[x_{k-2}, ..., x_k]$ & $f[x_{k-3}, ..., x_k]$ & $f[x_{k-4}, ..., x_k]$\\
\hline
0 & 0 & 1 & - & - & - & -\\
1 & 1 & 0 & $-1$ & - & - & -\\
2 & 1 & 0 & 1 & $2$ & - & -\\
3 & 1 & 0 & 1 & 0 & $-2$ & -\\
4 & 2 & 1 & $1$ & 0 & $0$ & $1$\\ 
\end{tabularx}

Damit erhalten wir das Polynom 
\begin{equation*}\begin{split}
    p(x) &= f[x_0] + f[x_0, x_1](x-x_0) + f[x_0, x_1, x_2](x - x_0)(x-x_1)\\
    &+f[x_0, .. ,x_3](x - x_0)(x - x_1)(x - x_2)\\
    &+ f[x_0, .., x_4](x - x_0)(x - x_1)(x - x_2)(x - x_3)\\
    &= 1 - x + 2x(x-1) - 2x(x-1)^2 + x(x-1)^3\\
    &= 1 - x +2x^2 - 2x - 2x(x^2 -2x +1)+ x(x-1)(x^2 - 2x +1)\\
    &= 1 - x + 2x^2 - 2x - 2x^3 + 4x^2 - 2x + x^4 - 3x^3 + 3x^2 - x\\
    &= x^4 - 5x^3 + 9 x^2 - 6x + 1
\end{split}\end{equation*}
als Lösung.

\label{LastPage}
\end{document}
