\documentclass[11pt,a4paper,ngerman]{article}
\usepackage[bottom=2.5cm,top=2.5cm]{geometry} 
\usepackage{babel}
\usepackage[utf8]{inputenc} 
\usepackage[T1]{fontenc} 
\usepackage{ae} 
\usepackage{amssymb} 
\usepackage{amsmath}
\usepackage{amsthm} 
\usepackage{graphicx}
\usepackage{fancyhdr}
\usepackage{fancyref}
\usepackage{enumerate}
\usepackage{listings}
\usepackage{xcolor}
\usepackage{paralist}
\usepackage{tabularx}

\usepackage[pdftex, bookmarks=false, pdfstartview={FitH}, linkbordercolor=white]{hyperref}
\usepackage{fancyhdr}
\pagestyle{fancy}
\fancyhead[C]{Numerik I}
\fancyhead[L]{Übung 7}
\fancyhead[R]{SoSe 2013}
\fancyfoot{}
\fancyfoot[L]{}
\fancyfoot[C]{\thepage \hspace{1px} of \pageref{LastPage}}
\renewcommand{\footrulewidth}{0.5pt}
\renewcommand{\headrulewidth}{0.5pt}
\setlength{\parindent}{0pt} 
\setlength{\headheight}{0pt}

\date{Tutor : Christina Schulz}
\title{Übung 7}
\author{Max Wisniewski, Alexander Steen}


%%
%% Enviroments for proofs and lemmas
%%
\newtheorem{lemma}{\bfseries Claim}

\begin{document}

\lstset{language=Pascal, basicstyle=\ttfamily\fontsize{10pt}{10pt}\selectfont\upshape, commentstyle=\rmfamily\slshape, keywordstyle=\rmfamily\bfseries, breaklines=true, frame=single, xleftmargin=3mm, xrightmargin=3mm, tabsize=2, mathescape=true}

\renewcommand{\figurename}{Figure}

\maketitle
\thispagestyle{fancy}

%%%%%%%%%%%%%%%%%%%%%%%%%%%%%%
%% Aufgabe 1                                                    %%%%%%%%%%%%%%%%
%%%%%%%%%%%%%%%%%%%%%%%%%%%%%%
\subsection*{Aufgabe 1}
Gegeben sei das Integral $J$ mit folgender (exakter) Lösung:
$$ J := \int_0^1 \sin(\pi x) \, dx = \frac{2}{\pi} $$

Nach Skript  ist der Fehler bei summierter Trapezregel beschränkt durch $h^2 \|f^{(2)} \|_\infty$ und damit für uns gegeben durch

$$ h^2 \|f^{(2)} \|_\infty =$h^2 \| -\pi^2 \sin(\pi x) \|_\infty = h^2 \pi^2$$

Da der Fehler durch $10^{-4}$ beschränkt sein soll ergibt sich folgende Ungleichung

$$ h^2 \pi^2 < 10^{-4} \Leftrightarrow n > 100 \pi$$



%%%%%%%%%%%%%%%%%%%%%%%%%%%%%%
%% Aufgabe 2                           %%%%%%%%%%%%%%%%
%%%%%%%%%%%%%%%%%%%%%%%%%%%%%%
\subsection*{Aufgabe 2}

check

%%%%%%%%%%%%%%%%%%%%%%%%%%%%%%
%% Aufgabe 3                   %%%%%%%%%%%%%%%%
%%%%%%%%%%%%%%%%%%%%%%%%%%%%%%
\subsection*{Aufgabe 3}




\label{LastPage}
\end{document}