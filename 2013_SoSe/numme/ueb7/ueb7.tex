\documentclass[11pt,a4paper,ngerman]{article}
\usepackage[bottom=2.5cm,top=2.5cm]{geometry} 
\usepackage{babel}
\usepackage[utf8]{inputenc} 
\usepackage[T1]{fontenc} 
\usepackage{ae} 
\usepackage{amssymb} 
\usepackage{amsmath}
\usepackage{amsthm} 
\usepackage{graphicx}
\usepackage{fancyhdr}
\usepackage{fancyref}
\usepackage{enumerate}
\usepackage{listings}
\usepackage{xcolor}
\usepackage{paralist}
\usepackage{tabularx}

\usepackage[pdftex, bookmarks=false, pdfstartview={FitH}, linkbordercolor=white]{hyperref}
\usepackage{fancyhdr}
\pagestyle{fancy}
\fancyhead[C]{Numerik I}
\fancyhead[L]{Übung 7}
\fancyhead[R]{SoSe 2013}
\fancyfoot{}
\fancyfoot[L]{}
\fancyfoot[C]{\thepage \hspace{1px} of \pageref{LastPage}}
\renewcommand{\footrulewidth}{0.5pt}
\renewcommand{\headrulewidth}{0.5pt}
\setlength{\parindent}{0pt} 
\setlength{\headheight}{0pt}

\date{Tutor : Christina Schulz}
\title{Übung 7}
\author{Max Wisniewski, Alexander Steen}


%%
%% Enviroments for proofs and lemmas
%%
\newtheorem{lemma}{\bfseries Claim}

\begin{document}

\lstset{language=Pascal, basicstyle=\ttfamily\fontsize{10pt}{10pt}\selectfont\upshape, commentstyle=\rmfamily\slshape, keywordstyle=\rmfamily\bfseries, breaklines=true, frame=single, xleftmargin=3mm, xrightmargin=3mm, tabsize=2, mathescape=true}

\renewcommand{\figurename}{Figure}

\maketitle
\thispagestyle{fancy}

%%%%%%%%%%%%%%%%%%%%%%%%%%%%%%
%% Aufgabe 1                                                    %%%%%%%%%%%%%%%%
%%%%%%%%%%%%%%%%%%%%%%%%%%%%%%
\subsection*{Aufgabe 1}

\subsubsection*{(a)}

Gegeben sei das Integral $J$ mit folgender (exakter) Lösung:
\begin{equation*}
    J := \int_0^1 \sin(\pi x) \, dx = \frac{2}{\pi}
\end{equation*}

Nach Skript  ist der Fehler bei summierter Trapezregel beschränkt durch $h^2 \|f^{(2)} \|_\infty$ und damit für uns gegeben durch

\begin{equation*}\begin{array}{crcl}
    & h^2 \| f^{(a)} \|_\infty &<& 10^{-4}\\
\Leftrightarrow & \frac{1}{n}^2 \| -\pi^2 \sin (\pi x) \|_\infty &<& 10^{-4}\\
\Leftrightarrow & \frac{pi}{n}^2 &<& 10^{-4}\\
\Leftrightarrow & n &>& 100 \pi
\end{array}\end{equation*}

Nun müssen wir nach der Formel
$$
    I_\Delta (f) = \overset{m}{\underset{k=1}{\sum}} \overset{n}{\underset{i=0}{\sum}} \lambda_{ik} f(x_{ik}) h_k
$$
die Funktion insgesammt $2n$ mal auswerten oder aber, da sich die Werte immer überschneiden, müssen wir sie nur $n$ mal berechnen
und uns den letzten Wert für das nächste Trapez merken.

\subsubsection*{(b)}

Gegeben ist das Integral $\int_0^1 f(x) dx$ und die Knoten $x_0 = 0, x_1 = \frac{1}{4}, x_2 = 1$.
Bestimmen Sie die Quadraturformel mit Ordnung 3.\\

\textbf{Solution:}\\

Die Quadraturformel mit Ordnung 3 ist nach Newton-Côtes Formel die simpsonregel. Da wir nur 3 Punkte haben, betrachten wir
nur dieses Interval.\\

\begin{equation*}\begin{split}
    I_\Delta (f) &= \overset{2}{\underset{i=0}{\sum}} \lambda_{i} f(x_i) h_{i}\\
    &= \frac{f(0)}{6} + \frac{4 \cdot f(\frac{1}{4})}{6} + \frac{f(1)}{6}
\end{split}\end{equation*}

Äh, ich glaub man sollte noch 3 Punkte in die beiden Intervalle je setzten....
%%%%%%%%%%%%%%%%%%%%%%%%%%%%%%
%% Aufgabe 2                           %%%%%%%%%%%%%%%%
%%%%%%%%%%%%%%%%%%%%%%%%%%%%%%
\subsection*{Aufgabe 2}

Es sollen zwei Programme zur berechnung von
$$
    \int_a^b f(x) dx
$$
geschriebe werde.

\subsubsection*{(a)}

Schreiben Sie eine Funktion zur Berechnung mit summierter Trapezregel.\\

\textbf{Lösung:}\\

tbd

\subsubsection*{(b)}

Schreibe Sie eine FUnktion zur Berechnung mit summierter Simpsonregel.\\

\textbf{Lösung:}\\

tbd

\subsubseciton*{(c)}

Testen Sie die Funktionen und plotten Sie die Laufzeit.\\

\textbf{Lösung:}\\

tbd

%%%%%%%%%%%%%%%%%%%%%%%%%%%%%%
%% Aufgabe 3                   %%%%%%%%%%%%%%%%
%%%%%%%%%%%%%%%%%%%%%%%%%%%%%%
\subsection*{Aufgabe 3}

Gewinnung einer Quadraturformel in höheren Dimensionen.

\subsubsectin*{(a)}

Wie sieht die Quadratur auf einem Dreieck mit den Ecke $e_1 = (0,0), e_2 = (a,0), e_3 = (0,b)$ aus.\\

\textbf{Lösung:}\\

tbd

\subsubsection*{(b)}

Gewinnen Sie eine Quadraturformel für das Dreieck durch Integration der Lagrange-Polynome ähnlich
durch $L_k(e_i) = \delta_{ki}$ gegeben sind.\\

\textbf{Lösung:}\\

tbd

\subsubsection*{(c)}

Leiten Sie die Formel für das Recheck her.\\

\textbf{Lösung:}\\

tbd


\label{LastPage}
\end{document}
