\documentclass[11pt,a4paper,ngerman]{article}
\usepackage[bottom=2.5cm,top=2.5cm]{geometry} 
\usepackage{babel}
\usepackage[utf8]{inputenc} 
\usepackage[T1]{fontenc} 
\usepackage{ae} 
\usepackage{amssymb} 
\usepackage{amsmath}
\usepackage{amsthm} 
\usepackage{graphicx}
\usepackage{fancyhdr}
\usepackage{fancyref}
\usepackage{enumerate}
\usepackage{listings}
\usepackage{xcolor}
\usepackage{paralist}
\usepackage{tabularx}

\usepackage[pdftex, bookmarks=false, pdfstartview={FitH}, linkbordercolor=white]{hyperref}
\usepackage{fancyhdr}
\pagestyle{fancy}
\fancyhead[C]{Numerik I}
\fancyhead[L]{Übung 5}
\fancyhead[R]{SoSe 2013}
\fancyfoot{}
\fancyfoot[L]{}
\fancyfoot[C]{\thepage \hspace{1px} of \pageref{LastPage}}
\renewcommand{\footrulewidth}{0.5pt}
\renewcommand{\headrulewidth}{0.5pt}
\setlength{\parindent}{0pt} 
\setlength{\headheight}{0pt}

\date{Tutor : Christina Schulz}
\title{Übung 5}
\author{Max Wisniewski, Alexander Steen}


%%
%% Enviroments for proofs and lemmas
%%
\newtheorem{lemma}{\bfseries Claim}

\begin{document}

\lstset{language=Pascal, basicstyle=\ttfamily\fontsize{10pt}{10pt}\selectfont\upshape, commentstyle=\rmfamily\slshape, keywordstyle=\rmfamily\bfseries, breaklines=true, frame=single, xleftmargin=3mm, xrightmargin=3mm, tabsize=2, mathescape=true}

\renewcommand{\figurename}{Figure}

\maketitle
\thispagestyle{fancy}

%%%%%%%%%%%%%%%%%%%%%%%%%%%%%%
%% Aufgabe 1 %%%%%%%%%%%%%%%%
%%%%%%%%%%%%%%%%%%%%%%%%%%%%%%
\subsection*{Aufgabe 1}

Zeigen Sie, dass der Interpolationsoperator
\begin{equation*}
    \mathcal{C}[a,b] \ni f \mapsto \varphi (f) = p_n \in \mathcal{P}_n
\end{equation*}
eine Projektion ist und die Norm $\| \varphi \|_\infty = \Lambda_n$ hat, wobei
$$
    \Lambda_n = \underset{x \in [a,b]}{\max} \overset{n}{\underset{k=0}{\sum}} |L_k(x)|
$$
die Lebesque-Konstante ist und $L_k$ die Lagrange-Polyonme bezeichnet.\\

\textbf{Lösung:}\\

tbd

\subsection*{Aufgabe 2}

Betrachte die Funktionsklasse
\begin{equation*}
    \mathcal{F} = \{f \in C^{n+1}[-1,1] \, | \, \|f^{(n+1)}\|_\infty \leq (n+1)! \}.
\end{equation*}
Für $f \in \mathcal{F}$ bezeichne $p_n(f)$ das Interpolationspolynom $n$-ten Grades zu den Knoten
$K = \{t_0, ..., t_n \} \subset [-1,1]$.

\subsubsection*{(a)}

Zeigen Sie
\begin{equation*}
    \varepsilon_n(K) = \underset{f \in \mathcal{F}}{\sup} |f - p_n(f)| = \| \omega_{n+1} \|_\infty, \omega_{n+1}(t) = (t-t_0)...(t-t_n).
\end{equation*}

\textbf{Lösung:}\\

tbd

\subsubsection*{(b)}

Zeigen Sie $\varepsilon_n(K) \geq 2^{-n}$ und dass Gleicheit genau dann gilt, wenn $K$ die Menge
der Tschebyscheff-Knoten ist.\\

\textbf{Lösung:}\\

tbd


\label{LastPage}
\end{document}
