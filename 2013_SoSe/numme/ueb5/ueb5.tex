\documentclass[11pt,a4paper,ngerman]{article}
\usepackage[bottom=2.5cm,top=2.5cm]{geometry} 
\usepackage{babel}
\usepackage[utf8]{inputenc} 
\usepackage[T1]{fontenc} 
\usepackage{ae} 
\usepackage{amssymb} 
\usepackage{amsmath}
\usepackage{amsthm} 
\usepackage{graphicx}
\usepackage{fancyhdr}
\usepackage{fancyref}
\usepackage{enumerate}
\usepackage{listings}
\usepackage{xcolor}
\usepackage{paralist}
\usepackage{tabularx}

\usepackage[pdftex, bookmarks=false, pdfstartview={FitH}, linkbordercolor=white]{hyperref}
\usepackage{fancyhdr}
\pagestyle{fancy}
\fancyhead[C]{Numerik I}
\fancyhead[L]{Übung 5}
\fancyhead[R]{SoSe 2013}
\fancyfoot{}
\fancyfoot[L]{}
\fancyfoot[C]{\thepage \hspace{1px} of \pageref{LastPage}}
\renewcommand{\footrulewidth}{0.5pt}
\renewcommand{\headrulewidth}{0.5pt}
\setlength{\parindent}{0pt} 
\setlength{\headheight}{0pt}

\date{Tutor : Christina Schulz}
\title{Übung 5}
\author{Max Wisniewski, Alexander Steen}


%%
%% Enviroments for proofs and lemmas
%%
\newtheorem{lemma}{\bfseries Claim}

\begin{document}

\lstset{language=Pascal, basicstyle=\ttfamily\fontsize{10pt}{10pt}\selectfont\upshape, commentstyle=\rmfamily\slshape, keywordstyle=\rmfamily\bfseries, breaklines=true, frame=single, xleftmargin=3mm, xrightmargin=3mm, tabsize=2, mathescape=true}

\renewcommand{\figurename}{Figure}

\maketitle
\thispagestyle{fancy}

%%%%%%%%%%%%%%%%%%%%%%%%%%%%%%
%% Aufgabe 1 %%%%%%%%%%%%%%%%
%%%%%%%%%%%%%%%%%%%%%%%%%%%%%%
\subsection*{Aufgabe 1}

Zeigen Sie, dass der Interpolationsoperator
\begin{equation*}
    \mathcal{C}[a,b] \ni f \mapsto \varphi (f) = p_n \in \mathcal{P}_n
\end{equation*}
eine Projektion ist und die Norm $\| \varphi \|_\infty = \Lambda_n$ hat, wobei
$$
    \Lambda_n = \underset{x \in [a,b]}{\max} \overset{n}{\underset{k=0}{\sum}} |L_k(x)|
$$
die Lebesque-Konstante ist und $L_k$ die Lagrange-Polyonme bezeichnet.\\

\textbf{Lösung:}\\

Seien $a \leq x_0 < ... < x_n \leq b$ die Stützstellen für den Interpolationsoperator.
Dann zeigen wir zunächst, dass für $f,g \in \mathcal{C}[a,b]$ und $\lambda \in \mathbb{R}$
\begin{equation*}
    \varphi(\lambda (f + g)) = \lambda (\varphi(f) + \varphi(g))
\end{equation*}
gilt.\\

\textbf{Beweis:}
Da der Interpolationsopertor nur über
\begin{equation}\label{numme:ueb5:intopdef}
    f(x_i) = \varphi(f)(x_i), \quad \forall 0 \leq i \leq n
\end{equation}
definiert ist, müssen wir auch nur jede der Stütztstellen untersuchen.\\

Sei $x_i$ eine der Stützstellen, beliebig aber fest.
\begin{equation}\begin{split}
    \varphi(\lambda (f + g))(x_i)
    &\stackrel{(\ref{numme:ueb5:intopdef})}{=} (\lambda (f + g))(x_i)\\
    &\stackrel{f,g\text{ stetig}}{=} \lambda (f(x_i) + g(x_i))\\
    &\stackrel{(\ref{numme:ueb5:intopdef})f,g}{=} \lambda (\varphi(f)(x_i) + \varphi(g)(x_i))\\
    &\stackrel{\varphi(f),\varphi(g)\text{ stetig}}{=} (\lambda (\varphi(f) + \varphi(g))) (x_i)
\end{split}\end{equation}
\mbox{}\hfill$\square$

Als nächstes zeigen wir, dass $\varphi^2 = \varphi$ ist.\\

\textbf{Beweis:}
Sei $f \in \mathcal{C}[a,b]$ beliebig aber fest.
Dann ist $p_n = \varphi(f) \in P_n$ ein Polynom vom Grad $n$.

Nun ist $q_n = \varphi(p_n) = \varphi^2(f) \in P_n$ ebenfalls ein Polynom vom Grad $n$,
bei dem nach Definition vom Interpolationsoperator gilt, dass
$\forall 0 \leq i \leq n \, : \, q_n(x_i) = p_n(x_i)$ gilt.

Nun gilt, dass zwei Polynome vom Grad $n$, die an $n+1$ Stellen den selben Wert haben die selben Polynome
sind. Also gilt $p_n = q_n$.\\
\mbox{}\hfill$\square$

Als nächstes zeigen wir, dass die Norm $\| \varphi \|_\infty = \Lambda_n$ ist.\\

\textbf{Beweis:}\\

Wir nehmen an, dass wir auf dem Zielraum $P_n$ wiederum die supremumsnorm verwenden wollen.

Es gilt zunächst, dass wenn wir ein Polynom haben, dass das Lagrangepolynom wieder das selbe beschreibt.
Das Lagrangepolynom $L(f)$ sieht wie folg aus.
\begin{equation}\begin{split}
    L(f)(x) &:= \overset{k}{\underset{j=0}{\sum}} y_i L_j(x)\\
        &\text{und}\\
    L(f)(x) = \varphi(L(f))(x) = \varphi(f)(x)
\end{split}\end{equation}

Zeigen wir zunächst, dass $\|\varphi\|_\infty \leq \Lambda_n$ ist.

\begin{equation*}\begin{split}
    \| \varphi \|_\infty  &= \underset{f \in \mathcal{C}[a,b]}{\sup} \| \varphi(f) \|_\infty \\
    &= \underset{f \in \mathcal{C}[a,b]}{\sup} \underset{x \in [a,b]}{\sup} |\varphi(f)(x)|\\
    &= \underset{f \in \mathcal{C}[a,b]}{\sup} \underset{x \in [a,b]}{\sup} |L(f)(x)|\\
    &= \underset{f \in \mathcal{C}[a,b]}{\sup} \underset{x \in [a,b]}{\sup} |\overset{k}{\underset{j=0}{\sum}} y_i L_j(x)|\\
    &\stackrel{\sup}{\leq} \underset{f \in \mathcal{C}[a,b]}{\sup} \underset{x \in [a,b]}{\sup} \overset{k}{\underset{j=0}{\sum}} |y_i L_j(x)|\\
    &\stackrel{\sup}{\leq} \underset{f \in \mathcal{C}[a,b]}{\sup} \underset{x \in [a,b]}{\max} \overset{k}{\underset{j=0}{\sum}} |L_j(x)|\\
    &= \underset{f \in \mathcal{C}[a,b]}{\sup} \Lambda_n\\
    &= \Lambda_n
\end{split}\end{equation*}

Und nun zeigen wir, dass $\Lambda_n \leq \|\varphi \|_\infty$ ist.

\begin{equation*}\begin{split}
    \Lambda_n &= \underset{x \in [a,b]}{\max} \overset{n}{\underset{k=0}{\sum}} |L_j(x)|\\
        &\leq \underset{x \in [a,b]}{\sup} \overset{n}{\underset{k=0}{\sum}} |L_j(x)|\\
        &\overset{\exists f\,:\,|y_i|\geq 1}{\leq} \underset{f \in \mathcal{C}[a,b]}{\sup} \underset{x \in [a,b]}{\sup} \overset{n}{\underset{k=0}{\sum}} |y_i||L_i(x)|\\
        &\leq \underset{f \in \mathcal{C}[a,b]}{\sup} \underset{x \in [a,b]}{\sup} \overset{n}{\underset{k=0}{\sum}} |y_i L_i(x)|\\
        &= \underset{f \in \mathcal{C}[a,b]}{\sup} \underset{x \in [a,b]}{\sup} \varphi(f)(x)\\
        &\stackrel{\text{Def.}}{=} \| \varphi \|_\infty
\end{split}\end{equation*}
\mbox{}\hfill$\square$

\subsection*{Aufgabe 2}

Betrachte die Funktionsklasse
\begin{equation*}
    \mathcal{F} = \{f \in C^{n+1}[-1,1] \, | \, \|f^{(n+1)}\|_\infty \leq (n+1)! \}.
\end{equation*}
Für $f \in \mathcal{F}$ bezeichne $p_n(f)$ das Interpolationspolynom $n$-ten Grades zu den Knoten
$K = \{t_0, ..., t_n \} \subset [-1,1]$.

\subsubsection*{(a)}

Zeigen Sie
\begin{equation*}
    \varepsilon_n(K) = \underset{f \in \mathcal{F}}{\sup} |f - p_n(f)| = \| \omega_{n+1} \|_\infty, \omega_{n+1}(t) = (t-t_0)...(t-t_n).
\end{equation*}

\textbf{Lösung:}\\

tbd

\subsubsection*{(b)}

Zeigen Sie $\varepsilon_n(K) \geq 2^{-n}$ und dass Gleicheit genau dann gilt, wenn $K$ die Menge
der Tschebyscheff-Knoten ist.\\

\textbf{Lösung:}\\

tbd


\label{LastPage}
\end{document}
