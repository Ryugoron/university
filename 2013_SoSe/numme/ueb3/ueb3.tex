\documentclass[11pt,a4paper,ngerman]{article}
\usepackage[bottom=2.5cm,top=2.5cm]{geometry} 
\usepackage{babel}
\usepackage[utf8]{inputenc} 
\usepackage[T1]{fontenc} 
\usepackage{ae} 
\usepackage{amssymb} 
\usepackage{amsmath}
\usepackage{amsthm} 
\usepackage{graphicx}
\usepackage{fancyhdr}
\usepackage{fancyref}
\usepackage{enumerate}
\usepackage{listings}
\usepackage{xcolor}
\usepackage{paralist}

\usepackage[pdftex, bookmarks=false, pdfstartview={FitH}, linkbordercolor=white]{hyperref}
\usepackage{fancyhdr}
\pagestyle{fancy}
\fancyhead[C]{Numerik I}
\fancyhead[L]{Übung 3}
\fancyhead[R]{SoSe 2013}
\fancyfoot{}
\fancyfoot[L]{}
\fancyfoot[C]{\thepage \hspace{1px} of \pageref{LastPage}}
\renewcommand{\footrulewidth}{0.5pt}
\renewcommand{\headrulewidth}{0.5pt}
\setlength{\parindent}{0pt} 
\setlength{\headheight}{0pt}

\date{}
\title{Übung 3}
\author{Max Wisniewski, Alexander Steen}


%%
%% Enviroments for proofs and lemmas
%%
\newtheorem{lemma}{\bfseries Claim}

\begin{document}

\lstset{language=Pascal, basicstyle=\ttfamily\fontsize{10pt}{10pt}\selectfont\upshape, commentstyle=\rmfamily\slshape, keywordstyle=\rmfamily\bfseries, breaklines=true, frame=single, xleftmargin=3mm, xrightmargin=3mm, tabsize=2, mathescape=true}

\renewcommand{\figurename}{Figure}

\maketitle
\thispagestyle{fancy}

%%%%%%%%%%%%%%%%%%%%%%%%%%%%%%
%% Aufgabe 1 %%%%%%%%%%%%%%%%
%%%%%%%%%%%%%%%%%%%%%%%%%%%%%%
\subsection*{Aufgabe 1}
Sei $X$ ein Prähilbertraum und $P: X \to X$ linear. Z.z: Folgende drei Aussagen sind äquivalent
\begin{enumerate}[(a)]
\item $(x-Px,y) = 0$, für alle $x \in X$, $y \in R(P)$ 
\item $P^2 = P$ und $(Px,y) = (x,Py)$, für alle $x,y \in X$ 
\item $P^2 = P$ und $\|P\| \leq 1$
\end{enumerate}
\textbf{Beweis} durch Ringschluss:\\
(a) $\Rightarrow$ (b) \\
asd \\
(b) $\Rightarrow$ (c) \\
P selbstadjungiert und $P^2 = P$ => P diagonalisierbar (reell), durch $P^2 = p$ folgt ew = 0 od. 1 => nach basiswechsel bla ,kann nicht mehr werden. \\
(c) $\Rightarrow$ (a) \\
asd 
\mbox{} \hfill $\square$

%%%%%%%%%%%%%%%%%%%%%%%%%%%%%%
%% Aufgabe 2 %%%%%%%%%%%%%%%%
%%%%%%%%%%%%%%%%%%%%%%%%%%%%%%
\subsection*{Aufgabe 2}
Seien $w: [0,1] \to \mathbb{R}$, mit $w(x) := - \ln x$ und $(u,v)$ definiert durch
$$ (u,v) := \int_0^1 w(x)u(x)v(x) \, dx $$
Dann ergeben sich die Orthogonalpolynome $v_0,v_1,v_2$ vom Grad 0,1,2 (respektive) durch Anwendung
des Gram-Schmidt-Verfahrens wie folgt: \\

Wähle $\{w_0, w_1, w_2\}$ mit $w_i = x^i$ als Basis des $P_2$.
\begin{enumerate}
\item $v_0 := w_0 = 1$ 
\item $v_1 = w_1 - \frac{(v_0,w_1)}{(v_0,v_0)} v_0 = x - \frac{\int_0^1 - x \cdot \ln x \, dx}{\int_0^1 - \ln x \, dx} x
           = x - \frac{1}{4} x = \frac{3}{4}x$
\item $v_2 = w_2 - \frac{(v_1,w_2)}{(v_1,v_1)} v_1
           = \frac{3}{4} \left( x^2 - \frac{\int_0^1 - \frac{3}{4}x^3 \ln x \, dx }{ \int_0^1 - \frac{9}{16}x^2\ln x \, dx} x \right)
           = \frac{3}{4} \left(x^2 - \frac{\frac{3}{64}}{\frac{625}{10000}} x\right)
           = \frac{3}{4} \left(x^2 - \frac{3}{4} x\right) = \frac{3}{4}x^2 - \frac{9}{16}x$
\end{enumerate}

Also ergibt sich $\{1,\frac{3}{4}x,\frac{3}{4}x^2 - \frac{9}{16}x \}$ als Orthogonalbasis.
%%%%%%%%%%%%%%%%%%%%%%%%%%%%%%
%% Aufgabe 3 %%%%%%%%%%%%%%%%
%%%%%%%%%%%%%%%%%%%%%%%%%%%%%%
\subsection*{Aufgabe 3}



\label{LastPage}
\end{document}
