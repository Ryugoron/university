\documentclass[11pt,a4paper,ngerman]{article}
\usepackage[bottom=2.5cm,top=2.5cm]{geometry} 
\usepackage{babel}
\usepackage[utf8]{inputenc} 
\usepackage[T1]{fontenc} 
\usepackage{ae} 
\usepackage{amssymb} 
\usepackage{amsmath}
\usepackage{amsthm} 
\usepackage{graphicx}
\usepackage{fancyhdr}
\usepackage{fancyref}
\usepackage{enumerate}
\usepackage{listings}
\usepackage{xcolor}
\usepackage{paralist}

\usepackage[pdftex, bookmarks=false, pdfstartview={FitH}, linkbordercolor=white]{hyperref}
\usepackage{fancyhdr}
\pagestyle{fancy}
\fancyhead[C]{Numerik I}
\fancyhead[L]{Übung 3}
\fancyhead[R]{SoSe 2013}
\fancyfoot{}
\fancyfoot[L]{}
\fancyfoot[C]{\thepage \hspace{1px} of \pageref{LastPage}}
\renewcommand{\footrulewidth}{0.5pt}
\renewcommand{\headrulewidth}{0.5pt}
\setlength{\parindent}{0pt} 
\setlength{\headheight}{0pt}

\date{}
\title{Übung 3}
\author{Max Wisniewski, Alexander Steen}


%%
%% Enviroments for proofs and lemmas
%%
\newtheorem{lemma}{\bfseries Claim}

\begin{document}

\lstset{language=Pascal, basicstyle=\ttfamily\fontsize{10pt}{10pt}\selectfont\upshape, commentstyle=\rmfamily\slshape, keywordstyle=\rmfamily\bfseries, breaklines=true, frame=single, xleftmargin=3mm, xrightmargin=3mm, tabsize=2, mathescape=true}

\renewcommand{\figurename}{Figure}

\maketitle
\thispagestyle{fancy}

%%%%%%%%%%%%%%%%%%%%%%%%%%%%%%
%% Aufgabe 1 %%%%%%%%%%%%%%%%
%%%%%%%%%%%%%%%%%%%%%%%%%%%%%%
\subsection*{Aufgabe 1}
Sei $X$ ein Prähilbertraum und $P: X \to X$ linear. Z.z: Folgende drei Aussagen sind äquivalent
\begin{enumerate}[(a)]
\item $(x-Px,y) = 0$, für alle $x \in X$, $y \in R(P)$ 
\item $P^2 = P$ und $(Px,y) = (x,Py)$, für alle $x,y \in X$ 
\item $P^2 = P$ und $\|P\| \leq 1$
\end{enumerate}
\textbf{Beweis}:\\
(a) $\Leftrightarrow$ (b): \\
(i) "$\Rightarrow$": \\
Wir zeigen zunächst den ersten Teil.\\
Sei $x \in X$ beliebig aber fest.
\begin{equation*}\begin{array}{rl}
& \left\langle Px - P^2 x, Px - P^2 x \right\rangle\\
\stackrel{\text{Bilinear}}{=}&
    \left\langle Px - P^2 x, Px \right\rangle - \left\langle Px - P^2 x, P^2 x\right\rangle\\
\stackrel{Px \in X\text{, Vor.}}{=}&
    0 - 0 = 0
\end{array}\end{equation*}

Da $\left\langle . \right\rangle$ ein Skalarprodukt ist, kann nur
\begin{equation*}\begin{split}
    Px - P^2 x &= 0\\
    Px      &= P^2 x
\end{split}\end{equation*}
und somit muss $P = P^2$ gelten.\\

Zum zweiten Teil:
Seien $x, y \in X$, dann gilt nach a)
\begin{equation*}\begin{array}{crcl}
& \left\langle x - Px, Py \right\rangle &=& 0\\
\Leftrightarrow &
\left\langle x , Py \right\rangle - \left\langle Px, Py \right\rangle &=& 0\\
\Leftrightarrow &
\left\langle x, Py \right\rangle &=& \left\langle Px, Py + y - y \right\rangle\\
\Leftrightarrow &
\left\langle x , Py \right\rangle &=&
\left\langle Px, y \right\rangle + \left\langle Px, Py - y \right\rangle\\
\stackrel{\left\langle y - Py, Px \right\rangle = 0}{\Leftrightarrow}&
\left\langle x , Py \right\rangle &=& \left\langle Px , y \right\rangle
\end{array}\end{equation*}

Somit haben wir diesen Teil durch zweifache Anwendung der Vorraussetzung erreicht. \\
(ii) "$\Leftarrow$": \\
Es gelte $P^2 = P$ und $P$ selbstadjungiert. Dann gilt für $y = Pa$, für ein $a \in X$:
\begin{equation*}\begin{split}
(x-Px,y) &= (x,Pa) - (Px,Pa) \\
      &\stackrel{P^2 = P}{=} (x,P^2 a) - (Px,Pa) \\
      &\stackrel{P \text{ selbstadj.}}{=} (Px, Pa) - (Px,Pa) = 0
\end{split}\end{equation*}
(b) $\Leftrightarrow$ (c): \\
(i) "$\Rightarrow$": \\
Da $P^2 = P$ und $P$ selbstadjungiert (also $P = P^t$) folgt $P = PP = P P^t$. Also ist $P$ symmetrisch und damit (reell) diagonalisierbar. Sei $B$ die Basis aus Eigenvektoren sodass
\begin{equation*}
 P_B = \left(\begin{array}{cccc}
      \lambda_1 & 0 & \hdots & 0\\
      0 & \lambda_2 & \hdots & \vdots \\
      \vdots & \vdots & \vdots & \vdots \\
      0 & 0 & \hdots & \lambda_n
      \end{array} \right)
\end{equation*}
wobei $\lambda_i$ die Eigenwerte von $P$ sind.
Wegen $P^2 = P$ ist $\lambda_i \in \{0,1 \}$. Also ist
$\| Px \| \leq \|x\|$ und dadurch auch $\|P \| \leq 1$.

(ii) "$\Leftarrow$": Gute Frage ;)\\

%%%%%%%%%%%%%%%%%%%%%%%%%%%%%%
%% Aufgabe 2 %%%%%%%%%%%%%%%%
%%%%%%%%%%%%%%%%%%%%%%%%%%%%%%
\subsection*{Aufgabe 2}
Seien $w: [0,1] \to \mathbb{R}$, mit $w(x) := - \ln x$ und $(u,v)$ definiert durch
$$ (u,v) := \int_0^1 w(x)u(x)v(x) \, dx $$
Dann ergeben sich die Orthogonalpolynome $v_0,v_1,v_2$ vom Grad 0,1,2 (respektive) durch Anwendung
des Gram-Schmidt-Verfahrens wie folgt: \\

Wähle $\{w_0, w_1, w_2\}$ mit $w_i = x^i$ als Basis des $P_2$.
\begin{enumerate}
\item $v_0 := w_0 = 1$ 
\item $v_1 = w_1 - \frac{(v_0,w_1)}{(v_0,v_0)} v_0 = x - \frac{\int_0^1 - x \cdot \ln x \, dx}{\int_0^1 - \ln x \, dx} x
           = x - \frac{1}{4} x = \frac{3}{4}x$
\item $v_2 = w_2 - \frac{(v_1,w_2)}{(v_1,v_1)} v_1
           = \frac{3}{4} \left( x^2 - \frac{\int_0^1 - \frac{3}{4}x^3 \ln x \, dx }{ \int_0^1 - \frac{9}{16}x^2\ln x \, dx} x \right)
           = \frac{3}{4} \left(x^2 - \frac{\frac{3}{64}}{\frac{625}{10000}} x\right)
           = \frac{3}{4} \left(x^2 - \frac{3}{4} x\right) = \frac{3}{4}x^2 - \frac{9}{16}x$
\end{enumerate}

Also ergibt sich $\{1,\frac{3}{4}x,\frac{3}{4}x^2 - \frac{9}{16}x \}$ als Orthogonalbasis.
%%%%%%%%%%%%%%%%%%%%%%%%%%%%%%
%% Aufgabe 3 %%%%%%%%%%%%%%%%
%%%%%%%%%%%%%%%%%%%%%%%%%%%%%%
\subsection*{Aufgabe 3}
Sei $h := x_{i+1} - x_i = \frac{1}{n}$ der Abstand des Gitters.
\begin{enumerate}[a)]
\item Es gilt $(\varphi_i,\varphi_j) = 0$ für $|i - j| \geq 2$. Allerdings ist $(\varphi_i,\varphi_{i+1}) = \frac{1}{6}h = \frac{1}{6n}$, für $i = 0,\ldots,n-1$ bzw. $(\varphi_{i-1},\varphi_i) = \frac{1}{6n}$, für $i = 1,\ldots,n$. Also ist die Knotenbasis nicht orthogonal.
Beim Rest haben wir leider keine Ahnung was gemacht werden soll :D
\end{enumerate}


\label{LastPage}
\end{document}
