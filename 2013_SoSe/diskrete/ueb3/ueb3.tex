\documentclass[11pt,a4paper,ngerman]{article}
\usepackage[bottom=2.5cm,top=2.5cm]{geometry} 
\usepackage{babel}
\usepackage[utf8]{inputenc} 
\usepackage[T1]{fontenc} 
\usepackage{ae} 
\usepackage{amssymb} 
\usepackage{amsmath}
\usepackage{amsthm} 
\usepackage{graphicx}
\usepackage{fancyhdr}
\usepackage{fancyref}
\usepackage{listings}
\usepackage{xcolor}
\usepackage{paralist}
\usepackage{tabularx}
\usepackage{tikz}
\usetikzlibrary{arrows,automata}
\usepackage[pdftex, bookmarks=false, pdfstartview={FitH}, linkbordercolor=white]{hyperref}
\usepackage{fancyhdr}
\pagestyle{fancy}
\fancyhead[C]{Diskrete Mathematik}
\fancyhead[L]{Übung 3}
\fancyhead[R]{SoSe 2013}
\fancyfoot{}
\fancyfoot[L]{}
\fancyfoot[C]{\thepage \hspace{1px} of \pageref{LastPage}}
\renewcommand{\footrulewidth}{0.5pt}
\renewcommand{\headrulewidth}{0.5pt}
\setlength{\parindent}{0pt} 
\setlength{\headheight}{0pt}
\newcommand\mdoubleplus{\ensuremath{\mathbin{+\mkern-10mu+}}}
\date{}
\title{Übung 3}
\author{Max Wisniewski, Alexander Steen}


%%
%% Enviroments for proofs and lemmas
%%
\newtheorem{lemma}{\bfseries Claim}

\begin{document}

\lstset{language=Pascal, basicstyle=\ttfamily\fontsize{10pt}{10pt}\selectfont\upshape, commentstyle=\rmfamily\slshape, keywordstyle=\rmfamily\bfseries, breaklines=true, frame=single, xleftmargin=3mm, xrightmargin=3mm, tabsize=2, mathescape=true}

\renewcommand{\figurename}{Figure}

\maketitle
\thispagestyle{fancy}

%%%%%%%%%%%%%%%%%%%%%%%%%%%%
%% Aufgabe 1
%%%%%%%%%%%%%%%%%%%%%%%%%%%%
\subsection*{Aufgabe 1.}
\mbox{}\hfill$\square$

%%%%%%%%%%%%%%%%%%%%%%%%%%%%
%% Aufgabe 2
%%%%%%%%%%%%%%%%%%%%%%%%%%%%
\subsection*{Aufgabe 2.}
\begin{enumerate}
  \item Zu zeigen: $\sum_{i=r}^{n}{\binom{i}{r}} = \binom{n+1}{r+1}$ \\
        \textbf{Beweis}:\\
        Induktionsanfang: $n = r \in \mathbb{N}$
        \begin{equation*}
          \sum_{i=r}^{r}{\binom{i}{r}} = \binom{r}{r} = 1 = \binom{r+1}{r+1}
        \end{equation*}
        Induktionsschritt: $n+1 > r \in \mathbb{N}$
        \begin{equation*}\begin{split}
          \sum_{i=r}^{n+1}{\binom{i}{r}} &= \sum_{i=r}^{n}{\binom{i}{r}} + \binom{n+1}{r} \\
          &\stackrel{IV}{=} \binom{n+1}{r+1} + \binom{n+1}{r}\\
          &\stackrel{Rekur.}{=} \binom{n+2}{r+1}
        \end{split}\end{equation*}
\mbox{}\hfill$\square$
  \item Zu zeigen: $|M| =\binom{n+r-1}{r-1}$, mit $M =  \{ (k_1,\ldots,k_r) \in \mathbb{N}^r | \sum_{i=1}^r k_i = n\} $, für $n,r \in \mathbb{N}, r \geq 1$. \\
        \textbf{Beweis}: Sei $n \in \mathbb{N}$.\\
        Induktionsanfang: $r = 1$
        \begin{equation*}
           |M| \stackrel{}{=} 1 = \binom{n}{0} = \binom{n+r-1}{r-1}
        \end{equation*}
        Induktionsschritt: $r+1 > 1$
        \begin{equation*}\begin{split}
            |M| =&  |\{ (k_1,\ldots,k_r,k_{r+1}) \in \mathbb{N}^{r+1} | \sum_{i=1}^{r+1} k_i = n\}| \\
              =& |\{ (k_1,\ldots,k_r,0)| \sum_{i=1}^r k_i = n\}|
               + |\{ (k_1,\ldots,k_r,1)| \sum_{i=1}^r k_i = n-1\}| \\
              &+ \ldots 
              + |\{ (k_1,\ldots,k_r,n-1)| \sum_{i=1}^r k_i = 1\}|
               + |\{ (k_1,\ldots,k_r,n) | \sum_{i=1}^r k_i = 0\}| \\
              \stackrel{IV}{=}& \binom{n+r-1}{r-1} + \binom{n+r-2}{r-1} + \ldots + \binom{r-1}{r-1} \\
              \stackrel{1.)}{=}& \binom{n+r}{r} = \binom{n+(r+1)-1}{(r+1)-1}
        \end{split}\end{equation*}
\end{enumerate}
\mbox{}\hfill$\square$

%%%%%%%%%%%%%%%%%%%%%%%%%%%%
%% Aufgabe 3
%%%%%%%%%%%%%%%%%%%%%%%%%%%%
\subsection*{Aufgabe 3}

Wir betrachten den Namen \textbf{Max}: Wir können $3! = 6$ verschiedene Worte damit bilden.\\
Wir betrachten den Namen \textbf{Alexander}: Wir können damit $\frac{9!}{2\cdot 2} = 90720$ verschiedene Worte bilden. Die Division durch vier ergibt sich aus den doppelten Buchstaben des Namens (nämlich zwei doppelte).

\subsection*{Aufgabe 4}
Sei $M$ eine Menge mit $|M| =: n \in \mathbb{N}$ Elementen. Z.z: Die Hälfe der Teilmengen von $M$ hat eine gerade Anzahl von Elementen. \\
\textbf{Beweis} durch Induktion über $n$: \\
  Induktionsanfang: $n = 1$ \\
        Sei o.B.d.A $M = \{ a \}$. Dann ist
        \begin{equation*}
          2^M = \{\emptyset, {a} \}
        \end{equation*}
        wobei die Teilmenge $\emptyset$ eine gerade Anzahl von Elementen hat (nämlich Null) und
        die Teilmenge $M$ eine ungerade Anzahl von Elementen hat. Damit besitzt die Hälfte aller
        Teilmengen eine gerade Kardinalität. \\
        Induktionsschritt: $n > 1$ \\
        Sei o.B.d.A. $M = M' \mathaccent\cdot\cup \{ a \}$.
        Die Teilmengen von $M$ sind nun die Teilmengen von $M'$ zusammen mit den um $\{a\}$ 
        erweiterten Teilmengen von $M'$.
        
        Sei hierfür $2^M \mdoubleplus A$ für zwei Mengen $M,A$ definiert als
        $2^M \mdoubleplus A := \{m \cup A | m \in 2^M\} $. Dann ist
        \begin{equation*}\begin{split}
          2^M &= 2^{M'} \mathaccent\cdot\cup (2^{M'} \mdoubleplus \{a\})
        \end{split}\end{equation*}
        Die Hälfte der Mengen von $2^{M'}$ hat nach Induktionsvoraussetzung eine gerade
        Kardinalität. Die Hälfte der Mengen von $2^{M'} \mdoubleplus \{a\}$ hat ebenfalls
        eine gerade Kardinalität, da in jede Menge von $2^{M'}$ jeweils ein Element hinzugefügt
        wird und dadurch die Mengen von $2^{M'} \mdoubleplus \{a\}$,
        die vorher eine ungerade Kardinalität basaßen nun eine
        gerade Kardinalität besitzen und andersherum. Da Vereinigung disjunkt ist, 
        besitzen nun auch die Hälfte der Mengen von $2^M$ eine gerade Anzahl von Elementen.

\label{LastPage}

\end{document}
