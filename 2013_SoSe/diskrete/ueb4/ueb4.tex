\documentclass[11pt,a4paper,ngerman]{article}
\usepackage[bottom=2.5cm,top=2.5cm]{geometry} 
\usepackage{babel}
\usepackage[utf8]{inputenc} 
\usepackage[T1]{fontenc} 
\usepackage{ae} 
\usepackage{amssymb} 
\usepackage{algpseudocode}
\usepackage{algorithm}
\usepackage{amsmath}
\usepackage{amsthm} 
\usepackage{graphicx}
\usepackage{fancyhdr}
\usepackage{fancyref}
\usepackage{listings}
\usepackage{xcolor}
\usepackage{paralist}
\usepackage{tabularx}
\usepackage{tikz}
\usetikzlibrary{arrows,automata}
\usepackage[pdftex, bookmarks=false, pdfstartview={FitH}, linkbordercolor=white]{hyperref}
\usepackage{fancyhdr}
\pagestyle{fancy}
\fancyhead[C]{Diskrete Mathematik}
\fancyhead[L]{Übung 3}
\fancyhead[R]{SoSe 2013}
\fancyfoot{}
\fancyfoot[L]{}
\fancyfoot[C]{\thepage \hspace{1px} of \pageref{LastPage}}
\renewcommand{\footrulewidth}{0.5pt}
\renewcommand{\headrulewidth}{0.5pt}
\setlength{\parindent}{0pt} 
\setlength{\headheight}{0pt}
\newcommand\mdoubleplus{\ensuremath{\mathbin{+\mkern-10mu+}}}
\date{}
\title{Übung 4}
\author{B. Sc. Max Wisniewski, B. Sc. Alexander Steen}
\newcommand{\argmin}{\text{argmin}}
\newcommand{\argmax}{\text{argmax}}
\renewcommand{\algorithmicrequire}{\textbf{Input:}}
\renewcommand{\algorithmicensure}{\textbf{Output:}}
%%
%% Enviroments for proofs and lemmas
%%
\newtheorem{lemma}{\bfseries Claim}

\begin{document}

\lstset{language=Pascal, basicstyle=\ttfamily\fontsize{10pt}{10pt}\selectfont\upshape, commentstyle=\rmfamily\slshape, keywordstyle=\rmfamily\bfseries, breaklines=true, frame=single, xleftmargin=3mm, xrightmargin=3mm, tabsize=2, mathescape=true}

\renewcommand{\figurename}{Figure}

\maketitle
\thispagestyle{fancy}

%%%%%%%%%%%%%%%%%%%%%%%%%%%%
%% Aufgabe 1
%%%%%%%%%%%%%%%%%%%%%%%%%%%%
\subsection*{Aufgabe 1.}
Wir betrachten folgende Algorithmus als ''dualen Greedyalgorithmus'':
\begin{algorithmic}[1]
\Require Graph $G = ([n],E)$ zusammenhängend, $c \in \mathbb{R}^n$
\Ensure $T^* = ([n], E^*) \in \argmin_{T \in T_G} c(T)$
\State $E^* \gets E$
\While{$|E^*| > n-1$}
  \State wähle $e \in \argmax_{e \in E^*} c_e$, sodass $e$ keine Brücke
  \State $E^* \gets E^* \setminus \{e \}$
\EndWhile
\end{algorithmic}
Terminierung: Der Algorithmus terminiert, da jeder Baum mit $n$ Knoten genau $n-1$ Kanten besitzt und
pro Iteration genau eine Kante entfernt wird. \\
Korrektheit: 

%%%%%%%%%%%%%%%%%%%%%%%%%%%%
%% Aufgabe 2
%%%%%%%%%%%%%%%%%%%%%%%%%%%%
\subsection*{Aufgabe 2.}
Idee: Fasse am Anfang jeden Knoten des Eingabegraphen als eigene Zusammenhangskomponente auf und verschmelze diese nach und nach, bis die Anzahl der Zusammenhangskomponenten nicht mehr sinken kann.
Bei dem Verschmelzen werden maximale Kanten zuerst gewählt. Im schlimmsten Fall müssen $n-1$ Verschmelzungen durchgeführt werden (dann ist der Graph zusammenhängend und der Output ist ein MST).

Damit ergibt sich folgender Algorithmus:
\begin{algorithmic}[1]
\Require Graph $G = ([n],E)$, $c \in \mathbb{R}^E$
\Ensure Wald $W = ([n], E^*) \in \argmax_{W \in W_G} c(W)$
\State $U_i \gets \{ i\}, E_i \gets \emptyset, T_i = (U_i, E_i), i \in [n]$
\For{$k = 1,\ldots,n-1$}
\State wähle $U_i \neq \emptyset$, sodass $\delta(U_i) \neq \emptyset$
\State wähle $e_i = u_i v_i \in \argmax_{e \in \delta (U_i)} c_e$
\State bestimme $j: v_i \in U_j$
\State $U_i \gets U_i \cup U_j$, $E_i \gets E_i \cup E_j \cup \{u_iv_i \}$, $T_i \gets (U_i, E_i)$
\State $U_j \gets \emptyset$, $E_j \gets \emptyset$, $T_j \gets (U_j, E_j)$.
\EndFor
\State $E^* \gets \bigcup E_i$, $W \gets ([n], E^*)$
\end{algorithmic}

Korrektheit folgt aus Satz 4.4 für jede einzelne Zusammenhangskomponente $E_i \neq \emptyset$ von $G$, also jeweils $F = E_i$, $U = U_i$, $e = e_i$ für $k = 1,\ldots,n-1$ und $E_i \neq \emptyset$.
%%%%%%%%%%%%%%%%%%%%%%%%%%%%
%% Aufgabe 3
%%%%%%%%%%%%%%%%%%%%%%%%%%%%
\subsection*{Aufgabe 3}
Sei $G = (V,E)$ Graph und $T$ ein aufspannender BFS-Baum zur Wurzel $v \in V$. Sei $w \in V$ ein Knoten und $P$ ein vw-Weg in $G$ und $Q$ der eindeutige vw-Weg in $T$. Zz. $|E(Q)| \leq |E(P)|$. \\

\textbf{Beweis} durch Induktion über $n := dist(v,w)$ \\
\textbf{I.A.} $n = 1$ \\
Da $v$ und $w$ inzident, wurde $w$ in der ersten Iteration der Breitensuche gefunden. Also ist $|E(Q)| = 1 \leq |E(W)|$ für jeden vw-Weg $W$ in $G$. \\

\textbf{I.S.} $n-1 \rightsquigarrow n$ \\
Sei $Q = v,u_1,\ldots,u_j,w$ für $j \geq 1$ der eindeutig bestimmte $vw$-Weg in $T$.
Für den Weg $Q' = v,u_1,\ldots,u_j$ in $T$ gilt nach I.V. $n-1 = |E(Q')| \leq |E(W')|$ für jeden $vu_j$-Weg $W'$ in $G$.
Nun wird $w$ in der nächsten Breitensuche-Iteration gefunden und es ist $Q = Q'w$ und damit $|E(Q')|+1 =|E(Q)| = n  \leq |E(W)|$ für jeden $vw$-Weg $W$ in $G$.
\mbox{} \hfill $\square$
%%%%%%%%%%%%%%%%%%%%%%%%%%%%
%% Aufgabe 4
%%%%%%%%%%%%%%%%%%%%%%%%%%%%
\subsection*{Aufgabe 4}
\begin{enumerate}[i)]
\item Breitensuche: Im schlimmsten Fall wird jede Kante und jeder Knoten besucht, also beträgt die Laufzeit $O(|V| + |E|)$.
\item Prim: $O(|E| \log |V|)$ für die Implementierung mit Prioritätswarteschlange (Heap) und einer Adjazenzliste.
\end{enumerate}

\label{LastPage}

\end{document}
