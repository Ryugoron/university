\documentclass[11pt,a4paper,ngerman]{article}
\usepackage[bottom=2.5cm,top=2.5cm]{geometry} 
\usepackage{babel}
\usepackage[utf8]{inputenc} 
\usepackage[T1]{fontenc} 
\usepackage{ae} 
\usepackage{amssymb} 
\usepackage{amsmath}
\usepackage{amsthm} 
\usepackage{graphicx}
\usepackage{fancyhdr}
\usepackage{fancyref}
\usepackage{listings}
\usepackage{xcolor}
\usepackage{paralist}
\usepackage{tabularx}
\usepackage{tikz}
\usetikzlibrary{arrows,automata}
\usepackage[pdftex, bookmarks=false, pdfstartview={FitH}, linkbordercolor=white]{hyperref}
\usepackage{fancyhdr}
\pagestyle{fancy}
\fancyhead[C]{Diskrete Mathematik}
\fancyhead[L]{Übung 5}
\fancyhead[R]{SoSe 2013}
\fancyfoot{}
\fancyfoot[L]{}
\fancyfoot[C]{\thepage \hspace{1px} of \pageref{LastPage}}
\renewcommand{\footrulewidth}{0.5pt}
\renewcommand{\headrulewidth}{0.5pt}
\setlength{\parindent}{0pt} 
\setlength{\headheight}{0pt}
\newcommand\mdoubleplus{\ensuremath{\mathbin{+\mkern-10mu+}}}
\date{}
\title{Übung 5}
\author{Max Wisniewski, Alexander Steen}
\newcommand{\argmin}{\text{argmin}}
\newcommand{\argmax}{\text{argmax}}
\newcommand{\code}[1]{\langle {#1} \rangle}
\newcommand{\Z}{\mathbb{Z}}
%%
%% Enviroments for proofs and lemmas
%%
\newtheorem{lemma}{\bfseries Claim}

\begin{document}

\lstset{language=Pascal, basicstyle=\ttfamily\fontsize{10pt}{10pt}\selectfont\upshape, commentstyle=\rmfamily\slshape, keywordstyle=\rmfamily\bfseries, breaklines=true, frame=single, xleftmargin=3mm, xrightmargin=3mm, tabsize=2, mathescape=true}

\renewcommand{\figurename}{Figure}

\maketitle
\thispagestyle{fancy}

%%%%%%%%%%%%%%%%%%%%%%%%%%%%
%% Aufgabe 1
%%%%%%%%%%%%%%%%%%%%%%%%%%%%
\subsection*{Aufgabe 1.}
\begin{enumerate}[a)]
\item Zu zeigen: $\forall a,b \in \Z: \, \code{ab} \leq \code{a} + \code{b}$ 
\begin{equation*}
 \code{ab} \leq \log(|ab| + 1)+2 \leq (\log(|a|+1)+1) + (\log(|b|+1)+1) \leq \code{a} + \code{b}
\end{equation*}
\item Zu zeigen: $\forall A \in \Z^{n \times n}: \code{\det A} \leq \code{A} - n^2 + 1$
\begin{equation*}\begin{split}
\code{\det A} &= \code{\sum_{\sigma \in S_n} {sgn(\sigma) \prod_{i=1}^n a_{i \sigma(i)}}} \\
 &= \sum_{\sigma \in S_n} \code{{\prod_{i=1}^n a_{i \sigma(i)}}} \\
  &\stackrel{(a)}{\leq} \sum_{\sigma \in S_n} \sum_{i=1}^n \code{{a_{i \sigma(i)}}} \\
  &= \sum_{i=1}^n \sum_{j=1}^n \code{a_{ij}} - \sum_{\sigma \notin S_n} \sum_{i=1}^n \code{{a_{i \sigma(i)}}} \\
  &= \code{A} - \sum_{\sigma \notin S_n} \sum_{i=1}^n \code{{a_{i \sigma(i)}}} \\
  &\stackrel{(*)}{\leq} \code{A} - \left( \sum_{i=1}^n \sum_{j=1}^n \code{a_{ij}} \right) +1 \\
  &\stackrel{\code{.} \geq 1}{\leq} \code{A} - \left( \sum_{i=1}^n \sum_{j=1}^n 1 \right) +1 \\
  &= \code{A} - n^2 + 1
\end{split}\end{equation*}
(*) gilt, da wir in $S_n$ über weniger iterieren als in der nächsten Zeile. Wir ergänzen alles
  und da $S_n \not= \emptyset$ wissen wir, dass wir eins übrigbehalten können. Fügen wir also
  den Rest hinzu, so erhalten wir etwa die nächste Zeile.
\item Zu zeigen: $\forall k \in \Z: k \leq 2^{\code{k}}$ \\
\begin{equation*}
2^{\code{k}} \geq 2^{\log(|k| + 1)} \geq 2^{\log(|k|)} = |k| \geq k
\end{equation*}
\end{enumerate}
\subsection*{Aufgabe 2.}
\begin{enumerate}[a)]
\item Z.z. \textsc{Clique} ist NP-vollständig \\
Sei $G = (V,E)$ ein Graph und $k \in \mathbb{N}$.\\
(1) \textsc{Clique} $\in$ NP \\
Als Zeuge wird die Menge der Knoten $\{v_1,\ldots,v_k \} \subseteq V $ in der Clique genutzt. Dann kann in $O(k^2)$ Zeit überprüft werden (Adjazenzmatrix), ob es sich um eine Clique in $G$ handelt.\\
(2) \textsc{Independent Set} $\preceq$ \textsc{Clique} \\
Die Reduktion ist gegeben durch
$$ f: (G, k) \mapsto (G^*, k)$$
wobei $G^* = (V, V^2 \setminus E)$ den zu $G = (V,E)$ komplementären Graphen bezeichnet. \\
Z.z. $(G,k) \in $ \textsc{Independent Set} $\Leftrightarrow f(G,k) \in $ \textsc{Clique}\\
"$\Rightarrow$": Sei $(G,k) \in$ \textsc{Independent Set}.
Dann ex. $U \subseteq V$, mit $|U| = k$ und $U$ stabile Menge, also $\forall u_1, u_2 \in U: (u_1,u_2) \notin E$. Damit gilt dann $\forall u_1, u_2 \in U: (u_1,u_2) \in V^2 \setminus E$. Also ist $f(G,k) \in$ \textsc{Clique}. \\
"$\Leftarrow$": Sei $(G,k) \notin$ \textsc{Independent Set}.
Also existiert keine stabile Knotenteilmenge $U \subseteq V$ der Größe $k$. Also gibt es auch keine $k$-Clique im Komplementgraphen $G^*$ und damit $f(G,k) \notin$ \textsc{Clique}.
\mbox{} \hfill $\square$
\item Z.z. \textsc{Longest Path} ist NP-vollständig \\
Sei $G = (V,E)$ ein Graph, $v,w \in V$ und $k \in \mathbb{N}$. \\
(1) \textsc{Longest Path} $\in$ NP \\
Als Zeuge für $(G,v,w,k) \in$ \textsc{Longest Path} wird eben die Kantenfolge $e_1,\ldots,e_z \subseteq E$ mit $z \geq k$ und $e_1 = vv' \in E$, $e_z = w'w \in E$ benutzt. Dies kann in linearer Zeit verifiziert werden. \\
(2) \textsc{Hamilton Pfad} $\preceq$ \textsc{Longest Path} \\
Die Reduktion ist gegeben durch
$$ f: G \mapsto (G',v,w,n+1)$$
wobei $v,w \not\in V$ in $V'$ liegen und wir fügen Kanten hinzu, so dass $v,w$ zu jedem Knoten in $V$ adjazent sind. \\
Z.z. $G \in $ \textsc{Hamilton Pfad} $\Leftrightarrow f(G) \in $ \textsc{Longest Path}\\
\begin{description}
    \item[$\Rightarrow$]:\\
        Sei $G(V,E)$ ein Graph der einen Hamiltonpfad entält.
        Wir wissen, dass ein Hamiltonpfad $|V| = n-1$ Kanten enthält und jeden Knoten besucht.
        Seien $a,b$ die letzen Knoten auf diesem Pfad. Da $v,w$ zu jedem Knoten adjazent sind, nehmen wir nun $va$ und $bw$ in unseren
        Pfad auf. Da diese Kanten in $V$ nicht existiert haben können wir nichts doppelt oder kreuzendes wählen.
        Dieser Pfad von $u$ nach $v$ ist $n+1$ Kanten lang.
    \item[$\Leftarrow$]:\\
        Sei $G(V,E)$ ein Graph und $G'(V',E')$ enthält einen Pfad der Länge $n+1$ von $v$ nach $w$.
        Da wir $v,w$ nicht doppelt anlaufen können sind beide nur an den beiden Enden des Pfades vorhanden.
        Löschen wir nun die erste Kante $va$ und die letzte Kante $bw$ aus dem Pfad. So erhalten wir einen Pfad der länge
        $n-1$ der von $a$ nach $b$ führt.
        Da der Pfad jeden Knoten nur einmal anlaufen sollte, wissen wir, dass auf diesem Pfad $n$ verschiedene Knoten liegen.
        Dies bedeutet, dass komplett $V$ auf dem Pfad liegt und damit ist es ein Hamilton Pfad.
\end{description}
\mbox{}\hfill $\square$


\item Z.z. \textsc{Exact Cover} ist NP-vollständig \\
Sei $M$ eine endliche Menge und $\mathcal{M} \subseteq 2^M$.\\
(1) \textsc{Exact Cover} $\in$ NP \\
Als Zeuge für $(M,\mathcal{M}) \in$ \textsc{Exact Cover} wird die Teilmenge $\mathcal{P} \subseteq \mathcal{M}$ mit $M = \dot\bigcup_{M \in \mathcal{P}} M$ benutzt. Die Überprüfung der Disjunktheit benötigt $O(|\mathcal{P}|^2 \cdot m)$ Schritte, wobei $m = \max_{M \in \mathcal{P}} |M|$, die Überprüfung der Überdeckung dauert $O(|M|)$ Schritte. Insgesamt also polynomielle Zeit in der Eingabe.\\
(2) \textsc{3-SAT} $\preceq$ \textsc{Exact Cover} \\
Ziemlich technische Reduktion. Tenor: Es funzt ;)
\end{enumerate}
\subsection*{Aufgabe 3.}

\subsubsection*{(a)}

Das Unabhängigkeitssystem der stabilen Mengen eines Graphen ist ein Matroid.\\

\textbf{Lösung:}\\

Diese Aussage ist falsch.

Betrachten wir einen einfachen Pfad $v_1v_2v_3$ als Graph. Dann ist $S_1 = \{v_1, v_3 \}$ eine stabile Menge und $S_2 = \{v_2\}$ ist eine stabile
Menge. Da nun $|S_1| = |S_2| + 1$ ist müsste eine Element in der Differenz liegen $y \in S_1 \setminus S_2$, so dass $S_2 \cup \{y\}$ wieder eine
stabile Menge ist. Doch die einzigen Möglichkeiten $\{v_1, v_2\}$ und $\{v_2, v_3\}$ sind nicht unabhängig, da zwischen diesen Knoten eine Kante existiert.

\subsubsection*{(b)}

Das Unabhängigkeitssystem der Teilmengen von Hamiltonkreisen eines Graphen ist ein Matroid.\\

\textbf{Lösung:}\\

Diese Aussage ist falsch.

Wir betrachten einen vollständigen Graphen $G(V,E)$. Sei $v \in V$ ein beliebiger Knoten. Sei o.B.d.A. $|V| > 3$.
Sei $U_1 = \{ (a,v_1), (v_1, b), (b, c) \}$ ein Pfad mit 3 Kanten durch $v_1$ und $U_1 \{ (d, v_1), (v_1, e) \}$

\subsubsection*{(c)}

Sei $G=(V,E)$ ein Graph und $s,t \in V$ zwei verschiedene Knoten. Das System $I := \{ F \subseteq E \, | \, \not \exists \text{st-Weg in } (V,F) \}$
ist ein Unabhängigkeitssystem.\\

\textbf{Lösung:}\\

Die Aussage stimmt.

Zunächst ist $\emptyset \in I$. Wenn es keine Kanten gibt, dann gibt es insbesondere auch keinen Weg. Daher existiert kein Weg von $s$ nach $t$.

Wenn $X \subset F$ und $F \in I$ ist, dann ist $x \in I$. Gäbe es einen Weg von $s$ nach $t$ in $X$, so besteht dieser nur aus Kanten von $X$.
Damit müsste der selbe Weg aber auch in $F$ existieren. Damit ist ausgeschlossen, dass ein solcher Weg in $X$ existiert.

\mbox{}\hfill $\square$

\subsubsection*{(d)}

Das Problem, einen minimalen aufspannenden Baum zu finden, lässt sich als ein Maximierungsproblem über einem Matroid formulieren.\\

\textbf{Lösung:}\\

Wir wählen uns das Unabhängigkeitsystem $I = (E, W)$ der Wälder in einem Graphen $G(V,E)$ mit Kantengewichten $\nu \, : \, E \rightarrow \mathbb{R_+}$.

Das Unabhängigkeitsystem ist ein Matroid. Seien $W_1$, $W_2$ Wälder in $G$ mit $|W_1| = |W_2| + 1$.
Dann existiert in $W_2$ ein Knoten, der noch mit keiner Kanteverbunden ist, der in $W_1$ mit einer Kante verbunden ist.
Wählen wir diese Kante $y$, dann ist $W_2 \cup \{y \}$ ein Wald, da kein Kreis entstehen kann, da von $y$ keine Kante weg führt.\\

Wir brauchen zunächst das Maximale Kantengewicht $m := 1 + \subset{e \in E}{\max} \nu (e)$ und bauen uns so
die Gewichtsfunktion für das Maximierungsproblem $c \, : \, E \rightarrow R_+$ mit
$c(e) = m - \nu (e)$.
Dann ist liefert uns das Maximierungsproblem
$$
    \underset{w \in W}{\max} c(w)
$$
einen minimalen aufspannenden Baum, da die vorher minimalsten Kanten das gewicht nun maximieren werden.

\subsection*{Aufgabe 4.}

\begin{description}
    \item[$a) \Rightarrow b)$]:\\
        Seien $X,Y \in I$ beliebig, aber fest mit $|X| > |Y|$.\\
        Es gilt $\exists Z \subseteq X \setminus Y$ mit $|Z| = |X| - |Y|$, so dass
        $Y \cup Z \in I$.\\
        Induktion über $|Z|$.
        \begin{description}
            \item[I.A.] $|Z| = 1$. Dann ist $|X| = |Y| + 1$ und nach a) existiert ein
                Element im Schnitt $y \in X \setminus Y$, so dass $Y \cup \{y\} \in I$.
                Setze $Z = \{ y \}$.
            \item[I.S.] $|Z| \leadsto |Z|+1$.\\
                Da $I$ Unabhängiges Mengensystem ist, können wir ein $y \in X \setminus Y$ nehmen
                und betrachten dann $X' := X \setminus \{y\}$.
                Nun gilt nach Induktionsvorraussetzung, dass für $X'$ und $Y$ eine Menge $Z'$ existiert
                mit $Z' \subseteq X' \setminus Y$ und $|Z'| = |X'| - |Y|$.
                Nun ist $Y' := Y \cup Z' \in I$. Nun lässt sich auf $X$ und $Y'$ wiederum a) anwenden,
                da nun $|X| = |Y'| + 1$ gilt.
                Wir nehmen wieder $x \in X \setminus Y'$. Da $x \not\in Z'$ ist, können wir es hinzunehmen.
                Da $Y' \cup \{x \} \in I$ und $Y' \cup \{x\} = Y \cup Z' \cup \{x\} = Y \cup (Z' \cup \{x\})$
                können wir $Z := Z' \cup \{x\}$ wählen.
        \end{description}
        \mbox{}\hfill$\square$
    \item[$b) \Rightarrow c)$]:\\
        Sei $Z \in I$ beliebig, aber fest. Seien $X, Y$ zwei Basen von $Z$.
        Angenommen $|X| \not= |Y|$, dann ist o.B.d.A. $|X| > |Y|$. Nach b) existiert nun eine Menge
        $M \subseteq X \setminus Y$ mit $|Z| = |X| - |Y|$ so dass $Z \cup Y \in I$. Da alle Elemente aus $Z \subseteq X$
        auch aus $Z$ kommen ist $Z \cup Y$ genauso aus Elementen von $Z$ gebildet. $Y$ war also nicht maximal und damit nicht Basis.
    \item[$c) \Rightarrow a)$]:\\
    Keine Lust mehr $\ddot\smile$
%        Beweis durch Kontraposition.\\
%        Es existieren zwei Mengen $X,Y \in I$, so dass $|Y| = |X| + 1$ und $\not \exists y \in Y \setminus X \, : \, X \cup \{y\} \in I$.
        
        
\end{description}

\label{LastPage}

\end{document}
