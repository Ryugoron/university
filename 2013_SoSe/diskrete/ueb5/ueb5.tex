\documentclass[11pt,a4paper,ngerman]{article}
\usepackage[bottom=2.5cm,top=2.5cm]{geometry} 
\usepackage{babel}
\usepackage[utf8]{inputenc} 
\usepackage[T1]{fontenc} 
\usepackage{ae} 
\usepackage{amssymb} 
\usepackage{amsmath}
\usepackage{amsthm} 
\usepackage{graphicx}
\usepackage{fancyhdr}
\usepackage{fancyref}
\usepackage{listings}
\usepackage{xcolor}
\usepackage{paralist}
\usepackage{tabularx}
\usepackage{tikz}
\usetikzlibrary{arrows,automata}
\usepackage[pdftex, bookmarks=false, pdfstartview={FitH}, linkbordercolor=white]{hyperref}
\usepackage{fancyhdr}
\pagestyle{fancy}
\fancyhead[C]{Diskrete Mathematik}
\fancyhead[L]{Übung 5}
\fancyhead[R]{SoSe 2013}
\fancyfoot{}
\fancyfoot[L]{}
\fancyfoot[C]{\thepage \hspace{1px} of \pageref{LastPage}}
\renewcommand{\footrulewidth}{0.5pt}
\renewcommand{\headrulewidth}{0.5pt}
\setlength{\parindent}{0pt} 
\setlength{\headheight}{0pt}
\newcommand\mdoubleplus{\ensuremath{\mathbin{+\mkern-10mu+}}}
\date{}
\title{Übung 5}
\author{Max Wisniewski, Alexander Steen}
\newcommand{\argmin}{\text{argmin}}
\newcommand{\argmax}{\text{argmax}}
\newcommand{\code}[1]{\langle {#1} \rangle}
\newcommand{\Z}{\mathbb{Z}}
%%
%% Enviroments for proofs and lemmas
%%
\newtheorem{lemma}{\bfseries Claim}

\begin{document}

\lstset{language=Pascal, basicstyle=\ttfamily\fontsize{10pt}{10pt}\selectfont\upshape, commentstyle=\rmfamily\slshape, keywordstyle=\rmfamily\bfseries, breaklines=true, frame=single, xleftmargin=3mm, xrightmargin=3mm, tabsize=2, mathescape=true}

\renewcommand{\figurename}{Figure}

\maketitle
\thispagestyle{fancy}

%%%%%%%%%%%%%%%%%%%%%%%%%%%%
%% Aufgabe 1
%%%%%%%%%%%%%%%%%%%%%%%%%%%%
\subsection*{Aufgabe 1.}
\begin{enumerate}[a)]
\item Zu zeigen: $\forall a,b \in \Z: \, \code{ab} \leq \code{a} + \code{b}$ 
\begin{equation*}
 \code{ab} \leq \log(|ab| + 1)+2 \leq (\log(|a|+1)+1) + (\log(|b|+1)+1) \leq \code{a} + \code{b}
\end{equation*}
\item Zu zeigen: $\forall A \in \Z^{n \times n}: \code{\det A} \leq \code{A} - n^2 + 1$ \\
\item Zu zeigen: $\forall k \in \Z: k \leq 2^{\code{k}}$ \\
\begin{equation*}
2^{\code{k}} \geq 2^{\log(|k| + 1)} \geq 2^{\log(|k|)} = |k| \geq k
\end{equation*}
\end{enumerate}
\subsection*{Aufgabe 2.}
\begin{enumerate}[a)]
\item Z.z. \textsc{Clique} ist NP-vollständig \\
Sei $G = (V,E)$ ein Graph und $k \in \mathbb{N}$.\\
(1) \textsc{Clique} $\in$ NP \\
Als Zeuge wird die Menge der Knoten $\{v_1,\ldots,v_k \} \subseteq V $ in der Clique genutzt. Dann kann in $O(k^2)$ Zeit überprüft werden (Adjazenzmatrix), ob es sich um eine Clique in $G$ handelt.\\
(2) \textsc{Independent Set} $\preceq$ \textsc{Clique} \\
Die Reduktion ist gegeben durch
$$ f: (G, k) \mapsto (G^*, k)$$
wobei $G^* = (V, V^2 \setminus E)$ den zu $G = (V,E)$ komplementären Graphen bezeichnet. \\
Z.z. $(G,k) \in $ \textsc{Independent Set} $\Leftrightarrow f(G,k) \in $ \textsc{Clique}\\
"$\Rightarrow$": Sei $(G,k) \in$ \textsc{Independent Set}.
Dann ex. $U \subseteq V$, mit $|U| = k$ und $U$ stabile Menge, also $\forall u_1, u_2 \in U: (u_1,u_2) \notin E$. Damit gilt dann $\forall u_1, u_2 \in U: (u_1,u_2) \in V^2 \setminus E$. Also ist $f(G,k) \in$ \textsc{Clique}. \\
"$\Leftarrow$": Sei $(G,k) \notin$ \textsc{Independent Set}.
Also existiert keine stabile Knotenteilmenge $U \subseteq V$ der Größe $k$. Also gibt es auch keine $k$-Clique im Komplementgraphen $G^*$ und damit $f(G,k) \notin$ \textsc{Clique}.
\mbox{} \hfill $\square$
\item Z.z. \textsc{Longest Path} ist NP-vollständig \\
Sei $G = (V,E)$ ein Graph, $v,w \in V$ und $k \in \mathbb{N}$. \\
(1) \textsc{Longest Path} $\in$ NP \\
Als Zeuge für $(G,v,w,k) \in$ \textsc{Longest Path} wird eben die Kantenfolge $e_1,\ldots,e_z \subseteq E$ mit $z \geq k$ und $e_1 = vv' \in E$, $e_z = w'w \in E$ benutzt. Dies kann in linearer Zeit verifiziert werden. \\
(2) \textsc{Hamiltonian Path} $\preceq$ \textsc{Longest Path} \\
Die Reduktion ist gegeben durch
$$ f: G \mapsto (G,v,w,n-1)$$
wobei $vw \in E$ beliebig und $|V| = n$. \\
Z.z. $G \in $ \textsc{Hamiltonian Path} $\Leftrightarrow f(G) \in $ \textsc{Longest Path}\\
HIER FEHLT NOCH DER BEWEIS
\item Z.z. \textsc{Exact Cover} ist NP-vollständig \\
Sei $M$ eine endliche Menge und $\mathcal{M} \subseteq 2^M$.\\
(1) \textsc{Exact Cover} $\in$ NP \\
Als Zeuge für $(M,\mathcal{M}) \in$ \textsc{Exact Cover} wird die Teilmenge $\mathcal{P} \subseteq \mathcal{M}$ mit $M = \dot\bigcup_{M \in \mathcal{P}} M$ benutzt. Die Überprüfung der Disjunktheit benötigt $O(|\mathcal{P}|^2 \cdot m)$ Schritte, wobei $m = \max_{M \in \mathcal{P}} |M|$, die Überprüfung der Überdeckung dauert $O(|M|)$ Schritte. Insgesamt also polynomielle Zeit in der Eingabe.\\
(2) \textsc{3-SAT} $\preceq$ \textsc{Exact Cover} \\
\end{enumerate}
\subsection*{Aufgabe 3.}

\subsection*{Aufgabe 4.}


\label{LastPage}

\end{document}
