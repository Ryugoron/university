\documentclass[11pt,a4paper,ngerman]{article}
\usepackage[bottom=2.5cm,top=2.5cm]{geometry} 
\usepackage{babel}
\usepackage{algpseudocode}
\usepackage{algorithm}

\usepackage[utf8]{inputenc} 
\usepackage[T1]{fontenc} 
\usepackage{ae} 
\usepackage{amssymb} 
\usepackage{amsmath}
\usepackage{amsthm} 
\usepackage{graphicx}
\usepackage{fancyhdr}
\usepackage{fancyref}
\usepackage{listings}
\usepackage{xcolor}
\usepackage{paralist}
\usepackage{tabularx}
\usepackage{tikz}
\usetikzlibrary{arrows,automata}
\usepackage[pdftex, bookmarks=false, pdfstartview={FitH}, linkbordercolor=white]{hyperref}
\usepackage{fancyhdr}
\pagestyle{fancy}
\fancyhead[C]{Diskrete Mathematik}
\fancyhead[L]{Übung 11}
\fancyhead[R]{SoSe 2013}
\fancyfoot{}
\fancyfoot[L]{}
\fancyfoot[C]{\thepage \hspace{1px} of \pageref{LastPage}}
\renewcommand{\footrulewidth}{0.5pt}
\renewcommand{\headrulewidth}{0.5pt}
\setlength{\parindent}{0pt} 
\setlength{\headheight}{0pt}
\newcommand\mdoubleplus{\ensuremath{\mathbin{+\mkern-10mu+}}}
\date{Tutor : Stefan Schwarz}
\title{Übung 11}
\author{Max Wisniewski, Alexander Steen}
\renewcommand{\algorithmicrequire}{\textbf{Input:}}
\renewcommand{\algorithmicensure}{\textbf{Output:}}
\newcommand{\argmin}{\text{argmin}}
\newcommand{\argmax}{\text{argmax}}
\newcommand{\code}[1]{\langle {#1} \rangle}
\newcommand{\Z}{\mathbb{Z}}
%%
%% Enviroments for proofs and lemmas
%%
\newtheorem{lemma}{\bfseries Claim}

\begin{document}

\lstset{language=Pascal, basicstyle=\ttfamily\fontsize{10pt}{10pt}\selectfont\upshape, commentstyle=\rmfamily\slshape, keywordstyle=\rmfamily\bfseries, breaklines=true, frame=single, xleftmargin=3mm, xrightmargin=3mm, tabsize=2, mathescape=true}

\renewcommand{\figurename}{Figure}

\maketitle
\thispagestyle{fancy}

%%%%%%%%%%%%%%%%%%%%%%%%%%%%
%% Aufgabe 1
%%%%%%%%%%%%%%%%%%%%%%%%%%%%
\subsection*{Aufgabe 1.}
\begin{enumerate}[a)]
\item Rekursion für $a_n$: \\
$a_n = 2 a_{n-1} + \sum_{k=1}^{n-2} a_k$, wobei $a_1 := 1$.
\item spezifische Werte bzw. geschlossene Formel: \\
Es gilt $a_2 = 1, a_3 = 3, a_4 = 8, a_5 = 21$.
\item Beweis von (b): \\
\end{enumerate}

%%%%%%%%%%%%%%%%%%%%%%%%%%%%
%% Aufgabe 2
%%%%%%%%%%%%%%%%%%%%%%%%%%%%
\subsection*{Aufgabe 2.}

\begin{enumerate}[a)]
\item Blabla
\item Es gilt $a_n = \frac{1}{2} \left( \left(1-\sqrt{2} \right)^{1+n}+\left(1+\sqrt{2}\right)^{1+n} \right)$. \\
\textbf{Beweis} durch Induktion: \\
Basis: $n = 0,1$

\begin{equation*}
\frac{1}{2} \left((1-\sqrt{2})+(1+\sqrt{2}) \right) = 1 = a_0
\end{equation*}
und
\begin{equation*}\begin{split}
& \frac{1}{2} \left((1-\sqrt{2})^2+(1+\sqrt{2})^2 \right) \\
& = \frac{1}{2} \left(1-2\sqrt{2}+2+1+2\sqrt{2}+2 \right) \\
& = \frac{1}{2} 6 = 3 = a_1
\end{split}\end{equation*}
I.V. Behauptung gilt für alle $n$ mit $1 \geq n_0 < n$ \\
I.S. $n-1 \rightsquigarrow n$ \\
\begin{equation*}\begin{split}
a_{n} &= 2a_{n-1} + a_{n-2} \\
      &\stackrel{IV}{=} 2* \frac{1}{2} \left( \left(1-\sqrt{2} \right)^{n}+\left(1+\sqrt{2}\right)^{n} \right)
        +  \frac{1}{2} \left( \left(1-\sqrt{2} \right)^{n-1}+\left(1+\sqrt{2}\right)^{n-1} \right) \\
      &= \frac{1}{2} \left(\left(1-\sqrt{2} \right)^{n}+\left(1+\sqrt{2}\right)^{n} + \left(1-\sqrt{2} \right)^{n-1}+\left(1+\sqrt{2}\right)^{n-1} \right) \\
      &= \frac{1}{2} \left(\left(2-\sqrt{2} \right) \left(1-\sqrt{2} \right)^{n-1} + \left(2+\sqrt{2} \right) \left(1+\sqrt{2} \right)^{n-1} \right)
\end{split}\end{equation*}
\end{enumerate}

%%%%%%%%%%%%%%%%%%%%%%%%%%%%
%% Aufgabe 3
%%%%%%%%%%%%%%%%%%%%%%%%%%%%
\subsection*{Aufgabe 3.}


%%%%%%%%%%%%%%%%%%%%%%%%%%%%
%% Aufgabe 4
%%%%%%%%%%%%%%%%%%%%%%%%%%%%
\subsection*{Aufgabe 4.}


\label{LastPage}
\end{document}
