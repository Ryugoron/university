\documentclass[11pt,a4paper,ngerman]{article}
\usepackage[bottom=2.5cm,top=2.5cm]{geometry} 
\usepackage{babel}
\usepackage{algpseudocode}
\usepackage{algorithm}

\usepackage[utf8]{inputenc} 
\usepackage[T1]{fontenc} 
\usepackage{ae} 
\usepackage{amssymb} 
\usepackage{amsmath}
\usepackage{amsthm} 
\usepackage{graphicx}
\usepackage{fancyhdr}
\usepackage{fancyref}
\usepackage{listings}
\usepackage{xcolor}
\usepackage{paralist}
\usepackage{tabularx}
\usepackage{tikz}
\usetikzlibrary{arrows,automata}
\usepackage[pdftex, bookmarks=false, pdfstartview={FitH}, linkbordercolor=white]{hyperref}
\usepackage{fancyhdr}
\pagestyle{fancy}
\fancyhead[C]{Diskrete Mathematik}
\fancyhead[L]{Übung 11}
\fancyhead[R]{SoSe 2013}
\fancyfoot{}
\fancyfoot[L]{}
\fancyfoot[C]{\thepage \hspace{1px} of \pageref{LastPage}}
\renewcommand{\footrulewidth}{0.5pt}
\renewcommand{\headrulewidth}{0.5pt}
\setlength{\parindent}{0pt} 
\setlength{\headheight}{0pt}
\newcommand\mdoubleplus{\ensuremath{\mathbin{+\mkern-10mu+}}}
\date{Tutor : Stefan Schwarz}
\title{Übung 11}
\author{Max Wisniewski, Alexander Steen}
\renewcommand{\algorithmicrequire}{\textbf{Input:}}
\renewcommand{\algorithmicensure}{\textbf{Output:}}
\newcommand{\argmin}{\text{argmin}}
\newcommand{\argmax}{\text{argmax}}
\newcommand{\code}[1]{\langle {#1} \rangle}
\newcommand{\Z}{\mathbb{Z}}
%%
%% Enviroments for proofs and lemmas
%%
\newtheorem{lemma}{\bfseries Claim}

\begin{document}

\lstset{language=Pascal, basicstyle=\ttfamily\fontsize{10pt}{10pt}\selectfont\upshape, commentstyle=\rmfamily\slshape, keywordstyle=\rmfamily\bfseries, breaklines=true, frame=single, xleftmargin=3mm, xrightmargin=3mm, tabsize=2, mathescape=true}

\renewcommand{\figurename}{Figure}

\maketitle
\thispagestyle{fancy}

%%%%%%%%%%%%%%%%%%%%%%%%%%%%
%% Aufgabe 1
%%%%%%%%%%%%%%%%%%%%%%%%%%%%
\subsection*{Aufgabe 1.}
\begin{enumerate}[a)]
\item Rekursion für $a_n$: \\
$a_n = 2 a_{n-1} + \sum_{k=1}^{n-2} a_k$, wobei $a_1 := 1$.
\item spezifische Werte bzw. geschlossene Formel: \\
Es gilt $a_2 = 1, a_3 = 3, a_4 = 8, a_5 = 21$. \\
Geschlossene Formel: Gute Frage!
\item Beweis von (b): Hier gibt's nichts zu tun ;-)\\
\end{enumerate}

%%%%%%%%%%%%%%%%%%%%%%%%%%%%
%% Aufgabe 2
%%%%%%%%%%%%%%%%%%%%%%%%%%%%
\subsection*{Aufgabe 2.}

\begin{enumerate}[a)]
\item Kurze Erklärung, warum die Rekursion Sinn ergibt: \\
Für $n=0$ haben wir eine Möglichkeit ''keinen'' Pfad zu beschreiben, bei $n=1$ drei Stück (nach links, nach rechts, nach oben. \\
Für einen Pfad der Länge $n+1$ können wir für jeden Endpunkt eines Pfades der Länge $n$ den Weg nach oben erweitern (das geht immer, $a_n$ Möglichkeiten) und entweder nach links oder nach rechts (je nachdem, ob vorher links oder rechts gegangen wurde, $a_n$ Möglichkeiten). Die Pfade, die im letzten Schritt einen Schritt nach oben gemacht haben, können wir sowohl links als auch rechts erweitern (also zusätzlich $a_{n-1}$ Möglichkeiten, da wir die Entscheidung ''rückgängig machen können'' um nach oben zu erweitern).

\item Es gilt $a_n = \frac{1}{2} \left( \left(1-\sqrt{2} \right)^{1+n}+\left(1+\sqrt{2}\right)^{1+n} \right)$. \\
\textbf{Beweis} durch Induktion: \\
Basis: $n = 0,1$

\begin{equation*}
\frac{1}{2} \left((1-\sqrt{2})+(1+\sqrt{2}) \right) = 1 = a_0
\end{equation*}
und
\begin{equation*}\begin{split}
& \frac{1}{2} \left((1-\sqrt{2})^2+(1+\sqrt{2})^2 \right) \\
& = \frac{1}{2} \left(1-2\sqrt{2}+2+1+2\sqrt{2}+2 \right) \\
& = \frac{1}{2} 6 = 3 = a_1
\end{split}\end{equation*}
I.V. Behauptung gilt für alle $n$ mit $n < n_0$ \\
I.S. $n-1 \rightsquigarrow n$ \\
\begin{equation*}\begin{split}
a_{n} &= 2a_{n-1} + a_{n-2} \\
      &\stackrel{IV}{=} 2* \frac{1}{2} \left( \left(1-\sqrt{2} \right)^{n}+\left(1+\sqrt{2}\right)^{n} \right)
        +   \left( \left(1-\sqrt{2} \right)^{n-1}+\left(1+\sqrt{2}\right)^{n-1} \right) \\
      &= \left(1-\sqrt{2} \right)^{n}+\left(1+\sqrt{2}\right)^{n} + \frac{1}{2}\left(1-\sqrt{2} \right)^{n-1}+\frac{1}{2}\left(1+\sqrt{2}\right)^{n-1} \\
      &= \frac{1}{2}\left(2 \left(1-\sqrt{2}\right)^{n} + \left(1-\sqrt{2}\right)^{n-1} \right)
          + \frac{1}{2}\left(2 \left(1+\sqrt{2}\right)^{n} + \left(1+\sqrt{2}\right)^{n-1} \right) \\
      &= \frac{1}{2} \left(1-\sqrt{2} \right)^{n+1}
         + \frac{1}{2} \left(1+\sqrt{2} \right)^{n+1} \\
      &= \frac{1}{2}\left(\left(1-\sqrt{2} \right)^{n+1} +\left(1+\sqrt{2} \right)^{n+1}  \right)
\end{split}\end{equation*}
\end{enumerate}

%%%%%%%%%%%%%%%%%%%%%%%%%%%%
%% Aufgabe 3
%%%%%%%%%%%%%%%%%%%%%%%%%%%%
\subsection*{Aufgabe 3.}

\begin{enumerate}[a)]
\item Z.z. $ \sum_{k=0}^\infty \binom{\alpha +k -1}{k} z^k = (1-z)^{-\alpha} $, für $\alpha \in \mathbb{R}$ \\

\textbf{Beweis}: \\
Es gilt die Identität $\binom{-\alpha}{k} = (-1)^k \binom{\alpha + k -1}{k}$. Dann ist
\begin{equation*}\begin{split}
\sum_{k=0}^\infty \binom{\alpha +k -1}{k} z^k
&= \sum_{k=0}^\infty (-1)^k (-1)^k \binom{\alpha +k -1}{k} z^k \\
&= \sum_{k=0}^\infty (-1)^k \binom{-\alpha}{k} z^k \\
&= \sum_{k=0}^\infty \binom{-\alpha}{k} (-z)^k \\
&\stackrel{Binom.Lehrsatz}{=} (1 + (-z))^{-\alpha} = (1 - z)^{-\alpha}
\end{split}\end{equation*}
\item Z.z. $\sum_{k=0}^\infty (-1)^k \binom{n+k}{k} z^k = (1-z)^{n+1}$, für $n \in \mathbb{N}$ \\
Nicht bearbeitet.
%\textbf{Beweis}:
%\begin{equation*}\begin{split}
%\sum_{k=0}^\infty (-1)^k \binom{n+k}{k} z^k 
%&= \sum_{k=0}^\infty (-1)^k \binom{n+k-1}{k-1} z^k + (-1)^k \binom{n+k-1}{k} z^k \\
%\end{split}\end{equation*}
%
\end{enumerate}

%%%%%%%%%%%%%%%%%%%%%%%%%%%%
%% Aufgabe 4
%%%%%%%%%%%%%%%%%%%%%%%%%%%%
\subsection*{Aufgabe 4.}

Für das Zählen erschaffen wir uns eine Indikatorfunktion
$$
    f(0) = 1, \, f(2) = 1, \, f(4) = 1 \, f(x) = 0 \, \text{sonst}
$$
mit der wir nun die Kombinationen zählen.

Wir können uns $2b$ Berliner wählen (damit ist die Anzahl durch 2 teilbar)
und $6m$ Münchener (Bayern, damit wir die bezeichner unterscheiden können).

Dann ist die Anzahl beschrieben durch
$$
    \overset{50}{\underset{b=0}{\sum}} \overset{17}{\underset{m=0}{\sum}} f(100 - 2b - 6m)
$$
Wir testen einfach, ob bei $b$ Berlinern und $m$ Münchern noch $0,2,4$ übrig bleiben. Ist dies
der Fall handelt es sich um eine gültige Lösung.

Um diese gegebene Form zu lösen haben wir in Haskel das Programm geschrieben

\begin{lstlisting}[language=Haskell]
ind :: Integer -> Integer
ind 0   =   1
ind 2   =   1
ind 4   =   1
ind _   =   0

sumOf :: Integer -> Integer -> (Integer -> Integer) -> Integer
sumOf a b f   = foldr (+) 0 [f x | x <- [a..b]]

amount = sumOf 0 50 (\b -> sumOf 0 17 (\m -> ind (100 - 2*x - 6*y)))
\end{lstlisting}

Die Grenzen $50$ und $17$ sind so gewählt, da wir nicht mehr als $50*2 = 100$ Berliner oder
$17*6 = 102$ Münchener geben kann.

Führen wir das Programm aus, so erhalten wir das Ergebnis $51$.

\label{LastPage}
\end{document}
