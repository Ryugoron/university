\documentclass[11pt,a4paper,ngerman]{article}
\usepackage[bottom=2.5cm,top=2.5cm]{geometry} 
\usepackage{babel}
\usepackage{algpseudocode}
\usepackage{algorithm}

\usepackage[utf8]{inputenc} 
\usepackage[T1]{fontenc} 
\usepackage{ae} 
\usepackage{amssymb} 
\usepackage{amsmath}
\usepackage{amsthm} 
\usepackage{graphicx}
\usepackage{fancyhdr}
\usepackage{fancyref}
\usepackage{listings}
\usepackage{xcolor}
\usepackage{paralist}
\usepackage{tabularx}
\usepackage{tikz}
\usetikzlibrary{arrows,automata}
\usepackage[pdftex, bookmarks=false, pdfstartview={FitH}, linkbordercolor=white]{hyperref}
\usepackage{fancyhdr}
\pagestyle{fancy}
\fancyhead[C]{Diskrete Mathematik}
\fancyhead[L]{Übung 7}
\fancyhead[R]{SoSe 2013}
\fancyfoot{}
\fancyfoot[L]{}
\fancyfoot[C]{\thepage \hspace{1px} of \pageref{LastPage}}
\renewcommand{\footrulewidth}{0.5pt}
\renewcommand{\headrulewidth}{0.5pt}
\setlength{\parindent}{0pt} 
\setlength{\headheight}{0pt}
\newcommand\mdoubleplus{\ensuremath{\mathbin{+\mkern-10mu+}}}
\date{}
\title{Übung 7}
\author{Max Wisniewski, Alexander Steen}
\renewcommand{\algorithmicrequire}{\textbf{Input:}}
\renewcommand{\algorithmicensure}{\textbf{Output:}}
\newcommand{\argmin}{\text{argmin}}
\newcommand{\argmax}{\text{argmax}}
\newcommand{\code}[1]{\langle {#1} \rangle}
\newcommand{\Z}{\mathbb{Z}}
%%
%% Enviroments for proofs and lemmas
%%
\newtheorem{lemma}{\bfseries Claim}

\begin{document}

\lstset{language=Pascal, basicstyle=\ttfamily\fontsize{10pt}{10pt}\selectfont\upshape,
commentstyle=\rmfamily\slshape, keywordstyle=\rmfamily\bfseries, breaklines=true, frame=single,
xleftmargin=3mm, xrightmargin=3mm, tabsize=2, mathescape=true}

\renewcommand{\figurename}{Figure}

\maketitle
\thispagestyle{fancy}

%%%%%%%%%%%%%%%%%%%%%%%%%%%%
%% Aufgabe 1
%%%%%%%%%%%%%%%%%%%%%%%%%%%%
\subsection*{Aufgabe 1.}
Reduzieren der folgenden Probleme auf Max-Flow

\begin{enumerate}[a)]
\item Gegeben sei ein Netzwerk $N = (V,E,c,S,T)$ mit $(V,E)$ gerichteter Graph, 
$c \in \mathbb{N}_0^E$ Kapazitätsfunktion,
$S \subseteq V$ Quellenmenge und $T \subseteq V$ Senkenmenge. \\
\textbf{Idee}: Füge explizite Quelle/Senke ein und verbinde diese mit allen vorigen Quellen/Senken (respektive).
Daraus ergibt sich die Konstruktion der ''normales Netzwerk'' $N' = (V', E', c', s, t)$ (mit einer
Quelle/Senke) wie folgt: \\

Für zwei neue Knoten $s,t$ setzen wir \\
$V' := V \cup \{ s,t \}$ \\
$E' := E \cup \{ (s,s') | s' \in S \} \cup \{(t',t) | t' \in T\}$ \\
$c' := e \mapsto \begin{cases}
			c_e & \text{, falls $e \in E$} \\
			c^* & \text{, sonst}
		      \end{cases} \quad$
		      mit $c^* = \max_{s \in S} \left( \sum_{(u,v) \in \delta^+(s)} c_{uv} -  \sum_{(u,v) \in \delta^-(s)} c_{uv}\right)$ \\

Das Gewicht der neuen Kanten $c^*$ ist so gewählt, dass genügend
viel über diese Kanten fließen kann; nämlich mindestens das,
was auch aus den jeweiligen Quellen ausfließen kann.
\item Gegeben sei ein Netzwerk $N = (V,E,c,\tilde{c},s,t)$ mit $(V,E)$ gerichteter Graph, 
$c \in \mathbb{N}_0^E$ Kapazitätsfunktion der Kanten,
$\tilde{c} \in \mathbb{N}_0^V$ Kapazitätsfunktion der Knoten,
$s,t \in V$ Quelle bzw. Senke (respektive). \\
\textbf{Idee}: Zerteile alle Knoten in zwei separate Knoten;
dem einen Knoten werden alle eingehenden Kanten zugeordnet,
dem anderen Knoten werden alle ausgehenden Kanten zugeordet
(jeweils mit deren gegebenen Kapazität).
Die beiden Knoten werden miteinander durch eine Kante verbunden,
die das Kantengewicht des ursprünglichen Knotengewichts trägt.
Daraus ergibt sich die Konstruktion der ''normales Netzwerk'' 
$N' = (V', E', c', s', t')$ (mit einer Kapazitätsfunktion) wie folgt: \\

$V' := \{ v_{in} | v \in V \} \cup \{v_{out} | v \in V \}$ \\
$E' := \{ (u,v_{in}) | u \in \delta^-(v)\} \cup \{ (v_{out},u) | u \in \delta^+(v) \} \cup \{(v_{in},v_{out}) | v \in V \}$ \\
$c' := e \mapsto \begin{cases}
			c_e & \text{, falls $e \in E$} \\
			\tilde{c}_v & \text{, falls $e = (v_{in},v_{out}) \notin E$}
		      \end{cases} $ \\
$s' := s_{in}$ \\
$t' := t_{out}$ \\

Die Einhaltung der zusätzlichen Knotenkapazität ist nun dadurch gewährleistet,
dass durch die '' künstliche Kante'' zwischen $v_{in}$ und $v_{out}$ erzwungen wird,
dass nie mehr als diese Kapazität durch $v_{in}$ und $v_{out}$, also auch nicht 
durch den ursprünglichen Knoten, fließen.
\end{enumerate}

%%%%%%%%%%%%%%%%%%%%%%%%%%%%
%% Aufgabe 2
%%%%%%%%%%%%%%%%%%%%%%%%%%%%
\subsection*{Aufgabe 2.}
Sei $G = (V,E)$ ein gerichteter Graph, $k \in \mathbb{N}$. \\
Z.z. $G$ ist $k$-fach stark kantenzusammenhängend \\
  $\Leftrightarrow \forall s,t \in V: \; \left|\{ \text{ disjunkte gerichtete $st$-Wege } \}\right| \geq k$ \\

\textbf{Beweis}: \\
"$\Rightarrow$": Sei $G$ $k$-fach stark kantenzusammenhängend, $s,t \in V$. \\
Dann gilt nach Annahme $\forall B \subseteq V, |B| \leq k-1: \, (V,E \setminus B)$ enthält einen $st$-Weg.


asd \\

"$\Leftarrow$": Sei $s,t \in V$ und $\mathcal{M} = \{M_1, M_2, \ldots \}$ die Menge aller kantendisjunkter gerichteten $st$-Wege, $|\mathcal{M}| \geq k$.\\
Sei $B \subseteq V$ mit $|B| \leq k-1$, dann ist $|B \cap \left(\cup_{i=1}^{|\mathcal{M}|} M_i\right)| \leq k-1$.
Da außerdem alle $M_i$ kantendisjunkt, ex. also ein $j \in \{1,\ldots,|\mathcal{M}| \}$, sodass $M_j \cap B = \emptyset$.
Also enthält der Graph $(V, E \setminus B)$ mindestens einen $st$-Weg, nämlich $M_j$.
\mbox{} \hfill $\square$
%%%%%%%%%%%%%%%%%%%%%%%%%%%%
%% Aufgabe 3
%%%%%%%%%%%%%%%%%%%%%%%%%%%%
\subsection*{Aufgabe 3.}
Z.z. jeder ganzzahliger $st$-Fluss $f$ mit $|f| \geq 1$ kann in $st$-Wege $P_i$ und Kreise $C_j$ zerlegt werden.

\textbf{Beweis}: Wir geben einen Algorithmus an, der uns die Zerlegung liefert. Bezeichne $f(e)$ den Fluss entlang einer Kante $e \in E$.

\begin{algorithmic}[1]
\Require Netzwerk $N = (V,E,c,s,t)$, $c \in \mathbb{R}^E, f$ gültiger Fluss 
\Ensure Zerlegung von $f$ in $st$-Wege $P$ und Kreise $C$
\While{ex. $st$-Weg $v_1,v_2,\ldots,v_k$ mit $f(v_iv_{i+1}) > 0, \forall i < k$}
\State $E \gets st$-Weg $v_1,v_2,\ldots,v_k$ mit $f(v_iv_{i+1}) > 0, \forall i<k$
\State $f(e) \gets f(e) - 1$, für alle Kanten $e$ des $st$-Weges $E$
\State füge $E$ zu $P$ hinzu
\EndWhile
\While{ex. Kante $e = v_1v_2$ mit $f(e) > 0$}
\State $K \gets $ Kreis $v_1,v_2,\ldots,v_k,v_{k+1}=v_1$ mit $f(v_iv_{i+1}) > 0, \forall i \leq k$
\State $f(e) \gets f(e) - 1$, für alle Kanten $e$ des Kreises $K$
\State füge $K$ zu $C$ hinzu
\EndWhile
\end{algorithmic}

\textbf{Korrektheit}: In den Zeilen 1..5 werden die $st$-Wege extrahiert. Falls in dem Fluss $f$ etwas fließt, also $|f| \geq 1$ gilt, können wir solche Wege nach Knotenbedingung finden.
In den Zeilen 6..10 werden dann die Kreise extrahiert. Falls es nach ''entfernen'' der $st$-Wege noch Kanten mit positivem Fluss gibt, müssen solche Kreise nach Knotenbedingung existieren.
Sowohl Wege als auch Kreise werden nur dann
betrachtet, falls diese auf jeder Kante mindestens die Flussmenge eins besitzen.

%%%%%%%%%%%%%%%%%%%%%%%%%%%%
%% Aufgabe 4
%%%%%%%%%%%%%%%%%%%%%%%%%%%%
\subsection*{Aufgabe 4.}
\begin{enumerate}[a)]
\item Beweis durch Widerspruch: \\
      Angenommen $d^{i+1}(v) < d^i(v)$ für einen Knoten $v \in V \setminus \{s,t\}$.
      Sei $v$ o.B.d.A. der Knoten kürzester Distanz von $s$ der nach der Augmentierung
      die Distanz verkürzte. Sei dann $w = s,\ldots,u,v$ ein kürzester Pfad nach der Augmentierung.
      Dann gilt aber
      \begin{equation*}\begin{split}
        d^{i}(v) & \leq d^{i}(u) + 1 \\
        &\stackrel{(*)}{\leq} d^{i}(u) + 1 \\
        &\leq d^{i+1}(v)
      \end{split}\end{equation*}
      was ein Widerspruch ist.
      (*) gilt, da sich nach Wahl des Pfades $w$ die Distanz von $s$ nach $u$ nicht reduziert.
\item Sei $(v,u) \in E; v,u \in V$. Wird $u$ saturiert, so lag dieser vor der Augmentierung auf einem         
      augmentierendem Weg kürzester Länge, also gilt $d^i(u) = d^i(v) + 1$. Frühestens bei der nächsten   
      Saturierung muss nun die umgekehrte Kante $(u,v)$ auf einem augmentierenden Pfad liegen,
      hier gilt also $d^{i+1}(v) = d^{i+1}(u) + 1$. Dann gilt
      \begin{equation*}\begin{split}
        d^{i+1}(v) & = d^{i+1}(u) + 1 \\
        &\stackrel{(a)}{\geq} d^{i}(u) + 1 \\
        &= d^{i}(v) +2
      \end{split}\end{equation*}
      Also kann ein Knoten höchstens $\frac{|V|}{2} = \frac{n}{2}$ mal saturiert werden.
\item Da im Residualgraphen $O(|E|)$ Kanten enthalten sind und jede Augmentierung mindestens einen Knoten
      saturiert, ist (nach (b)) ist die Anzahl der Augmentierungen $O(|V|\cdot |E|)$. Die Suche des kürzesten
      augmentierenden Pfad kann durch Breitensuche gelöst werden, damit ergibt sich eine Laufzeit von
      $O(|V|\cdot |E|^2)$. Also ist die Laufzeit polynomiell.
\end{enumerate}


\label{LastPage}
\end{document}
