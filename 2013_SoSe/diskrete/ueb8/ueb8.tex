\documentclass[11pt,a4paper,ngerman]{article}
\usepackage[bottom=2.5cm,top=2.5cm]{geometry} 
\usepackage{babel}
\usepackage{algpseudocode}
\usepackage{algorithm}

\usepackage[utf8]{inputenc} 
\usepackage[T1]{fontenc} 
\usepackage{ae} 
\usepackage{amssymb} 
\usepackage{amsmath}
\usepackage{amsthm} 
\usepackage{graphicx}
\usepackage{fancyhdr}
\usepackage{fancyref}
\usepackage{listings}
\usepackage{xcolor}
\usepackage{paralist}
\usepackage{tabularx}
\usepackage{tikz}
\usetikzlibrary{arrows,automata}
\usepackage[pdftex, bookmarks=false, pdfstartview={FitH}, linkbordercolor=white]{hyperref}
\usepackage{fancyhdr}
\pagestyle{fancy}
\fancyhead[C]{Diskrete Mathematik}
\fancyhead[L]{Übung 7}
\fancyhead[R]{SoSe 2013}
\fancyfoot{}
\fancyfoot[L]{}
\fancyfoot[C]{\thepage \hspace{1px} of \pageref{LastPage}}
\renewcommand{\footrulewidth}{0.5pt}
\renewcommand{\headrulewidth}{0.5pt}
\setlength{\parindent}{0pt} 
\setlength{\headheight}{0pt}
\newcommand\mdoubleplus{\ensuremath{\mathbin{+\mkern-10mu+}}}
\date{}
\title{Übung 7}
\author{Max Wisniewski, Alexander Steen}
\renewcommand{\algorithmicrequire}{\textbf{Input:}}
\renewcommand{\algorithmicensure}{\textbf{Output:}}
\newcommand{\argmin}{\text{argmin}}
\newcommand{\argmax}{\text{argmax}}
\newcommand{\code}[1]{\langle {#1} \rangle}
\newcommand{\Z}{\mathbb{Z}}
%%
%% Enviroments for proofs and lemmas
%%
\newtheorem{lemma}{\bfseries Claim}

\begin{document}

\lstset{language=Pascal, basicstyle=\ttfamily\fontsize{10pt}{10pt}\selectfont\upshape,
commentstyle=\rmfamily\slshape, keywordstyle=\rmfamily\bfseries, breaklines=true, frame=single,
xleftmargin=3mm, xrightmargin=3mm, tabsize=2, mathescape=true}

\renewcommand{\figurename}{Figure}

\maketitle
\thispagestyle{fancy}

%%%%%%%%%%%%%%%%%%%%%%%%%%%%
%% Aufgabe 1
%%%%%%%%%%%%%%%%%%%%%%%%%%%%
\subsection*{Aufgabe 1.}
Reduzieren der folgenden Probleme auf Max-Flow

\begin{enumerate}[a)]
\item Gegeben sei ein Netzwerk $N = (V,E,c,S,T)$ mit $(V,E)$ gerichteter Graph, 
$c \in \mathbb{N}_0^E$ Kapazitätsfunktion,
$S \subseteq V$ Quellenmenge und $T \subseteq V$ Senkenmenge. \\
\textbf{Idee}: Füge explizite Quelle/Senke ein und verbinde diese mit allen vorigen Quellen/Senken (respektive).
Daraus ergibt sich die Konstruktion der ''normales Netzwerk'' $N' = (V', E', c', s, t)$ (mit einer
Quelle/Senke) wie folgt: \\

Für zwei neue Knoten $s,t$ setzen wir \\
$V' := V \cup \{ s,t \}$ \\
$E' := E \cup \{ (s,s') | s' \in S \} \cup \{(t',t) | t' \in T\}$ \\
$c' := e \mapsto \begin{cases}
			c_e & \text{, falls $e \in E$} \\
			c^* & \text{, sonst}
		      \end{cases} \quad$
		      mit $c^* = \max_{s \in S} \left( \sum_{(u,v) \in \delta^+(s)} c_{uv} -  \sum_{(u,v) \in \delta^-(s)} c_{uv}\right)$ \\

Das Gewicht der neuen Kanten $c^*$ ist so gewählt, dass genügend
viel über diese Kanten fließen kann; nämlich mindestens das,
was auch aus den jeweiligen Quellen ausfließen kann.
\item Gegeben sei ein Netzwerk $N = (V,E,c,\tilde{c},s,t)$ mit $(V,E)$ gerichteter Graph, 
$c \in \mathbb{N}_0^E$ Kapazitätsfunktion der Kanten,
$\tilde{c} \in \mathbb{N}_0^V$ Kapazitätsfunktion der Knoten,
$s,t \in V$ Quelle bzw. Senke (respektive). \\
\textbf{Idee}: Zerteile alle Knoten in zwei separate Knoten;
dem einen Knoten werden alle eingehenden Kanten zugeordnet,
dem anderen Knoten werden alle ausgehenden Kanten zugeordet
(jeweils mit deren gegebenen Kapazität).
Die beiden Knoten werden miteinander durch eine Kante verbunden,
die das Kantengewicht des ursprünglichen Knotengewichts trägt.
Daraus ergibt sich die Konstruktion der ''normales Netzwerk'' 
$N' = (V', E', c', s', t')$ (mit einer Kapazitätsfunktion) wie folgt: \\

$V' := \{ v_{in} | v \in V \} \cup \{v_{out} | v \in V \}$ \\
$E' := \{ (u,v_{in}) | u \in \delta^-(v)\} \cup \{ (v_{out},u) | u \in \delta^+(v) \} \cup \{(v_{in},v_{out}) | v \in V \}$ \\
$c' := e \mapsto \begin{cases}
			c_e & \text{, falls $e \in E$} \\
			\tilde{c}_v & \text{, falls $e = (v_{in},v_{out}) \notin E$}
		      \end{cases} $ \\
$s' := s_{in}$ \\
$t' := t_{out}$ \\

Die Einhaltung der zusätzlichen Knotenkapazität ist nun dadurch gewährleistet,
dass durch die '' künstliche Kante'' zwischen $v_{in}$ und $v_{out}$ erzwungen wird,
dass nie mehr als diese Kapazität durch $v_{in}$ und $v_{out}$, also auch nicht 
durch den ursprünglichen Knoten, fließen.
\end{enumerate}

%%%%%%%%%%%%%%%%%%%%%%%%%%%%
%% Aufgabe 2
%%%%%%%%%%%%%%%%%%%%%%%%%%%%
\subsection*{Aufgabe 2.}


%%%%%%%%%%%%%%%%%%%%%%%%%%%%
%% Aufgabe 3
%%%%%%%%%%%%%%%%%%%%%%%%%%%%
\subsection*{Aufgabe 3.}


%%%%%%%%%%%%%%%%%%%%%%%%%%%%
%% Aufgabe 4
%%%%%%%%%%%%%%%%%%%%%%%%%%%%
\subsection*{Aufgabe 4.}



\label{LastPage}
\end{document}
