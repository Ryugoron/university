\documentclass[11pt,a4paper,ngerman]{article}
\usepackage[bottom=2.5cm,top=2.5cm]{geometry} 
\usepackage{babel}
\usepackage[utf8]{inputenc} 
\usepackage[T1]{fontenc} 
\usepackage{ae} 
\usepackage{amssymb} 
\usepackage{amsmath}
\usepackage{amsthm} 
\usepackage{graphicx}
\usepackage{fancyhdr}
\usepackage{fancyref}
\usepackage{listings}
\usepackage{xcolor}
\usepackage{paralist}

\usepackage[pdftex, bookmarks=false, pdfstartview={FitH}, linkbordercolor=white]{hyperref}
\usepackage{fancyhdr}
\pagestyle{fancy}
\fancyhead[C]{Diskrete Mathematik}
\fancyhead[L]{Übung 1}
\fancyhead[R]{SoSe 2013}
\fancyfoot{}
\fancyfoot[L]{}
\fancyfoot[C]{\thepage \hspace{1px} of \pageref{LastPage}}
\renewcommand{\footrulewidth}{0.5pt}
\renewcommand{\headrulewidth}{0.5pt}
\setlength{\parindent}{0pt} 
\setlength{\headheight}{0pt}

\date{}
\title{Übung 1}
\author{Max Wisniewski, Alexander Steen}


%%
%% Enviroments for proofs and lemmas
%%
\newtheorem{lemma}{\bfseries Claim}

\begin{document}

\lstset{language=Pascal, basicstyle=\ttfamily\fontsize{10pt}{10pt}\selectfont\upshape, commentstyle=\rmfamily\slshape, keywordstyle=\rmfamily\bfseries, breaklines=true, frame=single, xleftmargin=3mm, xrightmargin=3mm, tabsize=2, mathescape=true}

\renewcommand{\figurename}{Figure}

\maketitle
\thispagestyle{fancy}

\subsection*{Aufgabe 1.}
Zeigen Sie, dass durch
$$
	u \sim v \, : \, \Leftrightarrow u \text{ und } v \text{ hängen zusammen}
$$
eine Äquivalenzrelation auf den Knoten eines Graphen definiert wird.\\

\textbf{Beweis:}\\
Sei  $ G(V,E) $ ein ungerichteter Graph.

Zwei Knoten $u,v$ hängen zusammen, wenn ein Weg $(a_i)_{1\leq i \leq n}$ existiert, mit $u = a_1$, $v = a_n$ und 
$\forall 1 \leq i < n \; : \; a_ia_{i+1} \in E$.\\

\begin{description}
	\item[\bfseries Reflexivität:]
		Sei $u \in V$, dann ist $(u)$  ein Weg in $G$, da es keine Kanten gibt die nicht in $E$ liegen und Start- und Endknoten
		$u$ sind. $\Rightarrow$ $u \sim v$.

	\item[\bfseries Symmetrisch:]
		Sei $u,v \in V$ mit $u \sim v$.\\

		Dann existiert ein Weg $(a_i)_{1 \leq i \leq n}$ nach Definition.
		Sei $(b_i)_{1 \leq i \leq n}$ ein Weg mit $b_j = (a_{n-j+1})$ für alle $1 \leq j \leq n$. 
		Da alle Knoten aus $V$ kommen, liegt dieser Weg auch in $G$.
		Es gilt $b_1 = a_{n-1+1} = a_n = v$ und $b_n = a_{n - n + 1} = a_1 = u$.

		Nun gilt für alle $1 \leq i < n$, dass
		$$
			b_ib_{i+1} = a_{n-i+1}a_{n-i} \in E
		$$
		da $G$ ungerichtet ist und nach Vorraussetzung $a_{n-i}a_{n-i+1}$ in $E$ liegt.
		Damit ist
		$$
			v \sim u
		$$
	\item[\bfseries Transitivität:]
		Seien $u,v,w \in V$ mit $u \sim v$ und $v \sim w$
		Sei $(a_i)_{1 \leq i \leq n}$ ein Weg von $u$ nach $v$ und
		$(b_i)_{1\leq i \leq n}$ ein Weg von $v$ nach $w$.

		Dann ist 
		$(c_i)_{1 \leq i \leq 2n}$ ein Weg mit
		$$
			c_i = \left\{ \begin{array}{lr} a_i &, i \leq n \\ b_{i-n} &, i > n \end{array}\right.
		$$
		
		Es gilt $c_0 = a_0 = u$ und $c_{2n} = b_n = w$.
		Desweiteren gilt für alle $1 \leq i \leq n$, dass $c_ic_{i+1} = a_ia_{i+1}\in E$ gilt
		und für alle $n+1 \leq i \leq 2n$, dass $c_{i-n}c_{i-n+1}=b_ib_{i+1} \in E$.\\

		$\Rightarrow u \sim w$.
\end{description}

Damit ist $\sim$ eine Äquivalenzrelation.\\
\mbox{}\hfill$\square$

\subsection*{Aufgabe 2.}

Zeigen Sie, dass in jedem Graphen die Anzahl der Knoten mit ungeradem Grad gerade ist.\\

\textbf{Beweis:}\\

Sei $G(V,E)$ ein ungerichter Graph.\\

Wir betrachten nun $g = \underset{v\ in V}{\sum} \text{grad}(v) = 2 |E|$.
Dies gilt, da wir für jede Kante $(u,v) \in E$ sie einmal für den Grad von $u$ und einmal
für den Grad von $v$ zählen.\\

Nun wissen wir, dass die summe von
\begin{enumerate}
	\item gerade und gerade ist gerade.
	\item gerade und ungrade ist ungerade.
	\item ungerade und ungerade ist gerade.
\end{enumerate}
Da die Summe aller Grade gerade ist, müssen es eine gerade Anzahl von Knoten mit geradem Grad sein.\\

\mbox{}\hfill$\square$

\subsection*{Aufgabe 3.}

Zeigen Sie: Eine Kante ist eine Brücke genau dann, wenn sie in keinem Kreis enthalten ist.\\

\textbf{Beweis:}\\
Sei $G(V,E)$ ein Graph und $(u,v) \in E$.

"$\Rightarrow$":\\
Sei $(u,v)$ eine Brücke und $G' = (V,E \setminus \{ (u,v) \})$ der Graph nach dem Entfernen von Kante $(u,v)$ aus $G$.
Da $(u,v)$ Brücke war, zerfällt die Komponente in der $(u,v)$ lag, in zwei Komponenten $K_1$ und $K_2$.
Insbesondere gibt in $G'$ keinen Weg von $K_1$ nach $K_2$ und umgekehrt. Damit kann $(u,v)$ in $G$ in
keinem Kreis enthalten sein, da sonst ein solcher Weg (o.B.d.A. von $K_1$ nach $K_2$) in $G'$ existiere.

"$\Leftarrow$":\\

\subsection*{Aufgabe 4.}

Zeigen Sie, dass ein Graph genau dann bipartit ist, wenn er keinen ungeraden Kreis enthält.\\

\textbf{Beweis:}\\

tbd

\label{LastPage}

\end{document}
