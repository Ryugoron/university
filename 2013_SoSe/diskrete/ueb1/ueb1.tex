\documentclass[11pt,a4paper,ngerman]{article}
\usepackage[bottom=2.5cm,top=2.5cm]{geometry} 
\usepackage{babel}
\usepackage[utf8]{inputenc} 
\usepackage[T1]{fontenc} 
\usepackage{ae} 
\usepackage{amssymb} 
\usepackage{amsmath}
\usepackage{amsthm} 
\usepackage{graphicx}
\usepackage{fancyhdr}
\usepackage{fancyref}
\usepackage{listings}
\usepackage{xcolor}
\usepackage{paralist}

\usepackage[pdftex, bookmarks=false, pdfstartview={FitH}, linkbordercolor=white]{hyperref}
\usepackage{fancyhdr}
\pagestyle{fancy}
\fancyhead[C]{Diskrete Mathematik}
\fancyhead[L]{Übung 1}
\fancyhead[R]{SoSe 2013}
\fancyfoot{}
\fancyfoot[L]{}
\fancyfoot[C]{\thepage \hspace{1px} of \pageref{LastPage}}
\renewcommand{\footrulewidth}{0.5pt}
\renewcommand{\headrulewidth}{0.5pt}
\setlength{\parindent}{0pt} 
\setlength{\headheight}{0pt}

\date{}
\title{Übung 1}
\author{Max Wisniewski, Alexander Steen}


%%
%% Enviroments for proofs and lemmas
%%
\newtheorem{lemma}{\bfseries Claim}

\begin{document}

\lstset{language=Pascal, basicstyle=\ttfamily\fontsize{10pt}{10pt}\selectfont\upshape, commentstyle=\rmfamily\slshape, keywordstyle=\rmfamily\bfseries, breaklines=true, frame=single, xleftmargin=3mm, xrightmargin=3mm, tabsize=2, mathescape=true}

\renewcommand{\figurename}{Figure}

\maketitle
\thispagestyle{fancy}

\subsection*{Aufgabe 1.}
Zeigen Sie, dass durch
$$
	u \sim v \, : \, \Leftrightarrow u \text{ und } v \text{ hängen zusammen}
$$
eine Äquivalenzrelation auf den Knoten eines Graphen definiert wird.\\

\textbf{Beweis:}\\
Sei  $ G(V,E) $ ein ungerichteter Graph.

Zwei Knoten $u,v$ hängen zusammen, wenn ein Weg $(a_i)_{1\leq i \leq n}$ existiert, mit $u = a_1$, $v = a_n$ und 
$\forall 1 \leq i < n \; : \; a_ia_{i+1} \in E$.\\

\begin{description}
	\item[\bfseries Reflexivität:]
		Sei $u \in V$, dann ist $(u)$  ein Weg in $G$, da es keine Kanten gibt die nicht in $E$ liegen und Start- und Endknoten
		$u$ sind. $\Rightarrow$ $u \sim v$.

	\item[\bfseries Symmetrisch:]
		Sei $u,v \in V$ mit $u \sim v$.\\

		Dann existiert ein Weg $(a_i)_{1 \leq i \leq n}$ nach Definition.
		Sei $(b_i)_{1 \leq i \leq n}$ ein Weg mit $b_j = (a_{n-j+1})$ für alle $1 \leq j \leq n$. 
		Da alle Knoten aus $V$ kommen, liegt dieser Weg auch in $G$.
		Es gilt $b_1 = a_{n-1+1} = a_n = v$ und $b_n = a_{n - n + 1} = a_1 = u$.

		Nun gilt für alle $1 \leq i < n$, dass
		$$
			b_ib_{i+1} = a_{n-i+1}a_{n-i} \in E
		$$
		da $G$ ungerichtet ist und nach Vorraussetzung $a_{n-i}a_{n-i+1}$ in $E$ liegt.
		Damit ist
		$$
			v \sim u
		$$
	\item[\bfseries Transitivität:]
		Seien $u,v,w \in V$ mit $u \sim v$ und $v \sim w$
		Sei $(a_i)_{1 \leq i \leq n}$ ein Weg von $u$ nach $v$ und
		$(b_i)_{1\leq i \leq n}$ ein Weg von $v$ nach $w$.

		Dann ist 
		$(c_i)_{1 \leq i \leq 2n}$ ein Weg mit
		$$
			c_i = \left\{ \begin{array}{lr} a_i &, i \leq n \\ b_{i-n} &, i > n \end{array}\right.
		$$
		
		Es gilt $c_0 = a_0 = u$ und $c_{2n} = b_n = w$.
		Desweiteren gilt für alle $1 \leq i \leq n$, dass $c_ic_{i+1} = a_ia_{i+1}\in E$ gilt
		und für alle $n+1 \leq i \leq 2n$, dass $c_{i-n}c_{i-n+1}=b_ib_{i+1} \in E$.\\

		$\Rightarrow u \sim w$.
\end{description}

Damit ist $\sim$ eine Äquivalenzrelation.\\
\mbox{}\hfill$\square$

\subsection*{Aufgabe 2.}

Zeigen Sie, dass in jedem Graphen die Anzahl der Knoten mit ungeradem Grad gerade ist.\\

\textbf{Beweis:}\\

Sei $G(V,E)$ ein ungerichter Graph.\\

Wir betrachten nun $g = \underset{v\ in V}{\sum} \text{grad}(v) = 2 |E|$.
Dies gilt, da wir für jede Kante $(u,v) \in E$ sie einmal für den Grad von $u$ und einmal
für den Grad von $v$ zählen.\\

Nun wissen wir, dass die summe von
\begin{enumerate}
	\item gerade und gerade ist gerade.
	\item gerade und ungrade ist ungerade.
	\item ungerade und ungerade ist gerade.
\end{enumerate}
Da die Summe aller Grade gerade ist, müssen es eine gerade Anzahl von Knoten mit geradem Grad sein.\\

\mbox{}\hfill$\square$

\subsection*{Aufgabe 3.}

Zeigen Sie: Eine Kante ist eine Brücke genau dann, wenn sie in keinem Kreis enthalten ist.\\

\textbf{Beweis:}\\
Sei $G(V,E)$ ein Graph und $(u,v) \in E$.

"$\Rightarrow$":\\
Sei $(u,v)$ eine Brücke und $G' = (V,E \setminus \{ (u,v) \})$ der Graph nach dem Entfernen von Kante $(u,v)$ aus $G$.
Da $(u,v)$ Brücke war, zerfällt die Komponente in der $(u,v)$ lag, in zwei Komponenten $K_1$ und $K_2$.
Insbesondere gibt in $G'$ keinen Weg von $K_1$ nach $K_2$ und umgekehrt. Damit kann $(u,v)$ in $G$ in
keinem Kreis enthalten sein, da sonst ein solcher Weg von $u$ nach $v$ existieren würde
und so $K_1$ mit $K_2$ verbinden würde, deshalb könnte $(u,v)$ keine Brücke sein.

"$\Leftarrow$":\\
Sei $(u,v)$ eine Kante die auf einem Kreis liegt.
Entfernen wir $(u,v)$ aus dem $G$ so existiert noch der Rest des Kreises in $G$, der nun einen Weg von
$u$ nach $v$ bildet. Somit kann $(u,v)$ keine Brücke sein.\\
\mbox{}\hfill$\square$

\subsection*{Aufgabe 4.}

Zeigen Sie, dass ein Graph genau dann bipartit ist, wenn er keinen ungeraden Kreis enthält.\\

\textbf{Beweis:}\\

Ein Graph $G(V,E)$ ist nun bipartit, wenn man eine Partition der Knoten findet also $V = V_0 \cup V_1$ und $V_0 \cap V_1 = \emptyset$,
so dass $\forall v_1 \in V_i v_2 \in V_i : (v_1,v_2) \not \int E$ für $i=0,1$ gilt.

Sei $G(V,E)$ nun ein beliebiger ungerichteter Graph.\\

$\Rightarrow$:\\

Sei $k = a_1a_2..a_n$ ein Kreis in $G$. Da $k$ ein Kreis ist, gilt $a_1=a_n$.
Sei o.B.d.A. $a_1 \in V_0$, dann wissen wir, da $G$ bipartit ist, dass $\forall 1 \leq i < n \, : \, a_i \in V_x \Rightarrow a_{i+1} \in V_{1-x}$ gilt.

Nach iterativer Anwendung ist für ungerade Indizes der Knoten in $V_0$ und für gerade Indizes der Knoten in $V_1$. Da $a_n = a_1 \in V_0$ muss auch $n$ ungerade sein.
Daher ist der Kreis gerade.

$\Leftarrow$:\\

Wir wissen, dass alle Kreise gerade sind.
Wir betrachten Graphen die o.B.d.A. nur aus Kreisen bestehen.
Wir färben die Kreise nun induktiv.

\begin{description}
   \item[\bfseries I.A.] $n = 1$.\\
      Sie $k = a_1 ... a_n$ ein Kreis.
      Wir stecken für alle ungeraden $i$ $a_i$ in $V_1$ und
      für alle geraden $i$ $a_i$ in $V_0$. Alle Kanten des Kreises
      laufen nun nur von $V_0$ nach $V_1$ und nie innerhalb.
      Wir nennen den Graphen, der nur aus $k$ besteht $G^1$
   \item[\bfseries I.V.] Der Graph $G^n$ bestehend aus den ersten
      $n$ Kreisen ist bipartit.
   \item[\bfseries I.S.] $n \leadsto n+1$.\\
     Sei $k = a_1 ... a_m$ der $n$-te gerade Kreis.
     Teilt der Graph keinen Knoten mit $G^n$ so teilen wir ihn, wie im Anker
     auf.\\
     Trifft er jede Zusammenhangskomponente höchstens einmal, so
     vereinigen wir die induzierten Partitionen der Kompenenten konsistent
     indem wir den Knoten, der auf dem Kreis liegt als ursprung komponente
     annehmen. (Entweder belassen wir die Mengen oder wir tauschen $V_0$ und
     $V_1$). Die Knoten des Kreises werden, wie im Anker aufgenommen.\\

     Trifft der Kreis eine Komponente mehrfach, ist die Aufteilung trotzdem
     Konsistent.
     Seien $u, v \in V$ Knoten auf dem Kreis, die die selbe
     Zusammenhangskomponente treffen. Dann schließt sich nun
     durch die Kompentente und ein Teil des Kreises $k$ ein neuer Kreis,
     der aufgrund der Vorraussetzung gerade ist.
     Starten wir mit der selben Aufteiliung für $u$ so erreichen wir auf
     $k$ und dem neuen Kreis die selbe Einteilung für $v$.\\

     Durch Aufnahme des Kreises $k$ erhalten wir $G^{n+1}$.
\end{description}

In der Induktion konnten wir Kreise schon aufgenommen haben, bevor wir sie betrachten. In diesem Fall wird sich nach dem Schema allerdings nichts verändern.

\mbox{}\hfill$\square$

\label{LastPage}

\end{document}
