\documentclass[11pt,a4paper,ngerman]{article}
\usepackage[bottom=2.5cm,top=2.5cm]{geometry} 
\usepackage{babel}
\usepackage{algpseudocode}
\usepackage{algorithm}

\usepackage[utf8]{inputenc} 
\usepackage[T1]{fontenc} 
\usepackage{ae} 
\usepackage{amssymb} 
\usepackage{amsmath}
\usepackage{amsthm} 
\usepackage{graphicx}
\usepackage{fancyhdr}
\usepackage{fancyref}
\usepackage{listings}
\usepackage{xcolor}
\usepackage{paralist}
\usepackage{tabularx}
\usepackage{tikz}
\usetikzlibrary{arrows,automata}
\usepackage[pdftex, bookmarks=false, pdfstartview={FitH}, linkbordercolor=white]{hyperref}
\usepackage{fancyhdr}
\pagestyle{fancy}
\fancyhead[C]{Diskrete Mathematik}
\fancyhead[L]{Übung 9}
\fancyhead[R]{SoSe 2013}
\fancyfoot{}
\fancyfoot[L]{}
\fancyfoot[C]{\thepage \hspace{1px} of \pageref{LastPage}}
\renewcommand{\footrulewidth}{0.5pt}
\renewcommand{\headrulewidth}{0.5pt}
\setlength{\parindent}{0pt} 
\setlength{\headheight}{0pt}
\newcommand\mdoubleplus{\ensuremath{\mathbin{+\mkern-10mu+}}}
\date{}
\title{Übung 9}
\author{Max Wisniewski, Alexander Steen}
\renewcommand{\algorithmicrequire}{\textbf{Input:}}
\renewcommand{\algorithmicensure}{\textbf{Output:}}
\newcommand{\argmin}{\text{argmin}}
\newcommand{\argmax}{\text{argmax}}
\newcommand{\code}[1]{\langle {#1} \rangle}
\newcommand{\Z}{\mathbb{Z}}
%%
%% Enviroments for proofs and lemmas
%%
\newtheorem{lemma}{\bfseries Claim}

\begin{document}

\lstset{language=Pascal, basicstyle=\ttfamily\fontsize{10pt}{10pt}\selectfont\upshape, commentstyle=\rmfamily\slshape, keywordstyle=\rmfamily\bfseries, breaklines=true, frame=single, xleftmargin=3mm, xrightmargin=3mm, tabsize=2, mathescape=true}

\renewcommand{\figurename}{Figure}

\maketitle
\thispagestyle{fancy}

%%%%%%%%%%%%%%%%%%%%%%%%%%%%
%% Aufgabe 1
%%%%%%%%%%%%%%%%%%%%%%%%%%%%
\subsection*{Aufgabe 1.}
Es gibt $\sum_{i=1}^n \binom{n}{i} 2^i$ viele Möglichkeiten.

%%%%%%%%%%%%%%%%%%%%%%%%%%%%
%% Aufgabe 2
%%%%%%%%%%%%%%%%%%%%%%%%%%%%
\subsection*{Aufgabe 2.}
Wie viele Möglichkeiten gibt es, sieben Elfen und fünf Kobolde in einer Reihe aufzustellen,
sodass nie zwei Kobolde direkt hintereinander stehen?
\begin{enumerate}[a)]
\item Elfen bzw. Kobolde nicht unterscheidbar \\
    Stehen die sieben Elfen hintereinander, so gibt es acht Positionen, an denen man die Koblde
    platzieren könnte. Also gibt es $\binom{8}{5} = 56$ Möglichkeiten.
\item Elfen bzw. Kobolde unterscheidbar \\
    Für jede Aufstellung können wir uns jede Permutation der Elfen bzw. Kobolde betrachten. Für eine
    feste Aufstellung haben wir $7! \cdot 5!$ Permutationen, also insgesamt
    $\binom{8}{5} \cdot 7! \cdot 5! = 33 868 800$.
\end{enumerate}

%%%%%%%%%%%%%%%%%%%%%%%%%%%%
%% Aufgabe 3
%%%%%%%%%%%%%%%%%%%%%%%%%%%%
\subsection*{Aufgabe 3.}
Zu zeigen: Für alle $r,n \in \mathbb{N}_0: \sum_{k=0}^n \binom{r+k}{k} = \binom{n+r+1}{n}$

\textbf{Beweis} durch Induktion: \\
Sei $r \in \mathbb{N}_0$.
Basis: $n = 0$, dann ist
    \begin{equation*}
      \sum_{k=0}^0 \binom{r+k}{k} = \binom{r}{0} = 1 = \binom{r+1}{0}
    \end{equation*}
I.V.: Behauptung gilt für ein $n$. \\
Schritt: $n \rightsquigarrow n+1$
    \begin{equation*}\begin{split}
      \sum_{k=0}^{n+1} \binom{r+k}{k} &= \sum_{k=0}^{n} \binom{r+k}{k} + \binom{r+n+1}{n+1} \\
      &\stackrel{(IV)}{=}\binom{n+r+1}{n} + \binom{n+r+1}{n+1}  \\
      &\stackrel{(Def.)}{=}\binom{n+r+2}{n+1} = \binom{(n+1)+r+1}{n+1}
    \end{split}\end{equation*}
\mbox{} \hfill $\square$
%%%%%%%%%%%%%%%%%%%%%%%%%%%%
%% Aufgabe 4
%%%%%%%%%%%%%%%%%%%%%%%%%%%%
\subsection*{Aufgabe 4.}
Sei $n \in \mathbb{N}, \tilde{n} \in \mathbb{N}_0$.
\begin{enumerate}[a)]
\item $s_{n,1} = (n-1)!$ \\
I.A. $n=1$; $s_{1,1} = 1 = 0! = (1-1)!$ \\
I.S. $n \rightsquigarrow n+1$;
  \begin{equation*}\begin{split}
    s_{n+1,1} &= s_{n,0} + n \cdot s_{n,1} 
              \stackrel{(IV)}{=} 0 + n \cdot (n-1)! 
              = n! = ((n+1)-1)!
  \end{split}\end{equation*}
\item $s_{n,n-1} = \binom{n}{2}$ \\
I.A. $s_{1,0} = 0 = \binom{1}{2}$ \\
I.S. $n \rightsquigarrow n+1$;
  \begin{equation*}s_{n+1,n} = s_{n,n-1} + n s_{n,n} \stackrel{(IV)}{=} \binom{n}{2} + n = \binom{n+1}{2}\end{equation*}
\item $s_{\tilde{n},\tilde{n}} = 1$ \\
Gilt nach Definition.
\item $s_{n,2} = (n-1)! H_{n-1}$ \\
I.A. $n=1$; $s_{1,2} = 0 = 1 \cdot 0 = 0! \cdot H_{0} $ \\
I.S. $n \rightsquigarrow n+1$;
  \begin{equation*}\begin{split}
    s_{n+1,2} &= s_{n,1} + n s_{n,2} \stackrel{(IV)}{=} (n-1)! + n (n-1)! H_{n-1} \\
              &=  (n-1)! + n! H_{n-1} ...
  \end{split}\end{equation*}
\item $S_{n,1} = S_{n,n} = 1$ \\
$S_{n,n} = 1$ gilt nach Definition. \\
I.A. $n=1$; $S_{1,1} \stackrel{Def.}{=} 1 $ \\
I.S. $n \rightsquigarrow n+1$;
  \begin{equation*}\begin{split}
    S_{n+1,1} &= S_{n,0} + S_{n,1} \stackrel{(IV)}{=}0 + 1 = 1\end{split}\end{equation*}
\item $S_{n,n-1} = \binom{n}{2}$ \\
I.A. $n=1$; $S_{1,0} \stackrel{Def.}{=} 0 = \binom{1}{2} $ \\
I.S. $n \rightsquigarrow n+1$;
  \begin{equation*}\begin{split}
    S_{n+1,n} &= S_{n,n-1} + n S_{n,n} \stackrel{(IV)}{=} \binom{n}{2} + n = \binom{n+1}{2}
  \end{split}\end{equation*}
\item $S_{n,2} = 2^{n-1} -1 $
I.A. $n=1$; $S_{1,2} \stackrel{Def.}{=} 0 = 2^{1-1} - 1 $ \\
I.S. $n \rightsquigarrow n+1$;
  \begin{equation*}\begin{split}
    S_{n+1,2} &= S_{n,1} + 2 S_{n,2} \stackrel{(e)+(IV)}{=} 1 + 2\cdot (2^{n-1} -1) \\
              &= 1+ 2^n - 2 = 2^n - 1 = 2^{(n+1)-1} - 1
  \end{split}\end{equation*}

\end{enumerate}

\label{LastPage}
\end{document}
