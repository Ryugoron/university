\documentclass[11pt,a4paper,ngerman]{article}
\usepackage[bottom=2.5cm,top=2.5cm]{geometry} 
\usepackage{babel}
\usepackage{algpseudocode}
\usepackage{algorithm}

\usepackage[utf8]{inputenc} 
\usepackage[T1]{fontenc} 
\usepackage{ae} 
\usepackage{amssymb} 
\usepackage{amsmath}
\usepackage{amsthm} 
\usepackage{graphicx}
\usepackage{fancyhdr}
\usepackage{fancyref}
\usepackage{listings}
\usepackage{xcolor}
\usepackage{paralist}
\usepackage{tabularx}
\usepackage{tikz}
\usetikzlibrary{arrows,automata}
\usepackage[pdftex, bookmarks=false, pdfstartview={FitH}, linkbordercolor=white]{hyperref}
\usepackage{fancyhdr}
\pagestyle{fancy}
\fancyhead[C]{Diskrete Mathematik}
\fancyhead[L]{Übung 10}
\fancyhead[R]{SoSe 2013}
\fancyfoot{}
\fancyfoot[L]{}
\fancyfoot[C]{\thepage \hspace{1px} of \pageref{LastPage}}
\renewcommand{\footrulewidth}{0.5pt}
\renewcommand{\headrulewidth}{0.5pt}
\setlength{\parindent}{0pt} 
\setlength{\headheight}{0pt}
\newcommand\mdoubleplus{\ensuremath{\mathbin{+\mkern-10mu+}}}
\date{Tutor : Stefan Schwarz}
\title{Übung 10}
\author{Max Wisniewski, Alexander Steen}
\renewcommand{\algorithmicrequire}{\textbf{Input:}}
\renewcommand{\algorithmicensure}{\textbf{Output:}}
\newcommand{\argmin}{\text{argmin}}
\newcommand{\argmax}{\text{argmax}}
\newcommand{\code}[1]{\langle {#1} \rangle}
\newcommand{\Z}{\mathbb{Z}}
%%
%% Enviroments for proofs and lemmas
%%
\newtheorem{lemma}{\bfseries Claim}

\begin{document}

\lstset{language=Pascal, basicstyle=\ttfamily\fontsize{10pt}{10pt}\selectfont\upshape, commentstyle=\rmfamily\slshape, keywordstyle=\rmfamily\bfseries, breaklines=true, frame=single, xleftmargin=3mm, xrightmargin=3mm, tabsize=2, mathescape=true}

\renewcommand{\figurename}{Figure}

\maketitle
\thispagestyle{fancy}

%%%%%%%%%%%%%%%%%%%%%%%%%%%%
%% Aufgabe 1
%%%%%%%%%%%%%%%%%%%%%%%%%%%%
\subsection*{Aufgabe 1.}
Die Herleitung ist analog zu der Herleitung in Aufgabe 5 (die wir zuerst gemacht haben), nur dass dieses Mal der jeweils ersten und zweiten Tänzer von $1 \ldots n$ durchnummeriert werden und der eine Partner $i$ zum anderen Partner $i$ gehört. Die Anzahl der Paare, bei denen niemand mit seinem eigentlich Tanzpartner tanzt, ist nun wieder genau die Anzahl der fixpunktfreien Permutation von $n$ Elementen $D(n)$. Nach Aufgabe 5 gilt
$D(n) = n! \left(\sum_{k=0}^n \frac{ (-1)^{k} }{k!} \right)$ (siehe Aufgabe 5 für die genaue Herleitung).
%%%%%%%%%%%%%%%%%%%%%%%%%%%%
%% Aufgabe 2
%%%%%%%%%%%%%%%%%%%%%%%%%%%%
\subsection*{Aufgabe 2.}

Die Anzahl der Aufteilungen der drei Personentypen kann (vorerst ohne Berücksichtigung der verbotenen Situtationen) durch $\binom{12}{3,4,5} = 27 720$ ermittelt werden.
Nun betrachten wir, wie viele Möglichkeiten es gibt, sodass (a) alle Amerikaner (b) alle Chinesen und (c) alle
Russen hintereinander stehen.\\
(a) Sind drei Amerikanerpositionen vergeben, ergeben sich $\binom{8}{5}$ Möglichkeiten für die restliche Aufteilung. Für die geschlossene Aufstellung der drei Amerikaner haben wir 9 Möglichkeiten (jeweils am Ende der Gruppe und sieben Positionen zwischen den acht restlichen Leuten).
Es ergibt sich $ 9 \cdot \binom{8}{5} = 504$. \\
(b) Analog ergibt sich eine restliche Aufteilung von $\binom{7}{3}$ Möglicheiten auf acht verschiedene Arten. Also $8 \cdot \binom{7}{3} = 280$. \\
(c) Analog ergibt sich eine restliche Aufteilung von $\binom{9}{5}$ Möglicheiten auf zehn verschiedene Arten. Also $10 \cdot \binom{9}{5} = 1260$. \\

Nun betrachten wir die Möglichkeiten, bei denen jeweils zwei Nationalitätsgruppen jeweils zusammenstehen:
Davon gibt es sechs Möglichkeiten, nämlich: \\
(1) Amerikaner und Russen: $\binom{6}{2} = 15 $ Möglichkeiten \\
(2) Amerikaner und Chinesen: $\binom{4}{2} = 6$ Möglichkeiten \\
(3) Chinesen und Russen: $\binom{5}{2} = 10$ Möglichkeiten \\

Nun betrachten wir die Möglichkeiten, bei denen alle drei Nationalitäten jeweils geschlossen zusammenstehen, das sind genau $3! = 6$ Möglichkeiten.

Also ergibt sich insgesamt eine Anzahl von $27 720 - 504 - 280 - 1260 +15 + 6+ 10 - 6 = 25701$.
%%%%%%%%%%%%%%%%%%%%%%%%%%%%
%% Aufgabe 3
%%%%%%%%%%%%%%%%%%%%%%%%%%%%
\subsection*{Aufgabe 3.}
Nicht bearbeitet.
%%%%%%%%%%%%%%%%%%%%%%%%%%%%
%% Aufgabe 4
%%%%%%%%%%%%%%%%%%%%%%%%%%%%
\subsection*{Aufgabe 4.}
\textbf{Beweis}: \\
Sei $\sigma \in S_{n}$ eine fixpunktfreie Permutation. Also ist insbesondere $\sigma(1) = i \neq 1$.
Dann gilt entweder: \\
Fall 1: $\sigma(i) = 1$ \\
Dann verbleiben für die verbliebenen $n-2$ Elemente $\{ 2,\ldots,n \} \setminus \{i \}$ genau $D_{n-2}$ Möglichkeiten, eine fixpunktfreie Permutation zu erzeugen. \\
Fall 2: $\sigma(i) \neq 1$ \\
Dann verbleiben für die restlichen $n-1$ Elemente genau $D_{n-1}$ Möglichkeiten, eine fixpunktfreie Permutation zu erzeugen.\\

Für die Wahl von $i$ haben für nun $n-1$ Möglichkeiten (nämlich genau $\{2,\ldots,n \}$). Daraus ergibt sich die Rekursion $D_n = (n-1) (D_{n-1} + D_{n-2})$.
%%%%%%%%%%%%%%%%%%%%%%%%%%%%
%% Aufgabe 5
%%%%%%%%%%%%%%%%%%%%%%%%%%%%
\subsection*{Aufgabe 5.}
Seien die Herren und deren Hüte mit $1 \ldots n$ durchnummeriert (wobei der Hut $i$ dem Herrn $i$ gehört).
Die Rückgabe der Gardrobiere entspricht der Wahl einer Permutation $\sigma \in S_n$, wobei
der Hut korrekt zurückgegeben wird, wenn $\sigma(i) = i, i = 1,\ldots,n$. Keiner der $n$ Herren bekommt
seinen Hut zurück, falls die Permutation $\sigma$ fixpunktfrei (also ein Derangement) ist. Sei $D(n)$ die
Anzahl der fixpunktfreien Permutation $\sigma \in S_n$.
Gesucht ist also die Wahrscheinlichkeit
$$p_n = Pr[\text{keiner der $n$ Herren bekommt seinen Hut zurück}] \stackrel{\text{Laplace}}{=} \frac{D(n)}{|S_n|} = \frac{D(n)}{n!}$$
Die Anzahl $D(n)$ können wir nun mit dem Prinzip der Inklusion und Exklusion bestimmen: \\
Wir zählen die Permutation, die mindestens einem Herren den korrekten Hut geben und ziehen diese Zahl von $n!$ ab. Betrachten wir die Menge $X_i = \{\sigma \in S_n | \sigma(i) = i \}$, so gilt $\bigcap_{i \in I \subset [n]} X_i = (n - |I|)!$ und damit

$$ |\bigcup_{i = 1}^n X_i | = \sum_{k=1}^n (-1)^{k-1} \binom{n}{k} (n-k)! =  \sum_{k=1}^n (-1)^{k-1} \frac{n!}{k!}$$

Daraus ergibt sich 
$$D(n) = n! - \sum_{k=1}^n (-1)^{k-1} \frac{n!}{k!} = n! \left(\sum_{k=0}^n \frac{ (-1)^{k} }{k!} \right)$$
und damit für $p_n$
$$ p_n = \frac{D(n)}{n!} = \sum_{k=0}^n \frac{ (-1)^{k} }{k!} \stackrel{n \to \infty}{\longrightarrow} \frac{1}{e}$$
\label{LastPage}
\end{document}
