\documentclass[11pt,a4paper,ngerman]{article}
\usepackage[bottom=2.5cm,top=2.5cm]{geometry} 
\usepackage{babel}
\usepackage[utf8]{inputenc} 
\usepackage[T1]{fontenc} 
\usepackage{ae} 
\usepackage{amssymb} 
\usepackage{amsmath}
\usepackage{amsthm} 
\usepackage{graphicx}
\usepackage{fancyhdr}
\usepackage{fancyref}
\usepackage{listings}
\usepackage{xcolor}
\usepackage{paralist}

\usepackage[pdftex, bookmarks=false, pdfstartview={FitH}, linkbordercolor=white]{hyperref}
\usepackage{fancyhdr}
\pagestyle{fancy}
\fancyhead[C]{Computational Geometry}
\fancyhead[L]{Exercise sheet 8}
\fancyhead[R]{SoSe 2013}
\fancyfoot{}
\fancyfoot[L]{}
\fancyfoot[C]{\thepage \hspace{1px} of \pageref{LastPage}}
\renewcommand{\footrulewidth}{0.5pt}
\renewcommand{\headrulewidth}{0.5pt}
\setlength{\parindent}{0pt} 
\setlength{\headheight}{0pt}

\date{}
\title{Exercise sheet 8}
\author{Max Wisniewski, Alexander Steen}


%%
%% Enviroments for proofs and lemmas
%%
\newtheorem{lemma}{\bfseries Claim}

\begin{document}

\lstset{language=Pascal, basicstyle=\ttfamily\fontsize{10pt}{10pt}\selectfont\upshape, commentstyle=\rmfamily\slshape, keywordstyle=\rmfamily\bfseries, breaklines=true, frame=single, xleftmargin=3mm, xrightmargin=3mm, tabsize=2, mathescape=true}

\renewcommand{\figurename}{Figure}

\maketitle
\thispagestyle{fancy}

%%%%%%%%%%%%%%%%%%%%%%%%
%% Aufgabe 1 
%%%%%%%%%%%%%%%%%%%%%%%%
\subsection*{Problem 1}

Let $P = \{ x_1, ..., x_n\}$ be $n$ numbers, given no particular order.\\

Consider the algortihm that picks a random permutation of $P$ and then inserts the elements of $P$ in this order into
an (unbalanced) binary search tree $T$.

\subsubsection*{(a)}

Explain how this algorithm can be interpreted as a one-dimensional version of the algorithm in class.\\

\textbf{Solution:}\\

The search structure from the lecture dicedes by points $p$ from the planar subdevision whether the searched point
lies right or left of this point and then descendes into this subtree. Given a line of the subdevision it decides
wether it the searched point lies above or below.\\

In the one dimensional case we can only do a choice in one dimension, therefore we have either of the two cases
mentioned above. In the two-dimensional case we choose a random permutation of lines, which are in one dimension points,
and insert them into the structure, by finding its position in the search tree and update by inserting new choiced
for the endpoints and the line two the new faces. If only given a point we only insert it at the right place.\\

The result partitions the previous line segment \emph{next($p$)} \emph{prev($p$)}. In two dimensions the line
can partition a face into at most four new faces.\\

Therefore the interpretation of the searchtree and the search is essantially the same as the algorithm in the lecture.

\subsubsection*{(b)}

Show that the expected running time for the construction of $T$ is $O(n \log n)$.\\

\textbf{Soluton:}\\

If the first element of the permuation is inserted into the search tree all following elements will be partitioned by
this choice. If $x \in P$ is inserted we insert all $P_<(x) = \{ p \in P \, | \, p < x \}$ into the left and
all $P_>(x) = \{ p \in P \, | \, p > x \}$ into the right tree. Because of the itertive construction of a binary search tree,
there procede just the same. Hence we are interested in the expected size of $P_<$ and $P_>$.\\

If the permutation is choosen random at equal, for an element of $P$ to be first we have a probability of $\frac{1}{n}$.
Because $P$ is a set each number occurs at most once. If we sort $P$ we know, that $P_<(i)$ for te $i$th element
has size $i-1$ and $P_>(i) = n-i$.
\begin{equation*}\begin{split}
    E[|P_<(p)|] &= \overset{n}{\underset{i=1}{\sum}} Pr(p = i) \cdot |P_<(i)|\\
                &= \frac{1}{n} \cdot \overset{n}{\underset{i=1}{\sum}} i-1\\
                &= \frac{1}{n} \cdot \overset{n-1}{\underset{i=0}{\sum}} i\\
                &= \frac{1}{n} \cdot \frac{(n-1)(n-2)}{2}\\
                &< \frac{n}{2}\\
    E[|P_>(p)|] &= \overset{n}{\underset{i=1}{\sum}} Pr(p=i) \cdot |P_>(i)|\\
                &= \frac{1}{n} \cdot \overset{n}{\underset{i=1}{\sum}} n-i\\
                &< \frac{n}{2}
\end{split}\end{equation*}

We see that we half the number of elements at each level of the iteration.
Therefor the height of the tree is $O(\log n)$. Because of that each insertion operation takes
at most $O(\log n)$ time. Hence all insertions take $O(n \log n )$ time.

\subsubsection*{(c)}

Let $z$ be a fixed number. Give a bound on the expected time it takes to search for $z$ in $T$. Here, the expectation is
over the random permutation used to construct $T$.\\

\textbf{Solution:}\\

As shown in \emph{(b)} the height of the tree is expected $O(\log n)$. Hence a search takes at most $O(\log n)$ steps.
In each step we make a constant number of operations.

\subsubsection*{(d)}

Explain how this algorithm resembles randomized quicksort.\\

\textbf{Solution:}\\

In quicksort we partition the list of elements by a pivot element which is in out case given by the first element
of the random permutation. Our algorithm partions it by the first element, because it is the root of the search tree.\\

The tree will be computed by quicksort implicitly as the recursion tree. The left iteration is the left side of the tree
and the right as that.\\

The constructed search tree will produce a sorted list if it is flattened again.
By \emph{fold fusion law} the known quicksort is exaclty the same as first creating the data structure and then flatten it
again.


\subsection*{Problem 2}

Give an example of $n$ line segments with an order on them that makes the algorithm conustruct a search structure of size $\Theta (n^2)$ and worst-case
query time $\Theta (n)$.\\

\textbf{Solution:}\\

tbd

\subsection*{Problem 3}

Let $P$ be a polygon given an array of its $n$ vertices in sorted order along the boundary. Give an algorithm, that given a query point $q$, decides
whether $q$ lies inside of $P$ in $O(\log n)$ time for the following cases.

\subsubsecetion*{(a)}

$P$ is convex.\\

\textbf{Solution:}\\

Choose a point $p \in P$ and look search for the intersection $\overline{qp}$ with $P$. Because $P$ is convex there can be at most two.
If we found the second intersection (it can be on an edge or a vertex of $P$) we can check, if $q$ lies between the two points.
If both points lie on one side of $q$, then $q$ lies outside. This holds if $q$ lies on the proper interior. If $q$ lies on the boundary
we can not find a second point other then $q$.\\

If we can find the second point in $O(\log n)$ time we can check the property in $O(\log n)$ time.\\

We again search a point where the line $\overline{pq}$ is right (or on) but the next point in $P$ is on the left. As we have already seen
this binary search in a circular structure, we know that we can find this point in $O(\log n)$ time.

\subsubsection*{(b)}

$P$ is y-monoton.\\

\textbf{Solution:}\\

We search the two line segments enclosing the point $q$ in y-coordinates. There exists at most two because it is y-monoton.

We can find both the line segements $s,t$ by binary search. Let $x,y$ be the upper most and lower most point of the polygon.
Then the part between $x,y$ and $y,x$ are both sorted lists, where we can find via binary search the segment, where
the moment point is below and the next is above $q$ and vis-versa.

Given $s$ and $t$ we check if $q$ lies between this both points.\\

This binary search takes $O(\log n)$ time and the check takes $O(1)$ such that this algorithm takes $O(\log n)$ time.

\subsubsection*{(c)}

$P$ is star-shaped.\\

\textbf{Solution:}\\

We reduce this problem to a y-monoton case. From $p$ we shoot rays to two abritrary disjoint points. For one we do a
binary search for the vertex, where the line from $p$ to this vertex has $q$ to the left, but the next vertex on $P$ has
$q$ to the right. And the other line accordingly wit h$q$ to the right.\\

Because $P$ is star shaped both lines are completly in $P$. Taken the polygon defined by the both lines and the line segment
between both endpoints is y-monoton. If it were not the case there would be an ear pointing into the new polygon $P'$ (no
other case is possible because it is star shaped and $p$ can see every point). If we assume general position $q$ can not lie
an the line from $p$ to the vertex of the ear. An is therefor left or right. But then we could have moved one of the two rays
closer to $q$.\\

The search takes $O(\log n)$ time for both rays. As shown in \emph{(b)} the search in the y-monoton subsegment takes $O(\log n)$ time
such that this algorithm takes $O(\log n)$ time.

\label{LastPage}


\end{document}
