\documentclass[11pt,a4paper,ngerman]{article}
\usepackage[bottom=2.5cm,top=2.5cm]{geometry} 
\usepackage{babel}
\usepackage[utf8]{inputenc} 
\usepackage[T1]{fontenc} 
\usepackage{ae} 
\usepackage{amssymb} 
\usepackage{amsmath}
\usepackage{amsthm} 
\usepackage{graphicx}
\usepackage{fancyhdr}
\usepackage{fancyref}
\usepackage{listings}
\usepackage{xcolor}
\usepackage{paralist}

\usepackage[pdftex, bookmarks=false, pdfstartview={FitH}, linkbordercolor=white]{hyperref}
\usepackage{fancyhdr}
\pagestyle{fancy}
\fancyhead[C]{Computational Geometry}
\fancyhead[L]{Exercise sheet 6}
\fancyhead[R]{SoSe 2013}
\fancyfoot{}
\fancyfoot[L]{}
\fancyfoot[C]{\thepage \hspace{1px} of \pageref{LastPage}}
\renewcommand{\footrulewidth}{0.5pt}
\renewcommand{\headrulewidth}{0.5pt}
\setlength{\parindent}{0pt} 
\setlength{\headheight}{0pt}

\date{}
\title{Exercise sheet 6}
\author{Max Wisniewski, Alexander Steen}


%%
%% Enviroments for proofs and lemmas
%%
\newtheorem{lemma}{\bfseries Claim}

\begin{document}

\lstset{language=Pascal, basicstyle=\ttfamily\fontsize{10pt}{10pt}\selectfont\upshape, commentstyle=\rmfamily\slshape, keywordstyle=\rmfamily\bfseries, breaklines=true, frame=single, xleftmargin=3mm, xrightmargin=3mm, tabsize=2, mathescape=true}

\renewcommand{\figurename}{Figure}

\maketitle
\thispagestyle{fancy}

%%%%%%%%%%%%%%%%%%%%%%%%
%% Aufgabe 1 
%%%%%%%%%%%%%%%%%%%%%%%%
\subsection*{Problem 1}

Let $P$ be a simple planar polygon with $n$ vertices. Let $x$ be a point inside $P$. The \emph{visible polygon} $\text{vis}_P(x)$ of $x$ is the set of all points $q \in P$ that are \emph{visible} from $x$, i.e., the line segment $xq$ does not intersect the outside of $P$.

\subsubsection*{(a)}
Draw an interesting example of a simple polygon $P$ and visibility polygons
for two points in $P$.\\

\textbf{Solution:}\\

\vspace{15cm}

\subsubsection*{(b)}
Describe the structure of $\text{vis}_P(x)$. What are its vertices and edges?\\

\textbf{Solution:}\\

The \emph{visible polygon} is a simple planar polygon if no vertices of $P$ lie on a line with $x$. $\text{vis}_P$ is enclosed by $P$ and its vertices are
either vertices of $P$ or a point on an edge of $P$. For each vertex $v$ of $\text{vis}_P$ on an edge of $P$ there exists a vertex $q$ in $P$ such that $v$ lies
on the line $xq$.

\subsubsection*{(c)}
Suppose that we have a triangulation $T$ of $P$ available. Given a point $x$ inside $P$, show how to find $\text{vis}_P$ in $O(n)$ time.\\

\textbf{Algorithm:}\\

We assume, that $x$ is in the proper interior of $P$ and that no two vertices
of $P$ on a line $xp$ for $p \in P$. And no three vertices of $P$ lie on a line.

\begin{enumerate}[1.]
    \item Find the triangle $T$ that contains $x$.
    \item Fix one vertex $p_0$ of the triangle $T$.
    \item loop until $p_{i+1} = p_0$
        \begin{enumerate}[a.]
                \item (Descending to new bottom)\\
                    If edge right to $p_i$ is a diagonal of $P$.
                \begin{enumerate}[(i)]
                    \item for the current triangle $T$ test if
                        two points are on the right of $\overline{xq}$
                        then take the left subtree. Else the right one.
                    \item If this subtree does not exist, mark the intersection
                        of $\overline{xq}$ with the edge of the triangle as
                        $p_{i+1}$
                \end{enumerate}
                \item (Moving right / up)\\
                    Else let $q$ be the next point right of the line 
                    $\overline{xp_i}$ in the triangle containing $p_i$.
                    \begin{enumerate}[(i)]
                        \item If $q = p_i$ we move one triangle back in the
                            traversal of the triangulation. (We keep staying at
                            $p_i$ which is also a vertex of the next upper
                            triangle.)
                            We take the next vertex $z$ in cw order in
                            the upper triangle.
                            We remove $p_i$ (and decrement $i$ as long
                            as $p_i$ is right to $\overline{xz}$.\\
                            
                            If the removed point $p'$ was connected by an edge
                            in $P$ with $p_i$ we compute the intersection of
                            $\overline{xz}$ with this edge. We add the intersection
                            point to the list after $p_i$ and add $z$ as the new $p_{i+1}$.
                        \item Else we set $p_i = q$.
                    \end{enumerate} 
        \end{enumerate}
\end{enumerate}

\textbf{Correctness:} (Sketch)\\

We traverse the triangulation only, if we have a line of side with $x$
therefor each point in the new polygon should be visible. On
the way up, we remove every vertex in $p_i$ that is right
to the new line. Because we always
descent if it is possible we should no skip a point.\\

The algorithm works by shooting a ray from $x$ onto some starting direction in $P$. Then it iterates by moving the ray cw around searching for a next point of interest (lighthouse). Until it reaches the starting point.\\

The deleting of the points already reached is important, because a corner
in the polygon above might block the vision to an already worked part.\\
This can only happen on the way up, because we keep track of the line of side
on the way down.\\

The starting point $p_0$ is in $\text{vis}_P$ because it is visible (the
triangle containing $x$ is convex) and every point behind $p_0$ is not visible.
Furthermore $p_0$ is a vertex of $\text{vis}_P$ because it cannot lie on a
line in $P$ due to the general position assumed. Such that the edge in $P$
to $p_0$ also are contained at least partially (connected to $p_0$) to $\text{vis}_P$.\\

\textbf{Runtime:}\\

To find the triangle takes $O(n)$ time.\\
We consider each triangle (except the start triangle) three times.
On on moving down where we make a constant number of steps to check
if we can still descent.

And on the way up. On the way up we have to delete all non visible points.
But each added point can only be added/removes once through the algorithm.
We can have at most one point per edge of $P$ and at most all points in $P$.
Thefore in total this step takes linear time.\\

The initial triangle can have three children, but it remains the same analysis
as for the rest.\\

Because we iterate at most over all triangles and do in total a linear
number of deletions on $p_i$. The algorithm
runs in $O(n)$ if the triangulation is given.

\subsection*{Problem 2}

Give an example that the deterministic incremental algorithm for two dimensional LP may take $\Omega(n^2)$ time.\\

\textbf{Solution:}\\

\vspace{15cm}

The example was given in the book \emph{Computational Complexity} by \emph{de Berg, et al.}. In each iteration we have to compute the next optimal solution $v_i$ as the intersection of the halfplanes in $H_i$. This takes $O(i)$ time when formulated as a linear program.
The lines are sorted in a way, that $v_{i-1} \not\in H_i$ as can be seen
in the picture, but the new $v_i$ always moves up. Therefor it takes
\begin{equation}\begin{split}
    T(n) &= \overset{n}{\underset{i=0}{\sum}} O(i)\\
        &= O(n^2)
\end{split}\end{equation}
time to run the deterministic algorithm on this example.

\subsection*{Problem 3}
Let $P$ be a convex poylgon in the plane. We would like to compute a disk
of maximum size that is encolsed in $P$. Formulate this problem as a linear
program.\\

\textbf{Solution:}\\

The size of the disk grows monoton with its radius $r$. Hence we do not have
to check the size but can rather check the radius of the disk.
Let $p_0, p_1, ..., p_n$ be the vertices in ccw order of the convex polygon.
The for each triangle. $(p_i, p_{i+1}, (x,y))$ the radius has to be smaller
then the normal on $\overline{p_ip_{i+1}}$ and $(x,y)$ To check this
we calculate the point in the middle of the line and simply check
if the radius is smaller then the line to each of these points. Furthermore
$(x,y)$ has to be on the left of this line. Let $p_{n+1}=p_0$ then we
formulate both these conditions as linear constraints.
Für jede der Kanten $\overline{p_ip_{i+1}}$ we can compute the
normal vector $\overline{n_i}$ in advance and use it in the subjective function.

\begin{equation}\label{alge:ueb6:disklp}\begin{array}{lrclr}
    \text{maximize} & 1 \cdot r + 0 \cdot x + 0 \cdot y\\
    \text{subject to} & (x-p_i^x) \cdot n_i^x + (y-p_i^y) \cdot n_i^y & \leq & r & \forall 0 \leq i \leq n\\
                & (x - p_i^x) \cdot n_i^x + (y-p_i^y) \cdot n_i^y & \geq & 0 & \forall 0 \leq i \leq n
\end{array}\end{equation}

The first subject assures that the center of the circel has a distance of at most $r$ to the line between $\overline{p_ip_{i+1}}$.
The second one assures, that the center of the disk is to the left of the line.
Because only $x,y,r$ are variables the equation is linear.
(For the standard form we might move the $r$ to the other side).

\label{LastPage}


\end{document}
