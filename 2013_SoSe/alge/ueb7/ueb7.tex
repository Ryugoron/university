\documentclass[11pt,a4paper,ngerman]{article}
\usepackage[bottom=2.5cm,top=2.5cm]{geometry} 
\usepackage{babel}
\usepackage[utf8]{inputenc} 
\usepackage[T1]{fontenc} 
\usepackage{ae} 
\usepackage{amssymb} 
\usepackage{amsmath}
\usepackage{amsthm} 
\usepackage{graphicx}
\usepackage{fancyhdr}
\usepackage{fancyref}
\usepackage{listings}
\usepackage{xcolor}
\usepackage{paralist}

\usepackage[pdftex, bookmarks=false, pdfstartview={FitH}, linkbordercolor=white]{hyperref}
\usepackage{fancyhdr}
\pagestyle{fancy}
\fancyhead[C]{Computational Geometry}
\fancyhead[L]{Exercise sheet 7}
\fancyhead[R]{SoSe 2013}
\fancyfoot{}
\fancyfoot[L]{}
\fancyfoot[C]{\thepage \hspace{1px} of \pageref{LastPage}}
\renewcommand{\footrulewidth}{0.5pt}
\renewcommand{\headrulewidth}{0.5pt}
\setlength{\parindent}{0pt} 
\setlength{\headheight}{0pt}

\date{}
\title{Exercise sheet 7}
\author{Max Wisniewski, Alexander Steen}


%%
%% Enviroments for proofs and lemmas
%%
\newtheorem{lemma}{\bfseries Claim}

\begin{document}

\lstset{language=Pascal, basicstyle=\ttfamily\fontsize{10pt}{10pt}\selectfont\upshape, commentstyle=\rmfamily\slshape, keywordstyle=\rmfamily\bfseries, breaklines=true, frame=single, xleftmargin=3mm, xrightmargin=3mm, tabsize=2, mathescape=true}

\renewcommand{\figurename}{Figure}

\maketitle
\thispagestyle{fancy}

%%%%%%%%%%%%%%%%%%%%%%%%
%% Aufgabe 1 
%%%%%%%%%%%%%%%%%%%%%%%%
\subsection*{Problem 1}

Let $P$ be a set of $n$ points in the plane. We would like to compute the minimal disk enclosing all points in $P$.
Consider the randomized inkremental algorithm from the lecture.

\subsubsection*{(a)}
For an order $p_1, ... , p_n$ of points in $P$, let $P_i =  \{ p_1, ... , p_i \}$. Also let $D_i$
be the smallest enclosing disk of $P_i$. Consider how the optimum disc changes when $p_i$ is added and
show that for $(2 < i < n)$ if $p_i \not\in D_{i-1}$, then $p_i$ lies on the boundary of $D_i$.\\

\textbf{Proof.}\\

If $p_i \in D_{i-1}$ then $D_i = D_{i-1}$. $D_i$ cannot be any larger, because $D_{i-1}$ would be smaller and contain
every point in $P_i$. Therefor $D_i$ would not be optimal. On the otherhand $D_i$ cannot be smaller, because
otherwise $D_{i-1}$ could have been smaller already, because $D_i$ contains in particular $P_i \setminus \{ q_i \}$.
Therefor $D_{i-1}$ would not be optimal.\\

Now consider $p_i \not\in D_{i-1}$.\\
First it is the case, that if two minimal disk $D_1, D_2$ contain the same set of points $Q$ $D_1 = D_2$.
\textbf{Proof.}\\
Consider $D_1 \not= D_2$, that $D_1 \cap D_2 \ni P \not= \emptyset$. We take the middle point $p$ of both centers
of the disks. It holds that a disk $D_3$ centered at $P$ and with a diameter of the maximal distance in $D_1 \cap D_2$.\\
We can easily see, that the disk $P \in D_2 \cap D_1 \in D_3$. But because $D_1 \not= D_2$ the diameter is strictly
smaller than the diameter of one of $D_1, D_2$ because otherwise one would be contained strictly in the other.
But this is not possible because $D_1, D_2$ were optimal.
\mbox{} \hfill $\square$

Now assume $p_i$ is not on the boundary of $D_i$. Because by lemma we know, that $D_i$ is strictly larger than $D_{i-1}$.
This is true, because otherwise we saw that because $D_i$ contains $P_{i-1}$ and both are minimal they would be the
same, which is not possible because $p_i$ is in $D_i$ but not in $D_{i-1}$.

But if $p_i$ is not on the boundary of $D_i$ we can move the disk. We keep track, that we keep $D_{i-1}$ contained in $D_{i}$
and $p_i$ until $p_i$ is on the boundary. This is possible because both $D_{i-1}$ and $D_i$ are convex, so we can not move the
boundary to $p_i$ letting $D_{i-1}$ leaving the interiour of $D_i$ partially.\\

But by lemma the new $D_i'$ with $p_i$ on the boundary has to be equal to $D_i$, because both contain $P_i$ and have the same
size and in the case of $D_i$ it was optimal.\\

Therefor $p_i$ has to be on the boundary of $D_i$.

\subsubsection*{(b)}
Show that the algorithm described in the lecture runs in $O(n)$ expected time.\\

\textbf{Proof.}\\

First observe, that a circle is fixed by three non-colinear points on the boundary. If we remove one of 
three points on the boundary, we are able to construct a smaller disk.\\

In every iteration we check if $p_i \in D_{i-1}$ which takes constant time.
If it is true we do nothing. If it is false we have to compute a new disk. This
procedure iterates over all yet seen points. Therefor it takes $O(i)$ time in round $i$
to compute the next disk.\\

We only have to look at the probability for the disk to change in iteration $i$. We therefor
do a backwards analysis as seen in the lecture.\\

We know from the algorithm that $q$ is on the boundary and can not vanish. Let $D_i$ be
the smallest disk for the points in $P_i$. We know that the disk can only change if we drop
below 3 points on the boundary. So if $p_i$ is not in $D_{i-1}$ it has to change
and this means we have to remove one of the 3 points on the boundary. Because $q$ is fixed
we have a choice of $\frac{1}{p}$ to take one node and there are two critical nodes.
Hence $Pr[p_i \not\in D_i] = \frac{2}{p}$.\\

The total running time is therefor
\begin{equation*}\begin{split}
    T(n) &= \overset{n}{\underset{i=0}{\sum}} \underbrace{O(1)}_{\text{Test}} + \underbrace{O(i)}_{\text{Compute next disk}} \cdot Pr[p_i \not\in D_i]\\
        &= O(n) + \overset{n}{\underset{i=0}{\sum}} c \cdot i \cdot \frac{2}{i}\\
        &= O(n) + O(n) = O(n)
\end{split}\end{equation*}

\subsection*{Problem 2}

Consider the procedure \lstinline|RandomPerumtation(A)| that we have described in the lecture and show that it is correct.

\textbf{Proof.}\\

If the algorithm chooses in iteration $i$ a different element, than another Permutation we get a completly different
permutation. This is obvious, because the first one has at position $i$ another element than the other.\\

We now claim, that for $|A| = n$ the algorithm can produce $n!$ different permutations.
Let $|A| = 1$ then the algorithm can only produce one Permutation and this is $1! = 1$.\\

Let $|A| = n + 1$. Then the algorithm can choose
$n + 1$ elements for the first position. As shown before each of the results will be different and computes
by induction $n!$ different permutations per element.
This produces $\overset{n+1}{\underset{i=1}{\sum}} n! = (n+1)!$.\\

Because we know there is no doubeling of permutations in this algorithm and we get every permutation we iterate
the choose pick.\\

For $n$ elements in round $i$ we have a probability of $\frac{1}{n-i}$ to pick an element. This is due to the
fact, that we have only $n-i$ Elements in round $i$ obviously.\\

As shown before the probability is independent, we can simply sum up the probability for a path.
$Pr[\text{RandomPermutation(A)}=A'] = \overset{n-1}{\underset{i=0}{\prod}} \frac{1}{n-i} = \frac{1}{n!}$.

Therefor we know that each Permutation has the same probability to be choosen.

\subsection*{Problem 3}

Give an algorithm to compute whether a given poylgon is a \emph{star-shaped} polygon whose expected running time is
linear.\\

\textbf{Solution}:\\

Let $P$ be a simple polygon. For every edge $(a,b) \in P$ we can define a half-plane $h_{a,b}$.
$h_{a,b}$ is the half-plane where $(a,b)$ is in the boundary $\delta h_{a,b}$ and the halfplane
is directed towards the inside of the polygon. Such that it
can be described by the equation $h_{a,b}(x) = \left[x - a \right] n_{a,b} \geq 0$ where $n_{a,b}$ is 
the normal vektor directed to the inside of the polygon.\\

As one can see $h_{a,b}(x)$ is a linear property in $x$.\\

For a point to see a point of this halfplane it means to be inside of it. A point can therefor see
the hole polygon if $\forall (a,b) \in P \, : \, h_{a,b}(x)$ holds. It lies in all half-planes.\\

We formulate the LP
\begin{equation*}\begin{array}{rlcr}
    \min & (1,1)^T x\\
    \text{s.t.} & \left[ x - a \right] n_{a,b} & \geq 0
\end{array}\end{equation*}
and we know that the solution can see every point in $P$ by construction
and the LP can be solved in $O(n)$ running time by lecture.

\label{LastPage}


\end{document}
