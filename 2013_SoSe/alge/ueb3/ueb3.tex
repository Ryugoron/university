\documentclass[11pt,a4paper,ngerman]{article}
\usepackage[bottom=2.5cm,top=2.5cm]{geometry} 
\usepackage{babel}
\usepackage[utf8]{inputenc} 
\usepackage[T1]{fontenc} 
\usepackage{ae} 
\usepackage{amssymb} 
\usepackage{amsmath}
\usepackage{amsthm} 
\usepackage{graphicx}
\usepackage{fancyhdr}
\usepackage{fancyref}
\usepackage{listings}
\usepackage{xcolor}
\usepackage{paralist}

\usepackage[pdftex, bookmarks=false, pdfstartview={FitH}, linkbordercolor=white]{hyperref}
\usepackage{fancyhdr}
\pagestyle{fancy}
\fancyhead[C]{Computational Geometry}
\fancyhead[L]{Exercise sheet 3}
\fancyhead[R]{SoSe 2013}
\fancyfoot{}
\fancyfoot[L]{}
\fancyfoot[C]{\thepage \hspace{1px} of \pageref{LastPage}}
\renewcommand{\footrulewidth}{0.5pt}
\renewcommand{\headrulewidth}{0.5pt}
\setlength{\parindent}{0pt} 
\setlength{\headheight}{0pt}

\date{}
\title{Exercise sheet 3}
\author{Max Wisniewski, Alexander Steen}


%%
%% Enviroments for proofs and lemmas
%%
\newtheorem{lemma}{\bfseries Claim}

\begin{document}

\lstset{language=Pascal, basicstyle=\ttfamily\fontsize{10pt}{10pt}\selectfont\upshape, commentstyle=\rmfamily\slshape, keywordstyle=\rmfamily\bfseries, breaklines=true, frame=single, xleftmargin=3mm, xrightmargin=3mm, tabsize=2, mathescape=true}

\renewcommand{\figurename}{Figure}

\maketitle
\thispagestyle{fancy}

%%%%%%%%%%%%%%%%%%%%%%%%
%% Aufgabe 1 
%%%%%%%%%%%%%%%%%%%%%%%%
\subsection*{Task 1}

Let $P$ and $Q$ be two convex polygons with $n$ and $m$ vertices respectively, each polygon is given as
a list of its vertices sorted in counter-clockwise (or clock-wise) order. Give a sweep-line algorithm
that computes all intersections between $P$ and $Q$ in $O(n + m)$ times.

\textbf{Solution:}\\

tbd

%%%%%%%%%%%%%%%%%%%%%%%%
%%  Aufgabe 2
%%%%%%%%%%%%%%%%%%%%%%%
\subsection*{Task 2}
Let $P$ be a polygon with $n$ vertices and $h$ holes.

\subsubsection*{(a)}
Give a reasonable definition for a triangulation of $P$.\\

\textbf{Solution:}\\
tbd

\subsubsection*{(b)}
Show that $P$ has a triangulation.\\

\textbf{Proof:}\\

tbd

\subsubsection*{(c)}

Find a formula for the number of triangles in any triangulation of $T$, and proof that it is correct.\\

\textbf{Solution:}\\

tbd

%%%%%%%%%%%%%%%%%%%%%%%%%%
%%  Aufgabe 3
%%%%%%%%%%%%%%%%%%%%%%%%%%
\subsection*{Task 3}

Let $P$ be a simple polygon with $n$ vertices and let $T$ be a triangulation of $P$. The \emph{dual graph} of $T$,
named $T^*$, is the graph whose vertices are the triangles of $T$ in which two triangles are adjazent if and only if
they share a diagonal.

\subsubsection*{(a)}
Show that $T^*$ is a tree.\\

\textbf{Proof:}\\

Because $T$ is a simple polygon $T^*$ is connected. Assume $T^*$ is not a tree. Then it contains at least one
cicle $C$. A cicle of triangles encloses some area in the interior of the circle of triangles. On a euklidian
topological surface these circles have at least on point in the middle. All triangles ar in the interior of the polygon
therefor this inner point cannot be on a edge of the polygon.

This means one endpoint of the diagonals the triangles are made of is not a vertex of the polygon $P$. Therefor these
were not diagonals and the circle could not be made of triangles of a triangulation.\\
\mbox{}\hfill$\square$

\subsubsection*{(b)}

Use $T^*$ to give an alternative proof that $T$ is 3-colorable.\\

\textbf{Proof:}\\

tbd

\subsubsection*{(c)}

Suppose $n \geq 4$. An \emph{ear} of $T$ is a triangle in $T$ that has two polygon edges
as sides. Show that $T$ contains at least two ears.\\

\textbf{Proof:}\\

If $T$ has more than $4$ vertices, that every dual graph $T^*$ representing 
a triangulation is made of at least two nodes.\\
As we have shown in \emph{(a)} $T^*$ is a tree.

For a leave of $T^*$ it holds that it only shares on diagonal with an other triangle.
Therefor the other two edges can only be edges of the polygon $P$ itself.

This implies that every leave of the dual-graph is an ear.\\

Next we know that every tree with $n\geq 4$ contains at least two leaves
\footnote{If we cannot assume this, we take an maximal path in $T^*$, that exsists because there are no circles.
The endpoints of the path are leaves because there exist no other edge except the ones we already took for the path.
Therefor these endpoints have a degree of $\leq 1$.}.
\mbox{} \hfill $\square$

\subsubsection*{(d)}

Let $n \geq 4$. Show that $P$ has a diagonal that partitions $P$ into two
simple poylgons with at least $\frac{n-3}{3} + 2$ vertices.\\

\textbf{Proof:}\\

We assume a polygon wit $n\geq 5$. Therefor we have a dualgraph $T^*$ with at least
three nodes. We can assume this savely, because in a polygon with $n = 4$ vertices
there exists only one diagonal that partitions the polygon equaly.

Some triangulation $T$ of $P$ and look at the dual graph $T^*$ of $T$.

\begin{claim}\label{alge:ueb3:centerpoint}
There exists a node $v$ in $T^*$ were
\begin{itemize}
    \item $d(v) \leq 2$ and the both subtrees $l, r$ hold that
        \begin{equation}\begin{split}
            |l| &\leq |r| + 1\\
            |r| &\leq |l| + 1
        \end{split}\end{equation}
    \item or $d(v) \leq 3$ and for the subtrees $t_1, t_2, t_3$ it holds that
        \begin{equation}\begin{split}
            |t_1| &\leq |t_2| + |t_3| + 1\\
            |t_2| &\leq |t_1| + |t_3| + 1\\
            |t_3| &\leq |t_1| + |t_2| + 1
        \end{split}\end{equation}
\end{itemize}
\end{claim}



\label{LastPage}

\end{document}
