\documentclass[11pt,a4paper,ngerman]{article}
\usepackage[bottom=2.5cm,top=2.5cm]{geometry} 
\usepackage{babel}
\usepackage[utf8]{inputenc} 
\usepackage[T1]{fontenc} 
\usepackage{ae} 
\usepackage{amssymb} 
\usepackage{amsmath}
\usepackage{amsthm} 
\usepackage{graphicx}
\usepackage{fancyhdr}
\usepackage{fancyref}
\usepackage{listings}
\usepackage{xcolor}
\usepackage{paralist}

\usepackage[pdftex, bookmarks=false, pdfstartview={FitH}, linkbordercolor=white]{hyperref}
\usepackage{fancyhdr}
\pagestyle{fancy}
\fancyhead[C]{Computational Geometry}
\fancyhead[L]{Exercise sheet 3}
\fancyhead[R]{SoSe 2013}
\fancyfoot{}
\fancyfoot[L]{}
\fancyfoot[C]{\thepage \hspace{1px} of \pageref{LastPage}}
\renewcommand{\footrulewidth}{0.5pt}
\renewcommand{\headrulewidth}{0.5pt}
\setlength{\parindent}{0pt} 
\setlength{\headheight}{0pt}

\date{}
\title{Exercise sheet 3}
\author{Max Wisniewski, Alexander Steen}


%%
%% Enviroments for proofs and lemmas
%%
\newtheorem{lemma}{\bfseries Claim}

\begin{document}

\lstset{language=Pascal, basicstyle=\ttfamily\fontsize{10pt}{10pt}\selectfont\upshape, commentstyle=\rmfamily\slshape, keywordstyle=\rmfamily\bfseries, breaklines=true, frame=single, xleftmargin=3mm, xrightmargin=3mm, tabsize=2, mathescape=true}

\renewcommand{\figurename}{Figure}

\maketitle
\thispagestyle{fancy}

%%%%%%%%%%%%%%%%%%%%%%%%
%% Aufgabe 1 
%%%%%%%%%%%%%%%%%%%%%%%%
\subsection*{Task 1}

Let $P$ and $Q$ be two convex polygons with $n$ and $m$ vertices respectively, each polygon is given as
a list of its vertices sorted in counter-clockwise (or clock-wise) order. Give a sweep-line algorithm
that computes all intersections between $P$ and $Q$ in $O(n + m)$ times.

\textbf{Solution:}\\

We will commence just as in the standard sweepline algorithm but we change the datastructures. Due to the fact
that we have the vertices already sorted we do not need an event structure.\\
What we will save is for each polygon two points. On is the next point on the upper half of the polygon
and the other one is the next point on the lower half of the polygon.

The start point can be computed in $O(n + m)$ be searching the minimum x-value. The next point on the upper half
can be found by taken the next in cw order and the next on the lower half is the previous in cw order.\\

Next we observe that each side of the polygon can be intersected at most twice by the other polygon. Otherwise
it could not be convex. Therefor we have to keep track of at most 8 points, that we have to consider. The rest of
the points can be computed as soon as we reached the next vertex of the polygon.\\

Therefor we get the next event point in $O(1)$ time and can store the intersections as another convex polygon also in $O(1)$ time
per round. Thus for vertex of each convex hull we half a constant number of events and intersections which can be computed in constant time.\\

The computation with sweepline on this simpler data structure has a total running time of $O(n+m)$.

%%%%%%%%%%%%%%%%%%%%%%%%
%%  Aufgabe 2
%%%%%%%%%%%%%%%%%%%%%%%
\subsection*{Task 2}
Let $P$ be a polygon with $n$ vertices and $h$ holes.\\

We assume that the holes do not intersect each other. If that would be the case we would merge the two holes.

\subsubsection*{(a)}
Give a reasonable definition for a triangulation of $P$.\\

\textbf{Solution:}\\


We first redefine what a diagonal is in this context.

\begin{def}\label{alge:ueb3:holes:def1}
    A \emph{diagonal} in a polygon $P$ with holes $H_1,...,H_h$ is a line $\overline{uv}$ where $u,v \in P \cup \bigcup_{1\leq i \leq h} H_i$
    some vertex such that $\overline{uv}$ is completely in the interior of $P$ and is never in the interior of any $H_i$.\mbox{}\hfill$\lrcorner$
\end{def}

And now we are able to define a triangulation.

\begin{def}\label{alge:ueb3:hole:def2}
    A \emph{triangulation} of a polygon $P$ with $h$ holes $H_1, ..., H_h$ is a maximal set of diagonals 
    that do not cross each other in $P$ with $H_1, ... ,H_h$.
    \mbox{}\hfill$\lrcorner$
\end{def}

\subsubsection*{(b)}
Show that $P$ has a triangulation.\\

\textbf{Proof:}\\

Let $P$ be a simple polygon and $H_1, ... , H_h$ holes. We do an induction on $h$.
\begin{description}
    \item[\bfseries ind. anc.] $h=0$.\\
        $P$ is a simple polygon and has a triangulation as shown in the lecture.
    \item[\bfseries ind. hyp.] For any simple polygon $P$ with $H_1, ... , H_h$ there exist a triangulation.
    \item[\bfseries ind. step.] Let $P$ be a simple polygon and $H_1, ... , H_h, H_{h+1}$ holes.\\
        Let $d_{h+1}$ be a diagonal with some endpoint in $H_{h+1}$.\\
        
        This diagonal exists. Otherwise there would not be any diagonal in $P$ without the hole $H_{h+1}$
        which would be in the interior of $H_{h+1}$, because we could move this diagonal to some vertex
        of this edge. If such an edge would not exists we can simply ignore $H_{h+1}$ because it is not
        recognizable by any measurement.\\

        If we have this diagonal we virtually split the polygon along this edge. This can be done by doubling the
        endpoints and moving them apart in some $\varepsilon > 0$ Ball such that the both resulting lines would not collide.\\

        We get a new Polygon in which $H_{h+1}$ is either part of $P$ or merged with one of $H_i$.
        The resulting polygon has only $h$ holes. And there exists a triangulation $T'$.\\

        The diagonals of this triangulation $T'$ do not cross $d_{h+1}$ because in the construction $d_{h+1}$ has a small space
        that is not part of the polygon, therefor there exists no diagonal crossing this space.

        We obtain the triangulation $T$ by renaming the vertices which we split again into one and adding $d_{h+1}$.
\end{description}

\subsubsection*{(c)}

Find a formula for the number of triangles in any triangulation of $P$, and proof that it is correct.\\

\textbf{Solution:}\\

By construction in (b) we add for each hole in $P$ one diagonal, add therefor two points, and in the end we triangulate a simple polygon.
The number of triangles in a triangulation for a simple polygon are $n-2$.

Therefor by construction we need $n + 2*h -2$ triangles.\\

\textbf{Proof:}\\
Leaves us with the proof that this is always the case.\\
For every triangulation we have for each hole at least one diagonal connected to it. Therefor we can proceed as in the
algorithm above to merge the hole along this diagonal. The rest of the polygon still contains all the original triangles,
except we moved the ones at the merge diagonal a bit.\\

We proceed until we have no more holes and get a Polygon with $n + 2h$ nodes and all triangles from the original one.\\
From this new Triangulation we know from the lecture, that it contains $n + 2h -2$ triangles. Hence the original
triangulation has $n + 2h - 2$ triangles.\\
\mbox{} \hfill $\square$

%%%%%%%%%%%%%%%%%%%%%%%%%%
%%  Aufgabe 3
%%%%%%%%%%%%%%%%%%%%%%%%%%
\subsection*{Task 3}

Let $P$ be a simple polygon with $n$ vertices and let $T$ be a triangulation of $P$. The \emph{dual graph} of $T$,
named $T^*$, is the graph whose vertices are the triangles of $T$ in which two triangles are adjacent if and only if
they share a diagonal.

\subsubsection*{(a)}
Show that $T^*$ is a tree.\\

\textbf{Proof:}\\

Because $T$ is a simple polygon $T^*$ is connected. Assume $T^*$ is not a tree. Then it contains at least one
cycle $C$. A cycle of triangles encloses some area in the interior of the circle of triangles. On a euclidean
topological surface these circles have at least on point in the middle. All triangles are in the interior of the polygon
therefor this inner point cannot be on a edge of the polygon.

This means one endpoint of the diagonals the triangles are made of is not a vertex of the polygon $P$. Therefor these
were not diagonals and the circle could not be made of triangles of a triangulation.\\
\mbox{}\hfill$\square$

\subsubsection*{(b)}

Use $T^*$ to give an alternative proof that $T$ is 3-colorable.\\

\textbf{Proof:}\\

We prove the following:\\
Given a diagonal and two colors for each edge, we can find a coloring for the triangulation $T$.\\
Induction on the size of $T^*$.
\begin{description}
    \item[ind. anc.] $|T^*| = 1$\\
        We have exactly on triangle and this one has already given two colors. We take the third one
        for the last edge.
    \item[ind. step] $|T^*| \leq n \leadsto |T^*| = n+1$\\
        We are given two colors on one edge. This diagonal is part of at least one triangle. We take the missing
        third color for this triangle. Then we know in $T^*$ this triangle has a set of subtrees. Each subtree does not share
        an edge and at most one node, of which we already know the color. For each subtree we have two colors on one edge, namely
        the edges of the triangle. Therefor we can color the subtrees with 3 colors keeping the colors of the initial triangle
        by induction hypotheses.\\

        The coloring of the subtrees does not collide on the vertices,  because we initially gave them the same color.
        This leads to a 3-coloring of the whole tree $T^*$.
\end{description}
\mbox{}\hfill $\square$

\subsubsection*{(c)}

Suppose $n \geq 4$. An \emph{ear} of $T$ is a triangle in $T$ that has two polygon edges
as sides. Show that $T$ contains at least two ears.\\

\textbf{Proof:}\\

If $T$ has more than $4$ vertices, that every dual graph $T^*$ representing 
a triangulation is made of at least two nodes.\\
As we have shown in \emph{(a)} $T^*$ is a tree.

For a leave of $T^*$ it holds that it only shares on diagonal with an other triangle.
Therefor the other two edges can only be edges of the polygon $P$ itself.

This implies that every leave of the dual-graph is an ear.\\

Next we know that every tree with $n\geq 4$ contains at least two leaves
\footnote{If we cannot assume this, we take an maximal path in $T^*$, that exists because there are no circles.
The endpoints of the path are leaves because there exist no other edge except the ones we already took for the path.
Therefor these endpoints have a degree of $\leq 1$.}.
\mbox{} \hfill $\square$

\subsubsection*{(d)}

Let $n \geq 4$. Show that $P$ has a diagonal that partitions $P$ into two
simple polygons with at least $\frac{n-3}{3} + 2$ vertices.\\

\textbf{Proof:}\\

We assume a polygon wit $n\geq 5$. Therefor we have a dual graph $T^*$ with at least
three nodes. We can assume this safely, because in a polygon with $n = 4$ vertices
there exists only one diagonal that partitions the polygon equally.

Some triangulation $T$ of $P$ and look at the dual graph $T^*$ of $T$.

\begin{lemma}\label{alge:ueb3:centerpoint}
There exists a node $v$ in $T^*$ were
\begin{itemize}
    \item $d(v) \leq 2$ and the both subtrees $l, r$ hold that
        \begin{equation}\begin{split}
            |l| &\leq |r| + 1\\
            |r| &\leq |l| + 1
        \end{split}\end{equation}
    \item or $d(v) \leq 3$ and for the subtrees $t_1, t_2, t_3$ it holds that
        \begin{equation}\begin{split}
            |t_1| &\leq |t_2| + |t_3| + 1\\
            |t_2| &\leq |t_1| + |t_3| + 1\\
            |t_3| &\leq |t_1| + |t_2| + 1
        \end{split}\end{equation}
\end{itemize}
or there exists two nodes $u,v$ which share a diagonal along both dissatisfy this property.
\end{lemma}

\textbf{Proof \ref{alge:ueb3:centerpoint}.}\\
Note that if we are in a vertex of degree $2$ or $3$ at most one of the equations can be false.\\

If there exist no such node we can find an infinite sequence of vertices $(v_i)_{i\in \mathbb{N}}$
where for any $i \in \mathbb{N}$
\begin{itemize}
    \item if $v_i$ is a leave $v_{i+1}$ is its parent.
    \item if $v_i$ has degree $2$ $v_{i+1}$ is the vertex in $l$ if $|l| \geq |r| + 1$ and vis-verca.
    \item if $v_i$ has degree $3$ $v_{i+1}$ is the vertex in the subtree that does not satisfy the condition.
\end{itemize}
If this sequence would not exist, there would be a vertex that has the property we are searching for.\\

Assume there is a loop that repeats the sequence $v_{i+1}v_i$. Then both mutually dissatisfy the property.
Otherwise they could not loop.\\

Any other cycle is not possible. If $v_{i}v_{i+1}$ is in the sequence $v_{i+1}v_i$ can not be in there later on.
As shown before at most one successor can dissatisfy the property such that if we left from $v_{i+1}$ not to $v_i$ we
will do this any other time we enter $v_{i+1}$. Therefor $v_i$ cannot be a successor.\\

Because $T^*$ is a tree we have no other possibility to create a circle in $T^*$.

Due to the fact, that $T^*$ is finite either $(v_i)_{i \in \mathbb{N}}$ is looping in two nodes as in the claim
or the sequence ends in a node that satisfies the property.\\
\mbox{}\hfill $\square$

\textbf{Proof 3d.}\\
First we observe, that with the formulas for the triangles ins any triangulation we have $n-2$ triangles in
each triangulation. Therefor the number of nodes in a tree with $k$ triangles is $k+2$.\\

Using claim \ref{alge:ueb3:centerpoint} we know there either exists a point with satisfies the property or two,
that share an edge along which the property is not satisfied.\\

In the second case, we will simple cut this edge.
\begin{itemize}
     \item If $u,v$ had both degree $2$ we know that we have a graph $T_1 - u - v - T_2$
        with $|T_1| + 1 > |T_2| + 1$ but also $|T_2| + 1 > |T_1| + 1$. Therefor this
        constellation is not possible.
    \item If $u,v$ have both degree $3$ we know that mutually both subtrees on $u,v$ must hold
        more than $\frac{2}{3}$ of the tree. Because they share no nodes the tree would have $\frac{4}{3}n$ nodes
        which is not possible.
    \item If one of them has degree on both of them must have, because their subtrees cannot be larger then one.
        But we explicitly cut this case.
    \item If $u,v$ have degree 2 and 3 we have a picture as $T_1 - u - v = (T_2, T_3)$ if w.l.o.g. $u$ has degree
        $2$ and $v$ has degree $3$. Then we know, that $|T_1| + 1 > \frac{2}{3}n$ and $|T_2| + |T_3| + 1 > \frac{1}{2} n$.
        Cutting this edge is save for our theorem.
\end{itemize}

If we have a node that satisfies the property.
\begin{itemize}
    \item If $v$ is the first case. We can cut any of the two edges. If $T_1 - v - T_2$ is
        the tree and we cut $T_1 - v$ then $|T_1| = \frac{k}{2} -1 = \frac{n+2 - 2}{2}$ vertices. The other have is bigger.
        Because we assumed $n \geq 5$ the theorem is satisfied.
    \item If $v$ is the second case. We can cut any of the edges. If $T_1 - v = (T_2, T_3)$ is the
        tree and we cut $T_1 - v$ then $|T_1| = \frac{n-3}{3} + 2$ vertices big.
\end{itemize}
\mbox{}\hfill $\square$
\label{LastPage}

\end{document}
