\documentclass[11pt,a4paper,ngerman]{article}
\usepackage[bottom=2.5cm,top=2.5cm]{geometry} 
\usepackage{babel}
\usepackage[utf8]{inputenc} 
\usepackage[T1]{fontenc} 
\usepackage{ae} 
\usepackage{amssymb} 
\usepackage{amsmath}
\usepackage{amsthm} 
\usepackage{graphicx}
\usepackage{fancyhdr}
\usepackage{fancyref}
\usepackage{listings}
\usepackage{xcolor}
\usepackage{paralist}

\usepackage[pdftex, bookmarks=false, pdfstartview={FitH}, linkbordercolor=white]{hyperref}
\usepackage{fancyhdr}
\pagestyle{fancy}
\fancyhead[C]{Computational Geometry}
\fancyhead[L]{Exercise sheet 5}
\fancyhead[R]{SoSe 2013}
\fancyfoot{}
\fancyfoot[L]{}
\fancyfoot[C]{\thepage \hspace{1px} of \pageref{LastPage}}
\renewcommand{\footrulewidth}{0.5pt}
\renewcommand{\headrulewidth}{0.5pt}
\setlength{\parindent}{0pt} 
\setlength{\headheight}{0pt}

\date{}
\title{Exercise sheet 5}
\author{Max Wisniewski, Alexander Steen}


%%
%% Enviroments for proofs and lemmas
%%
\newtheorem{lemma}{\bfseries Claim}

\begin{document}

\lstset{language=Pascal, basicstyle=\ttfamily\fontsize{10pt}{10pt}\selectfont\upshape, commentstyle=\rmfamily\slshape, keywordstyle=\rmfamily\bfseries, breaklines=true, frame=single, xleftmargin=3mm, xrightmargin=3mm, tabsize=2, mathescape=true}

\renewcommand{\figurename}{Figure}

\maketitle
\thispagestyle{fancy}

%%%%%%%%%%%%%%%%%%%%%%%%
%% Aufgabe 1 
%%%%%%%%%%%%%%%%%%%%%%%%
\subsection*{Problem 1}

\subsubsection*{(a)}

Let $P$ be a simple polygon and let $s,t \in P$. Assume general position. Show that there exists exactly one shortest path
from $s$ to $t$ inside $P$.\\

\textbf{Proof.}\\

Let $\pi_1$ and $\pi_2$ be two shortest paths from $s$ to $t$ not equal. Let $a, b \pi_1 \cap \pi_2$ two nodes
with $a \overset{\pi_1}{\rightarrow} b$ and $a \overset{\pi_2}{\rightarrow} b$ do not share an edge.
Because $P$ is a simple polygon without holes the space enclosed by these two chains lies completely in the polygon.

We look at the straight line $\overline{ab}$. If it does not intersect one of $a \overset{\pi_1}{\rightarrow} b$ 
or $a \overset{\pi_2}{\rightarrow} b$ by triangle inequality
$\overline{ab}$ is the shortest path and therefor $\pi_1 = \pi_2$. 
Otherwise we cut one of the paths 
$a \overset{\pi_1}{\rightarrow}$, $a \overset{\pi_2}{\rightarrow}$ and are outside of out section to look at.

Let $\pi_c$ be the path on which the edge lies and $c$ be the vertex next in order to cut the line. 
And let $d$ be the point where we enter the section again.
Then path $v \rightarrow c \overset{\pi_c}{\rightarrow} d \overset{\pi_c}{\rightarrow} b$ is shorter than both $\pi_1$ and $\pi_2$
by triangle inequality. We know that a straight line is the shortest path in euklidiean vektorspaces.
Because we take a route closer to this straight line we can only get shorter.\\

This includes the case, that we cut the first and then the last edge of one path. Because both paths cannot cross each other one
of both of these takes a larger curver and can therefor be shortcutted.

Hence $\pi_1$, $\pi_2$ were not the shortest paths.
\mbox{}\hfill$\square$

\subsubsection*{(b)}

Consider the shortest paths $\pi_i^l$ and $\pi_i^r$ inside the sleeve from $s$ to the left and the right endpoint
 of $d_i$. Let $v$ be the last common vertex of $\pi_i^l$ and $\pi_i^r$.

\begin{enumerate}[(i)]
    \item Show that $s \overset{\pi_i^l}{\rightarrow} v = s \overset{\pi_i^r}{\rightarrow} v$.\\
    \textbf{Proof.}\\
        We know $s, v \in p$ and by \emph{(a)} we know that shortest paths are unique.
        Hence both paths are the same.
        \mbox{} \hfill $\square$
    \item Show that all inner vertices of $v \overset{\pi_i^l}{\rightarrow} d_i^l$ lie on the left chain and
        that all inner vertices of $v \overset{\pi_i^r}{\rightarrow} d_i^r$ lie on the right chain.\\
    \textbf{Proof.}\\
        Both cases can a symmetric. Therefor we look only at the case $v \overset{\pi_i^l}{\rightarrow} d_i^l$ lie on the left chain.

        Assume there is a point $v_r in v \overset{\pi_i^l}{\rightarrow} d_i^l$ that is on the right chain after $v$.
        We know that $v_r$ cannot be in $v \overset{\pi_i^r}{\rightarrow} d_i^r$, because then $v_r$ would be the last common point.
        
        Next $v_r$ is a point of the polygon and to the right is the outside and by lemma of the lecture the
        path goes throug $v_r$ or is left of it.
        If $\pi_i^r$ is left of $v_r$ then $\pi_i^l$ is crossing $\pi_i^r$, which again is impossible by lemma.
        There is no other possible way vor $\pi_i^r$ to go by $v_r$. Hence $v_r$ cannot be part of $v \overset{\pi_i^l}{\rightarrow} d_i^l$.
        \mbox{} \hfill $\square$        
    \item Show that $v \overset{\pi_i^l}{\rightarrow} d_i^l$, $v \overset{\pi_i^r}{\rightarrow}d_i^r$ and $d_i$ form a simple polygon, that 
        is bounded by two inwardly concave chains.\\
    \textbf{Proof.}\\
        A shortest path cannot cross itself and as shown $v \overset{\pi_i^r}{\rightarrow} d_i^r$ and $v \overset{\pi_i^l}{\rightarrow} d_i^l$ 
        do not cross or share a node. Therefor
        $v \overset{\pi_i^r} d_i^r \overset{d_i^r}{\rightarrow} d_i^l \overset{d_i^l}{\rightarrow} v$ is a crossing-free polynomial chain
        and by definition a simple polygon.

        The chains a inwardly concave. Assume it is not the case, than we have a convex part from some vertex $v_a$ to $v_b$ in $P$. Then
        there exists a shorter path from $v_a$ to $v_b$ either by taking the direct diagonal $\overline{v_a v_b}$ or by shortcutting over
        the other chain. Either way a concave section can always be replaced by a shorter path.
        \mbox{} \hfill $\square$
\end{enumerate}

\subsection*{Problem 2}

Extra sheet.\\

Gibt es eine langweiligere Aufgabe, als das hier?

\subsection*{Problem 3}

Let $P$ be a planar $n$-point set in general position. Show how to find in $O(n \log n)$ time a simple polygon
whose vertices are exactly the points in $P$.\\

\textbf{Solution.}\\

We assume that no two points share the same $x$ - coordinate.

\begin{enumerate}[1.]
    \item Sort the points by $x$ - coordinate.
    \item Take $p_0$ as the point with the smallest $x$ - coordinate and $p_n$ as the one with
        the highest $x$-coordinate.
    \item Split the points into two sets $P_1 = \{ p \in P \, | \, \text{right of or on } \overline{p_0 p_n} \}$
        and $P_2 = \{ p \in P \, | \, \text{left of } \overline{p_0 p_n} \}$. $p_0$ and $p_n$ should be in there both.
    \item for both sets starting with $p_i = p_0$.
        \begin{enumerate}[a.]
            \item take the next vertex $p_{i+1}$ from the set and add the
                edge $\overline{p_ip_{i+1}}$.
            \item continue with $p_{i+1}$.
        \end{enumerate}
    \item Connect both chains.
\end{enumerate}

\begin{lemma}\label{alge:ueb5:simp:cor}
    The algorithm above is correct and runs in time $O(n \log n)$.
\end{lemma}

\textbf{Proof \ref{alge:ueb5:simp:cor}.}\\

On each point set the chains moving left to right have no edges crossing because the points
are sorted by $x$-coordinate. A later edge is always right to all previous edges and cannot cross 
an earlier one.

The two chains do not intersect mutually, because the points from $P_1, P_2$ are divided by
the line $\overline{p_0p_n}$ and therefor the chains do not cross this line. And at most the chain in $P_1$ can 
overlap with this line.

We obtain a polynomial chain with no crossing that is a circle and therefor it is a simple polygon, because
the start end endpoint of both lines equal and we took each point of $P$.

The running time is in $O(n \log n)$.
Step one runs in $O(n \log n)$ because it sorts the points. The second then can be computed in $O(1)$.
The split takes $O(n)$ time and after that the list remain sorted.

Iterating over the list takes per round $O(1)$ time and $O(n)$ total. The last step takes constant time.\\

Therefor the algorithm has the mentioned running time.
\mbox{} \hfill $\square$\\

The general position can be dropped if we assume at same $x$ - coordinate the one with the smaller $y$ coordinate smaller.
We get a consistent sorting and the crossing - free argument from the proof remains true, because on the same level we can only
move from bottom to top and never back.

\label{LastPage}


\end{document}
