\documentclass[11pt,a4paper,ngerman]{article}
\usepackage[bottom=2.5cm,top=2.5cm]{geometry} 
\usepackage{babel}
\usepackage[utf8]{inputenc} 
\usepackage[T1]{fontenc} 
\usepackage{ae} 
\usepackage{amssymb} 
\usepackage{amsmath}
\usepackage{amsthm} 
\usepackage{graphicx}
\usepackage{fancyhdr}
\usepackage{fancyref}
\usepackage{listings}
\usepackage{xcolor}
\usepackage{paralist}

\usepackage[pdftex, bookmarks=false, pdfstartview={FitH}, linkbordercolor=white]{hyperref}
\usepackage{fancyhdr}
\pagestyle{fancy}
\fancyhead[C]{Computational Geometry}
\fancyhead[L]{Exercise sheet 2}
\fancyhead[R]{SoSe 2013}
\fancyfoot{}
\fancyfoot[L]{}
\fancyfoot[C]{\thepage \hspace{1px} of \pageref{LastPage}}
\renewcommand{\footrulewidth}{0.5pt}
\renewcommand{\headrulewidth}{0.5pt}
\setlength{\parindent}{0pt} 
\setlength{\headheight}{0pt}

\date{}
\title{Exercise sheet 2}
\author{Max Wisniewski, Alexander Steen}


%%
%% Enviroments for proofs and lemmas
%%
\newtheorem{lemma}{\bfseries Claim}

\begin{document}

\lstset{language=Pascal, basicstyle=\ttfamily\fontsize{10pt}{10pt}\selectfont\upshape, commentstyle=\rmfamily\slshape, keywordstyle=\rmfamily\bfseries, breaklines=true, frame=single, xleftmargin=3mm, xrightmargin=3mm, tabsize=2, mathescape=true}

\renewcommand{\figurename}{Figure}

\maketitle
\thispagestyle{fancy}

\begin{description}
%%%%%%%%%%%%%%%%%%%%%%%%
%% Aufgabe 1 
%%%%%%%%%%%%%%%%%%%%%%%%
\item[Problem 1] Convexity
  \begin{enumerate}
    \item Let $\{C_i\}_{i \in I}$ be a set of convex sets. Show that $\bigcap_{i \in I} C_i$ is convex.\\

          \textbf{Proof}: Since each $C_i$ is convex, it holds that the line $\overline{pq}$,
                          for $p,q \in C_i$, is completely contained in $C_i$. 
                          %%More formally
                          %%$\forall p,q \in C_i \, \forall \alpha \in [0,1]:
                          %%(1- \alpha) p + \alpha (q-p) \in C_i$, for all $i$. 
                          Since for all $p,q \in \bigcap_{i \in I} C_i$, the segment 
                          $\overline{pq}$ is contained in each $C_i$, it holds that
                          $\overline{pq} \in \bigcap_{i \in I} C_i$.\\
                          $\Rightarrow \bigcap_{i \in I} C_i$ is convex.
                          \mbox{} \hfill $\square$ \\

          A similar property can be found for unions of convex sets: \\        
          \textbf{Claim}: For any non-decreasing series of
                          convex sets $(C_i)_{i \in I}$ (with respect to set inclusion),
                          the set $\bigcup_{i \in I} C_i$ is convex.\\
          \textbf{Proof}: For all $p,q \in \bigcup_{i \in I} C_i$ there exists a $\tilde{i} \in I$
                          such that $p,q \in C_{\tilde{i}}$, hence $\overline{pq} \in C_{\tilde{i}}$.
                          Since $(C_i)_{i \in I}$ is non-decreasing,
                          $\overline{pq} \in \bigcup_{i \in I} C_i$.
                          \mbox{} \hfill $\square$ 
                          
    \item Let $P$ be a finite point set in the plane.
          Show that the boundary of the convex hull $CH(P)$ of $P$ is a convex polygon
          whose vertices are points of $P$. \\
    
          TBA
    \item Show that the segment between two points $p, q \in P$ is an edge of $CH(P)$ if
          and only if all points of $P$ lie on the same side of the line through $p$ and $q$. \\

          \textbf{Proof}: \\
          "$\Rightarrow$": By contraposition. Let $\tilde{p}$ a point on the one side of the line
                           through $p$ and $q$ and $\tilde{q}$ a point on the other side.
                           Let $s \in \overline{pq}$. Then, one of $\overline{s\tilde{p}}$ 
                           and $\overline{s\tilde{q}}$ is not contained in $P$. Hence,
                           $\overline{pq}$ cannot be an edge of $CH(P)$. \\
          "$\Leftarrow$": TBA
          \mbox{} \hfill $\square$
  \end{enumerate}

%%%%%%%%%%%%%%%%%%%%%%%%
%% Aufgabe 2
%%%%%%%%%%%%%%%%%%%%%%%%
\item[Problem 2] Computing the tangents to a polygon \\

TBA
%%%%%%%%%%%%%%%%%%%%%%%%
%% Aufgabe 3
%%%%%%%%%%%%%%%%%%%%%%%%
\item[Problem 3] Chan's algorithm and superexponential search
  \begin{enumerate}
    \item 
    \item 
  \end{enumerate}

\end{description}

\label{LastPage}

\end{document}
