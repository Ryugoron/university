\documentclass[11pt,a4paper,ngerman]{article}
\usepackage[bottom=2.5cm,top=2.5cm]{geometry} 
\usepackage{babel}
\usepackage[utf8]{inputenc} 
\usepackage[T1]{fontenc} 
\usepackage{ae} 
\usepackage{amssymb} 
\usepackage{amsmath}
\usepackage{amsthm} 
\usepackage{graphicx}
\usepackage{fancyhdr}
\usepackage{fancyref}
\usepackage{listings}
\usepackage{xcolor}
\usepackage{paralist}

\usepackage[pdftex, bookmarks=false, pdfstartview={FitH}, linkbordercolor=white]{hyperref}
\usepackage{fancyhdr}
\pagestyle{fancy}
\fancyhead[C]{Computational Geometry}
\fancyhead[L]{Exercise sheet 2}
\fancyhead[R]{SoSe 2013}
\fancyfoot{}
\fancyfoot[L]{}
\fancyfoot[C]{\thepage \hspace{1px} of \pageref{LastPage}}
\renewcommand{\footrulewidth}{0.5pt}
\renewcommand{\headrulewidth}{0.5pt}
\setlength{\parindent}{0pt} 
\setlength{\headheight}{0pt}

\date{}
\title{Exercise sheet 2}
\author{Max Wisniewski, Alexander Steen}


%%
%% Enviroments for proofs and lemmas
%%
\newtheorem{lemma}{\bfseries Claim}

\begin{document}

\lstset{language=Pascal, basicstyle=\ttfamily\fontsize{10pt}{10pt}\selectfont\upshape, commentstyle=\rmfamily\slshape, keywordstyle=\rmfamily\bfseries, breaklines=true, frame=single, xleftmargin=3mm, xrightmargin=3mm, tabsize=2, mathescape=true}

\renewcommand{\figurename}{Figure}

\maketitle
\thispagestyle{fancy}

\begin{description}
%%%%%%%%%%%%%%%%%%%%%%%%
%% Aufgabe 1 
%%%%%%%%%%%%%%%%%%%%%%%%
\item[Problem 1] Convexity
  \begin{enumerate}
    \item Let $\{C_i\}_{i \in I}$ be a set of convex sets. Show that $\bigcap_{i \in I} C_i$ is convex.\\

          \textbf{Proof}: Since each $C_i$ is convex, it holds that the line $\overline{pq}$,
                          for $p,q \in C_i$, is completely contained in $C_i$. 
                          %%More formally
                          %%$\forall p,q \in C_i \, \forall \alpha \in [0,1]:
                          %%(1- \alpha) p + \alpha (q-p) \in C_i$, for all $i$. 
                          Since for all $p,q \in \bigcap_{i \in I} C_i$, the segment 
                          $\overline{pq}$ is contained in each $C_i$, it holds that
                          $\overline{pq} \in \bigcap_{i \in I} C_i$.\\
                          $\Rightarrow \bigcap_{i \in I} C_i$ is convex.
                          \mbox{} \hfill $\square$ \\

          A similar property can be found for unions of convex sets: \\        
          \textbf{Claim}: For any non-decreasing series of
                          convex sets $(C_i)_{i \in I}$ (with respect to set inclusion),
                          the set $\bigcup_{i \in I} C_i$ is convex.\\
          \textbf{Proof}: For all $p,q \in \bigcup_{i \in I} C_i$ there exists a $\tilde{i} \in I$
                          such that $p,q \in C_{\tilde{i}}$, hence $\overline{pq} \in C_{\tilde{i}}$.
                          Since $(C_i)_{i \in I}$ is non-decreasing,
                          $\overline{pq} \in \bigcup_{i \in I} C_i$.
                          \mbox{} \hfill $\square$ 
                          
    \item Let $P$ be a finite point set in the plane.
          Show that the boundary of the convex hull $CH(P)$ of $P$ is a convex polygon
          whose vertices are points of $P$. \\

	 \textbf{Proof:}\\
	 We first proove, that $CH(P)$ is a polygon and second, the 
	 points of the polygon are points of $P$.\\

   	 Let $C$ be an arbitrary convex set containing $P$. Assume $C$ is
	not a polygon. Fix a point $x$ on that curve 
	\footnote{We assume complete sets. If the set has no boundary
	we consider a series of points converging to the boundary.}.
	We now take $\varepsilon > 0$ steps on the curve to the point
	$x_\varepsilon$ and look at the 
	set $C'$ where we substituted the line $\overline{xx_\varepsilon}$
	for the curveseqment that connected them before.

	We know if $\varepsilon$ is small enough there is no point of $P$
	in the cut part. If the curve was bend left, the resulting line will
	only do left turns at the end. Therefore $C'$ is convex.

	Hence $C$ could not be the convex hull.\\

	Next we fix a convex polygon $CP$ that has points of $P$ on the
	nodes. This can be found by intersecting all halfplanes that contain all
	points in $P$ as in the Brute-Force algorithm.\\

	Because $CP$ is a convex set containing $P$ only 
	$CH(P) \subset CP$ can hold.\\

	Assume there is a point on the polygon $CH(P)$ $y=p_k \not \in P$.
	Then $p_{k-1}p_kp_{k+1}$ is a triangle pointing to the outside
	of the polygon. Otherwise we would have taken a left turn in cw
	order.
	But this is not possible due to the previous shown fact, that
	a polygon only of points of $P$ is a convex set containing $P$.
	Hence $p_k$ can not be on the hull.\\
\mbox{}\hfill$\square$

    \item Show that the segment between two points $p, q \in P$ is an edge of $CH(P)$ if
          and only if all points of $P$ lie on the same side of the line through $p$ and $q$. \\

          \textbf{Proof}: \\
          "$\Rightarrow$": By contraposition. Let $\tilde{p}$ a point on the one side of the line
                           through $p$ and $q$ and $\tilde{q}$ a point on the other side.
                           Let $s \in \overline{pq}$. Then, one of $\overline{s\tilde{p}}$ 
                           and $\overline{s\tilde{q}}$ is not contained in $P$. Hence,
                           $\overline{pq}$ cannot be an edge of $CH(P)$. \\
          "$\Leftarrow$": \\
		By the previous part, we know that the $CH(P)$ is a polygon
		of the points in $P$. Because $p,q \in P$ we know that 
		$\overline{pq} \in CH(P)$.\\ 

		Assume $\overline{pq}$ is not
		on the boundary. Then there exists points $u,v \in P$
		on both sides of $\overline{pq}$ by the previous part.
		But this is not possible due to the premise.
                
          \mbox{} \hfill $\square$
  \end{enumerate}

%%%%%%%%%%%%%%%%%%%%%%%%
%% Aufgabe 2
%%%%%%%%%%%%%%%%%%%%%%%%
\item[Problem 2] Computing the maximal tangent to a subconvex hull in Chens Algorithm in $O(log h^*)$. \\

\textbf{Proof:}\\

Let $p_{k-1}, p_k$ be the last computed points on the convex hull.
Let $q_1, ..., q_{h^*}$ be the points on the convex hull of the subconvex hull
given in ccw or cw order. Then we will compute the point on the hull that
maximizes the angle as follows. Assume we have the points given in an array
\lstinline|hull|.

\begin{lstlisting}[mathescape=true]
pos $\leftarrow$ 1
h'$\leftarrow h^*$ / 2
while h' > 0 do
  l $\leftarrow$ deg($p_{k-1},p_k$,hull[pos-1 mod h^*]))
  c $\leftarrow$ deg($p_{k-1},p_k$,hull[pos mod h^*]))
  r $\leftarrow$ deg($p_{k-1},p_k$,hull[pos+1 mod h^*]))
  if l < c && c < r || l == r
    pos $\leftarrow$ pos + h' mod $h^*$
  else if l > c && c > r
    pos $\leftarrow$ pos - h' mod $h^*$
  h'$\leftarrow$ h' / 2
od
return pos
\end{lstlisting}

Next we have to verify the algorithm.
\begin{lemma}\label{alge:ueb2:logh}
  The above algorithm maximizes the angle and can be
  computed in $O(\log \, h^*)$ time.\\
\mbox{}\hfill$\lrcorner$
\end{lemma}

\textbf{Proof \ref{alge:ueb2:logh}:}\\
The running time of the algorithm is obviously $O(\log \, h^*)$.
We start with $\frac{h^*}{2}$. In the while loop we only
compute for constant time $c$. We decrease $h'$ until we hit zero.
$$
\Rightarrow T(n) = log h^*.
$$

We know that $\overset{\infty}{\underset{i=1}{\sum \frac{h^*}{2}}} = h^*$
and if our steps are discrete we can reach the point in $log h^*$ time
(Binary Search).
Let $R_i$ be the set of reachable points on the hull in step $i$
of the algorithm and $q$ the optimal point. Then
$$
	\forall i \in \mathbb{N} : q \in R_i, |R_i| = \frac{h^*}{2^i}
$$
Inducion on $i$.
\begin{description}
	\item[\bfseries I.A.] $i=1$\\
	In the first step we can do at most 
	$\overset{\infty}{\underset{i=1}{\sum}} \frac{h^*}{2^i} = h^*$
	steps. Therefore all points are reachable espacially $q$.
	\item[\bfseries I.S.] $i \leadsto i+1$\\
	We now in $q \in R_i$. On the last step we were
	according to the algorithm at one endpoint and we were on the
	edge of $R_i$. In the algorithm we devide \lstinline|h'| by two,
	therefore we are in the middle of the set according to the given order
	because it had size \lstinline|h'| $= |R_i|$.
	We divide $R_i$ into two subsets $R_{i+1}^l$ and $R_{i+1}^r$.
	If the first case is true, we know $q \in R_{i+1}^l$ because
	the hull was convex therefore we can only decrease the angle again
	on the otherside of the optimum. We take $R_{i+1}=R_{i+1}^l$
	as the algorithm says. The othercase is equivalent with $R_{i+1}^r$.\\
	$|R_{i+1}| = |\frac{R_i}{2}|$.
\end{description}
BUILD UP MORE!!
\mbox{}\hfill$\square$



%%%%%%%%%%%%%%%%%%%%%%%%
%% Aufgabe 3
%%%%%%%%%%%%%%%%%%%%%%%%
\item[Problem 3] Chan's algorithm and superexponential search
  \begin{enumerate}
    \item Show that the first $h^*$ points of the convex hull
	can be computed in $O(n)$ time given the $r$ subconvex hulls.\\
    \textbf{Solution:}\\
	tbd
    \item Show that by $2^{2^i}$ as the iteration for $h^*$ we obtain
	a total running time of $O(n \log h)$.\\
    \textbf{Solution:}\\
	tbd
  \end{enumerate}

\end{description}

\label{LastPage}

\end{document}
