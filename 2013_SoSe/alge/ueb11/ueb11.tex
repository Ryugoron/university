\documentclass[11pt,a4paper,ngerman]{article}
\usepackage[bottom=2.5cm,top=2.5cm]{geometry} 
\usepackage{babel}
\usepackage[utf8]{inputenc} 
\usepackage[T1]{fontenc} 
\usepackage{ae} 
\usepackage{amssymb} 
\usepackage{amsmath}
\usepackage{amsthm} 
\usepackage{graphicx}
\usepackage{fancyhdr}
\usepackage{fancyref}
\usepackage{listings}
\usepackage{xcolor}
\usepackage{paralist}

\usepackage[pdftex, bookmarks=false, pdfstartview={FitH}, linkbordercolor=white]{hyperref}
\usepackage{fancyhdr}
\pagestyle{fancy}
\fancyhead[C]{Algorithmische Geometrie}
\fancyhead[L]{Aufgabenblatt 101}
\fancyhead[R]{SoSe 2013}
\fancyfoot{}
\fancyfoot[L]{}
\fancyfoot[C]{\thepage \hspace{1px} of \pageref{LastPage}}
\renewcommand{\footrulewidth}{0.5pt}
\renewcommand{\headrulewidth}{0.5pt}
\setlength{\parindent}{0pt} 
\setlength{\headheight}{0pt}

\date{}
\title{Aufgabenblatt 11}
\author{Max Wisniewski, Alexander Steen}


%%
%% Enviroments for proofs and lemmas
%%
\newtheorem{lemma}{\bfseries Behauptung}

\begin{document}

\lstset{language=Pascal, basicstyle=\ttfamily\fontsize{10pt}{10pt}\selectfont\upshape, commentstyle=\rmfamily\slshape, keywordstyle=\rmfamily\bfseries, breaklines=true, frame=single, xleftmargin=3mm, xrightmargin=3mm, tabsize=2, mathescape=true}

\renewcommand{\figurename}{Figur}

\maketitle
\thispagestyle{fancy}

%%%%%%%%%%%%%%%%%%%%%%%%
%% Aufgabe 1 
%%%%%%%%%%%%%%%%%%%%%%%%
\subsection*{Aufgabe 1}
Sei $\left( a_n \right)_{n \in \mathbb{N}}$ eine Folge gegeben durch $a_n = (1-\frac{1}{n})^n$ für $n \geq 2$. \\

\begin{enumerate}[(1)]
\item $\left( a_n \right)$ ist monoton steigend \\
z.Z. $a_{n+1} \geq a_n$, $n \in \mathbb{N}, n \geq 2$ \\
Sei $n \in \mathbb{N}$. Dann ist
\begin{equation*}\begin{split}
\frac{a_{n+1}}{a_n} &= \frac{(1-\frac{1}{n+1})^{n+1}}{(1-\frac{1}{n})^n} \\
  &= \frac{ (\frac{n}{n+1})^{n+1} }{(\frac{n-1}{n})^n } 
  = \left(\frac{n}{n+1}\right)^{n+1} \left(\frac{n}{n-1}\right)^n  \\
  &= \left(\frac{n}{n+1}\right)^{n+1} \left(\frac{n}{n-1}\right)^{n+1} \frac{n-1}{n} \\
  &= \left(\frac{n^2}{(n+1)(n-1)}\right)^{n+1} \frac{n-1}{n} \\
%  &= (\frac{n^2}{(n^2-1)})^{n+1} \frac{n-1}{n} \\
  &= \left(1+\frac{1}{n^2-1}\right)^{n+1} \frac{n-1}{n} \\
  &\stackrel{Bernoulli}{\geq} \frac{n-1}{n}  \left(1 + \frac{n+1}{n^2-1}\right) \\
  &= \frac{n-1}{n}  + \frac{1}{n} = 1
\end{split}\end{equation*}
und daraus folgt die Behauptung.

\item $\left( a_n \right)$ ist nach oben beschränkt
Sei $n \in \mathbb{N}, n \geq 2$. \\
Dann ist 
\begin{equation*}\begin{split}
a_n &= \left(1 - \frac{1}{n}\right)^n \stackrel{*}{\leq} 1
\end{split}\end{equation*}
\end{enumerate}

(*) gilt, da $1- \frac{1}{n} \leq 1$.

%%%%%%%%%%%%%%%%%%%%%%%%
%% Aufgabe 2
%%%%%%%%%%%%%%%%%%%%%%%%
\subsection*{Aufgabe 2}

%%%%%%%%%%%%%%%%%%%%%%%%
%% Aufgabe 3
%%%%%%%%%%%%%%%%%%%%%%%%
\subsection*{Aufgabe 3}

\label{LastPage}
\end{document}
