\documentclass[11pt,a4paper,ngerman]{article}
\usepackage[bottom=2.5cm,top=2.5cm]{geometry} 
\usepackage{babel}
\usepackage[utf8]{inputenc} 
\usepackage[T1]{fontenc} 
\usepackage{ae} 
\usepackage{amssymb} 
\usepackage{amsmath}
\usepackage{amsthm} 
\usepackage{graphicx}
\usepackage{fancyhdr}
\usepackage{fancyref}
\usepackage{listings}
\usepackage{xcolor}
\usepackage{paralist}

\usepackage[pdftex, bookmarks=false, pdfstartview={FitH}, linkbordercolor=white]{hyperref}
\usepackage{fancyhdr}
\pagestyle{fancy}
\fancyhead[C]{Algorithmische Geometrie}
\fancyhead[L]{Aufgabenblatt 101}
\fancyhead[R]{SoSe 2013}
\fancyfoot{}
\fancyfoot[L]{}
\fancyfoot[C]{\thepage \hspace{1px} of \pageref{LastPage}}
\renewcommand{\footrulewidth}{0.5pt}
\renewcommand{\headrulewidth}{0.5pt}
\setlength{\parindent}{0pt} 
\setlength{\headheight}{0pt}

\date{}
\title{Aufgabenblatt 11}
\author{Max Wisniewski, Alexander Steen}


%%
%% Enviroments for proofs and lemmas
%%
\newtheorem{lemma}{\bfseries Behauptung}

\begin{document}

\lstset{language=Pascal, basicstyle=\ttfamily\fontsize{10pt}{10pt}\selectfont\upshape, commentstyle=\rmfamily\slshape, keywordstyle=\rmfamily\bfseries, breaklines=true, frame=single, xleftmargin=3mm, xrightmargin=3mm, tabsize=2, mathescape=true}

\renewcommand{\figurename}{Figur}

\maketitle
\thispagestyle{fancy}

%%%%%%%%%%%%%%%%%%%%%%%%
%% Aufgabe 1 
%%%%%%%%%%%%%%%%%%%%%%%%
\subsection*{Aufgabe 1}
Sei $\left( a_n \right)_{n \in \mathbb{N}}$ eine Folge gegeben durch $a_n = (1-\frac{1}{n})^n$ für $n \geq 2$. \\

\begin{enumerate}[(1)]
\item $\left( a_n \right)$ ist monoton steigend \\
z.Z. $a_{n+1} \geq a_n$, $n \in \mathbb{N}, n \geq 2$ \\
Sei $n \in \mathbb{N}$. Dann ist
\begin{equation*}\begin{split}
\frac{a_{n+1}}{a_n} &= \frac{(1-\frac{1}{n+1})^{n+1}}{(1-\frac{1}{n})^n} \\
  &= \frac{ (\frac{n}{n+1})^{n+1} }{(\frac{n-1}{n})^n } 
  = \left(\frac{n}{n+1}\right)^{n+1} \left(\frac{n}{n-1}\right)^n  \\
  &= \left(\frac{n}{n+1}\right)^{n+1} \left(\frac{n}{n-1}\right)^{n+1} \frac{n-1}{n} \\
  &= \left(\frac{n^2}{(n+1)(n-1)}\right)^{n+1} \frac{n-1}{n} \\
%  &= (\frac{n^2}{(n^2-1)})^{n+1} \frac{n-1}{n} \\
  &= \left(1+\frac{1}{n^2-1}\right)^{n+1} \frac{n-1}{n} \\
  &\stackrel{Bernoulli}{\geq} \frac{n-1}{n}  \left(1 + \frac{n+1}{n^2-1}\right) \\
  &= \frac{n-1}{n}  + \frac{1}{n} = 1
\end{split}\end{equation*}
und daraus folgt die Behauptung.

\item $\left( a_n \right)$ ist nach oben beschränkt
Sei $n \in \mathbb{N}, n \geq 2$. \\
Dann ist 
\begin{equation*}\begin{split}
a_n &= \left(1 - \frac{1}{n}\right)^n \stackrel{*}{\leq} 1
\end{split}\end{equation*}
\end{enumerate}

(*) gilt, da $1- \frac{1}{n} \leq 1$.

%%%%%%%%%%%%%%%%%%%%%%%%
%% Aufgabe 2
%%%%%%%%%%%%%%%%%%%%%%%%
\subsection*{Aufgabe 2}

Beschreiben Sie einen 2D randomzied incremental algorithm um die Konvexe Hülle zu berechnen. Gehen Sie vom 3D Algortihmus aus und vereinfachen Sie ihn so weit es geht.\\

\textbf{Lösung:}

Wir sehen uns als erstes an, wie die Konfliktmengen und die Konfliktzeiger aussehen.\\

Wie in Übung 8 gezeigt, können wir in $O(\log n)$ Zeit bestimmen, ob wir in
der bisherigen Konvexen Hülle liegen. Dieser
Ansatz brachte uns insbesondere eine Kante der Hülle, bei
der wir außen lagen. Diese beiden Punkte sind
nun unsere Konfliktzeiger. Da wir nun eine Konvexe Hülle haben und
einen Punkt außerhalb, wenden wir nun den Algorithmus von Übung 2 (b) an.
(Falls wir ihn nicht so hatten: Nimm einen Punkte $q$ auf der Konvexen Hülle
und ziehe eine Gerade vom aktuellen Punkt $p$. Suche nun den anderen
Schnittpunkt außer $q$ von $\overline{pq}$ mit der konvexen Hülle.
Dazu spannen wir ein Dreieck auf. Startdreieck ist $q+1,q-1$ und $p+\frac{n}{2}$ der Gegenüberliegende Punkt. Die gerade wird durch $q+1,q-1$ gehen, sonst
war $q$ ein Extrempunkt und wir konnten ihn schon sehen, und wird
daher das Dreieck $q+1,q-1,q+\frac{n}{2}$ wieder verlassen. Wir nehmen die Verlassende Seite und nehmen den Mittelpunkt der Endpunkte iterativ, bis wir
die Kante gefunden haben an der $\overline{pq}$ das Dreieck verlässt. Dies
dauert $O(\log n)$ Zeit, da wir jedesmal den Suchbereich halbieren können und
pro Schritt müsssen wir nur 2 Seiten auf einen Schnitt mit einer Geraden testen.)


Wir setzen einen Punkt der gerade gefundenen Kante als Referenzpunkt fest
und ziehen eine linie zum neuen Punkt. Nun suchen wir die beiden Winkelmaximierenden Punkt links und recht zu dieser Strecke an die Konvexe Hülle.\\

Dies geht, wie wir in Übung 2 gezeigt haben in $O(\log n)$ Zeit. Da die
Konvexe Hülle ein Geradenzug ist, sind auch alle betroffenen Konfliktmengen
zwischen den beiden Gefundenen Punkten enthalten und wir müssen diese nicht
explizit suchen.\\

Wie im 3D Algorithmus verbinden wir nun die beiden Endpunkte (Winkelmaximierenden) mit dem neu einzufügendem Punkt und lassen alle alten Geradenstücke
zwischen den Maximierenden Punkten fallen, da sie im Dreieck der Maximierenden
und des neuen liegen.\\

\begin{lstlisting}[frame=single]
ConvexHull(P Points)
    Hull <- P(1), P(2), P(3)    // ccw order
    for i = 4 to n
        p <- P(i)
        q, r, test <- inside(p, Hull)
        if test
            w1 <- maximizeAngle(r,p)
            w2 <- minimizeAngle(r,p)
            for j = w1+1; j<w2; j++
                Delete(Hull,j)
            Insert(w1,p)
            Insert(p,w2)
\end{lstlisting}

Der Algorithmus ist korrekt, da wir wie in Chen's Algorithmus mit
den Tangenten an die Hülle vom letzten Schritt heran gehen und wir dort
bewiesen haben, da wir dann wieder etwas Konvexes erhalten. Somit ist
der Algorithmus korrekt.\\

Nun erschaffen wir pro Schritt höchstens zwei Seiten und damit
werden wir maximal $2n$ Kanten erzeugen und auch höchstens $2n-3$ Kanten
wieder löschen.
Davon abgesehen braucht jede der $n$ Runden $O(\log n)$ Zeit.

Der Algorithmus hat also eine Laufzeit von $O(n \log n)$ und aufgrund
der besseren Sucheigenschaften auf der Konvexen Hülle müssen wir noch nicht
einmal die punkte vorher permutieren.

%%%%%%%%%%%%%%%%%%%%%%%%
%% Aufgabe 3
%%%%%%%%%%%%%%%%%%%%%%%%
\subsection*{Aufgabe 3}

Ein \emph{Platonischer Körper} ist ein Konvexes Polytop im $\mathbb{R}^3$, so
dass jeder Vertex mit genau $a$ Kanten verbunden ist und jede Seitenfläche
von genau $b$ Kanten begrenzt ist.

Zeigen Sie, dass es höchstens 5 verschiedene Platonische Körper gibt.\\

\textbf{Lösung:}\\

Eulers Formel besagt
$$
    n - m + f = 2.
$$
Nun wissen wir, dass für einen Platonischen Körper über doppeltes Abzählen
gilt
$$ \begin{array}{rcl}
    a*n &=& 2m\\
    b*f &=& 2m.    
\end{array}$$
Da $a,b$ nicht $0$ sein können, da der Körper sonst nicht zusammenhängt,
können wir die Formeln nach $n,f$ umstellen und in Eulers Formel einsetzten.

$$\begin{array}{lrcl}
    &\frac{2m}{a} - m + \frac{2m}{b} &=& 2\\
\Leftrightarrow &
    \frac{2bm -abm + 2am}{ab} &=& 2\\
\Leftrightarrow &
    2bm - abm + 2am &=& 2ab\\
\Leftrightarrow &
    2m( a + b ) - ab (m + 2) &=& 0
\end{array}$$
\label{LastPage}
\end{document}
