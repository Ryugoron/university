\documentclass[11pt,a4paper,ngerman]{article}
\usepackage[bottom=2.5cm,top=2.5cm]{geometry} 
\usepackage{babel}
\usepackage[utf8]{inputenc} 
\usepackage[T1]{fontenc} 
\usepackage{ae} 
\usepackage{amssymb} 
\usepackage{amsmath}
\usepackage{amsthm} 
\usepackage{graphicx}
\usepackage{fancyhdr}
\usepackage{fancyref}
\usepackage{listings}
\usepackage{xcolor}
\usepackage{paralist}

\usepackage[pdftex, bookmarks=false, pdfstartview={FitH}, linkbordercolor=white]{hyperref}
\usepackage{fancyhdr}
\pagestyle{fancy}
\fancyhead[C]{Algorithmische Geometrie}
\fancyhead[L]{Aufgabenblatt 12}
\fancyhead[R]{SoSe 2013}
\fancyfoot{}
\fancyfoot[L]{}
\fancyfoot[C]{\thepage \hspace{1px} of \pageref{LastPage}}
\renewcommand{\footrulewidth}{0.5pt}
\renewcommand{\headrulewidth}{0.5pt}
\setlength{\parindent}{0pt} 
\setlength{\headheight}{0pt}

\date{}
\title{Aufgabenblatt 12}
\author{Max Wisniewski, Alexander Steen}


%%
%% Enviroments for proofs and lemmas
%%
\newtheorem{lemma}{\bfseries Behauptung}

\begin{document}

\lstset{language=Pascal, basicstyle=\ttfamily\fontsize{10pt}{10pt}\selectfont\upshape, commentstyle=\rmfamily\slshape, keywordstyle=\rmfamily\bfseries, breaklines=true, frame=single, xleftmargin=3mm, xrightmargin=3mm, tabsize=2, mathescape=true}

\renewcommand{\figurename}{Figur}

\maketitle
\thispagestyle{fancy}

%%%%%%%%%%%%%%%%%%%%%%%%
%% Aufgabe 1 
%%%%%%%%%%%%%%%%%%%%%%%%
\subsection*{Aufgabe 1}

Beweisen Sie die Formeln
$$
    \sum^{m}_{i=1} a_i (b_{i+1} - b_i) = a_{m+1}b_{m+1} - a_1b_1 - \sum^{m}_{i=1} b_{i+1} (a_{i+1} - a_i)
$$
und folgern Sie daraus
$$
    \sum^{n-4}_{k=1} \frac{6}{k^2} (|L_{\leq k}| - |L_{\leq (k-1)}|) = \frac{6}{(n-3)^3}|L_{\leq(n-4)} - 6 |L_0| + \sum^{n-4}_{k=1} |L_{\leq k}|(\frac{6}{k^2}-\frac{6}{(k+1)^2}
$$

\textbf{Beweis:}\\

Zum ersten Teil
\begin{equation*}\begin{array}{rcl}
    a_{m+1}b_{m+1} - a_1b_1 - \overset{m}{\underset{i=1}{\sum}} b_{i+1}(a_{i+1} - a_i)
        &=& a_{m+1}b_{m+1} - a_1b_1 - \overset{m}{\underset{i=1}{\sum}} b_{i+1}a_{i+1} + \overset{m}{\underset{i=1}{\sum}} b_{i+1}a_i\\
        &=& -a_1b_1 - \overset{m-1}{\underset{i=1}{\sum}} b_{i+1}a_{i+1} + \overset{m}{\underset{i=1}{\sum}} b_{i+1}a_i\\
        &=& -a_1b_1 - \overset{m}{\underset{i=2}{\sum}} b_ia_i + \overset{m}{\underset{i=1}{\sum}} b_{i+1}a_i\\
        &=& - \overset{m}{\underset{i=1}{\sum}} b_ia_i + \overset{m}{\underset{i=1}{\sum}} b_{i+1}a_i\\
        &=& \overset{m}{\underset{i=1}{\sum}} (- b_ia_i + b_{i+1}a_i)\\
        &=& \overset{m}{\underset{i=1}{\sum}} a_i(b_{i+1} - b_i)
\end{array}\end{equation*}

Dies wenden wir das ganze auf den zweiten Teil an:
\begin{equation*}\begin{array}{rcl}
    \overset{n-4}{\underset{k=1}{\sum}} \frac{6}{k^2} (|L_{\leq k}| - |L_{\leq (k-1)}|)
        &\stackrel{\text{sumParts}}{=}&
            \frac{6}{(n-3)^2)}|L_{\leq (n-4)}| - \frac{6}{1^2}|L_{\leq 0}| - \overset{n-4}{\underset{k=1}{\sum}}|L_{\leq k}|\left( \frac{6}{(k+1)^2} - \frac{6}{k^2} \right)\\
        &=& \frac{6}{(n-3)^2)}|L_{\leq (n-4)}| - 6|L_0| + \overset{n+4}{\underset{k=1}{\sum}}|L_{\leq k}| \left( \frac{6}{k^2} - \frac{6}{(k+1)^2}\right)
\end{array}\end{equation*}

\mbox{}\hfill$\square$

\subsection*{Aufgabe 2}


Sei $\mathcal{P} \subseteq \mathbb{R}^3$ ein Simplicial Polytop gegeben durch eine DCEL. Beschreiben Sie eine
Anfragedatenstrucktur, die in $O(\log n)$ erwarteter Laufzeit entscheidet, ob ein gegebener Punkt $q \in \mathbb{R}^3$
innerhalb von $\mathcal{P}$ liegt.\

\textbf{Beweis:}\\

Wir nutzen die Konstruktion aus der Vorlesung um das Polytop auf eine Ebene zu projezieren.\\
Dazu legen wir zunächst eine Kugel $B_r(x_0)$ um das Polytop, so dass $\mathcal{P} \subset B_r(x_0)$ und $x_0 \in \mathcal{P}$
und projezieren von $x_0$ aus alle Ecken und Kanten auf die Kugel $B_r$.

Danach projezieren wir vom Nordpol der Kugel alle Ecken und Kanten auf eine Tangentialebene am Südpol. (Wir nehmen an, dass
nicht am Nord- und Südpol ein Punkte liegt).\\

Auf dieser Projektion können wir nun eine planare point-location ausführen. Dazu stehen uns Trapezoidal Maps und
eine Sequence von Triangulierungen (wie in der letzten Vorlesung gezeigt wurde) zur Verfügung.\\

Da ein Simplicial Polytop schon Trianguliert ist, entscheiden wir uns für die zweite Datenstrucktur. Diese hat
$O(n)$ Vorverarbeitungszeit und $O(n)$ Speicherbedarf.\\

Wollen wir nun entscheiden, ob ein Punkt $q \in \mathbb{R}^3$ in $\mathcal{P}$ liegt, testen wir zunächst, ob $q \in B_r(x_0)$ liegt
(konstante zeit). Ist dies nicht der Fall verwerfen wir. Sonst wenden wir beide Projektion an um den Punkt in die Ebene
zu übertragen (projektion in 3D dauern konstante Zeit) und führen die Pointlocation aus. Daraus erhalten wir in welcher Facette wir liegen.\\

\vspace{5cm}

Wie im Bild dargestellt wissen wir, dass $q$ in der Pyramide mit Grundfläche $f$ (der Facette, die wir ermittelt haben) liegt, wobei
diese potentiell über die Facette erweitert werden muss.\\
Wir müssen also nur noch einen Orientierungstest mit der Facette machen. Liegt der Punkt außerhalb, verwerfen wir und sonst akzeptieren wir.\\

Wir müssen keine andere Facette prüfen, da wir wissen, dass $q$ in der ermittelten Pyramide liegt.\\

Die Anfragezeit dauert also $O(\log n)$ determiniert. (Erwartet, falls man Trapezoidal Maps verwendet.)

\mbox{}\hfill$\square$

\subsection*{Aufgabe 3}

\subsubsection*{(a)}

\vspace{7cm}

\subsubsection*{(b)}

Rechtfertigen Sie die Annahmen aus dem dritten Schritt.\\

\textbf{Lösung:}\\

Alle Voronoi Knoten in $x \in D$ stehen im Konflikt mit $s$. Das bedeutet, dass der Kreis $C$, der $x$ erzeugt hat,
nicht leer ist. Wäre nun $D$ nicht verbunden, so gäbe es auf der Strecke von $s$ zu $x \in D$ einen Knoten $y$ , der nicht
Berührt wurde. Da aber $s \in C_x \Rightarrow y \in C_x$. Damit erfüllt auch $y$ die Empty-Circle Eigenschaft nicht und
muss Teil von $D$ sein.\\

Nachdem $D$ gelöscht wurde, gehen wir alle umgebenden Punkte ab und erstellen jeweils Bisectors zwischen $s$ und dem Vertex
zu der Zelle in $D$, die wir gerade gelöscht haben. Da $\mathcal{P}$ konvex war müssen alle so entstandenen Bisectors in $VD(S)$ liegen.
Außerhalb von $D$ galt die Empty-Circle Property noch. Innerhalb von $D$ haben wir sie gerade bestimmt. Daher gilt sie überall
und es handelt sich um ein Voronoi-Diagram.

\subsubsection*{(c)}

Zeigen Sie, wie man den dritten Schritt in $O(d)$ Zeit implementieren kann, wobei $d$ die Anzahl der Voronoi vertices der Voronoizelle von $s$ ist.\\

\textbf{Lösung:}\\

Wir führen eine Tiefensuche von der ermittelten Zelle $x$ aus und brechen ab, wenn die Empty-Circle Property erfüllt ist ($s$ liegt nicht im erschaffenen
Kreis). Das ermittelte Set $D$ ist wie in \emph{(b)} beschrieben verbunden, damit können wir so vorgehen. Nun muss jeder Knoten in
$D$ eine Kante hinzufügen (also auch höchstens zwei Knoten), da wie vorher beschrieben die Punktemenge Konvex war.

\subsubsection*{(d)}

Zeigen Sie, dass der dritte Schritt konstante Zeit benötigt.\\

\textbf{Lösung:}\\

Wir müssen ermitteln wie viele Knoten die Voronoizelle für $s$ hat. Dazu benutzen wir Backward Analysis und bestimmen
stattdessen was die erwartete Anzahl von Knoten pro Voronoizelle ist.\\

Dabei benutzen wir, dass es sich beim Voronoi Diagram um einen planaren Graphen handelt und dass die Anzahl
der Kanten demnach $\leq 3 n - 6$ ist. Dies ist wichtig, da eine Nachbarvoronoizelle durch eine Kante abgegrenzt wird.\\

Wir hatten hier keine Zeit mehr. Klausurstress.


\subsubsection*{(e)}

Folgern Sie, dass der Algorithmus erwartet $O(n)$ Zeit benötigt.\\

\textbf{Lösung:}\\

Haben wir $3$ Knoten oder weniger können wir es direkt ausrechnen.\\
Darüber fügen wir je einen Knoten hinzu und benötigen dafür $d$ Zeit, wobei wie in \emph{(d)} gezeigt konstant ist.
Demnach ist
$$
    \sum^{n-3}_{k=1} d = O(n)
$$
\mbox{} \hfill $\square$

\subsubsection*{(f)}

Ohne die konvexe Eigenschaft, wissen wir nicht in welcher Voronoizelle aus $VD(S \setminus \{s \})$ der Knoten $s$ liegt
und wir müssten somit noch eine Pointlocation ausführen, was uns wieder auf die gegebene Laufzeit bringt.

\label{LastPage}
\end{document}
