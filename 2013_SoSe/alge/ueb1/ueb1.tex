\documentclass[11pt,a4paper,ngerman]{article}
\usepackage[bottom=2.5cm,top=2.5cm]{geometry} 
\usepackage{babel}
\usepackage[utf8]{inputenc} 
\usepackage[T1]{fontenc} 
\usepackage{ae} 
\usepackage{amssymb} 
\usepackage{amsmath}
\usepackage{amsthm} 
\usepackage{graphicx}
\usepackage{fancyhdr}
\usepackage{fancyref}
\usepackage{listings}
\usepackage{xcolor}
\usepackage{paralist}

\usepackage[pdftex, bookmarks=false, pdfstartview={FitH}, linkbordercolor=white]{hyperref}
\usepackage{fancyhdr}
\pagestyle{fancy}
\fancyhead[C]{Computational Geometry}
\fancyhead[L]{Exercise sheet 1}
\fancyhead[R]{SoSe 2013}
\fancyfoot{}
\fancyfoot[L]{}
\fancyfoot[C]{\thepage \hspace{1px} of \pageref{LastPage}}
\renewcommand{\footrulewidth}{0.5pt}
\renewcommand{\headrulewidth}{0.5pt}
\setlength{\parindent}{0pt} 
\setlength{\headheight}{0pt}

\date{}
\title{Exercise sheet 1}
\author{Max Wisniewski, Alexander Steen}


%%
%% Enviroments for proofs and lemmas
%%
\newtheorem{lemma}{\bfseries Claim}

\begin{document}

\lstset{language=Pascal, basicstyle=\ttfamily\fontsize{10pt}{10pt}\selectfont\upshape, commentstyle=\rmfamily\slshape, keywordstyle=\rmfamily\bfseries, breaklines=true, frame=single, xleftmargin=3mm, xrightmargin=3mm, tabsize=2, mathescape=true}

\renewcommand{\figurename}{Figure}

\maketitle
\thispagestyle{fancy}

\begin{description}
%%%%%%%%%%%%%%%%%%%%%%%%
%% Aufgabe 1 
%%%%%%%%%%%%%%%%%%%%%%%%
\item[Problem 1] Computing the minimum
  \begin{enumerate}
    \item Let $X_i \in \{ 0,1 \}$ be the random variable that indicates whether line (*) is executed
          in the $i$-th iteration of the for-loop. Show that $E[X] = \sum_{i=2}^{n}{E[X_i]}$.\\

          Since $X = \sum_{i=2}{n}{X_i}$, it holds that
          \begin{equation}\begin{split}
            E[X] = E\left[ \sum_{i=2}^n X_i \right] \stackrel{(*)}{=} \sum_{i=2}^{n}{E[X_i]}
          \end{split}\end{equation}
          where (*) holds by linearity of the expected value.
    \item Find $E[X_i]$. \\
          
          Since the numbers are pairwise distinct, the probability that at some fixed position $k$
          $A[k]$ is minimal, is given by $Pr[A[k] \text{ minimal}] = \frac{1}{i}$, where $i$ is the number
          of distinct numbers. It follows that
          \begin{equation}\begin{split}
            E[X_i] & \stackrel{\text{Def.}}{=} \sum_{a \in \{0,1\}}{a \cdot Pr[X_i = a]}\\
                                          &\;= Pr[X_i = 1] = Pr[A[i] \text{ minimal in $i$ elements}]
                                           = \frac{1}{i}
          \end{split}\end{equation}
    \item Conclude that $E[X] = O(\log n)$.
          \begin{equation}\begin{split}
            E[X] &\stackrel{(1)}{=} \sum_{i = 2}^{n}{E[X_i]} = \sum_{i = 2}^{n}{\frac{1}{i}} \\
                 &\leq \sum_{i = 1}^{n}{\frac{1}{i}} = H_n = O(\log n)
          \end{split}\end{equation}
  \end{enumerate}

%%%%%%%%%%%%%%%%%%%%%%%%
%% Aufgabe 2
%%%%%%%%%%%%%%%%%%%%%%%%
\item[Problem 2] Induction \\


%%%%%%%%%%%%%%%%%%%%%%%%
%% Aufgabe 3
%%%%%%%%%%%%%%%%%%%%%%%%
\item[Problem 3] O-Notation
  \begin{enumerate}
    \item $\log(n!) = \Theta(n \log(n))$\\
          (1) $\log(n!) = O(n \log(n))$ \\
              \begin{equation}\begin{split}
                \log(n!) &= \sum_{i=1}^n {\log i} 
                         \leq n \cdot \log n
              \end{split}\end{equation}
          (2) $\log(n!) = \Omega(n \log(n))$ \\
              \begin{equation}\begin{split}
                TBD
              \end{split}\end{equation}
    \item $\log(mn) = O(\log(n+m))$ holds since
      \begin{equation}\begin{split}
        \log(mn) &= \log(m) + \log(n) \leq \log(m+n) + \log(n+m)\\
                 &= 2 \log(n+m) = O(\log(n+m))
      \end{split}\end{equation}
  
    \item Let $f,g \geq 2$ and $f(n) = O(g(n))$
    \begin{enumerate}
      \item $\sqrt{f(n)} = O(\sqrt{g(n)})$
            \begin{equation}\begin{split}
              \sqrt{f(n)} &= (f(n))^{\frac{1}{2}} \leq (c\cdot g(n))^{\frac{1}{2}}\\
                          &= \sqrt{c}\cdot \sqrt{g(n)} = O(\sqrt{g(n)})
            \end{split}\end{equation}
      \item $2^f(n) = O(2^g(n))$
            \begin{equation}\begin{split}
              TBD
            \end{split}\end{equation}
    \end{enumerate}
  \end{enumerate}

\end{description}

\label{LastPage}

\end{document}
