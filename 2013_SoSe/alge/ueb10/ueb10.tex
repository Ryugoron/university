\documentclass[11pt,a4paper,ngerman]{article}
\usepackage[bottom=2.5cm,top=2.5cm]{geometry} 
\usepackage{babel}
\usepackage[utf8]{inputenc} 
\usepackage[T1]{fontenc} 
\usepackage{ae} 
\usepackage{amssymb} 
\usepackage{amsmath}
\usepackage{amsthm} 
\usepackage{graphicx}
\usepackage{fancyhdr}
\usepackage{fancyref}
\usepackage{listings}
\usepackage{xcolor}
\usepackage{paralist}

\usepackage[pdftex, bookmarks=false, pdfstartview={FitH}, linkbordercolor=white]{hyperref}
\usepackage{fancyhdr}
\pagestyle{fancy}
\fancyhead[C]{Algorithmische Geometrie}
\fancyhead[L]{Aufgabenblatt 10}
\fancyhead[R]{SoSe 2013}
\fancyfoot{}
\fancyfoot[L]{}
\fancyfoot[C]{\thepage \hspace{1px} of \pageref{LastPage}}
\renewcommand{\footrulewidth}{0.5pt}
\renewcommand{\headrulewidth}{0.5pt}
\setlength{\parindent}{0pt} 
\setlength{\headheight}{0pt}

\date{}
\title{Aufgabenblatt 10}
\author{Max Wisniewski, Alexander Steen}


%%
%% Enviroments for proofs and lemmas
%%
\newtheorem{lemma}{\bfseries Behauptung}

\begin{document}

\lstset{language=Pascal, basicstyle=\ttfamily\fontsize{10pt}{10pt}\selectfont\upshape, commentstyle=\rmfamily\slshape, keywordstyle=\rmfamily\bfseries, breaklines=true, frame=single, xleftmargin=3mm, xrightmargin=3mm, tabsize=2, mathescape=true}

\renewcommand{\figurename}{Figur}

\maketitle
\thispagestyle{fancy}

%%%%%%%%%%%%%%%%%%%%%%%%
%% Aufgabe 1 
%%%%%%%%%%%%%%%%%%%%%%%%
\subsection*{Aufgabe 1}

Geben Sie ein Beispiel an, dass der Term $\sqrt{n}$ in der oberen Schranke $O( \sqrt{n} + z)$ für die Anfragezeit
an zwei Dimensionale kd-Bäume nicht verbessert werden kann.\\

\textbf{Lösung:}\\

Wir nehmen uns eine Linie als Bereichsanfrage, parrallel zur x-Achse, die jeden Knoten, der nach y-Koordinate
fragt und im richtigen x-Abschnitt liegt durchschneidet.

\vspace{5cm}

Pro x-Knoten machen wir einen Vergelich und gehen in nur einen der beiden Teilbäume. Diesen Schritt vernachlässigen wir
(oder fassen in den y-Knoten zusammen). Dort vergleichen wir und arbeiten auf beiden Bäumen weiter.
Da wir durch 2 Knoten durch sind, verlieren wir in einer x/y Runde die Hälfte der Knoten, so dass uns 
pro besuchten Teilbaum $\frac{n}{4}$ bleiben. Der Baum hat immer noch eine Höhe von $\log n$,

\begin{equation*}\begin{array}{rcl}
    T(1) &=& O(1)\\
    T(n) &=& 2 T(\frac{n}{4}) + O(1)\\
        &=& 2^k (T\frac{n}{2^{2k}}) + \overset{k}{\underset{i=1}{\sum}} 2^i \cdot c\\
        &\stackrel{k=\frac{1}{2}\log n}{=}& 2^{\frac{\log n}{2}} c + c * \overset{\frac{\log n}{2}}{\underset{i=1}{\sum}} 2^i\\
        &\leq& c * \sqrt{n} + \sqrt{\frac{n}{2}} - 1\\
        &=& O(\sqrt{n})
\end{array}\end{equation*}

Dabei sollte bei einer einfachen linie unser Ergebniss höchstens ein Elemente enthalten also $z \leq 1$.

\subsection*{Aufgabe 2}

Sei $P \subseteq \mathbb{R}^k$ mit $|P| = n$, wobei $k$ nicht von $n$ abhängen soll.

\subsubsection*{(a)}

Geben Sie einen Algorithmus an, für die Konstruktion eines kd-Baums an, der $O(n \log n)$ Zeit und
$O(n)$ Speicher benötigt.\\

\textbf{Lösung:}\\

Der Algorithmus nimmt eine Liste von Zahlen entgegen, $i$ als die Dimension,
nach der in der aktuellen Phase geteilt werden soll und $k$ als die Dimension von $P$.\\
Wir wissen, dass einen linearzeit Algorithmus \lstinline|median| gibt,
der zu einer Liste von Zahlen den Median berechnet. Wir wollen den Median der $i$ ten
Komponente und setzen den Algorithmus vorraus.

\begin{lstlisting}[frame=single, mathescape = true]
kdTree(list, i, k)
    if |list| = 0 
        return Nil
    m = median(list, i)
    l = $\{x \in list \, | \, x(i) < m(i) \}$
    r = $\{x \in list \, | \, x(i) > m(i) \}$
    
    return Node (kdTree (l, (i+1) % k), (m,i) , kdTree (r, (i+1) % k))
\end{lstlisting}

Jedes Element kann höchstens einmal Median sein (da wir es danach löschen) und damit hat der
Baum höchstens $n$ Knoten, was einer linearen Größe entspricht.\\

Die Laufzeit lässt sich beschreiben durch
$$
    T(n) \leq 2 \cdot T(\left\lceil \frac{n}{2} \right\rceil) + T_m(n)
$$
und da $T_m(n) = c \cdot n$ für ein $c > 0$ ist, können wir die alt bekannte Abschätzung
für Mergesort benutzen und erhalten
$$
    T(n) \leq 2 n \log n = O(n \log n).
$$
Daraus ergibt sich auch, dass die Höhe des Baumes in $O(\log n)$ ist.

\subsubsection*{(b)}

Beschreiben Sie einen Algorithmus, der die Bereichsabfrage an Ihren Baum in $O(n^{1-\frac{1}{k} + z})$ beantwortet.\\

\textbf{Lösung:}\\

Wir setzen vorraus, dass es eine Methode \lstinline|geti : Nat -> Range -> (R,R)| gibt,
die uns die Grenzen des Rechtecks bezüglich der $i$-ten Komponente gibt und in konstante Zeit läuft.

\begin{lstlisting}
query(Nil, i, k)
    return []
query(Node l (m,i) r, range)
    (lb, rb) = geti(i, query)
    erg = []
    if lb < m(i)
        erg = erg ++ query(l, range))
    if rb > m(i)
        erg = erg ++ query(r, range))
    if m $\in$ Range
        erg = erg ++ [m]

    return erg
\end{lstlisting}

Der Algorithmus geht analog zum zwei dimensionalen vor. Wenn die Bereichsabfrage in der $i$-ten Komponente kleiner wird als
der Knoten, müssen wir nach links gehen, und wenn er größer wird als der Knoten, müssen wir nach rechts.
Wenn beides zutrifft müssen wir natürlich in beide Teilbäume absteigen.

Zuletzt prüfen wir den Knoten auf alle $k$ Dimensionen, ob wir ihn nehmen können.\\

\begin{lemma}\label{alge:ueb10:kdquery}
    Der Algorithmus ist korrekt und hat die Laufzeit $O(n^{1-\frac{1}{k}} + z)$.
\end{lemma}

\textbf{Beweis \ref{alge:ueb10:kdquery}.}\\
    Nach Konstruktion befinden sich im linken Unterbaumbaum alle nur Punkte mit $x(i) < m(i)$. Sollte der
    Query als in dieser Komponente größer sein, kann das Rechteck keinen Punkt aus dem linken Unterbaum enthalten.
    Dies gilt ebenso für den rechten Unterbaum. Es ist also korrekt diesen Unterbaum nicht zu besuchen.

    Demnach müssen wir also alle Knoten besuchen, die in der Range liegen.\\

    Betrachten wir nun die Laufzeit.\\

    Wie in Teil \emph{(a)} und in der VL konstruieren wir unseren Worstcase.
    
    In jedem Schritt haben wir die Möglichkeit und entweder aufzuteilen oder nur in eine Richtung zu gehen.
    Wir wollen hierbei nun die Anzeil der aufteilungsoperationen zu maximieren. Dies erreichen wir,
    indem wir eine Dimension fixieren und uns in dieser nicht verzweigen. In allen anderen verzweigen wir uns.\\

    Wir brauchen diese eine "nicht-verzweigung", da sonst alle inneren Knoten, wie im ein-dimensionalen Fall
    komplett von unserem Query-Rechteck eingeschlossen werden und somit zum Ergebnis gehören. Somit würden
    wir also nur eine Laufzeit von $O(\log n + z)$ erreichen.\\

    Lassen wir aber das Rechteck noch je von einer Hyperebene schneiden, so besteht bis zum Schluss die 
    Möglichkeit, dass ein Element nicht zum Ergebnis gehört.\\

    Wie in \emph{(a)} gehen wir durch $k$ Dimensionen, wobei wir immer die Anzahl der Elemente halbieren.
    Da wir in einer Dimension verzweigen verlieren wir bei einer Runde vom Durchlaufen aller Dimensionen
    je die Hälfte der Elemente.
    $$\begin{array}{rcl}
        T(1) &=& 0\\
        T(n) &\leq& 2^{k-1} \cdot T(\frac{n}{2^{k}}) + O(1)\\
            &=& \left( 2^{k-1} \right)^i \cdot T(\frac{n}{\left( 2^k \right)^i}) + c \cdot \overset{i}{\underset{j=1}{\sum}} \left(2^{k-1}\right)^j\\
            &\stackrel{i=(\log n)/k}{=}& c \cdot \overset{(\log n)/k}{\underset{j=1}{\sum}} \left(2^{k-1} \right)^j\\
            &=& c \cdot \left( \left( 2^{k-1} \right)^{(\log n)/k} - 1\right)\\
            &=& c \cdot \left( \left( 2^{\log n} \right)^{(k-1)/k} - 1 \right)\\
            &=& c n^{1 - \frac{1}{k}} - c\\
            &=& O(n^{1 - \frac{1}{k}})
    \end{array}$$

    Wie schon gesagt, können wir nicht mehr Verzweigungen erreichen, da sonst der innere Teil komplett im Rechteck liegt.
    Sollten wir weniger verzweigungen haben, so wird der Rekursionschritt nur kleiner, da wir weniger Teile betrachten müssen.
    Damit ist $O(n^{1-\frac{1}{k}})$ eine obere Schranke und wir wissen, dass diese Schranke fest ist, da wir wie in \emph{(a)} das
    Beispiel konstruieren können, so dass unsere Abschätzung mit Gleichheit erfüllt ist.

    \mbox{}\hfill$\square$
\subsection*{Aufgabe 3}

Sei $P \subseteq \mathbb{R}^k$. Zeigen Sie, wie man \emph{Range Trees} auf $k$ dimensionen erweitert und wie man
für die Anfrage $R$ die Größe $P \cap R$ berechnet, so dass die Anfrage Zeit in $O(\log^k n)$ liegt.\\

\textbf{Lösung:}\\

In der zweidimensionalen Fassung wurde ein balancierter Suchbaum, nach der ersten Komponente erstellt und
in jedem Knoten ein Array gehalten, mit der sortierten Liste für die zweite Komponente.\\

Dies erweitern wir nun indem wir einen RangeTree(i) definieren. Dieser
ist ein balancierter Suchbaum nach der $i$-ten Komponente und speichert in jedem Knoten einen
RaneTree($i-1$), statt einem Array.\\

Wir wissen, dass man RangeTree(2) mit Platz $O(n \log n)$ speichern kann und in Zeit $O(n \log n)$ berechnen kann.\\

Wir erstellen einen Range wie folgt Rekursiv, wobei $i=2$ der bekannte RangeTree aus der VL ist.
\begin{enumerate}[1.]
    \item Sei \emph{list} die aktuelle List und wir sortieren nach der $i$ den Komponente.
    \item Erstelle einen $u = $ RangeTree($i-1$) aus \emph{list}.
    \item Berechne den Median $m$ bezüglich der $i$ten Komponente von \emph{list}.
    \item Berechne \lstinline|l = RangeTree(i)| begzüglich $\{ x \in list \, | \, x(i) \leq m(i) \}$
        und \lstinline|r = RangeTree(i)| bezüglich $\{x \in list \, | \, x(i) > m(i) \}$.
    \item Gebe den Knoten zurück mit $u$ als Wert und $l,r$ als Unterbäumen.
\end{enumerate}

Es gilt offensichtlich, dass ein RangeTree einer Partition von $P$ in der Summe höchstens so groß ist
wie der RangeTree von gesamt $P$ und das selbe gilt für die Laufzeit.

Für den Platz gilt also $O(n \log^{k-1} n)$.\\
\textbf{Beweis:}\\
I.A. Für $k=2$ wissen wir, nimmt der Baum die Größe $O(n \log n)$ an.\\
I.S. $k \leadsto k+1$\\
    Pro Ebene ist der RangeTree eine Partition von $P$ mit RangeTrees(k). Diese
    sind nach höchstens so große wie ein RangeTree(k) über gesammt $P$.
    Und da der Baum die Höhe $O(\log n)$ (Da es ein binärer Suchbaum in einer Dimension ist),
    folgt
    $$
        S_{k+1}(n) \leq c \log n * S_{k}(n) = c' \log n * (n \log^{k-1} n) = O(n \log^k n)
    $$
\mbox{}\hfill$\square$

Die Vorverarbeitungszeit ist $O(n \log^{k-1} n)$.\\
\textbf{Beweis:}\\
    Der Beweis läuft analog zum Platz beweis, da wir wie beim platz die subadditivität für
    die Partition benutzen können.
    Dabei können wir pro Knoten das aufteilen und die Median bestimmung vernachlässigen,
    da beides linear ist aber die Verarbeitungszeit für den RangeTree(k-1) mindestens $O(n \log n)$
    dauert.\\
\mbox{}\hfill$\square$

Nun führen wir eine Bereichsabfrage wie folgt durch:
\begin{enumerate}[a.]
    \item Falls $k=1$ sind wir in einem Array. Wir machen eine Binäre Suche nach dem kleinsten
        und dem Größten Element, das noch in der Range liegt, wie in der Vl und geben
        die Differenz der Indizes zurück.
    item Falls $k>1$
\begin{enumerate}[1.]
    \item Gehe den Baum hinunter, bis wir einen Knoten gefunden haben, für den $m(k) \in query(k)$ gilt,
        also der Knoten liegt in der $k$ten Komponente im Suchinterval.
    \item Laufe den linken Baum hinunter. Gehen wir nach links, führen wir eine Bereichsabfrage auf
        den RangeTree(k-1) zur rechten aus und mergen das Ergebnis mit der Rekursion aus dem linken Unterbaum.\\
        Sind wir in einem Blatt, prüfen wir, ob der einzelne Knoten im query liegt und geben das Ergebnis zurück.
    \item Ebenso verfahren wir mit dem rechten Baum spiegelverkehrt.
    \item Wir addieren beide Ergebnisse.
\end{enumerate}
\end{enumerate}

Der Algorithmus läuft analog zum Algorithmus aus der Vorlesung, nur dass wir beim links gehen den rechten Knoten
nicht als als Array betrachten, sondern wieder als RangeTree.

\begin{lemma}\label{alge:ueb10:rangetreecor}
    Der Algorithmus ist korrekt und läuft in Zeit $O(\log^{k} n)$.
\end{lemma}

\textbf{Beweis \ref{alge:ueb10:rangetreecor}.}\\

Wir müssen den Pfad bis zum \emph{Split} Knoten nicht betrachten, da die Werte nicht innerhalb der Range liegen.\\
Haben wir den Split Knoten erreicht und sind o.B.d.A. auf dem linken Pfad. Gehen wir nach rechts,
so ist der linke Unterbaum nicht im query enthalten, da er nach der $k$ten Komponente außerhalb lag.

Gehen wir allerdings nach links weiter, so liegen alle Knoten im rechten Unterbaum im Intervall ter $k$ten Komponente der
Suchanfrage, da wir links vom \emph{Split} Knoten sind , der kleiner als die obere Grenze der $k$ten Komponente war, und
rechts vom aktuellen Knoten sind, der da wir nach links gehen größer als die untere Grenze der $k$ ten Komponente war.
Wir müssen hier also weiter suchen.\\
Da jede Ebene eine Partition von allen Elementen ist, können wir also keinen Wert aus $P$ übersehen.\\

Der Algorithmus läuft in $O(\log^{k} n)$. Wir wissen nach VL (ohne das $z$ da wir nur Indizes zählen und
nicht die Elemente aufsammeln müssen), dass für $k=2$ der Algorithmus
in $O(\log^2 n)$ läuft.\\

Für einen RangeTree mit $n$ Elemente der Höche $\log n$ können wir den Rückgriff auf den Algorithmus für $k-1$ höchstens
2 mal pro Ebene machen. Darüber hinaus, können wir erst auf Ebene 2 damit Anfangen, da auf Ebene 1 (also die Wurzel) die
früheste Möglichkeit für den Split Knoten ist.

Daraus ergibt sich
\begin{equation*}\begin{array}{rcl}
    T_{k+1}(n) &\leq& \overset{\log n}{\underset{i = 2}{\sum}} 2 \cdot T_k(\frac{n}{2^i})\\
        &=& \overset{\log n}{\underset{i=2}{\sum}} 2 c \cdot \log^k \frac{n}{2^i}\\
        &=& \overset{\log n}{\underset{i=2}{\sum}} 2 \cdot (\log^k n - \log 2^i)\\
        &=& \overset{\log n}{\underset{i=2}{\sum}} 2 c \log^k n - 2i\\
        &=& \left( 2c \log^k n \overset{\log n}{\underset{i=2}{\sum}} 1 \right) - 2c \overset{\log n}{\underset{i=2}{\sum}} i\\
        &\leq& 2c  \log^{k+1} n - 2c \frac{(\log n)((\log n) + 1)}{2}\\
        &=& O(\log^{k+1} n)
\end{array}\end{equation*}
\mbox{}\hfill $\square$
\label{LastPage}

\end{document}
