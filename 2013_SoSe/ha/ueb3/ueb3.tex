\documentclass[11pt,a4paper,ngerman]{article}
\usepackage[bottom=2.5cm,top=2.5cm]{geometry} 
\usepackage{babel}
\usepackage[utf8]{inputenc} 
\usepackage[T1]{fontenc} 
\usepackage{ae} 
\usepackage{amssymb} 
\usepackage{amsmath}
\usepackage{amsthm} 
\usepackage{graphicx}
\usepackage{fancyhdr}
\usepackage{fancyref}
\usepackage{listings}
\usepackage{xcolor}
\usepackage{paralist}

\usepackage[pdftex, bookmarks=false, pdfstartview={FitH}, linkbordercolor=white]{hyperref}
\usepackage{fancyhdr}
\pagestyle{fancy}
\fancyhead[C]{Höhere Algorithmik}
\fancyhead[L]{Übungsblatt Nr. 3}
\fancyhead[R]{SoSe 2013}
\fancyfoot{}
\fancyfoot[L]{}
\fancyfoot[C]{\thepage \hspace{1px} of \pageref{LastPage}}
\renewcommand{\footrulewidth}{0.5pt}
\renewcommand{\headrulewidth}{0.5pt}
\setlength{\parindent}{0pt} 
\setlength{\headheight}{0pt}

\date{}
\title{Übungsblatt Nr. 3}
\author{Max Wisniewski, Alexander Steen}


%%
%% Enviroments for proofs and lemmas
%%
\newtheorem{lemma}{\bfseries Lemma}
\newtheorem{claim}{\bfseries claim}
\newtheorem{theorem}{\bfseries Theorem}

\lstloadlanguages{Haskell}
\lstset{
  language=Haskell,
  basicstyle=\ttfamily,
  flexiblecolumns=false,
  basewidth={0.5em,0.45em},
  literate={+}{{$+$}}1 {/}{{$/$}}1 {*}{{$*$}}1 {=}{{$=$}}1
           {>}{{$>$}}1 {<}{{$<$}}1 {\\}{{$\lambda$}}1
           {\\\\}{{\char`\\\char`\\}}1
           {->}{{$\rightarrow$}}2 {>=}{{$\geq$}}2 {<-}{{$\leftarrow$}}2
           {<=}{{$\leq$}}2 {=>}{{$\Rightarrow$}}2 
           {\ .}{{$\circ$}}2 {\ .\ }{{$\circ$}}2
           {>>}{{>>}}2 {>>=}{{>>=}}2
           {|}{{$\mid$}}1               
}

\lstnewenvironment{code}
    {\lstset{basicstyle=\footnotesize\ttfamily}%
      \csname lst@SetFirstLabel\endcsname}
    {\csname lst@SaveFirstLabel\endcsname}
\long\def\ignore#1{}

\begin{document}

\renewcommand{\figurename}{Figure}
\maketitle
\thispagestyle{fancy}

\subsection*{Aufgabe 1}

Sei $K$ ein Körper, der eine primitive $k$-te Einheitswurzel mit $k=2^{\lceil \log (n+1) \rceil}$ enthält. Zeigen Sie, wie man effizient die Koeffizeinten eins Polynoms $P(x)$ finden kann, das eine gegebene Folge $z_0,...,z_n$ von Elementen von $K$ als Nullstellen hat und sonst keine Nullstellen. Geben Sie dazu einen Algorithmus an, der $O(n\log^2 n)$ Operationen auf $K$ benötigt.\\

\textbf{Lösung:}\\

Wir schreiben das Polynom als Multiplikation seiner Nullstellen auf.
\begin{equation}
    P(x) = \underset{i=0}{\overset{n}{\prod}} (x-z_i)
\end{equation}
Dieses Polynom hat $z_i$ als seine Nullstellen. Sei nun
$P_{i}(x) = x-z_i$ eins der Einzelpolynome. Dann ist $P(x)$ das Produkt aller $P_i(x)$.
Von $P_i(x)$ wissen wir, dass der Koeffizientenvektor $\left( \begin{array}{c}1\\-z_i\end{array}\right)$ ist.

Wir benutzen nun die schnelle Fouriertransformation um aus den $n$ Polynomen $P(x)$ zu bestimmen.

\begin{lstlisting}[language=Pascal, mathescape=true]
procedure P($P_1$, $...$ , $P_n$)
  for i := 1 to log n
      for j := 0 to n / 2^(i+1)
          $P_{j}$ := convulate $P_{2j}$ $P_{2j + 1}$
  return $P_0$
\end{lstlisting}

Wir multiplizieren je zwei Polynome miteinander um zur nächsten Ebene zu gelangen. Dies tun wir, bis wir nur noch ein Polynom
haben.
\begin{lemma}\label{ha2:ueb3:poly} (Korrektheit)
    Der Algorithmus $P$ berechnet $P(x) = \prod P_i(x)$ in der Zeit $O(n \log^2 n)$.
\end{lemma}

\textbf{Beweis \ref{ha2:ueb3:poly}:}\\
Wir multiplizieren pro Runde je zwei Polynome, wobei wir pro Runde keines doppelt multiplizieren. Das bedeutet nach $k$ Runden
haben wir noch $\frac{n}{2^k}$ Polynome übrig. Also ist, wie im Algorithmus korrekt angegeben, nach $k=\log n$ Runden nur noch
ein Polynom übrig. Nun gilt, dass in Runde $k$ im $0 \leq i \leq \frac{n}{2^k}$-ten Polynom 
\begin{equation*}
    P_i(x) = \underset{j=0}{\overset{2^k}{\prod}} P_{i\cdot 2^k+j}(x)
\end{equation*}
steht.
Induktion über $k$.
\begin{description}
    \item[\bfseries I.A.] $k = 0$.\\
        Zu Beginn steht in $P_i(x) = P_{i\cdot 2^0 + 0}(x)$.
    \item[\bfseries I.S.] $k \leadsto k+1$.\\
        Nach Induktionvoraussetzung 
        haben wir $P_{2i}(x) = \underset{j=0}{\overset{2^k}{\prod}} P_{2i\cdot 2^k + j}(x)$ und
        $P_{2i+1}(x) = \underset{j=0}{\overset{2^k}{\prod}} P_{(2i+1)\cdot 2^k + j}(x)$.
        Multiplizieren wir beide, so erhalten wir
        $P_{2i}(x) = \underset{j=0}{\overset{2^{k+1}}{\prod}} P_{2i\cdot 2^{k+1} + j}(x)$.
\end{description}

Nun wissen wir, dass wir zu Beginn Polynome in Runde $k$ vom Grad $2^k$ auf Grad $2^{k+1}$ wechseln. Da wir nun einen Körper $K$
mit primitiver $k$-ter Einheitswurzel $\omega$ mit $k=2^{\left\lceil \log (n+1)\right\rceil}$ haben, wissen wir, dass wir auf jeden Fall genug
Einheitswurzeln für jede Runde haben, um die schnelle Fouriertransformation auszuführen.
Wie gezeigt ist $\omega^2$ eine $\frac{k}{2}$-te primitive Einheitswurzel. Demnach können wir in der $i$-ten Runde $\omega^{2i}$ benutzen um eine $i+1$-te Einheitswurzel
zu erhalten (Da es sich bei $k$ um eine Zweierpotenz handelte, ist die Operation wohldefiniert für Einheitswurzeln).\\

Die Konvolution über die schnelle Fouriertransformation hat, wie in der Vorlesung gezeigt, eine Laufzeit von $O(n)$. Sei $c\cdot n$ die Laufzeit für eine Multiplikation
von twei Polynomen vom Grad $\frac{n}{2}$ (das resultierende Polynom hat Grad $n$).
Dann müssen wir in der $i$-ten Runde $\frac{n}{2^i}$ Polynome mit Grad $2^{i-1}$ multiplizieren.
\begin{equation*}\begin{split}
    T(n) &= \overset{\log n}{\underset{i=1}{\sum}} c \cdot 2^i \cdot \frac{n}{2^i}\\
        &= \overset{\log n}{\underset{i=1}{\sum}} c \cdot n\\
        &= cn \overset{\log n}{\underset{i=1}{\sum}} 1\\
        &= cn \log^2 n\\
        &= O(n \log^2 n)
\end{split}\end{equation*}
\mbox{}\hfill $\square$


\subsection*{Aufgabe 2}

Implementieren sie die dieskrete Fouriertransformation durch FFT, die inverse DFT und die Faltung.\\

\textbf{Lösung:}\\
\input{Field.lhs}
\input{FFT.lhs}

\textbf{Test:}\\
Unser letzter Test
liegt noch in den Variablen $a,b,w$ und kann mit \emph{ghci}
einfach ausgeführt werden.
\begin{lstlisting}
> ghci FFT.lhs
...
>let a = expandTo 8 (Z 0 17) [Z 2 17, Z 3 17, Z 5 17]
>let b = expandTo 8 (Z 0 17) [Z 3 17, Z 2 17, Z 7 17]
>let w = Z 2 17
> convulate w a b
[6, 13, 1, 14, 1, 0, 0, 0]
> a
[2, 3, 5, 0, 0, 0, 0, 0]
> dft w $ fft w a
[2, 3, 5, 0, 0, 0, 0, 0]
\end{lstlisting}

Wobei wir überprüft haben, dass $2$ in $\mathbb{Z}_{17}$ eine $8$te primitive Einheitswurzel ist.

\label{LastPage}

\end{document}
