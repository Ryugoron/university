\documentclass[11pt,a4paper,ngerman]{article}
\usepackage[bottom=2.5cm,top=2.5cm]{geometry} 
\usepackage{babel}
\usepackage[utf8]{inputenc} 
\usepackage[T1]{fontenc} 
\usepackage{ae} 
\usepackage{amssymb} 
\usepackage{amsmath}
\usepackage{amsthm} 
\usepackage{graphicx}
\usepackage{fancyhdr}
\usepackage{fancyref}
\usepackage{listings}
\usepackage{xcolor}
\usepackage{paralist}

\usepackage[pdftex, bookmarks=false, pdfstartview={FitH}, linkbordercolor=white]{hyperref}
\usepackage{fancyhdr}
\pagestyle{fancy}
\fancyhead[C]{Höhere Algorithmik II}
\fancyhead[L]{Übung 2}
\fancyhead[R]{SoSe 2013}
\fancyfoot{}
\fancyfoot[L]{}
\fancyfoot[C]{\thepage \hspace{1px} of \pageref{LastPage}}
\renewcommand{\footrulewidth}{0.5pt}
\renewcommand{\headrulewidth}{0.5pt}
\setlength{\parindent}{0pt} 
\setlength{\headheight}{0pt}

\date{}
\title{Übung 2}
\author{Max Wisniewski, Alexander Steen}


%%
%% Enviroments for proofs and lemmas
%%
\newtheorem{lemma}{\bfseries Claim}

\begin{document}

\lstset{language=Pascal, basicstyle=\ttfamily\fontsize{10pt}{10pt}\selectfont\upshape, commentstyle=\rmfamily\slshape, keywordstyle=\rmfamily\bfseries, breaklines=true, frame=single, xleftmargin=3mm, xrightmargin=3mm, tabsize=2, mathescape=true}

\renewcommand{\figurename}{Figure}

\maketitle
\thispagestyle{fancy}

%%%%%%%%%%%%%%%%%%%%%%%%%%%%%%%%%%%%%%%%%%%
%%%%% Aufgabe 1 %%%%%%%%%%%%%%%%%%%%%%%%%%%
\subsection*{Aufgabe 1}
Gesucht ist ein $O(M(n))$-Algorithmus zur Bestimmung des Betrags der Determinanten, wobei $M(n)$ die Anzahl
der arithmetischen Operationen für die Matrizenmultiplikation von $n \times n$-Matrizen ist. \\

\textbf{Lösung}: \\
Sei $A$ eine invertierbare $n \times n$-Matrix. Wir zerteilen die Matrix wie in der VL in folgende Teile:

\begin{equation*}
A = XYZ = \left( \begin{array}{cc}
          I & 0 \\
          A_{21}A_{11}^{-1} & I
          \end{array} \right)
          \left( \begin{array}{cc}
          A_{11} & 0 \\
          0 & D
          \end{array} \right)
          \left( \begin{array}{cc}
          I & A_{11}^{-1}A_{12} \\
          0 & I
          \end{array} \right)
\end{equation*}

wobei

\begin{equation*}
A = \left( \begin{array}{cc}
          A_{11} & A_{12} \\
          A_{21} & A_{22}
    \end{array} \right), \quad
D = A_{22} - A_{21}A_{11}^{-1}A_{12}
\end{equation*}
Wir nehmen vorerst an, dass $A_{11}$ ebenfalls invertierbar ist. Wir kommen später darauf zurück.
Für die Determinante von $A$ gilt nun:

\begin{equation*}\begin{split}
\det A &= \det \left( \begin{array}{cc}
          I & 0 \\
          A_{21}A_{11}^{-1} & I
          \end{array} \right)
          \det \left( \begin{array}{cc}
          A_{11} & 0 \\
          0 & D
          \end{array} \right)
          \det \left( \begin{array}{cc}
          I & A_{11}^{-1}A_{12} \\
          0 & I
          \end{array} \right) \\
      &= \det \left( \begin{array}{cc}
          A_{11} & 0 \\
          0 & D
          \end{array} \right)
       = \det A_{11} \det D
\end{split}\end{equation*}

Stellen wir nun die Rekursionsgleichung für $D(n)$ (Anz. Operationen für die Berechnung der Determinanten) auf und lösen sie:

\begin{equation*}\begin{split}
D(1) &= 1 \\
D(n) &= 2\cdot D\left(\frac{n}{2}\right) + 3O\left(M\left(\frac{n}{2}\right) \right) + \left(\frac{n}{2}\right)^2 + 1 \\
     &= 2\cdot D\left(\frac{n}{2}\right) + c M\left(\frac{n}{2}\right) \\
     &= \ldots \\
     &= 2^k D\left(\frac{n}{2^k} \right) + c \left(M\left(\frac{n}{2}\right)+ 2M\left(\frac{n}{4}\right) + \ldots +  2^{k-1}M\left(\frac{n}{2^k}\right) \right) \\
     &\stackrel{(*)}{=} 2^k D\left(\frac{n}{2^k} \right) + M(n) \cdot c\sum_{i=0}^{k} {(2a)^i} \\
     &\stackrel{k = \log n}{=}2^{\log n} + M(n) \cdot c' \\
     &= O(M(n))
\end{split}\end{equation*}

Umformung (*) gilt, da $M(\frac{n}{2}) \leq a\cdot M(n)$ für ein $a < 1/2$.

Die Annahme, dass $A_{11}$ invertierbar ist, können wir treffen, da wir im Zweifelsfall den Algorithmus
auf $B := A\cdot A^{t}$ anwenden, und dort die linke obere Teilmatrix immer invertierbar ist. \\
Da dann $\det B = (\det A)^2$ gilt hier $|\det A| = \sqrt{\det B}$.


%%%%%%%%%%%%%%%%%%%%%%%%%%%%%%%%%%%%%%%%%%%
%%%%% Aufgabe 2 %%%%%%%%%%%%%%%%%%%%%%%%%%%
\subsection*{Aufgabe 2}

%%%%%%%%%%%%%%%%%%%%%%%%%%%%%%%%%%%%%%%%%%%
%%%%% Aufgabe 3 %%%%%%%%%%%%%%%%%%%%%%%%%%%
\subsection*{Aufgabe 3}



\label{LastPage}
\end{document}
