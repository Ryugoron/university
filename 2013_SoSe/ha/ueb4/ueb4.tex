\documentclass[11pt,a4paper,ngerman]{article}
\usepackage[bottom=2.5cm,top=2.5cm]{geometry} 
\usepackage{babel}
\usepackage[utf8]{inputenc} 
\usepackage[T1]{fontenc} 
\usepackage{ae} 
\usepackage{amssymb} 
\usepackage{amsmath}
\usepackage{amsthm} 
\usepackage{graphicx}
\usepackage{fancyhdr}
\usepackage{fancyref}
\usepackage{listings}
\usepackage{xcolor}
\usepackage{paralist}

\usepackage[pdftex, bookmarks=false, pdfstartview={FitH}, linkbordercolor=white]{hyperref}
\usepackage{fancyhdr}
\pagestyle{fancy}
\fancyhead[C]{Höhere Algorithmik}
\fancyhead[L]{Übungsblatt Nr. 4}
\fancyhead[R]{SoSe 2013}
\fancyfoot{}
\fancyfoot[L]{}
\fancyfoot[C]{\thepage \hspace{1px} of \pageref{LastPage}}
\renewcommand{\footrulewidth}{0.5pt}
\renewcommand{\headrulewidth}{0.5pt}
\setlength{\parindent}{0pt} 
\setlength{\headheight}{0pt}

\date{}
\title{Übungsblatt Nr. 4}
\author{Max Wisniewski, Alexander Steen}


%%
%% Enviroments for proofs and lemmas
%%
\newtheorem{lemma}{\bfseries Lemma}
\newtheorem{claim}{\bfseries claim}
\newtheorem{theorem}{\bfseries Theorem}

\begin{document}

\renewcommand{\figurename}{Figure}
\maketitle
\thispagestyle{fancy}

\subsection*{Aufgabe 1}

%%%%%%%%%%%%%%%%%%%%%%%%%%%%%
%%      Aufgabe 1
%%%%%%%%%%%%%%%%%%%%%%%%%%%%
\subsubsection*{(a)}

Schreiben Sie den Pseudocode für die Schulmethode zum Berechnen der Quadratwurzel von einer Binärzahl, die den ganzzahligen Anteil berechnet.\\

\textbf{Lösung:}\\

Wir betrachten unsere eingegebene Zahl $n$ als liste von Bits $n=b_{k-1}...b_0$, wobei $k < \log n$ die Länge der Binärdarstellung ist.

\begin{lstlisting}
root([]bit b){
    akt := 0
    k := length(b)
    rest := mod k 2 = 0 ? b[k]<<1 + b[k-1] : b[k]
    k := mod k 2 = 0 ? k-2 : k-1
    do{
        if (rest < akt<<1 + 1){
            akt := akt<<1
            rest := rest<<2 + b[k]<<1 + b[k-1]
            k := k - 2
        } else {
            akt := akt<<1 + 1
            rest := (rest-akt)<<2 + b[k]<<1 +b[k-1]
            k := k - 2
        }
    }until {k < 0}
    return akt 
}
\end{lstlisting}

In dieser Darstellung ist (\lstinline|x<<k|) ein Linksshift von $x$ um $k$ Stellen, was in $k$ Schritten gehen sollte und $+1$ auf ein verschobenes
Polynom ändert nur die letzte Stelle (da diese vorher 0 war) geht also in einem Schritt.\\

\subsubsection*{(b)}

Die in \emph{(a)} gezeigt Funktion berechnet den ganzzahligen Anteil der Quadratwurzel einer Zahl $n$.\\

\textbf{Beweis:}\\

tbd

\subsubsection*{(c)}

Bestimmen Sie asymptotisch die Zahl der Binäroperationen des Algorithmuses für $n$ Bitzahlen.\\

\textbf{Lösung:}\\

tbd
%%%%%%%%%%%%%%%%%%%%%%%%%
%%      Aufgabe 2
%%%%%%%%%%%%%%%%%%%%%%%%
\subsection*{Aufgabe 2}

Bei der Newton-Iteration zuer Bestimmung der Reziproken $\frac{1}{a}$ sei der Fehler zum korrekten Wert nach $i$ Schritten $\varepsilon_i$. Geben Sie eine Abschätzung von $\varepsilon_{i+1}$ durch $\varepsilon_i$ an.

Was folgt als Bedingung für den Startwert $x_0$, damit das Verfahren konvergiert?\\

\textbf{Lösung:}\\

Wir berechnen den nächsten Wert nach Newtoniteration durch
\begin{equation}\label{ha:ueb4:newtonit}
    x_{i+1} = 2x_i - ax_i^2.
\end{equation}

Der Fehler wird beschrieben durch
\begin{equation}\label{ha:ueb4:newtonerror}
    \varepsilon_i = \left| x_i - \frac{1}{a} \right|,
\end{equation} wobei $a \not= 0$ gilt da wir nur
diese Werte invertieren können.

Stellen wir nun den aposteriori Fehlerabschätzer auf so erhalten wir
für $\varepsilon_{i+1}$.

\begin{equation}\label{ha:ueb4:aposteriori}\begin{array}{rcl}
    \varepsilon_{i+1} &\stackrel{(\ref{ha:ueb4:newtonerror})}{=}&
        \left| \frac{1}{a} - x_{i+1} \right|\\
    &\stackrel{(\ref{ha:ueb4:newtonit})}{=}&
        \left| ax_i^2 - 2x_i + \frac{1}{a} \right|\\
    &\stackrel{\text{Binomie}}{=}&
        \left| a \left( x_i - \frac{1}{a} \right)^2 \right|\\
    &=& |a| \cdot \varepsilon_i^2
\end{array}\end{equation}

Nun müssen wir untersuchen, wann $(\varepsilon_i)_{i\in \mathbb{N}}$ gegen $0$ konvergiert. 
Dazu muss $(\varepsilon_i)$ zunächst streng monoton fallen.
\begin{equation}\label{ha:ueb4:monoton} \begin{array}{lrcl}
    &\varepsilon_{i+1} &<& \varepsilon_i \\
    \stackrel{(\ref{ha:ueb4:aposteriori})}{\Leftrightarrow}&
         \left| a \right| \varepsilon_i &<& \varepsilon_i\\
    \Leftrightarrow& \varepsilon_i &<& \left| \frac{1}{a} \right|
\end{array}\end{equation}

Da es sich um eine geometrische Folge handelt, wird sie nun so gegen 0 konvergieren.
Müssen wir uns nur noch um $x_0$ kümmern, damit $\varepsilon_0$ klein genug ist.
\begin{equation}\label{ha:ueb4:start}\begin{split}
    \varepsilon_0 < \left| \frac{1}{a} \right|\\
    \Leftrightarrow \left| x_0 - \frac{1}{a} \right| < \left| \frac{1}{a} \right|\\
\end{split}\end{equation}

Da der Fehler in jedem Schritt sinkt, können wir durch eine Exponentialsuche den Startwert in $O(\log a)$ iterationen finden.

\subsection*{Aufgabe 3}

\subsubsection*{(a)}

Bestimmen Sie die Iterationsvorschrift, wenn man die Bestimmung der Quadratwurzel mittels Newton-Iteration auf Multiplikation
und Division zurück.\\

\textbf{Lösung:}\\

Wir wollen die Nullstelle der Funktion
\begin{equation}
    f(x) = x^2 - a,
\end{equation}
da in der Nulstelle
\begin{equation*}\begin{split}
    0 &= x^2 - a\\
    a &= x^2
\end{split}\end{equation*}
was genau die Definition der Wurzel ist.\\

Die Iteration ergbit die Folgende Rekursion.
\begin{equation}\begin{split}
    x_{i+1} &= x_i - \frac{f(x_i)}{f'(x_i)}\\
    \Rightarrow x_{i+1} &= x_i - \frac{x_i^2 - a}{2x_i}\\
    \Rightarrow x_{i+1} &= \frac{1}{2} (x_i + \frac{a}{x_i})
\end{split}\end{equation}

\subsubsection*{(b)}

Führen Sie die selbe Analyse wie bei \emph{Aufgabe 2} aus.\\

\textbf{Lösung:}\\

tbd

\label{LastPage}

\end{document}
