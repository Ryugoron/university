\documentclass[11pt,a4paper,ngerman]{article}
\usepackage[bottom=2.5cm,top=2.5cm]{geometry} 
\usepackage{babel}
\usepackage[utf8]{inputenc} 
\usepackage[T1]{fontenc} 
\usepackage{ae} 
\usepackage{amssymb} 
\usepackage{amsmath}
\usepackage{amsthm} 
\usepackage{graphicx}
\usepackage{fancyhdr}
\usepackage{fancyref}
\usepackage{listings}
\usepackage{xcolor}
\usepackage{paralist}

\usepackage[pdftex, bookmarks=false, pdfstartview={FitH}, linkbordercolor=white]{hyperref}
\usepackage{fancyhdr}
\pagestyle{fancy}
\fancyhead[C]{Höhere Algorithmik}
\fancyhead[L]{Übungsblatt Nr. 4}
\fancyhead[R]{SoSe 2013}
\fancyfoot{}
\fancyfoot[L]{}
\fancyfoot[C]{\thepage \hspace{1px} of \pageref{LastPage}}
\renewcommand{\footrulewidth}{0.5pt}
\renewcommand{\headrulewidth}{0.5pt}
\setlength{\parindent}{0pt} 
\setlength{\headheight}{0pt}

\date{}
\title{Übungsblatt Nr. 4}
\author{Max Wisniewski, Alexander Steen}


%%
%% Enviroments for proofs and lemmas
%%
\newtheorem{lemma}{\bfseries Lemma}
\newtheorem{claim}{\bfseries claim}
\newtheorem{theorem}{\bfseries Theorem}

\begin{document}

\renewcommand{\figurename}{Figure}
\maketitle
\thispagestyle{fancy}

\subsection*{Aufgabe 1}

%%%%%%%%%%%%%%%%%%%%%%%%%%%%%
%%      Aufgabe 1
%%%%%%%%%%%%%%%%%%%%%%%%%%%%
\subsubsection*{(a)}

Schreiben Sie den Pseudocode für die Schulmethode zum Berechnen der Quadratwurzel von einer Binärzahl, die den ganzzahligen Anteil berechnet.\\

\textbf{Lösung:}\\

Wir betrachten unsere eingegebene Zahl $n$ als liste von Bits $n=b_k...b_0$, wobei $k < \log n$ die Länge der Binärdarstellung ist.
Wir betrachten nur Zahlen mit geradzahliger Anzahl von Stellen. Sonst müsste die Zahl von mit einer führenden $0$ ergänzt werden.

\begin{lstlisting}[frame=single]
root([]bit b)
    akt := 0
    k := length(b)
    rest := 0
    while( k >= 0 )
        rest := rest<<2 + b[k]<<1 + b[k-1]
        if (rest < akt << 1 + 1)
            akt := akt << 1
        else
            akt := akt << 1 + 1
            rest := rest - akt
        k := k - 2
    return akt
\end{lstlisting}

In dieser Darstellung ist (\lstinline|x<<k|) ein Linksshift von $x$ um $k$ Stellen, was in $k$ Schritten gehen sollte und $k+1$ auf ein verschobenes
Polynom ändert nur die letzte Stelle (da diese vorher 0 war) geht also in einem Schritt. Sollte $k<0$ sein, soll der Wert von $x$ zurückgegeben werden.\\

\subsubsection*{(b)}

Die in \emph{(a)} gezeigt Funktion berechnet den ganzzahligen Anteil der Quadratwurzel einer Zahl $n$.\\

\textbf{Beweis:}\\

Wir zeigen, dass
\begin{equation}
    n = \left(\textit{akt}^2 + \textit{rest} \right) << k + b_{k} ... b_0 \land rest < akt
\end{equation}
eine Invariante des Programmes ist. Daraus folgt direkt, dass bei einer Quadratzahl
am Ende 
\begin{equation}
   n = \left(\textit{akt}^2 + 0 \right) + 0 = \textit{akt}^2 
\end{equation}
ist und in \lstinline|akt| das Ergebnis steht. Sollte $n$ keine Quadratzahl sein, so wissen wir
das in $rest$ eine Zahl steht, die kleiner ist als $akt$ und kann daher als Rest des Quadratwurzel ziehens gelten.\\

Der Beweis dieses Programmes ist nunmehr fleißarbeit, da der Hoare Kalkül und insbesondere der wlp Transformator uns automatisch
die Lösung bestimmen kann.

Der Beweiser wird am Anfang der Schleife zwei Optionen haben,
wenn $rest<<2 + b[k]<<1 + b[k-1] < akt << 1 + 1$,
dann ist
\begin{equation*}\begin{split}
    n &= ((\textit{akt} << 1)^2 + \textit{rest}<<2 + b[k]<<1 + b[k-1])<< (k-2) + b_{k-2} ... b_0\\
    &= (\textit{akt}^2 << 2 + \textit{rest}<<2 + b[k] << 1 + b[k-1]) << (k-2) + b_{k-2} ... b_0\\
    &= ((\textit{akt}^2 + \textit{rest})<<2)<<(k-2) + (b[k]<<1)<<(k-2) + (b[k-1])<<(k-2)\\
    & + b_{k-2}...b_0\\
    &= (\textit{akt}^2 + \textit{rest})<<k + b_{k}b_{k-1}b_{k-2}...b_0
\end{split}\end{equation*}
und im anderen Fall ist
\begin{equation*}\begin{split}
    n &= ((\textit{akt} << 1 + 1)^2 + (\textit{rest}-\textit{akt})<<2 + b[k]<<1 + b[k-1])<<(k-2) + b_{k-3}...b_0\\
    &= (((\textit{akt}<<1)^2 + 2(\textit{akt}<<1) + 1) + (\textit{rest}-(\textit{akt}<<1 + 1))<<2 + b[k]<<1 + b[k-1])<<(k-2)\\
    & + b_{k-3}...b_0\\
    &= (\textit{akt}^2 + \textit{rest})<<k + b_k ... b_0 + (\textit{akt}<<2 + 2 - (\textit{akt}<<3 + 4))\\
    &\stackrel{\text{Why}}{=} (\textit{akt}^2 + \textit{res})<<k + b_k ... b_0
\end{split}\end{equation*}

Der zweite Teil der Invariante ist leicht zu zeigen, da wir vom Rest immer wieder $akt$ abziehen. Daher kann, wenn $rest < akt$ vorher galt,
danach auch nur wieder $rest<akt$ gelten. Zu Begin ist die Invariante erfüllt, da $akt = rest = 0$ gesetzt ist und damit
$n = b_k ... b_0$ ist, was nach Defintion stimmt.
\mbox{} \hfill $\square$
\subsubsection*{(c)}

Bestimmen Sie asymptotisch die Zahl der Binäroperationen des Algorithmuses für $n$ Bitzahlen.\\

\textbf{Lösung:}\\

Gehen wir davon aus, das ein Shift eine Operation von konstanter Laufzeit ist, so
haben wir in Runde $i$ eine Zahl $akt$ mit $i$ Stellen. Da $rest < akt$ hat auch $rest$ weniger als $i$ Stellen.
Wir haben bei $akt$ $i$ Stellen, da wir in jeder Runde um eins shiften und hinten eine 1 anhängen. Da nach einem Shift
die letzte Stelle 0 ist, dauert die addition einer eins konstante Zeit.
Der Vergleich, ob $rest<<2 +b[k] + b[k-1] <  akt << 1 + 1$ dauert daher $i$ Schritte (lexikographisch vergleichen).
Die gesammte Laufzeit beträgt daher
\begin{equation}\begin{split}
    T(k) &= \overset{k}{\underset{i=1}{\sum}} c \cdot i\\
        &= c \overset{\frac{k}{2}}{\underset{i=1}{\sum}} i\\
        &= O (k^2)
\end{split}\end{equation}
und damit ist die Laufzeit von der schulmethode quadratisch in der Anzahl der Stellen.
%%%%%%%%%%%%%%%%%%%%%%%%%
%%      Aufgabe 2
%%%%%%%%%%%%%%%%%%%%%%%%
\subsection*{Aufgabe 2}

Bei der Newton-Iteration zuer Bestimmung der Reziproken $\frac{1}{a}$ sei der Fehler zum korrekten Wert nach $i$ Schritten $\varepsilon_i$. Geben Sie eine Abschätzung von $\varepsilon_{i+1}$ durch $\varepsilon_i$ an.

Was folgt als Bedingung für den Startwert $x_0$, damit das Verfahren konvergiert?\\

\textbf{Lösung:}\\

Wir berechnen den nächsten Wert nach Newtoniteration durch
\begin{equation}\label{ha:ueb4:newtonit}
    x_{i+1} = 2x_i - ax_i^2.
\end{equation}

Der Fehler wird beschrieben durch
\begin{equation}\label{ha:ueb4:newtonerror}
    \varepsilon_i = \left| x_i - \frac{1}{a} \right|,
\end{equation} wobei $a \not= 0$ gilt da wir nur
diese Werte invertieren können.

Stellen wir nun den aposteriori Fehlerabschätzer auf so erhalten wir
für $\varepsilon_{i+1}$.

\begin{equation}\label{ha:ueb4:aposteriori}\begin{array}{rcl}
    \varepsilon_{i+1} &\stackrel{(\ref{ha:ueb4:newtonerror})}{=}&
        \left| \frac{1}{a} - x_{i+1} \right|\\
    &\stackrel{(\ref{ha:ueb4:newtonit})}{=}&
        \left| ax_i^2 - 2x_i + \frac{1}{a} \right|\\
    &\stackrel{\text{Binomie}}{=}&
        \left| a \left( x_i - \frac{1}{a} \right)^2 \right|\\
    &=& |a| \cdot \varepsilon_i^2
\end{array}\end{equation}

Nun müssen wir untersuchen, wann $(\varepsilon_i)_{i\in \mathbb{N}}$ gegen $0$ konvergiert. 
Dazu muss $(\varepsilon_i)$ zunächst streng monoton fallen.
\begin{equation}\label{ha:ueb4:monoton} \begin{array}{lrcl}
    &\varepsilon_{i+1} &<& \varepsilon_i \\
    \stackrel{(\ref{ha:ueb4:aposteriori})}{\Leftrightarrow}&
         \left| a \right| \varepsilon_i^2 &<& \varepsilon_i\\
    \Leftrightarrow& \varepsilon_i &<& \left| \frac{1}{a} \right|
\end{array}\end{equation}

Da es sich um eine geometrische Folge handelt, wird sie nun so gegen 0 konvergieren.
Müssen wir uns nur noch um $x_0$ kümmern, damit $\varepsilon_0$ klein genug ist.
\begin{equation}\label{ha:ueb4:start}\begin{split}
    \varepsilon_0 < \left| \frac{1}{a} \right|\\
    \Leftrightarrow \left| x_0 - \frac{1}{a} \right| < \left| \frac{1}{a} \right|\\
\end{split}\end{equation}

Wir können nun mit $a$ multiplizieren und so ein leichtes Argument finden. Falls
$a>0$ ist, dann muss $\left| ax_0 - 1 \right| < 1$ und falls $a<0$ ist, dann muss
$\left| ax_0 - 1\right| > 1$ sein.

\subsection*{Aufgabe 3}

\subsubsection*{(a)}

Bestimmen Sie die Iterationsvorschrift, wenn man die Bestimmung der Quadratwurzel mittels Newton-Iteration auf Multiplikation
und Division zurück.\\

\textbf{Lösung:}\\

Wir wollen die Nullstelle der Funktion
\begin{equation}
    f(x) = x^2 - a,
\end{equation}
da in der Nulstelle
\begin{equation*}\begin{split}
    0 &= x^2 - a\\
    a &= x^2
\end{split}\end{equation*}
was genau die Definition der Wurzel ist.\\

Die Iteration ergbit die Folgende Rekursion.
\begin{equation}\begin{split}\label{ha:ueb4:quadit}
    x_{i+1} &= x_i - \frac{f(x_i)}{f'(x_i)}\\
    \Rightarrow x_{i+1} &= x_i - \frac{x_i^2 - a}{2x_i}\\
    \Rightarrow x_{i+1} &= \frac{1}{2} (x_i + \frac{a}{x_i})
\end{split}\end{equation}

\subsubsection*{(b)}

Führen Sie die selbe Analyse wie bei \emph{Aufgabe 2} aus.\\

\textbf{Lösung:}\\

Wir sind nun wieder am Fehler interessiert. Dazu lösen wir wiederum auf.

\begin{equation}\begin{split}
    \varepsilon_{i+1} &\stackrel{\text{Fehler}}{=} \left| x_{i+1} - \sqrt{a} \right|\\
    &\stackrel{(\ref{ha:ueb4:quadit})}{=} \left| \frac{x_i}{2} + \frac{a}{2x_i} - \sqrt{a} \right|\\
    &\stackrel{\text{Binomie}}{=} \left| \left( \frac{\sqrt{a}}{bla}\right)^2 \right| 
\end{split}\end{equation}

\label{LastPage}

\end{document}
