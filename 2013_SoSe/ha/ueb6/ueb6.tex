\documentclass[11pt,a4paper,ngerman]{article}
\usepackage[bottom=2.5cm,top=2.5cm]{geometry} 
\usepackage{babel}
\usepackage[utf8]{inputenc} 
\usepackage[T1]{fontenc} 
\usepackage{ae} 
\usepackage{amssymb} 
\usepackage{amsmath}
\usepackage{amsthm} 
\usepackage{graphicx}
\usepackage{fancyhdr}
\usepackage{fancyref}
\usepackage{listings}
\usepackage{xcolor}
\usepackage{paralist}

\usepackage[pdftex, bookmarks=false, pdfstartview={FitH}, linkbordercolor=white]{hyperref}
\usepackage{fancyhdr}
\pagestyle{fancy}
\fancyhead[C]{Höhere Algorithmik}
\fancyhead[L]{Übungsblatt Nr. 6}
\fancyhead[R]{SoSe 2013}
\fancyfoot{}
\fancyfoot[L]{}
\fancyfoot[C]{\thepage \hspace{1px} of \pageref{LastPage}}
\renewcommand{\footrulewidth}{0.5pt}
\renewcommand{\headrulewidth}{0.5pt}
\setlength{\parindent}{0pt} 
\setlength{\headheight}{0pt}

\date{}
\title{Übungsblatt Nr. 6}
\author{Max Wisniewski, Alexander Steen}


%%
%% Enviroments for proofs and lemmas
%%
\newtheorem{lemma}{\bfseries Lemma}
\newtheorem{claim}{\bfseries claim}
\newtheorem{theorem}{\bfseries Theorem}

\begin{document}

\renewcommand{\figurename}{Figure}
\maketitle
\thispagestyle{fancy}


%%%%%%%%%%%%%%%%%%%%%%%%%%%%%
%%      Aufgabe 1
%%%%%%%%%%%%%%%%%%%%%%%%%%%%
\subsection*{Aufgabe 1}

Lösen Sie das folgende lineare Programm mit dem zweiphasigen Simplex-Algorithmus:
\begin{equation*}\begin{array}{rlcr}
    \max &c^Tx\\
    \text{s.t.} & A x & \leq & b
\end{array}\end{equation*}
wobei 
\begin{equation*}
c = \left( \begin{array}{c} 3 \\ 1 \end{array}\right), A = \left( \begin{array}{cc} 1 & -1 \\ -1 & -1 \\ 2 & -1\end{array}\right), b =
\left( \begin{array}{c} -1 \\ -3 \\ 2 \end{array}\right)
\end{equation*}
gilt.

\textbf{Lösung:}\\

Wir formen unser LP zunächst in die Standardform um.
Wobei $x_1 = x_1 - x_3$ und $x_2 = x_2 + x_4$.
\begin{equation*}\begin{array}{rlcr}
    \min & -3x_1 + 3x_3 - x_2 + x_4\\
    \text{s.t.} &
        x_1 - x_3 - x_2 + x_4 + x_5 & = & -1\\
    & - x_1 + x_3 - x_2 + x_4 + x_6 & = & -3\\
    & 2x_1 - 2x_3 - x_2 + x_4 + x_7 & = & 2\\
    & x & \geq & 0
\end{array}\end{equation*}
Wir sehen, dass $(0,0,0,0,-1,-3,2)$ keine bfs ist. Daher formulieren wir das neue LP um
die Startecke zu finden.

\begin{enumerate}[1.]
    \item Wir starten mit dem Dictionary:
        \begin{equation*}\begin{array}{rcl}
            f & = & x_0\\
            x_5 & = & -1 - x_1 + x_3 + x_2 - x_4 + x_0\\
            x_6 & = & -3 + x_1 - x_3 + x_2 - x_4 + x_0\\
            x_7 & = & 2 - 2x_1 + 2x_3 + x_2 - x_4 + x_0\\
        \end{array}\end{equation*}
    \item Wir setzen in unserem resultierenden LP $x_0$ als Exitvariable und setzen es auf $3$, damit
        alle Variablen korrekt größer als $0$ sind. Damit ist $x_6$ Entryvariable, da die Gleichung
        für seine Nebenbedingung $0$ wird.\\
        Wir erhalten das Dictionary:
    \begin{equation*}\begin{array}{rcl}
        f = x_0 & = & 3 - x_1 + x_3 -x_2 + x_4 + x_6\\
        x_5 & = & 2 - 2x_1 + 2x_3 + x_6\\
        x_7 & = & 5 - 3x_1 + 3x_3 + x_6\\
        \text{bfs} & = & \left( 3 , 0, 0 ,0 ,0 , 2 , 0, 5 \right)  
    \end{array}\end{equation*}
    \item Wir sehen,dass wir $x_2$ als Exitvariable nehmen können und und $x_0$ als Entryvariable um so eine
        bfs zu finden. Damit (einsetzen) ist
        $\text{bfs} = (0, 3, 0, 0, 2, 0, 5)$ eine bfs für das ursprüngliche LP mit der Startecke
        $x_2, x_5, x_7$.
    \item Erstellen des Initialen Dictionaries.
        \begin{equation*}\begin{array}{rcl}
            f & = & -3x_1 + 3x_3 - (3 - x_1 + x_3 + x_4 + x_6) + x_4\\
                & = & -3 - 2x_1 + 2x_3 - x_6\\
            x_5 & = & 2 - x_1 + x_3 - x_4\\
            x_2 & = & 3 - x_1 + x_3 + x_4 + x_6\\
            x_7 & = & 5 - 2x_1 + 2x_3 - x_4\\
            \text{bfs} & = & (0, 3, 0, 0, 2, 0, 7)
        \end{array}\end{equation*}
    \item Wir nehmen $x_1$ als Exitvariable. Wir sehen, dass $x_5$ als erstes $0$ wird
        mit $x_1 = 2$.\\
        \begin{equation*}\begin{array}{rcl}
            f   & = & -3 - 2 (2 + x_3 - x_4 - x_5) + 2x_3 - x_6\\
                & = & -7 + 2x_4 + 2x_5 - x_6\\
            x_1 & = & 2 + x_3 - x_4 - x_5\\
            x_2 & = & 3 - (2 + x_3 - x_4 - x_5) + x_3 + x_4 + x_6\\
                & = & 1 + 2x_4 + x_5 + x_6\\
            x_7 & = & 5 - 2 (2 + x_3 - x_4 - x_5) + 2x_3 - x_4\\
                & = & 1 + x_4 + 2x_5\\
            \text{bfs} & = & (2, 3, 0, 0, 0, 0, 1)
        \end{array}\end{equation*}
    \item Nun ist $x_6$ einzige verbleibende Exitvariable, aber es gibt keine Bedingung, so dass mit $x_6 \geq 0$ eine Nulstelle
        erreicht wird. Demnach können wir das LP beliebig groß machen und die Zielfunktion ist unbeschränkt.
\end{enumerate}

\subsection*{Aufgabe 2}

Wie reagiert der Simplex-Algorithmus, falls für das lineare Programm gilt:

\subsubsection*{(a)}
Das LP hat mehr als eine optimale Lösung.\\

\textbf{Lösung:}\\

Wenn der Simplex-Algorithmus mehr als eine optimale Lösung hat, so liegt diese in einer Facette des Polyeders mit einer
Dimension größer $0$. Der Simplex-Algorithmus mit der Regel von Bland wird eine
Ecke dieser Facette finden.
Nun gilt in dieser Ecke:
\begin{itemize}
    \item Die Hyperebenen, die nicht diese Facette bilden, werden die Zielfunktion kleiner machen, wenn
        wir ihre Variablen vergrößern, da die Facette sonst nicht optimal wäre. 
        Daher werden sie positiv in der Zielfunktion stehen. Werden also vom Algorithmus nicht für
        eine Austrittsvariable in betracht gezogen.\\
    \item Die Hyperebenen, die diese Facette bilden, können von den Bedingungsfunktionen zwar
        noch vergrößert werden, aber da der Wert überall in der Facette gleich ist, werden
        diese mit einem Gewicht von $0$ in der Zielfunktion stehen. Also werden auch
        diese Hyperebenen nicht für ein Fortfahren in betracht gezogen.
\end{itemize}
Es gibt also in einer Zielecke keine Variable, die wir erhöhen können. Der Simplex-Algorithmus terminiert
also in der ersten Ecke der Zielfacette, die er erreicht.

\subsubsection*{(b)}
Das LP hat keine zulässige Lösung.\\

\textbf{Lösung:}\\

Wenn es keine zulässige Lösung gibt, so enthält es insbesondere keine bfs.
Dies bedeuted für uns, dass unser erstes Hilfs-LP für das finden der Startecke,
das die neue Variable $x_0$ minimieren sollte, einen Wert $>0$ liefert. Wir wissen,
dass nach Blands Regel, dass es eine Lösung finden wird. Nach den Vorzeichenbedingungen
ist $x \geq 0$ und da $x=0$ bedeuted, dass es eine bfs für das ursprüngliche LP gibt,
wird das Hilfs-LP eben einen Wert $>0$ liefern und der Algorithmus kann abbrechen.

\subsubsection*{(c)}
Die Zielfunktion ist unbeschränkt.\\

\textbf{Lösung:}\\

Wenn die Zielfunktion nicht beschränkt ist, wird der Algorithmus irgendwann eine
Kante finden (da der Quadrant immer von $x_1, ... , x_n \geq 0$ begränzt ist, existieren
zumindest immer Kanten und ein Punkt in dem wir sind), entlang dessen wir die Zielfunktion
verkleinern können ($b_i < 0$), aber wenn wir die aktuellen Bedingungsfunktionen betrachten
und die Nullstelle mit diesen berechen, sind alle Nullstellen für $x_i < 0$ erreicht.
Da die Funktionen linear ist es die einzige Nullstelle und folglich können wir $x_i$ beliebig groß machen.

Der Algorithmus sieht aber das es keine Nulstellen geben wird und kann abbrechen.

\subsection*{Aufgabe 3}

Reduktionen auf ILP (Integer Lineare Programming).

\subsubsection*{(a)}
SAT $\preceq_P$ ILP\\

\textbf{Lösung:}\\

Wir wissen, dass wir jede Aussagenlogische, mit linear mehr Variablen als KNF in polynomieller Größe
der eigentlichen Formel darstellen können. Wir nehmen daher an, dass die Eingaben für SAT hier in KNF gegeben sind.\\

Sei $\Phi = \overset{m}{\underset{i=1}{\bigwedge}} \overset{n}{\underset{j=1}{\bigvee}} l_{ij}$
wobei $l_{ij}$ das Literal in Klausel $i$ zur Variable $j$
\begin{itemize}
    \item FALSE ist, wenn die Variable $j$ nicht in der Klausel $i$ steht.
    \item $\overline{x_j}$ ist, wenn die Variable $x_j$ negativ in der Klausel $i$ steht.
    \item $x_j$ ist, wenn die Varibale $x_j$ psoitiv in der Klausel $i$ steht.
\end{itemize}

Nun ist $f(\Phi)$ definiert durch:
Wir verdoppeln jede Variable $x_i$ in $x_i^+$ und $x_i^-$.
Als Vorzeichen-/Integerbedingung verlangen wir $x_i \in \{0,1\}$.\\

$c = (1 , ... , 1) \in \mathbb{R}^{2n}$,
$A = (a_{ij})_{1 \leq i \leq m+2n, 1 \leq j \leq 2n}$
mit 
\begin{equation*}\begin{split}
\forall 1 \leq i \leq m, 1 \leq j \leq n \, : \, a_{ij} &= \left \{ \begin{array}{rl} 1 \quad &, l_{ij} = x_j \\ 0 &, \text{sonst}\end{array}\right. ,\\
\forall 1 \leq i \leq m, 1 \leq j \leq n \, : \, a_{i(j+n)} &= \left \{\begin{array}{rl} 1 \quad &, l_{ij} = \overline{x_j} \\ 0 &, \text{sonst}\end{array}\right. \\
\forall 1 \leq i \leq n \; : \; a_{(i+m)i} &= 1 \land a_{(i+m)(i+n)} = 1.
\end{split}\end{equation*}

Wir verlangen nun, dass für die ersten $m$ Zeilen gelten muss, dass in der Summe $\geq 1$ herauskommen soll.
Für den zweiten Teil der Matrix, soll genau $1$ herauskommen. (Wir haben sorum also keine Normalform).\\

Unser LP führt also für jede Variable einen positiven und einen negativen Teil ein.
Danach addieren wir pro Klausel die Literale, die wir gesetzt haben.
Am Ende verlangen wir noch, dass jede Variable auf true ODER false gesetzt wird und nicht beides
gleichzeit sein kann.\\

Erhalten wir vom ILP also eine Lösung $x^*$ so gilt für diese:
$\overset{2n}{\underset{j=1}{\sum}} a_{ij}x_j \geq 1$ für alle Bedingungen $i$.
Dies Bedeuted, dass wir für jede Klausel mindestens ein Literal positiv gesetzt haben,
dass in $i$ vorkommt.\\
Da darüber hinaus gilt, dass $x_i + x_{2i} = 1$ ist also eine Variable entweder auf true oder auf false gesetzt.\\
Also ist für das SAT Problem eine Lösung gefunden indem wir $y_i =$ true, wenn $x_i = 1$ und $y_1 =$ false, wenn $x_{2i} = 1$ ist.\\

Wenn die Formel $\Phi$ erfüllbar ist, können wir unsere Variablen,
wie folgt setzten: Falls $y_i = $ true, dann ist $x_i = 1$ und $x_{2i} = 0$. Falls $y_i = $ false, dann ist $x_i = 0$ und $x_{2i} = 1$.
Die Vorzeichenbedingung ist erfüllt. Addieren wir beide Variablen erhalten wir $1$ und nach
der überlegung aus der anderen Richtung, wissen wir, dass jede Summe größer als $1$ sein muss.\\

Das größte was wir konstruieren ist eine Matrix der Größe $(m+2n)\times(2n)$. Damit ist unser Algorithmus polynomiell.

\subsubsection*{(b)}
MAX-INDEPENDENT-SET $\preceq_P$ ILP\\

\textbf{Lösung:}\\

Sei $G(V,E)$ ein Graph. Dann führen wir für jeden Knoten eine Variable ein $x_v \in \{ 0 , 1 \} \forall v \in V$.
Wir maximieren nun $\underset{v \in V}{\sum} x_v$ unter der Bedingung,
dass $\forall (i,j) \in E \, : \, x_i + x_j \leq 1$ gilt.\\

Dies Bedeutet, dass wir für jede Kante höchstens einen Knoten genommen haben, was genau der Definition von
unabhängiger Knotenmenge entspricht.\\

Wir zeigen zunächst, dass jede feasable solution von diesem LP eine unabhängige Knotenmenge der Größe
als als Ursprung hat und umgekehrt.\\

Sie $x^*$ eine feasable solution. Dann wissen wir, dass für jede Kante $(i,j) \in E$ nur ein Knoten von $i,j$
genommen worden sein konnte. Da für alle Variablen die Bedingungen erfüllt ist, gibt es in $G$ eine unabhängige
Knotenmenge der Größe $\sum x^*$.\\
Umgekehrt könen wir zu einer Unabhängigen Knotenmenge $v^*$ auch alle Variablen zu jedem Knoten auf $1$ setzen und
den Rest auf $0$. Da die Knotenmenge unabhängig war, erfüllt diese Belegung die Bedingungsfunktion und die Zielfunktion
hat den Wert $|v^*|$.\\

Sei $v^* \subseteq V$ nun eine maximale unabhängige Knotenmenge.
Dann ist $x^*$ wie oben gezeigt eine feasable solution für das ILP mit $|x^*| = |v^*|$. Wäre $x^*$ nicht optimal,
gäbe es ein $x'$ mit $|x'| > |x^*|$. Wie wir aber schon gezeigt haben, gibt es dann auch eine Knotenmenge
$v'$ zu $x'$, mit $|v'| = |x'| > |x^*| = |v^*|$. Damit war $v^*$ nicht optimal.

Die Rückrichtung läuft unter der vorher gezeigten Bedingung analog.\\

Der Algoritmus konstruiert einen Zielvektor der Größe $|A|$, eine Matrix der Größe $|A|\times1$ und einen Bedingungsvektor
der größe $|A|$. Damit ist die Reduktion linear.

\subsubsection*{(c)}
SUBSET-SUM $\preceq_P$ ILP\\

\textbf{Lösung:}\\

Wir konstruieren für die Menge $A$ von Zahlen ein ILP,
wobei wir jedem Elemente $a \in A$ eine Variable $x_a \in \{0 , 1 \}$ zuweisen.
Dies steht am Ende dafür, ob ein Element $a$ in der Lösung enthalten ist oder nicht.
Die Zielfunktion ist daher
$\max \underset{ a \in A }{\sum} x_a$.\\

Nun verlangen wir noch das die Nebenbedingung an die Lösung erfüllt ist, also
$\underset{ a \in A }{\sum} x_a * a \leq b$.
Dies ist linear, da $a \in \mathbb{R}$ im LP nur eine konstante ist.\\

Es sollte nun offensichtlich sein, dass es sich hierbei um eine Reduktion handelt,
da wir nur die Definition des Problems linear formuliert haben. Der Beweis ist
somit trivial.\\

Die Laufzeit ist ebenso simple linear wie in Aufgabenteil \emph{b)}.

\subsubsection*{(d)}
MIN-VERTEX-COVER $\preceq_P$ ILP\\

\textbf{Lösung:}\\

Analog zu Independent-Set können wir Vertex-Cover formulieren.\\

Sei $G(V,E)$ ein Graph. Wir suchen eine minimale Anzahl von Knoten $v \in V' \subset V$, so
dass jede Kante $e \in E$ zu mindestens einem Knoten adjazent ist $e \cap V' \not= \emptyset$.\\

Wir suchen die minimale Menge (minimum), so dass die Bedingung erfüllt ist, also minimieren
wir die Zielfunktion $\underset{v \in V}{\sum} x_v$, wobei wir für jeden Knoten $v\in V$
eine Variable $x_v \in \{ 0, 1 \}$ enführen.

Die Nebenbeding lässt Darstellen als
$x_i + x_j \geq 1$ für alle $e = (i,j) \in E$.

Der Beweis ist nun Aufgrund der leichten umformulierung nicht mehr trivial, aber noch so hinreichend
einfach, dass es offensichtlich ist, dass die Reduktion stimmt.\\

Wir konstriueren eine Zielfunktion der Größe $|V|$, eine Matrix und eine Bedingung der Größe $O(|E|)$.
Damit ist die Reduktion polynomiell.

\label{LastPage}
\end{document}
