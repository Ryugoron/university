\documentclass[11pt,a4paper,ngerman]{article}
\usepackage[bottom=2.5cm,top=2.5cm]{geometry} 
\usepackage{babel}
\usepackage[utf8]{inputenc} 
\usepackage[T1]{fontenc} 
\usepackage{ae} 
\usepackage{amssymb} 
\usepackage{amsmath}
\usepackage{amsthm} 
\usepackage{graphicx}
\usepackage{fancyhdr}
\usepackage{fancyref}
\usepackage{listings}
\usepackage{xcolor}
\usepackage{paralist}

\usepackage[pdftex, bookmarks=false, pdfstartview={FitH}, linkbordercolor=white]{hyperref}
\usepackage{fancyhdr}
\pagestyle{fancy}
\fancyhead[C]{Höhere Algorithmik}
\fancyhead[L]{Übungsblatt Nr. 8}
\fancyhead[R]{SoSe 2013}
\fancyfoot{}
\fancyfoot[L]{}
\fancyfoot[C]{\thepage \hspace{1px} of \pageref{LastPage}}
\renewcommand{\footrulewidth}{0.5pt}
\renewcommand{\headrulewidth}{0.5pt}
\setlength{\parindent}{0pt} 
\setlength{\headheight}{0pt}

\date{}
\title{Übungsblatt Nr. 8}
\author{Max Wisniewski, Alexander Steen}


%%
%% Enviroments for proofs and lemmas
%%
\newtheorem{lemma}{\bfseries Lemma}
\newtheorem{claim}{\bfseries claim}
\newtheorem{theorem}{\bfseries Theorem}

\begin{document}

\renewcommand{\figurename}{Figure}
\maketitle
\thispagestyle{fancy}


%%%%%%%%%%%%%%%%%%%%%%%%%%%%%
%%      Aufgabe 1
%%%%%%%%%%%%%%%%%%%%%%%%%%%%
\subsection*{Aufgabe 1}

%%%%%%%%%%%%%%%%%%%%%%%%%%%%%
%%      Aufgabe 2
%%%%%%%%%%%%%%%%%%%%%%%%%%%%
\subsection*{Aufgabe 2}
Es ist $t_0 = 0, C_0 = E_n$ und die NB $x_1 \leq -1$ wurde verletzt.
Zu zeigen:
\begin{enumerate}[(a)]
\item $t_1$ und $C_1$ haben die Gestalt wie auf dem Zettel \\
    Da die NB $x_1 \leq -1$ verletzt wurde, gilt in der ersten Iteration
    \begin{equation}\label{eq:1}
      a^t = \left(1,0,\ldots,0 \right)
    \end{equation}
    Durch die Berechnungsvorschrift ergibt sich für $t_1$:
    \begin{equation*}\begin{split}
      t_1 &= t_0 -\frac{1}{n+1} \frac{C_0 a}{\sqrt{a^t C_0 a}}
          = -\frac{1}{n+1} \frac{a}{\sqrt{a^t a}} \\
          &\stackrel{(\ref{eq:1})}{=} -\frac{1}{n+1} \left(1,0,\ldots,0 \right)
          = \left(-\frac{1}{n+1},0,\ldots,0 \right)^t
    \end{split}\end{equation*}
    
    und für $C_1$:
    \begin{equation*}\begin{split}
      C_1 &= \frac{n^2}{n^2-1} \left(E_n - \frac{2}{n+1} \cdot \frac{a a^t}{a^t a} \right) 
          \stackrel{(\ref{eq:1})}{=} \frac{n^2}{n^2-1} \left(E_n - \frac{2}{n+1} \cdot
                  \left(\begin{array}{cccc} 1 & 0 & \hdots & 0 \\
                                            0 & 0 & \hdots & 0 \\
                                            \vdots & \vdots & \ddots & \vdots \\
                                            0 & 0 & \hdots & 0
                  \end{array} \right)
              \right) \\
          &=  \frac{n^2}{n^2-1}  \left(\begin{array}{cccc} 1-\frac{2}{n+1} & 0 & \hdots & 0 \\
                                            0 & 1 & \hdots & 0 \\
                                            \vdots & \vdots & \ddots & \vdots \\
                                            0 & 0 & \hdots & 1
                  \end{array} \right) 
          = \left(\begin{array}{cccc} \frac{n^2}{n^2-1} \cdot \frac{n-1}{n+1} & 0 & \hdots & 0 \\
                                            0 & \frac{n^2}{n^2-1} & \hdots & 0 \\
                                            \vdots & \vdots & \ddots & \vdots \\
                                            0 & 0 & \hdots & \frac{n^2}{n^2-1}
                  \end{array} \right) \\
          &= \left(\begin{array}{cccc} \frac{n^2}{(n+1)^2} & 0 & \hdots & 0 \\
                                            0 & \frac{n^2}{n^2-1} & \hdots & 0 \\
                                            \vdots & \vdots & \ddots & \vdots \\
                                            0 & 0 & \hdots & \frac{n^2}{n^2-1}
                  \end{array} \right)
    \end{split}\end{equation*}
\item $C_1$ ist symmetrisch und positiv definit \\
      \begin{itemize}
        \item symmetrisch: Naja, haben wir ja gerade berechnet. $C_1$ hat Diagonalform und ist damit
              insbesondere symmetrisch.
        \item positiv definit: \\
              Sei $x \in \mathbb{R}^n$, $x \neq 0$. Dann gilt
              \begin{equation}\label{eq:2}\begin{split}
                x^t C_1 x = x_1^2 \cdot \overbrace{\frac{n^2}{(n+1)^2}}^{> 0}
                            + \overbrace{\frac{n^2}{n^2-1}}^{> 0} \sum_{i=2}^n  x_i^2
              \end{split}\end{equation}
              Da $x \neq 0$ ex. ein $1 \leq i \leq n$ mit $x_i \neq 0$. Dann ist
              $x_1^2 \cdot \frac{n^2}{(n+1)^2} + \frac{n^2}{n^2-1} \sum_{i=2}^n  x_i^2 > 0$.
      \end{itemize}
\item $\frac{1}{2}B_n = \{x| x^tx \leq 1, x_1 \leq 0 \} \subseteq E_1$ \\
      Sei $x \in \frac{1}{2}B_n$. Dann ist (*) $\sum_{i=1}^n x_i^2 \leq 1$, (**) $x_1 \leq 0$.\\
      Es ist 
      \begin{equation*}\begin{split}
        (x-t_1)^t C_1^{-1} (x-t_1) &= (x_1 - t_1^1)^2 \cdot \frac{(n+1)^2}{n^2} + \frac{n^2-1}{n^2} \sum_{i=2}^n  (x_i - t_1^i)^2 \\
        &= (x_1 + \frac{1}{n+1})^2 \cdot \frac{(n+1)^2}{n^2} + \frac{n^2-1}{n^2} \sum_{i=2}^n  x_i^2 \\
        &= \frac{(n+1)^2}{n^2} \cdot x_1^2 + 2\frac{n+1}{n^2} \cdot x_1 + \frac{1}{n^2} + \frac{n^2-1}{n^2} \sum_{i=2}^n  x_i^2 \\
        &= \frac{(x_1(n+1) + 1)^2}{n^2} + \frac{n^2-1}{n^2} \sum_{i=2}^n  x_i^2 \\
        &\geq BRINGT UNS NIX :( \underbrace{\frac{1+ 2x_1(n+1)}{n^2}}_{(**), \leq 1} +  \underbrace{\frac{n^2-1}{n^2} \sum_{i=2}^n  x_i^2}_{(*), \leq 1}  
      \end{split}\end{equation*}
\item $\frac{vol(E_1)}{vol(B_n)} < 2^{- \frac{1}{2(n+1)}}$ \\
\item ...
\end{enumerate}

\label{LastPage}
\end{document}
