\documentclass[11pt,a4paper,ngerman]{article}
\usepackage[bottom=2.5cm,top=2.5cm]{geometry} 
\usepackage{babel}
\usepackage[utf8]{inputenc} 
\usepackage[T1]{fontenc} 
\usepackage{ae} 
\usepackage{amssymb} 
\usepackage{amsmath}
\usepackage{amsthm} 
\usepackage{graphicx}
\usepackage{fancyhdr}
\usepackage{fancyref}
\usepackage{listings}
\usepackage{xcolor}
\usepackage{paralist}

\usepackage[pdftex, bookmarks=false, pdfstartview={FitH}, linkbordercolor=white]{hyperref}
\usepackage{fancyhdr}
\pagestyle{fancy}
\fancyhead[C]{Höhere Algorithmik}
\fancyhead[L]{Übungsblatt Nr. 8}
\fancyhead[R]{SoSe 2013}
\fancyfoot{}
\fancyfoot[L]{}
\fancyfoot[C]{\thepage \hspace{1px} of \pageref{LastPage}}
\renewcommand{\footrulewidth}{0.5pt}
\renewcommand{\headrulewidth}{0.5pt}
\setlength{\parindent}{0pt} 
\setlength{\headheight}{0pt}

\date{}
\title{Übungsblatt Nr. 8}
\author{Max Wisniewski, Alexander Steen}


%%
%% Enviroments for proofs and lemmas
%%
\newtheorem{lemma}{\bfseries Lemma}
\newtheorem{claim}{\bfseries claim}
\newtheorem{theorem}{\bfseries Theorem}

\begin{document}

\renewcommand{\figurename}{Figure}
\maketitle
\thispagestyle{fancy}

    
%%%%%%%%%%%%%%%%%%%%%%%%%%%%%
%%      Aufgabe 1
%%%%%%%%%%%%%%%%%%%%%%%%%%%%
\subsection*{Aufgabe 1}

Sei $Q \in \mathbb{R}^{n\times n}, A \in \mathbb{R}^{m \times n}, x,c \in \mathbb{R}^n, b \in \mathbb{R}^m$.
Bei der \emph{quadratischen Optimierung} ist eine Zielfunktion
$$
    x^T Q x + c^Tx
$$
zu maximieren oder minimieren unter der Nebenbedingung
$$
   A x \leq b
$$
\subsubsection*{(a)}
Zeigen Sie, dass quadratische Optimierung NP-schwer ist.\\

\textbf{Beweis:}\\
Wir reduzieren 3-Sat auf dieses Problem.\\

Die Idee ist uns nach Aufgabenteil b) gekommen.\\
Wir bilden die Nebenbedinung wie auf dem letzen Zettel, über die Summe der Literale pro Klausel. Nur können wir
im Gegensatz zum ILP nicht $x \in \{ 0 , 1 \}$ verlangen. Stattdessen verlangen wir $x \in [-1, 1]$; die Bedingungen $x \geq -1$ und $x \leq 1$ sind linear.\\

In der Zielfunktion minimieren wir nun das folgende.
$$
    x^T Q x + c^tx = \overset{n}{\underset{k=1}{\sum}} x_i^2 - 1
$$
Diese wollen wir nun maximieren. Wie wir sehen gilt für jeden einzelnen Summanden $x_i^2 - 1 \leq 0$, und $x_i^2 - 1 = 0$ falls $x_i \in \{ -1 , 1\}$.\\

Wir benötigen nun eine Entscheidungsvariante um von 3-Sat reduzieren zu können. Wir wählen die Variante, dass wir fragen,
ob eine Lösung mit dem Wert $0$ existiert.\\

Die Nebenbedinung sind wie beim letzen Mal gewählt, nur dass wir statt $x_i$ zu benutzen
die Makros $v_i = \frac{x_i + 1}{2}$ wählen. Im Fall, der uns interessiert ($x_i \in \{-1 , 1\}$) ist $v_i \in \{ 0, 1 \}$.
Damit ist der Rest der Reduktion wie auf dem letzen Zettel, da wir uns nun nur noch die Nebenbedinungen ansehen müssen
und über $v_i$ argumentieren.

\mbox{}\hfill$\square$

\subsubsection*{(b)}
Zeigen Sie, dass Optimieren einer linearen Zielfunktion mit quadratischen Nebenbedingungen NP-schwer ist.\\

\textbf{Beweis:}\\

Wie auf dem letzten Zettel erstellen wir ein Optimierungsproblem, bei dem wir daran interessiert sind,
ob eine Lösung existiert und nicht direckt an dessen Wert. Damit haben wir auch wieder eine Entscheidungsvariante des Problems.\\

Nun reduzieren wir 3-SAT auf dieses Problem.\\

Wir führen für jede boolsche Variable $b_i$ eine Variable $x_i$ ein.\\
Wir stellen nun eine "Vorzeichenbedingung" an die Variablen mit
$x_i * x_i = 1$. Diese Bedingung ist quadratisch.

Um für die Klauseln zu rechnen bedienen wir uns des Makros $v_i = \frac{x_i + 1}{2}$.
Die Interpretation ist, dass $x_i = -1$ bedeutet, dass $b_i$ false ist und $x_i = 1$ bedeutet $b_i$ ist true.
Daraus folgt $b_i = true \Rightarrow v_i = 1$ und $b_i = false \Rightarrow v_i = 0$, so wie wir es schoneinmal hatten.
Für ein Literal $l_i$ aus einer Klausel setzen wir nun nun 
$$l_i = \left\{ \begin{array}{lr} v_i \quad &, l_i \text{ positiv} \\ 1-v_i &, l_i \text{ negativ} \end{array}\right.$$ \\

Für eine Klausel $k_j = (b_x \land b_y \land b_z)$ führen wir die Bedingung\\
$$
    l_x + l_y + l_z \geq 1
$$

Zuletzt erstellen wir die Zielfunktion, die uns aber eigentlich egal ist, indem wir
z.B. die richtigen Klauseln zählen und maximieren $\overset{m}{\underset{k=1}{\sum}} l_{x_k} + l_{y_k} + l_{z_k}$.\\

$w \in $ 3-Sat $\Leftrightarrow$ $f(w) \in$ Problem\\
Sei $w = \overset{m}{\underset{i=1}{\bigwedge}} k_i$ und es existiert eine gültige Belegung von $b_i$ bezeichnet als $\overline{b_i}$.\\
Dann ist $x_i = 1$, falls $\overline{b_i} = true$ und $x_i = -1$, falls $\overline{b_i} = false$, eine Lösung für das Optimierungsproblem.\\

Die "Vorzeichenbedingungen" sind erfüllt, da $(-1)^2 = 1^2 = 1$ gilt. Wie gezeigt, ist $v_i = 0$ falls $\overline{b_i} = false$ und 1 sonst.\\
Wie schon auf dem letzten Zettel gezeigt ist so die Nebenbedinung auch erfüllt.\\

$\Rightarrow$:\\
Wie schon gesagt, folgt aus der Nebenbedinung, dass alle $x_i \in \{ -1, 1\}$ liegen. Wir setzen wieder $\overline{b_i} = true$ falls
$x_i = 1$, $\overline{b_i} = -1$ falls $x_i = 1$. Aus dem zweiten Teil der Nebenbedinung wissen wir, dass für jede Klausel $l_1 + l_2 + l_3 \geq 1$ gilt.
Da alle $l_1 \in \{ 0 , 1 \}$ liegen, gilt daher, dass mindestens ein $l_i$ eins sein muss und damit ist pro Klausel mindestens ein Literal wahr.\\

\mbox{} \hfill $\square$

%%%%%%%%%%%%%%%%%%%%%%%%%%%%%
%%      Aufgabe 2
%%%%%%%%%%%%%%%%%%%%%%%%%%%%
\subsection*{Aufgabe 2}
Es ist $t_0 = 0, C_0 = E_n$ und die NB $x_1 \leq -1$ wurde verletzt.
Zu zeigen:
\begin{enumerate}[(a)]
\item $t_1$ und $C_1$ haben die Gestalt wie auf dem Zettel \\
    Da die NB $x_1 \leq -1$ verletzt wurde, gilt in der ersten Iteration
    \begin{equation}\label{eq:1}
      a^t = \left(1,0,\ldots,0 \right)
    \end{equation}
    Durch die Berechnungsvorschrift ergibt sich für $t_1$:
    \begin{equation*}\begin{split}
      t_1 &= t_0 -\frac{1}{n+1} \frac{C_0 a}{\sqrt{a^t C_0 a}}
          = -\frac{1}{n+1} \frac{a}{\sqrt{a^t a}} \\
          &\stackrel{(\ref{eq:1})}{=} -\frac{1}{n+1} \left(1,0,\ldots,0 \right)
          = \left(-\frac{1}{n+1},0,\ldots,0 \right)^t
    \end{split}\end{equation*}
    
    und für $C_1$:
    \begin{equation*}\begin{split}
      C_1 &= \frac{n^2}{n^2-1} \left(E_n - \frac{2}{n+1} \cdot \frac{a a^t}{a^t a} \right) 
          \stackrel{(\ref{eq:1})}{=} \frac{n^2}{n^2-1} \left(E_n - \frac{2}{n+1} \cdot
                  \left(\begin{array}{cccc} 1 & 0 & \hdots & 0 \\
                                            0 & 0 & \hdots & 0 \\
                                            \vdots & \vdots & \ddots & \vdots \\
                                            0 & 0 & \hdots & 0
                  \end{array} \right)
              \right) \\
          &=  \frac{n^2}{n^2-1}  \left(\begin{array}{cccc} 1-\frac{2}{n+1} & 0 & \hdots & 0 \\
                                            0 & 1 & \hdots & 0 \\
                                            \vdots & \vdots & \ddots & \vdots \\
                                            0 & 0 & \hdots & 1
                  \end{array} \right) 
          = \left(\begin{array}{cccc} \frac{n^2}{n^2-1} \cdot \frac{n-1}{n+1} & 0 & \hdots & 0 \\
                                            0 & \frac{n^2}{n^2-1} & \hdots & 0 \\
                                            \vdots & \vdots & \ddots & \vdots \\
                                            0 & 0 & \hdots & \frac{n^2}{n^2-1}
                  \end{array} \right) \\
          &= \left(\begin{array}{cccc} \frac{n^2}{(n+1)^2} & 0 & \hdots & 0 \\
                                            0 & \frac{n^2}{n^2-1} & \hdots & 0 \\
                                            \vdots & \vdots & \ddots & \vdots \\
                                            0 & 0 & \hdots & \frac{n^2}{n^2-1}
                  \end{array} \right)
    \end{split}\end{equation*}
\item $C_1$ ist symmetrisch und positiv definit \\
      \begin{itemize}
        \item symmetrisch: Naja, haben wir ja gerade berechnet. $C_1$ hat Diagonalform und ist damit
              insbesondere symmetrisch.
        \item positiv definit: \\
              Sei $x \in \mathbb{R}^n$, $x \neq 0$. Dann gilt
              \begin{equation}\label{eq:2}\begin{split}
                x^t C_1 x = x_1^2 \cdot \overbrace{\frac{n^2}{(n+1)^2}}^{> 0}
                            + \overbrace{\frac{n^2}{n^2-1}}^{> 0} \sum_{i=2}^n  x_i^2
              \end{split}\end{equation}
              Da $x \neq 0$ ex. ein $1 \leq i \leq n$ mit $x_i \neq 0$. Dann ist
              $x_1^2 \cdot \frac{n^2}{(n+1)^2} + \frac{n^2}{n^2-1} \sum_{i=2}^n  x_i^2 > 0$.
      \end{itemize}
\item $\frac{1}{2}B_n = \{x| x^tx \leq 1, x_1 \leq 0 \} \subseteq E_1$ \\
      Sei $x \in \frac{1}{2}B_n$. Dann ist (*) $\sum_{i=1}^n x_i^2 \leq 1$, (**) $x_1 \leq 0$.\\
      Es ist 
      \begin{equation*}\begin{split}
        (x-t_1)^t C_1^{-1} (x-t_1) &= (x_1 - t_1^1)^2 \cdot \frac{(n+1)^2}{n^2} + \frac{n^2-1}{n^2} \sum_{i=2}^n  (x_i - t_1^i)^2 \\
        &= (x_1 + \frac{1}{n+1})^2 \cdot \frac{(n+1)^2}{n^2} + \frac{n^2-1}{n^2} \sum_{i=2}^n  x_i^2 \\
        &= (x_1 + \frac{1}{n+1})^2 \cdot \frac{(n+1)^2}{n^2} + \left( \frac{n^2-1}{n^2} \cdot (- x_1^2 + \sum_{i=1}^n  x_i^2) \right) \\
        &= \frac{(x_1(n+1) + 1)^2}{n^2} + \left( \frac{n^2-1}{n^2} \cdot (- x_1^2 + \sum_{i=1}^n  x_i^2) \right) \\
        &\stackrel{(*)}{\leq} \frac{(x_1(n+1) + 1)^2 - x_1^2(n^2 -1)}{n^2} \\
        &= \frac{2x_1^2 (n+1) + 2x_1 (n+1) + 1}{n^2} \stackrel{n \geq 2}{\leq} 1
       % &= \frac{(n+1)^2}{n^2} \cdot x_1^2 + 2\frac{n+1}{n^2} \cdot x_1 + \frac{1}{n^2} + \frac{n^2-1}{n^2} \sum_{i=2}^n  x_i^2 \\
       % &= \frac{(x_1(n+1) + 1)^2}{n^2} + \frac{n^2-1}{n^2} \sum_{i=2}^n  x_i^2 \\
      %  &\geq BRINGT UNS NIX :( \underbrace{\frac{1+ 2x_1(n+1)}{n^2}}_{(**), \leq 1} +  \underbrace{\frac{n^2-1}{n^2} \sum_{i=2}^n  x_i^2}_{(*), \leq 1}  
      \end{split}\end{equation*}
\end{enumerate}

\label{LastPage}
\end{document}
