\documentclass[11pt,a4paper,ngerman]{article}
\usepackage[bottom=2.5cm,top=2.5cm]{geometry} 
\usepackage{babel}
\usepackage[utf8]{inputenc} 
\usepackage[T1]{fontenc} 
\usepackage{ae} 
\usepackage{amssymb} 
\usepackage{amsmath}
\usepackage{amsthm} 
\usepackage{graphicx}
\usepackage{fancyhdr}
\usepackage{fancyref}
\usepackage{listings}
\usepackage{xcolor}
\usepackage{paralist}

\usepackage[pdftex, bookmarks=false, pdfstartview={FitH}, linkbordercolor=white]{hyperref}
\usepackage{fancyhdr}
\pagestyle{fancy}
\fancyhead[C]{Höhere Algorithmik II}
\fancyhead[L]{Übung 1}
\fancyhead[R]{SoSe 2013}
\fancyfoot{}
\fancyfoot[L]{}
\fancyfoot[C]{\thepage \hspace{1px} of \pageref{LastPage}}
\renewcommand{\footrulewidth}{0.5pt}
\renewcommand{\headrulewidth}{0.5pt}
\setlength{\parindent}{0pt} 
\setlength{\headheight}{0pt}

\date{}
\title{Übung 1}
\author{Max Wisniewski, Alexander Steen}


%%
%% Enviroments for proofs and lemmas
%%
\newtheorem{lemma}{\bfseries Claim}

\begin{document}

\lstset{language=Pascal, basicstyle=\ttfamily\fontsize{10pt}{10pt}\selectfont\upshape, commentstyle=\rmfamily\slshape, keywordstyle=\rmfamily\bfseries, breaklines=true, frame=single, xleftmargin=3mm, xrightmargin=3mm, tabsize=2, mathescape=true}

\renewcommand{\figurename}{Figure}

\maketitle
\thispagestyle{fancy}

\subsection*{Aufgabe 1.}

Im Folgenden werden zwei Vorhergehensweisen angeben, wie man den Algorithmus
von Strassen zur Multiplikation zweier $n \times n$ Matrizen verwenden kann, falls
$n$ nicht unbedingt eine Zweierpotenz ist.

Bestimmen Sie die jeweilige Laufzeit einschließlich der Konstante im signifikantesten
Term genau und berechnen Sie, für welche $n$ diese Algorithmen weniger Operationen als
die klassische Methode benötigen.

\subsubsection*{(a)}

Die Matrizen werden bis zur nächsten Zweierpotenz geeignet aufgefüllt.\\

\textbf{Lösung:}\\

tbd

\subsubsection*{(b)}

Ist $n$ gerade so führt man einen Rekursionsschritt nach Strassen aus.
Andernfalls zerlegt man
$$
A = \left( \begin{array}{cc} A_{11} & A_{12}\\A_{21} & A_{22}\end{array} \right),
\quad
B = \left( \begin{array}{cc} B_{11} & B_{12}\\B_{21} & B_{22}\end{array} \right),
$$
wobei $A_{11}$ und $B_{11}$ $(n-1) \times (n-1)$-, $A_{12}$ und $B_{12}$ $(n-1)\times 1$-, $A_{21}$ und $B_{21}$ $1\times (n-1)$- und $A_{22}$ und $B_{22}$ $1 \times 1$- Matrizen sind.

Dann berechnet man $AB$ in der Aufteilung, wie bei der klassischen Multiplikation von $2\times 2$ Matrizen, wobei $A_{11}B_{11}$ rekursiv, die übrigen Produkte klassisch berechnet werden.\\

\textbf{Lösung:}\\

tbd

\subsection*{Aufgabe 2.}

\subsubsection*{(a)}

Zeigen Sie, dass die Multiplikation von $n \times n$- Matrizen mit $O(I(n))$ Operationen durchführbar ist, falls man mit $I(n)$ Operationen Matrizen invertieren kann.\\

\textbf{Lösung:}\\
Zur Multiplikation den Matrizen $A,B$ betrachten wir das folgende Inverse einer $3n \times 3n$-Matrix $M$:
\begin{equation}\label{eq:inverse}
  M^{-1} = \left(\begin{array}{ccc}
              1_n & A   & 0_n \\
              0_n & 1_n & B \\
              0_n & 0_n & 1_n
           \end{array}\right)^{-1}
         = \left(\begin{array}{ccc}
              1_n & -A   & AB \\
              0_n & 1_n & -B \\
              0_n & 0_n & 1_n
           \end{array}\right)
\end{equation}
Wir sehen also in Gleichung~(\ref{eq:inverse}), dass durch Invertierung der Matrix $M$ das Produkt $AB$
in der oberen rechte Ecke enthalten ist. Damit gilt $M(n) = I(3n)$, wobei $M(n)$ die Kosten der Multiplikation
darstellt.
Eine Abschätzung erhalten wir durch Nutzung der Abschätzung $(*)$: $I(n) = O(n^3)$. Dann gilt:

\begin{equation}
  M(n) = I(3n) \stackrel{(*)}{\leq} 9c\cdot I(n) = O(I(n))
\end{equation}
\subsubsection*{(b)}

Zeigen Sie, dass die Multiplikation von $n \times n$- Matrizen mit $O(S(n))$ Operationen durchführtbar ist, falls man mit $S(n)$ Operationen Matrizen quadrieren kann.\\

\textbf{Lösung:}\\
Zur Multiplikation den Matrizen $A,B$ betrachten wir die gleiche $3n \times 3n$-Matrix $M$ wie in a):
\begin{equation}\label{eq:square}
  M^{2} = \left(\begin{array}{ccc}
              1_n & A   & 0_n \\
              0_n & 1_n & B \\
              0_n & 0_n & 1_n
           \end{array}\right)^{2}
         = \left(\begin{array}{ccc}
              1_n & 2A   & AB \\
              0_n & 1_n & 2B \\
              0_n & 0_n & 1_n
           \end{array}\right)
\end{equation}
Wieder steht das Produkt $AB$ oben rechts. Es gilt also wieder $M(n) = S(3n)$. Auch für Quadrieren gilt
die Abschätzung $S(n) = O(n^3)$, also:

\begin{equation}
  M(n) = S(3n) {\leq} 9c\cdot S(n) = O(S(n))
\end{equation}


\subsection*{Aufgabe 3.}

Bei der Multiplikation Boolescher Matrizen wird $+$ durch $\lor$ und $\cdot$ durch $\land$ ersetzt. Strassens Algorithmus ist nicht direkt anwendbar, da $(\{0,1\}, \lor, \land)$ kein Ring ist.

Zeigen Sie, dass die Boolsche Matrizenmultiplikation mit $O( n^{\omega + \varepsilon} )$ Operationen aus $\{ \lor, \land, \neg \}$ für jedes $\varepsilon > 0$ möglich ist, wenn die Matrizenmultiplikation für ganze Zahlen mit $O(n^\omega )$ arithmetischen Operationen möglich ist.\\

\textbf{Lösung:}\\

tbd

\label{LastPage}

\end{document}
