\documentclass[11pt,a4paper,ngerman]{article}
\usepackage[bottom=2.5cm,top=2.5cm]{geometry} 
\usepackage{babel}
\usepackage[utf8]{inputenc} 
\usepackage[T1]{fontenc} 
\usepackage{ae} 
\usepackage{amssymb} 
\usepackage{amsmath}
\usepackage{amsthm} 
\usepackage{graphicx}
\usepackage{fancyhdr}
\usepackage{fancyref}
\usepackage{listings}
\usepackage{xcolor}
\usepackage{paralist}

\usepackage[pdftex, bookmarks=false, pdfstartview={FitH}, linkbordercolor=white]{hyperref}
\usepackage{fancyhdr}
\pagestyle{fancy}
\fancyhead[C]{Höhere Algorithmik}
\fancyhead[L]{Übungsblatt Nr. 5}
\fancyhead[R]{SoSe 2013}
\fancyfoot{}
\fancyfoot[L]{}
\fancyfoot[C]{\thepage \hspace{1px} of \pageref{LastPage}}
\renewcommand{\footrulewidth}{0.5pt}
\renewcommand{\headrulewidth}{0.5pt}
\setlength{\parindent}{0pt} 
\setlength{\headheight}{0pt}

\date{}
\title{Übungsblatt Nr. 5}
\author{Max Wisniewski, Alexander Steen}


%%
%% Enviroments for proofs and lemmas
%%
\newtheorem{lemma}{\bfseries Lemma}
\newtheorem{claim}{\bfseries claim}
\newtheorem{theorem}{\bfseries Theorem}

\begin{document}

\renewcommand{\figurename}{Figure}
\maketitle
\thispagestyle{fancy}


%%%%%%%%%%%%%%%%%%%%%%%%%%%%%
%%      Aufgabe 1
%%%%%%%%%%%%%%%%%%%%%%%%%%%%
\subsection*{Aufgabe 1}

\textbf{Bezeichner}:\\
Sei $1 \leq i \leq 4$. Dann bezeichnet \\
$r_i$ die Anzahl der gebauten Radios in der $i$-ten Woche,\\
$p_i$ die Anzahl der angestellten Personen in der $i$-ten Woche (es gilt $p_1 = 40$), \\
$w_i$ die Anzahl der ''bauenden'' Arbeiter in der $i$-ten Woche, \\
$t_i$ die Anzahl der ausbildenden Arbeiter in der $i$-ten Woche, \\
$a_i$ die Anzal der neuen Auszubildenden in der $i$-ten Woche. \\

Wir nehmen hierbei an, dass die Angestellten-Bezeichner die Veränderungen der Personalgröße reflektieren; so ist z.B. $p_2 = p_1 + a_1$ (die Auszubildenden wurden also ''übernommen''). \\

\textbf{Gewinnfunktion}: Die Gewinnfunktion~(\ref{eq:gewinn}) ergibt sich aus Umsatz abzüglich Kosten, also als Summe der Erlöse minus die Summe der Personal- und Produktionskosten. 

\begin{equation}\label{eq:gewinn}
 \underbrace{40r_1 + 36r_2 + 32r_3 + 28r_4}_{Umsatz} - \left(\overbrace{10\sum_i r_i}^{Produktionskosten}
 + \underbrace{\sum_i 400p_i + 200a_i}_{Personalkosten} \right)
\end{equation}

\textbf{Nebenbedingungen}: \\
\begin{tabular}{c|c}
Bedingung & Erklärung \\
\hline \hline
$\sum_{i=1}^4 r_i \leq 20000$ & Nicht mehr als 20000 Radios\\
$r_i \leq 50w_i$, für $i = 1,\ldots,4$ & Jeder bauende Arbeiter kann max. 50 Radios bauen \\
$p_i = w_i + t_i$, für $i = 1,\ldots,4$ & Einige Arbeiter bauen, andere bilden aus \\
$a_i \leq 3 t_i$, für $i = 1,\ldots,4$ & Jeder Ausbildende kann max. 3 Personen einweisen \\
$p_{i+1} = p_i + a_i$, für $i = 1,2,3$ & Nach einer Woche sind die neuen Mitarbeiter eingearbeitet 
\end{tabular}

Die Gewinnstrategie sieht nun vor, die Gewinnfunktion~(\ref{eq:gewinn}) zu maximieren
und dabei die Nebenbedingungen einzuhalten. Dies kann als LP formuliert werden, sollte aber
eigentlich als Integer-Programm formuliert werden, da die Anzahl von Mitarbeiten etc. keine
nicht-ganzzahligen Werte haben kann.
%%%%%%%%%%%%%%%%%%%%%%%%%%%%%
%%      Aufgabe 2
%%%%%%%%%%%%%%%%%%%%%%%%%%%%
\subsection*{Aufgabe 2}

%%%%%%%%%%%%%%%%%%%%%%%%%%%%%
%%      Aufgabe 3
%%%%%%%%%%%%%%%%%%%%%%%%%%%%
\subsection*{Aufgabe 3}




\label{LastPage}
\end{document}
