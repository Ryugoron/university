\documentclass[11pt,a4paper,ngerman]{article}
\usepackage[bottom=2.5cm,top=2.5cm]{geometry} 
\usepackage{babel}
\usepackage[utf8]{inputenc} 
\usepackage[T1]{fontenc} 
\usepackage{ae} 
\usepackage{amssymb} 
\usepackage{amsmath}
\usepackage{amsthm} 
\usepackage{graphicx}
\usepackage{fancyhdr}
\usepackage{fancyref}
\usepackage{listings}
\usepackage{xcolor}
\usepackage{paralist}

\usepackage[pdftex, bookmarks=false, pdfstartview={FitH}, linkbordercolor=white]{hyperref}
\usepackage{fancyhdr}
\pagestyle{fancy}
\fancyhead[C]{Höhere Algorithmik}
\fancyhead[L]{Übungsblatt Nr. 5}
\fancyhead[R]{SoSe 2013}
\fancyfoot{}
\fancyfoot[L]{}
\fancyfoot[C]{\thepage \hspace{1px} of \pageref{LastPage}}
\renewcommand{\footrulewidth}{0.5pt}
\renewcommand{\headrulewidth}{0.5pt}
\setlength{\parindent}{0pt} 
\setlength{\headheight}{0pt}

\date{}
\title{Übungsblatt Nr. 5}
\author{Max Wisniewski, Alexander Steen}


%%
%% Enviroments for proofs and lemmas
%%
\newtheorem{lemma}{\bfseries Lemma}
\newtheorem{claim}{\bfseries claim}
\newtheorem{theorem}{\bfseries Theorem}

\begin{document}

\renewcommand{\figurename}{Figure}
\maketitle
\thispagestyle{fancy}


%%%%%%%%%%%%%%%%%%%%%%%%%%%%%
%%      Aufgabe 1
%%%%%%%%%%%%%%%%%%%%%%%%%%%%
\subsection*{Aufgabe 1}

\textbf{Bezeichner}:\\
Sei $1 \leq i \leq 4$. Dann bezeichnet \\
$r_i$ die Anzahl der gebauten Radios in der $i$-ten Woche,\\
$p_i$ die Anzahl der angestellten Personen in der $i$-ten Woche (es gilt $p_1 = 40$), \\
$w_i$ die Anzahl der ''bauenden'' Arbeiter in der $i$-ten Woche, \\
$t_i$ die Anzahl der ausbildenden Arbeiter in der $i$-ten Woche, \\
$a_i$ die Anzal der neuen Auszubildenden in der $i$-ten Woche. \\

Wir nehmen hierbei an, dass die Angestellten-Bezeichner die Veränderungen der Personalgröße reflektieren; so ist z.B. $p_2 = p_1 + a_1$ (die Auszubildenden wurden also ''übernommen''). \\

\textbf{Gewinnfunktion}: Die Gewinnfunktion~(\ref{eq:gewinn}) ergibt sich aus Umsatz abzüglich Kosten, also als Summe der Erlöse minus die Summe der Personal- und Produktionskosten. 

\begin{equation}\label{eq:gewinn}
 \underbrace{40r_1 + 36r_2 + 32r_3 + 28r_4}_{Umsatz} - \left(\overbrace{10\sum_i r_i}^{Produktionskosten}
 + \underbrace{\sum_i 400p_i + 200a_i}_{Personalkosten} \right)
\end{equation}

\textbf{Nebenbedingungen}: \\

\begin{tabular}{c|c}
Bedingung & Erklärung \\
\hline \hline
$\sum_{i=1}^4 r_i \leq 20000$ & Nicht mehr als 20000 Radios\\
$r_i \leq 50w_i$, für $i = 1,\ldots,4$ & Jeder bauende Arbeiter kann max. 50 Radios bauen \\
$p_i = w_i + t_i$, für $i = 1,\ldots,4$ & Einige Arbeiter bauen, andere bilden aus \\
$a_i \leq 3 t_i$, für $i = 1,\ldots,4$ & Jeder Ausbildende kann max. 3 Personen einweisen \\
$p_{i+1} = p_i + a_i$, für $i = 1,2,3$ & Nach einer Woche sind die neuen Mitarbeiter eingearbeitet \\
\hline
$p_i, r_i, a_i, w_i, t_i \geq 0, i = 1,\ldots,4$ & Vorzeichenbed. 
\end{tabular}
$\quad$ \\
Die Gewinnstrategie sieht nun vor, die Gewinnfunktion~(\ref{eq:gewinn}) zu maximieren
und dabei die Nebenbedingungen einzuhalten. Dies kann als LP formuliert werden, sollte aber
eigentlich als Integer-Programm formuliert werden, da die Anzahl von Mitarbeiten etc. keine
nicht-ganzzahligen Werte haben kann.
%%%%%%%%%%%%%%%%%%%%%%%%%%%%%
%%      Aufgabe 2
%%%%%%%%%%%%%%%%%%%%%%%%%%%%
\subsection*{Aufgabe 2}
\begin{enumerate}[a)]
\item maximaler Netzwerkfluss als LP \\
Wir formen unser ungerichtetes Netzwerk $G = (V,E)$ so um, dass wir jede Kante $e$ durch zwei Flusskanten $e^+, e^-$
ersetzen, die den Fluss in die eine bzw. die andere Richtung beschreiben. Dann führen wir für jede solcher
gerichteten Kanten eine Variable $x_{e^+}$ (bzw. $x_{e^-}$) ein, die den Betrag des Flusses in die jeweilige 
Richtung beschreibt. Die Quelle des Netzwerkflusses sei $s$, die Senke sei $t$, die Kapazität sei gegeben durch $c \in \mathbb{R}^{E}$. Bezeichne $\delta^+(v)$ alle ausgehenden Kanten vom Knoten $v$, $\delta^-(v)$ alle eingehenden Kanten von $v$. Dann ist das LP für den max. Fluss gegeben durch:

$$ \max \sum_{e \in \delta^+(s)} x_e - \sum_{e \in \delta^-(s)} x_e$$

Die Nebenbedingungen sind durch die Flussbeschränkung und die Knotenregel gegeben: \\
(1) $x_{e^+} - x_{e^-} \leq c(e), x_{e^-} - x_{e^+} \leq c(e), \forall e \in E$ -- die Flüsse müssen betragsmäßig kleiner als die Kapazität sein \\

(2) $\sum_{e \in \delta^+(v)} x_e = \sum_{e \in \delta^-(v)} x_e, \forall v \in V \setminus \{s,t\}$ -- in jedem Knoten fließt genausoviel rein wie raus \\

(3) (Vorzeichenbed.) $x_{e^+} \geq 0, x_{e^-} \geq 0$, für alle Kanten $e$.

\item Länge des kürzesten Wegs von $s$ nach $t$ in $G = (V,E)$ gerichtet \\
Siehe Hinweis aus dem Blatt für die Benennung der Variablen; dann ergibt sich das LP durch: \\
$ \min x_t $ \\
mit den Nebenbed. \\
(1) $x_s = 0$ \\
(2) $x_u \leq x_v + c(u,v), \forall (u,v) \in E$ und \\
(3) Vorzeichenbed. $x_u \geq 0, \forall u \in V$. \\

\textbf{Korrektheit}: Yada bla
\end{enumerate}
%%%%%%%%%%%%%%%%%%%%%%%%%%%%%
%%      Aufgabe 3
%%%%%%%%%%%%%%%%%%%%%%%%%%%%
\subsection*{Aufgabe 3}
\begin{enumerate}[a)]
\item Zielfunktion unbeschränkt, wenn \\
(1) $s = 0, t > 0$, oder \\
(2) $s > 0, t = 0$, oder \\
(3) $s > 0, t > 0$. \\
In den ersten beiden Fällen verschwindet ein $x_i$ aus der Nebenbed. und darum kann dieser beliebig wachsen. Im dritten Fall wird für genügend große $x_i$ die Nebenbed. für immer gelten.
\item Keine zulässige Lösung gibt es, wenn \\
(1) $s < 0, t < 0$, oder \\
(2) $s < 0, t = 0$, oder \\
(3) $s = 0, t < 0$, oder \\
(4) $s = 0, t = 0$
\item Mehr als eine optimale Lösung \\
Wenn es mehr als eine optimale Lösung geben sollte, muss die Gerade $sx_1 + tx_2 = 1$ (Ebenenbegrenzung)) 
orthogonal zu der Zielfunktion sein. Die Zielfunktion hat Steigung 1, also muss die Gerade
$sx_1 + tx_2 = 1$ Steigung $-1$ haben. Durch Umformen erhalten wir
$sx_1 + tx_2 = 1 \Leftrightarrow x_2 = \frac{1}{t} - \frac{s}{t}x_1$ für $t \neq 0$.

Um die Steigung $-1$ zu erhalten muss also $\frac{s}{t} = 1$ gelten, also wenn $s = t$ gilt. Ist $s,t < 0$ gibt es aber nach (b) keine Lösung. Ist $s,t > 0$, so ist nach (a) die Zielfunktion unbeschränkt.\\
Also kann es nicht mehrere optimale Lösungen geben.
\item Genau eine optimale Lösung
In allen anderen Fällen.

%s < 0
%  t < 0
%    keine zulässige lösung
%  t = 0
%    keine zulässige lösung
%  t > 0
%    x1 <= 1/s - t*x2/s
%    x2 >= 1/t - s*x1/t
%    eine opt
%    
%s = 0
%  t < 0
%    keine zulässige lösung
%  t = 0
%    keine zulässige lösung
%  t > 0
%    x2 >= 1/t
%    x1 beliebig -> zielfunktion unbeschränkt
%
%s > 0
%  t < 0
%    s*x1 + t*x2 >= 1 ???
%  t = 0
%    s*x1 >= 1  <-> x1 >= 1/s, x2 beliebig
%    zielfunktion unbeschränkt
%  t > 0
%    s*x1 + t*x2 >= 1
%    zielfunktion unbeschränkt
\end{enumerate}

\label{LastPage}
\end{document}
