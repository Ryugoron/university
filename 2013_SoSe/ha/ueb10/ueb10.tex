\documentclass[11pt,a4paper,ngerman]{article}
\usepackage[bottom=2.5cm,top=2.5cm]{geometry} 
\usepackage{babel}
\usepackage[utf8]{inputenc} 
\usepackage[T1]{fontenc} 
\usepackage{ae} 
\usepackage{amssymb} 
\usepackage{amsmath}
\usepackage{amsthm} 
\usepackage{graphicx}
\usepackage{fancyhdr}
\usepackage{fancyref}
\usepackage{listings}
\usepackage{xcolor}
\usepackage{paralist}

\usepackage[pdftex, bookmarks=false, pdfstartview={FitH}, linkbordercolor=white]{hyperref}
\usepackage{fancyhdr}
\pagestyle{fancy}
\fancyhead[C]{Höhere Algorithmik}
\fancyhead[L]{Übungsblatt Nr. 10}
\fancyhead[R]{SoSe 2013}
\fancyfoot{}
\fancyfoot[L]{}
\fancyfoot[C]{\thepage \hspace{1px} of \pageref{LastPage}}
\renewcommand{\footrulewidth}{0.5pt}
\renewcommand{\headrulewidth}{0.5pt}
\setlength{\parindent}{0pt} 
\setlength{\headheight}{0pt}

\date{}
\title{Übungsblatt Nr. 10}
\author{Max Wisniewski, Alexander Steen}


%%
%% Enviroments for proofs and lemmas
%%
\newtheorem{lemma}{\bfseries Lemma}
\newtheorem{claim}{\bfseries claim}
\newtheorem{theorem}{\bfseries Theorem}

\begin{document}

\renewcommand{\figurename}{Figure}
\maketitle
\thispagestyle{fancy}

    
%%%%%%%%%%%%%%%%%%%%%%%%%%%%%
%%      Aufgabe 1
%%%%%%%%%%%%%%%%%%%%%%%%%%%%
\subsection*{Aufgabe 1}

Sie $G=(V,E)$ ein zusammenhängender ungerichteter gewichteter Graph mit $n$ Knoten und $m$ Kanten. Die Kantengewichte seien alle verschieden.

\subsubsection*{(a)}
Zeigen Sie, dass der Algorithmus von Boruvka tatsächlich einen MST berechnet.\\

\textbf{Beweis:}\\

\begin{lemma} Sei $F \subseteq V$ eine Teilmenge der Kanten. Dann ist
    $S(F) = \{ \{ v_1, v_2 \} \in E \, | \, v_1 \in F \land v_2 \in E \setminus F \}$ ein Schnitt. Und es gilt für
    $e := \underset{e \in S(F)}{\text{argmin }} w(e)$,
    $e \in$ MST $(V,E)$.\\
\end{lemma}

\textbf{Beweis:}
    Sei $S(F)$ ein Schnitt im Graph $G$. Dann ist im resultierenden
    MST sowohl der auf $F$ eingeschränkte Baum ein MST, als auch 
    $E \setminus E$. Nun müssen diese beiden Komponenten verbunden werden
    und es verlaufen nur $S(F)$ Kanten zwischen den beiden Bäumen.\\
    Da der MST aber minimal ist, muss 
    $\underset{e \in S(F)}{\text{argmin }} w(e)$ ausgewählt werden,
    da der Baum sonst nicht minimal ist.

\mbox{}\hfill$\square$

Für einen Knoten $v$ in einer Boruvka-Phase $n$ sein $V^n(v)$ die
Knotenmenge die in diesen Knoten kontrahiert wurde. Sei $E^n(v)$ die
Kantenmenge die bis in Runde $n$ zwischen den Knoten $V^n(v)$ ausgewählt
wurden.\\

Wir zeigen nun $\forall n \in \mathbb{N} \forall v \in V^n: G(V^n(v), E^n(v))$ ist ein MST.\\

\begin{description}
    \item[I.A.] Zu Beginn ist $V^0(v) = \{ v \}$ und $E^0(v) = \emptyset$
        und damit ein MST.\\
    \item[I.S.] Sei $v \in V^{n+1}$ und $v_1, ..., v_k \in V^n$ die 
        Knotenmenge, die in Runde $n$ in $e$ konstrahiert wurde.
        Nach konstruktion ist die Kante $e$ zwischen $v_i, v_j$
        die kleinste Kante zwischen $V^n(v_i)$ und $V^n(v_j)$.\\

        Sei nun $e$ eine Kante entlang der zu $v$ konstrahiert wurde
        und verlaufe $e$ zwischen $v_i, v_j$ und sei $v_i$ o.B.d.A.
        der Knoten für den $e$ minimales Gewicht hatte.\\

        Nun stellt $S(V^{n}(v_i))$ einen Schnitt dar und deshalb
        muss $e$ im resultieren MST sein. Dies gilt insbesondere
        für den reduzierten Graphen $V^{n+1}(v)$.

        Wir wissen nun für alle Kanten, entlang denen kontrahiert wurde,
        dass diese im MST sind und $E^{n}(v_i)$ sind auch alle Kanten
        im MST, da sich die Schnitteigenschaft immer auf den gesammten
        Graph bezog.\\

        Damit ist $(V^{n+1}, E^{n+1})$ ein MST.

        \mbox{}\hfill$\square$
\end{description}

        Im letzten Schritt sind haben wir nun noch einen Knoten
        und nach Induktion ist der Repräsentierte Graph ein MST.
    
        \mbox{}\hfill$\square$
\subsubsection*{(b)}

Zeigen Sie, dass sich eine Boruvka-Phase in Zeit $O(m')$ implementieren lässt,
wobei $m'$ die Anzahl der Kanten im aktuellen Graphen bezeichnet.\\

\textbf{Beweis:}\\
Wir nehmen an, dass die Knoten $V = \{1, ..., n \}$ nummeriert sind.
Weiter nehmen  wir an, dass alle adjazenten Knoten aufsteigend
ihrer Nummerierung in der Adjazenzliste stehen.\\

Zunächst müssen wir in jeder Runde die Kante mit minimalem Gewicht für
jeden Knoten suchen. Dazu betrachten wir einfach jede Kante,
die adjazent zu einem Knoten ist, was $2m'$ angesehene Kanten sind.\\

Danach müssen wir die Komponenten kontrahieren. Dazu bestimmen
wir zunächst ein Representaten pro Komponenten (den kleinsten
nach Nummerierung) und speichern und
ein Mapping, dass in $O(n) = O(m')$ geht.\\

Für jeden Representanten legen wir ein Bucket an.\\

Danach sehen wir uns alle Knoten einer neuen Komponente an. Wir gehen
die Listen von vorne an durch. Da die Listen die Knoten sortiert enthalten
sehen wir also, falls wir auf den selben Knoten mappen, immer gleichzeitig.
Wir wählen die kleinste Kante aus uns speichern sie im Bucket, für den Representanten.\\

Danach lassen wir alle Pointer eins weiterrücken in der Adjazenzliste,
die gemeinsam auf den zur Zeit kleinsten gezeigt haben. (Falls ein
Knoten nicht in der Adjazenzliste stand, so wird er hier nicht ein
Knoten übersprungen). 
Für diesen Schritt sehen wir uns wiederum nur jede Kante aus 2 Richtungen
an. Was zu einer Laufzeit von $O(m')$ führt.\\

Als letztes müssen wir nun die Buckets durchgehen und uns aus
den gleichen Representanten die kleinste Kante herraussuchen. Da wir
die Representanten aber zusammengeschrieben haben, werden wir über
jeden Teil nur einmal durchlaufen müssen und schreiben dann die Listen
wiederum zusammen. Da wird die Buckets schon sortiert anlegen können,
müssen wir nur noch über alle Buckets gehen und das Minimum aufnehmen,
was uns $O(n') = O(m')$ kostet, wir müssen dabei unseren eigenen Bucket immer
überspringen, da dieser eine Selbstkante hinzufügen würde.
Dieser Teil wird weniger als $3m'$ Schritte benötigen, da wir schon Kanten
eliminiert haben.\\

Insgesamt brauchen wir also $O(m')$ Schritte.

\mbox{}\hfill$\square$

\subsubsection*{(c)}

Folgern Sie, dass der Algorithmus $O(\min \{ m log n, n^2 \})$ Schritte benötigt.

\textbf{Lösung:}\\

Wir fassen die Laufzeit $T$ als eine Funktion von $n,m$ abhängig auf.
Dann erhalten wir für $n',m'$ im nächsten Schritt
$$
    T(n,m) = T(n',m') + O(m)
$$
wir müssen nun analysieren, wie sich $n'$ und $m'$ verhalten. Für $n'$ wissen
wir nur, dass sich die Anzahl der Knoten in jedem Schritt mindestens halbiert.
$$
    T(n,m) = T(\frac{n}{2}, m') + O(m)
$$
Schreiben wir also $m_i$ als die Knoten im $i$-ten Schritt erhalten wir als Laufzeit
$$
    T(n,m) = \overset{\log n}{\underset{i = 1}{\sum}} m_i
$$
Sollten sich die Kanten im Graphen von einem zum nächsten Schritt kaum verändern (wir verlieren nur $\frac{n}{2}$ Kanten),
so erhalten wir $m_0 = ... = m_{\log n}$, da wir die Differenz in der Landau-Notation fallen lassen können. Dies gibt uns eine obere Schranke und
dafür erhalten wir $m \log n$. 

Sollte der Graph voller besetzt sein, so dass wir in jedem Schritt einen Bruchteil von Kanten behalten,
so ergibt sich für $0 < c < 1$ als Anteil
$$
    T(n,m) = m \overset{\log n}{\underset{i = 1}{\sum}} c^i.
$$
Wir sehen, dass dies in $O(m)$ liegt. Da dieses Szenario aber nur in Graphen der Fall sein kann, die $O(n^2)$ Kanten besitzten,
da wir sonst nicht diesen Anteil erhalten, ist der zweite Teil des Minimums erklärt.

Für annährend Vollbesetzte Graphen ist die Laufzeit $O(n^2)$ und für schwach besetzte Graphen $O(m \log n)$.

\subsubsection*{(d)}

Kombinieren Sie Prim und Boruvka um einen MST-Algorithmus zu erhalten, der $O(m \log \log n)$ Schritte benötigt.\\

\textbf{Lösung:}\\

Wir führen zunächst $\log \log n$ Runden Boruvka aus. Auf dem resultierenden kontrohierten Graphen führen wir
Prim aus und fügen die Kanten zum Ergebnis der Boruvka Phasen hinzu.\\

Der Algorithmus ist korrekt. Wir in (a) gezeigt sind die einzelnen Knoten representanten für unterbäume im MST und diese
müssen mit MST Kanten verbunden werden. Da PRIM diese findet, ist das gesammtergebnis korrekt.\\

Die ersten $\log \log n$ Runden brauchen, wie in (c) gezeigt
maximal $O(m \log \log n)$ Zeit.
Nun halbiert sich die Anzahl der Knoten in jeder Runde, so dass wir
$n' = n / 2^{\log \log n} = n / \log n$ Knoten haben und $O(m)$ Kanten (im schlimmsten Fall).

Geben wir diese Zahlen in die Laufzeit für PRIM erhalten wir
$$
   \frac{n}{\log n} \log \frac{n}{\log n} + m = \frac{n}{\log n} \log n - \frac{n}{\log n} \log\log n + m = O(n + m)
$$

Addieren wir nun beide auf, so ergibt sich, da $n < m \log \log n$ und $m < m\log \log n$,
dass
$$
    T(n,m) = O(m \log \log n) + O(n + m) = O(m \log \log n)
$$
gilt.

\subsection*{Aufgabe 2}

Sei $G=(V,E)$ ein zusammenhängender ungerichteter gewichteter Graph mit $n$ Knoten
 und $m$ Kanten, so dass alle Kantengewichte paarweise verschieden sind.
Sei $F \subseteq E$ kreisfrei. Eine Kante $e \in E$ heißt F-\emph{leicht}, wenn
(i) $F \cup \{ e \}$ kreisfrei ist oder (ii) der eindeutige Kreis $C$ in $F \cup \{ e \}$ eine Kante enthält, die schwerer ist als $e$.

\subsubsection*{(a)}

Beweisen Sie, dass $F$ genau dann ein minimaler aufspannender Baum von $G$ ist, wenn $E \setminus F$ keine F-leichten Kanten enthält.\\

\textbf{Beweis:}\\

$\Rightarrow$:\\

Sei $F$ ein MST von $G$.\\
Sei $e \in E \setminus F$ eine F-leichte Kante. Dann gilt nach Def. von F-leicht
\begin{enumerate}[(1)]
    \item $F \cup \{ e \}$ ist kreisfrei. Da aber $F$ ein Spannbaum ist, ist dies unmöglich,
        da sonst vorher die Endpunkte $(a,b) := e$ nicht miteinander verbunden wären. Dies
        ist ein Wiederspruch zur Spannbaumeigenschaft von $F$.
    \item In $F \cup \{ e \}$ entsteht ein Kreis $C$, so dass eine Kante $e'$ in $C$ existiert mit $e' \not= e$ und
        einem höheren Gewicht als $e$.

        Zunächst muss $e$ auf dem Kreis liegen, da durch dass entfernen von $e$ der Kreis wieder verschwindet.\\
        Entfernen wir $e'$ vom Kreis so entsteht wieder ein Spannbaum, der aber geringer im Gewicht ist als
        $F$. Dies ist ein Widerspruch zur minimalität von $F$.
\end{enumerate}

$\Leftarrow$:\\
    Sei $F \subseteq E$ kreisfrei und $E \setminus F$ enthalte keine F-leichten Kanten.\\
    
    Zunächst ist $F$ ein Spannbaum, da wir keine weitere Kante hinzufügen können, ohne dass sich ein Kreis schließt.
    Desweiteren ist $F$ minimal.

    Angenommen es wäre nicht so, dann gäbe es eine Kante in $e' E \setminus F$, so dass wir durch das austauschen mit
    einer Kante $e \in F$ einen kleineren Spannbaum erhalten könnten.\\
    
    Da die erste Eigenschaft schon erfüllt ist, wissen wir, dass $F \cup \{ e' \}$ einen Kreis enthält. Damit
    durch das entfernen von $e$ wieder ein Spannbaum entsteht, muss $e$ auf dem geschlossenen Kreis liegen.\\

    Nun liegt aber auf dem Kreis, denn $e'$ geschlossen hat eine Kante mit geringerem Gewicht und damit ist $e'$ F-leicht,
    was ein Widerspruch zu unserer Annahme ist.

    \mbox{} \hfill $\square$

\subsubsection*{(b)}

Beweisen Sie, dass $e\in E$ genau dann F-leicht ist, wenn $e \in MSF(F \cup \{ e\})$ ist, wobei MSF den minimalen aufspannenden \emph{Wald} bezeichne.\\

\textbf{Beweis:}\\

$\Rightarrow$:\\
    Sei $e \in E$ F-leicht. 
    \begin{enumerate}[(1)]
        \item $F \cup \{ e \}$ ist kreisfrei. Da $MSF(F)$ eine Teilmenge von $F$ ist, wird $e$ in $MSF(F)$ keinen Kreis schließen.
            Sollte daher $e \not\in MSF(F)$ sein, könnten wir sie hinzufügen und $MSF(F)$ wäre kein Spannwald gewesen.
        \item $F \cup \{ e \}$ schließt einen Kreis $C$ so, dass eine Kante $e'$ in $C$ existiert mit $w(e') > w(e)$.\\
            Hier gilt zu beobachten, dass $MSF(F) = F$ gilt, da $F$ kreisfrei ist. Da $e$ einen Kreis in $F$ schließt,
            wird auch in $MSF(F) \cup \{ e \}$ ein Kreis geschlossen und wieder liegt $e'$ auf diesem Kreis.

            Nehmen wir an $e \not\in MSF(F \cup \{ e \})$ so ist $MSF(F \cup \{ e \}) = MSF(F) = F$, da wir nicht genommenene
            Kanten herrausnehmen können ohne das Ergebnis zu verändern. Dies ist aber ein Widerspruch zur Minimalität von $MSF(F \cup \{ e \})$,
            da wir schon wissen, dass $e$ in $MSF(F)$ F-leicht ist.
    \end{enumerate}

$\Leftarrow$:\\
    Sei $e \in MSF(F \cup \{ e\})$.
    \begin{enumerate}[(1)]
        \item Falls $MSF(F) \cup \{ e \}$ keinen Kreis schließt, dann verbindet $e = (a,b)$ zwei Komponenten an den Knoten $a,b$, die
            vorher nicht verbunden waren. Da $MSF(F)$ ein Spannbaum ist, gibt es auch in $F$ keinen Weg von $a$ nach $b$.
            Daher schließt $F \cup \{ e \}$ keinen Kreis und $e$ ist daher F-leicht.
        \item Falls $MSF(F) \cup \{ e \}$ einen Kreis $C$ schließt, dann gibt es auf diesem Kreis eine Kante $e'$ mit kleinerem Gewicht als $e$,
            da $e \in MSF(F \cup \{ e \})$ und die Kantengewichte eindeutig sind. Da $MSF(F) \subseteq F$ existiert der Kreis auch in $F$.

            Daraus folgt, dass $e$ F-leicht sein muss.
    \end{enumerate}
    \mbox{}\hfill $\square$
\label{LastPage}
\end{document}
