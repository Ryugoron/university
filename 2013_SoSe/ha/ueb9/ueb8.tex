\documentclass[11pt,a4paper,ngerman]{article}
\usepackage[bottom=2.5cm,top=2.5cm]{geometry} 
\usepackage{babel}
\usepackage[utf8]{inputenc} 
\usepackage[T1]{fontenc} 
\usepackage{ae} 
\usepackage{amssymb} 
\usepackage{amsmath}
\usepackage{amsthm} 
\usepackage{graphicx}
\usepackage{fancyhdr}
\usepackage{fancyref}
\usepackage{listings}
\usepackage{xcolor}
\usepackage{paralist}

\usepackage[pdftex, bookmarks=false, pdfstartview={FitH}, linkbordercolor=white]{hyperref}
\usepackage{fancyhdr}
\pagestyle{fancy}
\fancyhead[C]{Höhere Algorithmik}
\fancyhead[L]{Übungsblatt Nr. 9}
\fancyhead[R]{SoSe 2013}
\fancyfoot{}
\fancyfoot[L]{}
\fancyfoot[C]{\thepage \hspace{1px} of \pageref{LastPage}}
\renewcommand{\footrulewidth}{0.5pt}
\renewcommand{\headrulewidth}{0.5pt}
\setlength{\parindent}{0pt} 
\setlength{\headheight}{0pt}

\date{}
\title{Übungsblatt Nr. 9}
\author{Max Wisniewski, Alexander Steen}


%%
%% Enviroments for proofs and lemmas
%%
\newtheorem{lemma}{\bfseries Lemma}
\newtheorem{claim}{\bfseries claim}
\newtheorem{theorem}{\bfseries Theorem}

\begin{document}

\renewcommand{\figurename}{Figure}
\maketitle
\thispagestyle{fancy}

    
%%%%%%%%%%%%%%%%%%%%%%%%%%%%%
%%      Aufgabe 1
%%%%%%%%%%%%%%%%%%%%%%%%%%%%
\subsection*{Aufgabe 1}

Sei $A$ ein Array mit $n$ Einträgen. Wir wissen, dass $A$ einen bestimmten Schlüssel $k$ enthält. Wir
wollen den Eintrag finden, an dem $k$ vorkommt.

\subsubsection*{(a)}

Zeigen Sie, dass jeder deterministische Algorithmus im schlimmsten Fall mindestens $n-1$ Zugriffe auf $A$ durchführen muss,
um die Position von $k$ zu finden.\\

\textbf{Beweis:}\\

Wir haben zunächst keine näheren Anhaltspunkt, wie der Algorithmus aussieht. Was wir aber aussagen können ist, dass wir
jeden Algorithmus, der auf ein Feld mehr als einmal zugreift dahingehend optimieren können, dass er das Feld nur einmal
liest, indem wir uns die Information des Feldes speichern. Da wir nur die Anzahl der Zugriffe auf $A$ zählen, erhöht dies
die Komplexität nicht.\\

Wir sehen also, dass wir nur neue Informationen gewinnen, wenn wir uns ein neues Feld ansehen. Da der Algorithmus deterministisch ist,
wird er sich für eine Eingabe eine Permutation $\sigma \in \pi (n)$ betrachten und diese Schritt für Schritt ausführen.\\

Nun kann man für jede Eingabelänge $n$ eine Folge konstruieren, so dass die Permuatation des Algorithmus $k$ auf den Platz
$n$ setzt. $k = \sigma (n)$.

Der Algorithmus wird also $n-1$ Schritte benötigen, da er sich die ersten $n-1$ Felder anguckt und danach weiß,
dass das Element auf dem letzten Platz ist. (Das Element ist zugesichert im Array enthalten).\\

Wie gezeigt, ist dies für jeden Algorithmus konstruierbar und ein Algorithmus kann höchstens mehr Zugriffe auf das Array durchführen.

\mbox{} \hfill $\square$

\subsubsection*{(b)}

Entwickeln Sie einen randomisierten Algorithmus, der das Problem mit $\frac{n}{2}$ erwarteten Zugriffen löst.\\

\textbf{Lösung:}\\

Der Algorithmus ist recht simpel (Zu beachten ist die Indexverschiebung innerhalb des Arrays).

\begin{lstlisting}[mathescape=true]
$\sigma \leftarrow$ zufaellige Permutation von $[n]$
for i = 1 to n-1
    if a[$\sigma$(i) - 1] = k
        return $\sigma$(i) - 1
    return $\sigma$(n)-1
\end{lstlisting}

Der Algorithmus hat die erwartete Anzahl von $\frac{n}{2}$ Zugriffen.\\

\textbf{Beweis:}\\
    Wir führen die Zufallsvariable
    $$
        X_i = \left\{ \begin{array}{ll} 1 \quad &,\text{falls }k=a[\sigma(i)-1]\\ 0 &,\text{sonst} \end{array}\right.
    $$
    Bei einer gleichverteilten Wahl der Permutationen ist das Element $k$ an Position $i$
    mit der Wahrscheinlichkeit
    $$
        Pr[X_i = 1] = \frac{1}{n}
    $$

    Für das letzte Feld machen wir keinen Zugriff, daher können wir den Erwartungs wie folgt ausrechen:
    \begin{equation*}\begin{array}{rcl}
        E[X] &=& \overset{n-1}{\underset{i=1}{\sum}} Pr[X_i = 1]\\
            &=& \overset{n-1}{\underset{i=1}{\sum}} \frac{1}{n}
            = \frac{1}{n} \overset{n-1}{\underset{i=1}{\sum}} 1\\
            &=& \frac{1}{n} \frac{(n-1)n}{2}\\
            &=& \frac{n-1}{2}
    \end{array}\end{equation*}

    Wir kommen also mit $\frac{n-1}{2}$ erwarteten Zugriffen aus, was der Anforderung an den Algorithmus entsprach.
    \mbox{}\hfill$\square$.

\subsection*{Aufgabe 2}

\subsubsection*{(a)}
Sei $X$ eine nichtnegative Zufallsvariable mit Erwartungswert $\mu$. Beweisen Sie die \emph{Markov}-Ungleichung.
$$
    \forall t > 0 \, : \, Pr[X \geq t\mu] \leq \frac{1}{t}.
$$    

\textbf{Beweis:}\\

Da wir keine näheren Informationen über die Verteilung der und Anzahl der Werte haben, nehmen wir die allgemeinste
Definition.

Sei $\mathcal{L}$ ein Wahrscheinlichkeitsmaß.
Dann gilt
$$
    E[X] = \int_{x \in \Omega (X)} x d\mathcal{L} = \int_{x \in \Omega (X)} x \mathcal{L}(x) dx
$$
Wobei allgemein gilt $\mathcal{L}(x) = Pr[X = x]$.

Daraus folgt
\begin{equation*}\begin{split}
    \mu &= \int_{x \in \Omega (X)} x Pr[X = x] dx\\
        &= \underset{\underset{x \geq t \mu}{x \in \Omega (X)}}{\int} x \cdot Pr [X = x] dx + 
            \underset{\underset{x < t \mu}{x \in \Omega (X)}}{\int} x \cdot Pr [X = x]\\
        &\stackrel{x \geq 0}{\geq} \underset{\underset{x \geq t\mu}{ x\in \Omega (X)}}{\int} x \cdot Pr [X = x]\\
        &\stackrel{x \geq t \mu}{\geq} t \mu  \cdot \underset{\underset{x \geq t\mu}{ x \in \Omega (X) }}{\int} Pr [X = x]\\
        &\stackrel{\mathcal{L}\text{ subadditiv}}{\geq} t \mu \cdot Pr[X \geq t \mu]\\
\end{split}\end{equation*}

Für $t \mu > 0$ teilt man beide Sieten der Gleichung und es folgt
$$
    Pr[X \geq t \mu] \leq \frac{1}{t}
$$


\subsubsection*{(b)}

Sei $ X = \overset{n}{\underset{i=1}{\sum}} X_i$, mit $X_i \in \{ 0 , 1\}$ undabhängig und $Pr[X_i = 1] = p$. Zeigen Sie,
dass für alle $\lambda , t \geq 0$ gilt
$$
        Pr[X \geq (p + t)n] = Pr[e^{\lambda X} \geq e^{\lambda (p + t)n}] \leq E[e^{\lambda X}] / e^{\lambda(p+t)n}.
$$

\textbf{Beweis:}\\

Als erstes gilt
\begin{equation*}\begin{array}{crcl}
& X &\geq& (p + t)n\\
\stackrel{\lambda > 0}{\Leftrightarrow} & \lambda X & \geq & \lambda (p + t) n\\
\Leftrightarrow & e^{\lambda X} & \geq & e^{\lambda ( p + t) n}
\end{array}\end{equation*}

Als nächstes wenden wir die \emph{Markov}-Ungleichung an.
\begin{equation*}\begin{array}{rl}
    Pr[X \geq \lambda (p + t) n]
        &= Pr[X \geq \frac{\lambda (p + t) n}{\lambda \mu} \mu]\\
        &\stackrel{\emph{Markov}}{\leq} \frac{\lambda \mu}{\lambda (p + t) n}
        = \frac{\lambda E[X]}{\lambda (p + t) n}\\
        &= \frac{E[\lambda X]}{\lambda (p + t) n}
        = \frac{e^{E[\lambda X]}}{e^{\lambda (p + t) n}}\\
        &\leq \frac{e^{E[\lambda X]}}{e^{\lambda (p + t) n}}
\end{array}\end{equation*}

\mbox{}\hfill$\square$

\subsubsection*{(c)}

Benutzen Sie die Unabhängigkeit der $X_i$ und die Eigenschaft der Exponentialfunktion, um zu zeigen, dass
$$
    E[e^{\lambda X}] = (pe^\lambda + 1 - p)^n.
$$

\textbf{Beweis:}\\

\begin{equation*}\begin{split}
    E[e^{\lambda X}] &= E[e^{\lambda \overset{n}{\underset{i=1}{\sum}} X_i}]\\
        &= E[ \overset{n}{\underset{i=1}{\prod}} e^{\lambda X_i}]\\
        &\stackrel{\text{Unabh.}}{=} \overset{n}{\underset{i=1}{\prod}} E[e^{\lambda X_i}]\\
        &= \overset{n}{\underset{i=1}{\prod}} \left( e^{\lambda} Pr[X_i = 1] + e^0 Pr[X_i = 0] \right)\\
        &= \overset{n}{\underset{i=1}{\prod}} \left( e^{\lambda}\cdot p + 1 \cdot (1 - p) \right)\\
        &= \left( pe^\lambda + 1 - p \right)^n
\end{split}\end{equation*}

\subsubsection*{(d)}

Optimieren Sie $\lambda$, um eine möglichst gute Abschätzung für $Pr[X \geq (p + t)n]$ zu erhalten.\\

\textbf{Lösung:}\\

Nach \emph{(c)} wissen wir, dass
$$
    Pr[ X \geq (p + t)n] \leq \frac{\left( p e^{\lambda} + 1 - p\right)^n}{e^{\lambda (p + t) n}}
$$
gilt. Nun wollen wir das Minimum bezüglich $\lambda$ finden, dazu bilden wir zunächst die erste ableitung bezüglich $\lambda$.

\begin{equation*}\begin{array}{rl}
    \frac{\lambda}{\partial \lambda} \left( \frac{p e^{\lambda} + 1 - p}{e^{\lambda (p + t)}} \right)^n
        &= n \cdot \left( \frac{pe^\lambda + 1 - p}{e^{\lambda (p + t)}} \right)^{n-1} 
        \cdot \frac{\lambda}{\partial \lambda} \left( \frac{pe^\lambda + 1 - p}{e^{\lambda (p + t)}}\right)\\
        &= n \cdot \left( \frac{pe^\lambda + 1 - p}{e^{\lambda (p + t)}} \right)^{n-1} \cdot
            \frac{pe^\lambda - (pe^\lambda + 1 - p)(p+t)}{e^(\lambda(p+t)}
\end{array}\end{equation*}

Nun such wir die Nullstellen der ersten Ableitung. Da wir hier nur Produkte haben müssen wir nur die einzelnen Faktoren überprüfen.
Bei Brüchen müssen wir nur den Zähler testen.\\

\begin{enumerate}[1. {Fall}]
    \item 
        \begin{equation*}\begin{array}{crcl}
            & 0 &=& pe^\lambda + 1 - p\\
            \Leftrightarrow & p-1 &=& pe^\lambda\\
            \Leftrightarrow & e^\lambda &=& \frac{p-1}{p}
        \end{array}\end{equation*}
        Diese Nullstelle ist für uns nicht möglich, da $\lambda > 0$ immer $e^\lambda > 1$ zur Folge hat.
    \item
        \begin{equation*}\begin{array}{crcl}
            & 0 &=& pe^\lambda - (pe^\lambda + 1 -p)(p+t)\\
            \Leftrightarrow & 0 &=& pe^\lambda (1 - p - t) - (1-p)(p+t)\\
            \Leftrightarrow & e^\lambda &=& \frac{(1-p)(p+t)}{(1-p-t)p}
        \end{array}\end{equation*}
        Wir haben auch nachgeprüft, dass es sich hierbei um ein Minimum handelt, hatten aber keine Lust die Rechung dafür auch noch
        aufzuschreiben.
\end{enumerate}

Mit dem Minimum können wir nun den optimalen Wert errechnen.

\begin{equation*}\begin{split}
    Pr[X \geq (p+t)n] & \leq \left( \frac{p e^\lambda + 1 -p}{(e^\lambda)^{p+t}} \right)^n\\
        &\stackrel{Min.}{=} \left( \frac{ p (\frac{(1-p)(p+t)}{(1-p-t)p}) + 1 - p}{(\frac{(1-p)(p+t)}{(1-p-t)p})^{p+t}} \right)^n\\
        &= \left[ \left(\frac{p}{p+t}\right)^{p+t} \left( \frac{1-p}{1-p-t}\right)^{1-p-t} \right]^n
\end{split}\end{equation*}

\subsubsection*{(e)}

Die Abschätzung aus der VL war
$$
    Pr [|X - \mu| \geq \delta \mu] \leq 2e^{\frac{\min \{ \delta, \delta^2\}}{4} \mu}
$$

Zunächst sind wir immer noch Exponentiell in $n$ beschränkt. Nun haben wir eine geringfügig andere Betrachtung.
Wo wir vorher $|X-\mu| \geq \delta \mu$  betrachten wir nun
$$
    X \geq (p+t)n = \mu + tn \Leftrightarrow X - \mu \geq \frac{tn}{\mu} \mu
$$
haben wir den Fehler nur zu einer Seite betrachtet, da in der Abschätzung aus der Vorlesung der Betrag stand.
Dies erklärt das Fehler des Faktors $2$ aus der Abschätzung der VL, da das Bernouli Experiment symmetrisch ist,
sollte die Wahrscheinlichkeit für die andere Hälfte des Betrages gleich zur ersten sein.\\


Was wir nicht wissen ist, wo da $\delta^2$ hin verschwunden ist. Wir haben uns nicht genauer angesehen,
was passiert, wenn $\frac{p+t}{n}$ klein wird, ob sich dort die Funktion anders Verhält.\\

Ansonsten sehen beide Funktionen recht ähnlich aus, wenn man einen Basiswechsel nach $e$ durchführt.
\label{LastPage}
\end{document}
