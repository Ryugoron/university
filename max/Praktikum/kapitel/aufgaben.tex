\section{Aufgaben und Tätigkeiten}

\subsection{Tätigkeitsumfeld}

Wie schon vorher erwähnt in \emph{Tätigkeiten und Arbeitsergebnisse} noch einmal genauer erläutert wird. Wurde kurz nach dem Beginn meines Praktikums der komplette Code weggeschmissen und das System als ganzen neu Entworfen.\\
An diesem Teil der Arbeit hat sich das komplette Programmierteam inklusive meines Betreuers und Programmierkoordinators Janosch Asen beteiligt. Hier fand ein Großteil der Arbeit an Whiteboard und Diskussionsrunden statt. Alles in allem eine gute Zusammenarbeit, die den Zusammenhalt innerhalb des Teams auch gefördert hat. Dieses refactoring war nötig, da bis zu diesem Zeitpunkt eher alles als Prototyp entworfen wurde. Vieles konnte schon getestet werden, aber alles zusammengenommen wurde die Struktur grausam und nicht skalierbar.\\

Nachdem der Großteil des Planes stand, musste zunächst die Struktur umgesetzt werden. Dies fand größtenteils im Side-by-Side programming statt, da auf diese weise ein umfassendes Wissen über das Gesamtsystem geschaffen wurde und wir bei einer schnell wechselnden Belegschaft (es waren fast alle Programmierer nur Praktikanten) kein Expertenwissen schaffen wollten. Zum anderen konnten so schnell Probleme gelöst werden, wenn schnell viele Ideen zu diesem Problem kommen, ohne das jeder neu in die Problematik eingearbeitet werden musste.\\


Als auch dieser Schritt getan war, konnte überwiegend an einzelnen kleinen Projekten gearbeitet werden. Diese waren größtenteils unabhängig von einander und in einzelnen Tickets und Milestones angeordnet. Trotzdem war es immer noch möglich mit Problemen umzugehen und sich helfen zu lassen oder anderen zu helfen, aber es wurde nicht mehr so eng zusammen gearbeitet, wie in der groben Programmstruktur. Der einzige, mit dem man immer eng zusammen gearbeitet hat, war Herr Asen, da dieser bis zum jetzigen Zeitpunkt unser Kunde war und uns immer schnell Rückmeldung geben konnte, ob das getane so gemeint war oder schon richtig umgesetzt war.


\subsection{Aufgabe und Ziele}

Im Projekt sollte ich das Programmierteam unterstützen und die Entwicklung des selbigem voran bringen. Da des Unternehmen wenig Geld zur Verfügung hatte, gerade um Vollzeit Programmierer einzustellen, sollte ich in der Zeit meines Praktikums einen regulären Programmierer darstellen. Da keiner der Anwesenden Erfahrung mit großen Projekten hatte, war die Devise Lerning-By-Doing. Im Grunde war es meine Aufgabe, die von Herrn Asen gestellt Aufgaben in Zusammenarbeit mit den anderen Programmieren zu erledigen. Sobald ich einige Erfahrung gesammelt hatte. Sollte ich auch zunehmend selbstständig an meinen Aufgaben arbeiten.\\
Da zu Beginn noch keine Aufgaben gesteckt waren, wurde an keiner Stelle festgesteckt, was ich im Verlauf erreichen sollte. Das vorrangige Ziel, war es nach dem Refactoring wieder eine Lauffähige Version zu erhalten. Dies hat auch die meiste Zeit des Praktikums in Anspruch genommen.\\

Meine eigenen Ziele sahen nicht ganz so Ziel gerichtet aus. Da ich bis zum damaligen Zeitpunkt noch nie ein Arbeitsverhältnis eingenommen habe, hatte ich keine Ahnung, wie ich auf die Forderungen, die auf mich zukommen würden, reagieren würde. Ich wollte also erst einmal sehen, ob ich unter Arbeitsanforderungen produktiv arbeiten könne.\\
Als nächstes musste ich Verifizieren, ob meine übliche Herangehensweise in einem Unternehmen angenommen wird. Ich habe die Eigenschaft immer lange über ein Problem nachzudenken, bevor ich auch nur anfange etwas zu tun. Dies hätte sich als negativ auswirken können, wenn schnelle Ergebnisse verlangt sind. Was mich auf jedenfall noch interessiert hatte, war es das gelernte aus Softwaretechnik in der Praxis zu erleben und umzusetzen. Die Themen der Vorlesung sind sehr praxisorientiert und da ich zu diesem Zeitpunkt noch nie gearbeitet hatte, konnte ich viele Themen der Vorlesung nicht nachvollziehen. Wenn man aber alles einmal erlebt hat, macht dies die Vorlesung viel leichter verständlich. Es bietet sich unter Umständen ein, erst sein Praktikum zu machen und danach Softwaretechnik zu hören. Zu guter Letzt wollte ich erfahren, wie ich in größeren Gruppen arbeite, da ich von Natur aus eher eine stille Person bin.

\subsection{Tätigkeiten und Arbeitsergebnisse}

Zu Beginn des Praktikums habe ich eine Geheimhaltungserklärung unterschrieben, dass ich nichts genaues über das Projekt erklären darf. Da zum Zeitpunkt zu dem ich den Bericht nun schreibe, die Beta-Version schon öffentlich ist, kann ich konkreter werden, als ich es zum Ende meine Praktikums schon gekonnte hätte, aber ins Detail kann ich immer noch nicht gehen.

\subsubsection{Einarbeitung}

In den ersten zwei Wochen, in denen ich noch neben der Uni gearbeitet habe, habe ich mich zunächst in das schon vorhandene Programm eingelesen. Bei Guido handelt es sich im groben um eine Webseite und ich habe bis zum damaligen Zeitpunkt nur selten mit HTML und PHP und eigentlich noch nie mit Javascript gearbeitet. Zum Programmieren haben wir jedoch GWT\footnote{GWT (Google Web Toolkit) ist ein Crosscompiler, der Javacode mit speziellen Bibliotheken nach Javascript compilieren kann.} verwendet. GWT ist eine Sammlung von Bibliotheken und eine Sammlung von Precompilern und Compilern, die aus Java Code Javascript generieren können. Da ich durch die Uni Java Programmieren konnte, musste ich mich zunächst nur in den ungefähren Aufbau eines HTML Dokuments einlesen und wie man in Javascript an Probleme heran geht. Später habe ich mich genauer damit beschäftigt um einige Methode, zu denen es von GWT keine Interfaces gab in nativem Javascriptcode zu schreiben und in Java einzubinden.\\

Bei der Einarbeitung gab es keine Probleme, aber meinen Zielen hat es nicht unbedingt geholfen. Dass ich mich in neue Situationen einlesen kann wusste ich vorher schon.

\subsubsection{Refactoring und Strukturdesign}

Wie schon mehrfach erwähnt wurde nach wenigen Wochen der bisherige Prototyp des Programms weggeworfen und wir haben uns daran gemacht das komplette Programm neu zu entwerfen. Da wir an viele Stellen gesehen hatte, wo es Probleme geben kann und die Anforderungen langsam eine klarere Struktur bekamen, konnten wir in etwa abschätzen, was wir brauchen würden.\\

Der Plan war, ein reaktives System umzusetzen, dass in großen Teilen indirekt über einen Eventbus kommuniziert. So kann ein Objekt ganz allgemein sagen, dass gerade ein Ereignisse eingetroffen ist (z.B. ein neues Item soll zum aktuellen Zeitpunkt angezeigt werden) und alle Teile des Systems, die es interessiert, können diese Nachricht erhalten. Der Teil, der durch Mausklick oder anderen Trigger, was passieren muss, brauch so keine Ahnung zu haben, wen er alles ansprechen muss. Bei etwaigen Erweiterungen, muss dieser Teil des Codes nicht mehr verändert werden, sondern nur noch die neue Komponenten darauf eingestellt werden, die Nachricht zu empfangen. Der andere wichtige Teil, die wir im System umgesetzt haben, ist MVP (Model View Presenter), der analog zum in Softwaretechnik gezeigten MVC (Model View Controller) funktioniert. Später haben wir aufgrund der Komplexität den Presenter doch nochmal aufgeteilt, sodass wir sowohl Presenter als auch Controller hatten.\\

Was meine Ziele betrifft war dieser Teil des Praktikums ein großer Erfolg. Meine Natur eher länger über ein Problem nachzudenken, war an dieser Stelle von Vorteil, da im Entwurf an vieles Gedacht wurde, was erst ein halbes Jahr später erst in Angriff genommen wurde. Ein kleiner Nachteil war, dass wir im Team alle dazu geneigt waren zu viel nachzudenken und das System über zu designen. Es täte dem Team gut, wenn eine durchsetzungsfähige Person da wäre, die einspringt, wenn genug entworfen wurde. Sonst ist man mit dieser Phase nie fertig. Meine Schüchternheit war auch kein Problem. Die ganze Atmosphäre war sehr familiär und man konnte gut mit den anderen diskutieren.\\

Ein kleiner Misserfolg ist allerdings, dass ich im Nachhinein, nicht jeder ganz an den Entwurf gehalten hat. Das ist, solange es Verbesserungen sind, kein Problem, allerdings wurde der Entwurf zu Gunsten von Entscheidungen abgeändert, die wir eigentlich verhindern wollten und die das System an manchen Stellen schon nach ein paar Monaten schlecht Wartbar gemacht haben.

\subsubsection{Umsetzung und Erweiterung}

Nachdem der Entwurf soweit fertig war, haben wir das Grundgerüst fertig gestellt. Wie zuvor beschrieben, haben wir dabei noch eng zusammen gearbeitet. Gerade weil an vielen Stellen noch nicht fertig geklärt war, wie wir es Umsetzten wollen. Wir haben im Programmentwurf noch keine große Erfahrung gehabt, so dass wir nicht an alles denken konnten.\\
Nachdem das Eventsystem stand und wir für MVP alle Interfaces und schon mehrere abstrakte Klassen fertig hatten, konnten wir Anfangen aus dem Prototypen vorhandene Funktionalität nach zubauen. An dieser Stelle habe ich erst einmal einfache Aufgaben übernommen, da ich mich mit den Probleme, die schon einmal aufgetreten sind, noch nicht aus kannte.

Als das System soweit stand, konnte dazu übergegangen werden neue und schon einmal vorhandene Features einzubauen. Mein Tätigkeitsfeld waren unter anderem das einbinden der Youtube-API. Dieses API wurde bis zum jetzigen Zeitpunkt bestimmt acht mal implementiert von unserer Seite und funktioniert immer noch nicht einwandfrei. Als nächstes musste zu jeder Tour eine zeitliche Anordnung realisiert werden, über die ich mir Gedanken machen durfte. Und neben vielen Kleinigkeiten durfte ich zuletzt noch eine Verwaltung schreiben, wie z-Layer (in HTML) an unsere dynamisch angeordneten Items vergeben werden, ohne dass man das komplette Dokument neu aufbauen muss, da man auf einem Layer keinen Platz mehr hat.\\

An dieser Stelle habe ich gelernt, wie ich damit umgehen kann, wenn ich vorher lange über ein Problem nachdenke, ohne anfange zu Programmieren. Ich hab alles, über das ich nachgedacht habe gut dokumentiert und in textueller Form im Projekt hinterlegt. Das hat erstens den Fortschritt belegt und zweitens schon in mehreren Bereichen dem nachfolgenden Programmierer gezeigt, wo Probleme auftreten können. Im Falle der Timeline habe ich gar nicht angefangen etwas umzusetzen, aber habe alle Fehler und Probleme dokumentiert, die allgemein oder in der aktuellen Struktur auftreten können. Dies hat bei der Umsetzung von der nächsten Person geholfen, als diese die ersten fünf Fehler gemacht hatte, die ich aufgeschrieben habe.\\

\subsubsection{Nachwirkung}

Dies kann ich Rückblickend auf die Hinarbeit der Beta-Phase ein halbes Jahr später sagen, da es sich im Praktikum bereits angebahnt hat. Da Aufgrund der knappen Finanzlage ein Zeitdruck hinter allem Stand und wir getrieben wurden eine präsentationsfähige Software aufzustellen, mussten wir an vielen Stellen Kompromisse Eingehen.\\
Gegen Ende hieß es immer, dass die Beta nächste Woche herauskommen sollte. Dies hat dazu geführt, das an vielen Stellen nur Bugfixes eingefügt wurden und das Problem an sich nicht mehr gelöst werden konnte. Was nach dem Beta release dazu führte, dass wir nicht mir wissen, wo nun gerade Fehler im Code sind, die wir noch korrigieren müssen und das ganze System äußerst labil läuft.\\

Dies ist ein großer Fehler von unserer Programmiererseite gewesen. Erstens hätte man die Bugfixes besser dokumentieren sollen und zweitens die Deadline einmal weit hinausschieben sollen und nicht viele male ein bisschen.\\

\subsubsection{Ergebnis}

Insgesamt war das Praktikum erfolgreich. Wir habe für die Beta-Phase einen Grundstein gelegt, der innerhalb des Praktikums schon funktionierte. Ich habe vieles gelernt, wie man in größeren Gruppen arbeitet und wie man die Anforderungen von einer Person bestimmen kann, die keine Ahnung von Informatik hat. Ich habe viele Fehler gemacht und hoffentlich aus allen etwas gelernt. Darüber hinaus hat das Praktikum mir Spaß gemacht und ich helfe dem Unternehmen immer mal wieder, wenn etwas getan werden muss.
