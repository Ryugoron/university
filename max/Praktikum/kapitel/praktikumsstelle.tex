\section{Praktikumsstelle}

Mein Berufspraktikum habe ich bei der Firma scheideggerasen absolviert. Scheideggerasen ist eine Agentur für asthetische Kommunikation, die neben Werbung und Kommunikationsberatung auch an Projekten aus dem Kultur- und Medienwissenschaftlichen Bereich arbeitet. Das momentane Projekt von scheideggerasen ist FarFromHompage (FFHP), an dem auch ich gearbeitet habe, und das zur eigenen Gründung führen soll. Am Projekt FFHP arbeiteten als ich dort war, sechs fest mitarbeitende Personen, die ergänzt wurden durch freie Mitarbeiter, Praktikanten und Kooperationsvereinbahrungen, die Teilweise von der Gründerinitiative der FU - Berlin vermittelt wurden. Die Anzahl der Praktikanten schwankte über meine Anwesenheit, aber im Durchschnitt haben etwa 18 Personen an dem Projekt gearbeitet. Davon waren zu Beginn vier Programmierer. Die Branche des Unternehmens kann man der IT-, dem Kommunikationsmedien- oder auch dem Kulturbereich zuordnen.\\
Als ich dort arbeitete war das Unternehmen in der Vorgründungsphase und wurde durch verschiedene Förderungen von Seiten der FU - Berlin unterstützt. Ziel war es das Programm soweit zu entwickeln, dass man eine Beta-Phase zu testen erreichen konnte. Diese Phase wurde ein Halbesjahr nach Abschluss meines Praktikums mit freundlicher Hilfe aller ehemaligen Praktikanten erreicht. Bei dem von FFHP entwickelten Programm Guido handelt es sich um eine Art \emph{Schnittprogramm} für das Internet. Man kann als \emph{Autor} es allen Internetmedien eine \emph{Tour} zusammenstellen und dann auf der Internetseite von FFHP presentieren. Ein Betrachter dieser Tour kann sie sich einfach ansehen oder in den gezeigten Median weiter surfen, da die ganze Tour interaktiv mit dem WorldWIdeWeb arbeitet.\\
Da es sich um ein kleines Unternehmen handelt, ist die Leitungsstruktur sehr flach. Gegründet wurde das Unternehmen vom Medien-, Theaterwissenschaftler und Philosophen Manuel Scheidegger und dem Medienwissenschaftler und Medienkünstler Janosch Asen, deren Nachnamen das Unternehmen auch seinen Namen verdankt. Mein Betreuer war der zweit genannte Gründer Janosch Asen.\\

Ich war als Praktikant bei der Entwicklung und der Programmierung des Frontends von Guido tätig. Da es zum damaligen Zeitpunkt noch recht wenig Programmierer und sehr vielfälitge Aufgaben gab, kann man nicht direkt einen Aufgabenbereich herrausstellen, für den ich Tätig war. Und bei einer Größe von vier Programmierern gibt es auch keine unterschiedlichen Abteilungen. Dadurch war das Arbeitsklima dennoch sehr angenehm. Sind Probleme an einer Stelle aufgetreten, konnte man die anderen leicht hinzuziehen, da zu diesem Zeitpunkt Kenntnisse über das gesammte System noch nicht schwer zu bekommen war und das Projekt auch noch nicht groß war. Da unser Chef und Betreuer immer bei uns war und obwohl er von Programmierung keine Ahnung hat, konnte er uns immer konkret neue und Sinnvolle Aufgaben geben. Neben der Koordination der Programmierer ist Janosch Asen für die Bürokratie, die Öffentlichkeitsarbeit und die Konzept- und Ideenentwicklung in der Firma zuständig.\\

Auf die Praktikumsstelle wurde ich erstmals im Sommer 2010 aufmerksam, als die Stellenausschreibung auf dem Mailverteiler von Spline herum geschickt wurde. Damals wurde aber zunächst nur ein Programmierer gesucht und mein Freund Alexander Steen  hat die Stelle bekommen. Ein halbes Jahr später wurden wieder neue Leute gesucht und Alex hat mich zum einen geworben und zum anderen Empfohlen. So habe ich vor Weihnachten 2010 ein Vorstellungsgespräch gehabt und konnte gegen Beginn des neuen Jahres (2011) dort anfangen zu arbeiten. Vor diesem Zeitpunkt habe ich weder dort noch irgendwo anders jemals gearbeitet.\\
Zu Beginn war mein Praktikum in der Studienzeit und ich hab 1-2 Tage die Woche für etwa 12h gearbeitet. Sobald das Semester vorrüber war habe ich auf eine 40h Woche erhöht. Die Arbeitszeiten waren frei wählbar ich bin dennoch meist morgens gegen 10 Uhr ins Büro gekommen. Da ich keinen Schlüssel zum Gebäude hatte, war ich immer darauf angewiesen, dass schon eine Person vor mir im Büro war.\\
Das Praktikum an sich war nicht vergütet, ich wurde allerdings für eine Teilstelle über das darauf folgende Semester bezahlt.\\

Das mehrwöchige Berufspraktikum war für mich sehr interessant, da ich zu diesem Zeitpunkt noch nie in einem größeren Projekt gearbeitet habe und es mir auch nicht recht vorstellen konnte. Da es sich um ein recht junges Unternehmen handelte, hat man einen tiefen Einblick in die komplette Strucktur und Bürokratie eines solchen Unternehmens erhalten. Vom Programmiertechnischen Standpunkt auch, habe ich auch viel gelernt, da noch kurzer Zeit, als ich angefangen hatte zu arbeiten, der komplette Code refactored wurde.\\
So konnte ich daran teilhaben, die komplette Strucktur eines Projektes mitzudiskutieren, zu entwerfen und mitzubestimmen. Dabei konnte man vieles vom Programmentwurf was man in Softwaretechnik gelernt hatte umsetzen. Der Entwurf an sich, war von außen betrachtet, ein bisschen zu viel für so ein Projekt. Die Modularisierung und Trennung der einzelnen Teile war recht lose gehalten. Nach einem halben Jahr allerdings konnte man beobachten, dass, obwohl das große Konzept nicht durchbrochen wurde, in Einzelfällen ein solcher Zusammenhalt bestand, dass es nicht zu trennen war und vor allem auch nicht mehr Wartbar. Die Geschwindigkeit, mit der wir den Code an eine nicht mehr wartbare Stelle gebracht haben, hat uns alle (zumindest hoffe ich das) ein bisschen Disziplin gelehrt.

