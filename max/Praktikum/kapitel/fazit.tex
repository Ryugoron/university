\section{Einsichten und Fazit}
Dieser Abschnitt beschreibt, welche wichtigsten Einsichten (Aha-Erlebnisse) ich aus dem
Praktikum mitgenommen habe.

\subsection{Technik}

\subsubsection{GWT}

Das Google Web Toolkit war für mich in jedem Fall wunderbar zu benutzten, da ich von Javascript keine Ahnung hatte und auf diese Weise erst einmal langsam heran geführt werfen und dabei schon trotzdem effizient arbeiten konnte. Eine Falle in die man am Anfang aber schnell läuft ist, dass durch die Javascript Umsetzung im Hintergrund manche Operationen und Datentypen nicht so effizient sind, wie man glaubt. HashMaps sind in Javascript nach versuchen auf jeden Fall nicht so schnell gewesen, wie wir erfahren mussten. Auf anderen Datentypen sind manche Operationen nicht portiert worden. Zum Beispiel haben wir auf TreeMaps kein floorKey und ceilKey gehabt, die aber dringend gebraucht wurden. Nach dem rein lesen und den ersten Wochen Verzweiflung konnte man aber recht gut damit umgehen.

\subsubsection{Git}

Wir haben mit dem Versionsverwaltungssystem Git gearbeitet. Git an sich ist sehr mächtig, da wir aber keine Erfahrung damit hatten, ist es zu Beginn öfters passiert, dass man seine kompletten Fortschritte vernichtet hat.\\
Ein kleiner Crashkurs sollte dem Abhilfe schaffen, wir hatten aber leider keinen.

\subsubsection{Whiteboards}

Whiteboards sind wunderbar, wenn man über einen Entwurf diskutieren will. Das einzige Problem ist es ohne einen Schriftführer alles zu dokumentieren. Und den hatten wir aufgrund der knappen Belegschaft nicht, da alle mit diskutiert haben. Aber ein Whiteboard, Flipchart oder eine Tafel sollte in keiner Besprechung fehlen. Softwaregestütze Präsentationen wirkten dagegen meist fehl am Platz bei den Besprechungen.

\subsection{Methodik}

Wenn man einen Entwurf gemacht hat, dann muss man sich nicht sklavisch daran halten. Aber wenn man entdeckt, dass es in einer speziellen Situation keinen Sinn macht nach dem Entwurf vorzugehen, dann sollte man nicht eigenmächtig beschließen ihn abzuändern und schon gar nicht das ganze undokumentiert zu tun. Wenn man an einen solchen Punkt kommt, sollte man mit den anderen darüber reden und zusammen entscheiden, ob eine Änderung nötig ist oder wie diese Auszusehen hat. Bei uns ist es an einer Stelle vorgekommen, dass eine Person eigenmächtig Direktzugriff durch 3 Abstraktionsebenen gewährt hat, weil es an der Stelle gut besser ging. Wir haben gesehen, dass die Abstraktion an dieser Stelle zu schwer fällig war, aber wir haben den Eingriff erst bemerkt, als wir durch Zufall über einen Fehler an dieser Stelle gestolpert sind. Durch die Veränderung haben wir diesen Fehler auch sehr schlecht gefunden.\\

Wenn man eine Beschreibung bekommen hat, die nicht vollständig Spezifiziert ist, sollte man auf jeden Fall nochmal nachfragen. Dies tritt gerade auf, wenn der andere kein Informatiker ist. Man sollte sich an dieser Stelle die Lücken nicht selber zu denken, da am Ende etwas heraus kommen kann, was überhaupt nicht gewünscht ist. Da wir einen engen Kontakt mit unserem Kunden, der immer im selben Raum saß, konnten wir als Lösung immer einen schnelle Prototypen bauen und weiter entwickeln, in die Richtung, die er meinte. Sollte er aber einmal nicht so perfekt zur Verfügung stehen, sollte man sich alle Fragen vorher überlegen oder man läuft Gefahr alles mehrfach neu schreiben zu müssen.

Ruhe bewahren ist das oberste Motto einer Besprechung oder in einer Diskussion. Ich selber neige dazu, wenn ich nicht verstanden werden, etwas aggressiv in meiner Sprechweise zu werden. Dies hat aber noch nicht geholfen, dass es der andere besser versteht. Dies viel aber nur auf, da unsere Besprechungen immer allgemein sehr ruhig waren. Als Training war es sehr gut Tutor zu sein. Man darf die selben Fragen ziemlich oft erklären und bleibt darüber recht ruhig, auch wenn man es zehn mal erklären darf.

\subsection{Sonstiges}

Wir waren zu Beginn eine recht kleine Gruppe, in der sich die meisten Probleme von selbst gelöst haben. Sobald wir aber etwas mehr wurden, stellte sich raus, dass wir zu wenig Disziplin und Dokumentation aller unserer Arbeitsergebnisse, Aufgabenverteilung und Probleme hatten. In größere Unternehmen hat sich an dieser Stelle schon ein Workflow eingestellt, den wir noch finden mussten. Gerade regelmäßige Besprechung sind bei einer Gruppengröße von mehr als drei Mitgliedern unabdingbar, da sonst niemand einen Überblick hat.\\

Es gibt manche Personen, die reagieren zwar, darauf, wenn man ihnen sagt, dass sie etwas falsch tun oder es anders besser geht, aber trotzdem weiter versuchen ihre Idee durchzusetzen. Auf diese Leute sollte man ein Auge haben. Es ist nicht schlecht zum entwerfen eines Prototyps, wie wir feststellen konnten, aber im Endergebnis ist ein bisschen mehr Disziplin nötig. Also sollte man darauf achten, dass diese Personen sich an die Richtlinien halten.\\

Ich selbst habe mich unterschätzt, wenn es darum geht, große Aufgaben zu bewältigen. Ich bin, wie nicht nur beim Umsetzung von großer Software, von einer großen Aufgabe überwältigt, so dass ich nicht weiß, wo ich beginnen soll, oder wie ich das ganze Überhaupt schaffen kann. Doch wenn man sich immer nur auf kleine Teilstücke konzentriert, kann man es doch auch große Aufgaben zügig und gut schaffen.

\subsection{Fazit}

Das Praktikum hat mir an einigen Stellen gezeigt, wie ich mein Wissen in die Praxis umsetzen kann. Das war schön einmal zu sehen, aber ich kann es mir nicht vorstellen später einmal als Programmierer zu arbeiten. Diese Aufgabe ist mir doch zu eintönig und Öde. Im Praktikum habe ich allerdings gelernt, mehr Lösungen von anderen zu lesen, die das selbe Problem, wie ich hatten. Im Informatik Studium wird man sollten dazu geführt, sich Lösungen von anderen durchzulesen um sein Problem zu lösen. Dort muss man oft seine Probleme von Grund auf selber lösen und darf jeden Fehler einmal selber begehen. Im Grunde genommen gut, da man aus seinen Fehlern lernen kann, aber es ist unvernünftig schon vorhandenes Wissen zu verwerfen um alles noch einmal neu zu machen.\\
Das Studium bringt mir an sich aber mehr Wissen. Im Praktikum war es oft immer wieder das selbe das man tun musste. Im Studium gibt es mehr Abwechslung und vielfältigere Probleme. Es macht mir insgesamt mehr Spaß.\\
Mein Studien verhalten werde ich nach meinen Erfahrungen im Praktikum eher nicht ändern. Vieles von dem was ich erfahren habe war ein Erfolg oder eine Bestätigung. Die Richtung meines Praktikums, das heißt insbesondere Webprogrammierung, ist nicht mein Interessengebiet, es hat mich eher bestätigt, nicht weiter in diese Richtung zu gehen.\\

Allen anderen, die ein Praktikum suchen, kann ich es auf jeden Fall empfehlen, in ein kleines Unternehmen zu gehen. In großen Unternehmen kann man unter Umständen ein feste Struktur lernen, wie man an große Projekte heran geht. Man wird aber nicht lernen, warum dies so gemacht wird, sondern wird nur dazu gedrängt es so zu machen. In einem kleinen Unternehmen ist man sehr nah an der Entscheidungsfindung und man wird vielleicht sogar daran beteiligt. Man darf einmal jeden Fehler begehen, den man begehen kann. Gerade zum lernen ist dies ein Vorteil. In kleinen Unternehmen hat man auch eine größere Auswahl woran man arbeiten möchte. Bei großen Unternehmen wird man meist ohne Wahl vor eine Aufgabe gesetzt.\\

Deshalb kann ich jedem nur ein Praktikum bei einem Startup empfehlen. Wir habe unter anderem auch schon viele Mitstudenten an FFHP vermittelt und den meisten hat das Praktikum Spaß gemacht.

