\documentclass[11pt,a4paper,ngerman]{article}
\usepackage[bottom=2.5cm,top=2.5cm]{geometry}
\usepackage[ngerman]{babel}
\usepackage[utf8]{inputenc}
\usepackage[T1]{fontenc}
\usepackage{ae}
\usepackage{amssymb}
\usepackage{amsmath}
\usepackage{amsthm}
\usepackage{graphicx}
\usepackage{fancyhdr}
\usepackage{fancyref}
\usepackage{listings}
\usepackage{xcolor}
\usepackage{stmaryrd}
\usepackage{paralist}

\usepackage[pdftex, bookmarks=false, pdfstartview={FitH}, linkbordercolor=white]{hyperref}
\usepackage{fancyhdr}
\pagestyle{fancy}
\fancyhead[C]{Stochastik I}
\fancyhead[L]{Übung 10}
\fancyhead[R]{WiSe 2013/14}
\fancyfoot{}
\fancyfoot[L]{}
\fancyfoot[C]{\thepage \hspace{1px} of \pageref{LastPage}}
\renewcommand{\footrulewidth}{0.5pt}
\renewcommand{\headrulewidth}{0.5pt}
\newcommand{\set}[1]{ \{ #1 \}}
\newcommand{\Prob}{\mathbb{P}}
\setlength{\parindent}{0pt}
\setlength{\headheight}{0pt}

\newcommand{\N}{\mathbb{N}}
\newcommand{\Q}{\mathbb{Q}}
\newcommand{\R}{\mathbb{R}}
\newcommand{\bigO}{\mathcal{O}}
\newcommand{\Rarr}{\Rightarrow}
\newcommand{\rarr}{\rightarrow}
\newcommand{\Pot}{\mathcal{P}}
\newcommand{\abs}[1]{ |#1|}
\newcommand{\solved}{$\mbox{}$ \hfill $\square$}
\newcommand{\Epsilon}{\mathcal{E}}

\newcommand{\maxw}{$^\curlywedgedownarrow$}
\newcommand{\alex}{$^\dagger$}
\newcommand{\marcel}{$^\diamondsuit$}

\date{}
\title{Übung 10}
\author{Max Wisniewski\maxw, Alexander Steen\alex, Marcel Ehrhardt\marcel}


%%
%% Enviroments for proofs and lemmas
%%
\newtheorem{prop}{\bfseries Behauptung}
\newtheorem{lemma}{\bfseries Lemma}

\begin{document}

\lstset{language=Pascal, basicstyle=\ttfamily\fontsize{10pt}{10pt}\selectfont\upshape, commentstyle=\rmfamily\slshape, keywordstyle=\rmfamily\bfseries, breaklines=true, frame=single, xleftmargin=3mm, xrightmargin=3mm, tabsize=2, mathescape=true}

\renewcommand{\figurename}{Figure}

\maketitle
\thispagestyle{fancy}

\begin{center}
\textbf{Hinweis}: Die Angabe, wer welche Aufgabe in \LaTeX\ formuliert hat, ist als hochgestelltes Symbol der Aufgabenüberschrift zu entnehmen.
\end{center}


\section*{Aufgabe1}
Konstruieren Sie Zufallsvariablen $X$ und $Y$, die nicht unabhängig sind, so dass die
Erwartungswerte von $X$, $Y$ und $XY$ existieren und $E(XY) = E(X)E(Y)$. \\

\textbf{Lösung}: \\
Wahrscheinlichkeitsraum $\Omega = {\omega_1, \omega_2, \omega_3}$ mit Gleichverteilung.

Sei $X$ gegeben durch
\begin{equation}
X(x) = \begin{cases}
         -1 & \text{, falls $x = \omega_1$}\\
         0 & \text{, falls $x = \omega_2$}\\
         1 & \text{, falls $x = \omega_3$}
       \end{cases}
\end{equation}
und $Y$ gegeben durch
\begin{equation}
Y(y) = \begin{cases}
         1 & \text{, falls $y = \omega_1$}\\
         0 & \text{, falls $y = \omega_2$}\\
         1 & \text{, falls $y = \omega_3$}
       \end{cases}
\end{equation}

$XY(\omega_1) = X(\omega_1)Y(\omega_1) = -1 * 1 = -1$

$XY(\omega_2) = ... = 0*0 = 0$

$XY(\omega_3) = 1$

$E(XY) = 0$

$E(X) = 0$

$E(Y) = 2/3$

z.z. x x, y s.d. ereignisse ${X >= x}$ und ${Y >= y}$ nicht unabh. sind

${X >=1}  = {omega3}$

${Y >=1} = {omega1, omega3}$


$P({X >=1} cap {Y>=1}) = P(omega3} = 1/3$

$P({X>=1})P({Y>=1}) = 1/3 * 2/3 = 2/9$


\label{LastPage}
\end{document}
