\documentclass[11pt,a4paper,ngerman]{article}
\usepackage[bottom=2.5cm,top=2.5cm]{geometry}
\usepackage[ngerman]{babel}
\usepackage[utf8]{inputenc}
\usepackage[T1]{fontenc}
\usepackage{ae}
\usepackage{amssymb}
\usepackage{amsmath}
\usepackage{amsthm}
\usepackage{graphicx}
\usepackage{fancyhdr}
\usepackage{fancyref}
\usepackage{listings}
\usepackage{xcolor}
\usepackage{stmaryrd}
\usepackage{paralist}

\usepackage[pdftex, bookmarks=false, pdfstartview={FitH}, linkbordercolor=white]{hyperref}
\usepackage{fancyhdr}
\pagestyle{fancy}
\fancyhead[C]{Stochastik I}
\fancyhead[L]{Übung 10}
\fancyhead[R]{WiSe 2013/14}
\fancyfoot{}
\fancyfoot[L]{}
\fancyfoot[C]{\thepage \hspace{1px} of \pageref{LastPage}}
\renewcommand{\footrulewidth}{0.5pt}
\renewcommand{\headrulewidth}{0.5pt}
\newcommand{\set}[1]{ \{ #1 \}}
\newcommand{\Prob}{\mathbb{P}}
\setlength{\parindent}{0pt}
\setlength{\headheight}{0pt}

\newcommand{\N}{\mathbb{N}}
\newcommand{\Q}{\mathbb{Q}}
\newcommand{\R}{\mathbb{R}}
\newcommand{\bigO}{\mathcal{O}}
\newcommand{\Rarr}{\Rightarrow}
\newcommand{\rarr}{\rightarrow}
\newcommand{\Pot}{\mathcal{P}}
\newcommand{\abs}[1]{ |#1|}
\newcommand{\solved}{$\mbox{}$ \hfill $\square$}
\newcommand{\Epsilon}{\mathcal{E}}

\newcommand{\maxw}{$^\curlywedgedownarrow$}
\newcommand{\alex}{$^\dagger$}
\newcommand{\marcel}{$^\diamondsuit$}

\date{}
\title{Übung 10}
\author{Max Wisniewski\maxw, Alexander Steen\alex, Marcel Ehrhardt\marcel}


%%
%% Enviroments for proofs and lemmas
%%
\newtheorem{prop}{\bfseries Behauptung}
\newtheorem{lemma}{\bfseries Lemma}

\begin{document}

\lstset{language=Pascal, basicstyle=\ttfamily\fontsize{10pt}{10pt}\selectfont\upshape, commentstyle=\rmfamily\slshape, keywordstyle=\rmfamily\bfseries, breaklines=true, frame=single, xleftmargin=3mm, xrightmargin=3mm, tabsize=2, mathescape=true}

\renewcommand{\figurename}{Figure}

\maketitle
\thispagestyle{fancy}

\begin{center}
\textbf{Hinweis}: Die Angabe, wer welche Aufgabe in \LaTeX\ formuliert hat, ist als hochgestelltes Symbol der Aufgabenüberschrift zu entnehmen.
\end{center}


\section*{Aufgabe1}
Konstruieren Sie Zufallsvariablen $X$ und $Y$, die nicht unabhängig sind, so dass die
Erwartungswerte von $X$, $Y$ und $XY$ existieren und $E(XY) = E(X)E(Y)$. \\

\textbf{Lösung}: \\
Seu $(\Omega, \Epsilon, \Prob)$ ein Wahrscheinlichkeitsraum mit $\Omega = {\omega_1, \omega_2, \omega_3}$ und Gleichverteilung.

Sei $X$ gegeben durch
\begin{equation}
X(x) = \begin{cases}
         -1 & \text{, falls $x = \omega_1$}\\
         0 & \text{, falls $x = \omega_2$}\\
         1 & \text{, falls $x = \omega_3$}
       \end{cases}
\end{equation}
und $Y$ gegeben durch
\begin{equation}
Y(y) = \begin{cases}
         1 & \text{, falls $y = \omega_1$}\\
         0 & \text{, falls $y = \omega_2$}\\
         1 & \text{, falls $y = \omega_3$}
       \end{cases}
\end{equation}
Die Erwartungswerte von $X$ bzw. $Y$ sind gegeben durch $E(X) = 0$, $E(Y) = 2/3$. \\
Da $(XY)(\omega_i) = X(\omega_i) \cdot Y(\omega_i)$ ist die Produkt-Zufallsvariable gegeben durch: \\
$XY(\omega_1) = X(\omega_1)Y(\omega_1) = -1  \cdot 1 = -1$ \\
$XY(\omega_2) = X(\omega_2)Y(\omega_2) = 0 \cdot 0 = 0$ \\
$XY(\omega_3) =X(\omega_3)Y(\omega_3)  = 1$ \\
und damit der Erwartungswert (weil gleichverteilt) $E(XY) = 0$. \\

Nun suchen wir $x, y$ sodass die Ereignisse $\set{X >= x}$ und $\set{Y >= y}$ nicht unabhängig sind.
Dies ist gegeben bei den Ereignissen $\set{X >=1}  = \set{\omega_3}$ und $\set{Y >=1} = \set{\omega_1, \omega_3}$, da

\begin{equation}\begin{split}
 &\Prob(\set{X >=1} \cap \set{Y>=1})=  \Prob(\set{\omega_3}) =  \frac{1}{3} \\
&\neq \frac{2}{9}= \frac{1}{3} \cdot \frac{2}{3} = \Prob(\set{X>=1}) \cdot \Prob(\set{Y>=1})
\end{split}\end{equation}

\section*{Aufgabe 2}
Die Dichtefunktionen für $X, Y, Z$ sind gegeben durch $f_X \equiv f_Y \equiv f_Z \equiv x \mapsto 1$.
Bezeichne $f_X, f_Y, f_Z$ im Folgenden die auf $\mathbb{R}$ mit 0 fortgesetzten Dichtefunktionen, also
\begin{equation}
f_X \equiv f_Y \equiv f_Z \equiv x \mapsto \begin{cases}
								1 & \text{, falls $x \in [0,1]$} \\
								0 & \text{, sonst}
							\end{cases}
\end{equation}

\begin{equation}\begin{split}
f_{XY}(x) := (X+Y)(x) = (f_X * f_Y)(x) &= \int_\mathbb{R} f_X(a) f_Y(x-a) \;da \\
&= \begin{cases}
		0 & \text{, falls $x < 0$} \\
		x & \text{, falls $0 \leq x \leq 1$} \\
		2-x  & \text{, falls $1 < x \leq 2$} \\
		0 & \text{, falls $x > 2$}
     \end{cases}
\end{split}\end{equation}
Puh, das ist nervig. Kein Bock mehr.

\label{LastPage}
\end{document}
