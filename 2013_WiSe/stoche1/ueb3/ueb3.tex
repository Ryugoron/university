\documentclass[11pt,a4paper,ngerman]{article}
\usepackage[bottom=2.5cm,top=2.5cm]{geometry}
\usepackage[ngerman]{babel}
\usepackage[utf8]{inputenc}
\usepackage[T1]{fontenc}
\usepackage{ae}
\usepackage{amssymb}
\usepackage{amsmath}
\usepackage{amsthm}
\usepackage{graphicx}
\usepackage{fancyhdr}
\usepackage{fancyref}
\usepackage{listings}
\usepackage{xcolor}
\usepackage{stmaryrd}
\usepackage{paralist}

\usepackage[pdftex, bookmarks=false, pdfstartview={FitH}, linkbordercolor=white]{hyperref}
\usepackage{fancyhdr}
\pagestyle{fancy}
\fancyhead[C]{Stochastik I}
\fancyhead[L]{Übung 3}
\fancyhead[R]{WiSe 2013/14}
\fancyfoot{}
\fancyfoot[L]{}
\fancyfoot[C]{\thepage \hspace{1px} of \pageref{LastPage}}
\renewcommand{\footrulewidth}{0.5pt}
\renewcommand{\headrulewidth}{0.5pt}
\newcommand{\set}[1]{ \{ #1 \}}
\newcommand{\Prob}{\mathbb{P}}
\setlength{\parindent}{0pt}
\setlength{\headheight}{0pt}

\newcommand{\N}{\mathbb{N}}
\newcommand{\Q}{\mathbb{Q}}
\newcommand{\R}{\mathbb{R}}
\newcommand{\bigO}{\mathcal{O}}
\newcommand{\Rarr}{\Rightarrow}
\newcommand{\Pot}{\mathcal{P}}
\newcommand{\abs}[1]{ |#1|}
\newcommand{\solved}{$\mbox{}$ \hfill $\square$}

\newcommand{\maxw}{$^\curlywedgedownarrow$}
\newcommand{\alex}{$^\dagger$}
\newcommand{\marcel}{$^\diamondsuit$}

\date{}
\title{Übung 3}
\author{Max Wisniewski\maxw, Alexander Steen\alex, Marcel Ehrhardt\marcel}


%%
%% Enviroments for proofs and lemmas
%%
\newtheorem{prop}{\bfseries Behauptung}
\newtheorem{lemma}{\bfseries Lemma}

\begin{document}

\lstset{language=Pascal, basicstyle=\ttfamily\fontsize{10pt}{10pt}\selectfont\upshape, commentstyle=\rmfamily\slshape, keywordstyle=\rmfamily\bfseries, breaklines=true, frame=single, xleftmargin=3mm, xrightmargin=3mm, tabsize=2, mathescape=true}

\renewcommand{\figurename}{Figure}

\maketitle
\thispagestyle{fancy}

\begin{center}
\textbf{Hinweis}: Die Angabe, wer welche Aufgabe in \LaTeX\ formuliert hat, ist als hochgestelltes Symbol der Aufgabenüberschrift zu entnehmen.
\end{center}


\subsection*{Aufgabe 1\alex}
Gegeben ist ein fairer W"urfel mit 6 Seiten. Wie wahrscheinlich ist es dass nie eine
6 vorkommt wenn wir k-mal werfen bzw. abz"ahlbar oft werfen? Vergessen Sie nicht
Ihre Antwort zu begr"unden (wie immer)!  \\

Bezeichne $6_k$ das Ereignis, in der $k$-ten Runde eine 6 gewürfelt zu haben.
Da wir es mit einem fairen Würfel zu tun haben, gilt für ein festes $k \in \mathbb{N}$:
$\Prob(6_k) = \frac{1}{6}$ (1,2,3,4,5,6 sind die möglichen Ergebnisse des Wurfs).
Damit gilt ebenfalls $\Prob(\text{keine 6 im $k$-ten Wurf}) = \Prob(6_k^c) = \frac{5}{6}$.

Für zwei verschiedene Ereignisse $6_i^c$ und $6_j^c$ gilt $\Prob(6_i^c \cup 6_j^c) = \Prob(6_i^c) \Prob(6_j^c)$: \\
Betrachten wir alle möglichen Ausgänge für zwei Würfe \\
($(1,1), (1,2), (1,3), (1,4), (1,5), (1,6), \ldots, (6,4), (6,5), (6,6)$), so gibt es $36 - 11 = 25$ Elemente für das Ereignis $6_i^c \cup 6_j^c$ (nämlich alle ausser $(6,j)$ und $(i,6)$ für $j = 1,\ldots,6$, $i = 1,\ldots,5$), insgesamt aber $36$ verschiedene Ausgänge.
Es gilt also $\Prob(6_i^c \cup 6_j^c) = \frac{25}{36} = \left(\frac{5}{6}\right)^2$. \\

Dann gilt für das Ereignis "keine 6 in $k$ Würfen":\\
$\Prob(\text{keine 6 in $k$ Würfen}) = \Prob(\bigcup_{i=1}^k 6_i^c)
= \prod_{i=1}^k \Prob(6_i^c) = \left(\frac{5}{6}\right)^k$.

Für den abzählbaren Fall erhalten wir \\
$\Prob(\bigcup_{i=1}^\infty 6_i^c) = \lim_{k\to \infty} \Prob(\bigcup_{i=1}^k 6_i^c)
= \lim_{k\to \infty} \left(\frac{5}{6}\right)^k = 0$.

\subsection*{Aufgabe 2\maxw} 
Es seien $A, B \in E$ zwei Ereignisse mit $\Prob(A) = \dfrac{3}{5}$ und
$\Prob(B) = \dfrac{1}{2}.$

\begin{enumerate}[(i)]
  \item Zeigen Sie, dass \[
      \dfrac{1}{10} \le \Prob(A \cap B) \le \dfrac{1}{2}
    \]

    \textbf{Beweis}:

    Aus $A \cap B \subseteq A$ folgt $\Prob(A \cap B) \le \Prob(A)$, sowie aus
    $A \cap B \subseteq B$ folgt $\Prob(A \cap B) \le \Prob(B)$. Somit ist \[
      \Prob(A \cap B) \le \min\set{\Prob(A), \Prob(B)} = \dfrac{1}{2}.
    \]

    Wir Wissen, dass $\Prob(A\cup B) = \Prob(A) + \Prob(B) - \Prob(A\cap B)$ ist.\\
    Also auch $\Prob(A\cap B) = \Prob(A) + \Prob(B) - \Prob(A\cup B)$.
    \[
      \Prob(A\cap B) = \dfrac{11}{10} - \Prob(A\cup B) \ge 0
    \]

    $\Prob(A\cup B) = 1$ ist der gr"o"ste Wert den $\Prob(A\cup B)$ annehmen
    kann. Demzufolge ist $\Prob(A\cap B) = \frac{1}{10}$ der kleinst m"ogliche
    Wert.

  \item Finden Sie Beispiele, in denen die Schranken, angenommen werden.

    \textbf{Beispiele}:

    $\Omega = \set{1,2,\ldots,10}$ mit $\Prob(\set i) = \frac{1}{10}$.

    F"ur die untere Grenze sei $A = \set{1,\ldots,6}$ und
    $B = \set{6, \ldots 10}$. \[
      \Prob(A) = \dfrac{3}{5} \\ \qquad\qquad
      \Prob(B) = \dfrac{1}{2} \\ \qquad\qquad
      \Prob(A\cap B) = \Prob(\set 6) = \dfrac{1}{10}.
    \]

    F"ur die obere Grenze sei $A = \set{1,\ldots,6}$ und
    $B = \set{1,\ldots, 5}$. \[
      \Prob(A) = \dfrac{3}{5} \\ \qquad\qquad
      \Prob(B) = \dfrac{1}{2} \\ \qquad\qquad
      \Prob(A\cap B) = \Prob(\set{1,\ldots,5}) = \dfrac{1}{2}.
    \]

  \item Finden und beweisen Sie analoge Schranken f"ur $\Prob(A \cup B)$.

    Es gilt \[
      \dfrac{3}{5} \le \Prob(A \cup B) \le 1
    \]

    \textbf{Beweis}:

    Da nicht ausgeschlossen werden kann, dass $A \cup B = \Omega$ gilt. Kann
    $\Prob(A\cup B) = 1$ sein.

    Aus $A \subseteq A\cup B$ folgt $\Prob(A) \le \Prob(A\cup B)$ und aus
    $B \subseteq A\cup B$ folgt $\Prob(B) \le \Prob(A\cup B)$, wodurch \[
      \dfrac{3}{5} = \max\set{\Prob(A), \Prob(B)} \le \Prob(A\cup B).
    \]

    \textbf{Beispiele}:

    F"ur die obere Grenze sei $A = \set{1,\ldots,6}$ und
    $B = \set{6, \ldots 10}$. \[
      \Prob(A) = \dfrac{3}{5} \\ \qquad\qquad
      \Prob(B) = \dfrac{1}{2} \\ \qquad\qquad
      \Prob(A\cup B) = \Prob(\Omega) = 1.
    \]

    F"ur die untere Grenze sei $A = \set{1,\ldots,6}$ und
    $B = \set{1,\ldots, 5}$. \[
      \Prob(A) = \dfrac{3}{5} \\ \qquad\qquad
      \Prob(B) = \dfrac{1}{2} \\ \qquad\qquad
      \Prob(A\cup B) = \Prob(\set{1,\ldots,6}) = \dfrac{3}{5}.
    \]
\end{enumerate}

\subsection*{Aufgabe 3\marcel}

\newcommand{\SetM}{M_{3,4}}

Gegeben sei eine beliebige Menge $\Omega$. Sei \[
  \SetM := \set{E \in \Pot(\Omega)\ :\ \abs{E} = 3 \mbox{ oder } \abs{E} = 4}.
\]
Beschreiben Sie die von $\SetM$ erzeugte $\sigma$-Algebra.\\

\textbf{Beweis}:

\begin{itemize}
  \item[Fall 1:] $\abs{\Omega} \le 2$.

    Dann ist $\SetM = \emptyset$, wodurch $\sigma(\SetM) = \set{\emptyset,
    \Omega}$

  \item[Fall 2:] $\abs{\Omega} = 3$.

    Dann ist $\SetM = \set{\Omega}$, wodurch $\sigma(\SetM) = \set{\emptyset,
    \Omega}$

  \item[Fall 3:] $\abs{\Omega} \ge 4$ und $\Omega$ ist abz"ahlbar (unendlich).

    Dann ist $\sigma(\SetM) = \Pot(\Omega)$.

    \begin{itemize}
      \item Dazu zeige $x \in \Omega \Rarr \set x \in \sigma(\SetM)$. \\

        Seien $A = \set{x, y, z} \subseteq \Omega$ und $B = \set{x', y, z}
        \subseteq \Omega$ mit $x,x',y,z$ paarweise verschieden (das geht, da
        $\abs{\Omega} \ge 4$).

        Da $\abs{A} = \abs{B} = 3$, ist $A,B \in \sigma(\SetM)$, wodurch auch
        $\set x = A\setminus B \in \sigma(\SetM)$.

      \item Zeige $\sigma(\SetM) = \Pot(\Omega)$. \\

        Die Richtung ``$\subseteq$'' ist offensichtlich.

        Sei $X \in \Pot(\Omega)$, dann ist $X$ abz"ahlbar, weil $\Omega$
        abz"ahlbar ist. Schreibe $X$ als $X = \set{x_1, \ldots} \subseteq
        \Omega$, dann ist \[
          \bigcup_{i\in\N} \set{x_i} = X,
        \]
        wodurch $X \in \sigma(\SetM)$.

    \end{itemize}

  \item[Fall 4:] $\Omega$ ist "uberabz"ahlbar.

    TODO!!

\end{itemize}

\subsection*{Aufgabe 4\marcel}

Zeigen Sie nur unter Benutzung der Definition von Borelmengen, dass die folgenden
Teilmengen von $\R$ Borelmengen sind (Bereits bewiesene Ergebnisse "uber
$\sigma$-Algebren d"urfen dabei benutzt werden):

\begin{enumerate}[(i)]
  \item die Menge der positiven rationalen Zahlen, deren Wurzel kleiner als $3$
    ist.\\

    \textbf{Beweis}:

    Die in der Aufgabe beschriebene Menge l"asst sich wie folgt ausdr"ucken: \[
      \set{x \in \Q\ :\ 0 \le \sqrt x < 3} =
      \set{x \in \R\ :\ 0 \le x < 3 ^ 2, x \in \Q} = [0, 3^2) \cap \Q.
    \]

    Betrachte $[0, 3^2) = \set{0} \cup (0, 3^2)$, wobei $(0, 3^2)$ eine
    Borelmenge ist, da sie offen ist. Somit ist nur z.z., dass jede ein-
    elementige Menge und $\Q$ Borelmengen sind. \\

    Wir wissen, dass alle Mengen in der $\sigma$-Algebra $\sigma(\bigO)$ als
    Borelmengen bezeichnet werden. Da die Komplemente von offenen Mengen
    abgeschlossen sind, sind abgeschlossene Mengen Borelmengen. Insbesondere ist
    jede ein-elementige Menge abgeschlossen und somit eine Borelmenge.

    Seien $q_1, q_2, \ldots$ alle Elemente aus $\Q$ (ist abz"ahlbar).
    Damit ist \[
      \Q = \bigcup_{i\in\N} \set{q_i}
    \]
    eine abz"ahlbar unendliche Vereinigung von ein-elementigen Mengen, welche
    wie oben gezeigt Borelmengen sind, wodurch auf Grund der Eigenschaft von
    $\sigma$-Algebren auch $\Q$ eine Borelmenge ist.

  \item $(-2, 8] \setminus \Q.$\\

    \textbf{Beweis}:

    Es ist $(-2, 8] = (-2, 8) \cup \set 8$ eine Borelmenge, da $(-2, 8)$ und
    $\set 8$ welche sind. Damit ist auch $(-2, 8] \setminus \Q$ eine Borelmenge,
    da in $\sigma$-Algebren relative Komplemente nicht aus der $\sigma$-Algebra
    herausf"uhren.

\end{enumerate}

\label{LastPage}
\end{document}
