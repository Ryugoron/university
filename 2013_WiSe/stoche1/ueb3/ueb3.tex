\documentclass[11pt,a4paper,ngerman]{article}
\usepackage[bottom=2.5cm,top=2.5cm]{geometry} 
\usepackage[ngerman]{babel}
\usepackage[utf8]{inputenc} 
\usepackage[T1]{fontenc} 
\usepackage{ae} 
\usepackage{amssymb} 
\usepackage{amsmath}
\usepackage{amsthm} 
\usepackage{graphicx}
\usepackage{fancyhdr}
\usepackage{fancyref}
\usepackage{listings}
\usepackage{xcolor}
\usepackage{stmaryrd}
\usepackage{paralist}

\usepackage[pdftex, bookmarks=false, pdfstartview={FitH}, linkbordercolor=white]{hyperref}
\usepackage{fancyhdr}
\pagestyle{fancy}
\fancyhead[C]{Stochastik I}
\fancyhead[L]{Übung 3}
\fancyhead[R]{WiSe 2013/14}
\fancyfoot{}
\fancyfoot[L]{}
\fancyfoot[C]{\thepage \hspace{1px} of \pageref{LastPage}}
\renewcommand{\footrulewidth}{0.5pt}
\renewcommand{\headrulewidth}{0.5pt}
\newcommand{\set}[1]{ \{ #1 \}}
\newcommand{\Prob}{\mathbb{P}}
\setlength{\parindent}{0pt} 
\setlength{\headheight}{0pt}

\newcommand{\R}{\mathbb{R}}
\newcommand{\Q}{\mathbb{Q}}
\newcommand{\Pot}{\mathcal{P}}
\newcommand{\abs}[1]{ |#1|}
\newcommand{\solved}{$\mbox{}$ \hfill $\square$}

\newcommand{\maxw}{$^\curlywedgedownarrow$}
\newcommand{\alex}{$^\dagger$}
\newcommand{\marcel}{$^\diamondsuit$}

\date{}
\title{Übung 3}
\author{Max Wisniewski\maxw, Alexander Steen\alex, Marcel Ehrhardt\marcel}


%%
%% Enviroments for proofs and lemmas
%%
\newtheorem{prop}{\bfseries Behauptung}
\newtheorem{lemma}{\bfseries Lemma}

\begin{document}

\lstset{language=Pascal, basicstyle=\ttfamily\fontsize{10pt}{10pt}\selectfont\upshape, commentstyle=\rmfamily\slshape, keywordstyle=\rmfamily\bfseries, breaklines=true, frame=single, xleftmargin=3mm, xrightmargin=3mm, tabsize=2, mathescape=true}

\renewcommand{\figurename}{Figure}

\maketitle
\thispagestyle{fancy}

\begin{center}
\textbf{Hinweis}: Die Angabe, wer welche Aufgabe in \LaTeX formuliert hat, ist als hochgestellter Symbol der Aufgabenüberschrift zu entnehmen.
\end{center}


\subsection*{Aufgabe 1}
Gegeben ist ein fairer W"urfel mit 6 Seiten. Wie wahrscheinlich ist es dass nie eine
6 vorkommt wenn wir k-mal werfen bzw. abz"ahlbar oft werfen? Vergessen Sie nicht
Ihre Antwort zu begr"unden (wie immer)!

\subsection*{Aufgabe 2}
Es seien $A, B \in E$ zwei Ereignisse mit $\Prob(A) = \dfrac{3}{5}$ und $\Prob(B) = \dfrac{1}{2}.$

\begin{enumerate}[(i)]
  \item Zeigen Sie, dass \[\dfrac{1}{10} \le \Prob(A \cap B) \le \dfrac{1}{2}\]
  \item Finden Sie Beispiele, in denen die Schranken, angenommen werden.
  \item Finden und beweisen Sie analoge Schranken f"ur $\Prob(A \cup B)$.
\end{enumerate}

\subsection*{Aufgabe 3}

\newcommand{\SetM}{M_{3,4}}

Gegeben sei eine beliebige Menge $\Omega$. Sei \[
  \SetM := \set{E \in \Pot(\Omega)\ :\ \abs{E} = 3 \mbox{ oder } \abs{E} = 4}.
\]
Beschreiben Sie die von $\SetM$ erzeugte $\sigma$-Algebra.

\subsection*{Aufgabe 4}

Zeigen Sie nur unter Benutzung der Definition von Borelmengen, dass die folgenden
Teilmengen von $\R$ Borelmengen sind (Bereits bewiesene Ergebnisse "uber
$\sigma$-Algebren d"urfen dabei benutzt werden):

\begin{enumerate}[(i)]
  \item die Menge der positiven rationalen Zahlen, deren Wurzel kleiner als $3$ ist.
  \item $(-2, 8] \setminus \Q.$
\end{enumerate}

\label{LastPage}
\end{document}
