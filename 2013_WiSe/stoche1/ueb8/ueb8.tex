\documentclass[11pt,a4paper,ngerman]{article}
\usepackage[bottom=2.5cm,top=2.5cm]{geometry}
\usepackage[ngerman]{babel}
\usepackage[utf8]{inputenc}
\usepackage[T1]{fontenc}
\usepackage{ae}
\usepackage{amssymb}
\usepackage{amsmath}
\usepackage{amsthm}
\usepackage{graphicx}
\usepackage{fancyhdr}
\usepackage{fancyref}
\usepackage{listings}
\usepackage{xcolor}
\usepackage{stmaryrd}
\usepackage{paralist}

\usepackage[pdftex, bookmarks=false, pdfstartview={FitH}, linkbordercolor=white]{hyperref}
\usepackage{fancyhdr}
\pagestyle{fancy}
\fancyhead[C]{Stochastik I}
\fancyhead[L]{Übung 8}
\fancyhead[R]{WiSe 2013/14}
\fancyfoot{}
\fancyfoot[L]{}
\fancyfoot[C]{\thepage \hspace{1px} of \pageref{LastPage}}
\renewcommand{\footrulewidth}{0.5pt}
\renewcommand{\headrulewidth}{0.5pt}
\newcommand{\set}[1]{ \{ #1 \}}
\newcommand{\Prob}{\mathbb{P}}
\setlength{\parindent}{0pt}
\setlength{\headheight}{0pt}

\newcommand{\N}{\mathbb{N}}
\newcommand{\Q}{\mathbb{Q}}
\newcommand{\R}{\mathbb{R}}
\newcommand{\bigO}{\mathcal{O}}
\newcommand{\Rarr}{\Rightarrow}
\newcommand{\rarr}{\rightarrow}
\newcommand{\Pot}{\mathcal{P}}
\newcommand{\abs}[1]{ |#1|}
\newcommand{\solved}{$\mbox{}$ \hfill $\square$}
\newcommand{\Epsilon}{\mathcal{E}}

\newcommand{\maxw}{$^\curlywedgedownarrow$}
\newcommand{\alex}{$^\dagger$}
\newcommand{\marcel}{$^\diamondsuit$}

\date{}
\title{Übung 8}
\author{Max Wisniewski\maxw, Alexander Steen\alex, Marcel Ehrhardt\marcel}


%%
%% Enviroments for proofs and lemmas
%%
\newtheorem{prop}{\bfseries Behauptung}
\newtheorem{lemma}{\bfseries Lemma}

\begin{document}

\lstset{language=Pascal, basicstyle=\ttfamily\fontsize{10pt}{10pt}\selectfont\upshape, commentstyle=\rmfamily\slshape, keywordstyle=\rmfamily\bfseries, breaklines=true, frame=single, xleftmargin=3mm, xrightmargin=3mm, tabsize=2, mathescape=true}

\renewcommand{\figurename}{Figure}

\maketitle
\thispagestyle{fancy}

\begin{center}
\textbf{Hinweis}: Die Angabe, wer welche Aufgabe in \LaTeX\ formuliert hat, ist als hochgestelltes Symbol der Aufgabenüberschrift zu entnehmen.
\end{center}


\subsection*{Aufgabe 1}

Es werden 4 Kugeln auf gut Glück auf 4 Fächer verteilt. Wie groß ist die Wahrscheinlichkeit dafür, dass genau ein Fach leer bleibt?\\

\textbf{Lösung:}\\

Wir nehmen erstmal an, dass es sich bei emph{auf gut Glück} um eine gleichverteilung der Aufteilungen handelt. Und darüber hinaus nehmen
wir an, dass am Ende auch 4 Kugeln in den 4 Fächern sind (nicht, dass er auf gut Glück eine daneben fallen lässt).\\

Dann ist $\left| \Omega \right| = 4^4$, da wir für jede Kugel $4$ Möglichkeiten haben, sie in ein Fach zu legen.\\

Betrachten wir das Ereignis \emph{genau ein Fach bleibt leer}, dann haben wir zunächst $\binom{4}{1} = 4$ Möglichkeiten, das Fach zu wählen,
dass leer bleibt. Nun wollen wir $4$ Kugeln auf $3$ Fächer aufteilen, so dass jedes Fach mindestens einmal getroffen wird.

Dazu könnten wir die Aufteilung jetzt als \emph{surjektiv} betrachten, mit einer unterscheidbaren Menge an Objekten, die auf eine unterscheidbare
Menge an Objekten aufgeteilt wird, was der Größe $r!S_{n,r}$\footnote{Hierbei ist $S_{n,r}$ die Stirlingzahl zweiter Art} entspricht. Aber wir können uns
das ganze auch einfacher machen. Wenn jedes Fach mindestens einmal getroffen wird, bleibt genau eine Kugel über, die wir auf eins der 3 Fächer aufteilen können.

Wir wählen also eins der drei aus $\binom{3}{1} = 3$.\\

$\left|E\right| = \binom{4}{1} \cdot \binom{3}{1} = 12$.\\

\[
    P(E) = \frac{\left|E\right|}{\left|\Omega\right|} = \frac{12}{4^4} = \frac{3}{4^3} = \frac{3}{64}
\]

\subsection*{Aufgabe 2}

Eine Ameise steht auf dem Nullpunkt des Gitters $\mathbb{Z}^2$ und bewegt sich einen Gitterpunkt mit jedem Schritt entweder nach rechts oder nach
oben mit gleicher Wahrscheinlichkeit. Es gibt einen elektrischen Zaun zwischen den Punkten $(0,14)$ und $(23,14)$, sowie zwischen den Punkten $(25,0)$ und
$(25,12)$; wenn die Ameise auf einen Zaun trifft, stirbt sie. Wie Warhscheinlich ist es, dass die Ameise überlebt und die Freiheit erreicht?\\

\textbf{Lösung:}\\

Zunächst wieder eine Annahmen:\\
Die Randpunkte $(0,14), (23,14), (25,0), (25,12)$ gehören zum Zaun und dürfen für das überleben ebenfalls nicht betreten werden.\\

Betrachten wir uns das Feld der Ameise, so ist sie von einem (nahezu) Rechteck umrahmt und der einzige Weg in die Freiheit führt über
das Feld $(24,13)$. Sind wir auf dieses Feld gelangt können wir vorher keinen der Zaune berührt haben, da wir jede Komponente des Tupels nur erhöhen,
aber nicht verringern können durch einen Schritt.
Umgekehrt wird jeder Schritt von hieraus den Zaun passieren.\\

Wir wollen also die Wahrscheinlichkeit berechnen auf dieses Feld zu gelangen.\\
Dies ist ein kanonischer Anwendungsfall für den Binomialkoeffizienten. Wir erhöhen in jedem Schritt eine Komponente des Tupels um 1, d.h.
um von $(0,0)$ auf $(24,13)$ zu kommen machen wir $24+13 = 37$ Schritte. Nun fragen wir uns, wie viele Wege es gibt.
Da wir unterwegs auf keinen Zaun treffen, ist jede Kombination von Oben/Rechts erlaubt, was insgesammt
$ \binom{37}{13} $ ist.\\

Ein Weg von $37$ Schritten hat, da beide Möglichkeiten gleich Wahrscheinlich sind, die Wahrscheinlichkeit $0.5^{37}$. Wir erhalten für
das Ereignis $E$ \emph{Überleben} also

\[
    P(E) = \binom{37}{13} * 0.5^{37} = 0.02592036107671447098255157470703125 \approx 2.5\%
\]

\subsection*{Aufgabe 3}


\subsection*{Aufgabe 4}

Wir versehen $[0, 1]$ mit der Dichtefunktion $f(x) = cx^2$.
\begin{itemize}
  \item Bestimmen Sie c.
  \item Bestimmen Sie alle Intervalle $[a, b] \subseteq [0, 1]$, die von $[0,
  0.4]$ unabh"angig sind.

\end{itemize}

\textbf{Lösung:}\\

Es muss
\[
    \int_0^1 f(x) dx = 1
\] gelten und damit
\[
    \int_0^1 c x^2 = 1 \Leftrightarrow \frac{1}{3}c = 1 \Leftrightarrow c = 3.
\]

Es ist $[a,b]$ von $[0,0.4]$ unabhängig gdw. $\Prob([a,b] \cap [0,0.4]) = \Prob([a,b]) \Prob([0,0.4])$.\\
Es ist
\[
    \Prob([a,b]) \Prob([0,0.4]) = \int_a^b 3x^2 dx \cdot \int_0^{0.4} 3x^2 dx = 0.064 (b^3 - a^3).
\]

\begin{description}

    \item[\bfseries 1. Fall:] $b \leq 0.4$ \\
        Dann ist $[a,b] \cap [0,0.4] = [a,b]$ und damit
        \[
            \Prob([a,b] \cap [0,0.4]) = \Prob([a,b]) = b^3 - a^3.
        \]
        Dann ist $ b^3 - a^3 =  0.064 (b^3 - a^3) \Leftrightarrow b = a$. \\
        Also $[a,b] = [a,a] = \{a\}$ und hat die Wahrscheinlichekit $\Prob ([a,a]) = 0$. \\
    \item[\bfseries 2. Fall:] $b > 0.4$ und $a \leq 0.4$: \\
        Dann ist $[a,b] \cap [0,0.4] = [a, 0.4]$ und damit
        \[
            \Prob([a,b] \cap [0,0.4]) = \Prob([a,0.4]) = 0.064 - a^3.
        \]
        Dann ist $ 0.064 - a^3 =  0.064 (b^3 - a^3) \Leftrightarrow b = \frac{1}{2} \sqrt[3]{8-117a^3}$. \\
        Also $[a,b] = [a,\frac{1}{2} \sqrt[3]{8-117a^3}]$ für $a \leq 0.4$, $b > 0.4$. \\

    \item[\bfseries 3. Fall:] $b > 0$ und $ a > 0.4$: \\
        Dann ist $[a,b] \cap [0,0.4] = \emptyset$ und damit  $\Prob([a,b] \cap [0,0.4]) = \Prob(\emptyset) =0$.
        Dann ist $ 0=  0.064 (b^3 - a^3) \Leftrightarrow b = a$. \\
        Also $[a,b] = [a,a] = \{a\}$.\\
\end{description}

    Es gilt also nur im 2. Fall und für alle $0$-Mengen, also ein elementige und Intervalle $[a,b]$ mit $a>b$, da diese
    hier $[a,b] = \emptyset$ auch die Bedingung erfüllt.

\label{LastPage}
\end{document}
