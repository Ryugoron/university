\documentclass[11pt,a4paper,ngerman]{article}
\usepackage[bottom=2.5cm,top=2.5cm]{geometry}
\usepackage[ngerman]{babel}
\usepackage[utf8]{inputenc}
\usepackage[T1]{fontenc}
\usepackage{ae}
\usepackage{amssymb}
\usepackage{amsmath}
\usepackage{amsthm}
\usepackage{graphicx}
\usepackage{fancyhdr}
\usepackage{fancyref}
\usepackage{listings}
\usepackage{xcolor}
\usepackage{stmaryrd}
\usepackage{paralist}

\usepackage[pdftex, bookmarks=false, pdfstartview={FitH}, linkbordercolor=white]{hyperref}
\usepackage{fancyhdr}
\pagestyle{fancy}
\fancyhead[C]{Stochastik I}
\fancyhead[L]{Übung 8}
\fancyhead[R]{WiSe 2013/14}
\fancyfoot{}
\fancyfoot[L]{}
\fancyfoot[C]{\thepage \hspace{1px} of \pageref{LastPage}}
\renewcommand{\footrulewidth}{0.5pt}
\renewcommand{\headrulewidth}{0.5pt}
\newcommand{\set}[1]{ \{ #1 \}}
\newcommand{\Prob}{\mathbb{P}}
\setlength{\parindent}{0pt}
\setlength{\headheight}{0pt}

\newcommand{\N}{\mathbb{N}}
\newcommand{\Q}{\mathbb{Q}}
\newcommand{\R}{\mathbb{R}}
\newcommand{\bigO}{\mathcal{O}}
\newcommand{\Rarr}{\Rightarrow}
\newcommand{\rarr}{\rightarrow}
\newcommand{\Pot}{\mathcal{P}}
\newcommand{\abs}[1]{ |#1|}
\newcommand{\solved}{$\mbox{}$ \hfill $\square$}
\newcommand{\Epsilon}{\mathcal{E}}

\newcommand{\maxw}{$^\curlywedgedownarrow$}
\newcommand{\alex}{$^\dagger$}
\newcommand{\marcel}{$^\diamondsuit$}

\date{}
\title{Übung 8}
\author{Max Wisniewski\maxw, Alexander Steen\alex, Marcel Ehrhardt\marcel}


%%
%% Enviroments for proofs and lemmas
%%
\newtheorem{prop}{\bfseries Behauptung}
\newtheorem{lemma}{\bfseries Lemma}

\begin{document}

\lstset{language=Pascal, basicstyle=\ttfamily\fontsize{10pt}{10pt}\selectfont\upshape, commentstyle=\rmfamily\slshape, keywordstyle=\rmfamily\bfseries, breaklines=true, frame=single, xleftmargin=3mm, xrightmargin=3mm, tabsize=2, mathescape=true}

\renewcommand{\figurename}{Figure}

\maketitle
\thispagestyle{fancy}

\begin{center}
\textbf{Hinweis}: Die Angabe, wer welche Aufgabe in \LaTeX\ formuliert hat, ist als hochgestelltes Symbol der Aufgabenüberschrift zu entnehmen.
\end{center}


\subsection*{Aufgabe 1}



\subsection*{Aufgabe 2}



\subsection*{Aufgabe 3}


\subsection*{Aufgabe 4}

Es muss $\int_0^1 f(x) dx = 1$ sein und damit $\int_0^1 c x^2 = 1 \Leftrightarrow \frac{1}{3}c = 1 \Leftrightarrow c = 3$.

Es ist $[a,b]$ von $[0,0.4]$ unabhängig gdw. $\Prob([a,b] \cap [0,0.4]) = \Prob([a,b]) \Prob([0,0.4])$.
Es ist $\Prob([a,b]) \Prob([0,0.4]) = \int_a^b 3x^2 dx \cdot \int_0^{0.4} 3x^2 dx = 0.064 (b^3 - a^3)$. \\

1. Fall: $b \leq 0.4$ \\
Dann ist $[a,b] \cap [0,0.4] = [a,b]$ und damit $\Prob([a,b] \cap [0,0.4]) = \Prob([a,b]) = b^3 - a^3$.
Dann ist $ b^3 - a^3 =  0.064 (b^3 - a^3) \Leftrightarrow b = a$. \\
Also $[a,b] = [a,a] = \emptyset$. \\

2. Fall $b > 0.4$ \\
(i) Fall $a \leq 0.4$: \\
Dann ist $[a,b] \cap [0,0.4] = [a, 0.4]$ und damit  $\Prob([a,b] \cap [0,0.4]) = \Prob([a,0.4]) = 0.064 - a^3$.
Dann ist $ 0.064 - a^3 =  0.064 (b^3 - a^3) \Leftrightarrow b = \frac{1}{2} \sqrt{8-117a^3}^3$. \\
Also $[a,b] = [a,\frac{1}{2} \sqrt{8-117a^3}^3]$ für $a \leq 0.4$, $b > 0.4$. \\

(ii) Fall $a > 0.4$: \\
Dann ist $[a,b] \cap [0,0.4] = \emptyset$ und damit  $\Prob([a,b] \cap [0,0.4]) = \Prob(\emptyset) =0$.
Dann ist $ 0=  0.064 (b^3 - a^3) \Leftrightarrow b = a$. \\
Also $[a,b] = [a,a] = \emptyset$. \\

Macht das Sinn? Sieht irgendwie komisch aus.
Ich mach morgen ab ca. 9 Uhr weiter (Alex)


\label{LastPage}
\end{document}
