\documentclass[11pt,a4paper,ngerman]{article}
\usepackage[bottom=2.5cm,top=2.5cm]{geometry} 
\usepackage{babel}
\usepackage[utf8]{inputenc} 
\usepackage[T1]{fontenc} 
\usepackage{ae} 
\usepackage{amssymb} 
\usepackage{amsmath}
\usepackage{amsthm} 
\usepackage{graphicx}
\usepackage{fancyhdr}
\usepackage{fancyref}
\usepackage{listings}
\usepackage{xcolor}
\usepackage{paralist}

\usepackage[pdftex, bookmarks=false, pdfstartview={FitH}, linkbordercolor=white]{hyperref}
\usepackage{fancyhdr}
\pagestyle{fancy}
\fancyhead[C]{Stochastik I}
\fancyhead[L]{Übung 1}
\fancyhead[R]{WiSe 2013/14}
\fancyfoot{}
\fancyfoot[L]{}
\fancyfoot[C]{\thepage \hspace{1px} of \pageref{LastPage}}
\renewcommand{\footrulewidth}{0.5pt}
\renewcommand{\headrulewidth}{0.5pt}
\newcommand{\set}[1]{ \{ #1 \}}
\setlength{\parindent}{0pt} 
\setlength{\headheight}{0pt}

\date{}
\title{Übung 1}
\author{Max Wisniewski, Alexander Steen}


%%
%% Enviroments for proofs and lemmas
%%
\newtheorem{prop}{\bfseries Behauptung}
\newtheorem{lemma}{\bfseries Lemma}

\begin{document}

\lstset{language=Pascal, basicstyle=\ttfamily\fontsize{10pt}{10pt}\selectfont\upshape, commentstyle=\rmfamily\slshape, keywordstyle=\rmfamily\bfseries, breaklines=true, frame=single, xleftmargin=3mm, xrightmargin=3mm, tabsize=2, mathescape=true}

\renewcommand{\figurename}{Figure}

\maketitle
\thispagestyle{fancy}


\subsection*{Aufgabe 1}
Sei $\Omega = \{1,2,3,4\}$. Geben Sie alle $\sigma$-Algebren über $\Omega$ an. \\

\textbf{Lösung}: \\
Es bezeichnen $\xi_i$ die $\sigma$-Algebren über $\Omega$:
\begin{enumerate}[$\xi_1 = $]
% Partition 1,2,3,4
\item $\{\emptyset, \Omega \}$
%Partition 1;2,3,4
\item $\set{\emptyset, \set{1}, \set{2,3,4}, \Omega}$
%Partition 2; 1,3,4
\item $\set{\emptyset, \set{2}, \set{1,3,4},\Omega}$
%Partition 3;1,2,4
\item $\set{\emptyset, \set{3}, \set{1,2,4},\Omega}$
%Partition 4;1,2,3
\item $\set{\emptyset, \set{4}, \set{1,2,3},\Omega}$
% Partition 1,2;    3,4
\item $\{\emptyset, \set{1,2}, \set{3,4}, \Omega \}$
% Partition 1,3;    2,4
\item $\{\emptyset, \set{1,3}, \set{2,4}, \Omega \}$
% Partition 1,4;    2,3
\item $\{\emptyset, \set{1,4}, \set{2,3}, \Omega \}$
%Partition 1,2;3;4
\item $\{\emptyset, \set{3}, \set{4}, \set{1,2}, \set{3,4}, \set{1,2,3}, \set{1,2,4}, \Omega \}$
%Partition 1,3;2;4
\item $\{\emptyset, \set{2}, \set{4}, \set{1,3}, \set{2,4}, \set{1,2,3}, \set{1,3,4}, \Omega \}$
%Partition 1,4;2;3
\item $\{\emptyset, \set{2}, \set{3}, \set{1,4}, \set{2,3}, \set{1,2,4}, \set{1,3,4}, \Omega \}$
%Partition 2,3;1;4
\item $\{\emptyset, \set{1}, \set{4}, \set{1,4}, \set{2,3}, \set{1,2,3}, \set{2,3,4}, \Omega \}$
%Partition 2,4;1;3
\item $\{\emptyset, \set{1}, \set{3}, \set{1,3}, \set{2,4}, \set{1,2,4}, \set{2,3,4}, \Omega \}$
%Partition 3,4;1;2
\item $\{\emptyset, \set{1}, \set{2}, \set{1,2}, \set{3,4}, \set{1,3,4}, \set{2,3,4}, \Omega \}$
% Partition 1;2;3;4
\item $\mathcal{P}(\Omega) = \{\emptyset, \set{1},\set{2},\set{3}, \set{4}, \set{1,2}, \ldots, \set{2,3,4}, \Omega \}$
\end{enumerate}

\begin{lemma}
Sei $\Omega$ eine endliche Menge. Dann existiert eine Bijektion zwischen der Menge $\Xi := \set{\xi | \xi \in \mathcal{P}(\Omega) \text{ ist $\sigma$-Algebra}}$ aller $\sigma$-Algebren über $\Omega$ und der Menge der Partitionen von $\Omega$.	
\end{lemma}

\textbf{Beweis}: todo $\mbox{}$ \hfill $\square$

\begin{prop}
 Es gibt keine weiteren $\sigma$-Algebren über $\Omega$. 
\end{prop}
\textbf{Beweis}: Wie man leicht nachprüfen kann, sind alle obigen Mengensysteme gültige $\sigma$-Algebren. Bezeichne $S_{n,k}$ die Stirlingzahl zweiter Art.
Dann gilt für die Anzahl $N$ der verschiedenen Partitionen von $\Omega$: 
\begin{equation}
N = \sum_{k=1}^4 S_{4,k} = 1 +6+ 7+ 1=15
\end{equation}
Da zusätzliche alle obenstehenden Mengensysteme paarweise verschiedenen sind, folgt die Aussage.
$\mbox{}$ \hfill $\square$

\subsection*{Aufgabe 2}
\subsection*{Aufgabe 3}
\subsection*{Aufgabe 4}


\label{LastPage}
\end{document}
