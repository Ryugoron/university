\documentclass[11pt,a4paper,ngerman]{article}
\usepackage[bottom=2.5cm,top=2.5cm]{geometry} 
\usepackage[ngerman]{babel}
\usepackage[utf8]{inputenc} 
\usepackage[T1]{fontenc} 
\usepackage{ae} 
\usepackage{amssymb} 
\usepackage{amsmath}
\usepackage{amsthm} 
\usepackage{graphicx}
\usepackage{fancyhdr}
\usepackage{fancyref}
\usepackage{listings}
\usepackage{xcolor}
\usepackage{paralist}

\usepackage[pdftex, bookmarks=false, pdfstartview={FitH}, linkbordercolor=white]{hyperref}
\usepackage{fancyhdr}
\pagestyle{fancy}
\fancyhead[C]{Stochastik I}
\fancyhead[L]{Übung 1}
\fancyhead[R]{WiSe 2013/14}
\fancyfoot{}
\fancyfoot[L]{}
\fancyfoot[C]{\thepage \hspace{1px} of \pageref{LastPage}}
\renewcommand{\footrulewidth}{0.5pt}
\renewcommand{\headrulewidth}{0.5pt}
\newcommand{\set}[1]{ \{ #1 \}}
\newcommand{\Prob}{\mathbb{P}}
\setlength{\parindent}{0pt} 
\setlength{\headheight}{0pt}

\date{}
\title{Übung 1}
\author{Max Wisniewski, Alexander Steen}


%%
%% Enviroments for proofs and lemmas
%%
\newtheorem{prop}{\bfseries Behauptung}
\newtheorem{lemma}{\bfseries Lemma}

\begin{document}

\lstset{language=Pascal, basicstyle=\ttfamily\fontsize{10pt}{10pt}\selectfont\upshape, commentstyle=\rmfamily\slshape, keywordstyle=\rmfamily\bfseries, breaklines=true, frame=single, xleftmargin=3mm, xrightmargin=3mm, tabsize=2, mathescape=true}

\renewcommand{\figurename}{Figure}

\maketitle
\thispagestyle{fancy}


\subsection*{Aufgabe 1\footnote{Niederschrift von Santa Claus}}
Sei $\Omega = \{1,2,3,4\}$. Geben Sie alle $\sigma$-Algebren über $\Omega$ an. \\

\textbf{Lösung}: \\
Es bezeichnen $\xi_i$ die $\sigma$-Algebren über $\Omega$:
\begin{enumerate}[$\xi_1 = $]
% Partition 1,2,3,4
\item $\{\emptyset, \Omega \}$
%Partition 1;2,3,4
\item $\set{\emptyset, \set{1}, \set{2,3,4}, \Omega}$
%Partition 2; 1,3,4
\item $\set{\emptyset, \set{2}, \set{1,3,4},\Omega}$
%Partition 3;1,2,4
\item $\set{\emptyset, \set{3}, \set{1,2,4},\Omega}$
%Partition 4;1,2,3
\item $\set{\emptyset, \set{4}, \set{1,2,3},\Omega}$
% Partition 1,2;    3,4
\item $\{\emptyset, \set{1,2}, \set{3,4}, \Omega \}$
% Partition 1,3;    2,4
\item $\{\emptyset, \set{1,3}, \set{2,4}, \Omega \}$
% Partition 1,4;    2,3
\item $\{\emptyset, \set{1,4}, \set{2,3}, \Omega \}$
%Partition 1,2;3;4
\item $\{\emptyset, \set{3}, \set{4}, \set{1,2}, \set{3,4}, \set{1,2,3}, \set{1,2,4}, \Omega \}$
%Partition 1,3;2;4
\item $\{\emptyset, \set{2}, \set{4}, \set{1,3}, \set{2,4}, \set{1,2,3}, \set{1,3,4}, \Omega \}$
%Partition 1,4;2;3
\item $\{\emptyset, \set{2}, \set{3}, \set{1,4}, \set{2,3}, \set{1,2,4}, \set{1,3,4}, \Omega \}$
%Partition 2,3;1;4
\item $\{\emptyset, \set{1}, \set{4}, \set{1,4}, \set{2,3}, \set{1,2,3}, \set{2,3,4}, \Omega \}$
%Partition 2,4;1;3
\item $\{\emptyset, \set{1}, \set{3}, \set{1,3}, \set{2,4}, \set{1,2,4}, \set{2,3,4}, \Omega \}$
%Partition 3,4;1;2
\item $\{\emptyset, \set{1}, \set{2}, \set{1,2}, \set{3,4}, \set{1,3,4}, \set{2,3,4}, \Omega \}$
% Partition 1;2;3;4
\item $\mathcal{P}(\Omega) = \{\emptyset, \set{1},\set{2},\set{3}, \set{4}, \set{1,2}, \ldots, \set{2,3,4}, \Omega \}$
\end{enumerate}

\begin{lemma}
Sei $\Omega$ eine endliche Menge. Dann existiert eine Bijektion zwischen der Menge $\Xi := \set{\xi | \xi \in \mathcal{P}(\Omega) \text{ ist $\sigma$-Algebra}}$ aller $\sigma$-Algebren über $\Omega$ und der Menge der Partitionen von $\Omega$.	
\end{lemma}

\textbf{Beweis}: todo $\mbox{}$ \hfill $\square$

\begin{prop}
 Es gibt keine weiteren $\sigma$-Algebren über $\Omega$. 
\end{prop}
\textbf{Beweis}: Wie man leicht nachprüfen kann, sind alle obigen Mengensysteme gültige $\sigma$-Algebren. Bezeichne $S_{n,k}$ die Stirlingzahl zweiter Art.
Dann gilt für die Anzahl $N$ der verschiedenen Partitionen von $\Omega$: 
\begin{equation}
N = \sum_{k=1}^4 S_{4,k} = 1 +6+ 7+ 1=15
\end{equation}
Da zusätzliche alle obenstehenden Mengensysteme paarweise verschiedenen sind, folgt die Aussage.
$\mbox{}$ \hfill $\square$

\subsection*{Aufgabe 2\footnote{Niederschrift von Santa Claus}}
Seien $\Omega$ eine Menge, $\xi$ eine $\sigma$-Algebra auf $\Omega$ und $\xi_C := \set{E \cap C | E \in \xi}$ für ein $C \subseteq \Omega$.
Zu zeigen:
\begin{enumerate}[(i)]
\item $\xi_C$ ist eine $\sigma$-Algebra \\
\textbf{Beweis}: $\xi_C$ ist $\sigma$-Algebra über $C$. \\
(i) $\emptyset ,C \in \xi_C$ \\
Da $\emptyset, \Omega \in \xi$ folgt $\emptyset \cap C = \emptyset \in \xi_C$ und $\Omega \cap C = C \in \xi_C$.\\
(ii) $\forall E \in \xi_C: E^c \in \xi_C$ \\
Sei $E \in \xi_C$. Dann ex. ein $E' \in \xi$ s.d. $E = E' \cap C$. Dann gilt auch $E'^c \in \xi$ (Komplement von $E$ bzgl. $\Omega$) und damit für 
$E^c$ (Komplement von $E$ bzgl. $C$) $E^c = E'^c \cap C \in \xi_C$, da $C \subseteq \Omega$. \\
(iii) $\forall \left(E_i\right)_{i \in I} \subseteq \xi_C: \bigcup_{i \in I} {E_i} \in \xi_C$  \\
Sei $\left(E_i\right)_{i \in I} \subseteq \xi_C$ eine Folge von Ereignissen. Dann gibt es für jedes $E_i$ ein $E'_i \in \xi$ mit $E'_i \cap C = E_i$.
Dann ist aber auch $\bigcup_{i \in I} E'_i \in \xi$ und damit $\left( \bigcup_{i \in I} E'_i \right) \cap C = \bigcup_{i \in I} \left( E'_i \cap C \right) = \bigcup_{i \in I} E_i \in \xi_C$
$\mbox{}$ \hfill $\square$
\item $C \in \xi \Rightarrow \xi_C = \set{E \in \xi | E \subseteq C} =: \xi'_C$ \\
\textbf{Beweis}: Seien $C \in \xi$, .\\
"$\subseteq$": \\
todo

"$\supseteq$": \\
Sei $E \in \xi'_C$. Dann gilt $E \in \xi$, $E \subseteq C$. Dann ist $E = E \cap C \in \xi_C$.
$\mbox{}$ \hfill $\square$
\end{enumerate}


\subsection*{Aufgabe 3\footnote{Niederschrift von Santa Claus}}
Seien $\Prob_1, \Prob_2$ Wahrscheinlichkeitsmaße auf $(\Omega, \xi)$ und $\Prob: \xi \to \mathbb{R}$ definiert durch $\Prob(E) = \min \set{\Prob_1(E), \Prob_2(E)}$. Z.z. $\Prob$ ist ein Wahrscheinlichkeitsmaß genau dann, wenn $\Prob_1 = \Prob_2$.

\textbf{Beweis}: \\
"$\Leftarrow$": \\
Wegen $\Prob_1 = \Prob_2$ folgt $\forall E \in \xi: \Prob_1(E) = \Prob_2(E)$ und damit
\begin{equation}
\Prob(E) = \min \set{\Prob_1(E), \Prob_2(E)} = \min \set{\Prob_1(E), \Prob_1(E)} = \Prob_1(E)
\end{equation}
Also gilt $\Prob = \Prob_1 = \Prob_2$ und damit die Behauptung.

"$\Rightarrow$": \\
Sei $E \in \xi$. Dann gilt
\begin{equation}\label{3}\begin{split}
1-\min \set{\Prob_1(E),\Prob_2(E)}
&= 1- \Prob(E) \\
&\stackrel{2}{=} \Prob(E^c)\\
&= \min \set{\Prob_1(E^c),\Prob_2(E^c)} \\
&= \min \set{1-\Prob_1(E),1-\Prob_2(E)} \\
&= 1- \max \set{\Prob_1(E),\Prob_2(E)} \\
\end{split}\end{equation}
Wegen $\max \set{a,b} = \min \set{a,b} \Leftrightarrow a = b$ folgt
$\Prob_1(E) = \Prob_2(E)$ und damit $\Prob_1 = \Prob_2$.

Gleichheit 2 in Gleichung (\ref{3}) gilt wegen $1 = \Prob(\Omega) = \Prob(E \dot{\cup} E^c) = \Prob(E) + \Prob(E^c)$.
$\mbox{}$ \hfill $\square$
\subsection*{Aufgabe 4\footnote{Niederschrift von Santa Claus}}


\label{LastPage}
\end{document}
