\documentclass[11pt,a4paper,ngerman]{article}
\usepackage[bottom=2.5cm,top=2.5cm]{geometry} 
\usepackage[ngerman]{babel}
\usepackage[utf8]{inputenc} 
\usepackage[T1]{fontenc} 
\usepackage{ae} 
\usepackage{amssymb} 
\usepackage{amsmath}
\usepackage{amsthm} 
\usepackage{graphicx}
\usepackage{fancyhdr}
\usepackage{fancyref}
\usepackage{listings}
\usepackage{xcolor}
\usepackage{stmaryrd}
\usepackage{paralist}

\usepackage[pdftex, bookmarks=false, pdfstartview={FitH}, linkbordercolor=white]{hyperref}
\usepackage{fancyhdr}
\pagestyle{fancy}
\fancyhead[C]{Stochastik I}
\fancyhead[L]{Übung 1}
\fancyhead[R]{WiSe 2013/14}
\fancyfoot{}
\fancyfoot[L]{}
\fancyfoot[C]{\thepage \hspace{1px} of \pageref{LastPage}}
\renewcommand{\footrulewidth}{0.5pt}
\renewcommand{\headrulewidth}{0.5pt}
\newcommand{\set}[1]{ \{ #1 \}}
\newcommand{\Prob}{\mathbb{P}}
\setlength{\parindent}{0pt} 
\setlength{\headheight}{0pt}

\date{}
\title{Übung 1}
\author{Max Wisniewski$^\curlywedgedownarrow$, Alexander Steen$^\dagger$}


%%
%% Enviroments for proofs and lemmas
%%
\newtheorem{prop}{\bfseries Behauptung}
\newtheorem{lemma}{\bfseries Lemma}

\begin{document}

\lstset{language=Pascal, basicstyle=\ttfamily\fontsize{10pt}{10pt}\selectfont\upshape, commentstyle=\rmfamily\slshape, keywordstyle=\rmfamily\bfseries, breaklines=true, frame=single, xleftmargin=3mm, xrightmargin=3mm, tabsize=2, mathescape=true}

\renewcommand{\figurename}{Figure}

\maketitle
\thispagestyle{fancy}

\begin{center}
\textbf{Hinweis}: Die Angabe, wer welche Aufgabe in \LaTeX formuliert hat, ist als hochgestellter Symbol der Aufgabenüberschrift zu entnehmen.
\end{center}


\subsection*{Aufgabe 1$^\curlywedgedownarrow$}
Sei $\Omega = \{1,2,3,4\}$. Geben Sie alle $\sigma$-Algebren über $\Omega$ an. \\

\textbf{Lösung}: \\
Es können alle $\sigma$-Algebren folgendermaßen erzeugt werden: Sei $\mathcal{P}$ eine Partition von $\Omega$. Dann ist $\set{\bigcup_i P_i|\set{P_1,P_2,\ldots} \subseteq \mathcal{P}}$ eine $\sigma$-Algebra auf $\Omega$. Dies können wir nun für alle Partitionen von $\Omega$ tun. Es bezeichnen $\xi^j_i$ die $\sigma$-Algebren über $\Omega$: \\

\emph{von 1-elementigen Partitionen}:
\begin{enumerate}[$\xi^1_1 = $]
% Partition 1,2,3,4
\item $\{\emptyset, \Omega \}$
\end{enumerate}
\emph{von 2-elementigen Partitionen}:
\begin{enumerate}[$\xi^2_1 = $]
%Partition 1;2,3,4
\item $\set{\emptyset, \set{1}, \set{2,3,4}, \Omega}$
%Partition 2; 1,3,4
\item $\set{\emptyset, \set{2}, \set{1,3,4},\Omega}$
%Partition 3;1,2,4
\item $\set{\emptyset, \set{3}, \set{1,2,4},\Omega}$
%Partition 4;1,2,3
\item $\set{\emptyset, \set{4}, \set{1,2,3},\Omega}$
% Partition 1,2;    3,4
\item $\{\emptyset, \set{1,2}, \set{3,4}, \Omega \}$
% Partition 1,3;    2,4
\item $\{\emptyset, \set{1,3}, \set{2,4}, \Omega \}$
% Partition 1,4;    2,3
\item $\{\emptyset, \set{1,4}, \set{2,3}, \Omega \}$
\end{enumerate}
\emph{von 3-elementigen Partitionen}:
\begin{enumerate}[$\xi^3_1 = $]
%Partition 1,2;3;4
\item $\{\emptyset, \set{3}, \set{4}, \set{1,2}, \set{3,4}, \set{1,2,3}, \set{1,2,4}, \Omega \}$
%Partition 1,3;2;4
\item $\{\emptyset, \set{2}, \set{4}, \set{1,3}, \set{2,4}, \set{1,2,3}, \set{1,3,4}, \Omega \}$
%Partition 1,4;2;3
\item $\{\emptyset, \set{2}, \set{3}, \set{1,4}, \set{2,3}, \set{1,2,4}, \set{1,3,4}, \Omega \}$
%Partition 2,3;1;4
\item $\{\emptyset, \set{1}, \set{4}, \set{1,4}, \set{2,3}, \set{1,2,3}, \set{2,3,4}, \Omega \}$
%Partition 2,4;1;3
\item $\{\emptyset, \set{1}, \set{3}, \set{1,3}, \set{2,4}, \set{1,2,4}, \set{2,3,4}, \Omega \}$
%Partition 3,4;1;2
\item $\{\emptyset, \set{1}, \set{2}, \set{1,2}, \set{3,4}, \set{1,3,4}, \set{2,3,4}, \Omega \}$
\end{enumerate}
\emph{von 4-elementigen Partitionen}:
\begin{enumerate}[$\xi^4_1 = $]
% Partition 1;2;3;4
\item $\mathcal{P}(\Omega) = \{\emptyset, \set{1},\set{2},\set{3}, \set{4}, \set{1,2}, \ldots, \set{2,3,4}, \Omega \}$
\end{enumerate}

\begin{lemma}
Sei $\Omega$ eine endliche Menge. Dann existiert eine Bijektion zwischen der Menge $\Xi := \set{\xi | \xi \in \mathcal{P}(\Omega) \text{ ist $\sigma$-Algebra}}$ aller $\sigma$-Algebren über $\Omega$ und der Menge $\mathfrak{P}$  der Partitionen von $\Omega$.	
\end{lemma}

\textbf{Beweis}: \\
"$\Xi \to \mathfrak{P}$": \\
Sei $\xi$ eine $\sigma$-Algebra auf $\Omega$. Dann ist 
\begin{equation*}P = \{ E \in \xi \setminus \emptyset \; | \; C \subseteq E \Rightarrow C = \emptyset \lor C = E  \} \end{equation*}
eine Partition von $\Omega$. Denn

....

"$\mathfrak{P} \to \Xi$": \\
Sei $P = \left\{ P_i \right\}_{i \in I} \in \mathfrak{P}$ eine Partition von $\Omega$ mit $I$ abzählbare Indexmenge. Dann ist die Menge
\begin{equation*}\xi := \{ \bigcup_{j \in J} P_j | J \subseteq I \} \end{equation*}
eine $\sigma$-Algebra auf $\Omega$. Denn\\

(1) $\emptyset, \Omega \in \xi$ \\
Wähle $J := \emptyset$ (bzw. $J := I$), dann ist $\bigcup_{j \in \emptyset} P_j = \emptyset \in \xi$ (bzw. $\bigcup_{j \in I} P_j = \Omega \in \xi$).

(2) $E \in \xi \Rightarrow E^c \in \xi$ \\
Sei $E \in \xi$. Dann ex. ein $J \subseteq I$ mit $\bigcup_{j \in J} P_j = E$. Wähle $J' := I \setminus J$.\\
Dann ist $\bigcup_{j' \in J'} P_{j'} = \Omega \setminus E = E^c \in \xi$.

(3) $\left( E_\ell \right)_{\ell \in L} \subseteq \xi \Rightarrow \bigcup_{\ell \in L} E_\ell \in \xi$ \\
Gilt nach Konstruktion.


 $\mbox{}$ \hfill $\square$

\begin{prop}
 Es gibt keine weiteren $\sigma$-Algebren über $\Omega$. 
\end{prop}
\textbf{Beweis}: Wie man leicht nachprüfen kann, sind alle obigen Mengensysteme gültige $\sigma$-Algebren. Bezeichne $S_{n,k}$ die Stirlingzahl zweiter Art.
Dann gilt für die Anzahl $N$ der verschiedenen Partitionen von $\Omega$: 
\begin{equation}
N = \sum_{k=1}^4 S_{4,k} = 1 +6+ 7+ 1=15
\end{equation}
Da zusätzliche alle obenstehenden Mengensysteme paarweise verschiedenen sind, folgt die Aussage.
$\mbox{}$ \hfill $\square$

\subsection*{Aufgabe 2$^\dagger$}
Seien $\Omega$ eine Menge, $\xi$ eine $\sigma$-Algebra auf $\Omega$ und $\xi_C := \set{E \cap C | E \in \xi}$ für ein $C \subseteq \Omega$.
Zu zeigen:
\begin{enumerate}[(i)]
\item $\xi_C$ ist eine $\sigma$-Algebra \\
\textbf{Beweis}: $\xi_C$ ist $\sigma$-Algebra über $C$. \\
(i) $\emptyset ,C \in \xi_C$ \\
Da $\emptyset, \Omega \in \xi$ folgt $\emptyset \cap C = \emptyset \in \xi_C$ und $\Omega \cap C = C \in \xi_C$.\\
(ii) $\forall E \in \xi_C: E^c \in \xi_C$ \\
Sei $E \in \xi_C$. Dann ex. ein $E' \in \xi$ s.d. $E = E' \cap C$. Dann gilt auch $E'^c \in \xi$ (Komplement von $E$ bzgl. $\Omega$) und damit für 
$E^c$ (Komplement von $E$ bzgl. $C$) $E^c = E'^c \cap C \in \xi_C$, da $C \subseteq \Omega$. \\
(iii) $\forall \left(E_i\right)_{i \in I} \subseteq \xi_C: \bigcup_{i \in I} {E_i} \in \xi_C$  \\
Sei $\left(E_i\right)_{i \in I} \subseteq \xi_C$ eine Folge von Ereignissen. Dann gibt es für jedes $E_i$ ein $E'_i \in \xi$ mit $E'_i \cap C = E_i$.
Dann ist aber auch $\bigcup_{i \in I} E'_i \in \xi$ und damit $\left( \bigcup_{i \in I} E'_i \right) \cap C = \bigcup_{i \in I} \left( E'_i \cap C \right) = \bigcup_{i \in I} E_i \in \xi_C$
$\mbox{}$ \hfill $\square$
\item $C \in \xi \Rightarrow \xi_C = \set{E \in \xi | E \subseteq C} =: \xi'_C$ \\
\textbf{Beweis}: Es gelte $C \in \xi$. \\
"$\subseteq$": \\
Sei $E \in \xi_C$. Dann ex. ein $E' \in \xi$ s.d. $E = E' \cap C$. \\
(1) z.z. $E \in \xi$: Es gilt
\begin{equation}\begin{split}
E^c &= \Omega \setminus E \\
&= \Omega \setminus \left(E' \cap C \right) \\
&= \Omega \setminus E' \cup \Omega \setminus C \\
&= E'^c \cup C^c
\end{split}\end{equation}
Da $C, E' \in \xi \Rightarrow C^c, E'^c \in \xi$ und damit auch $E'^c \cup C^c \in \xi$. Also ist $E^c \in \xi$ und damit auch $E \in \xi$.

(2) z.z. $E \subseteq C$: Da $E = E' \cap C$ für ein $E' \in \xi$, gilt $E \subseteq C$.\\
Also gilt $E \in \xi'_C$.

"$\supseteq$": \\
Sei $E \in \xi'_C$. Dann gilt $E \in \xi$, $E \subseteq C$. Dann ist  $E \cap C = E \in \xi_C$.
$\mbox{}$ \hfill $\square$
\end{enumerate}


\subsection*{Aufgabe 3$^\curlywedgedownarrow$}
Seien $\Prob_1, \Prob_2$ Wahrscheinlichkeitsmaße auf $(\Omega, \xi)$ und $\Prob: \xi \to \mathbb{R}$ definiert durch $\Prob(E) = \min \set{\Prob_1(E), \Prob_2(E)}$. Z.z. $\Prob$ ist ein Wahrscheinlichkeitsmaß genau dann, wenn $\Prob_1 = \Prob_2$.

\textbf{Beweis}: \\
"$\Leftarrow$": \\
Wegen $\Prob_1 = \Prob_2$ folgt $\forall E \in \xi: \Prob_1(E) = \Prob_2(E)$ und damit
\begin{equation}
\Prob(E) = \min \set{\Prob_1(E), \Prob_2(E)} = \min \set{\Prob_1(E), \Prob_1(E)} = \Prob_1(E)
\end{equation}
Also gilt $\Prob = \Prob_1 = \Prob_2$ und damit die Behauptung.

"$\Rightarrow$": \\
Sei $E \in \xi$. Dann gilt
\begin{equation}\label{3}\begin{split}
1-\min \set{\Prob_1(E),\Prob_2(E)}
&= 1- \Prob(E) \\
&\stackrel{2}{=} \Prob(E^c)\\
&= \min \set{\Prob_1(E^c),\Prob_2(E^c)} \\
&= \min \set{1-\Prob_1(E),1-\Prob_2(E)} \\
&= 1- \max \set{\Prob_1(E),\Prob_2(E)} \\
\end{split}\end{equation}
Wegen $\max \set{a,b} = \min \set{a,b} \Leftrightarrow a = b$ folgt
$\Prob_1(E) = \Prob_2(E)$ und damit $\Prob_1 = \Prob_2$.

Gleichheit 2 in Gleichung (\ref{3}) gilt wegen $1 = \Prob(\Omega) = \Prob(E \dot{\cup} E^c) = \Prob(E) + \Prob(E^c)$.
$\mbox{}$ \hfill $\square$
\subsection*{Aufgabe 4$^\dagger$}

Sei $\Omega$ eine Menge mit $\left| \Omega \right| \geq 2$ und sei $\mathcal{E} = \mathcal{P}\left( \Omega \right)$ die Potenzmenge von $\Omega$. Seien
zwei verschiedene Elemente $x_0, y_0 \in \Omega$ fest gewählt. Weiter seien $a,b,c,d \in \mathbb{R}$ reele Zahlen. Definieren wir die Abbildung
$$
    \mathbb{P}(E) := \left\{ \begin{array}{lr} 
        a, &\quad \text{wenn } x_0 \not\in E \text{ and } y_0 \not\in E\\
        b, &\quad \text{wenn } x_0 \in E \text{ and } y_0 \not\in E\\
        c, &\quad \text{wenn } x_0 \not\in E \text{ and } y_0 \in E\\
        d, &\quad \text{wenn } x_0 \in E \text{ and } y_0 \in E
    \end{array}\right.
$$

Finden Sie notwendige und hinreichende Bedingungen an $a,b,c,d$, dass durch $\mathbb{P}$ ein Wahrscheinlichkeitsmaß definiert wird.\\

\textbf{Lösung:}\\

Fangen wir zunächst mit den notwendigen Bedingungen an, d.h.
eine Eigenschaft $A$ mit
\[
    \mathbb{P}\text{ ist Warscheinlichkeitsmaß} \Rightarrow A
\]

\begin{enumerate}[1.]
    \item Für ein WMaß $\mathbb{P}$ muss gelten $\mathbb{P}(\emptyset) = 0$. Da nun $x_0, y_0 \not\in \emptyset$ gilt, muss auch
            $\mathcal{P}(\emptyset) = a = 0$ gelten. Daher ist die Bedingung $a = 0$ notwendig.
    \item Für ein WMaß $\mathbb{P}$ muss gelten $\mathbb{P}(\Omega) = 1$. Da $x_0, y_0 \in \Omega$ gilt, muss auch
            $\mathcal{P}(\Omega) = d = 1$. Daher ist die Bedingung $d = 1$ notwendig.
    \item Ist $\mathcal{P}$ ein WMaß, so muss $\mathcal{P}(A \dot\cup B) = \mathcal{P}(A) + \mathcal{P}(B)$. Sollte also
        eine Menge $E\in\mathcal{E}$ mit $x_0 \in E$ und $y_0 \not\in E$ existieren, so ist auch
            $b + c = 1$ notwendig. 
\end{enumerate}

Alle 3 zusammen sind hinreichend.\\
\textbf{Beweis:}\\
Sei $a = 0, b+c = 1, d = 1$.
\begin{enumerate}[1.]
    \item z.z. $\mathbb{P}(\emptyset) = 0$.\\
        Wie schon gezeigt ist $\mathbb{P}(\emptyset) = a = 0$.
    \item z.z. $\mathbb{P}(\Omega) = 1$.\\
        Wie schon gezeigt ist $\mathbb{P}(\Omega) = d = 1$.
    \item z.z. $\mathbb{P}(\dot\bigcup A_i) = \overset{\infty}{\underset{i=1}{\sum}} A_i$.\\
        Wir unterscheiden 4 Fälle.
        \begin{enumerate}[a.]
            \item $x_0, y_0 \not\in \dot\bigcup A_i$. Dann gilt insbesondere $\forall i\in \mathbb{N} \, : \, x_0, y_0 \not\in A_i$.\\
                Also ist $\mathbb{P}(\dot\bigcup A_i) = 0 = \overset{\infty}{\underset{i=1}{\sum}} A_i $.
            \item $x_0 \in \dot\bigcup A_i$ und $y_0 \not\in \dot\bigcup A_i$. Dann gilt $\forall i \in \mathbb{N} \, : \, y_0 \not\in A_i$ und
                $\exists ! i \in \mathbb{N} \, : \, x_0 \in A_i$. Sei $j$ das eindeutige Elemente mit $x_0 \in A_j$.\\
                Dann ist $\mathbb{P}(\dot\bigcup A_i) = b$ und
                \[
                    \overset{\infty}{\underset{i=1}{\sum}} \mathbb{P}(A_i) = \mathbb{P}(A_j) + \underset{i \in \mathbb{N} \setminus \{j \}}{\sum} A_i = b + 0.
                \]
                Analog geht $x_0 \not\in \dot\bigcup A_i$ und $y_0 \in \dot\bigcup A_i$.\\
            \item $x_0, y_0 \in \dot\bigcup A_i$. Falls $j$ existiert mit $x_0, y_0 \in A_i$, dann ist $\mathbb{P}(A_i) = 1$ und insbesondere gilt die Behauptung.
                Sonst muss es zwei Zahlen $i,j \in \mathbb{N}$ mit $i \not= j$ geben, so dass $x_0 \in A_i$ und $y_0 \in A_j$.
                Dann ist
                \[
                    \underset{k \in \mathbb{N}}{\sum} \mathbb{P}(A_k) = \mathbb{P}(A_i) + \mathbb{P}(A_j) + \underset{k \in \mathbb{N}}{\sum} \mathbb{P}(A_k) = b + c + 0 = 1
                \]
                Was genau $\mathbb{P}(\dot\bigcup A_i) = 1$ entspricht.
        \end{enumerate}
\end{enumerate}

Eine weitere hinreichende Bedingung ist:
\[
    a = 0, d = 1 \text{ und } \forall E \in \mathcal{E} \, : \, x_0, y_0 \in E \lor x_0, y_0 \not\in E
\]

Dies folgt aus dem letzten Beweis, da wir Fall b) für beide Möglichkeiten nicht mehr haben und der zweite Fall in Fall c) auch nicht möglich ist.

\label{LastPage}
\end{document}
