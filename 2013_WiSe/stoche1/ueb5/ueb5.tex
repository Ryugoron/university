\documentclass[11pt,a4paper,ngerman]{article}
\usepackage[bottom=2.5cm,top=2.5cm]{geometry}
\usepackage[ngerman]{babel}
\usepackage[utf8]{inputenc}
\usepackage[T1]{fontenc}
\usepackage{ae}
\usepackage{amssymb}
\usepackage{amsmath}
\usepackage{amsthm}
\usepackage{graphicx}
\usepackage{fancyhdr}
\usepackage{fancyref}
\usepackage{listings}
\usepackage{xcolor}
\usepackage{stmaryrd}
\usepackage{paralist}

\usepackage[pdftex, bookmarks=false, pdfstartview={FitH}, linkbordercolor=white]{hyperref}
\usepackage{fancyhdr}
\pagestyle{fancy}
\fancyhead[C]{Stochastik I}
\fancyhead[L]{Übung 5}
\fancyhead[R]{WiSe 2013/14}
\fancyfoot{}
\fancyfoot[L]{}
\fancyfoot[C]{\thepage \hspace{1px} of \pageref{LastPage}}
\renewcommand{\footrulewidth}{0.5pt}
\renewcommand{\headrulewidth}{0.5pt}
\newcommand{\set}[1]{ \{ #1 \}}
\newcommand{\Prob}{\mathbb{P}}
\setlength{\parindent}{0pt}
\setlength{\headheight}{0pt}

\newcommand{\N}{\mathbb{N}}
\newcommand{\Q}{\mathbb{Q}}
\newcommand{\R}{\mathbb{R}}
\newcommand{\bigO}{\mathcal{O}}
\newcommand{\Rarr}{\Rightarrow}
\newcommand{\rarr}{\rightarrow}
\newcommand{\Pot}{\mathcal{P}}
\newcommand{\abs}[1]{ |#1|}
\newcommand{\solved}{$\mbox{}$ \hfill $\square$}
\newcommand{\Epsilon}{\mathcal{E}}

\newcommand{\maxw}{$^\curlywedgedownarrow$}
\newcommand{\alex}{$^\dagger$}
\newcommand{\marcel}{$^\diamondsuit$}

\date{}
\title{Übung 5}
\author{Max Wisniewski\maxw, Alexander Steen\alex, Marcel Ehrhardt\marcel}


%%
%% Enviroments for proofs and lemmas
%%
\newtheorem{prop}{\bfseries Behauptung}
\newtheorem{lemma}{\bfseries Lemma}

\begin{document}

\lstset{language=Pascal, basicstyle=\ttfamily\fontsize{10pt}{10pt}\selectfont\upshape, commentstyle=\rmfamily\slshape, keywordstyle=\rmfamily\bfseries, breaklines=true, frame=single, xleftmargin=3mm, xrightmargin=3mm, tabsize=2, mathescape=true}

\renewcommand{\figurename}{Figure}

\maketitle
\thispagestyle{fancy}

\begin{center}
\textbf{Hinweis}: Die Angabe, wer welche Aufgabe in \LaTeX\ formuliert hat, ist als hochgestelltes Symbol der Aufgabenüberschrift zu entnehmen.
\end{center}


\subsection*{Aufgabe 1\alex}
Es Punkt $(a,b)$ wird zufällig im Einheitsquadrat gewählt und die Summe $s := a + b$ ausgegeben.
Bestimmen Sie die Verteilungsfunktion $F(x) = \Prob(a + b \leq x)$ und daraus die Dichtefunktion $f(x)$. \\

\textbf{Lösung}: \\
Bei der Gleichverteilung auf $[0,1] \times [0,1]$ gilt
$\Prob(E) = \frac{\text{Flächeninhalt von }E}{\text{Flächeninhalt von }\Omega} = \text{Flächeninhalt von }E$,
für alle messbaren Mengen $E \subseteq [0,1] \times [0,1]$. \\
Es bezeichne $E_x = \set{(a,b) \in [0,1]^2 | a+b \leq x}$.
Dann ist 
\begin{equation*}
\Prob(a+b \leq x) = \Prob(\set{(a,b) \in [0,1]^2 | a+b \leq x}) = \Prob(E_x)
\end{equation*}

Für fixiertes $a \in [0,1]$ können nun drei Fälle unterschieden werden in denen $(a,b) \in E_x$:
\begin{enumerate}[(i)]
	\item $0 \leq x \leq 1$: $b \leq x - a$
	\item $1 < x \leq 2$: $b \leq \min(1, x - a)$
	\item $2 < x$: $b \leq 1$
\end{enumerate}

Damit können wir nun eine Funktion $b_x: [0,1] \to \mathbb{R}$ beschreiben, die uns die obere Grenze der erlaubten $b$ für ein gegebenes $a$ zu einer Summengrenze $x$ gibt:
\begin{equation*}
b_x(a) = \begin{cases}
                  x - a & \text{, falls $0 \leq x \leq 1$} \\
                  \min(1,x - a) & \text{, falls $1 < x \leq 2$} \\
                  1 & \text{, falls $2 < x$} \\
               \end{cases}
\end{equation*}

Die Verteilungsfunktion $F(x)$ erhalten wir nun durch\\
$F(x) = \Prob(a+b \leq x) = \Prob(E_x) = \text{Flächeninhalt von $E_x$}$.
Dies ist für die drei Fälle gegeben durch:
\begin{enumerate}[(i)]
\item $0 \leq x \leq 1$: 
\begin{equation*}
 \Prob(E_x) = \int_0^1 b_x(a) \; da =  \int_0^1 x-a \; da = x - \frac{1}{2}
\end{equation*}
\item $1 < x \leq 2$: 
\begin{equation*}\begin{split}
 \Prob(E_x) &= \int_0^1 b_x(a) \; da =  \int_0^1 \min(1, x - a) \; da
\end{split}\end{equation*}
Um dieses Integral aufzulösen, teilen wir es in zwei Integrale auf: \\
Es gilt $1 \leq x-a \Leftrightarrow a \geq x-1$ und damit:
\begin{equation*}\begin{split}
\int_0^1 \min(1, x - a) \; da = \int_0^{x-1} 1 \; da + \int_{x-1}^1 (x-a) \; da = -\frac{1}{2}x^2 +2x-1
\end{split}\end{equation*}
\item $2 < x$: 
\begin{equation*}
 \Prob(E_x) = \int_0^1 b_x(a) \; da =  \int_0^1 1 \; da = x-1
\end{equation*}
\end{enumerate}

Es ist also
\begin{equation*}
F(x) = \Prob(E_x) = \begin{cases}
                          x - \frac{1}{2} & \text{, falls $0 \leq x \leq 1$} \\
                          -\frac{1}{2}x^2 +2x-1 & \text{, falls $1 < x \leq 2$} \\
                          x - 1 & \text{, falls $2 < x$} \\
                     \end{cases}
\end{equation*}
die Verteilungsfunktion.
Die Dichtefunktion $f$ erhalten wir nun durch Ableitung:
\begin{equation*}
f(x) = \frac{d\, F}{d \, x} = \begin{cases}
                          -x +2 & \text{, falls $1 < x \leq 2$} \\
                          1 & \text{, sonst} \\
                     \end{cases}
\end{equation*}
\subsection*{Aufgabe 2}


\subsection*{Aufgabe 3\alex}
Es sei $\Omega = [-1,1]$, verstehen mit der $\sigma$-Algebra der Borelmengen.
$\Prob_0$ sei das Diracmaß bei 0.\\
Zu zeigen: Es gibt keine stetige Funktion $f: \Omega \to  \mathbb{R}$,
sodass $\Prob_0$ das von $f$ induzierte Wahrscheinlichkeitsmaß ist. \\

\textbf{Beweis} durch Widerspruch: \\
Sei $f: \Omega \to \mathbb{R}$ eine stetige Funktion. Dann ist $f$ integrierbar und
die Stammfunktion $F = \int f$ existiert. Weiterhin sei $\Prob_0([a,b]) = \int_a^b f(x) dx$
das von $f$ induzierte Wahrscheinlichkeitsmaß. Da $[a,b]$ ein Intervall ist,
stimmt $\Prob_0([a,b]) = \int_a^b f(x) dx$ mit dem Riemann-Integral überein und es gilt

\begin{equation*}\begin{split}
1 = \Prob_0([-1,1]) &= \int_{-1}^1 f(x) dx \\
  &=  \int_{-1}^0 f(x) dx +  \int_{0}^1 f(x) dx \\
  &= \Prob_0([-1,0]) +  \Prob_0([0,1]) = 1 + 1 = 2 \quad \lightning
\end{split}\end{equation*}
Also kann ein solches $f$ nicht existieren.
\mbox{} \hfill $\square$

\subsection*{Aufgabe 4}

X sei eine reellwertige Zufallsvariable auf $\Omega$. Definiere $Y : \Omega \rarr \R$
durch $Y(\omega) := X(\omega)$ f"ur $\abs{X(\omega)}< 1$  und $Y(\omega) := 0$
sonst. Zeigen Sie, dass auch Y eine Zufallsvariable ist. \\

{\bf Beweis:}

Nach VL ist $X$ ZV $\Leftrightarrow \forall F\in \mathcal{F}:\ \set{X \in F} \in \Epsilon$. \\

Da $C = \R$ in $Y,X: \Omega \rarr C$, ist $\mathcal{F} = \sigma(\bigO)$ (also
die Borel-$\sigma$-Algebra). \\

Z.z. $\forall F \in \mathcal{F}:\ \set{Y \in F} \in \Epsilon$. \\

Sei nun ein $F \in \mathcal{F}$ beliebig.
Betrachte
\begin{eqnarray}
  \set{Y \in F} &=& \set{\omega\ |\ Y(\omega) \in F} \\
  &=& \set{\omega\ |\ Y(\omega) \in F,\ \abs{X(\omega)} < 1} \cup
      \set{\omega\ |\ Y(\omega) \in F,\ \abs{X(\omega)} \ge 1} \\
  &=& \set{\omega\ |\ X(\omega) \in F,\ \abs{X(\omega)} < 1} \cup
      \set{\omega\ |\ 0 \in F,\ \abs{X(\omega)} \ge 1} \\
  &=& \set{\omega\ |\ X(\omega) \in F\cap(-1, 1)} \cup
      \set{\omega\ |\ 0 \in F,\ \abs{X(\omega)} \ge 1}
\end{eqnarray}

Da $F\cap(-1, 1)$ eine Borelmenge ist und $X$ eine ZV ist, gilt
$\set{\omega\ |\ X(\omega) \in F\cap(-1, 1)} \in \Epsilon$. \\

Wir m"ussen also nur zeigen, dass die zweite Vereinigung $\in \Epsilon$ ist.
Angenommen $0 \not\in F$, dann ist sie $ = \emptyset \in \Epsilon$. \\

Sei also $0 \in F$, dann ist sie
\begin{eqnarray}
  \set{\omega\ |\ \abs{X(\omega)} \ge 1} &=& 
  \set{\omega\ |\ X(\omega) \ge 1\ \lor\ X(\omega) \le -1} \\
  &=& \set{\omega\ |\ X(\omega) \ge 1} \cup \set{\omega\ |\ X(\omega) \le -1} \\
  &=& \set{X \in [1, +\infty)} \cup \set{X \in (-\infty, -1]}
\end{eqnarray}

Da $(-\infty, -1]$ und $[1, +\infty)$ Borelmengen sind und $X$ eine ZV ist,
sind die beiden Mengen $\in \Epsilon$. Wodurch auch $Y$ eine ZV ist. \solved

\label{LastPage}
\end{document}
