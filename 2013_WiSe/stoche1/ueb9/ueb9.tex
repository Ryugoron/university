\documentclass[11pt,a4paper,ngerman]{article}
\usepackage[bottom=2.5cm,top=2.5cm]{geometry}
\usepackage[ngerman]{babel}
\usepackage[utf8]{inputenc}
\usepackage[T1]{fontenc}
\usepackage{ae}
\usepackage{amssymb}
\usepackage{amsmath}
\usepackage{amsthm}
\usepackage{graphicx}
\usepackage{fancyhdr}
\usepackage{fancyref}
\usepackage{listings}
\usepackage{xcolor}
\usepackage{stmaryrd}
\usepackage{paralist}

\usepackage[pdftex, bookmarks=false, pdfstartview={FitH}, linkbordercolor=white]{hyperref}
\usepackage{fancyhdr}
\pagestyle{fancy}
\fancyhead[C]{Stochastik I}
\fancyhead[L]{Übung 9}
\fancyhead[R]{WiSe 2013/14}
\fancyfoot{}
\fancyfoot[L]{}
\fancyfoot[C]{\thepage \hspace{1px} of \pageref{LastPage}}
\renewcommand{\footrulewidth}{0.5pt}
\renewcommand{\headrulewidth}{0.5pt}
\newcommand{\set}[1]{ \{ #1 \}}
\newcommand{\Prob}{\mathbb{P}}
\setlength{\parindent}{0pt}
\setlength{\headheight}{0pt}

\newcommand{\N}{\mathbb{N}}
\newcommand{\Q}{\mathbb{Q}}
\newcommand{\R}{\mathbb{R}}
\newcommand{\bigO}{\mathcal{O}}
\newcommand{\Rarr}{\Rightarrow}
\newcommand{\rarr}{\rightarrow}
\newcommand{\Pot}{\mathcal{P}}
\newcommand{\abs}[1]{ |#1|}
\newcommand{\solved}{$\mbox{}$ \hfill $\square$}
\newcommand{\Epsilon}{\mathcal{E}}

\newcommand{\maxw}{$^\curlywedgedownarrow$}
\newcommand{\alex}{$^\dagger$}
\newcommand{\marcel}{$^\diamondsuit$}

\date{}
\title{Übung 9}
\author{Max Wisniewski\maxw, Alexander Steen\alex, Marcel Ehrhardt\marcel}


%%
%% Enviroments for proofs and lemmas
%%
\newtheorem{prop}{\bfseries Behauptung}
\newtheorem{lemma}{\bfseries Lemma}

\begin{document}

\lstset{language=Pascal, basicstyle=\ttfamily\fontsize{10pt}{10pt}\selectfont\upshape, commentstyle=\rmfamily\slshape, keywordstyle=\rmfamily\bfseries, breaklines=true, frame=single, xleftmargin=3mm, xrightmargin=3mm, tabsize=2, mathescape=true}

\renewcommand{\figurename}{Figure}

\maketitle
\thispagestyle{fancy}

\begin{center}
\textbf{Hinweis}: Die Angabe, wer welche Aufgabe in \LaTeX\ formuliert hat, ist als hochgestelltes Symbol der Aufgabenüberschrift zu entnehmen.
\end{center}


\subsection*{Aufgabe 1\maxw}

\subsubsection*{(a)}
Sei die Menge $\Omega = [47]$ mit der Gleichverteilung versehen. Zeigen Sie, dass wenn Ereignisse $A$ und $B \in \mathcal{P}(\Omega)$ unabhängig sind,
dann eine der beiden Mengen leer oder $\Omega$ ist.\\

\textbf{Lösung:}\\

Seien $\Omega, A, B$ wie in der Aufgabe. $A$, $B$ heißen unabhängig, wenn
\[
    \Prob(A \cap B) = \Prob(A) \cdot \Prob(B)
\]
gilt.

Da wir es mit einer Gleichverteilung zu tun haben, ist die Wahrscheinlichkeit einer Menge $C$, $\Prob(C) = \frac{|C|}{|Omega|} = \frac{|C|}{47}$.

Setzen wir dies in die unabhängigkeit ein, so erhalten wir
\[
    \begin{array}{lrcl}
        &\frac{\left| A \cap B \right|}{47} &=& \frac{|A|}{47} \cdot \frac{|B|}{47}\\
        \Leftrightarrow & 47 \cdot \left| A \cap B \right| &=& |A| \cdot |B|
    \end{array}
\]

Sollte der Schnitt $A \cap B$ leer sein, so müsste $A$ oder $B$ die leere Menge sein, da $0 = |A| \cdot |B|$ impliziert, das einer der Faktoren die größe Null hat
und der Satz wäre bewiesen.\\

Sollte der Schnitt $A \cap B$ nicht leer sein, so wissen wir, dass die linke Seite der Gleichung den Faktor $47$ enthält. Da $47$ eine Primzahl ist, muss
aber $47$ ein Primteiler von $A$ oder $B$ sein (unteilbarkeit von $47$).\\

Nun ist $A \subseteq \Omega$, $B \subseteq \Omega$ und daher gilt $|A| \leq 47$ und $|B| \leq 47$. Da eins der beiden aber $47$ als Teiler hat muss $A$ oder $B$
$47$ Elemente enthalten, was bei $|\Omega| = 47$ komplett $\Omega$ ist.

\mbox{}\hfill$\square$

\subsubsection*{(b)}

Zeigen Sie, dass die analoge Behauptung für $\Omega = [48]$ nicht richtig ist.\\

\textbf{Lösung:}\\

Ist $\Omega = \left[ 48 \right]$, so trifft unser zweite Fall nicht mehr zu. Es gilt zwar immer noch

\[
    48 \cdot \left| A \cap B \right| = |A| \cdot |B|
\]

Aber ich kann nun die $48$ aufteilen. Z.B. ist $A = \{1,2\}$ und $B = \{2,\ldots,25\}$.\\

Hier ist
\[
    \begin{array}{rcl}
        A \cap B    &=& \{2\}\\
        |A \cap B| &=& 1\\
        |A|     &=& 2\\
        |B|     &=& 24\\
    \end{array}
\]
und daher auch
\[
    48 \cdot \left| A \cap B \right| = 48 \cdot 1 = 2 \cdot 24 = |A| \cdot |B|
\]

\subsection*{Aufgabe 2\marcel}

Drei Kästen $K_1,K_2,K_3$ enthalten gut durchmischte schwarze und weiße Kugeln.\\

\subsubsection*{(a)}

Aus Kasten $K_3$ wird dreimal mit zurücklegen gezogen. Wie groß ist die Wahrscheinlichkeit, dass die erste Kugel
weiß, die zweite schwarz und die dritte wieder weiß ist.\\

\textbf{Lösung:}\\
In $K_3$ ist eine schwarze und 3 weiße Kugeln. Da wir mit zurücklegen ziehen, ändert sich die 
Wahrscheinlichkeit für eine Farbe von Zug zu Zug nicht. Wir nehmen an, dass \emph{gut durchmischt} Gleichverteilung bedeutet.\\

Dann ist $\Prob(\text{weiß}) = \frac{3}{4}$ und $\Prob(\text{schwarz}) = \frac{1}{4}$.\\
Für die gewünschte Folge erhalten wir also:
\[
    \Prob(E_a) = \Prob(\text{weiß})\Prob(\text{schwarz})\Prob(\text{weiß}) = \frac{3}{4} \frac{1}{4} \frac{3}{4} = \frac{9}{64}
\]

\subsubsection*{(b)}

Nun wird zunächst ein Kasten mit der Wahrscheinlichkeit $\frac{1}{3}$ gewählt und dann eine Kugel gezogen. Mit welcher Wahrscheinlichkeit
ist es eine weiße Kugel.\\

\textbf{Lösung:}\\

Die Wahrscheinlichkeit einen Kasten zu bekommen ist $\Prob(K_i) = \frac{1}{3}$ für $i \in [3]$.
Im ersten Kasten haben wir die Wahrscheinlichkeit
$\Prob(W \,|\, K_1) = \frac{2}{3}$ eine weiße Kugel zu ziehen, im zweiten $\Prob(W \,|\, K_2) = \frac{5}{8}$ und im dritten $\Prob(W \,|\, K_3) = \frac{3}{4}$.
Daraus folgt
\[
    \begin{array}{rcl}
    \Prob(W) &=& \Prob(K_1)\Prob(W \, | \, K_1) + \Prob(K_2)\Prob(W \, | \, K_2) + \Prob(K_3)\Prob(W \, | \, K_3)\\
            &=& \frac{1}{3} \cdot \left( \frac{2}{3} + \frac{5}{8} + \frac{3}{4} \right)\\
            &=& \frac{49}{72}
    \end{array}
\]

\subsubsection*{(c)}

Wenn eine Ziehung wie in (b) eine weiße Kugel liefert, mit welcher Wahrscheinlichkeit wurde dann im ersten Schritt Kasten $K_2$ gewählt.\\

\textbf{Lösung:}\\

Wir benutzen hier den Satz von Bayes da wir in (b) schon alles notwendige berechnet haben.

\[
    \begin{array}{rcl}
        \Prob( K_2 \, | \, W) &=& \frac{\Prob(W \, | \, K_2) \cdot \Prob(K_2)}{\Prob(W)}\\
            &=& \frac{5}{8} \cdot \frac{1}{3} \cdot \frac{72}{49}\\
            &=& \frac{15}{49}
    \end{array}
\]

\subsection*{Aufgabe 3\alex}
\newcommand{\D}{\mathcal{D}}
Sei $(\Omega, \Epsilon, \Prob)$ ein W-Raum und $A,B \in \Epsilon$ zwei unabh. Ereignisse. \\
Z.z.: $\mathcal{D} = \set{E \in \Epsilon \; | \; \set{A,B, E} \text{ ist unabhängig}}$ ist Dynkin-System. \\

\textbf{Beweis}: Sei  $A,B \in \Epsilon$ zwei unabhängige Ereignisse.
\begin{enumerate}[(i)]
\item $\emptyset \in \D, \Omega \in \D$: \\
Es ist $\emptyset \in \Epsilon$ und $\Omega \in \Epsilon$ (da $\Epsilon$ $\sigma$-Algebra). \\
Weiterhin ist $\Prob(A \cap B \cap \emptyset) = \Prob(\emptyset) = 0 = \Prob(A) \Prob(B) \underbrace{\Prob(\emptyset)}_{=0}$ und \\
$\Prob(A \cap B \cap \Omega) = \Prob(A \cap B) = \Prob(A) \Prob(B) =  \Prob(A) \Prob(B) \underbrace{\Prob(\Omega)}_{=1}$. \\
Also ist $\emptyset \in \D$ und $\Omega \in \D$.
\item Für alle $E \in \D$ gilt  $\Omega \setminus E \in \D$: \\
Sei $E \in \D$. Also gilt $\Prob(A \cap B \cap E) = \Prob(A) \Prob(B) \Prob(E)$. Dann gilt aber
\begin{equation*}\begin{split}
 \Prob(A \cap B \cap E^c) &=  \Prob((A \cap B) \cap (\Omega \setminus E)) \\
&=  \Prob((A \cap B \cap \Omega) \setminus (A \cap B \cap E)) \\
&= \Prob((A \cap B) \setminus (A \cap B \cap E)) \\
&=  \Prob(A)\Prob(B) - \Prob(A)\Prob(B)\Prob(E) \\
&= \Prob(A)\Prob(B) \cdot (1 - \Prob(E)) \\
&= \Prob(A)\Prob(B) \Prob(E^c) \\
\end{split}\end{equation*}
und damit $E^c \in \D$.
\item Sei $\left( E_i \right)_{i \in \mathbb{N}}$ eine Folge von paarweise disjunkten Ereignissen mit $E_i \in \D$. \\
Z.z. $\bigcup_i E_i \in \D$. \\
Es ist
\begin{equation*}\begin{split}
 \Prob(A \cap B \cap \bigcup_i E_i) &=  \Prob((A \cap B) \cap \bigcup_i E_i) \\
&= \Prob(\bigcup_i A \cap B \cap E_i) \\
&\stackrel{(*)}{=} \sum_i \Prob(A \cap B \cap E_i) \\
&\stackrel{Unabh.}{=} \sum_i \Prob(A) \Prob(B) \Prob(E_i) \\
&= \Prob(A)\Prob(B) \sum_i \Prob(E_i) \\
&\stackrel{(*)}{=} \Prob(A)\Prob(B) \Prob(\bigcup_i E_i) \\
\end{split}\end{equation*}
und damit $\bigcup_i E_i \in \D$.
(*) gilt, da für disjunkte Folgen $A_1, A_2, \ldots$ gilt: $\Prob(\bigcup_i A_i) = \sum_i \Prob(A_i)$.
\end{enumerate}
Also ist $\D$ ein Dynkin-System.
\mbox{} \hfill $\square$

\subsection*{Aufgabe 4\maxw}

Sei $\Omega = [7]^{11}$ mit Gleichverteilung versehen. Wir definieren $E_{k,l} = \{(x_1, \ldots , x_11) \, | \, x_k = l \}$ für $k \in [11]$ und $l \in [7]$.
Was ist die maximale Anzahl der Ereignisse, die Sie aus diesen $77$ wählen können, so dass sie eine unabhängige Menge von Ereignissen formen.\\

\textbf{Lösung:}\\

Als erstes können wir uns die Wahrscheinlichkeit von einem Ereignis $E_{k,l}$ berechnen. Da alle Elemente gleich wahrscheinlich sind,
müssen wir nur die Anzahl der Elemente in $E_{k,l}$ zählen.\\
In $E_{k,l}$ ist nach definition eine Stelle auf ein Element fixiert und der  Rest ist frei und kann daher beliebig belegt werden.\\
Dies sind $10$ Stellen mit $7$ Ziffern, was $7^{10}$ Kombinationen erlaub.
\[
    \Rightarrow \Prob (E_{k,l}) = \frac{|E_{k,l}|}{|\Omega|} = \frac{7^{10}}{7^{11}} = \frac{1}{7} = 7^{-1}
\]

Wir betrachten als nächstes, wann für eine Menge $E_1, ..., E_n$ mit $E_i = E_{k_i, l_i}$ die Bedingung
\[
    \Prob( \bigcap_i E_i) = \prod_i \Prob (E_i) = 7^{-n}
\]
erfüllt ist.\\

Wir sehen zunächst, dass falls $k_i = k_j$ gilt, auch $l_i = l_j$ gelten muss.\\
\textbf{Beweis:}\\
    Nehmen wir an, dass $k_i = k_j$ und $l_i \not= l_j$ gelten würde.
    Dann folgt
    \[
        E_i \cap E_j = \{(x_1, \ldots, x_{11}) | x_{k_i} = l_i \land x_{k_j} = l_j\}
    \]
    und da $x_{k_i} = x_{k_j}$ gilt ist
    \[
        E_i \cap E_j = \emptyset.
    \]

    Nun ist aber $\Prob (\emptyset) \not= 7^{-2}$ und $E_i$ und $E_j$ können nicht unabhängig sein.\\
\mbox{}\hfill$\square$\\

Dies zeigt uns auch, dass falls $E_i$ und $E_j$ $k_i = k_j$ haben auch in der Tat $E_i = E_j$ gilt und wir daher keine 2 verschiedenen Mengen haben.\\

Wir können also schnell folgern, dass eine Menge von Ereignissen $\{E_1, \ldots, E_n\}$ höchstens $11$ Ereignisse enthält und dass $k_i \not= k_j$
für alle $\forall i,j \in [11] : i \not= j$. Wobei wir den zweiten Teil nun schon bewiesen haben.\\

\textbf{Beweis:}\\
Sei $D = \{E_1, \ldots, E_{n} \}$ mit $n>11$ eine maximale Menge von unabhängigen Ereignissen. Dann gibt es nach Taubenschlagprinzip 2 Ereignisse $E_i$ und $E_j$ mit
$k_i = k_j$. Wie schon gezeigt ist nun aber $E_i = E_j$ also ist auch $D \setminus \{ E_i \}$ eine maximale Menge (in der Tat ist es $D$).
Also hat $D$ maximal $n-1$ Elemente.\\
\mbox{}\hfill$\square$\\

Lässt uns noch den Beweis, dass diese Menge auch unabhängig ist.\\
\textbf{Beweis:}\\
Sei $D = \{E_1, \ldots E_{11}\}$ wie gegeben. Dann gilt für jede Teilmenge $D' = \{E'_1, \ldots, E'_m\}$ mit $2\leq m \leq 11$
\[
    \begin{array}{rcl}
        \Prob(\underset{i=1}{\overset{k}{\bigcap}} E'_i) &=& \Prob(\{ (x_1, \ldots, x_{11}) \, | \, \underset{i=1}{\overset{k}{\bigwedge}} k_i = l_i\}\\
            &\stackrel{*}{=}& \frac{7^{11-k}}{7^11}\\
            &=& 7^{-k}\\
            &=& \underset{i=1}{\overset{k}{\prod}} \Prob(E'_i)
    \end{array}
\]

(*) gilt, da alle $k_i$ verschieden sind. Wir zählen also wieder die Größe von $\bigcap E'_i$. Hier sind $k$ stellen fix und wir können noch
für $11-k$ Positionen eine Zahl aus $[7]$ auswählen.

Da dies für alle Teilmengen von $D$ gilt, ist $D$ unabhängig.\\
\mbox{}\hfill$\square$\\


Wir sehen also, dass wir zu einer $11$ elementigen Menge von Ereignissen kein weiteres hinzugeben können, ohne sie entweder abhängig zu machen oder
das selbe Ereignis nocheinmal hinzuzugeben und wir sehen, dass alle so gebildeten $11$ elementigen Mengen auch wirklich unabhängig sind.

Daher ist die maximale Anzahl von Ereignissen die man bilden kann $11$.\\
\mbox{}\hfill$\square$

\label{LastPage}
\end{document}
