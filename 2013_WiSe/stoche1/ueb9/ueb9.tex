\documentclass[11pt,a4paper,ngerman]{article}
\usepackage[bottom=2.5cm,top=2.5cm]{geometry}
\usepackage[ngerman]{babel}
\usepackage[utf8]{inputenc}
\usepackage[T1]{fontenc}
\usepackage{ae}
\usepackage{amssymb}
\usepackage{amsmath}
\usepackage{amsthm}
\usepackage{graphicx}
\usepackage{fancyhdr}
\usepackage{fancyref}
\usepackage{listings}
\usepackage{xcolor}
\usepackage{stmaryrd}
\usepackage{paralist}

\usepackage[pdftex, bookmarks=false, pdfstartview={FitH}, linkbordercolor=white]{hyperref}
\usepackage{fancyhdr}
\pagestyle{fancy}
\fancyhead[C]{Stochastik I}
\fancyhead[L]{Übung 9}
\fancyhead[R]{WiSe 2013/14}
\fancyfoot{}
\fancyfoot[L]{}
\fancyfoot[C]{\thepage \hspace{1px} of \pageref{LastPage}}
\renewcommand{\footrulewidth}{0.5pt}
\renewcommand{\headrulewidth}{0.5pt}
\newcommand{\set}[1]{ \{ #1 \}}
\newcommand{\Prob}{\mathbb{P}}
\setlength{\parindent}{0pt}
\setlength{\headheight}{0pt}

\newcommand{\N}{\mathbb{N}}
\newcommand{\Q}{\mathbb{Q}}
\newcommand{\R}{\mathbb{R}}
\newcommand{\bigO}{\mathcal{O}}
\newcommand{\Rarr}{\Rightarrow}
\newcommand{\rarr}{\rightarrow}
\newcommand{\Pot}{\mathcal{P}}
\newcommand{\abs}[1]{ |#1|}
\newcommand{\solved}{$\mbox{}$ \hfill $\square$}
\newcommand{\Epsilon}{\mathcal{E}}

\newcommand{\maxw}{$^\curlywedgedownarrow$}
\newcommand{\alex}{$^\dagger$}
\newcommand{\marcel}{$^\diamondsuit$}

\date{}
\title{Übung 9}
\author{Max Wisniewski\maxw, Alexander Steen\alex, Marcel Ehrhardt\marcel}


%%
%% Enviroments for proofs and lemmas
%%
\newtheorem{prop}{\bfseries Behauptung}
\newtheorem{lemma}{\bfseries Lemma}

\begin{document}

\lstset{language=Pascal, basicstyle=\ttfamily\fontsize{10pt}{10pt}\selectfont\upshape, commentstyle=\rmfamily\slshape, keywordstyle=\rmfamily\bfseries, breaklines=true, frame=single, xleftmargin=3mm, xrightmargin=3mm, tabsize=2, mathescape=true}

\renewcommand{\figurename}{Figure}

\maketitle
\thispagestyle{fancy}

\begin{center}
\textbf{Hinweis}: Die Angabe, wer welche Aufgabe in \LaTeX\ formuliert hat, ist als hochgestelltes Symbol der Aufgabenüberschrift zu entnehmen.
\end{center}


\subsection*{Aufgabe 1}

\subsection*{Aufgabe 2}
\subsection*{Aufgabe 3}
\newcommand{\D}{\mathcal{D}}
Sei $(\Omega, \Epsilon, \Prob)$ ein W-Raum und $A,B \in \Epsilon$ zwei unabh. Ereignisse. \\
Z.z.: $\mathcal{D} = \set{E \in \Epsilon \; | \; \set{A,B, E} \text{ ist unabhängig}}$ ist Dynkin-System. \\

\textbf{Beweis}: Sei  $A,B \in \Epsilon$ zwei unabhängige Ereignisse.
\begin{enumerate}[(i)]
\item $\emptyset \in \D, \Omega \in \D$: \\
Es ist $\emptyset \in \Epsilon$ und $\Omega \in \Epsilon$ (da $\Epsilon$ $\sigma$-Algebra). \\
Weiterhin ist $\Prob(A \cap B \cap \emptyset) = \Prob(\emptyset) = 0 = \Prob(A) \Prob(B) \underbrace{\Prob(\emptyset)}_{=0}$ und \\
$\Prob(A \cap B \cap \Omega) = \Prob(A \cap B) = \Prob(A) \Prob(B) =  \Prob(A) \Prob(B) \underbrace{\Prob(\Omega)}_{=1}$. \\
Also ist $\emptyset \in \D$ und $\Omega \in \D$.
\item Für alle $E \in \D$ gilt  $\Omega \setminus E \in \D$: \\
Sei $E \in \D$. Also gilt $\Prob(A \cap B \cap E) = \Prob(A) \Prob(B) \Prob(E)$. Dann gilt aber
\begin{equation*}\begin{split}
 \Prob(A \cap B \cap E^c) &=  \Prob((A \cap B) \cap (\Omega \setminus E)) \\
&=  \Prob((A \cap B \cap \Omega) \setminus (A \cap B \cap E)) \\
&= \Prob((A \cap B) \setminus (A \cap B \cap E)) \\
&=  \Prob(A)\Prob(B) - \Prob(A)\Prob(B)\Prob(E) \\
&= \Prob(A)\Prob(B) \cdot (1 - \Prob(E)) \\
&= \Prob(A)\Prob(B) \Prob(E^c) \\
\end{split}\end{equation*}
und damit $E^c \in \D$.
\item Sei $\left( E_i \right)_{i \in \mathbb{N}}$ eine Folge von paarweise disjunkten Ereignissen mit $E_i \in \D$. \\
Z.z. $\bigcup_i E_i \in \D$. \\
Es ist
\begin{equation*}\begin{split}
 \Prob(A \cap B \cap \bigcup_i E_i) &=  \Prob((A \cap B) \cap \bigcup_i E_i) \\
&= \Prob(\bigcup_i A \cap B \cap E_i) \\
&\stackrel{(*)}{=} \sum_i \Prob(A \cap B \cap E_i) \\
&\stackrel{Unabh.}{=} \sum_i \Prob(A) \Prob(B) \Prob(E_i) \\
&= \Prob(A)\Prob(B) \sum_i \Prob(E_i) \\
&\stackrel{(*)}{=} \Prob(A)\Prob(B) \Prob(\bigcup_i E_i) \\
\end{split}\end{equation*}
und damit $\bigcup_i E_i \in \D$.
(*) gilt, da für disjunkte Folgen $A_1, A_2, \ldots$ gilt: $\Prob(\bigcup_i A_i) = \sum_i \Prob(A_i)$.
\end{enumerate}
\mbox{} \hfill $\square$
\subsection*{Aufgabe 4}


\label{LastPage}
\end{document}
