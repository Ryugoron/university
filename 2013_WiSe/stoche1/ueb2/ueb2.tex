\documentclass[11pt,a4paper,ngerman]{article}
\usepackage[bottom=2.5cm,top=2.5cm]{geometry} 
\usepackage[ngerman]{babel}
\usepackage[utf8]{inputenc} 
\usepackage[T1]{fontenc} 
\usepackage{ae} 
\usepackage{amssymb} 
\usepackage{amsmath}
\usepackage{amsthm} 
\usepackage{graphicx}
\usepackage{fancyhdr}
\usepackage{fancyref}
\usepackage{listings}
\usepackage{xcolor}
\usepackage{stmaryrd}
\usepackage{paralist}

\usepackage[pdftex, bookmarks=false, pdfstartview={FitH}, linkbordercolor=white]{hyperref}
\usepackage{fancyhdr}
\pagestyle{fancy}
\fancyhead[C]{Stochastik I}
\fancyhead[L]{Übung 2}
\fancyhead[R]{WiSe 2013/14}
\fancyfoot{}
\fancyfoot[L]{}
\fancyfoot[C]{\thepage \hspace{1px} of \pageref{LastPage}}
\renewcommand{\footrulewidth}{0.5pt}
\renewcommand{\headrulewidth}{0.5pt}
\newcommand{\set}[1]{ \{ #1 \}}
\newcommand{\Prob}{\mathbb{P}}
\setlength{\parindent}{0pt} 
\setlength{\headheight}{0pt}

\date{}
\title{Übung 2}
\author{Max Wisniewski$^\curlywedgedownarrow$, Alexander Steen$^\dagger$}


%%
%% Enviroments for proofs and lemmas
%%
\newtheorem{prop}{\bfseries Behauptung}
\newtheorem{lemma}{\bfseries Lemma}

\begin{document}

\lstset{language=Pascal, basicstyle=\ttfamily\fontsize{10pt}{10pt}\selectfont\upshape, commentstyle=\rmfamily\slshape, keywordstyle=\rmfamily\bfseries, breaklines=true, frame=single, xleftmargin=3mm, xrightmargin=3mm, tabsize=2, mathescape=true}

\renewcommand{\figurename}{Figure}

\maketitle
\thispagestyle{fancy}

\begin{center}
\textbf{Hinweis}: Die Angabe, wer welche Aufgabe in \LaTeX formuliert hat, ist als hochgestellter Symbol der Aufgabenüberschrift zu entnehmen.
\end{center}


\subsection*{Aufgabe 1$^\curlywedgedownarrow$}
Sei $\Omega$ eine Menge und $\xi$ eine $\sigma$-Algebra auf $\Omega$. Beweisen Sie

\begin{enumerate}[a)]
\item Zwei verschiedene Atome sind disjunkt. \\
	\textbf{Beweis}: \\
Annahme: Seien $P \neq P' \in \xi$ Atome mit $P \cap P' = Q \neq \emptyset$.\\
 Da $P, P' \in \xi$, gilt dann aber $(P^c \cup P'^c)^c = Q \in \xi$.
Damit ist $Q \in\xi$ und $Q \subseteq P,P'$, jedoch $\emptyset \neq Q \neq P,P' \Rightarrow P,P'$ keine Atome. Widerspruch!
 $\mbox{}$ \hfill $\square$

\item 
\begin{enumerate}[(1)]
\item Sei $\Omega$ höchstens abzählbar. Sei $\xi$ eine $\sigma$-Algebra auf $\Omega$. Dann ist die Menge $\mathcal{P}$ der Atome von $\xi$ mit
\begin{equation*}\mathcal{P} = \{ E \in \xi \setminus \emptyset \; | \; C \in \xi \land C \subseteq E \Rightarrow C = \emptyset \lor C = E  \} \end{equation*}
eine Partition von $\Omega$. Denn 

\begin{enumerate}[(1)]
\item $\forall P \in \mathcal{P}: P \neq \emptyset$: \\
Gilt nach Konstruktion.

\item $\forall P \neq P' \in \mathcal{P}: P \cap P' = \emptyset$: \\
Annahme: Seien $P \neq P' \in \mathcal{P}$ mit $P \cap P' = Q \neq \emptyset$.\\
 Da $P, P' \in \xi$, gilt dann aber $(P^c \cup P'^c)^c = Q \in \xi$.
Damit ist $Q \in\xi$ und $Q \subseteq P$, jedoch $\emptyset \neq Q \neq P$. Widerspruch zur Definition von $\mathcal{P}$.

\item $\Omega = \bigcup_{P \in \mathcal{P}} P $ \\
Annahme: Es ex. ein $e \in \Omega$ mit $e \notin  \bigcup_{P \in \mathcal{P}} P$. \\
Da aber $e \in \Omega$ ex. eine Menge $E \in \xi$, sodass $e  \in E$. Dann ex. ebenfalls eine inklusionsminimale Menge $Q \in \xi$, $Q \neq \emptyset$ mit $Q \subseteq E$.
Es ist also $Q \in \mathcal{P}$. Widerspruch zur Annahme. \\
\end{enumerate}
 $\mbox{}$ \hfill $\square$


\item \textbf{ICH GLAUB IN DER AUFGABE WIRD ETWAS STÄRKERES GEFORDERT}
Sei $\mathcal{P}$ eine Partition von $\Omega$. Dann ist die Menge
\begin{equation*}\xi := \{ \bigcup_{P \in \mathcal{P'}} P | \mathcal{P'} \subseteq \mathcal{P} \} \end{equation*}
eine $\sigma$-Algebra auf $\Omega$. Denn

\begin{enumerate}[(1)]
\item $\emptyset, \Omega \in \xi$ \\
Wähle $\mathcal{P'} := \emptyset$ (bzw. $\mathcal{P'} := \mathcal{P}$), dann ist $\bigcup_{P \in \emptyset} P = \emptyset \in \xi$ (bzw. $\bigcup_{P \in \mathcal{P}} P = \Omega \in \xi$).

\item $E \in \xi \Rightarrow E^c \in \xi$ \\
Sei $E \in \xi$. Dann ex. ein $\mathcal{P'} \subseteq \mathcal{P}$ mit $\bigcup_{P \in \mathcal{P'}} P = E$. Wähle $\overline{\mathcal{P}} := \mathcal{P} \setminus \mathcal{P'}$.\\
Dann ist $\bigcup_{P \in \overline{\mathcal{P}}} P = \Omega \setminus E = E^c \in \xi$.

\item $\left( E_\ell \right)_{\ell \in L} \subseteq \xi \Rightarrow \bigcup_{\ell \in L} E_\ell \in \xi$ \\
Gilt nach Konstruktion.
\end{enumerate}
 $\mbox{}$ \hfill $\square$
\end{enumerate}
\end{enumerate}



\subsection*{Aufgabe 2$^\dagger$}
\begin{enumerate}[a)]
\item Sei $\Omega$ die Menge aller abzählbar unendlichen Folgen von Würfen mit einem Würfel. Es ist $A_k := \set{\text{6 im $k$-ten Wurf}}$. Beschrieben
Sie mit Worten das Ereignis $A = \cap_{n=1}^\infty \cup_{k=n}^\infty A_k$.

\item Es sei $\Omega$ eine Menge, $\xi, \xi_1, \xi_2,\ldots$ $\sigma$-Algebren auf $\Omega$ mit $\xi_1 \subseteq \xi_2 \subseteq \ldots \subseteq \xi$.
Z.z. $\cup_{n=1}^\infty \xi_n$ ist nicht notwendigerweise $\sigma$-Algebra.
\end{enumerate}




\subsection*{Aufgabe 3$^\curlywedgedownarrow$}
Sei $\Prob$ ein Wahrscheinlichkeitsmaß auf $\mathbb{N}$. Zu zeigen: Für jedes $\varepsilon > 0 \; $ ex. eine Primzahl $p$ sodass $\Prob(\set{p,p+1,\ldots}) < \varepsilon$.

\textbf{Beweis}: \\



\subsection*{Aufgabe 4$^\dagger$}
Zu zeigen: $\sigma(\mathcal{M}) = \sigma(\mathcal{O})$, wobei $\mathcal{M} = \set{[a, +\infty) | a \in \mathbb{Q}}$.



\label{LastPage}
\end{document}
