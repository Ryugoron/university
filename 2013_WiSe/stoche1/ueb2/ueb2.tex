\documentclass[11pt,a4paper,ngerman]{article}
\usepackage[bottom=2.5cm,top=2.5cm]{geometry} 
\usepackage[ngerman]{babel}
\usepackage[utf8]{inputenc} 
\usepackage[T1]{fontenc} 
\usepackage{ae} 
\usepackage{amssymb} 
\usepackage{amsmath}
\usepackage{amsthm} 
\usepackage{graphicx}
\usepackage{fancyhdr}
\usepackage{fancyref}
\usepackage{listings}
\usepackage{xcolor}
\usepackage{stmaryrd}
\usepackage{paralist}

\usepackage[pdftex, bookmarks=false, pdfstartview={FitH}, linkbordercolor=white]{hyperref}
\usepackage{fancyhdr}
\pagestyle{fancy}
\fancyhead[C]{Stochastik I}
\fancyhead[L]{Übung 2}
\fancyhead[R]{WiSe 2013/14}
\fancyfoot{}
\fancyfoot[L]{}
\fancyfoot[C]{\thepage \hspace{1px} of \pageref{LastPage}}
\renewcommand{\footrulewidth}{0.5pt}
\renewcommand{\headrulewidth}{0.5pt}
\newcommand{\set}[1]{ \{ #1 \}}
\newcommand{\Prob}{\mathbb{P}}
\setlength{\parindent}{0pt} 
\setlength{\headheight}{0pt}

\date{}
\title{Übung 2}
\author{Max Wisniewski$^\curlywedgedownarrow$, Alexander Steen$^\dagger$}


%%
%% Enviroments for proofs and lemmas
%%
\newtheorem{prop}{\bfseries Behauptung}
\newtheorem{lemma}{\bfseries Lemma}

\begin{document}

\lstset{language=Pascal, basicstyle=\ttfamily\fontsize{10pt}{10pt}\selectfont\upshape, commentstyle=\rmfamily\slshape, keywordstyle=\rmfamily\bfseries, breaklines=true, frame=single, xleftmargin=3mm, xrightmargin=3mm, tabsize=2, mathescape=true}

\renewcommand{\figurename}{Figure}

\maketitle
\thispagestyle{fancy}

\begin{center}
\textbf{Hinweis}: Die Angabe, wer welche Aufgabe in \LaTeX formuliert hat, ist als hochgestellter Symbol der Aufgabenüberschrift zu entnehmen.
\end{center}


\subsection*{Aufgabe 1$^\curlywedgedownarrow$}
Sei $\Omega$ eine Menge und $\xi$ eine $\sigma$-Algebra auf $\Omega$. Beweisen Sie

\begin{enumerate}[a)]
\item Zwei verschiedene Atome sind disjunkt. \\
	\textbf{Beweis}: \\
Annahme: Seien $P \neq P' \in \xi$ Atome mit $P \cap P' = Q \neq \emptyset$.\\
 Da $P, P' \in \xi$, gilt dann aber $(P^c \cup P'^c)^c = Q \in \xi$.
Damit ist $Q \in\xi$ und $Q \subseteq P,P'$, jedoch $\emptyset \neq Q \neq P,P' \Rightarrow P,P'$ keine Atome. Widerspruch!
 $\mbox{}$ \hfill $\square$

\item Sei $\Omega$ höchstens abzählbar. Dann ist die Menge $\mathcal{P}$ der Atome von $\xi$ mit
\begin{equation*}\mathcal{P} = \{ E \in \xi \setminus \emptyset \; | \; C \in \xi \land C \subseteq E \Rightarrow C = \emptyset \lor C = E  \} \end{equation*}
\begin{enumerate}[(1)]



\item Z.z. $\mathcal{P}$ ist eine Partition von $\Omega$.

\begin{enumerate}[(1)]
\item $\forall P \in \mathcal{P}: P \neq \emptyset$: \\
Gilt nach Konstruktion.

\item $\forall P \neq P' \in \mathcal{P}: P \cap P' = \emptyset$: \\
Annahme: Seien $P \neq P' \in \mathcal{P}$ mit $P \cap P' = Q \neq \emptyset$.\\
 Da $P, P' \in \xi$, gilt dann aber $(P^c \cup P'^c)^c = Q \in \xi$.
Damit ist $Q \in\xi$ und $Q \subseteq P$, jedoch $\emptyset \neq Q \neq P$. Widerspruch zur Definition von $\mathcal{P}$.

\item $\Omega = \bigcup_{P \in \mathcal{P}} P $ \\
Annahme: Es ex. ein $e \in \Omega$ mit $e \notin  \bigcup_{P \in \mathcal{P}} P$. \\
Da aber $e \in \Omega$ ex. eine Menge $E \in \xi$, sodass $e  \in E$. Dann ex. ebenfalls eine inklusionsminimale Menge $Q \in \xi$, $Q \neq \emptyset$ mit $Q \subseteq E$.
Es ist also $Q \in \mathcal{P}$. Widerspruch zur Annahme. \\
\end{enumerate}
 $\mbox{}$ \hfill $\square$


\item Z.z. es gilt
\begin{equation*}\xi = \{ \bigcup_{P \in \mathcal{P'}} P | \mathcal{P'} \subseteq \mathcal{P} \} =: \xi' \end{equation*}

\begin{description}
\item[$\subseteq$:] Sei $E \in \xi$. \\
(1) $E$ ist Atom \\
Also ist $E \in \mathcal{P}$ und damit $\set{E} \subseteq \mathcal{P}$. Also $E \in \xi'$.

(2) $E$ ist kein Atom. Dann gibt es ein $P_1 \in \mathcal{P}$, s.d. $P_1 \cup (E \setminus P_1) = E$. Da $E \setminus P_1 \subsetneq E$ können wir nach Induktion $E \setminus P_1$ als Vereinigung von Atomen $P_2,P_3,\ldots$ darstellen. Es ist also $E = \bigcup_{i = 1,2,\ldots}P_i$ und damit $E \in \xi'$.
\item[$\supseteq$:] Nach Konstruktion. $\mbox{}$ \hfill $\square$
\end{description}

%\begin{enumerate}[(1)]
%\item $\emptyset, \Omega \in \xi$ \\
%Wähle $\mathcal{P'} := \emptyset$ (bzw. $\mathcal{P'} := \mathcal{P}$), dann ist $\bigcup_{P \in \emptyset} P = \emptyset \in \xi$ (bzw. $\bigcup_{P \in \mathcal{P}} P = \Omega \in \xi$).
%
%\item $E \in \xi \Rightarrow E^c \in \xi$ \\
%Sei $E \in \xi$. Dann ex. ein $\mathcal{P'} \subseteq \mathcal{P}$ mit $\bigcup_{P \in \mathcal{P'}} P = E$. Wähle $\overline{\mathcal{P}} := \mathcal{P} \setminus \mathcal{P'}$.\\
%Dann ist $\bigcup_{P \in \overline{\mathcal{P}}} P = \Omega \setminus E = E^c \in \xi$.
%
%\item $\left( E_\ell \right)_{\ell \in L} \subseteq \xi \Rightarrow \bigcup_{\ell \in L} E_\ell \in \xi$ \\
%Gilt nach Konstruktion.
%\end{enumerate}
% $\mbox{}$ \hfill $\square$
\end{enumerate}
\end{enumerate}



\subsection*{Aufgabe 2$^\dagger$}
\begin{enumerate}[a)]
\item Sei $\Omega$ die Menge aller abzählbar unendlichen Folgen von Würfen mit einem Würfel. Es ist $A_k := \set{\text{6 im $k$-ten Wurf}}$. Beschrieben
Sie mit Worten das Ereignis $A = \cap_{n=1}^\infty \cup_{k=n}^\infty A_k$.\\

\textbf{Lösung:}\\

Die innere Vereinigung nimmt alle Ereignisse, bei denen ab dem $k=n$ten
Schritt irgendwann einmal eine $6$ gewürfelt wird.

Schneiden wir nur bis zu einer Obergrenze $n'$ so würden wir auf jedenfall
die Ereigniss behaten, die in Wurf $n'$ eine $6$ haben. Nun schieben
wir das $n' \rightarrow \infty$. Schieben wir allerdings auch die $6$ einzig und
allein so weit hinaus, sehen wir sie niemals.

Die einzige möglichkeit um alle der inneren Mengen zu befriedigen, ist
es für jeden Wurf $k$ muss es irgendwann in endlicher Zukunft eine $6$ geben.
Dann ist dieses $k$ auf jedenfall für den Schnitt befriedigt. Sind wir bei $k'$ nach der $6$ muss es wieder in endlicher Zukunft eine $6$ geben, usw.

Somit lässt sich $A$ als die Menge aller Folge von Würfen beschreiben, bei denen
unendlich oft eine $6$ geworfen wird.

\item Es sei $\Omega$ eine Menge, $\xi, \xi_1, \xi_2,\ldots$ $\sigma$-Algebren auf $\Omega$ mit $\xi_1 \subseteq \xi_2 \subseteq \ldots \subseteq \xi$.
Z.z. $\cup_{n=1}^\infty \xi_n$ ist nicht notwendigerweise $\sigma$-Algebra.\\

\textbf{Beweis:}\\

Sei $\xi_n = \sigma \{ \{1\} , ... , \{n\}\}$ und $\xi = \subset{n\rightarrow \infty}{\lim} \xi_n$.
Dann ist $\xi_1 \subseteq \xi_2 \subseteq ... \subseteq \xi$.

Nun gilt, dass wir mit $\underset{n\rightarrow\infty}{\lim}\cup_{k=1}^n \xi_k$
niemals die Menge $\{2k \, | \, k \in mathbb{N}\}$ erzeugen können.
Da wir in jedem $\xi_n$ nur die endlichen Vereinigungen der ersten $n$ Zahlen
zur Verfügung haben und die unendlichen kompletemente $\mathbb{N} \ \{x\}$ für alle $x \leq n$, kann kein $\xi_k$ jemals die Menge der geraden Zahlen enthalten.

Da aber $\cup_{n=1}^\infty \xi_n$ alle Atome $\{k\}$ für $k\in \mathbb{N}$ 
enthält, müsste auch die unendliche Vereinigung $\{2k \, | \, k \in \mathbb{N}\}$ in der $\sigma$-Algebra liegen.

Demnach kann $\cup_{n=1}^\infty \xi_n$ keine $\sigma$-Algebra sein.
 $\mbox{}$ \hfill $\square$

\end{enumerate}



\subsection*{Aufgabe 3$^\dagger$}
Sei $\Prob$ ein Wahrscheinlichkeitsmaß auf $\mathbb{N}$. Zu zeigen: Für jedes $\varepsilon > 0 \; $ ex. eine Primzahl $p$ sodass $\Prob(\set{p,p+1,\ldots}) < \varepsilon$.

\textbf{Beweis}: \\
Es bezeichne $p_i$ die $i$-te Primzahl (in natürlicher Ordnung). Sei $E_i := \mathbb{N} \setminus \set{1,2,\ldots,p_i}$. \\
Sei weiterhin $\varepsilon > 0$. Es gilt offensichtlich $E_1 \supset E_2 \supset E_3 \supset \ldots$ und es gilt
$\lim_{i \to \infty} \Prob(E_i) = 0$. Dann ex. nach Stetigkeit von $\Prob$ (und weil es unendlich viele Primzahlen gibt) ein $i$, sodass $\forall j \geq i: \Prob(E_j) < \varepsilon$.
Dann ist $\Prob(\set{p_i,p_i+1,\ldots}) <\varepsilon$. $\mbox{}$ \hfill $\square$



\subsection*{Aufgabe 4$^\curlywedgedownarrow$}
Zu zeigen: $\sigma(\mathcal{M}) = \sigma(\mathcal{O})$, wobei $\mathcal{M} = \set{[a, +\infty) | a \in \mathbb{Q}}$.

\textbf{Beweis:}\\

$\subseteq$:\\

Sei $m = [ a, \infty) \in \mathcal{M}$. z.z. $m \in \sigma (\mathcal{O})$,
da, falls alle Atome von $\mathcal{M}$ enthalten sind auch zwangsläufig die
komplette $\sigma$-Algebra enthalten ist, da die $\sigma$-Algebra und vereinigung und komplement abgeschlossen ist.


Sei $(a_i, \infty)$ mit $\lim_{n\rightarrow \infty} a_i = a$ und die $a_i$ fallen streng monoton gegen $a$ eine
Folge von offenen Teilmengen in $\mathbb{R}$.\\
Dann gilt $(a_k, \infty) \subset (a_{k+1}, \infty)$ für alle $k$.
Demnach ist $\cup_{n=1}^\infty (a_n, \infty) = [a, \infty)$, nach üblichen
Grenzwert regeln, da $a$ im Grenzwert enthalten ist und so die Menge in dieser
Richtung abgeschlossen ist.

$\supseteq$:\\

Wir wissen, dass $\mathcal{O}$ von der unendlichen Vereinigung von 
offenen Intervallen $(a,b)$ gebildet wird. Wir untersuchen also wiederum, ob
$(a,b)$ in $\sigma(\mathcal{M})$ liegt und nach Abschlusseigenschaften muss
somit die gesammte $\sigma$-Algebra enthalten sein.\\

Sei $(a,b)$ ein offenes Intervall in $\mathbb{R}$. Zunächst sehen wir,
dass wir aus $[x, \infty)$ mit $x \in \mathbb{Q}$ durch Komplementbildung in $\sigma$ $(\infty, x)$ erhalten können.
Wir suchen uns also zwei Intervalle $(-\infty, b)$ und $(a, \infty)$ mit $a,b \in \mathbb{R}$ und scheiden diese, so erhalten wir $(a,b)$. Beide vorgehen sind
analog, wir zeigen also nur eins der beiden.


Sei $[x_n, \infty)$ eine Folge von Intervallen mit $(x_n)_{n \in \mathbb{N}}$
ist streng monoton fallend und $\lim_{n\rightarrow\infty} x_n = b$.

Dann ist $\cup_{n=1}^\infty [x_n, \infty) = [b, \infty)$. Das Intervall wird
offen und wir wissen, dass es für jede Zahl $a \in \mathbb{R}$ eine Cauchy-Folge in $\mathbb{Q}$ gibt, so dass der Grenzwert der Folge genau $b$ ist.
Damit ist $[b, \infty)^c = (-\infty, b)$ und somit das eine Interval, das wir erhalten wollten.

\mbox{}\hfill$\square$

\label{LastPage}
\end{document}
