\documentclass[11pt,a4paper,ngerman]{article}
\usepackage[bottom=2.5cm,top=2.5cm]{geometry}
\usepackage[ngerman]{babel}
\usepackage[utf8]{inputenc}
\usepackage[T1]{fontenc}
\usepackage{ae}
\usepackage{amssymb}
\usepackage{amsmath}
\usepackage{amsthm}
\usepackage{graphicx}
\usepackage{fancyhdr}
\usepackage{fancyref}
\usepackage{listings}
\usepackage{xcolor}
\usepackage{stmaryrd}
\usepackage{paralist}

\usepackage[pdftex, bookmarks=false, pdfstartview={FitH}, linkbordercolor=white]{hyperref}
\usepackage{fancyhdr}
\pagestyle{fancy}
\fancyhead[C]{Stochastik I}
\fancyhead[L]{Übung 4}
\fancyhead[R]{WiSe 2013/14}
\fancyfoot{}
\fancyfoot[L]{}
\fancyfoot[C]{\thepage \hspace{1px} of \pageref{LastPage}}
\renewcommand{\footrulewidth}{0.5pt}
\renewcommand{\headrulewidth}{0.5pt}
\newcommand{\set}[1]{ \{ #1 \}}
\newcommand{\Prob}{\mathbb{P}}
\setlength{\parindent}{0pt}
\setlength{\headheight}{0pt}

\newcommand{\N}{\mathbb{N}}
\newcommand{\Q}{\mathbb{Q}}
\newcommand{\R}{\mathbb{R}}
\newcommand{\bigO}{\mathcal{O}}
\newcommand{\Rarr}{\Rightarrow}
\newcommand{\Pot}{\mathcal{P}}
\newcommand{\abs}[1]{ |#1|}
\newcommand{\solved}{$\mbox{}$ \hfill $\square$}

\newcommand{\maxw}{$^\curlywedgedownarrow$}
\newcommand{\alex}{$^\dagger$}
\newcommand{\marcel}{$^\diamondsuit$}

\date{}
\title{Übung 4}
\author{Max Wisniewski\maxw, Alexander Steen\alex, Marcel Ehrhardt\marcel}


%%
%% Enviroments for proofs and lemmas
%%
\newtheorem{prop}{\bfseries Behauptung}
\newtheorem{lemma}{\bfseries Lemma}

\begin{document}

\lstset{language=Pascal, basicstyle=\ttfamily\fontsize{10pt}{10pt}\selectfont\upshape, commentstyle=\rmfamily\slshape, keywordstyle=\rmfamily\bfseries, breaklines=true, frame=single, xleftmargin=3mm, xrightmargin=3mm, tabsize=2, mathescape=true}

\renewcommand{\figurename}{Figure}

\maketitle
\thispagestyle{fancy}

\begin{center}
\textbf{Hinweis}: Die Angabe, wer welche Aufgabe in \LaTeX\ formuliert hat, ist als hochgestelltes Symbol der Aufgabenüberschrift zu entnehmen.
\end{center}


\subsection*{Aufgabe 1}
Es ist $\Prob(\text{6 wird gewürfelt}) = \frac{1}{3}$ und $\Prob(\text{$i$ wird gewürfelt}) = \frac{2}{15}$ für $i = 1,2,3,4,5$. Es sei $\Omega = \set{1,2,3,4,5,6}^2$ die Menge der Ereignisse. Gesucht sind die Wahrscheinlichkeiten für

\begin{enumerate}[i)]
\item 6 erscheint genau einmal \\
Die hierzu gehörenden Elementarereignisse sind $(6,i)$, $(i, 6)$, für $i = 1,\ldots,5$.\\
Es gilt $\Prob(\set{(6,i)}) = \Prob(\set{(i,6)}) = \frac{2}{15} \frac{1}{3} = \frac{2}{45}$, für $ i = 1,\ldots,5$.
Dann ist 
\begin{equation*}\begin{split}
\Prob(\text{genau einmal 6}) &= \Prob \left( \bigcup_{i=1}^5 \set{(6,i)} \cup \bigcup_{i=1}^5 \set{(i,6)}\right)\\
&= \sum_{i=1}^5 \frac{2}{45} + \sum_{i=1}^5 \frac{2}{45} \\
&= \frac{20}{45}
\end{split}\end{equation*}
\item beide Zahlen sind gerade
Die hierzu gehörenden Elementarereignisse sind\\
$(2,2), (2,4), (2, 6), (4,2), (4,4), (4,6), (6,2), (6,4), (6,6)$. 
Es ist \\ $\Prob(\set{x}) = \frac{4}{225}$, für $x \in \set{(2,2), (2,4), (4,2), (4,4)}$, \\
$\Prob(\set{x}) = \frac{2}{45}$, für $x \in \set{(2,6), (4,6), (6,2), (6,4)}$, und \\
$\Prob(\set{6,6}) = \frac{1}{9}$ und es gilt
\begin{equation*}\begin{split}
\Prob(\text{beide Zahlen gerade}) &= 4 \Prob(\set{(2,2)}) + 4 \Prob(\set{(4,6)}) + \Prob( \set{(6,6)}) \\
&= 4 \frac{4}{225} + 4\frac{2}{45} + \frac{1}{9} \\
&= \frac{9}{25}
\end{split}\end{equation*}

\item Summe ist 7
Die hierzu gehörenden Elementarereignisse sind\\
$(1,6), (2, 5), (3, 4), (4, 3), (5, 2), (6, 1)$.
Es ist \\ $\Prob(\set{x}) = \frac{4}{225}$, für $x \in \set{(2,5), (3,4), (4,3), (5,2)}$, und \\
$\Prob(\set{x}) = \frac{2}{45}$, für $x \in \set{(1,6), (6,1)}$ und es gilt
\begin{equation*}\begin{split}
\Prob(\text{Summe ist 7}) &= 4 \Prob(\set{(2,5)}) + 2 \Prob(\set{(1,6)}) \\
&= 4 \frac{4}{225} + 2\frac{2}{45}\\
&= \frac{4}{25}
\end{split}\end{equation*}

\end{enumerate}
\subsection*{Aufgabe 2}
\subsection*{Aufgabe 3}
Ansatz: Da wir vier Einschränkungen in die Dichtefunktion unterbringen müssen (drei Bedingungen in der Aufgabe, eine Bedingung für die Integralsumme gleich 1), haben
wir es mit einem Polynom $p$ dritten Grades (vier Freiheitsgrade) versucht. Es ist also $p(x) = ax^3 + bx^2 + cx +d$ der Ansatz der Dichtefunktion.

Die Einschränkungen sind (1) $\int_0^2 p(x) dx = 0.6$, (2) $\int_1^4 p(x) dx = 0.5$, (3) $\int_3^5 p(x) = 0.2$, (4) $\int_0^6 p(x) dx = 1$.

Das Gleichungssystem ergibt sich wegen
\begin{equation}
\int_0^2 p(x) dx = 4 a + \frac{8b}{3} + 2 (c + d)
\end{equation}
\begin{equation}
\int_1^4 p(x) dx =\frac{3 (85 a + 28 b + 10 c + 4 d)}{4}
\end{equation}
\begin{equation}
\int_3^5 p(x) dx =136 a + \frac{98 b}{3} + 8 c + 2 d
\end{equation}
\begin{equation}
\int_0^6 p(x) dx =6 (54 a + 12 b + 3 c + d)
\end{equation}
zu
\begin{equation*}
\left\{
\begin{array}{rcl}
 4 a + \frac{8b}{3} + 2 (c + d) & = & 0.6 \\
\frac{3 (85 a + 28 b + 10 c + 4 d)}{4} & = & 0.5 \\
136 a + \frac{98 b}{3} + 8 c + 2 d &= & 0.2 \\
6 (54 a + 12 b + 3 c + d) &=& 1
\end{array}
\right.
\end{equation*}

Nach einigem Rechnen (z.B. Gauß-Eliminierungsverfahren) kommt man auf die Lösung\\
\begin{equation*}\begin{split}
a = -\frac{2}{495} & \qquad b = \frac{8}{165} \\
c = -\frac{109}{495} & \qquad d = \frac{51}{110}
\end{split}\end{equation*}

und damit auf die Dichtefunktion 
\begin{equation*}
f: [0,6] \to \mathbb{R}, x \mapsto  -\frac{2}{495} x^3 +  \frac{8}{165} x^2 - \frac{109}{495}x +  \frac{51}{110}
\end{equation*}
Da es sich um ein Polynom handelt, ist $f$ stetig.
\subsection*{Aufgabe 4}

\begin{enumerate}[i)]
\item Wie groß ist $\Prob([-0.67,0.38])$ unter $N(0,1)$? \\
Der Tabelle entnehmen wir $\Prob(]\infty, 0.38]) =0.6480 $ und $\Prob(]-\infty,-0.67]) =  0.2514$.
Dann ist  $\Prob([-0.67,0.38]) = 0.6480 -  0.2514 = 0.3966$.
\item Bestimme $a$, sodass $\Prob([-0.67, a]) = 0.5$ unter $N(0,1)$.\\
Es ist   $\Prob(]-\infty,-0.67]) =  0.2514$ (Tabelle). Für $a$ muss gelten: \\
$\Prob(]-\infty, a]) - \Prob(]-\infty,-0.67])  = \Prob(]-\infty, a]) -  0.2514=  0.5$, also $\Prob(]-\infty, a]) = 0.2486$.

Durch suchen in der Tablle erhalten wir: $a \approx -0.68$, denn $\Prob(]-\infty,-0.68]) =  0.2483$.

\item Wie groß ist $\sigma$, wenn man weiß, dass $\Prob([-2,2]) = 0.82$ unter $N(0,\sigma^2)$?\\
Es gilt: $\Prob_{N(a,\sigma^2)}([c,d]) = \Prob_{N(0,1)}([(c-a)/\sigma, (d-a)/ \sigma])$  (siehe Buch ''Elementare Stochastik, Behrends, Seite 53'').

Es ist also hier ein $\sigma$ für $\Prob_{N(0,1)}([-2/\sigma, 2/ \sigma]) = 0.82$ gesucht. \\
Da  $\Prob_{N(0,1)}([-2/\sigma, 2/ \sigma]) =   \Prob(]-\infty, 2/ \sigma]) -  \Prob(]-\infty, - 2/ \sigma])$ suchen wir nun symmetrisch vom Nullpunkt in der Tabelle,
bis die Differenz von rechter und linker Seite 0.82 entspricht. Nach einigem Suchen findet man  $\Prob(]-\infty, 1.34]) -  \Prob(]-\infty, -1.34]) = 0.8194 \approx 0.82$.
Dann ergibt sich für $\sigma$ ein Wert von $\sigma = 0.5 \cdot 1.34 = 0.67$.
\end{enumerate}


\label{LastPage}
\end{document}
