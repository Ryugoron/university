\documentclass[11pt,a4paper,ngerman]{article}
\usepackage[bottom=2.5cm,top=2.5cm]{geometry}
\usepackage[ngerman]{babel}
\usepackage[utf8]{inputenc}
\usepackage[T1]{fontenc}
\usepackage{ae}
\usepackage{amssymb}
\usepackage{amsmath}
\usepackage{amsthm}
\usepackage{graphicx}
\usepackage{fancyhdr}
\usepackage{fancyref}
\usepackage{listings}
\usepackage{xcolor}
\usepackage{stmaryrd}
\usepackage{paralist}

\usepackage[pdftex, bookmarks=false, pdfstartview={FitH}, linkbordercolor=white]{hyperref}
\usepackage{fancyhdr}
\pagestyle{fancy}
\fancyhead[C]{Stochastik I}
\fancyhead[L]{Übung 6}
\fancyhead[R]{WiSe 2013/14}
\fancyfoot{}
\fancyfoot[L]{}
\fancyfoot[C]{\thepage \hspace{1px} of \pageref{LastPage}}
\renewcommand{\footrulewidth}{0.5pt}
\renewcommand{\headrulewidth}{0.5pt}
\newcommand{\set}[1]{ \{ #1 \}}
\newcommand{\Prob}{\mathbb{P}}
\setlength{\parindent}{0pt}
\setlength{\headheight}{0pt}

\newcommand{\N}{\mathbb{N}}
\newcommand{\Q}{\mathbb{Q}}
\newcommand{\R}{\mathbb{R}}
\newcommand{\bigO}{\mathcal{O}}
\newcommand{\Rarr}{\Rightarrow}
\newcommand{\rarr}{\rightarrow}
\newcommand{\Pot}{\mathcal{P}}
\newcommand{\abs}[1]{ |#1|}
\newcommand{\solved}{$\mbox{}$ \hfill $\square$}
\newcommand{\Epsilon}{\mathcal{E}}

\newcommand{\maxw}{$^\curlywedgedownarrow$}
\newcommand{\alex}{$^\dagger$}
\newcommand{\marcel}{$^\diamondsuit$}

\date{}
\title{Übung 5}
\author{Max Wisniewski\maxw, Alexander Steen\alex, Marcel Ehrhardt\marcel}


%%
%% Enviroments for proofs and lemmas
%%
\newtheorem{prop}{\bfseries Behauptung}
\newtheorem{lemma}{\bfseries Lemma}

\begin{document}

\lstset{language=Pascal, basicstyle=\ttfamily\fontsize{10pt}{10pt}\selectfont\upshape, commentstyle=\rmfamily\slshape, keywordstyle=\rmfamily\bfseries, breaklines=true, frame=single, xleftmargin=3mm, xrightmargin=3mm, tabsize=2, mathescape=true}

\renewcommand{\figurename}{Figure}

\maketitle
\thispagestyle{fancy}

\begin{center}
\textbf{Hinweis}: Die Angabe, wer welche Aufgabe in \LaTeX\ formuliert hat, ist als hochgestelltes Symbol der Aufgabenüberschrift zu entnehmen.
\end{center}


\subsection*{Aufgabe 1}

Die Punktmenge der 2-Sphäre $S = \{ x \in \mathbb{R}^3 \, : \, \|x\| = 1\}$ ist zu $90\%$ rot und zu $10\%$ blau gefärbt. Beweisen Sie,
dass es einen in die Kugeloberfläche einbeschriebenen Würfel gibt mit nur roten Ecken.\\

\textbf{Beweis:}\\

tbd

\subsection*{Aufgabe 2}

\subsubsection*{a)}

Berechnen Sie den Erwartungswert der Exponentialverteilung mit Parameter $\lambda > 0$.\\

\textbf{Lösung:}\\

Die Dichtefunktion der Exponentialverteilung ist
\[
    f_\lambda (x) = \left\{ \begin{array}{rl} \lambda e^{-\lambda x} &, x\geq 0 \\ 0 &, x < 0 \end{array} \right.
\]

Der Erwartungswert ist nun
$$\begin{array}{rcl}
    \int_{-\infty}^\infty f_\lambda (x) &=& \int^\infty_0 x \cdot f_\lambda (x)\\
                &=& \int^\infty_0 \lambda x \cdot e^{-\lambda x} dx\\
                &=& \underset{a \rightarrow \infty}{\lim} \left[ - \frac{1}{\lambda} \cdot (\lambda x  + 1) e^{-\lambda x}\right]_0^a\\
                &=& \frac{1}{\lambda} (\lambda 0 + 1) e^{-\lambda 0} = \frac{1}{\lambda}
\end{array}$$

Wobei bei man das integral durch einmal Partielles Integrieren berechnen kann. Damit ist
$$E(X) = \frac{1}{\lambda}$$


\subsubsection*{b)}

Berechnen Sie den Erwartungswert der Normalverteilung $\mathcal{N}(a, \sigma^2)$.\\

\textbf{Lösung:}\\

Im Gegensatz zur Normalverteilung $\mathcal{N}(0,1)$ ist $\mathcal{N}(a,\sigma)$ achsensymmetrisch zur Achse $x = a$. Dies machen wir uns zu nutzen,
indem wir die Funktion geeignet verschieben. Betrachten wir $E(X)$ so können wir $X$ unter der standard Normalverteilung darstellen als
$X = X_{st} + a$. Damit ergibt sich
$$\begin{array}{rcl}
    E(X)    &=& E(X_{st} + a)\\
            &=& E(X) + a\\
            &=& a
\end{array}$$

Dies gilt, da $f_{(0,\sigma^2)}(x)$ symmetrisch ist und daher gilt
$$
    -\int_{-\infty}^0 x f(x) dx = \int_0^\infty x f(x)
$$
und so das gesammte Integral zu $0$ zerfällt.

\subsection*{Aufgabe 3}

Es sei $\Omega = [a,b]$, versehen mit der $\sigma$ Algebra der Borelmengen und einem beliebigen W-Maß $\mathbb{P}$. Weiterhin sei $X \ : \, \Omega \rightarrow \mathbb{R}$ eine
differenzierbare Zufallsvariable. Beweisen Sie, dass auch die Ableitung $X'$ eine Zufallsvariable ist.\\

\textbf{Beweis:}\\

tbd


\subsection*{Aufgabe 4}

Sei $\Omega = [1,27]$ mit der Dichte $f(x) = cx$ versehen. Eine Zufallsvariable $X \, : \, \Omega \rightarrow \mathbb{R}$ sei durch $X(x) := \sqrt[3]{x}$ definiert.
Bestimmen Sie den Wert von $c$, die Dichte von $\mathbb{P}_X$ und berechnen Sie damit $\mathbb{P}(\{X \in [1,2]\})$.\\

\textbf{Lösung}:\\

Damit es sich bei der Dichtefunktion um eine Dichtefunktion handelt, muss
\[
    \int_1^{27} f(x) = 1
\]
gelten.
Wir erhalten also
$$\begin{array}{crcl}
                & \int_1^{27} f(x) &=& 1\\
\Leftrightarrow & \left[ \frac{1}{2} cx^2\right]_1^{27} & = & 1\\
\Leftrightarrow & \frac{c*27^2 - c}{2} & = & 1\\
\Leftrightarrow & 364 * c &=& 1\\
\Leftrightarrow & c &=& \frac{1}{364}
\end{array}$$

Damit ist die Dichtefunktion also $f(x) = \frac{x}{364}$.

Berechnen wir nun $\mathbb{P}_X$. Dieser erhalten wir durch auflösen der Definition $\mathbb{P}_X([a,b]) = \mathbb{P}(X \in [a,b])$ mit $a,b \in [1,27]$.
$$\begin{array}{rlc}
    \mathbb{P}_X([a,b]) &=& \mathbb{P}(X \in [a,b])\\
                        &=& \mathbb{P}(\{x \, : \, X(x) \in [a,b]\})\\
                        &=& \mathbb{P}([\text{min}\{a^3,27\},\text{min}\{b^3,27\}])\\
\end{array}$$

    Nun folgt daraus
\[
    \mathbb{P}_X([a,b]) = \left\{ 
    \begin{array}{lr}
        0 &,\text{falls }a\geq3\\
        1 - \frac{1}{728} a^6 &,\text{falls }b\geq 3, a<3\\
        \frac{1}{728}\left( b^6 - a^6\right)&, \text{sonst}
    \end{array}
    \right.
\]

    Die Dichte ist die Ableitung der Verteilungsfunktions, die hier über $\Omega$ von $1$ bis $x$ geht.
\[
   \begin{array}{rcl}
        (\mathbb{P}([1,x]))'    &=& \left( \frac{1}{728} \left( x^6 - 1\right) \right)'\\
                                &=& \frac{3}{362} x^5
    \end{array}
\]
    Und zuletzt können wir $P_X([1,2])$ ausrechen.
\[
    \begin{array}{rcl}
        \mathbb{P}([1,2])   &=& \frac{1}{728} (2^6 - 1^6)\\
                            &=& \frac{1}{728} (63)\\
                            &=& \frac{9}{104}
    \end{array}
\]
\label{LastPage}
\end{document}
