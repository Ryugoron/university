\documentclass[11pt,a4paper,ngerman]{article}
\usepackage[bottom=2.5cm,top=2.5cm]{geometry}
\usepackage[ngerman]{babel}
\usepackage[utf8]{inputenc}
\usepackage[T1]{fontenc}
\usepackage{ae}
\usepackage{amssymb}
\usepackage{amsmath}
\usepackage{amsthm}
\usepackage{graphicx}
\usepackage{fancyhdr}
\usepackage{fancyref}
\usepackage{listings}
\usepackage{xcolor}
\usepackage{stmaryrd}
\usepackage{paralist}

\usepackage[pdftex, bookmarks=false, pdfstartview={FitH}, linkbordercolor=white]{hyperref}
\usepackage{fancyhdr}
\pagestyle{fancy}
\fancyhead[C]{Stochastik I}
\fancyhead[L]{Übung 12}
\fancyhead[R]{WiSe 2013/14}
\fancyfoot{}
\fancyfoot[L]{}
\fancyfoot[C]{\thepage \hspace{1px} of \pageref{LastPage}}
\renewcommand{\footrulewidth}{0.5pt}
\renewcommand{\headrulewidth}{0.5pt}
\newcommand{\set}[1]{ \{ #1 \}}
\newcommand{\Prob}{\mathbb{P}}
\setlength{\parindent}{0pt}
\setlength{\headheight}{0pt}

\newcommand{\N}{\mathbb{N}}
\newcommand{\Q}{\mathbb{Q}}
\newcommand{\R}{\mathbb{R}}
\newcommand{\bigO}{\mathcal{O}}
\newcommand{\Rarr}{\Rightarrow}
\newcommand{\rarr}{\rightarrow}
\newcommand{\Pot}{\mathcal{P}}
\newcommand{\abs}[1]{ |#1|}
\newcommand{\solved}{$\mbox{}$ \hfill $\square$}
\newcommand{\Epsilon}{\mathcal{E}}

\newcommand{\maxw}{$^\curlywedgedownarrow$}
\newcommand{\alex}{$^\dagger$}
\newcommand{\marcel}{$^\diamondsuit$}

\date{}
\title{Übung 12}
\author{Max Wisniewski\maxw, Alexander Steen\alex, Marcel Ehrhardt\marcel}


%%
%% Enviroments for proofs and lemmas
%%
\newtheorem{prop}{\bfseries Behauptung}
\newtheorem{lemma}{\bfseries Lemma}

\begin{document}

\lstset{language=Pascal, basicstyle=\ttfamily\fontsize{10pt}{10pt}\selectfont\upshape, commentstyle=\rmfamily\slshape, keywordstyle=\rmfamily\bfseries, breaklines=true, frame=single, xleftmargin=3mm, xrightmargin=3mm, tabsize=2, mathescape=true}

\renewcommand{\figurename}{Figure}

\maketitle
\thispagestyle{fancy}

\begin{center}
\textbf{Hinweis}: Die Angabe, wer welche Aufgabe in \LaTeX\ formuliert hat, ist als hochgestelltes Symbol der Aufgabenüberschrift zu entnehmen.
\end{center}


\subsection*{Aufgabe 1}

Jemand wirft solange, bis er das erste Mal "Augensumme 11" erhält.

\subsubsection*{a)}
Wie oft muss er im Mittel würfeln?\\

\textbf{Lösung:}\\

Es handelt sich um ein geometrisches Zufalls experiement.
Die Wahrscheinlichkeit bei einem mal werfen ist
$\Prob(W = 11) = \frac{1}{13}$, da es bei $36$ Kombinationen nur $2$ gibt,
bei denen die Augensumme $11$ ist.

Nun gilt bei geometrischen Verteilungen, dass der Mittelwert
$E[W] = \frac{1}{p}$ ist. In unserem Fall erwarten wir also $13$
durchgänge, bis wir eine $11$ bekommen.

\subsubsection*{b)}

Nun hat er schon 20 Mal gewürfelt, ohne dass die Augensumme einmal 11 gewesen ist. Mit welcher Wahrscheinlichkeit wird er auch in den nächsten beiden Versuchen keinen Erfolgt haben.\\

\textbf{Lösung:}\\

Eine geometrische Verteilung ist gedächnislos. D.h. die Anzahl der bisherigen
Misserfolge ist ohne belang für die Wahrscheinlichkeit. Also müssen wir
nur die Wahrscheinlichkeit für beide Hintereinander ausgeführten Versuche berechnen.

\[
    P(W_1 \not= 11 \land W_2 \not=11) = P(W_1 \not= 11) \cdot P(W_2 \not= 11) =
        (1-p)^2 = \frac{12^2}{13^2} = \frac{144}{169}
\]

\textbf{Kommt mir etwas zu leicht vor...}

\subsection*{Aufgabe 2}

Seien $X_1, X_2, ... : \Omega \rightarrow \mathbb{R}$ Zufallsvariablen auf dem Wahrscheinlichkeitsraum $\left( \Omega, \mathcal{E}, \Prob \right)$ wobei $\Omega$
höchstens abzählbar ist. Zeigen Sie, dass von $\lim_{\text{i.W.}} X_n = X$ folgt $\lim_{\text{f.s.}} X_n = X$.\\

\textbf{Beweis:}\\

Wir wissen, dass für die $X_n$ gilt:
\[
    \lim_{n \rightarrow \infty} \Prob(\left| X_n - X \right| > \varepsilon) = 0
\]
für alle $\varepsilon > 0$.

Nehmen wir nun auch an, dass
\[
    P(\left\{ \omega \in \Omega \, : \, \lim_{n \rightarrow \infty} X_n(\omega) = X(\omega) \right\}) < 1
\]

ist. Wir wissen also, dass eine Menge $N(\varepsilon) \subseteq \Omega$ von $\omega \in \Omega$ auf denen die Folge für ein festes $\varepsilon > 0$ 
von Zufallsvariablen nicht konvergiert, d.h. $\forall \omega \in N(\varepsilon) \forall n \in \mathbb{N} \, . \, X_n(\omega) - X > \varepsilon$.
und $\Prob(N(\varepsilon)) > 0$ also, gilt $|N(\varepsilon)| > 0$. 

Da $\Omega$ abzählbar ist kann $|X_n - X| > \varepsilon$ also nur $\Prob(N) > 0$ sein, und da $|N| > 0$ und abzählbar ist. Das ist aber ein Widerspruch zur annahme.

\subsection*{Aufgabe 3}

Ein Theater hat 200 Plätze und zwei Garderoben. Jeder Zuschauer gibt seinen Mantel an einer der beiden Garderoben ab, mit Wahrscheinlichkeit $\frac{1}{2}$, unabhängig
von den anderen. Der Herr Direktor will, dass in $99\%$ der Fälle jeder seinen Mantel an der gewünschten Garderobe abgeben kann. Wären 170 Haken dafür genug dafür?\\

\textbf{Lösung:}\\

Da beide Gerderoben die selbe Wahrscheinlichkeit haben, müssen wir nur einen Fall betrachten, wie viele Mäntel Wahrscheinlich an einer Garderobe hängen werden.\\
Wir haben hier ein einfaches Bernoulli Experiment, mit Gleichbleibenden Wahrscheinlichkeiten. Wir beschreiben die Zufallsvariable $X$ als, es haben so viele
Leute an der Garderobe abgegeben. Wir können daher $X$ darstellen als
\[
    X = \sum_{i=1}^{200} X_i
\]
wobei $X_i$ eine Indikatorvariable ist, die beschreibt ob der $i$te Gast an dieser Garderobe abgegeben hat.

Da wir nicht wissen mit welcher Genauigkeit die Abschätzung geschehen soll, arbeiten wir uns von unten nach oben vor.\\

\begin{description}
    \item[\bfseries Markov Ungleichung:] Der Erwartungswert ist $E[X] = 200 \cdot \frac{1}{2} = 100$, wie man es von einem Bernoulli Experiment kennt.
        Die Markov Ungleichung ergibt nun
        \[\begin{array}{rcl}
            P(X \geq 170) &=& \frac{E[X]}{170}\\ &=& \frac{100}{170}\\ &=& 0.588235.
        \end{array}\]
        Nach Markov würde es also nicht reichen.
    \item[\bfseries Tschebyscheff Ungleichung:] Die Varianz einer Bernoulli Verteilung ist $\text{Var}(X) = p*(1-p) = \frac{1}{4}$ in diesem Fall.
        Wir wissen aber, dass alle Mäntel auf die Gerderobe passen, wenn wir nicht mehr als $70$ vom Erwartungswert abweichen, also
        können wir nach Tschebyscheff Ungleichung bestimmen:
        \[\begin{array}{rcl}
            P(|X - 100| < 70) &\geq& 1 - \frac{\text{Var}^2(X)}{70^2}\\
                &=& 1 - \left(\frac{1}{4^2} \cdot \frac{1}{70^2}\right)\\
                &=& \frac{78399}{78400}\\
                &=& 0.999987
        \end{array}\]

        Damit hätten wir auch schon gezeigt, das wir die Wahrscheinlichkeit erreichen. Um aber noch eine schönere Abschätzung zu bekommen, schauen
        wir uns noch den nächsten Abschätzer an.
    \item[\bfseries Chernoff Schranke:] Wir befinden uns in der Situation eine Zufallsvariable zu haben, die aus der Summe von unabhängigen
        Indikatorvariablen mit je gleicher Wahrscheinlichkeit zu befinden. Damit können wir die folgende Schranke anwenden.
        \[
            \forall \delta > 0 \, : \, \Prob \left[\sum X_i \geq (1+\delta)\cdot p \cdot n \right] \leq e^{-\frac{\min{\delta, \delta^2}}{3}pn}
        \]
        Um die Schranke zu benutzen, müssen wir erst den relativen Faktor $\delta$ bestimmen. Wir wollen $X \geq 171$ untersuchen. 
        Also gilt $(1+ \delta) 100 = 171 \Rightarrow \delta = \frac{71}{100}$.

        Es folgt also für dieses $\delta$, dass
        \[
            P(X \geq 171) \leq e^{-\frac{\delta^2}{3}\cdot 100} = e^{-\frac{5041}{300}} \approx 5.0397 \cdot 10^{-8}
        \] 
        gilt. Wir sind also für $P(X \leq 170)$ mit einer Zahl recht nah an $1$ dabei. Wobei wir im gegensatz zu Tschebyscheff noch einmal 4 weiere Stellen
        mit $9$en fixieren.
\end{description}

\subsection*{Aufgabe 4}

Es seien $X, X_1, X_2, ... : \Omega \rightarrow \mathbb{R}$ Zufallsvariablen auf dem Wahrscheinlichkeitsraum $(\Omega, \mathcal{E}, \Prob)$,
so dass $\Prob(X_n \geq c) = 1$ für jedes $n$ und $\lim_{\text{i.V.}}X_n = X$. Zeigen Sie, dass auch $\Prob(X \geq c) = 1$.\\

\textbf{Beweis:}\\

blub.

\label{LastPage}
\end{document}
