\documentclass[11pt,a4paper,ngerman]{article}
\usepackage[bottom=2.5cm,top=2.5cm]{geometry}
\usepackage[ngerman]{babel}
\usepackage[utf8]{inputenc}
\usepackage[T1]{fontenc}
\usepackage{ae}
\usepackage{amssymb}
\usepackage{amsmath}
\usepackage{amsthm}
\usepackage{graphicx}
\usepackage{fancyhdr}
\usepackage{fancyref}
\usepackage{listings}
\usepackage{xcolor}
\usepackage{stmaryrd}
\usepackage{paralist}

\usepackage[pdftex, bookmarks=false, pdfstartview={FitH}, linkbordercolor=white]{hyperref}
\usepackage{fancyhdr}
\pagestyle{fancy}
\fancyhead[C]{Stochastik I}
\fancyhead[L]{Übung 7}
\fancyhead[R]{WiSe 2013/14}
\fancyfoot{}
\fancyfoot[L]{}
\fancyfoot[C]{\thepage \hspace{1px} of \pageref{LastPage}}
\renewcommand{\footrulewidth}{0.5pt}
\renewcommand{\headrulewidth}{0.5pt}
\newcommand{\set}[1]{ \{ #1 \}}
\newcommand{\Prob}{\mathbb{P}}
\setlength{\parindent}{0pt}
\setlength{\headheight}{0pt}

\newcommand{\N}{\mathbb{N}}
\newcommand{\Q}{\mathbb{Q}}
\newcommand{\R}{\mathbb{R}}
\newcommand{\bigO}{\mathcal{O}}
\newcommand{\Rarr}{\Rightarrow}
\newcommand{\rarr}{\rightarrow}
\newcommand{\Pot}{\mathcal{P}}
\newcommand{\abs}[1]{ |#1|}
\newcommand{\solved}{$\mbox{}$ \hfill $\square$}
\newcommand{\Epsilon}{\mathcal{E}}

\newcommand{\maxw}{$^\curlywedgedownarrow$}
\newcommand{\alex}{$^\dagger$}
\newcommand{\marcel}{$^\diamondsuit$}

\date{}
\title{Übung 7}
\author{Max Wisniewski\maxw, Alexander Steen\alex, Marcel Ehrhardt\marcel}


%%
%% Enviroments for proofs and lemmas
%%
\newtheorem{prop}{\bfseries Behauptung}
\newtheorem{lemma}{\bfseries Lemma}

\begin{document}

\lstset{language=Pascal, basicstyle=\ttfamily\fontsize{10pt}{10pt}\selectfont\upshape, commentstyle=\rmfamily\slshape, keywordstyle=\rmfamily\bfseries, breaklines=true, frame=single, xleftmargin=3mm, xrightmargin=3mm, tabsize=2, mathescape=true}

\renewcommand{\figurename}{Figure}

\maketitle
\thispagestyle{fancy}

\begin{center}
\textbf{Hinweis}: Die Angabe, wer welche Aufgabe in \LaTeX\ formuliert hat, ist als hochgestelltes Symbol der Aufgabenüberschrift zu entnehmen.
\end{center}


\subsection*{Aufgabe 1}

Ein Kartenspiel mit $n$ Karten enthält 2 Asse. Es wird gemischt und die Karten werden aufgedeckt. Sei $X$ die Anzahl der Karten
bis das erste Ass erscheint, $Y$ die Anzahl bis das zweite Ass erscheint. Beweisen Sie, dass $E(X) = \frac{n+1}{3}$ und $E(Y) = \frac{2}{3}(n+1)$.\\

\textbf{Lösung:}\\

\begin{description}

    \item[Erstes Ass:] Wir wollen die Wahrscheinlichkeit betrachten, dass das erste Ass an Position $i$ liegt.
        \[
            P(X=i) = 2 \cdot \begin{pmatrix} n - 2 \\ i - 1\end{pmatrix} \cdot (i-1)! \cdot (n-i)! \frac{1}{n!}
        \]
        gilt, da wir zunächst $i-1$ Elemente vor das erste Ass legen müssen und das können wir aus den $n-2$ nicht Assen tun.
        Danach können wir vor und hinter dem ersten Ass permutieren. Dies gibt uns alle Eregnisse für $X$. Zuletzt teilen wir
        durch alle Permutationen. Dies können wir für beide Asse machen, daher mal zwei.\\

        Für den Erwartungswert ergibt sich:
        \[\begin{array}{rcl}
            E[X] &=& \underset{i=1}{\overset{n}{\sum}} i \cdot P(X=i)\\
                &=& 2\cdot\underset{i=1}{\overset{n}{\sum}} i \cdot \begin{pmatrix} n - 2 \\ i - 1\end{pmatrix} \cdot (i-1)! \cdot (n-i)! \frac{1}{n!}\\
                &=& 2\cdot\underset{i=1}{\overset{n}{\sum}} \frac{i\cdot (n-i)}{(n-1)\cdot n}\\
                &=& 2\cdot \frac{n+1}{6}\\
                &=& \frac{n+1}{3}
        \end{array}\]
        Dabei können wir auch den Grenzfall $i=n$ betrachten, da wir eine Wahrscheinlichkeit von $0$ dafür haben.
    \item[Zweites Ass:]
        Wir können das Experiement genau umgekehrt betrachten. Wir zweite Element zunächst an einer Position $i$. Danach wählen wir
        für die hinteren $n-i$ Plätze aus $n-2$ nicht Assen die Karten und permutieren dann alles, wiederum für beide Asse.

        \[
            P(X=i) = 2 \cdot \begin{pmatrix} n - 2 \\ n-i \end{pmatrix} \cdot (i-1)! \cdot (n-i)! \frac{1}{n!}
        \]
        Dadurch ergibt sich der Erwartungsert:
        \[\begin{array}{rcl}
            E[X] &=& \underset{i=1}{\overset{n}{\sum}} i \cdot P(X=i)\\
                &=& \underset{i=1}{\overset{n}{\sum}} 2 \cdot i \cdot \begin{pmatrix} n - 2 \\ n - i \end{pmatrix} \cdot (i-1)! \cdot (n-i)! \frac{1}{n!}\\
                &=& 2 \cdot \underset{i=1}{\overset{n}{\sum}} \frac{ i \cdot (n-2)! (i-1)! (n-i)!} { (n-i)! (i - 2)!n!}\\
                &=& 2 \cdot \underset{i=1}{\overset{n}{\sum}} \frac{ i \cdot (i-1)} {n \cdot (n-1)}\\
                &=& \frac{2}{n \cdot (n-1)} \cdot \left( \underset{i=1}{\overset{n}{\sum}} i^2 - \underset{i=1}{\overset{n}{\sum}} i\right)\\
                &=& \frac{2}{n \cdot (n-1)} \cdot \left( \frac{n(n+1)(2 n+1)}{6} + \frac{n(n+1)}{2}\right)\\
                &=& \frac{2}{3} (n+1)
        \end{array}\]
        Ebenso ist hier der Grenzfall $i=1$ mit einer Wahrscheinlichkeit von $0$ betrachtet.
\end{description}

\subsection*{Aufgabe 2}

\subsubsection*{a)}

Berechnen Sie die Streuung der geometrischen Verteilung mit Parameter $q \in [0,1)$.\\

\textbf{Lösung:}\\

tbd

\subsubsection*{b)}

Bewesien Sie, dass $\sigma^2$ die Varianz der Normalverteilung $\mathcal{N}(a,\sigma^2)$ ist.\\

\textbf{Beweis:}\\

tbd

\subsection*{Aufgabe 3}

In einem Kasino wird nach folgenden Regeln mit drei Wurfeln gespielt. Ein Spieler
bekommt 1000 Euro fur drei Sechsen, 100 Euro für zwei Sechsen und 10 Euro für
eine Sechs. In allen anderen allen gibt es gar nichts. Welchen Mindesteinsatz wird
der Kasinobetreiber verlangen, wenn er nicht draufzahlen will?\\

\textbf{Lösung:}\\

Nehmen wir an, dass das Casino faire Würfel verwendet. Dann ist
$P( X = 3\times6) = \frac{1}{6^3}$, für einen 3er Treffer. $P( X = 2 \times 6) = \frac{6}{6^3} = \frac{1}{6^2}$ für
die 2 Treffer und $\frac{6^2}{6^3}$ werden auch benötigt. Der erwartet Velust ist demnach
\[
    E(Verl) = 1000 \cdot \frac{1}{6^3} + 100 \cdot \frac{1}{6^2} + 10 \cdot \frac{1}{6} = 9.07407407407
\]
Der erwartet Verlust für das Casino ist $9.07$ Dollar. Sollte das Casion also min $9.08$ Dollar verlangen, so
würde die Klausur gewinn machen. Alles darüber würde gewinn bedeuten.

\subsection*{Aufgabe 4}

Es sind dreizehn 1-Euro Münzen zwischen Wilhelm, Xavier, Yvette und Zora zufällig verteilt, so dass jede Verteilung gleich Wahrscheinlich ist.
Wie groß ist die Wahrscheinlichkeit, dass jemand kein Geld bekommt?\\

\textbf{Lösung:}\\

Um das Geld aufzuteilen betrachten wir ungeordnete Zahlpartitionen. D.h. wir wollen eine Zahl $n$ in $r$ Teile spalten, mit $n = n_0 + .. + n_{r-1}$.\\

Dieses Experiement stellen wir dar, indem wir $n$ Münzen hinstellen und uns noch $r-1$ Partitionsgegenstände hinzustellen. Danach wählen wir $r-1$ Position
aus diesen Elementen aus und alles, das zwischen 2 Gewählten Position liegt, sind die Münzen, des Spielers (die äußeren zählen zu je einem).

Dadurch erhalten wir $\left( \begin{array}{cc} n + r - 1 \\ r - 1\end{array}\right)$ als Anzahl von Zahlpartitionen.\\ In unserem Konkreten Fall ist dies
insgesammt $\begin{pmatrix} 16 \\ 3 \end{pmatrix} = 560$.\\

Nun betrachten wir gültige Ereignisse. Dazu müssen wir uns dem Inclusions-/Exclusions-Schema auf Mengen bedienen, da wir sonst Ereignisse doppelt Zählen.
Da die Spieler symmetrisch verlaufen, zählen wir
\[
    4 \cdot \text{min. 1 Spieler leer} - 6 \cdot \text{ min. 2 Spieler leer} + 6 \cdot \text{min . 3 Spieler leer} + 4 \cdot \text{min. 4 Spieler leer}
\]

Nun ist 1 Spieler leer eine Zahlpartion von $13$ auf $3$, 2 Spieler leer eine Zahlpartiont von $13$ auf $2$, und so weiter. Es ergibt sich
\[\begin{array}{ll}
    &3 \cdot \begin{pmatrix} 15 \\ 2\end{pmatrix} - 6 \cdot \begin{pmatrix} 14 \\ 1 \end{pmatrix} + 6 \cdot \begin{pmatrix} 13 \\ 0 \end{pmatrix} - 0\\
    = & 3 \cdot 105 - 6 \cdot 13 + 6 \cdot 1\\
    = & 315 - 78 + 6 = 243
\end{array}\]

Nun ist $P(X) = \frac{|X|}{|\Omega|} = \frac{243}{560} = 0.43392 \approx 43 \%$


\label{LastPage}
\end{document}
