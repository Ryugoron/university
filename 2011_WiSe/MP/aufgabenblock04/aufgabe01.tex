\subsection*{Aufgabe 4.1 : Taster}

\subsubsection*{A 4.1.1}
	Wir wollen die Interruptfähigkeit des Ports 1 nutzen. Der Effekt besteht darin, dass nicht ständig ein Polling des Zustands einer Portleitung erforderlich ist, 
	um den Zustand der Taster zu erkennen.\\

	Unsere Beispielanwendung besteht im Kern aus einer Endlosschleife, die eine wait() Funktion und eine Codezeile zum toggeln der LED (P4.2) beinhaltet.\\

	Beide Taster an Port1 (P1.0 und P1.1) sollen interruptfähig sein und auf eine LH-Flanke den Interrupt auslösen. Die notwendige Initialisierung der Register 
	des Ports P1 und die Freigabe des Interrupts muß vor der Endlosschleife erfolgen.\\

	In der ISR Interrupt-Service-Routine für den Port1 soll folgende Funktionalität integriert werden:
	\begin{itemize}
		\item für den Taster an P1.0:
			nach jedem 10-ten Tastendruck soll die LED (P4.0) getoggelt werden
        		\item für den Taster an P1.1:
			bei jedem Tastendruck soll die LED (P4.1) getoggelt werden
	\end{itemize}
	Im nächsten Schritt soll eine Lösung gefunden werden, die eine mögliche Interruptquelle ausschalten kann.\\

	Verändern Sie dazu die ISR in der Form:
	\begin{itemize}
		\item für den Taster an P1.0:\\
        			nach jedem 10-ten Tastendruck soll die LED (P4.0) getoggelt werden\\
        			nur bei leuchtender LED (P4.1) soll der Taster (P1.0) interruptfähig sein\\
		\item für den Taster an P1.1:\\
			bei jedem Tastendruck soll die LED (P4.1) getoggelt werden

	Testen und dokumentieren Sie ihre Lösung.\\

\textbf{Programm:}\\

\textbf{Erklärung:}\\

\textbf{Beobachtung:}