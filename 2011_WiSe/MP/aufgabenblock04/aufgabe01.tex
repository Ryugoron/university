\subsection*{Aufgabe 4.1 : Taster}

\subsubsection*{A 4.1.1}
	Wir wollen die Interruptfähigkeit des Ports 1 nutzen. Der Effekt besteht darin, dass nicht ständig ein Polling des Zustands einer Portleitung erforderlich ist, 
	um den Zustand der Taster zu erkennen.\\

	Unsere Beispielanwendung besteht im Kern aus einer Endlosschleife, die eine wait() Funktion und eine Codezeile zum toggeln der LED (P4.2) beinhaltet.\\

	Beide Taster an Port1 (P1.0 und P1.1) sollen interruptfähig sein und auf eine LH-Flanke den Interrupt auslösen. Die notwendige Initialisierung der Register 
	des Ports P1 und die Freigabe des Interrupts muß vor der Endlosschleife erfolgen.\\

	In der ISR Interrupt-Service-Routine für den Port1 soll folgende Funktionalität integriert werden:
	\begin{itemize}
		\item für den Taster an P1.0:
			nach jedem 10-ten Tastendruck soll die LED (P4.0) getoggelt werden
        		\item für den Taster an P1.1:
			bei jedem Tastendruck soll die LED (P4.1) getoggelt werden
	\end{itemize}
	Im nächsten Schritt soll eine Lösung gefunden werden, die eine mögliche Interruptquelle ausschalten kann.\\

	Verändern Sie dazu die ISR in der Form:
	\begin{itemize}
		\item für den Taster an P1.0:\\
        			nach jedem 10-ten Tastendruck soll die LED (P4.0) getoggelt werden\\
        			nur bei leuchtender LED (P4.1) soll der Taster (P1.0) interruptfähig sein\\
		\item für den Taster an P1.1:\\
			bei jedem Tastendruck soll die LED (P4.1) getoggelt werden

	Testen und dokumentieren Sie ihre Lösung.\\
	\end{itemize}

\textbf{Programm:}\\

\begin{lstlisting}
#pragma vector = PORT2_VECTOR
__interrupt void PORT1 (void) {
	if(P1IN & 0x01){
		dosmth1();
	}
	if(P1IN & 0x02){
		dosmth2();
	}
	P1IFG &= ~(0xFF);
}

char counter1 = 0;
char counter2 = 0;

void dosmth1(){
	P1IE &= ~(0x01);
	if(counter1%20 == 9){
		LED1ON;
	}
	if(counter1%20 == 19){
		LED1OFF;
	}
	counter1 = (counter1 + 1) %20;
	wait(10000);
	P1IE |= 0x01;
}

void dosmth2(){
	P1IE &= ~(0x02);
	if(counter2 == 1){
		P1IE &= ~(0x01);
		LED2OFF;
		counter2 = 0;
	}
	else{
		P1IE |= 0x01;
		LED2ON;
		counter2 = 1;
	}
	wait(10000);
	P1IC |= 0x02;
}
\end{lstlisting}

\textbf{Erklärung:}\\

In Zeile 1-2 haben wir eine neue ISR angelegt, die über den Compiler für den zweiten Port in die Interruptvektortabelle gelegt wird. Innerhalb dieser Testen wir zunächst, welcher Taster nun wirklich gedrückt wurde. Danach entscheiden wir, wie wir weiter verfahren. Am ende setzen wir die Interruptfalg auf 0 zurück, womit der Interrupt als erledigt gesetzt wird.\\

Wurde Taster 1 gedrückt, deaktivieren wir für diesen Taster zunächst die Interrupts. Ist der Counter nun im 10. Schritt, schalten wir die LED1 an und sind wir im 20. Schritt ist die LED1 aus. Danach aktivieren wir die Interrupts für diese Taste wieder.\\

Haben wir die zweite Taste gedrückt, testen wir, ob unsere LED an sein müsste. Ist sie an, deaktivieren wir den Interrupt für die erste Taste (zweiter Aufgabenteil, will man dies nicht, so muss man nur Zeile 31 und 36 auskommentieren) schalten die LED aus. Ist die LED aus, aktivieren wir sie und schalten den Interrupt für den Taster 1 weider an.\\

\textbf{Beobachtung:}\\

Das Programm verhaltet sich wie gewünscht. Wir haben uns diesmal allerdings nicht weiter um das Prellen gekümmert. Dadurch gestaltete sich das Testen manchmal schwierig. Der Taster 1 funktionierte aber nur, wenn die LED 2 an ist.