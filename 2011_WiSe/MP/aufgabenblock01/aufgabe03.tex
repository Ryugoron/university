\subsection*{Aufgabe 3: Ampelsteuerung}

\textbf{A 1.3.1} Nutzen Sie alle drei LED und den rechten Taster (P1.0), um eine Fußgängerampel (aus der Sichtweise des Autofahrers) zu programmieren. Folgender Ablauf soll dabei realisiert werden:

\begin{itemize}

\item Grundzustand alle LED aus
\item Wenn Taste gedrückt wird, sofort gelbe LED (P4.1) einschalten
\item Danach zeitverzögert gelbe LED aus und rote LED (P4.0) an
\item Nach einer Pause gelbe LED (P4.1) zur roten LED (P4.0) dazuschalten
\item Dann zeitverzögert nur die grüne LED (P4.2) einschalten
\item Nach einer weiteren Pause alle LEDs ausschalten
\item Erst danach mit einer größeren Wartezeit die Taste wieder abfragen

\end{itemize}

\vspace{\baselineskip}

Unsere Lösung sieht wie folgt aus:

\begin{lstlisting}
void myfunntion(){
	//Deine Mudda
}
\end{lstlisting}

Auswertung: