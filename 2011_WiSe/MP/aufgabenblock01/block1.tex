\documentclass[a4paper,ngerman]{article}

\usepackage{babel}
\usepackage[utf8]{inputenc} 
\usepackage[T1]{fontenc} 
\usepackage{tabularx}

\author{Paul Podlech \\ 3910583 \and Max Wisniewski\\4370074}
\title{Mikroprozessor Praktikum\\ Fr. HWP 2 \\ Aufgabenblock 1}
\date{\today}

\begin{document}

\maketitle

\newpage

\section*{Aufgabe 1: I/O Pots}

\subsection*{Output}

\begin{description}

\item{\bfseries A1.1.1}

Klären Sie die Funktion folgender Register:

\begin{description}

\item{\bfseries P4Sel:} Mit diesem Register kann man das Signal von oder an ein internes Modul weiterleiten. Das sorgt dafür, dass man entweder Daten direkt verarbeiten kann ohne das die CPU mitarbeiten muss, oder Signale automatisch rausgeschickt werden können, wie eine Lampe die im Takt blinken soll. Dabei steht \textbf{0} für I/O Funktionalität und \textbf{1} für Modul Funktionalität.

\item{\bfseries P4Dir:}  Legt fest, ob man von diesem Port liest oder auf ihn raus schreiben möchte. Dabei steht:\\
\textbf{0} für \emph{IN}, das heißt man will lesen und\\
\textbf{1} für \emph{OUT}, das heißt man will schreiben. 

\item{\bfseries P4Out:} Hier schreibt man die Daten hinein, wenn man etwas senden möchte. Sobald Direction auf \emph{OUT} steht, ist dieses Signal aussehn sichtbar.


\item{\bfseries P4In:} Von hier kann gelesen werden, welches Signal am vierten Port anliegt, wenn die Direction am \emph{P4Dir} auf \emph{IN} steht. Allerdings reflektiert es auch im \emph{OUT} Fall den Wert, den wir selber setzen.

\end{description}

Dies ist für für alle Ports gleich, wobei man für die anderen Ports die 4 durch die entsprechende Portnummer ersetzen muss.

\item{\bfseries A 1.1.2}

Erstellen Sie eine Liste von Bitoperationen auf und geben Sie Operationen zum Setzen, Toggeln, Rücksetzen an.

\begin{tabular}{c|c|c}

Name & Beschreibung & Symbol\\
\hline

AND & Verunded die die Zahlen Bitweise &  \&\\

OR & Verodert die Zahlen Bitweise & | \\

XOR &  VerXORt die zaheln Bitweise & HOCH \\

Komplement &  Bildet das Bitweise Komplemet & \textasciitilde \\

Rechtshift & Shiftet die Bits nach rechts und füllt mit 0en & >>\\

Linksshift & Shiftet die Bits nach links und füllt mit 0en & <<\\

\end{tabular}

\begin{description}

\item{\bfseries Setzen:} $P4OUT \; |= 1<< k$. Setzt das k-te Bit auf 1. Um mehrere Bits zu setzen, können wir mit der Bitmaske verodern, die an allen Stellen eine 1 hat, die wir setzen wollen und eine 0 sonst.

\item{\bfseries Toggeln:} $P4OUT \;  XOR= 1<<k$ Setzt das k-te Bit auf sein Komplement. Um mehrere Bits zu setzen, können wir mit der Bitmaske verodern, die an allen Stellen, die wir Toggeln wollen eine 1 und sonst eine 0.

\item{\bfseries Zurücksetzen:} $P4OUT \; \&= \textasciitilde (1<<k)$. Die Bitmaske hat an der k-ten Stelle eine 0 und sonst 1en. Damit Wird die k-te Stelle auf 0 gesetzt. Um mehrere Bits zu setzen, können wir mit der Bitmaske verunden, die an jeder Stelle eine 0 hat, die wir zurück setzen wollen und sonst 1en.

\end{description}

\item{\bfseries A1.1.3} Erläutern Sie anhand der Abbildung der internen Struktur einer Portleitung für die folgende Registerbelegung den Signalpfad und den Logikpegel der Portleitung P4.0

\begin{itemize}
\item define LED\_ROT(0x01)
\item P4SEL = 0x00
\item P4DIR = 0x0F
\item P4OUT |= LED\_ROT

\end{itemize}

Daraus könen wir herraus lesen, dass Select für alle auf 0 steht, dass heißt es ist als I/O Port geschaltet. Die Direction von 0x0F heißt, dass für Bit 0 eine 1 anliegt (da die unteren 4 Bit gesetzt sind). Wenn LED\_ROT mit unserem OUT verodert wird, kann man, wie in der letzen Aufgabe gezeigt, heißt dass, dass das 0te Bit gesetzt wird.\\
Aus unserem Bild kann man herrauslesen, das an P4.0 eine 1 herraus kommt.

\end{description}

\end{document}