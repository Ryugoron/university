\section{Aufgabe 5.1 : LPM und Interrupt}

\subsection{A 5.1.1}

In dieser Aufgabe soll der Strom und der Takt im Normalen Betrieb und im Lowpower Mode gemessen werden.\\

Dazu soll im Grundprojekt gestartet werden, die Messung von Strom und MCLK-Takt durchgeführt wrden. Danach soll in der \emph{main()} der Controller in den LPM4 versetzt werden und das Verhalten, Strom und Takt aufgeschrieben werden.\\

Als nächstes soll auf Tastendruck (P1.0) ein Interrupt ausgelöst werden, der den Cotroller 10s in einer While-Schleife arbeiten lässt. Analysieren Sie das Verhalten.\\

\textbf{Beobachtung:}\\

\begin{description}

\item{Normal}\\
	Strom : 5.65 mA\\
	Takt : $7,35$ MHz\\
	Beobachtung : Analog zu Aufgabe 2 

\item{LPM4}\\
	Strom : 0.32 mA\\
	Takt : 0 Hz\\
	Beobachtung : Sehr geringer Verbrauch und kein Takt

\item{Interrupt}\\
	Strom : 4.1 mA - 0.36 mA\\
	Takt :  7.35 MHz - 0 Hz\\
	Beobachtung : Wurde auf den Taster gedrückt, schnellt der Verbrauch für geschätzte 10s auf 4.1 mA bei 7.35 MHz. Danach fällt er wieder
			in den LPM zurück und hat dort den etwa selben Verbrauch.

\end{description}

\textbf{Erklärung:}\\

Das Programm zum ausgeben das Taktes wurde schon in Aufgabe 2 beschrieben und die konkreten Anweisungen stehen in der Aufgabe, daher wird hier auf die Erläuterung verzichtet.\\

Wie zu erwarten war, messen wir innerhalb der 10 Sekunden, nachdem wir den Interrupt bekommen haben, den selben Verbrauch und Takt, wie im Normalen Modus.

Im LowPower Modus ist der Verbrauch drastisch viel kleiner. Nach unseren Messung in Aufgabe 2, kann der Controller über eine sehr lange Zeit laufen, wenn man eine Batterie einsetzt.