\section{Aufgabe 5.2 : Auto Shutdown mit einer ON / OFF Logik}

\subsection{A 5.2.1}

In dieser Aufgabe soll ein Programm mit der folgenden Verhaltensweise umgesetzt werden:\\
Wenn das Gerät nicht benutzt wird, soll es in den LowPowerMode LPM4 versetzt werden. Auf Tastendruck soll das Gerät aufgeweckt werden und eine Zeit lang rechnen.\\

Das Hauptprogram sieht wie folgt aus:

\begin{itemize}
	\item Statusvariable auf LPM4 Mode setzen
	\item Tickvariable auf Null setzen
	\item Tastaturport (P1.0) interruptfähig initialisieren
	\item Eintreten in eine Endlosschleife
	\item Abfrage ob aktiver Mode aktuell ist, oder in LPM4 versetzt werden soll:
		\begin{itemize}
			\item Wenn aktiver Mode, dann schalte LED (P4.1) ein
			\item Wenn LPM4 Mode, dann alle LED aus und LPM4 an
			\item Wenn Taste (P1.0) gedrückt in einer Schleife Taste lesen, bis die Taste nicht mehr gedrück ist. Nach der Schleife testen, ob die Tick 	
				Variable > 2 ist. Ist dies der Fall den Mode auf LPM4 setzen.
		\end{itemize}
\end{itemize}	

Die ISR für den Taster soll das folgende realisieren:

\begin{itemize}
	\item Status der LP-Mode abfragen.
	\item Wenn Status auf LPM gesetzt ist:
		\begin{itemize}
			\item  Statusvariable auf aktiven Mode setzen
			\item Watchdog Timer programieren
		\end{itemize}
	\item Wenn der Status aktiv ist:
		\begin{itemize}
			\item Tickvariable auf Null setzen
		\end{itemize}
\end{itemize}

Die ISR für den Watchdog soll folgendes tun:

\begin{itemize}
	\item Zeittickvariable inkrementieren und LED (P4.2) toggeln
	\item Wenn der Status auf aktiven Mode ist, abtesten ob Timer > 60
	\item Ist dies der Fall, soll der LPM gesetzt und die LEDs deaktiviert werden.
\end{itemize}

\textbf{Programm}:\\

Zunächst haben wir die beiden Variablen angelegt, dabei soll das highest Bit für den Modus stehen und die verbleibenden Bits darunter für die Zeit.\\
0 steht an höchster Stelle für LPM und 1 für aktiven Mode.

\begin{lstlisting}
char state;
\end{lstlisting}

Danach bieten wir eine Initialisierungsmethode an:

\begin{lstlisting}
void init52(void){
	state = 0;
	IE1 |= 0x01;	// Interrupt für den Taster aktivieren
	IE1 |= WDT_IE; // Interrupt für den Watchdog aktivieren;

	P4SEL &= ~(0x07);
	P4DIR |= 0x07;
	P4OUT |= 0x07;

	BSCTL! |= DIVA_2;
	P1IES &= ~(0x01);
	P1SEL &= ~(0x01);
	P1DIR &= ~(0x01);
	P1OUT &= ~(0x01);
}
\end{lstlisting}

Nun legen wir die beiden ISRs an:

\begin{lstlisting}
#pragma vector = WDT_VECTOR
__interrupt void do_wdt52(void){
	state++;
	if(state & (1<<8) && (state | ((1<<8)-1)) > 60){
		state |= 1<<8;
		LED1OFF;
		LED2OFF;
		WDTCTL = WDTPW + WDTHOLD;
	} else {
		WDTCTL  = WDTPW + WDTCNTCL + WDTSSEL + WDTTMSEL;
	}
}

#pragma vector = PORT1_VECOTR
__interrupt void do_port152(void){
	if(state & (1<<8)){
		state &= (1<<8) - 1;
		WDTCTL  = DWTPW + WDTCNTCL + WDTSSEL + WDTTMSEL;
	} else {
		state = (1<<8);
	}
}
\end{lstlisting}

Und zuletzt das Hauptprogramm:

\begin{lstlisting}
void aufgabe52(void){
	if(state_aufgabe52 & (1<<8)){
		LED1ON;
	} else {
		LED1OFF;
		LED2OFF;
		LPM4;
	}
	while(P1IN & 0x01) {}

	if((state | ((1<<8)-1)) > 2) {
		state |= 1<<8;
	}
}
\end{lstlisting}

\textbf{Beobachtung}:\\

tada

\textbf{Erklärung}:\\

blub