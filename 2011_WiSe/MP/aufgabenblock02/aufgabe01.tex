\subsection*{Aufgabe 1: Taktfrequenz}

\begin{description}

\item{\bfseries A 2.1.1} Bestimmen Sie messtechnisch die Frequenz der \emph{LFXT1CLK} und  \emph{XT2CLK} Taktquelle.\\

Das Programm, dass deen Takt auf die Portleitung P5.0 legt, sieht wie folgt aus:

\begin{lstlisting}
void aufgabe21(){
	P5Sel |= (1<<4);	//  Module Mode
	P5Dir |= (1<<4);	// P5.4 auf Out
	BCSCTL2 = BCSCTL2 & SELM_3 | SELM_3;	// Die oberen beiden Bits sind Multiplexer
								// den wir auf LFXT1CLK setzen.

	//BCSCTL2 = BCSCTL2 & SELM_2 | SELM_2;	// Setzt das hoechste Bit auf 1 und das 2. auf 0
					// Damit wird XT2CLK 
	
	BCSCTL" = BCSCTL2 & DIVM_0 | DIVM_0;
}
\end{lstlisting}

\textbf{Bestimmung der Werte:}\\

\emph{LFXT1CLK} hat die Frequenz $32,7691$ kHz.\\

\emph{XT2CLK} hat die Frequenz $7,37326$ MHz.\\

\item{\bfseries A 2.1.2} Bestimmen Sie messtechnisch die minimale und maximale Takt- frequenz des MCLK-Taktes, die sich auf Basis der LFXT1CLK-, XT2CLK- und DCOCLK-Taktquellen bereitstellen läßt.\\

Unser letztes Programm muss nur angepasst werden, indem wir bei SEL noch die 0 testen (DCOCLK ist sowohl 0 als auch 1).\\

Die Minima und Maxima kann man über den Divider einstellen, dabei ist 0 ein Divisor von 1 und 4 ein Divisor von 8. Damit ergibt sich nach Messung:

\begin{center}
\begin{tabular}{lcc}
& Maximum & Minimum\\
LFTXT1CLK & 32,7 kHz & 4,09614 kHz \\
XT2CLK & 7,373 MHz & 921,656 kHz\\
DCOCLK & 7,259MhZ (ossziliert) & 925 kHz (osszilert) \\
\end{tabular}
\end{center}

\item{\bfseries A 2.1.3} An P2.5 ist ein Oszillatorwiderstand $R_{OSC}$ von 39kOhm angeschlossen. Erläutern Sie, wie der externe Widerstand für den DCOCLK-Taktgenerator nutzbar gemacht wird.\\

Der Oszilierende Wiederstand kann über das DCOR Bit, welches das unterste Bit in BCSCTL2 gesetzt werden kann.

\item{\bfseries A 2.1.4} Welchen Einfluss hat der Widerstand auf den DCOCLK-Taktgenerator? \\

Hat man den oszilierenden Wiederstand dazugeschaltet so ist die Frequenz bei 925kHz. Ist er nicht dabei haben wir eine Frequenz von 212 kHz gemessen. Bei einem 8 fachen Divisor haben wir also eine etwa die 4fache Frequenz.\\
Dies hat sich auch bei den anderen Divisoren bestätigt.

\end{description}