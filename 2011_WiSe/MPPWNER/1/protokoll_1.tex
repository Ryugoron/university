\documentclass[11pt,a4paper,ngerman]{article}
\usepackage[bottom=2.5cm,top=2.5cm]{geometry} 
\usepackage{babel}
\usepackage[utf8]{inputenc} 
\usepackage[T1]{fontenc} 
\usepackage{ae} 
\usepackage{amssymb} 
\usepackage{amsmath} 
\usepackage{graphicx}
\usepackage{fancyhdr}
\usepackage{fancyref}
\usepackage{listings}
\usepackage{xcolor}
\usepackage{paralist}
\usepackage{fancyhdr}
\usepackage{subfigure}
\pagestyle{fancy}
\fancyhead[C]{Mikroprozessorpraktikum}
\fancyhead[L]{Protokoll 1}
\fancyhead[R]{WS 2011/12}
\fancyfoot{}
\fancyfoot[L]{}
\fancyfoot[C]{\thepage / \pageref{LastPage}}
\renewcommand{\footrulewidth}{0.5pt}
\renewcommand{\headrulewidth}{0.5pt}
\setlength{\parindent}{0pt} 
\setlength{\headheight}{15pt}


\author{Teilnehmer:\\ \\Marco Träger, Matr. XXXXXXXX\\Alexander Steen, Matr. 4357549}
\date{Gruppe: XXX, Arbeitsplatz: XXX}
\title{Mikroprozessorpraktikum WS 2011/12\\ Aufgabenkomplex: 1}

\begin{document}

\lstset{language=c, basicstyle=\ttfamily\fontsize{10pt}{10pt}\selectfont\upshape, commentstyle=\rmfamily\slshape, keywordstyle=\rmfamily\bfseries, breaklines=true, frame=single, xleftmargin=3mm, xrightmargin=3mm, tabsize=2}

\maketitle
\thispagestyle{fancy}
\newpage
\section*{A 1.1 Output}
\subsection*{A 1.1.1}
\subsection*{A 1.1.2}
\subsection*{A 1.1.3}
\subsection*{A 1.1.4}
\subsection*{A 1.1.5}
\subsection*{A 1.1.6}
\section*{A 1.2	Input}
\subsection*{A 1.2.1}
\begin{description}
\item[P1DIR] entscheidet, ob der jeweilige Pin als Eingang oder Ausgang fungiert, dabei beschreibt 0 einen Eingang, 1 einen Ausgang
\item[P1IN] besteht aus einem Byte, deren Bits den aktuellen Logikpegel an dem jeweiligen Pin des Ports 1 darstellen
\item[P1OUT] zeigt an dem jeweiligen Bit an, welcher Logikpegel an dem zugehörigen Port anliegen soll, falls P1DIR auf Ausgang und P1SEL auf I/O-Funktion geschaltet ist
\item[P1SEL] gibt an, ob die einzelnen Pins des Port 1 direkt als I/O benutzt werden oder für ein angeschlossenes Modul genutzt werden
\item[P1IE] de-/aktiviert die Intertupt-Flags (P1IFG) für die Pins des ersten Ports.
\item[P1IES] entscheidet, ob man Interrupt durch eine low-high-Flanke (0) oder eine high-low-Flanke (1) auf dem jeweiligen Pin ausgelöst werden soll.
\item[P1IFG] bezeichnet die Interrupt-Flags der Pins von Port 1. Ist ein Bit von P1IFG auf 1 gesetzt, so wurde von dem zugehörigen Pin ein Interrupt ausgelöst.
\end{description}

\subsection*{A 1.2.2}
Der AND-Operator (\&) führt bitweise die Verundung der Bits der Arguments aus.\\
Geht man für das Codebeispiel
\begin{lstlisting}
if (P1IN & Taster) {...}
\end{lstlisting}
davon aus, dass man Pin $i$ ein Taster angeschlossen ist, so kann man durch Wahl der Variable \texttt{Taster} als Bitmaske, die nur an der Stelle $i$ eine 1 enthält (\texttt{Taster} = $2^i$), erreichen, dass die Abfrage genau dann erfolgreich ist, falls der Taster gedrückt wurde.
\subsection*{A 1.2.3}
\begin{lstlisting}[numbers=left]
#define Taster_rechts (0x01)
#define Taster_links (0x02)
P1DIR = 0x00;
P4DIR = 0xFF;
P4OUT = 0;
a = 7;
P4OUT = a;
P1OUT = a;
a = P1IN & 0x30; //beide Tasten
a = P1IN & 0x00; //Taste rechts
a = P1IN & 0x01; //Taste rechts
a = P1IN & 0x02; //Taste rechts
a = P1IN & 0x03; //Taste links
a = P1IN & 0x03; //beide Tasten
P4OUT = P1IN & Taster_rechts; //Taster an P1.0 nicht
P4OUT = P1IN & Taster_links; //Taster an P1.0
\end{lstlisting}
Alle Pins von Ports 1 werden auf Eingang geschaltet.
Hier werden nun alle Pins von Ports 4 auf Ausgang geschaltet.
An alle Pins von Port 4 werden die Logikpegel 0 angelegt.

\subsection*{A 1.2.4}
\section*{A 1.3 Ampel}
\subsection*{A 1.3.1}
\section*{A 1.4 Taster}
\subsection*{A 1.3.1}
\label{LastPage}
\end{document}