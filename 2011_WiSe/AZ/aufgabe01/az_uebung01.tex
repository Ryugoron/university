\documentclass[11pt,a4paper,ngerman]{article}
\usepackage[bottom=2.5cm,top=2.5cm]{geometry} 
\usepackage{babel}
\usepackage[utf8]{inputenc} 
\usepackage[T1]{fontenc} 
\usepackage{ae} 
\usepackage{amssymb} 
\usepackage{amsmath} 
\usepackage{graphicx}
\usepackage{fancyhdr}
\usepackage{fancyref}
\usepackage{listings}
\usepackage{xcolor}
\usepackage{paralist}
\usepackage[pdftex, bookmarks=false, pdfstartview={FitH}, linkbordercolor=white]{hyperref}
\usepackage{fancyhdr}
\pagestyle{fancy}
\fancyhead[C]{Alegbra und Zahlentheorie}
\fancyhead[L]{Übung Nr. 1}
\fancyhead[R]{WS 2011/12}
\fancyfoot{}
\fancyfoot[L]{}
\fancyfoot[C]{\thepage}
\renewcommand{\footrulewidth}{0.5pt}
\renewcommand{\headrulewidth}{0.5pt}
\setlength{\parindent}{0pt} 
\setlength{\headheight}{15pt}

%
\newcommand{\N}{\mathbb{N}}
\newcommand{\NN}{\mathbb{N} \setminus \{0\}}
\newcommand{\Z}{\mathbb{Z}}
%

\author{Tutor: not known}
\date{}
\title{Max Wisniewski , Alexander Steen}

\begin{document}

\lstset{language=Java, basicstyle=\ttfamily\fontsize{10pt}{10pt}\selectfont\upshape, commentstyle=\rmfamily\slshape, keywordstyle=\rmfamily\bfseries, breaklines=true, frame=single, xleftmargin=3mm, xrightmargin=3mm, tabsize=2}

\maketitle
\thispagestyle{fancy}

%% -----------------------------------------------------------------
%%                   	 AUFGABE 1
%% ----------------------------------------------------------------

\subsection*{Aufgabe 1 (Teilbarkeit)}

Gegeben seien natürliche Zahlen $k, m, n \in \NN $, so dass $ n = k \cdot m $.

\begin{enumerate}[\bfseries a)]

\item Beweisen Sie folgende Aussage:

$$\forall a,b \in \Z : \; (a^m - b^m) | (a^n - b^n).$$


%% ------------------------------------------------------------------
%%			Beweis Aufgabe 1 a)
%% ------------------------------------------------------------------

\textbf{Beweis}: \\
Seien $p_1, ... , p_s$ alle Primzahlen, die kleiner gleich $\max\{a,b\}$ sind.

\item Zeigen Sie weiter:

$$ k \text{ ungerade} \quad \Rightarrow \quad (\forall a,b \in \Z : (a^m + b^m) | (a^n + b^n))$$

%% ------------------------------------------------------------------
%%			Beweis Aufgabe 1 b)
%% ------------------------------------------------------------------

\textbf{Beweis}: \\

tbd by your mother

\end{enumerate}



%% -----------------------------------------------------------------
%%                   	 AUFGABE 2
%% ----------------------------------------------------------------


\subsection*{Aufgabe 2 (Primzahlen)}

\begin{enumerate}[\bfseries a)]

\item Bestimmen Sie mit dem Sieb des Erastrothenes alle Primzahlen zwischen 2 und 200. \\
$\{2, 3, 5, 7, 1, ,13, 17, 19,23, 29, 31, 37, 41, 43, 47, 53, 59, 61, 67, 71, 73, 79, 83,$\\
$ 89, 97, 101, 103, 107, 109, 113, 127, 131, 137, 139, 149, 151, 157, 163, 167, 173,$\\
$ 179, 181, 191, 193, 197, 199\}$\\

Und nu darf hier noch wer den Algorithmus runter brechen.

\item Geben Sie die Primfaktorzerlegung der Zahl $-1.601.320$ an.\\

$$-1.601.320 = -1 \cdot 43 \cdot 19 \cdot 7^2 \cdot 5 \cdot 2^3$$

\end{enumerate}



%% -----------------------------------------------------------------
%%                   	 AUFGABE 3
%% ----------------------------------------------------------------


\subsection*{Aufgabe 3 (Teiler)}

Für $n \in \N$ mit $n \geq 1$ sei $T_n := \{ l \geq 1 | \; l | n\}$ die Menge ihrer Teiler.

\begin{enumerate}[\bfseries a)]

\item Es sei $n = p_{1}^{k_1} \cdot ... \cdot p_{s}^{k_s}$ die Primfaktorzerlegung von $n$. Geben Sie eine Formel für die Anzahl $\#T_n$ der Teiler von $n$ an.\\

%% -----------------------------------------------------------------
%%                   	 Lösung Aufgabe 3a
%% ----------------------------------------------------------------


Für diese Formel reicht uns ein einfaches kombinatorisches Argument. Wir haben $s$ verschiedene Elemente mit jeweils $k_i$ vorkommen. Diese wollen wir nun in allen kombination Möglichkeiten haben. Dies führt zur Formel:
$$\#T_n = \prod_{j=0}^{s} ( k_j + 1)$$

\item Charakterisieren Sie diejenigen Zahlen, für die $\#T_n$ ungerade ist.


%% -----------------------------------------------------------------
%%                   	 Lösung Aufgabe 3b
%% ----------------------------------------------------------------

\textbf{Vermutung:} $\#T_n$ ungerade $\Leftrightarrow \exists a\in \N : a^2 = n$.\\
\textbf{Beweis:}\\


\end{enumerate}


%% -----------------------------------------------------------------
%%                   	 AUFGABE 4
%% ----------------------------------------------------------------


\subsection*{Augabe 4 (Die Amnestie)}

Ein Herrscher hält 500 Personen in Einzelzellen gefangen, die von 1 bis 500 durchnummeriert sind. Anlässlich seines fünfizgsten Geburtstags gewährt er eine Amnestie nach folgenden Regeln:

\begin{itemize}

\item Am ersten Tag werden alle Zellen aufgeschlossen.

\item Am Tag $i$ wird der Schlüssel der Zellen $i, 2i, 3i$ usw. einmal umgedreht, d.\,h. Zelle j wird versperrt, wenn sie offen war, und geöffnet, wenn sie verschlossen war, $j = i, 2i, 3i$ usw., $i = 2, ... , 500$.

\end{itemize}

Wie viele Gefangene kommen frei? Ist der Insasse von Zelle 179 unter den Freigelassenen?

\vspace{30px}
\textbf{Vermutung:} 



\end{document}