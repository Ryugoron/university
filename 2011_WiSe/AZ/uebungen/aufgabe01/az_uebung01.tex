\documentclass[11pt,a4paper,ngerman]{article}
\usepackage[bottom=2.5cm,top=2.5cm]{geometry} 
\usepackage{babel}
\usepackage[utf8]{inputenc} 
\usepackage[T1]{fontenc} 
\usepackage{ae} 
\usepackage{amssymb} 
\usepackage{amsmath} 
\usepackage{graphicx}
\usepackage{fancyhdr}
\usepackage{fancyref}
\usepackage{listings}
\usepackage{xcolor}
\usepackage{paralist}
\usepackage[pdftex, bookmarks=false, pdfstartview={FitH}, linkbordercolor=white]{hyperref}
\usepackage{fancyhdr}
\pagestyle{fancy}
\fancyhead[C]{Alegbra und Zahlentheorie}
\fancyhead[L]{Übung Nr. 1}
\fancyhead[R]{WS 2011/12}
\fancyfoot{}
\fancyfoot[L]{}
\fancyfoot[C]{\thepage / \pageref{LastPage}}
\renewcommand{\footrulewidth}{0.5pt}
\renewcommand{\headrulewidth}{0.5pt}
\setlength{\parindent}{0pt} 
\setlength{\headheight}{15pt}

%
\newcommand{\N}{\mathbb{N}}
\newcommand{\NN}{\mathbb{N} \setminus \{0\}}
\newcommand{\Z}{\mathbb{Z}}
%

\author{Tutor: David Müßig}
\date{}
\title{Max Wisniewski , Alexander Steen}

\begin{document}

\lstset{language=Java, basicstyle=\ttfamily\fontsize{10pt}{10pt}\selectfont\upshape, commentstyle=\rmfamily\slshape, keywordstyle=\rmfamily\bfseries, breaklines=true, frame=single, xleftmargin=3mm, xrightmargin=3mm, tabsize=2}

\maketitle
\thispagestyle{fancy}

%% -----------------------------------------------------------------
%%                   	 AUFGABE 1
%% ----------------------------------------------------------------

\subsection*{Aufgabe 1 (Teilbarkeit)}

Gegeben seien natürliche Zahlen $k, m, n \in \NN $, so dass $ n = k \cdot m $.

\begin{enumerate}[\bfseries a)]

\item Beweisen Sie folgende Aussage:

$$\forall a,b \in \Z : \; (a^m - b^m) | (a^n - b^n).$$


%% ------------------------------------------------------------------
%%			Beweis Aufgabe 1 a)
%% ------------------------------------------------------------------

\textbf{Beweis}: \\
Seien $a,b\in \Z$.\\
Faktorisiert man die Formel aus, so gilt:\\
$$(*) \quad a^n - b^n = \left( a - b \right) \cdot \sum_{s=0}^{n-1} a^s b^{n-1-s}$$

Nun können wir diese Formel verwenden.

$$\begin{array}{crl}
&\left(a^m - b^m \right)&|\left( a^n - b^n \right) \\
\stackrel{\text{Def.}}{\Leftrightarrow} & \left( a^m - b^m \right) &| \left( \left( a^m \right) ^k - \left( b^m \right)^k \right)\\ 
\stackrel{\text{Fakt.}}{\Leftrightarrow} &  \left( a^m - b^m \right) &| \left( a^m - b^m \right) \cdot \underset{s=0}{\overset{k-1}{\sum}} \left( a^m \right)^s \left( b^m \right)^{k-1-s}\\
\end{array}$$

Nach Eigenschaft 4 der Vorlesung ($\forall a,b \in \Z\;\forall c \in \Z \setminus \{0\} \; : \; a|b \Leftrightarrow a|(c \cdot b)$), dass die Aussage stimmen muss, da gilt $\forall p \in \Z \; : \; p | p$, was ebenfalls in der Vorlesung gezeigt wurde.

\item Zeigen Sie weiter:

$$ k \text{ ungerade} \quad \Rightarrow \quad (\forall a,b \in \Z : (a^m + b^m) | (a^n + b^n))$$

%% ------------------------------------------------------------------
%%			Beweis Aufgabe 1 b)
%% ------------------------------------------------------------------

\textbf{Beweis}: \\
Seinen $a,b\in \Z$.\\
Faktorisiert man die Formal aus, so gilt für ungerade n:
$$(**) \quad a^n + b^n = (a + b) \cdot \sum_{s = 0}^{n - 1} (-1)^s a^sb^{n-1-s}$$

Wenden wir diese Formel auf die Aussage an, kommen wir auf:

$$
\begin{array}{crl}

& \left( a^m + b^m \right) &| \left( a^n + b^n \right)\\
\stackrel{\text{Def.}}{\Leftrightarrow}& \left( a^m + b^m \right) &| \left( \left( a^m \right)^k + \left( b^m\right)^k \right)\\
\stackrel{\text{Formel}}{\Leftrightarrow} & \left( a^m + b^m \right) &| \left( a^m + b^m \right)\cdot \left( \underset{s=0}{\overset{k-1}{\sum}} (-1)^s \left( a^m \right)^s \left( b^m \right)^{k-1-s} \right)
\end{array}
$$

Nun gilt wieder nach den selben Überlegungen wie in \textbf{a)} muss diese Formel gelten.

\item Beweis (*)\\

Seien $a,b \in \Z$ und $n \in \N$.\\
Rechnen wir die Faktorisierte Form aus (von $n-1$ aus angefangen):
$$
\begin{array}{cl}
& \left( a - b\right) \cdot \overset{n-1}{\underset{s=0}{\sum}} a^s b^{n-1-s} \\
=& \left( \left( a^n - a^{n-1}b \right) + \left( a^{n-1} b - a^{n-2}b^2 \right) + ... + \left( a^2b^{n-2} - ab^{n-1} \right) + \left(ab^{n-1} - b^n \right) \right)\\
=&  a^n + \left( - a^{n-1}b +  a^{n-1} b \right)  + \left( - a^{n-2}b^2  + ... \right) +  ...+ \left( ...+   a^2b^{n-2}\right) + \left(  - ab^{n-1}  + ab^{n-1}\right) - b^n\\
=& a^n - b^n
\end{array}
$$

Bis auf die äußeren beiden Elemente stehen je 2 aufeinanderfolgende Summanden, die sich gegenseitig auslöschen.

\item Beweis (**)\\

Seien $a,b \in \Z$ und $n \in \N$ mit $\exists k \in \N \; : \; n = 2\cdot k + 1$.\\
Rechnen wir die Faktorisierte Form aus (von $n-1$ aus angefangen):
$$
\begin{array}{cl}
& \left( a + b \right) \cdot \overset{n-1}{\underset{s=0}{\sum}} (-1)^s a^s b^{n-1-s}\\
=& (-1)^{2k} \left( a^n + a^{n-1}b \right) + (-1)^{2k-1} \left( a^{n-1}b + a^{n-2}b^2 \right) \\
& + ... + (-1)^{1} \left(a^2b^{n-2} + ab^{n-1} \right) + (-1)^0 \left( ab^{n-1} b^n\right)\\
=& a^n + \left( a^{n-1}b - a^{n-1}b \right) + ... + \left(ab^{n-1} - ab^{n-1} \right) + b^n\\
=& a^n + b^n
\end{array}
$$

Wie bei c) löschen sich bis auf die äußersten beide alle Elemente gegenseitig aus.

\end{enumerate}



%% -----------------------------------------------------------------
%%                   	 AUFGABE 2
%% ----------------------------------------------------------------


\subsection*{Aufgabe 2 (Primzahlen)}

\begin{enumerate}[\bfseries a)]

\item Bestimmen Sie mit dem Sieb des Erastrothenes alle Primzahlen zwischen 2 und 200. \\
Das streichen der Elemente kann mit der Legende rechts der Zahlen nachvollzogen werden.\\
$$
\begin{array}{cccccccccc}
1 & 2 & 3 & 4 & 5 & 6 & 7 & 8 & 9 & 10 \\
11 & 12 & 13 & 14 & 15 & 16 & 17 & 18 & 19 & 20 \\
21 & 22 & 23 & 24 & 25 & 26 & 27 & 28 & 29 & 30 \\
31 & 32 & 33 & 34 & 35 & 36 & 37 & 38 & 39 & 40 \\
41 & 42 & 43 & 44 & 45 & 46 & 47 & 48 & 49 & 50 \\
51 & 52 & 53 & 54 & 55 & 56 & 57 & 58 & 59 & 60 \\
61 & 62 & 63 & 64 & 65 & 66 & 67 & 68 & 69 & 70 \\
71 & 72 & 73 & 74 & 75 & 76 & 77 & 78 & 79 & 80 \\
81 & 82 & 83 & 84 & 85 & 86 & 87 & 88 & 89 & 90 \\
91 & 92 & 93 & 94 & 95 & 96 & 97 & 98 & 99 & 100 \\
101 & 102 & 103 & 104 & 105 & 106 & 107 & 108 & 109 & 110 \\
111 & 112 & 113 & 114 & 115 & 116 & 117 & 118 & 119 & 120 \\
121 & 122 & 123 & 124 & 125 & 126 & 127 & 128 & 129 & 130 \\
131 & 132 & 133 & 134 & 135 & 136 & 137 & 138 & 139 & 140 \\
141 & 142 & 143 & 144 & 145 & 146 & 147 & 148 & 149 & 150 \\
151 & 152 & 153 & 154 & 155 & 156 & 157 & 158 & 159 & 160 \\
161 & 162 & 163 & 164 & 165 & 166 & 167 & 168 & 169 & 170 \\
171 & 172 & 173 & 174 & 175 & 176 & 177 & 178 & 179 & 180 \\
181 & 182 & 183 & 184 & 185 & 186 & 187 & 188 & 189 & 190 \\
191 & 192 & 193 & 194 & 195 & 196 & 197 & 198 & 199 & 200 \\
\end{array}
$$
Alle Primzahlen in diesem Interval:\\
$\{2, 3, 5, 7, 11 ,13, 17, 19,23, 29, 31, 37, 41, 43, 47, 53, 59, 61, 67, 71, 73, 79, 83,$\\
$ 89, 97, 101, 103, 107, 109, 113, 127, 131, 137, 139, 149, 151, 157, 163, 167, 173,$\\
$ 179, 181, 191, 193, 197, 199\}$\\

\item Geben Sie die Primfaktorzerlegung der Zahl $-1.601.320$ an.\\

$$-1.601.320 = -1 \cdot 43 \cdot 19 \cdot 7^2 \cdot 5 \cdot 2^3$$

\end{enumerate}



%% -----------------------------------------------------------------
%%                   	 AUFGABE 3
%% ----------------------------------------------------------------


\subsection*{Aufgabe 3 (Teiler)}

Für $n \in \N$ mit $n \geq 1$ sei $T_n := \{ l \geq 1 | \; l | n\}$ die Menge ihrer Teiler.

\begin{enumerate}[\bfseries a)]

\item Es sei $n = p_{1}^{k_1} \cdot ... \cdot p_{s}^{k_s}$ die Primfaktorzerlegung von $n$. Geben Sie eine Formel für die Anzahl $\#T_n$ der Teiler von $n$ an.\\

%% -----------------------------------------------------------------
%%                   	 Lösung Aufgabe 3a
%% ----------------------------------------------------------------


Für diese Formel reicht uns ein einfaches kombinatorisches Argument. Wir haben $s$ verschiedene Elemente mit jeweils $k_i$ vorkommen. Diese wollen wir nun in allen kombination Möglichkeiten haben. Dies führt zur Formel:
$$\#T_n = \prod_{j=0}^{s} ( k_j + 1)$$

\item Charakterisieren Sie diejenigen Zahlen, für die $\#T_n$ ungerade ist.


%% -----------------------------------------------------------------
%%                   	 Lösung Aufgabe 3b
%% ----------------------------------------------------------------

\begin{description}

\item{\bfseries Lemma} Seien $n, t_1, t_2 \in \NN$ mit $n=t_1 \cdot t_2$. Dann ist $\left( t_1,t_2\right)$ ein Teilerpaar, d.h. es existiert keine andere Zahl $t_3$ für die gilt: $n=t_1 \cdot t_3$ oder $n=t_2 \cdot t_3$. Die Teilerbeziehung ist jeweils eindeutig.

\item{\bfseries Beweis} Gelten die Bezeichner aus dem Lemma.\\
Nehmen wir an, es gäbe o.B.d.A. zu $t_1$ nicht nur $t_2$ sondern auch $t_3$.\\
$$t_1 \cdot t_2 = t_1  \cdot t_3 \Leftrightarrow t_1 \cdot \left( t_2 - t_3\right) = 0$$
Da $t_1 \not=0$ ist, da es Teiler ist, muss $t_2=t_3$ gelten. Damit ist es eindeutig.\\

\end{description}

\textbf{Vermutung:} $\#T_n$ ungerade $\Leftrightarrow \exists a\in \N : a^2 = n$.\\
\textbf{Beweis:}\\
"$\Rightarrow$"\\
Da wir eine ungerade Zahl an Teilern haben, muss es eine Zahl $a$ geben, die keinen von sich verschiedenen Partner hat, dalle anderen nach Lemma einen eindeutigen Partner haben. Da aber gilt $a\;|\;n$, kann nur $n = a \cdot a$ gelten, womit $n$ eine Quadratzahl ist.\\
"$\Leftarrow$"\\
Da $n$ Quadratzahl ist, gibt es den Teiler $a$, der sein eingenes Teilerpaar darstellt. Korrollar zum Lemma gilt, dass es keine zweite Zahl $b$ gibt mit $b\not=a \land n = b\cdot b$. Wir haben also einen Teiler und jeder weiter Teiler kommt als Teilerpaar.\\
Damit haben wir $2k + 1$ Teiler. $\Rightarrow \#T_n$ ist ungerade.\\
\mbox{} \hfill $\square$

\end{enumerate}


%% -----------------------------------------------------------------
%%                   	 AUFGABE 4
%% ----------------------------------------------------------------


\subsection*{Aufgabe 4 (Die Amnestie)}

Ein Herrscher hält 500 Personen in Einzelzellen gefangen, die von 1 bis 500 durchnummeriert sind. Anlässlich seines fünfizgsten Geburtstags gewährt er eine Amnestie nach folgenden Regeln:

\begin{itemize}

\item Am ersten Tag werden alle Zellen aufgeschlossen.

\item Am Tag $i$ wird der Schlüssel der Zellen $i, 2i, 3i$ usw. einmal umgedreht, d.\,h. Zelle j wird versperrt, wenn sie offen war, und geöffnet, wenn sie verschlossen war, $j = i, 2i, 3i$ usw., $i = 2, ... , 500$.

\end{itemize}

Wie viele Gefangene kommen frei? Ist der Insasse von Zelle 179 unter den Freigelassenen?

\vspace{30px}

\textbf{Eigentschaft:} Der Schlüssel eiiner Zelle wird genau dann umgedreht, wenn der Tag Teiler der Zahl ist.

\textbf{Beweis:} \\
"$\Leftarrow$"\\
1. Tag, werden alle Zellen geöffnet. $k\not= 0 \Rightarrow 1 | k $. Da die Zellen im Bereich $[1,500]$ liegen ist $k \not= 0$.
Am Tag $j$ gilt : $\forall k\in \N : k\cdot i$ wird geöffnet. $k \cdot i$.\\
Hat Zelle $z$ nun den Teiler $j$, so gilt: $\exists k' \in \N : z = j \cdot k'$. Dies erfüllt die drehen Vorraussetzung.\\
"$\Rightarrow$"\\
Sei $z$ Zelle und $j$ Tag und es gilt $j \not | z. => \exists k,r \in \N : k\cdot j + r = z \land 0<r<j$.\\
Da aber nur $t\cdot j$ für beliebige t gedreht wird, kann bei der Zelle das Schloss nicht nochmal gedreht werden.
\mbox{} \hfill $\square$


\textbf{Vermutung:} Die Zelle $z$ ist am Ende offen, genau dann wenn $\#T_z$ ungerade ist.\\
\textbf{Beweis:}\\
"$\Leftarrow$"\\
$\exists k \in \N : z = 2 \cdot k + 1$\\
Am ersten Tag werden alle Türen geöffnet. Bleibe $2\cdot k$ Drehvorgänge. Da aber nach beschreibung sich 2 Vorgänge paarweise aufheben,
wird die Tür am Ende geöffnet sein.\\
"$\Rightarrow$"\\
$\exists k \in \N : z = 2 \cdot k + 2$ ist möglich, da 0 keine unserer Türen ist.\\
Am ersten Tag wird die Tür wieder geöffnet. Bleiben $2\cdot k +1$ Drehvorgänge, von denen sich $2\cdot k$ gegenseitig aufheben. Bleib uns ein Drehvorgang, der die Tür abschließt.\\
\mbox{} \hfill $\square$

\vspace{15px}

Nach 3b) müssen wir jetzt nur noch sehen, welche der Zellen Quadratzahlen sind:
$1, 4, 9, 16, 25, 36, 49, 64, 81, 100, 121, 144$\\
$169, 196, 225, 256, 289, 324, 361, 400, 484$

\vspace{10px}

Da die 179 keine Quadratzahl ist, wird die Zelle am Ende der 500 Tage geschlossen sein.


\label{LastPage}
\end{document}