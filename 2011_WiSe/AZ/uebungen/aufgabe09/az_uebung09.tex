\documentclass[11pt,a4paper,ngerman]{article}
\usepackage[bottom=2.5cm,top=2.5cm]{geometry} 
\usepackage{babel}
\usepackage[utf8]{inputenc} 
\usepackage[T1]{fontenc} 
\usepackage{ae} 
\usepackage{amssymb} 
\usepackage{amsmath} 
\usepackage{graphicx}
\usepackage{fancyhdr}
\usepackage{fancyref}
\usepackage{listings}
\usepackage{xcolor}
\usepackage{paralist}
\usepackage{fancyhdr}
\usepackage{subfigure}
\pagestyle{fancy}
\fancyhead[C]{Alegbra und Zahlentheorie}
\fancyhead[L]{Übung Nr. 9}
\fancyhead[R]{WS 2011/12}
\fancyfoot{}
\fancyfoot[L]{}
\fancyfoot[C]{\thepage\, von \pageref{LastPage}}
\renewcommand{\footrulewidth}{0.5pt}
\renewcommand{\headrulewidth}{0.5pt}
\setlength{\parindent}{0pt} 
\setlength{\headheight}{15pt}

%
\newcommand{\N}{\mathbb{N}}
\newcommand{\NN}{\mathbb{N} \setminus \{0\}}
\newcommand{\Z}{\mathbb{Z}}
\newcommand{\C}{\mathbb{C}}
\newcommand{\R}{\mathbb{R}}
\newcommand{\Q}{\mathbb{Q}}
\newcommand{\ggT}{\text{ggT}}
\newcommand{\kgV}{\text{kgV}}
\newcommand{\sign}{\text{Sign}}
\newcommand{\ord}{\text{Ord}}
%

\author{Tutor: David Müßig}
\date{}
\title{Max Wisniewski, Alexander Steen}

\begin{document}

\lstset{language=Java, basicstyle=\ttfamily\fontsize{10pt}{10pt}\selectfont\upshape, commentstyle=\rmfamily\slshape, keywordstyle=\rmfamily\bfseries, breaklines=true, frame=single, xleftmargin=3mm, xrightmargin=3mm, tabsize=2}

\maketitle
\thispagestyle{fancy}

%% -------------------------------------------
%%                 AUFGABE 1
%% -------------------------------------------

\subsection*{Aufgabe 1 \mdseries (Untergruppen von Primzahlindex)}

Geben Sie für jede Primzahl $p > 2$ endliche Gruppen $G$ und $H$ an, so dass $\# (G / H) = p$ und $H$ kein Normalteiler von $G$ ist.\\

\textbf{Lösung:}\\

Es sei $G = S_p$ und damit $\# G = \# S_p = p!$. Sei nun $H$ die Gruppe der Permutationen, die alle Elemente bis auf das erste Permutiert, also $H = \left\{  \sigma \in G \; | \; \sigma \; 1 = 1 \right\}$.\\

$H$ erfüllt die Gruppenaxiome:
\begin{description}
\item{\bfseries Abgeschlossenheit:} \\ Seien $a,b \in H$, dann ist auch $a \cdot b \in H$, da $(a \cdot b)(1) = a(b(1)) \stackrel{b \in H} = a(1) = 1$

\item{\bfseries Inverses Element:} \\ Sei $a \in H$, dann ist auch $a^{-1} \in H$, da
$$
\begin{array}{crcl}
& a^{-1}(1) &=& 1\\
\Leftrightarrow & a\cdot a^{-1} (1) &=& a(1)\\
\Leftrightarrow & id(1) &=& a(1)\\
\Leftrightarrow & 1 &=& 1
\end{array}
$$ 

\item{\bfseries Neutrales Element:} \\ $id(1) = 1 \Rightarrow id \in H$.\\
\mbox{} \hfill $\square$

\end{description}

$H \cong S_{p-1}$, da wir bis auf ein Element alle Elemente permutieren.\\

Nun gilt nach dem Satz von Lagrange: $\# G = \# H \cdot \# (G / H)$. Das $\# (G / H ) = p$. Wir haben also eine Linksunterklasse gebildet, die genau $p$ Elemente enthält.\\

$H \vartriangleleft G \Leftrightarrow \forall g\in G \forall h \in H \; : \; ghg^{-1} \in H$.\\
Sei nun $g=g^{-1}=(1 \; 2) \in G$ und $h = (2 \; 3) \in H$.\\

$ghg^{-1} (1) = gh (2) = g (3) = 3 \not \in H$.

Damit kann $H$ nicht Normalteiler von $G$ sein.\\
\mbox{} \hfill $\square$

%% -------------------------------------------
%%                 AUFGABE 2
%% -------------------------------------------

\subsection*{Aufgabe 2 \mdseries (Die orthogonale Gruppe)}

\begin{enumerate}[\bfseries a)]
%% -------------------------------------------
%%                 a)
%% -------------------------------------------
\item Zeigen Sie, dass die orthogonale Gruppe $O(2)$ von Spiegelungen erzeugt wird.\\

\textbf{Lösung:}\\

tbd

%% -------------------------------------------
%%                 b)
%% -------------------------------------------
\item Es sei $N \vartriangleleft O(2)$ eine normale Untergruppe, die eine Spiegelung enthält. Beweisen Sie $N = O(2)$.\\

\textbf{Lösung:}\\

tbd

%% -------------------------------------------
%%                 c)
%% -------------------------------------------
\item Es seien $r \in O(2)$ eine Drehung und $G = \left< r \right>$. Weise Sie nach, dass $N$ eine normale Untergruppe ist.\\

\textbf{Lösung:}\\

tbd

%% -------------------------------------------
%%                 d)
%% -------------------------------------------
\item Wann ist die untergruppe $G$ aus Teil $c)$ endlich?\\

\textbf{Lösung:}\\

tbd

\end{enumerate}

%% -------------------------------------------
%%                 AUFGABE 3
%% -------------------------------------------

\subsection*{Aufgabe 3 \mdseries (Die Kommutatorenuntergruppe von $S_n$)}

\textbf{Satz:} Für $n>2$ gilt:
$$
\left[ S_n , S_n \right] = A_n.
$$

\textbf{Bew.:}\\

tbd

%% -------------------------------------------
%%                 AUFGABE 4
%% -------------------------------------------

\subsection*{Aufgabe 4 \mdseries (Die alternierende Gruppe $A_4$)}

\begin{enumerate}[\bfseries a)]
%% -------------------------------------------
%%                 a)
%% -------------------------------------------
\item Zeigen Sie, dass $e$ zusammen mit den Permutationen vom Zykeltyp $(2,2)$ eine normale Untergruppe $H \vartriangleleft A_4$ bildet.\\

\textbf{Lösung:}\\

tbd


%% -------------------------------------------
%%                 b)
%% -------------------------------------------
\item Geben Sie einen Homomorphismus $\varphi \; : \; A_4 \rightarrow \mathbb{Z}_3$ mit $Ker(\varphi ) = H$ an, $H$ die Untergruppe aus Teil a).\\

\textbf{Lösung:}\\

tbd

\end{enumerate}

\label{LastPage}


\end{document}