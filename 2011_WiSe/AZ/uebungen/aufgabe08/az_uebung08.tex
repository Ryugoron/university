\documentclass[11pt,a4paper,ngerman]{article}
\usepackage[bottom=2.5cm,top=2.5cm]{geometry} 
\usepackage{babel}
\usepackage[utf8]{inputenc} 
\usepackage[T1]{fontenc} 
\usepackage{ae} 
\usepackage{amssymb} 
\usepackage{amsmath} 
\usepackage{graphicx}
\usepackage{fancyhdr}
\usepackage{fancyref}
\usepackage{listings}
\usepackage{xcolor}
\usepackage{paralist}
\usepackage{fancyhdr}
\usepackage{subfigure}
\pagestyle{fancy}
\fancyhead[C]{Alegbra und Zahlentheorie}
\fancyhead[L]{Übung Nr. 8}
\fancyhead[R]{WS 2011/12}
\fancyfoot{}
\fancyfoot[L]{}
\fancyfoot[C]{\thepage\, von \pageref{LastPage}}
\renewcommand{\footrulewidth}{0.5pt}
\renewcommand{\headrulewidth}{0.5pt}
\setlength{\parindent}{0pt} 
\setlength{\headheight}{15pt}

%
\newcommand{\N}{\mathbb{N}}
\newcommand{\NN}{\mathbb{N} \setminus \{0\}}
\newcommand{\Z}{\mathbb{Z}}
\newcommand{\C}{\mathbb{C}}
\newcommand{\R}{\mathbb{R}}
\newcommand{\Q}{\mathbb{Q}}
\newcommand{\ggT}{\text{ggT}}
\newcommand{\kgV}{\text{kgV}}
\newcommand{\sign}{\text{Sign}}
\newcommand{\ord}{\text{Ord}}
%

\author{Tutor: David Müßig}
\date{}
\title{Max Wisniewski, Alexander Steen}

\begin{document}

\lstset{language=Java, basicstyle=\ttfamily\fontsize{10pt}{10pt}\selectfont\upshape, commentstyle=\rmfamily\slshape, keywordstyle=\rmfamily\bfseries, breaklines=true, frame=single, xleftmargin=3mm, xrightmargin=3mm, tabsize=2}

\maketitle
\thispagestyle{fancy}

%% -------------------------------------------
%%                 AUFGABE 1
%% -------------------------------------------

\subsection*{Aufgabe 1 \mdseries (Vorzeichen und Ordnung eines Zykels)}
\begin{enumerate}[\bfseries a)]
%% ------------------a) -----------------------
\item Es sei $c \in S_n$ ein Zykel der Länge $k$. Berechnen Sie das Vorzeichen $\sign(c)$. \\

Sei $c = (c_1 \ldots c_k) \in S_n$, $k \in \N,k > 1$. Dann gilt nach VL:
$$ c = (c_1 \ldots c_k) = (c_1 \; c_k) \cdot (c_1 \; c_{k-1}) \cdot \ldots \cdot (c_1 \; c_2)$$
Also wird das Zykel $c$ durch $k-1$ Transpositionen erzeugt. Da für jede Transposition $\tau \in S_n$ gilt: $\sign(\tau) = -1$ und ferner
$$ \sign((c_1 \; c_k) \cdot (c_1 \; c_{k-1}) \cdot \ldots \cdot (c_1 \; c_2)) = \sign(c_1 \; c_k) \cdot \sign(c_1 \; c_{k-1}) \cdot \ldots \cdot \sign(c_1 \; c_2)$$
gilt, folgt:
$$ \sign(c) = (-1)^{k-1} = \begin{cases}
  1,  & \text{falls }k\text{ ungerade,}\\
  -1, & \text{sonst.}
\end{cases} $$
%% ------------------b) -----------------------
\item  Welche Ordnung hat ein Zykel $c \in S_n$ der Länge $k$? \\

Sei $c = (c_1 \ldots c_k) \in S_n$, $k \in \N,k > 1$. \\
Dann ist die $\ord(c) = k$. So! Kein Bock mehr!
%% ------------------c) -----------------------
\item Es seien $\sigma \in S_n$ eine Permutation und $n=(n_1,...,n_s)$ ihr Zykeltyp. Welche Ordnung hat $\sigma$? \\
$\ord(\sigma)) = \kgV(n_1, n_2, ..., n_s)$ bitches!
\end{enumerate}



%% -------------------------------------------
%%		AUFGABE 2
%% ------------------------------------------

\subsection*{Aufgabe 2 \mdseries (Links- vs. Rechtswirkungen)}
Es seien $G$ eine Gruppe, $M$ eine Menge,
$\sigma : G \times M \to M$ eine Linkswirkung von $G$ auf $M$ und
$\rho : M \times G \to M$ eine Rechtswirkung. Es seien \\
 $\sigma^{*}: M \times G \to M$, $(m,g) \mapsto \sigma(g^{-1},m)$ und 
$\rho^{*}: G \times M \to M$, $(g,m) \mapsto \rho(m, g^{-1})$.

\begin{enumerate}[\bfseries i)]
\item Zu zeigen: $\sigma^{*}, \rho^{*}$ sind Wirkungen. \\
Sei $m \in M, g_1, g_2 \in G$. Dann gilt
\begin{enumerate}[\bfseries (1)]
\item für $\sigma^{*}$: \\
$$ \sigma^{*}(m,e) = \sigma(e^{-1},m) = \sigma(e,m) \stackrel{\sigma \text{ Wirking}}{=} m $$
$$ \sigma^{*}(\sigma^{*}(m,g_1),g_2) = \sigma^{*}(\sigma(g_1^{-1},m),g_2) = \sigma(g_2^{-1},\sigma(g_1^{-1},m)) $$
$$
\stackrel{\sigma \text{ Wirking}}{=} \sigma(g_2^{-1}\cdot g_1^{-1},m) 
= \sigma((g_1 \cdot g_2)^{-1},m) = \sigma^{*}(m,g_1 \cdot g_2)
$$
\mbox{} \hfill $\square$
\item für $\rho^{*}$: \\
$$ \rho^{*}(e,m) = \rho(m,e^{-1}) = \rho(m,e) \stackrel{\rho \text{ Wirking}}{=} m $$
$$ \rho^{*}(g_2,\rho^{*}(g_1,m)) = \rho^{*}(g_2,\rho(m,g_1^{-1})) = \rho(\rho(m,g_1^{-1}),g_2^{-1}) $$
$$
\stackrel{\rho \text{ Wirking}}{=} \rho(m,g_1^{-1}\cdot g_2^{-1}) 
= \rho(m,(g_2 \cdot g_1)^{-1}) = \rho^{*}(g_2 \cdot g_1,m)
$$
\mbox{} \hfill $\square$

\end{enumerate}

\item Vergleich der Bahnen und Standgruppen von $\sigma$ mit $\sigma^{*}$. \\

Ganz tolle Bahnen!

\end{enumerate}



%% -------------------------------------------
%%		AUFGABE 3
%% --------------------------------------------
\subsection*{Aufgabe 3 \mdseries (Das Zentrum)}

\begin{enumerate}[\bfseries a)]
\item Es seien $k$ ein Körper und $n \geq 1$. Bestimmen Sie das Zentrum von $GL_n(k)$. \\

alle $k \cdot E_n, k \in \Z$.

\item Geben Sie für $n \geq 1$ das Zentrum der symmetrischen Gruppe $S_n$ an. \\

$\{ id \}$
\end{enumerate}
%% -------------------------------------
%%		AUFGABE 4
%% -------------------------------------
\subsection*{Aufgabe 4 \mdseries (Ein Färbungsproblem)}


\label{LastPage}


\end{document}