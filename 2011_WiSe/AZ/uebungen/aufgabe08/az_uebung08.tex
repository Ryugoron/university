\documentclass[11pt,a4paper,ngerman]{article}
\usepackage[bottom=2.5cm,top=2.5cm]{geometry} 
\usepackage{babel}
\usepackage[utf8]{inputenc} 
\usepackage[T1]{fontenc} 
\usepackage{ae} 
\usepackage{amssymb} 
\usepackage{amsmath} 
\usepackage{graphicx}
\usepackage{fancyhdr}
\usepackage{fancyref}
\usepackage{listings}
\usepackage{xcolor}
\usepackage{paralist}
\usepackage{fancyhdr}
\usepackage{subfigure}
\pagestyle{fancy}
\fancyhead[C]{Alegbra und Zahlentheorie}
\fancyhead[L]{Übung Nr. 8}
\fancyhead[R]{WS 2011/12}
\fancyfoot{}
\fancyfoot[L]{}
\fancyfoot[C]{\thepage\, von \pageref{LastPage}}
\renewcommand{\footrulewidth}{0.5pt}
\renewcommand{\headrulewidth}{0.5pt}
\setlength{\parindent}{0pt} 
\setlength{\headheight}{15pt}

%
\newcommand{\N}{\mathbb{N}}
\newcommand{\NN}{\mathbb{N} \setminus \{0\}}
\newcommand{\Z}{\mathbb{Z}}
\newcommand{\C}{\mathbb{C}}
\newcommand{\R}{\mathbb{R}}
\newcommand{\Q}{\mathbb{Q}}
\newcommand{\ggT}{\text{ggT}}
\newcommand{\kgV}{\text{kgV}}
\newcommand{\sign}{\text{Sign}}
\newcommand{\ord}{\text{Ord}}
%

\author{Tutor: David Müßig}
\date{}
\title{Max Wisniewski, Alexander Steen}

\begin{document}

\lstset{language=Java, basicstyle=\ttfamily\fontsize{10pt}{10pt}\selectfont\upshape, commentstyle=\rmfamily\slshape, keywordstyle=\rmfamily\bfseries, breaklines=true, frame=single, xleftmargin=3mm, xrightmargin=3mm, tabsize=2}

\maketitle
\thispagestyle{fancy}

%% -------------------------------------------
%%                 AUFGABE 1
%% -------------------------------------------

\subsection*{Aufgabe 1 \mdseries (Vorzeichen und Ordnung eines Zykels)}
\begin{enumerate}[\bfseries a)]
%% ------------------a) -----------------------
\item Es sei $c \in S_n$ ein Zykel der Länge $k$. Berechnen Sie das Vorzeichen $\sign(c)$. \\

Sei $c = (c_1 \ldots c_k) \in S_n$, $k \in \N,k > 1$. Dann gilt nach VL:
$$ c = (c_1 \ldots c_k) = (c_1 \; c_k) \cdot (c_1 \; c_{k-1}) \cdot \ldots \cdot (c_1 \; c_2)$$
Also wird das Zykel $c$ durch $k-1$ Transpositionen erzeugt. Da für jede Transposition $\tau \in S_n$ gilt: $\sign(\tau) = -1$ und ferner
$$ \sign((c_1 \; c_k) \cdot (c_1 \; c_{k-1}) \cdot \ldots \cdot (c_1 \; c_2)) = \sign(c_1 \; c_k) \cdot \sign(c_1 \; c_{k-1}) \cdot \ldots \cdot \sign(c_1 \; c_2)$$
gilt, folgt:
$$ \sign(c) = (-1)^{k-1} = \begin{cases}
  1,  & \text{falls }k\text{ ungerade,}\\
  -1, & \text{sonst.}
\end{cases} $$
%% ------------------b) -----------------------
\item  Welche Ordnung hat ein Zykel $c \in S_n$ der Länge $k$?

Sei $c = (c_0 \ldots c_{k-1}) \in S_n$, $1 \leq k-1 < n$. \\
Dann ist die $\ord(c) = k$, denn für ein $i < k$ gilt:\\
$(c_0 \ldots c_{k-1})^k(c_i) = c_{(i+k \mod k)} = c_i$ und $(c_0 \ldots c_{k-1})^{k-1}(c_i) = c_{(i+k-1 \mod k)} \neq c_i$.\\
Für alle $i \leq n, i \notin \{c_0, \ldots ,c_{k-1} \}$ gilt natürlich auch $(c_0 \ldots c_{k-1})^k(i) = i$.
\mbox{} \hfill $\square$


% \textbf{Beweis:}\\
% \begin{description}
% \item[IA] $k = 2$, dann gilt für $c = (c_1 \; c_2) \in S_n$: $c^2 = id \land c^1=c \neq id \Rightarrow \ord(c) = 2$.
% \item[IV] Behauptung gilt für ein $k \in \N$.
% \item[IS] $k \rightsquigarrow k+1$ \\
% Es ist $c = (c_1 \; c_2 \; \ldots c_k \; c_{k+1})$. Dann gilt:
% $$ c^{k+1} = (c_1 \; c_2 \; \ldots c_k \; c_{k+1})^{k+1} = ((c_1 \; c_{k+1}) \cdot (c_1 \; c_2 \; \ldots c_k))^{k+1} $$
% \end{description}

%% ------------------c) -----------------------
\item Es seien $\sigma \in S_n$ eine Permutation und $n=(n_1,...,n_s)$ ihr Zykeltyp. Welche Ordnung hat $\sigma$?

Behauptung: $\ord(\sigma) = \kgV(n_1, n_2, ..., n_s)$. \\
\textbf{Beweis:}\\
Sei $\sigma = c_1 \cdot \ldots \cdot c_s, c_i \in S_n, 1 \leq i \leq s$. \\
Dann gilt nach b): $\ord(c_i) = n_i, 1 \leq i \leq s$. Da die Ordnung die kleinste Zahl $j \geq 1$ ist, für die $\sigma^j = id$ ist, ist die kleinste Zahl, die allen Zykeln $c_i$ zugleich gerecht wird, eben das kleinste gemeinsame Vielfache, also $\ord(\sigma) = \kgV(n_1, n_2, ..., n_s)$.
\end{enumerate}



%% -------------------------------------------
%%		AUFGABE 2
%% ------------------------------------------

\subsection*{Aufgabe 2 \mdseries (Links- vs. Rechtswirkungen)}
Es seien $G$ eine Gruppe, $M$ eine Menge,
$\sigma : G \times M \to M$ eine Linkswirkung von $G$ auf $M$ und
$\rho : M \times G \to M$ eine Rechtswirkung. Es seien \\
 $\sigma^{*}: M \times G \to M$, $(m,g) \mapsto \sigma(g^{-1},m)$ und 
$\rho^{*}: G \times M \to M$, $(g,m) \mapsto \rho(m, g^{-1})$.

\begin{enumerate}[\bfseries i)]
\item Zu zeigen: $\sigma^{*}, \rho^{*}$ sind Wirkungen. \\
Sei $m \in M, g_1, g_2 \in G$. Dann gilt
\begin{enumerate}[\bfseries (1)]
\item für $\sigma^{*}$: \\
$$ \sigma^{*}(m,e) = \sigma(e^{-1},m) = \sigma(e,m) \stackrel{\sigma \text{ Wirking}}{=} m $$
$$ \sigma^{*}(\sigma^{*}(m,g_1),g_2) = \sigma^{*}(\sigma(g_1^{-1},m),g_2) = \sigma(g_2^{-1},\sigma(g_1^{-1},m)) $$
$$
\stackrel{\sigma \text{ Wirking}}{=} \sigma(g_2^{-1}\cdot g_1^{-1},m) 
= \sigma((g_1 \cdot g_2)^{-1},m) = \sigma^{*}(m,g_1 \cdot g_2)
$$
\mbox{} \hfill $\square$
\item für $\rho^{*}$: \\
$$ \rho^{*}(e,m) = \rho(m,e^{-1}) = \rho(m,e) \stackrel{\rho \text{ Wirking}}{=} m $$
$$ \rho^{*}(g_2,\rho^{*}(g_1,m)) = \rho^{*}(g_2,\rho(m,g_1^{-1})) = \rho(\rho(m,g_1^{-1}),g_2^{-1}) $$
$$
\stackrel{\rho \text{ Wirking}}{=} \rho(m,g_1^{-1}\cdot g_2^{-1}) 
= \rho(m,(g_2 \cdot g_1)^{-1}) = \rho^{*}(g_2 \cdot g_1,m)
$$
\mbox{} \hfill $\square$

\end{enumerate}

\item Vergleich der Bahnen und Standgruppen von $\sigma$ mit $\sigma^{*}$. \\
Sei $x \in M$.
\begin{enumerate}[\bfseries (1)]
\item Bahnen: \\
Dann ist $G \cdot_{\sigma} x = \{ g \cdot_{\sigma} x | \, g \in G \}$ die Bahn von $x$ bzgl. $\sigma$. \\
Und $x \cdot_{\sigma^{*}} G  = \{ x \cdot_{\sigma^{*}} g | \, g \in G \}$ die Bahn von $x$ bzgl. $\sigma^{*}$. \\
Dann ist 
$$ x \cdot_{\sigma^{*}} G = \{ x \cdot_{\sigma^{*}} g | \, g \in G \} = \{ g^{-1} \cdot_{\sigma} x | \, g \in G \} $$
$$ \stackrel{*}{=} \{ g  \cdot_{\sigma} x | \, g \in G \} =  G \cdot_{\sigma} x $$
(*) gilt weil für jedes $g \in G$ auch $g^{-1}$ in $G$ enthalten ist.  \\
\item Standgruppen: \\
Es ist $G^{\sigma}_x = \{ g \in G | \; g \cdot_{\sigma} x = x \}$ die Standgruppe von x bzgl $\sigma$. \\
Es ist $G^{\sigma^{*}}_x = \{ g \in G | \; x \cdot_{\sigma^{*}} g = x \}$ die Standgruppe von x bzgl $\sigma^{*}$. \\
Dann ist
$$ G^{\sigma^{*}}_x = \{ g \in G | \; x \cdot_{\sigma^{*}} g = x \} = \{ g \in G | \; g^{-1} \cdot_{\sigma} x = x \} $$
$$ = \{ g^{-1} \in G | \; g \cdot_{\sigma} x = x \} = \{ g^{-1} | \; g \in G^{\sigma}_x \}$$
\end{enumerate}
\end{enumerate}
Es folgt insbesondere, dass die Mächtigkeiten der Standgruppen und Bahnen gleich groß sind.


%% -------------------------------------------
%%		AUFGABE 3
%% --------------------------------------------
\subsection*{Aufgabe 3 \mdseries (Das Zentrum)}

\begin{enumerate}[\bfseries a)]
\item Es seien $k$ ein Körper und $n \geq 1$. Bestimmen Sie das Zentrum von $GL_n(k)$. \\

Es gilt: $Z(GL_n(k)) = \{ k \cdot E_n, k \neq 0 \}$. \\
\textbf{Beweis:}\\
"$\subseteq$": \\
Für $M = (m_{ij}) \in Z(GL_n(k))$ kommutiert die Multiplikation mit allen Matrizen, insbesondere mit den Elemtarmatrizen $E_{kl}$, was einer elementaren Zeilen- bzw. Spaltentransformation entspricht. Daher muss $M$ diagonalgestalt haben.\\
Für $M \cdot E_{kl} = E_{kl} \cdot M$ erhält man durch Koeffizientenvergleich, dass $m_{ii} = m_{jj}$ gelten muss, für alle $i,j$. $\Rightarrow M = c \cdot E_n, c \neq 0$. \\
"$\supseteq$":\\
Für $c \neq 0$ ist ersichtlich, dass $c E_n \cdot M = M \cdot c E_n,\; \forall M \in GL_n(k)$. 
\mbox{} \hfill $\square$

\item Geben Sie für $n \geq 1$ das Zentrum der symmetrischen Gruppe $S_n$ an. \\
Für $n \in \{1,2 \}$ ist $Z(S_n) = S_n$ (kann man sehr leicht nachrechnen, da wenige Elemente). \\
Für $n > 2$ ist $Z(S_n) = \{ id \}$: Sei $\sigma \in S_n, \sigma \neq id$.\\
$ \Rightarrow \exists k,l \leq n: \; \sigma(k) = l \neq k$. Da nun aber $n > 2 \Rightarrow \exists m:\, m\neq k, m\neq l$. \\
Dann gilt 
$$ ((k \; m) \cdot \sigma)(k) = l = \sigma(k) \neq \sigma(m) = (\sigma \cdot (k \; m))(k) $$
$$ \Rightarrow \sigma \notin Z(S_n) $$
$$ \Rightarrow Z(S_n) = \{ id \} $$
\mbox{} \hfill $\square$
\end{enumerate}
%% -------------------------------------
%%		AUFGABE 4
%% -------------------------------------
\subsection*{Aufgabe 4 \mdseries (Ein Färbungsproblem)}
\begin{enumerate}[\bfseries a)]
\item Welche Symmetriegruppe benutzt man sinnvollerweise, um verschiedene Ketten zu identifizieren?

Wir können eine sechsgliedrige Kette als regelmäßiges Sechseck auffassen, also bietet sich dessen Symmetriegruppe, also $D_6$ an.

\item Bestimmen Sie die Anzahl der wesentlich verschiedenen Ketten, die man herstellen kann.

Sei nun $G = D_6, \#G = 12$ und $M = \{f: \, \{1,2,3,4,5,6 \} \to \{1,2, \ldots , n \}\}, \#M = n^6$.\\
Zählen der Fixpunkte für die Elemente in $G$:
\begin{enumerate}[\bfseries i)]
\item $g = e$: $\#M^e = n^6$.
\item $g =$ Drehung der Ordnung 6: $\#M^g = n$ (davon gibt es zwei)
\item $g =$ Drehung der Ordnung 3: $\#M^g = n^2$ (davon gibt es zwei)
\item $g =$ Drehung der Ordnung 2: $\#M^g = n^3$ (davon gibt es eine)
\item $g =$ Spiegelung: $\#M^g = n^3$ (davon gibt es 6).
\end{enumerate}
Dann gilt nach Formel für die Anzahl $s$ der wesentlich verschiedenen Färbungen: \\
$$ s = \frac{1}{12} \cdot (n^6 + 7n^3 + 2n^2 + 2n)$$

\item Geben Sie die Anzahl der wesentlich verschiedenen Ketten an, die aus drei weißen und drei schwarzen Perlen bestehen.

Wie in b) bilden wir $M$ mit $n=2$. Nun wählen wir die Perlen aus, die weiß gefärbt werden, also haben wir 
$$ P = \{ N \subset M | \, \#N = 3 \}, \#P = \binom{6}{3} = 20 $$
verschiedene Färbungen dafür zur Verfügung.\\
Das ergibt dann 5 verschiedene Färbungen.
\end{enumerate}
\label{LastPage}


\end{document}