\documentclass[11pt,a4paper,ngerman]{article}
\usepackage[bottom=2.5cm,top=2.5cm]{geometry} 
\usepackage{babel}
\usepackage[utf8]{inputenc} 
\usepackage[T1]{fontenc} 
\usepackage{ae} 
\usepackage{amssymb} 
\usepackage{amsmath} 
\usepackage{graphicx}
\usepackage{fancyhdr}
\usepackage{fancyref}
\usepackage{listings}
\usepackage{xcolor}
\usepackage{paralist}
\usepackage{fancyhdr}
\usepackage{subfigure}
\pagestyle{fancy}
\fancyhead[C]{Alegbra und Zahlentheorie}
\fancyhead[L]{Übung Nr. 8}
\fancyhead[R]{WS 2011/12}
\fancyfoot{}
\fancyfoot[L]{}
\fancyfoot[C]{\thepage\, von \pageref{LastPage}}
\renewcommand{\footrulewidth}{0.5pt}
\renewcommand{\headrulewidth}{0.5pt}
\setlength{\parindent}{0pt} 
\setlength{\headheight}{15pt}

%
\newcommand{\N}{\mathbb{N}}
\newcommand{\NN}{\mathbb{N} \setminus \{0\}}
\newcommand{\Z}{\mathbb{Z}}
\newcommand{\C}{\mathbb{C}}
\newcommand{\R}{\mathbb{R}}
\newcommand{\Q}{\mathbb{Q}}
\newcommand{\ggT}{\text{ggT}}
\newcommand{\kgV}{\text{kgV}}
\newcommand{\sign}{\text{Sign}}
%

\author{Tutor: David Müßig}
\date{}
\title{Max Wisniewski, Alexander Steen}

\begin{document}

\lstset{language=Java, basicstyle=\ttfamily\fontsize{10pt}{10pt}\selectfont\upshape, commentstyle=\rmfamily\slshape, keywordstyle=\rmfamily\bfseries, breaklines=true, frame=single, xleftmargin=3mm, xrightmargin=3mm, tabsize=2}

\maketitle
\thispagestyle{fancy}

%% -------------------------------------------
%%                 AUFGABE 1
%% -------------------------------------------

\subsection*{Aufgabe 1 \mdseries (Vorzeichen und Ordnung eines Zykels)}
\begin{enumerate}[\bfseries a)]
%% ------------------a) -----------------------
\item Es sei $c \in S_n$ ein Zykel der Länge $k$. Berechnen Sie das Vorzeichen $\sign(c)$. \\

Sei $c = (c_1 \ldots c_k) \in S_n$, $k \in \N,k > 1$. Dann gilt nach VL:
$$ c = (c_1 \ldots c_k) = (c_1 \; c_k) \cdot (c_1 \; c_{k-1}) \cdot \ldots \cdot (c_1 \; c_2)$$
Also wird das Zykel $c$ durch $k-1$ Transpositionen erzeugt. Da für jede Transposition $\tau \in S_n$ gilt: $\sign(\tau) = -1$ und ferner
$$ \sign((c_1 \; c_k) \cdot (c_1 \; c_{k-1}) \cdot \ldots \cdot (c_1 \; c_2)) = \sign(c_1 \; c_k) \cdot \sign(c_1 \; c_{k-1}) \cdot \ldots \cdot \sign(c_1 \; c_2)$$
gilt, folgt:
$$ \sign(c) = (-1)^{k-1} = \begin{cases}
  1,  & \text{falls }k\text{ ungerade,}\\
  -1, & \text{sonst.}
\end{cases} $$
%% ------------------b) -----------------------
\item  Welche Ordnung hat ein Zykel $c \in S_n$ der Länge $k$? \\

Sei $c = (c_1 \ldots c_k) \in S_n$, $k \in \N,k > 1$.
Dann ist die Ordnung k. So! Kein Bock mehr!
%% ------------------c) -----------------------
\item Es seien $\sigma \in S_n$ eine Permutation und $n=(n_1,...,n_s)$ ihr Zykeltyp. Welche Ordnung hat $\sigma$? \\
kgv(n1, n2, ..., ns) bitches!
\end{enumerate}



%% -------------------------------------------
%%		AUFGABE 2
%% ------------------------------------------

\subsection*{Aufgabe 2 \mdseries (Links- vs. Rechtswirkungen)}

\mbox{} \hfill $\square$\\


%% -------------------------------------------
%%		AUFGABE 3
%% --------------------------------------------
\subsection*{Aufgabe 3 \mdseries (Das Zentrum)}


%% -------------------------------------
%%		AUFGABE 4
%% -------------------------------------
\subsection*{Aufgabe 4 \mdseries (Ein Färbungsproblem)}


\label{LastPage}

\end{document}