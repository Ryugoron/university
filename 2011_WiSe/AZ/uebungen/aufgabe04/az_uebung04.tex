\documentclass[11pt,a4paper,ngerman]{article}
\usepackage[bottom=2.5cm,top=2.5cm]{geometry} 
\usepackage{babel}
\usepackage[utf8]{inputenc} 
\usepackage[T1]{fontenc} 
\usepackage{ae} 
\usepackage{amssymb} 
\usepackage{amsmath} 
\usepackage{graphicx}
\usepackage{fancyhdr}
\usepackage{fancyref}
\usepackage{listings}
\usepackage{xcolor}
\usepackage{paralist}
\usepackage{fancyhdr}
\usepackage{subfigure}
\pagestyle{fancy}
\fancyhead[C]{Alegbra und Zahlentheorie}
\fancyhead[L]{Übung Nr. 3}
\fancyhead[R]{WS 2011/12}
\fancyfoot{}
\fancyfoot[L]{}
\fancyfoot[C]{\thepage\, von \pageref{LastPage}}
\renewcommand{\footrulewidth}{0.5pt}
\renewcommand{\headrulewidth}{0.5pt}
\setlength{\parindent}{0pt} 
\setlength{\headheight}{15pt}

%
\newcommand{\N}{\mathbb{N}}
\newcommand{\NN}{\mathbb{N} \setminus \{0\}}
\newcommand{\Z}{\mathbb{Z}}
\newcommand{\R}{\mathbb{R}}
\newcommand{\ggT}{\text{ggT}}
\newcommand{\kgV}{\text{kgV}}
%

\author{Tutor: David Müßig}
\date{}
\title{Max Wisniewski, Alexander Steen}

\begin{document}

\lstset{language=Java, basicstyle=\ttfamily\fontsize{10pt}{10pt}\selectfont\upshape, commentstyle=\rmfamily\slshape, keywordstyle=\rmfamily\bfseries, breaklines=true, frame=single, xleftmargin=3mm, xrightmargin=3mm, tabsize=2}

\maketitle
\thispagestyle{fancy}

%% -------------------------------------------
%%                 AUFGABE 1
%% -------------------------------------------
\subsection*{Aufgabe 1 \mdseries Die Tetraedergruppe}

Gegeben seien folgende Punkte $\mathbb{R}^3:$
$$
P=(1,1,1),\; Q=(-1,-1,1),\; R=(1,-1,-1),\; S=(-1,1,-1)
$$
\begin{enumerate}[\bfseries (a)]

%% --------------------------------------------
%%			a)
%% -------------------------------------------
\item Berechnen Sie die Abstände $d(P,Q), d(P,R), d(P,S), d(Q,R), d(Q,S)$ und $d(R,S)$. Schließen Sie, dass die angegebenen Punkte die Eckpunkte eines regulären Tetraeders $T$ sind.\\

\textbf{Lösung:}

\begin{description}

\item{$d(P,Q)$} $=\sqrt{2^2 + 2^2 + 0} = \sqrt{8}$

\item{$ d(P,R)$} $= \sqrt{0 + 2^2 + 2^2} = \sqrt{8}$

\item{$d(P,S)$} $= \sqrt{2^2 + 0 + 2^2} = \sqrt{8}$

\item{$d(Q,R)$} $= \sqrt{2^2 + 0 + 2^2} = \sqrt{8}$

\item{$ d(Q,S)$} $=\sqrt{0 + 2^2 + 2^2} = \sqrt{8}$

\item{$d(R,S)$} $=\sqrt{2^2 + 2^2 + 0} = \sqrt{8}$

\end{description}

Die von $P,Q,R,S$ definierte Form hat 4 Punkte, die paarweise den selben Abstand ($\geq 0+ \varepsilon, \varepsilon > 0$) haben, darum erfüllen Sie die Bedingung, dass alle paarweise verscheiden sind. Daraus ergibt sich, dass die Oberfläche aus 4 gleichseitigen Dreiecken besteht. Deshalb handelt es sich hier bei um ein \emph{regelmäßiges} Tetraeder.

%% ---------------------------------------------
%% 			b)
%% --------------------------------------------

\item Wählen Sie eine Ecke $E \in \{  P,Q,R,S\} $. Dann sei $D$ die Achse durch $E$ und den Mittelpunkt der gegenüberliegenden Seite des Tetraeders $T$. Geben Sie die Abbildungsmatrix für die Drehung um $D$ um den Winkel $120^\circ$ und $240^\circ$ bzgl. der Standardbasis des $\R ^3$ an.\\

\textbf{Lösung:}\\
Wir gehen nach dem folgenden Prinzip vor. Wir wählen unsere Basis so, dass die Drehachse unsere z-Achse darstellt und Drehen nur in der x-y-Ebene. Danach führen wir einen Basiswechsel durch um in die Standardbasis zu kommen.\\

Wenn wir Berechnen zunächst $M_1$ die Drehung um $120^\circ$.\\
$$
M_1 \left( \begin{array}{c} 1 \\ 0 \\ 0 \end{array} \right) = \left( \begin{array}{c} \cos 120^\circ \\ \sin 120^\circ \\ 0 \end{array} \right), \; M_1 \left( \begin{array}{c} 0 \\ 1 \\ 0 \end{array} \right) = \left( \begin{array}{c} \sin 120^\circ \\ \cos 120^\circ \\ 0 \end{array} \right), \; M_1 \left( \begin{array}{c} 0 \\ 0 \\ 1 \end{array} \right) = \left( \begin{array}{c} 0 \\ 0 \\ 1 \end{array} \right)
$$
$$
\Rightarrow M_1 = \left(
\begin{array}{ccc}
\cos 120^\circ & \sin 120^\circ & 0 \\
\sin 120^\circ & \cos 120^\circ & 0 \\
0 & 0 & 1
\end{array} \right)
$$

Wenn wir Berechnen zunächst $M_2$ die Drehung um $240^\circ$.\\
$$
M_2 \left( \begin{array}{c} 1 \\ 0 \\ 0 \end{array} \right) = \left( \begin{array}{c} \cos 240^\circ \\ \sin 240^\circ \\ 0 \end{array} \right), \; M_2 \left( \begin{array}{c} 0 \\ 1 \\ 0 \end{array} \right) = \left( \begin{array}{c} \sin 240^\circ \\ \cos 240^\circ \\ 0 \end{array} \right), \; M_2 \left( \begin{array}{c} 0 \\ 0 \\ 1 \end{array} \right) = \left( \begin{array}{c} 0 \\ 0 \\ 1 \end{array} \right)
$$
$$
\Rightarrow M_2 = \left(
\begin{array}{ccc}
\cos 240^\circ & \sin 240^\circ & 0 \\
\sin 240^\circ & \cos 240^\circ & 0 \\
0 & 0 & 1
\end{array} \right)
$$

Nun müssen wir nur noch die Basiswechselmatrix $S$ und die Inversematrix $S^{-1}$ berechnen. (Diese existiert nach Satz aus Lina I).\\

Wählen wir zunächst eine Achse. Wir nehmen $E = P$. Da sich alle Punkte auf den Einheitskreis um den Nullpunkt bewegen, muss der Schwerpunkte des Tetraeders auch der Nullpunkt $( 0 , 0 ,0 )$ sein. Damit geht jede Drehachse durch diesen, also auch die Achse die auf den Mittelpunkt der gegenüberliegenden Seite geht.\\

Damit haben wir einen Vektor unseres neuen Vektorraumes bekommen. Nach dem Basiswechselsatz, kann ich mir nun aus der Standardbasis eine neue Basis wählen mit $B' = \left\{ \frac{1}{\sqrt{3}}(1,1,1), \frac{1}{\sqrt{2}}(1,0,-1), \frac{1}{\sqrt{6}}(1,-2,1) \right\}$. Dabei handelt es sich an dieser Stelle schon um eine Orthonormalbasis, da wir in der Drehung vorhin davon ausgagangen sind, dass es eine ist. (Eigenschaften dürfen gerne selbst nachgerechnet werden)\\

Damit ergeben sich:
$$
S (1,0,0) = (\frac{1}{\sqrt{3}} , \frac{1}{\sqrt{3}} , \frac{1}{\sqrt{3}}) , \; S (0,1,0) = (\frac{1}{\sqrt{2}},0,- \frac{1}{\sqrt{2}}), \; S (0,0,1) = (\frac{1}{\sqrt{6}}, - \frac{2}{\sqrt{6}}, \frac{1}{\sqrt{6}})
$$
$$
\Rightarrow S = \left( 
\begin{array}{ccc}
\frac{1}{\sqrt{3}} &  \frac{1}{\sqrt{2}} & \frac{1}{\sqrt{6}}\\
\frac{1}{\sqrt{3}} & 0 & - \frac{2}{\sqrt{6}} \\
\frac{1}{\sqrt{3}} & - \frac{1}{\sqrt{2}} & \frac{1}{\sqrt{6}})
\end{array}
\right)
$$
Daraus können wir $S^{-1}$ berechnen:
$$
\Rightarrow S^{-1} = \left(
\begin{array}{ccc}
0.57735 & 0.57735 & 0.57735\\
0.707107 & 0 & - 0.707107\\
0.408248 & -0.816497 & 0.408248 
\end{array}
\right)
$$

Als Ergebnis bekommen wir nun die folgenden Matrizen:

$$D_{120} = S^{-1} M_1 S \qquad \text{und} \qquad D_{240} = S^{-1} M_2 S$$

%% --------------------------------------------
%%			c)
%% --------------------------------------------
\item Wählen Sie zwei gegenüberliegende Kanten von $T$. Dann sei $D$ die Achse durch die Mittelpunkte dieser beiden Kanten. Geben Sie die Abbildungsmatrix für die Drehung um $D$ um den Winkel $180^\circ$ bzgl. der Standardbasis des $\R ^3$ an.

\pagebreak

\textbf{Lösung:}

Nehmen wir wieder an, dass die z-Achse, die Drehachse ist. Die Drehung sieht nun folgender Maßen aus:
$$
M = \left(
\begin{array}{ccc}
\cos 180^\circ & \sin 180^\circ & 0 \\
\sin 180^\circ & \cos 180^\circ & 0 \\
0 & 0 & 1
\end{array}
\right) = \left(
\begin{array}{ccc}
-1 & 0 & 0 \\
0 & -1 & 0 \\
0 & 0 & 1
\end{array}
\right)
$$

Nun müssten wir wiederum die Basiswechsel $S$ und $S^{-1}$ in die Orthonormalbasis der Drehachse machen. Aber auf die Punkte wollen wir an dieser Stelle verzichten.

\end{enumerate}

%%--------------------------------------------------
%%		AUFGABE 2
%% -------------------------------------------------
\subsection*{Aufgabe 2 \mdseries Würfel, Okta- und Dodekaeder}

\begin{enumerate}[\bfseries (a)]

%% ----------------------------------------
%%			a)
%% ---------------------------------------
\item Es sei $W \subset \R ^3$ der Würfel mit den Eckpunkten

$$
(1,1,\pm 1), \; (-1,1, \pm 1), \; (-1, -1, \pm 1), \; (1, -1, \pm 1)
$$
Bestimmen Sie die spezielle Symmetriegruppe $SO(W)$\\


\textbf{Lösung:}\\
Als Drehungen haben wir die folgenden Möglichkeiten:
\begin{itemize}
\item Identität
\item Achse durch die Mittelpunkte 2er gegenüberliegender Seiten\\
3 Achsen mit Drehungen um $90^\circ, \; 180^\circ , \; 270^\circ$
\item Achse durch zwei gegenüberliegende Eckpunkte\\
4 Achsen mit Drehungen um $120^\circ , 240^\circ$
\item Achse durch die Mittelpunkte zweier gegenüberliegender Kanten\\
6 Achsen mit Drehungen um $180^\circ$
\end{itemize}

Macht insgesammt $\# SO(W) = 24$.

%% -----------------------------------------
%%			b)
%% -----------------------------------------
\item Die Punkte
$$
(\pm 1, 0 , 0), \; (0, \pm 1, 0), \; (0,0,\pm 1)
$$
sind Eckpunkte eines Polyeders $O \subset \R ^ 3$. Skizzieren Sie $O$. Bestimmen Sie die spezielle Symmeriegruppe $SO(O)$. Vergleichen Sie $SO$(O) und $SO(W)$. Was stellen Sie fest? Haben Sie eine Erklärung dafür?\\

\textbf{Skizze:}

\pagebreak

\textbf{Lösung:}\\
Als Drehungen haben wir die folgenden Möglichkeiten:
\begin{itemize}
\item Identität
\item Achse durch die Mittelpunkte 2er gegenüberliegender Seiten\\
4 Achsen mit Drehungen um $120^\circ , \; 240^\circ$
\item Achse durch zwei gegenüberliegende Eckpunkte\\
3 Achsen mit Drehungen um $90^\circ , \; 180^\circ , \; 270^\circ$
\item Achse durch die Mittelpunkte zweier gegenüberliegender Kanten\\
6 Achsen mit Drehungen um $180^\circ$
\end{itemize}

Das Macht insgesammt $\# SO(O) = 24$. Wir sehen also, dass die beiden Symmetriegruppen die selben Anzahlen von Elementen haben. Die Drehungen teilen sich dabei auch noch gleich auf die selbe Anzahl von korrespondierenden Achsen auf. Damit sind die beiden Gruppen isomorph zu einander.

%% -------------------------------------------
%%			c)
%% -----------------------------------------
\item Ein Dodekaeder $D \subset \R ^3$ ist ein reguläres Polyeder, das von $12$ regelmäßigen Fünfecken begrenzt wird. Beschreiben Sie die spezielle Symmetriegruppe $SO(D)$\\

\textbf{Lösung:}\\
Das ganze verhält sich analog zu den Köpern aus \emph{a)} und \emph{b)}.

Als Drehungen haben wir die folgenden Möglichkeiten:
\begin{itemize}
\item Identität
\item Achse durch die Mittelpunkte 2er gegenüberliegender Seiten\\
6 Achsen mit Drehungen um $60^\circ , \; 120^\circ , \; 180^\circ , \; 240^\circ$
\item Achse durch zwei gegenüberliegende Eckpunkte\\
10 Achsen mit Drehungen um $120^\circ , \; 240^\circ $
\item Achse durch die Mittelpunkte zweier gegenüberliegender Kanten\\
15 Achsen mit Drehungen um $180^\circ$
\end{itemize}

Das Macht insgesammt $\# SO(D) = 60$. 

%% -----------------------------------------
%%			d)
%% ----------------------------------------
\item Recherchieren Sie den Begriff "platonischer Körper" {} und dokumentieren Sie Ihre Ergebnisse.\\

Platonische Körper sind dreidimensionale Polyeder, also Körper, die von Ebenen begrenzt werden. Das besondere an Platonischen Körpern ist ihre ''hohe Symmetrie'' und Gleichmäßigkeit. Sie werden alle aus jeweils gleichgroßen, regelmäßigen Flächen zusammengesetzt und besitzen viele Symmetrieeigenschaften:\\
So kann sowohl jeder Eckpunkt, jede Seitenfläche als auch jede Kante auf die jeweiligen anderen gedreht werden. Bereits in Euklids Werk ''Die Elemente'' wurde gezeigt, dass es nur fünf dieser speziellen Polyeder gibt.
\\Mögliche Platonische Körper sind der Tetraeder (vier gleichseitige Dreiecke), der Würfel (sechs Quadrate), der Oktaeder (acht gleichseitige Dreiecke), der Dodekaeder (12 regelmäßige Fünfecke) und der Ikosaeder (20 gleichseitige Dreiecke).
\\Diese Körper sind schon seit der Antike bekannt, ihre Symmetriegruppen treten an einigen Stellen der Mathematik wieder auf. Auch in der Natur treten einige Regelmäßigkeiten in Form von Platonischen Körpern auf (Kristalle von Mineralien, Struktur einiger Kohlenwasserstoffe.
\end{enumerate}

%% ------------------------------------------
%%		AUFGABE 3
%% -----------------------------------------
\subsection*{Aufgabe 3 \mdseries Solitaire}

\begin{enumerate}[\bfseries (a)]

%% ---------------------------------------------
%%			a)
%% --------------------------------------------
\item Bestimmen Sie alle Endpositionen des des Solitairspiels mit genau zwei Steinen.\\

\textbf{Lösung:}\\
Wir nehmen die Konstruktion des Spielfeldes aus der Vorlesung, die gefüllt mit den Elementen aus der \emph{Kleinschen Vierergruppe} folgendermaßen aussah:\\

\begin{center}
\begin{tabular}{|cc|ccc|cc|}
\hline
&&a&b&c&&\\
&&b&c&a&&\\
\hline
a&b&c&a&b&c&a\\
b&c&a&b&c&a&b\\
c&a&b&c&a&b&c\\
\hline
&&c&a&b&&\\
&&a&b&c&&\\
\hline
\end{tabular}
\end{center}

Nun wissen wir aus den Überlegungen aus der Vorlesung, dass die Kombination der Steine auf dem Spielfeld immer den selben Wert ergeben muss und unsere Startposition ist \emph{b}.\\
Da in der Endposition nur 2 Steine verbleiben, müssen diese Steine auf den Feldern $a$ und $c$ liegen, da allein diese Kombination aus 2 Elementen ein $b$ ergibt. (Das neutrale Element ist nicht Teil unseres Spielfeldes, kommt hierbei also nicht in betracht)\\

Die nächste Überlegung aus der Vorlesung, die wir hier anwenden können, ist das Symmetrieargument. Da wir das Spiel gedreht um den Mittelpunkt um $90^\circ , \; 180^\circ , \; 270^\circ $ und $360^\circ$ führen können und dabei auf das selbe kommen müssten, können nur Endpositionen in Frage kommen, die auch bei Drehung und Spiegelung ihren Wert erhalten. Dabei kann die Reihenfolge der Elemente bei der Projektion vertauscht werden. Aber dies ist zu vernachlässigen, da die Gruppe abelsch ist, wir also die Elemente beliebig tauschen können.\\

Als letztes gilt noch, dass die Felder sich für eine Endposition nicht gegenseitig auslöschen dürfen können, d.h. nicht neben einander liegen können.\\

Daraus ergeben sich die folgenden Endkonfigurationen (Drehungen (+3 Konfigurationen) und Spiegelungen (+4 Konfigurationen nach Drehung), werden nicht extra Aufgeführt):\\

\begin{center}
\begin{tabular}{ccc}
\begin{tabular}{|cc|ccc|cc|}
\hline
&&&&&&\\
&&&&&&\\
\hline
&&&&&&\\
&&&&&&\\
&&&&&&c\\
\hline
&&&a&&&\\
&&&&&&\\
\hline
\end{tabular} &
\begin{tabular}{|cc|ccc|cc|}
\hline
&&&&&&\\
&&&&&&\\
\hline
&&&&&&a\\
&&&&&&\\
&&&c&&&\\
\hline
&&&&&&\\
&&&&&&\\
\hline
\end{tabular}&
\begin{tabular}{|cc|ccc|cc|}
\hline
&&&&c&&\\
&&&&&&\\
\hline
&&&&&&\\
&&&&&&\\
&&&&&&\\
\hline
&&&&&&\\
&&a&&&&\\
\hline
\end{tabular}
\end{tabular}
\end{center}

Das ergibt für uns insgesammt diese Konfigurationen mulipliziert mit der Anzahl von Drehungen und Spiegelungen $= 12$ Konfigurationen.

%% ---------------------------------------------
%%			b)
%% ---------------------------------------------
\item Führen Sie die Diskussion aus der Vorlesung für das Spielfeld (siehe Zettel) durch.\\

\textbf{Lösung:}\\
Für die erste Betrachtung haben wir schon einmal die Elemente der \emph{Kleinschen Gruppe} in das Spielfeld eingezeichnet.

\begin{center}
\begin{tabular}{|ccc|ccc|ccc|}
\hline
&&&c&a&b&&&\\
&&&a&b&c&&&\\
&&&b&c&a&&&\\
\hline
c&a&b&c&a&b&c&a&b\\
a&b&c&a&b&c&a&b&c\\
b&c&a&b&c&a&b&c&a\\
\hline
&&&c&a&b&&&\\
&&&a&b&c&&&\\
&&&b&c&a&&&\\
\hline
\end{tabular}
\end{center}

Das Feld haben wir nach dem selben Muster, wie das ursprüngliche Feld, aus diesem erweitert. Insbesondere sind nur die äußersten 3 Felder pro äußerem Quadrat dazu gekommen. Diese ergeben aufaddiert (da wir in einer abelschen Gruppe die Elemente beliebig tauschen können) immer $e$, das neutrale Element.\\

Unsere Startkonfiguration hat also die Summe $b$. Wir suchen also wiederum die Felder, die ein $b$ enthalten und unter Drehung und Spiegelung abgeschlossen sind.\\

Dies sind die selben Felder, die schon in der Diskusion aus der Vorlesung hatten. Daraus ergibt sich, dass es 5 Endkonfigurationen mit einem Stein gibt und eine, die sogar auf der Mitte endet.

\end{enumerate}

\label{LastPage}
\end{document}