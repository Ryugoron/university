\documentclass[11pt,a4paper,ngerman]{article}
\usepackage[bottom=2.5cm,top=2.5cm]{geometry} 
\usepackage{babel}
\usepackage[utf8]{inputenc} 
\usepackage[T1]{fontenc} 
\usepackage{ae} 
\usepackage{amssymb} 
\usepackage{amsmath} 
\usepackage{graphicx}
\usepackage{fancyhdr}
\usepackage{fancyref}
\usepackage{listings}
\usepackage{xcolor}
\usepackage{paralist}
\usepackage{fancyhdr}
\usepackage{subfigure}
\pagestyle{fancy}
\fancyhead[C]{Alegbra und Zahlentheorie}
\fancyhead[L]{Übung Nr. 7}
\fancyhead[R]{WS 2011/12}
\fancyfoot{}
\fancyfoot[L]{}
\fancyfoot[C]{\thepage\, von \pageref{LastPage}}
\renewcommand{\footrulewidth}{0.5pt}
\renewcommand{\headrulewidth}{0.5pt}
\setlength{\parindent}{0pt} 
\setlength{\headheight}{15pt}

%
\newcommand{\N}{\mathbb{N}}
\newcommand{\NN}{\mathbb{N} \setminus \{0\}}
\newcommand{\Z}{\mathbb{Z}}
\newcommand{\C}{\mathbb{C}}
\newcommand{\R}{\mathbb{R}}
\newcommand{\Q}{\mathbb{Q}}
\newcommand{\ggT}{\text{ggT}}
\newcommand{\kgV}{\text{kgV}}
%

\author{Tutor: David Müßig}
\date{}
\title{Max Wisniewski, Alexander Steen}

\begin{document}

\lstset{language=Java, basicstyle=\ttfamily\fontsize{10pt}{10pt}\selectfont\upshape, commentstyle=\rmfamily\slshape, keywordstyle=\rmfamily\bfseries, breaklines=true, frame=single, xleftmargin=3mm, xrightmargin=3mm, tabsize=2}

\maketitle
\thispagestyle{fancy}

%% -------------------------------------------
%%                 AUFGABE 1
%% -------------------------------------------

\subsection*{Aufgabe 1 \mdseries (Die entgegengesetzte Gruppe)}
Es seien $(G, \cdot)$ eine Gruppe und $\star : G \times G, (g,h) \mapsto h \cdot g$. \\
Zeigen Sie, dass $G^{op} := (G, \star)$ eine Gruppe ist. \\
Es sei $e$ das neutrale Element von $G$, $e_op$ das neutrale Element von $G^{op}$. $a^{-1}$ beschreibt das inverse Element von $a$ bzgl. $G$, $a_{op}^{-1}$ das inverse Element von $a$ bzgl. $G^{op}$, falls es existiert.\\ \\
\textbf{Beweis:}\\
(i) $\exists e_{op} \in G^{op}$ \\
Es gilt $e_{op} = e$, da $\forall g\in G: e \star g = g \cdot e = g$ und $e \in G^{op}$ gilt nach Konstruktion. 
\\ \\
(ii) $\forall a,b,c \in G^{op}:\,(a \star b) \star c = a \star (b \star c)$ \\
Es seien $a,b,c \in G^{op}$. Dann gilt: \\
$$ (a \star b) \star c = (b \cdot a) \star c = c \cdot (b \cdot a) $$
$$ = (c \cdot b) \cdot a = a \star (c \cdot b) = a \star (b \star c)$$
\\
(iii) $\forall a \in G^{op}: \, a_{op}^{-1} \in G^{op}$ \\
Es gilt: $a_{op}^{-1} = a^{-1} \; \forall a \in G_{op}$ \\
Beweis: 
Sei $a \in G_{op}$.
$$ a \star a^{-1} = a^{-1} \cdot a = e$$
$\forall a \in G_{op}: \; a^{-1} \in G_{op}$ nach Konstruktion.
\\ \\
Rechtsinverse sind auch Linksinverse, analog für neutrale Elemente (nach VL und vorige Zettel).
\mbox{} \hfill $\square$\\


%% -------------------------------------------
%%		AUFGABE 2
%% ------------------------------------------

\subsection*{Aufgabe 2 \mdseries (Wirkungen von $\R$ auf $\C$)}

\begin{enumerate}[\bfseries a)]
\item Zeigen Sie, dass durch $\sigma:\;\R \times \C \to \C, (t, z) \mapsto \exp(t) \cdot z$
eine Linkswirkung von $\R$ auf $\C$ gegeben ist. Geben Sie für jede komplexe Zahl $z$ die
Bahn $\R \cdot z$ und die Standgruppe $\R_z$ an. Fertigen Sie eine Skizze der Bahnen an.
\begin{enumerate}[\bfseries (1)]
\item $\sigma$ ist Linkswirkung. \\
\textbf{Beweis:}\\
Fasse $\R$ als additive Gruppe auf. \\ \\
(i) $\forall z \in \C:\; \sigma(0,z) = z$ \\
Sei $z \in \C$.
$$ \sigma(0,z) = \exp(0) \cdot z = 1 \cdot z = z $$
\newpage
(ii) $\forall x_1, x_2 \in \R, \forall z \in \C: \; \sigma(x_1, \sigma(x_2,z)) = \sigma (x_1 + x_2, z)$ \\
Sei $ x_1, x_2 \in \R, z \in \C$.
$$ \sigma(x_1, \sigma(x_2,z)) = \sigma(x_1, \exp(x_2) \cdot z) $$
$$ \exp(x_1) \cdot (\exp(x_2) \cdot z) = (\exp(x_1) \cdot \exp(x_2)) \cdot z $$
$$ \exp(x_1 + x_2) \cdot z = \sigma(x_1 + x_2, z)$$
\mbox{} \hfill $\square$\\

\item Bahnen \\
Sei $z \in \C$. \\
$ \R \cdot z = \{ \exp(r) \cdot z | r \in \R \} = \{\rho \cdot z | \rho > 0 \}$, also alle reellen Vielfachen von $z$ (''Streckung'' von $z$).
\item Standgruppen
Sei $z \in \C$. \\
$\R_z = \{r \in \R | \exp(r) \cdot z = z \} = \left\{
\begin{array}{cc}
\R &, \text{falls } z = 0 \\
\{0\} & \text{sonst}
\end{array}
 \right.$
\item Skizze
\vspace{7cm}

\end{enumerate}

\item Zeigen Sie, dass durch $\sigma:\;\R \times \C \to \C, (t, z) \mapsto exp(i \cdot t) \cdot z$
eine Linkswirkung von $\R$ auf $\C$ gegeben ist. Geben Sie für jede komplexe Zahl $z$ die
Bahn $\R \cdot z$ und die Standgruppe $\R_z$ an. Fertigen Sie eine Skizze der Bahnen an.
\begin{enumerate}[\bfseries (1)]
\item $\sigma$ ist Linkswirkung. \\
\textbf{Beweis:}\\
Fasse $\R$ als additive Gruppe auf. \\ \\
(i) $\forall z \in \C:\; \sigma(0,z) = z$ \\
Sei $z \in \C$.
$$ \sigma(0,z) = \exp(i \cdot 0) \cdot z = \exp(0) \cdot z = 1 \cdot z = z $$
(ii) $\forall x_1, x_2 \in \R, \forall z \in \C: \; \sigma(x_1, \sigma(x_2,z)) = \sigma (x_1 + x_2, z)$ \\
Sei $ x_1, x_2 \in \R, z \in \C$.
$$ \sigma(x_1, \sigma(x_2,z)) = \sigma(x_1, \exp(i\cdot x_2) \cdot z) $$
$$ \exp(i \cdot x_1) \cdot (\exp(i \cdot x_2) \cdot z) = (\exp(i \cdot x_1) \cdot \exp(i \cdot x_2)) \cdot z $$
$$ \exp(i\cdot x_1 + i \cdot x_2) \cdot z = \exp(i \cdot (x_1 + x_2)) \cdot z = \sigma(x_1+x_2,z)$$
\mbox{} \hfill $\square$\\
\newpage
\item Bahnen \\
Sei $z \in \C$. \\
$ \R \cdot z = \{ \exp(i \cdot r) \cdot z | r \in \R \}$, also alle 'Drehungen' von $z$ (falls man $z$ als Vektor mit Länge $|z|$ interpretiert).
\item Standgruppen
Sei $z \in \C$. \\
$\R_z = \{r \in \R | \exp(i \cdot r) \cdot z = z \} = \left\{
\begin{array}{cc}
\R &, \text{falls } z = 0 \\
\{2 \pi k\,|\,k \in \Z\} & \text{sonst}
\end{array}
 \right.$
\item Skizze
\vspace{7cm}
\end{enumerate}
\end{enumerate}



%% -------------------------------------------
%%		AUFGABE 3
%% --------------------------------------------
\subsection*{Aufgabe 3 \mdseries (Wirkung einer Untergruppe von $S_8$)}
Es sei $\sigma := (1\;2\;3) (5\;6\;7\;8) \in S_8$.
\begin{enumerate}[\bfseries a)]
\item Bestimmen Sie die Ordnung von $\sigma$. \\
Den ersten Zykel müsste man ein Vielfaches von Drei mal ausführen, den zweiten ein Vielfaches von Vier mal ausführen. Damit nun  beiden gerecht wird, entspricht dies dem kGV. \\
Die Ordnung entspricht also $\kgV(3,4) \stackrel{\ggT(3,4)=1}{=} 3 \cdot 4 = 12$. 
\item Es seien $G := <\sigma> \subseteq S_8$ und
$$ \Gamma:\; G \times \{1,2,3,4,5,6,7,8\} \to \{1,2,3,4,5,6,7,8\} $$
$$(\sigma^i, j) \mapsto s^i(j).$$
Geben Sie für jedes der Elemente 1, 4 und 5 seine $G$-Bahn und seine Standgruppe in $G$ an. \\
$$ \sigma \cdot 1 = \{\sigma^i(1)\} =  \{1,2,3 \}$$
$$ \sigma_1 = \{\sigma^i \in G \, | \, \sigma^i(1) = 1 \} = \{ \sigma^{3k}\,|\,k \in \Z\}$$
\\
$$ \sigma \cdot 4 = \{\sigma^i(4)\} =  \{4\} $$
$$ \sigma_1 = \{\sigma^i \in G \, | \, \sigma^i(3) = 4 \} = G $$
\\
$$ \sigma \cdot 5 = \{\sigma^i(5)\} =  \{5,6,7,8 \} $$
$$ \sigma_1 = \{\sigma^i \in G \, | \, \sigma^i(5) = 5 \} = \{ \sigma^{4k}\,|\,k \in \Z\} $$
\end{enumerate}

%% -------------------------------------
%%		AUFGABE 4
%% -------------------------------------
\subsection*{Aufgabe 4 \mdseries (Zyklische Gruppen)}
Es seien $m, n > 0$ positive ganze Zahlen, so dass $\ggT(m,n) = 1$. \\
Zeigen Sie, dass das Element $(1,1) \in \Z_m \times \Z_n$ Ordnung $m\cdot n$ hat und folgern Sie
$$ \Z_m \times \Z_n \cong \Z_{m\cdot n} $$
\textbf{Lösung:}\\
Offenbar gilt $(1,1)^{m \cdot n} = (1,1)$. \\
Da das erste Element in $m$ zyklisch und das zweite Element in $n$ zyklisch ist, muss ein gemeinsames Vielfaches von $m$ und $n$ die Ordnung sein. Da nun $\ggT(m,n) = 1 \Rightarrow \kgV(m,n) = m \cdot n$. \\
$ \Rightarrow ord((1,1)) = m\cdot n $.
\\ \\
Sei nun o.B.d.A. $m > n$. Dann ist
$$ \varphi : \, \Z_m \times \Z_n \to \Z_{m\cdot n} $$
$$ (i,j) \mapsto i \cdot n + j $$
ein Gruppenisomorphismus.
\\
\textbf{Beweis:}\\
Sei $x=(a,b),y=(c,d) \in \Z_m \times \Z_n$. \\
$ \varphi (x + y) = \varphi (a+c,b+d) = (a+c) \cdot n + (b+d) $ \\
$ = a \cdot n + c \cdot n + b + d = a \cdot n + b + c \cdot n + d$ \\
$ = \varphi(a,b) + \varphi(c,d)$ \\
\\
$\varphi$ ist wohldefiniert, da wir minimal $\varphi(0,0) = 0$ und maximal $\varphi(m-1,n-1) = (m-1) \cdot n + (n-1)$ $ = m \cdot n - 1 \in \Z_{m\cdot n}$ haben. \\
$\varphi^{-1}$ existiert, mit
$$ \varphi^{-1} : \, \Z_{m\cdot n} \to \Z_m \times \Z_n$$
$$ j \mapsto (a,b) $$
mit $j = a \cdot n + b$ (ganzzahlige Division). \\
Da die Umkehrfunktion von $\varphi$ existiert, ist $\varphi$ bijektiv.
\mbox{} \hfill $\square$\\

\label{LastPage}

\end{document}