\documentclass[11pt,a4paper,ngerman]{article}
\usepackage[bottom=2.5cm,top=2.5cm]{geometry} 
\usepackage{babel}
\usepackage[utf8]{inputenc} 
\usepackage[T1]{fontenc} 
\usepackage{ae} 
\usepackage{amssymb} 
\usepackage{amsmath} 
\usepackage{graphicx}
\usepackage{fancyhdr}
\usepackage{fancyref}
\usepackage{listings}
\usepackage{xcolor}
\usepackage{paralist}
\usepackage{fancyhdr}
\usepackage{subfigure}
\pagestyle{fancy}
\fancyhead[C]{Alegbra und Zahlentheorie}
\fancyhead[L]{Übung Nr. 7}
\fancyhead[R]{WS 2011/12}
\fancyfoot{}
\fancyfoot[L]{}
\fancyfoot[C]{\thepage\, von \pageref{LastPage}}
\renewcommand{\footrulewidth}{0.5pt}
\renewcommand{\headrulewidth}{0.5pt}
\setlength{\parindent}{0pt} 
\setlength{\headheight}{15pt}

%
\newcommand{\N}{\mathbb{N}}
\newcommand{\NN}{\mathbb{N} \setminus \{0\}}
\newcommand{\Z}{\mathbb{Z}}
\newcommand{\R}{\mathbb{R}}
\newcommand{\Q}{\mathbb{Q}}
\newcommand{\ggT}{\text{ggT}}
\newcommand{\kgV}{\text{kgV}}
%

\author{Tutor: David Müßig}
\date{}
\title{Max Wisniewski, Alexander Steen}

\begin{document}

\lstset{language=Java, basicstyle=\ttfamily\fontsize{10pt}{10pt}\selectfont\upshape, commentstyle=\rmfamily\slshape, keywordstyle=\rmfamily\bfseries, breaklines=true, frame=single, xleftmargin=3mm, xrightmargin=3mm, tabsize=2}

\maketitle
\thispagestyle{fancy}

%% -------------------------------------------
%%                 AUFGABE 1
%% -------------------------------------------

\subsection*{Aufgabe 1 \mdseries (Die entgegengesetzte Gruppe)}
Es seien $(G, \cdot)$ eine Gruppe und $\star : G \times G, (g,h) \mapsto h \cdot g$. \\
Zeigen Sie, dass $G^{op} := (G, \star)$ eine Gruppe ist. \\ \\
\textbf{Beweis:}\\
(i) $e \in G^{op}$ \\
(ii) $\forall a,b,c \in G^{op}:\,(a \star b) \star c = a \star (b \star c)$ \\
(iii) $\forall a \in G^{op}: \, a^{-1} \in G^{op}$
\mbox{} \hfill $\square$\\


%% -------------------------------------------
%%		AUFGABE 2
%% ------------------------------------------

\subsection*{Aufgabe 2 \mdseries (Wirkungen von $\R$ auf $\C$)}

\begin{enumerate}[\bfseries a)]
\item 
\textbf{Beweis:}\\
beweis
\mbox{} \hfill $\square$\\

\item 
\textbf{Beweis:}\\
beweis
\mbox{} \hfill $\square$\\
\end{enumerate}



%% -------------------------------------------
%%		AUFGABE 3
%% --------------------------------------------
\subsection*{Aufgabe 3 \mdseries (Wirkung einer Untergruppe von $S_8$)}

\begin{enumerate}[\bfseries a)]
\item 
\item 
\end{enumerate}

%% -------------------------------------
%%		AUFGABE 4
%% -------------------------------------
\subsection*{Aufgabe 4 \mdseries (Zyklische Gruppen)}

\textbf{Beweis:}\\
beweis
\mbox{} \hfill $\square$\\

\label{LastPage}

\end{document}