\documentclass[11pt,a4paper,ngerman]{article}
\usepackage[bottom=2.5cm,top=2.5cm]{geometry} 
\usepackage{babel}
\usepackage[utf8]{inputenc} 
\usepackage[T1]{fontenc} 
\usepackage{ae} 
\usepackage{amssymb} 
\usepackage{amsmath} 
\usepackage{graphicx}
\usepackage{fancyhdr}
\usepackage{fancyref}
\usepackage{listings}
\usepackage{xcolor}
\usepackage{paralist}
\usepackage{fancyhdr}
\usepackage{subfigure}
\pagestyle{fancy}
\fancyhead[C]{Alegbra und Zahlentheorie}
\fancyhead[L]{Übung Nr. 3}
\fancyhead[R]{WS 2011/12}
\fancyfoot{}
\fancyfoot[L]{}
\fancyfoot[C]{\thepage\, von \pageref{LastPage}}
\renewcommand{\footrulewidth}{0.5pt}
\renewcommand{\headrulewidth}{0.5pt}
\setlength{\parindent}{0pt} 
\setlength{\headheight}{15pt}

%
\newcommand{\N}{\mathbb{N}}
\newcommand{\NN}{\mathbb{N} \setminus \{0\}}
\newcommand{\Z}{\mathbb{Z}}
\newcommand{\R}{\mathbb{R}}
\newcommand{\ggT}{\text{ggT}}
\newcommand{\kgV}{\text{kgV}}
%

\author{Tutor: David Müßig}
\date{}
\title{Max Wisniewski, Alexander Steen}

\begin{document}

\lstset{language=Java, basicstyle=\ttfamily\fontsize{10pt}{10pt}\selectfont\upshape, commentstyle=\rmfamily\slshape, keywordstyle=\rmfamily\bfseries, breaklines=true, frame=single, xleftmargin=3mm, xrightmargin=3mm, tabsize=2}

\maketitle
\thispagestyle{fancy}

%% -------------------------------------------
%%                 AUFGABE 1
%% -------------------------------------------
\section*{Aufgabe 1}

Es seien $G$ eine Menge und $\cdot : G \times G \to G, (g,h) \mapsto g \cdot h$, eine assoziative Verknüpfung mit einem linksneutralem Element $e \in G$ und einem linksinversen Element $g' \in G$ für jedes $g \in G$.

\begin{enumerate}[\bfseries a)]
%% -------------------------------------------
%%  Aufgabe 1a
%% -------------------------------------------
%% Aufgabenstellung
\item Es seien $g \in G$ und $g' \in G$ ein Element mit $g' \cdot g = e$. Zeigen Sie $g \cdot g' = e$.
\\ \\
%% -------------------------------------------
%% Lösung
\textbf{Beweis}: \\
Es seien $g,g' \in G$, sodass $g' \cdot g = e$. Es sei $g'' \in G$ ein Linksinverses zu $g'$. Dann gilt: \\
$$ e = g'' \cdot g' = g'' \cdot (e \cdot g') = g'' \cdot ((g' \cdot g) \cdot g') $$
$$ \stackrel{assoz.}{=} (g'' \cdot g') \cdot (g \cdot g') = e \cdot (g \cdot g') $$
$$ = g \cdot g' $$
\mbox{} \hfill $\square$
%% -------------------------------------------
%%  Aufgabe 1b
%% -------------------------------------------
%% Aufgabenstellung
\item Beweisen Sie, dass $g \cdot e = g$ für alle $g \in G$ gilt.
\\ \\
%% -------------------------------------------
%% Lösung
\textbf{Beweis}: \\
Es seien $g,g' \in G$, sodass $g' \cdot g = e$. Dann gilt: \\
$$ e \cdot g = (g' \cdot g) \cdot g \stackrel{a)}{=} (g \cdot g') \cdot g $$
$$ \stackrel{assoz.}{=} g \cdot (g' \cdot g) = g \cdot e $$
\mbox{} \hfill $\square$
\end{enumerate}

%% -------------------------------------------
%%                 AUFGABE 2
%% -------------------------------------------
\section*{Aufgabe 2}

Auf $\R$ wird folgende Verknüpfung $\star : \R \times \R \to \R$, mit $(a,b) \mapsto a \cdot b + a + b$ definiert.

\begin{enumerate}[\bfseries a)]

%% -------------------------------------------
%%  Aufgabe 2a
%% -------------------------------------------
%% Aufgabenstellung
\item Zeigen Sie, dass $\star$ das Assoziativgesetz erfüllt und es ein neutrales Element gibt.
\\ \\
%% -------------------------------------------
%% Lösung
\textbf{Beweis}: \\
Sei $a,b,c \in \R$. Dann gilt: \\
$$ a \star (b \star c) = a \star (b \cdot c + b +c) $$
$$ = a \cdot (b \cdot c + b + c) + a + (b \cdot c + b + c) $$
$$ = a \cdot b \cdot c + a \cdot b + a \cdot c + a + b \cdot c + b +c $$
$$ = (a \cdot b + a + b) \cdot c + (a \cdot b +a + b)+c $$
$$ = (a \star b) \cdot c + (a \star b) + c = (a \star b) \star c $$
\mbox{} \hfill $\square$
\newpage
\textbf{Behauptung}: $e = 0$ ist das neutrale Element bzgl. $\star$. \\
\textbf{Beweis}: \\
Sei $a \in \R$. Dann gilt: \\
$$ 0 \star a = 0 \cdot a + 0 + a = a $$
\mbox{} \hfill $\square$
%% -------------------------------------------
%%  Aufgabe 2b
%% -------------------------------------------
%% Aufgabenstellung
\item Welche Elemente in $\R$ besitzen bzgl. $\star$ keine Inversen? Geben Sie die kleinste Teilmenge $N \subset \R$ an, für die $(\R \setminus N,\star)$ eine Gruppe ist.
\\ \\
%% -------------------------------------------
%% Lösung
Suche Inverses $b' \in \R$ zu $b \in \R$: \\
$$ b' \star b = 0 \Leftrightarrow b' \cdot b + b' + b = 0 $$
$$ \Leftrightarrow b' = \frac{-b}{b+1} $$
Also besitzt $b = -1$ kein Inverses, da $\frac{-b}{b+1}$ für $b = -1$ nicht existiert. \\
$ \Rightarrow N = \{-1\} \Rightarrow (\R \setminus \{-1\}, \star)$ ist Gruppe.
\end{enumerate}

%% -------------------------------------------
%%                 AUFGABE 3
%% -------------------------------------------
\section*{Aufgabe 3}
\begin{enumerate}[\bfseries a)]
%% -------------------------------------------
%%  Aufgabe 3a
%% -------------------------------------------
%% Aufgabenstellung
\item Es sei $g \in G$ eine Gruppe, so dass $g^2 = e$ für alle $g \in G$ gilt. Weisen Sie nach, dass $G$ abelsch ist. Geben Sie für jedes $k \geq 1$ eine Gruppe $G$ mit $2k$ Elementen an, in der $g^2 = e$ für jedes Gruppenelement $g \in G$ gilt.
\\ \\
%% -------------------------------------------
%% Lösung
(1) $\forall g \in G: g^2 = e \Rightarrow G$ abelsch. \\


\textbf{Beweis}: \\
Sei $a,b \in G$. Dann gilt:
$$ a \cdot b = e \cdot a \cdot b = b^2 \cdot a \cdot b $$
$$ = b \cdot (b \cdot a) \cdot b \cdot e = b \cdot (b \cdot a) \cdot b \cdot a^2 $$
$$ = b \cdot (b \cdot a) \cdot  (b \cdot a) \cdot a = b \cdot (b \cdot a)^2 \cdot a $$
$$ = b \cdot e \cdot a = b \cdot a$$
\mbox{} \hfill $\square$
\\
(2) Je eine Gruppe $G$ wie oben mit $2^k$ Elementen. \\

Für jedes $k$ ist $\mathbb{Z}_2^k$ eine Gruppe, für die das obere gilt und die $2^k$ Elemente hat. Das es sich um eine Gruppe handelt, wurde in der VL gezeigt. Die Gruppe ist abelsch, da jede Komponente verknüpft mit sich selbst das neutrale Elemente ist (siehe $\mathbb{Z}_2$.
%% -------------------------------------------
%%  Aufgabe 3b
%% -------------------------------------------
%% Aufgabenstellung
\item Es sei $G$ eine endliche abelsche Gruppe. Zeigen Sie, dass 
$$ \prod_{g\in G}g^2 = e.$$
%% -------------------------------------------
%% Lösung
\textbf{Beweis}: \\
Sei $\kappa:\,G \to G, g \mapsto g^{-1}$ eine Funktion, wobei $g^{-1} \in G$ das Inverse zu $g \in G$ beschreibt. Da $G$ Gruppe, besitzt jedes Element ein eindeutiges Inverses $\Rightarrow \kappa$ bijektiv $\Rightarrow \kappa(G) = G$.
Also gilt: \\
$$ \prod_{g\in G}g^2 = \prod_{g\in G}g \cdot g \stackrel{\text{G abelsch}}{=} \prod_{g\in G}g \prod_{g\in G}g  $$
$$ \stackrel{\kappa \text{ bij.}}{=} \prod_{g\in G}g \prod_{g\in G}\kappa(g)
 \stackrel{\text{Def.}}{=} \prod_{g\in G}g \prod_{g\in G} g^{-1} $$
$$ \prod_{g\in G}g \cdot g^{-1} =  \prod_{g\in G}e = e $$
\mbox{} \hfill $\square$
\end{enumerate}

%% -------------------------------------------
%%                 AUFGABE 4
%% -------------------------------------------
\section*{Aufgabe 4}
Es sei $G$ eine endliche Gruppe und $\mathfrak{M} := \{ M \subset G \, | \, M \text{ zyklisch} \}$ die Menge aller zyklischen Teilmengen von $G$.
\begin{enumerate}[\bfseries a)]
%% -------------------------------------------
%%  Aufgabe 4a
%% -------------------------------------------
%% Aufgabenstellung
\item Zeigen Sie, dass der Durchschnitt zweier zyklischen Teilmengen von G zyklisch ist. \\ \\
%% -------------------------------------------
%% Lösung
Seien $A,B \in \mathfrak{M}$ zyklische Teilmengen von $G$.
Seien $a,b\in \mathfrak{M}$ Elemente, so dass
$$
\left< a \right> = A \quad \text{and} \quad \left< b \right> = B.
$$
Nun gilt: (*) $A \cap B \not= \emptyset$.
\begin{description}

\item{\bfseries Beweis:} Da für eine zyklische Teilmenge $\left< q \right>$ gilt: $q^n = q$. Damit ist $q^{n-1}$ da neutrale Element, da $e \cdot q = q$. Das neutrale Element ist also in jeder \emph{zyklischen Teilmenge} enthalten.

\end{description}

Da wir ein Element im Schnitt haben, können wir dieses über $a$ und $b$ bilden.\\
Sei $k = {\min \{k | k\geq 0 \; \land \; a^k \in B\}}$
Sei nun $x = a^k$ Element dieses Schnittes. Nun zeigen wir:

\begin{description}

\item{\bfseries Lemma 1:} $x$ (wie oben konstruiert) ist erzeugendes Element für den Schnitt, d.h. \\
$\left< x \right> = A \cap B$

\item{\bfseries Beweis:}
Da $x \in A\cap B \Rightarrow \exists \, l \geq 0 \; : \; b^l =x$. Wir zeigen nun, dass
$\left< x \right> = A \cap B$\\
$\subseteq$ : \\
Sei $m\geq 0$.\\
$x^m = \left( a^{k} \right)^m = a^{m \cdot k}\stackrel{m \cdot k \leq 0}{\Rightarrow} x^m \in A$\\
$x^m = \left(b^l \right)^m = b^{m \cdot l}\stackrel{m \cdot l \geq 0}{\Rightarrow} x^m \in B$\\
$\Rightarrow \forall m \geq 0 \; : \; x^m \in A \cap B$\\

$\supseteq$:\\
Angenommen es existiert ein $y \in A \cap B$, so dass $\forall t \geq 0 \; : \; x^t \not= y$.\\
Seien nun $a_y , b_y \geq 0$, so dass $a^{a_y} = y$ und $b^{b_y} = y$.
Nun müssten wir zeigen, dass es keine Zerlegung gibt, so dass gilt: $t\cdot k = a_y \; \land \; t \cdot l = b_y$.

\end{description}

Aus \emph{Lemma 1} folgt direkt, dass der Schnitt $A \cap B$ eine \emph{zyklische Teilmenge} von $G$ ist.

\pagebreak

%% -------------------------------------------
%%  Aufgabe 4b
%% -------------------------------------------
%% Aufgabenstellung
\item Zeichnen Sie die Zykelgraphen von $\Z_n, n\in \N$, der Gruppen aus Aufgabe 3 a) und der Diedergruppe $D_6$. \\
%% -------------------------------------------
%% Lösung
\pagebreak
%% handgemalt
%% -------------------------------------------
%%  Aufgabe 4c
%% -------------------------------------------
%% Aufgabenstellung
\item Geben Sie eine Gruppe mit dem Zykelgraphen vom Aufgabenblatt an. \\
%% -------------------------------------------
%% Lösung
Die Gruppe $\Z_3 \times \Z_3$ besitzt genau den Zykelgraphen vom Aufgabenblatt. \\
Dabei ist die Zuordnung der Gruppenelemente z.B.:\\
$$ e = (0,0) $$
$$ g_1 = (1,1), g_2 = (2,2) $$
$$ g_3 = (1,0), g_4 = (2,0) $$
$$ g_5 = (0,1), g_6 = (0,2) $$
$$ g_6 = (1,2), g_7 = (2,1) $$ %% stimmt nicht, glaub ich!
\end{enumerate}

%% -------------------------------------------
\label{LastPage}
\end{document}