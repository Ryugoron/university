\documentclass[11pt,a4paper,ngerman]{article}
\usepackage[bottom=2.5cm,top=2.5cm]{geometry} 
\usepackage{babel}
\usepackage[utf8]{inputenc} 
\usepackage[T1]{fontenc} 
\usepackage{ae} 
\usepackage{amssymb} 
\usepackage{amsmath} 
\usepackage{graphicx}
\usepackage{fancyhdr}
\usepackage{fancyref}
\usepackage{listings}
\usepackage{xcolor}
\usepackage{paralist}
\usepackage{fancyhdr}
\usepackage{subfigure}
\pagestyle{fancy}
\fancyhead[C]{Alegbra und Zahlentheorie}
\fancyhead[L]{Übung Nr. 3}
\fancyhead[R]{WS 2011/12}
\fancyfoot{}
\fancyfoot[L]{}
\fancyfoot[C]{\thepage / \pageref{LastPage}}
\renewcommand{\footrulewidth}{0.5pt}
\renewcommand{\headrulewidth}{0.5pt}
\setlength{\parindent}{0pt} 
\setlength{\headheight}{15pt}

%
\newcommand{\N}{\mathbb{N}}
\newcommand{\NN}{\mathbb{N} \setminus \{0\}}
\newcommand{\Z}{\mathbb{Z}}
\newcommand{\R}{\mathbb{R}}
\newcommand{\ggT}{\text{ggT}}
\newcommand{\kgV}{\text{kgV}}
%

\author{Tutor: David Müßig}
\date{}
\title{Max Wisniewski, Alexander Steen}

\begin{document}

\lstset{language=Java, basicstyle=\ttfamily\fontsize{10pt}{10pt}\selectfont\upshape, commentstyle=\rmfamily\slshape, keywordstyle=\rmfamily\bfseries, breaklines=true, frame=single, xleftmargin=3mm, xrightmargin=3mm, tabsize=2}

\maketitle
\thispagestyle{fancy}

%% -------------------------------------------
%%                 AUFGABE 1
%% -------------------------------------------
\section*{Aufgabe 1}

Es seien $G$ eine Menge und $\cdot : G \times G \to G, (g,h) \mapsto g \cdot h$, eine assoziative Verknüpfung mit einem linksneutralem Element $e \in G$ und einem linksinversen Element $g' \in G$ für jedes $g \in G$.

\begin{enumerate}[\bfseries a)]
%% -------------------------------------------
%%  Aufgabe 1a
%% -------------------------------------------
%% Aufgabenstellung
\item Es seien $g \in G$ und $g' \in G$ ein Element mit $g' \cdot g = e$. Zeigen Sie $g \cdot g' = e$.
\\ \\
%% -------------------------------------------
%% Lösung
\textbf{Beweis}: \\
Es seien $g,g' \in G$, sodass $g' \cdot g = e$. Es sei $g'' \in G$ ein Linksinverses zu $g'$. Dann gilt: \\
$$ e = g'' \cdot g' = g'' \cdot (e \cdot g') = g'' \cdot ((g' \cdot g) \cdot g') $$
$$ \stackrel{assoz.}{=} (g'' \cdot g') \cdot (g \cdot g') = e \cdot (g \cdot g') $$
$$ = g \cdot g' $$
\mbox{} \hfill $\square$
%% -------------------------------------------
%%  Aufgabe 1b
%% -------------------------------------------
%% Aufgabenstellung
\item Beweisen Sie, dass $g \cdot e = g$ für alle $g \in G$ gilt.
\\ \\
%% -------------------------------------------
%% Lösung
\textbf{Beweis}: \\
Es seien $g,g' \in G$, sodass $g' \cdot g = e$. Dann gilt: \\
$$ e \cdot g = (g' \cdot g) \cdot g \stackrel{a)}{=} (g \cdot g') \cdot g $$
$$ \stackrel{assoz.}{=} g \cdot (g' \cdot g) = g \cdot e $$
\mbox{} \hfill $\square$
\end{enumerate}

%% -------------------------------------------
%%                 AUFGABE 2
%% -------------------------------------------
\section*{Aufgabe 2}

Auf $\R$ wird folgende Verknüpfung $\star : \R \times \R \to \R$, mit $(a,b) \mapsto a \cdot b + a + b$ definiert.

\begin{enumerate}[\bfseries a)]

%% -------------------------------------------
%%  Aufgabe 2a
%% -------------------------------------------
%% Aufgabenstellung
\item Zeigen Sie, dass $\star$ das Assoziativgesetz erfüllt und es ein neutrales Element gibt.
\\ \\
%% -------------------------------------------
%% Lösung
\textbf{Beweis}: \\

%% -------------------------------------------
%%  Aufgabe 2b
%% -------------------------------------------
%% Aufgabenstellung
\item Welche Elemente in $\R$ besitzen bzgl. $\star$ keine Inversen? Geben Sie die kleinste Teilmenge $\N \subset \R$ an, für die $(\R \setminus \N,\star)$ eine Gruppe ist.
\\ \\
%% -------------------------------------------
%% Lösung

\end{enumerate}

%% -------------------------------------------
%%                 AUFGABE 3
%% -------------------------------------------
\section*{Aufgabe 3}
\begin{enumerate}[\bfseries a)]
%% -------------------------------------------
%%  Aufgabe 3a
%% -------------------------------------------
%% Aufgabenstellung
\item
%% -------------------------------------------
%% Lösung

%% -------------------------------------------
%%  Aufgabe 3b
%% -------------------------------------------
%% Aufgabenstellung
\item
%% -------------------------------------------
%% Lösung

\end{enumerate}

%% -------------------------------------------
%%                 AUFGABE 4
%% -------------------------------------------
\section*{Aufgabe 4}
\begin{enumerate}[\bfseries a)]
%% -------------------------------------------
%%  Aufgabe 4a
%% -------------------------------------------
%% Aufgabenstellung
\item
%% -------------------------------------------
%% Lösung

%% -------------------------------------------
%%  Aufgabe 4b
%% -------------------------------------------
%% Aufgabenstellung
\item
%% -------------------------------------------
%% Lösung


%% -------------------------------------------
%%  Aufgabe 4c
%% -------------------------------------------
%% Aufgabenstellung
\item
%% -------------------------------------------
%% Lösung

\end{enumerate}

%% -------------------------------------------
\label{LastPage}
\end{document}