\documentclass[11pt,a4paper,ngerman]{article}
\usepackage[bottom=2.5cm,top=2.5cm]{geometry} 
\usepackage{babel}
\usepackage[utf8]{inputenc} 
\usepackage[T1]{fontenc} 
\usepackage{ae} 
\usepackage{amssymb} 
\usepackage{amsmath} 
\usepackage{graphicx}
\usepackage{fancyhdr}
\usepackage{fancyref}
\usepackage{listings}
\usepackage{xcolor}
\usepackage{paralist}
\usepackage{fancyhdr}
\usepackage{subfigure}
\pagestyle{fancy}
\fancyhead[C]{Alegbra und Zahlentheorie}
\fancyhead[L]{Übung Nr. 6}
\fancyhead[R]{WS 2011/12}
\fancyfoot{}
\fancyfoot[L]{}
\fancyfoot[C]{\thepage\, von \pageref{LastPage}}
\renewcommand{\footrulewidth}{0.5pt}
\renewcommand{\headrulewidth}{0.5pt}
\setlength{\parindent}{0pt} 
\setlength{\headheight}{15pt}

%
\newcommand{\N}{\mathbb{N}}
\newcommand{\NN}{\mathbb{N} \setminus \{0\}}
\newcommand{\Z}{\mathbb{Z}}
\newcommand{\R}{\mathbb{R}}
\newcommand{\Q}{\mathbb{Q}}
\newcommand{\ggT}{\text{ggT}}
\newcommand{\kgV}{\text{kgV}}
%

\author{Tutor: David Müßig}
\date{}
\title{Max Wisniewski, Alexander Steen}

\begin{document}

\lstset{language=Java, basicstyle=\ttfamily\fontsize{10pt}{10pt}\selectfont\upshape, commentstyle=\rmfamily\slshape, keywordstyle=\rmfamily\bfseries, breaklines=true, frame=single, xleftmargin=3mm, xrightmargin=3mm, tabsize=2}

\maketitle
\thispagestyle{fancy}

%% -------------------------------------------
%%                 AUFGABE 1
%% -------------------------------------------

\subsection*{Aufgabe 1 \mdseries (Die additive Gruppen von $\Q$)}
Beweisen Sie, dass $\Q$ nicht endlich erzeugt ist.\\

\textbf{Beweis:}\\

Die Addition von zwei Zahlen $a,b \in \Q$ sieht folgender Maßen aus:\\
Sei $a := \frac{p_1}{q_1}, \; b := \frac{p_2}{q_2}$, mit $p_1,p_2 \in \Z$ und $q_1,q_2 \in \NN$.\\
Dann ist:
$$
a+b = \frac{p_1}{q_1} + \frac{p_2}{q_2} = \frac{p_1q_2 + p_2q_1}{q_1q_2}.
$$

Nun wollen wir für den Beweis nur noch vollig gekürzte Brüche betrachten.\\
Um nach der Addition noch zu kürzen, Teilen wir Zähler und Nenner durch deren ggT.
$$
p_1 = r \cdot \ggT (p_1,q_1), \; q_1 = s \cdot \ggT (p_1,q_1) \Rightarrow \frac{p_1}{q_1}=\frac{r \cdot \ggT (p_1,q_1)}{s \cdot \ggT (p_1 ,q_1)} = \frac{r}{s}
$$

\textbf{Lemma:}\\
Sei $M = \{ \frac{p_1}{q_1} , ... , \frac{p_n}{q_n} \} \subset \Q$ endliche Teilmenge, mit $p_i \in \Z, \, q_i \in \NN, 1 \leq i \leq n$. Dann ist:
$$\frac{1}{q_1 \cdot q_2 \cdot ... \cdot q_n + 1} \not\in \left< M \right>.$$

\textbf{Beweis:}\\
Sei o.B.d.A $q_i > 1$, da 1 als neutrales Element der Multiplikation das Ergebnis nicht verändert.\\

Analog zum Beweis von Euklid, wird hier ersichtlich, dass keiner unserer Nenner $q_i$ die Zahl $q_1 \cdot ... \cdot q_n + 1$ teilen kann. Da aber alle Zahlen im Nenner durch Multiplikation erzeugt werden, kann keiner unserer Nenner der Basis dieses Element erzeugen.\\

(Einen ähnliches Beweis kann man führen, wenn man zeigt, dass aus den bestehenden Primzahlen der Nenner (in Primfaktordarstellung) durch Multiplikation keine neue von den anderen verschiedene Primzahl gewonnen werden kann.)\\
\mbox{} \hfill $\square$\\

Mithilfe des Lemmas können wir nun leicht sehen, dass wir zu jeder beliebigen endlichen Basis eine Zahl wie im Lemma konstruieren können, die nicht von der Basis erzeugt werden kann. Damit kann $\Q$ nicht endlich erzeugt werden.

%% -------------------------------------------
%%		AUFGABE 2
%% ------------------------------------------

\pagebreak

\subsection*{Aufgabe 2 \mdseries (Zykelzerlegungen)}

\begin{enumerate}[\bfseries a)]
\item Es seien $c_1$ und $c_2$ zwei disjunkte Zykel in $S_n$. Zu zeigen ist $c_1 \cdot c_2 = c_2 \cdot c_1$.\\

\textbf{Beweis:}\\
Sei $a \in \{ 1 ,.. ,n \}$ beliebig, $c_1 = \left( \alpha_1 ... \alpha_k \right), c_2 = \left( \beta_1  ...  \beta_l \right)$ disjunkt.\\

\begin{enumerate}[\bfseries {Fall} 1]

\item $a \not\in \{ \alpha_1, ... , \alpha_k , \beta_1 , ... , \beta_l \}$.\\
$$c_1 a = a \; \land \; c_2 a = a \Longrightarrow (c_1 \cdot c_2) a = a = (c_2 \cdot c_1) a$$

\item $a \in \{ \alpha_1 , ... , \alpha_k\}$\\
$\Rightarrow \exists 1 \leq i \leq k \; : \; a = \alpha_i$. (Zur vereinfachung gilt gleich $\alpha_1 = \alpha_{k+1}$)\\
$\stackrel{\text{disjunkt}}{\Rightarrow} a=\alpha_i, \alpha_{i+1} \not\in \{ \beta_1 , ... , \beta_l \}$
$$
c_1 \cdot a = \alpha_{i+1} \, \land \, c_2 \cdot a = a \Longrightarrow (c_1 \cdot c_2) a = c_1 a = \alpha_{i+1} = c_2 \alpha_{i+1} = (c_2 \cdot c_1) a
$$

\item $a \in \{ \beta_1 , ... , \beta_l \}$\\
Analog zu \emph{Fall 2}.\\

\mbox{} \hfill $\square$
\end{enumerate}


\item Beweisen Sie folgende Aussage:\\
\textbf{Satz:} Es seien $c_1, ..., c_s$ und $d_1, ... , d_t$ Zykel, so dass $c_i$ und $c_j$ für $1 \leq i < j \leq s$ und $d_k$ und $d_l$ für $1 \leq k < l \leq t$ disjunkt sind. Dann gilt
$$
c_1 \cdot ... \cdot c_s = d_1 \cdot ... \cdot d_t \Rightarrow \{ c_1 , ..., c_s \} = \{ d_1 , ... , d_t \}
$$

\textbf{Beweis:}

Wir machen eine eine Rückwärtsinduktion:\\

Sei $c_1 = (\alpha_1 ... \alpha_p)$. Da beide Permutationen $\alpha_1$ auf das selbe Element abbilden, muss ein $k \in \{ 1,.., t \}$ existieren, so dass $d_k$ dieses $\alpha_1$ enthält. Dieses $k$ ist eindeutig, da die Zykel paarweise disjunkt sind.\\

\begin{description}

\item{\bfseries Behauptung} $\forall 1 \leq a \leq p \; : \; c_1 \alpha_a = d_k \alpha_a$

\item{I.A.} Da die Zykel disjunkt sind, aber die Komposition auf das selbe abbildet, gilt $c_1 \alpha_1 = \alpha_2 = d_k \alpha_1$, nach Auswahl von $c_1 , d_k$.

\item{I.S.} $\omega \rightarrow \omega+1$\\
Nach I.V. bildet $c_1 \alpha_{\omega-1} = \alpha_\omega = d_k \alpha_{\omega -1}$ auf das selbe Element. Nun wissen wir wieder über die Komposition, dass beide dieses Element, dass sich im zykel befindet auf das selbe Element abbildet. $c_1 \alpha_\omega = \alpha_{\omega+1} = d_k \alpha_\omega$.

\end{description}

Die Induktion bricht an einem Punkt ab. Wir können über den Zykel auch unendlich weiter laufen, da es keine Fehler erzeugt.

\textbf{Schritt:}\\
Nun können wir $c_1$ und $d_k$ aus der Permutation herrausnehmen und mit der Verkleinerten Permutation fortfahren:
$c_2 ... c_s = d_1 ... d_{k-1} d_{k+1} .. d_t$\\

Hat man den letzten Zykel von $c$ herraus genommen, bleibt hier nur noch die Identität. Damit muss auch $d$ nun die Identität sein und keinen Zykel übirig behalten haben. Da wir schrittweise gleiche Zykel herausgenommen haben folgt daraus $\{ c_1 , ... , c_s \} = \{ d_1 , ... , d_t \}$ und $s = t$. \\
\mbox{} \hfill $\square$

\item Leiten Sie folgendes Ergebnis ab: \\
\textbf{Folgerung:} Jede Permutation $\sigma \in S_n \setminus \{ e \}$ besitzen eine bis auf die Reihenfolge eindeutige Darstellung
$$
\sigma = c_1 \cdot ... \cdot c_s
$$
als Produkt paarweise disjunkter Zykel.\\

\textbf{Beweis:}\\
Sei $\sigma = d_1 \cdot ... \cdot d_t$ eine weitere Zerlegung paarweise disjunkter Zykel.\\

Nach \emph{b)} wissen wir, dass $\{c_1 , ..., c_s \} = \{ d_1 , ... , d_t \}$ gilt. Die benutzten Zykel sind also die selben. Da nach \emph{a)} disjunkte Zykel kommutativ sind, können wir die Reihenfolge in der Darstellung auch herstellen, egal welche Reihenfolge die Zykel haben.\\
\mbox{} \hfill $\square$

\end{enumerate}



%% -------------------------------------------
%%		AUFGABE 3
%% --------------------------------------------
\subsection*{Aufgabe 3 \mdseries (Rechnen in der symmetrischen Gruppe)}

\begin{enumerate}[\bfseries a)]
\item Schreiben Sie die Permutation
$$
\left( \begin{array}{cccccccccccccc}
1 & 2 & 3 & 4 & 5 & 6 & 7 & 8 & 9 & 10 & 11 & 12 & 13 & 14  \\
3 & 5 & 7 & 13 & 14 & 1 & 12 & 10 & 8 & 9 & 2 & 6 & 4 & 11  \\
\end{array} \right)
$$
als Produkt disjunkter Zykel.

\textbf{Lösung:}\\
$$
(3 \; 7 \; 12 \; 6  \;1) \cdot (5 \;14 \;11\; 2) \cdot (13 \;4) \cdot (10\; 9\; 8)
$$


\item Stellen Sie die Permutation
$$
\left( \begin{array}{cccccccccccccc}
1 & 2 & 3 & 4 & 5 & 6 & 7 & 8 & 9 & 10 & 11 & 12 & 13  \\
11 & 4 & 9 & 3 & 10 & 5 & 2 & 8 & 12 & 6 & 13 & 7 & 1  \\
\end{array} \right)
$$
als Produkt von Transposition dar.\\



\item Geben Sie das Vorzeichen der Permutation 
$$
\left( \begin{array}{cccccccccccccc}
1 & 2 & 3 & 4 & 5 & 6 & 7 & 8 & 9 & 10  \\
4 & 7 & 9 & 3 & 6 & 5 & 10 & 2 & 1 & 8  \\
\end{array} \right)
$$
an.

\end{enumerate}

%% -------------------------------------
%%		AUFGABE 4
%% -------------------------------------
\subsection*{Aufgabe 4 \mdseries (Gruppenwirkungen)}

\begin{enumerate}[\bfseries a)]
\item stuff
\end{enumerate}

\label{LastPage}

\end{document}