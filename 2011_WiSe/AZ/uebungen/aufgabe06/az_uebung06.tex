\documentclass[11pt,a4paper,ngerman]{article}
\usepackage[bottom=2.5cm,top=2.5cm]{geometry} 
\usepackage{babel}
\usepackage[utf8]{inputenc} 
\usepackage[T1]{fontenc} 
\usepackage{ae} 
\usepackage{amssymb} 
\usepackage{amsmath} 
\usepackage{graphicx}
\usepackage{fancyhdr}
\usepackage{fancyref}
\usepackage{listings}
\usepackage{xcolor}
\usepackage{paralist}
\usepackage{fancyhdr}
\usepackage{subfigure}
\pagestyle{fancy}
\fancyhead[C]{Alegbra und Zahlentheorie}
\fancyhead[L]{Übung Nr. 6}
\fancyhead[R]{WS 2011/12}
\fancyfoot{}
\fancyfoot[L]{}
\fancyfoot[C]{\thepage\, von \pageref{LastPage}}
\renewcommand{\footrulewidth}{0.5pt}
\renewcommand{\headrulewidth}{0.5pt}
\setlength{\parindent}{0pt} 
\setlength{\headheight}{15pt}

%
\newcommand{\N}{\mathbb{N}}
\newcommand{\NN}{\mathbb{N} \setminus \{0\}}
\newcommand{\Z}{\mathbb{Z}}
\newcommand{\R}{\mathbb{R}}
\newcommand{\Q}{\mathbb{Q}}
\newcommand{\ggT}{\text{ggT}}
\newcommand{\kgV}{\text{kgV}}
%

\author{Tutor: David Müßig}
\date{}
\title{Max Wisniewski, Alexander Steen}

\begin{document}

\lstset{language=Java, basicstyle=\ttfamily\fontsize{10pt}{10pt}\selectfont\upshape, commentstyle=\rmfamily\slshape, keywordstyle=\rmfamily\bfseries, breaklines=true, frame=single, xleftmargin=3mm, xrightmargin=3mm, tabsize=2}

\maketitle
\thispagestyle{fancy}

%% -------------------------------------------
%%                 AUFGABE 1
%% -------------------------------------------

\subsection*{Aufgabe 1 \mdseries (Die additive Gruppen von $\Q$)}
Beweisen Sie, dass $\Q$ nicht endlich erzeugt ist.

\textbf{Beweis:}\\

%% -------------------------------------------
%%		AUFGABE 2
%% ------------------------------------------

\subsection*{Aufgabe 2 \mdseries (Zykelzerlegungen)}

\begin{enumerate}[\bfseries a)]
\item Es seien $c_1$ und $c_2$ zwei disjunkte Zykel in $S_n$. Zu zeigen ist $c_1 \cdot c_2 = c_2 \cdot c_1$.

\item Beweisen Sie folgende Aussage: ...bla

\item Leiten Sie folgendes Ergebnis ab: ...bla
\end{enumerate}



%% -------------------------------------------
%%		AUFGABE 3
%% --------------------------------------------
\subsection*{Aufgabe 3 \mdseries (Rechnen in der symmetrischen Gruppe)}

\begin{enumerate}[\bfseries a)]
\item Schreiben Sie die Permutation
$$
\left( \begin{array}{cccccccccccccc}
1 & 2 & 3 & 4 & 5 & 6 & 7 & 8 & 9 & 10 & 11 & 12 & 13 & 14  \\
3 & 5 & 7 & 13 & 14 & 1 & 12 & 10 & 8 & 9 & 2 & 6 & 4 & 11  \\
\end{array} \right)
$$
als Produkt disjunkter Zykel.

\item Stellen Sie die Permutation
$$
\left( \begin{array}{cccccccccccccc}
1 & 2 & 3 & 4 & 5 & 6 & 7 & 8 & 9 & 10 & 11 & 12 & 13  \\
11 & 4 & 9 & 3 & 10 & 5 & 2 & 8 & 12 & 6 & 13 & 7 & 1  \\
\end{array} \right)
$$
als Produkt von Transposition dar

\item Geben Sie das Vorzeichen der Permutation 
$$
\left( \begin{array}{cccccccccccccc}
1 & 2 & 3 & 4 & 5 & 6 & 7 & 8 & 9 & 10  \\
4 & 7 & 9 & 3 & 6 & 5 & 10 & 2 & 1 & 8  \\
\end{array} \right)
$$
an.

\end{enumerate}

%% -------------------------------------
%%		AUFGABE 4
%% -------------------------------------
\subsection*{Aufgabe 4 \mdseries (Gruppenwirkungen)}

\begin{enumerate}[\bfseries a)]
\item stuff
\end{enumerate}

\label{LastPage}

\end{document}