\documentclass[11pt,a4paper,ngerman]{article}
\usepackage[bottom=2.5cm,top=2.5cm]{geometry} 
\usepackage{babel}
\usepackage[utf8]{inputenc} 
\usepackage[T1]{fontenc} 
\usepackage{ae} 
\usepackage{amssymb} 
\usepackage{amsmath} 
\usepackage{graphicx}
\usepackage{fancyhdr}
\usepackage{fancyref}
\usepackage{listings}
\usepackage{xcolor}
\usepackage{paralist}
\usepackage{fancyhdr}
\usepackage{subfigure}
\pagestyle{fancy}
\fancyhead[C]{Alegbra und Zahlentheorie}
\fancyhead[L]{Übung Nr. 10}
\fancyhead[R]{WS 2011/12}
\fancyfoot{}
\fancyfoot[L]{}
\fancyfoot[C]{\thepage\, von \pageref{LastPage}}
\renewcommand{\footrulewidth}{0.5pt}
\renewcommand{\headrulewidth}{0.5pt}
\setlength{\parindent}{0pt} 
\setlength{\headheight}{15pt}

%
\newcommand{\N}{\mathbb{N}}
\newcommand{\NN}{\mathbb{N} \setminus \{0\}}
\newcommand{\Z}{\mathbb{Z}}
\newcommand{\C}{\mathbb{C}}
\newcommand{\R}{\mathbb{R}}
\newcommand{\Q}{\mathbb{Q}}
\newcommand{\ggT}{\text{ggT}}
\newcommand{\kgV}{\text{kgV}}
\newcommand{\sign}{\text{Sign}}
\newcommand{\ord}{\text{Ord}}
%

\author{Tutor: David Müßig}
\date{}
\title{Max Wisniewski, Alexander Steen}

\begin{document}

\lstset{language=Java, basicstyle=\ttfamily\fontsize{10pt}{10pt}\selectfont\upshape, commentstyle=\rmfamily\slshape, keywordstyle=\rmfamily\bfseries, breaklines=true, frame=single, xleftmargin=3mm, xrightmargin=3mm, tabsize=2}

\maketitle
\thispagestyle{fancy}

%% -------------------------------------------
%%                 AUFGABE 1
%% -------------------------------------------

\subsection*{Aufgabe 1 \mdseries (Gruppen der Ordnung 10)}
Beweisen Sie, dass eine endliche Gruppe $G$ der Ordnung 1000 einen Normalteiler $H$
mit $\{e\} \subsetneq H \subsetneq G$ besitzt. \\

Die Ordnun von G ist $\#G = 1000 = 2^3 \cdot 5^3$. \\
$\stackrel{\text{1. Sylowsatz}}{\Rightarrow}$ existieren $h$ 5-Sylow-Untergruppen.  Für $h$ muss gelten: $h \equiv 1 \mod 5$ und $h \, | \, 2^3 = 8$. Die Teiler von 8 sind 1,2,4. Da aber nur $1 \equiv 1 \mod 5$ gilt, gibt es genau eine 5-Sylow-Untergruppe. \\
Wegen der Eindeutigkeit der Sylow-Untergruppe folgt aus dem zweiten Sylowsatz, dass diese ein Normalteiler von $G$ ist.
\mbox{} \hfill $\square$


%% -------------------------------------------
%%                 AUFGABE 2
%% -------------------------------------------

\subsection*{Aufgabe 2 \mdseries (Kommutierende Normalteiler)}
Es seien $G$ eine Gruppe und $H,J$ Normalteiler von $G$, so dass $H \cap J = \{e\}$.
\begin{enumerate}[\bfseries a)]
\item  Beweisen Sie $\forall h \in H\forall j \in J :\; h \cdot j = j \cdot h$. \\

Sei $h \in H, j \in J$. Betrachte $g = h \cdot j \cdot h^{-1} \cdot j^{-1} \in G$. \\
(1) Es gilt: $H \lhd G \Rightarrow j \cdot h^{-1} \cdot j^{-1} \in H \Rightarrow h \cdot (j \cdot h^{-1} \cdot j^{-1}) \in H$. \\
(2) Es gilt: $J \lhd G \Rightarrow h \cdot j \cdot h^{-1} \in J \Rightarrow (h \cdot j \cdot h^{-1}) \cdot j^{-1} \in J$. \\
$\Rightarrow g \in H \cap J \Rightarrow g = e$.
Also gilt nun:
$$
h \cdot j \cdot h^{-1} \cdot j^{-1} = e \Leftrightarrow h \cdot j = j \cdot h
$$
\mbox{} \hfill $\square$
\item  Nun sei $G$ eine endliche Gruppe mit $\#G = \#H \cdot \#J$. Zeigen Sie $G \cong H \times J$. \\

Da 

\mbox{} \hfill $\square$
\end{enumerate}


%% -------------------------------------------
%%                 AUFGABE 3
%% -------------------------------------------

\subsection*{Aufgabe 3 \mdseries (Zyklische Gruppen)}
Es seien $p < q$ Primzahlen, so dass $q \not\equiv 1 \mod p$, und $G$ eine endliche Gruppe der Ordnung $p \cdot q$.

\begin{enumerate}[\bfseries a)]
\item Geben Sie mindestens vier Beispiele für Paare $(p,q)$ mit den obigen Eigenschaften an. \\

$$ (3, 5), (5, 7), (7, 11), (11, 13) $$

\item Beweisen Sie, dass $G$ einen Normalteiler der Ordnung $p$ und einen Normalteiler der Ordnung $q$ hat. \\

Nach dem ersten Sylow-Satz existieren sowohl $p$-Sylow-Untergruppen als auch $q$-Sylow-Untergruppen.\\
Da $q \not\equiv 1 \mod p$ und $q$ Primzahl $\Rightarrow$ ex. genau eine $p$-Sylow-Untergruppe. Damit ist diese ein Normalteiler. \\
Da $p < q$ ist nur die 1 Teiler von $q$ mit Restklasse 1. $\Rightarrow$ ex. genau eine $q$-Sylow-Untergruppe. Damit ist diese ein Normalteiler.
\mbox{} \hfill $\square$

\item  Zeigen Sie, dass $G$ isomorph zu $\Z_p \times \Z_q$ ist, und folgern Sie, dass $G$ zyklisch ist. \\

Sei $A$ die $p$-Sylow-Gruppe und $B$ die $q$-Sylow-Gruppe aus b).\\
Da $\#A$ und $\#B$ Primzahl ist, sind $A, B$ zyklisch. \\
Für ein $g \in A \cap B, g \neq e$ gilt, dass $\ord(g) = p \land \ord(g) = q$ und weil $p < q \Rightarrow g = e$. Also ist $ A \cap B = \{ e \}$. Dann gilt nach Aufgabe 2b), dass $G \cong A \times B$ (Hier könnte man auch einfach einen Isomorphismus zwischen Potenzen der Erzeuger von A bzw. B und G aufstellen).\\
Da $A,B$ zyklisch gilt: $A \cong \Z_p$ und $B \cong \Z_q$ und damit 
$$
G \cong A \times B \cong \Z_p \times \Z_q
$$
\mbox{} \hfill $\square$
\end{enumerate}

%% -------------------------------------------
%%                 AUFGABE 4
%% -------------------------------------------

\subsection*{Aufgabe 4 \mdseries (Endliche abelsche Gruppen)}
Listen Sie alle Isomorphieklassen von endlichen abelschen Gruppen $A$ der Ordnung 36
auf. \\

Die Ordnung von A ist $\#A = 36 = 2^2 \cdot 3^2$.\\
Dann gilt einer der folgenden Isomorphien (wie man durch Sylow-Untergruppen-Betrachtung herausfinden kann):

$$
\begin{array}{l}	
A \cong \Z_4 \times \Z_9 \\
A \cong (\Z_2 \times \Z_2) \times \Z_9  \\
A \cong (\Z_2 \times \Z_2) \times (\Z_3 \times \Z_3)  \\
A \cong \Z_4 \times (\Z_3 \times \Z_3)  \\
\end{array}
$$
\label{LastPage}


\end{document}