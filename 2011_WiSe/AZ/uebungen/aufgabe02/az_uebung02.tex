\documentclass[11pt,a4paper,ngerman]{article}
\usepackage[bottom=2.5cm,top=2.5cm]{geometry} 
\usepackage{babel}
\usepackage[utf8]{inputenc} 
\usepackage[T1]{fontenc} 
\usepackage{ae} 
\usepackage{amssymb} 
\usepackage{amsmath} 
\usepackage{graphicx}
\usepackage{fancyhdr}
\usepackage{fancyref}
\usepackage{listings}
\usepackage{xcolor}
\usepackage{paralist}
\usepackage{fancyhdr}
\usepackage{subfigure}
\pagestyle{fancy}
\fancyhead[C]{Alegbra und Zahlentheorie}
\fancyhead[L]{Übung Nr. 2}
\fancyhead[R]{WS 2011/12}
\fancyfoot{}
\fancyfoot[L]{}
\fancyfoot[C]{\thepage / \pageref{LastPage}}
\renewcommand{\footrulewidth}{0.5pt}
\renewcommand{\headrulewidth}{0.5pt}
\setlength{\parindent}{0pt} 
\setlength{\headheight}{15pt}

%
\newcommand{\N}{\mathbb{N}}
\newcommand{\NN}{\mathbb{N} \setminus \{0\}}
\newcommand{\Z}{\mathbb{Z}}
\newcommand{\ggT}{\text{ggT}}
\newcommand{\kgV}{\text{kgV}}
%

\author{Tutor: David Müßig}
\date{}
\title{Max Wisniewski, Alexander Steen}

\begin{document}

\lstset{language=Java, basicstyle=\ttfamily\fontsize{10pt}{10pt}\selectfont\upshape, commentstyle=\rmfamily\slshape, keywordstyle=\rmfamily\bfseries, breaklines=true, frame=single, xleftmargin=3mm, xrightmargin=3mm, tabsize=2}

\maketitle
\thispagestyle{fancy}

%% -------------------------------------------
%%                 AUFGABE 1
%% -------------------------------------------

\section*{Aufgabe 1}

Es seien $a,b \in \N \setminus \{ 0,1 \}$ natürliche Zahlen, sowie
$$
a= p_1^{k_1}\cdot ...\cdot p_s^{k_s} \quad \text{und} \quad b = q_1^{l_1} \cdot ... \cdot q_s^{l_t}
$$
\begin{enumerate}[\bfseries a)]
%% -------------------------------------------
%%  Aufgabe 1a
%% -------------------------------------------
%% Aufgabenstellung
\item Geben Sie die Primfaktorzerlegung von $\ggT (a,b)$ und $\kgV (a,b)$ an.\\
%% -------------------------------------------
%% Lösung

%% -------------------------------------------
%%  Aufgabe 1b
%% -------------------------------------------
%% Aufgabenstellung
\item Beweisen Sie die Formel
$$ \ggT (a,b) \cdot \kgV (a,b) = a \cdot b $$
%% -------------------------------------------
%% Lösung

\end{enumerate}

%% -------------------------------------------
%%                 AUFGABE 2
%% -------------------------------------------
\section*{Aufgabe 2}
\begin{enumerate}[\bfseries a)]

%% -------------------------------------------
%%  Aufgabe 2a
%% -------------------------------------------
%% Aufgabenstellung
\item Bestimmen Sie den größten gemeinsamen Teiler von 165 und 585 mit dem euklidischen Algorithmus.
%% -------------------------------------------
%% Lösung
$585 = 3 \cdot 165 + 90$\\
$165 = 1 \cdot 90 + 75$\\
$90 = 1 \cdot 75 + 15$\\
$75 = 5 \cdot 15 + 0$\\
\\
$\Rightarrow \ggT (165,585) = 15$

%% -------------------------------------------
%%  Aufgabe 2b
%% -------------------------------------------
%% Aufgabenstellung
\item Geben Sie die Primfaktorzerlegungen von 165 und 585 an und überprüfen Sie das Ergebnis aus a) mit Ihrer Formel aus Aufgabe 1, a).
%% -------------------------------------------
%% Lösung
$$585 = 3^2 \cdot 5 \cdot 13$$
$$165 = 3 \cdot 5 \cdot 11$$

Nach \textbf{1 a)} gilt dann für den größten gemeinsamen Teiler: \\
$$ \ggT (585, 165) = 3 \cdot 5 = 15 $$

%% -------------------------------------------
%%  Aufgabe 2c
%% -------------------------------------------
%% Aufgabenstellung
\item Berechnen Sie mit dem erweiterten euklidischen Algorithmus den größten gemeinsamen Teiler $d$ von 142 und 202 und ganze Zahlen $a$ und $b$, sodass
$$ d = a \cdot 142 + b \cdot 202. $$
%% -------------------------------------------
%% Lösung
\begin{figure}[ht]
  %% eukl. Alg.
  \subtable{
	\begin{tabular}{ll}
		202 & $= 1 \cdot 142 + 60$\\
		142 & $= 2 \cdot 60 + 22$\\
		60 & $= 2 \cdot 22 + 16$\\
		22 & $= 1 \cdot 16 + 6$\\
		16 & $= 2 \cdot 6 + 4$\\
		6 & $= 1 \cdot 4 + 2$\\
		4 & $= 2 \cdot 2 + 0$\\
	\end{tabular}
  }
  %% Tabelle des erweiterten eukl. Alg.
  \subtable{
	\begin{tabular}{l||l|l|l|l}
		$i$ & $a_i$ & $q_i$ & $x_i$ & $y_i$ \\
		\hline \hline
		0 & 202 & - & 1 & 0\\
		1 & 142 & - & 0 & 1\\
		2 & 60 & 1 & 1 & -1\\
		3 & 22 & 2 & -2 & 3\\
		4 & 16 & 2 & 5 & -7\\
		5 & 6 & 1 & -7 & 10\\
		6 & 4 & 2 & 19 & -27\\
		7 & 2 & 1 & -26 & 37\\
		8 & 0 & 2 & 33 &\\
		\hline
	\end{tabular}
  }
\end{figure}

$\Rightarrow d = \ggT (142, 202) = 2, b = -26, a = 37$ \\
\end{enumerate}

%% -------------------------------------------
%%                 AUFGABE 3
%% -------------------------------------------
\section*{Aufgabe 3}
%% Aufgabenstellung
Es sei $(\varepsilon_k)_{k \in \N}$ die periodische Folge $(1,3,2,-1,-3,-2,...)$. \\
Beweisen Sie, dass eine ganze Zahl $a = \sum_{k=0}^{m}{a_k \cdot 10^k}$ genau dann durch 7 teilbar ist, wenn ihre gewichtete Quersumme es ist.\\
Testen Sie es mit diesem Kriterium die Zahlen 10.167.157 und 8.484.372 auf Teilbarkeit durch 7.\\
\\
%% -------------------------------------------
%% Lösung
Es gilt
$$ 10^0 \equiv 1 \mod 7 $$
$$ 10^1 \equiv 3 \mod 7 $$
$$ 10^2 \equiv 2 \mod 7 $$
$$ 10^3 \equiv -1 \mod 7 $$
$$ 10^4 \equiv -3 \mod 7 $$
$$ 10^5 \equiv -2 \mod 7 $$
$$ 10^6 \equiv 1 \mod 7 $$
$$ ... $$
$$ 10^k \equiv \varepsilon_k \mod 7 $$
Also ergibt sich für eine Zahl $a$
$$ a = \sum_{k=0}^{m}{a_k \cdot 10^k} \equiv \sum_{k=0}^{m}{a_k \cdot \varepsilon_k} \mod 7 $$
\mbox{} \hfill $\square$

%% -------------------------------------------
%%                 AUFGABE 4
%% -------------------------------------------
\section*{Aufgabe 4}
%% Aufgabenstellung
Vor einem Bienenvolk sind folgende Daten bekannt: Es hat zwischen 200 und 250 Mitglieder. Stellen sich die Bienen in 7er-Reihen auf, dann bleibt eine Biene alleine. Wenn sie sich in 5er-Reihen aufstellen, dann bleiben drei übrig. Wie viele Mitglieder hat das Bienenvolk?\\
\\
%% -------------------------------------------
%% Lösung
Aus der Aufgabenstellung erstellen wir folgendes Kongruenzsystem $(*)$: \\

$$ x \equiv 1 \mod 7 $$
$$ x \equiv 3 \mod 5 $$

Aus dem chinesischen Restsatz folgt direkt (da $\ggT(5,7)=1$): \\

$$ (*) \Leftrightarrow x \equiv 1 - (-2) \cdot 7 \cdot (1-2)$$
$$ \equiv -27 \mod 35$$

Wähle $k \in \N$ mit $200 \leq -27 + k \cdot 35 \leq 250$ \\
$ \Rightarrow k = 7 $\\
$ \Rightarrow $ Das Bienenvolk hat $ -27 + 7 \cdot 35 = 218 $ Mitglieder.
%% -------------------------------------------
\label{LastPage}
\end{document}