\documentclass[11pt,a4paper,ngerman]{article}
\usepackage[bottom=2.5cm,top=2.5cm]{geometry} 
\usepackage{babel}
\usepackage[utf8]{inputenc} 
\usepackage[T1]{fontenc} 
\usepackage{ae} 
\usepackage{amssymb} 
\usepackage{amsmath} 
\usepackage{graphicx}
\usepackage{fancyhdr}
\usepackage{fancyref}
\usepackage{listings}
\usepackage{xcolor}
\usepackage{paralist}
\usepackage[pdftex, bookmarks=false, pdfstartview={FitH}, linkbordercolor=white]{hyperref}
\usepackage{fancyhdr}
\pagestyle{fancy}
\fancyhead[C]{Alegbra und Zahlentheorie}
\fancyhead[L]{Übung Nr. 2}
\fancyhead[R]{WS 2011/12}
\fancyfoot{}
\fancyfoot[L]{}
\fancyfoot[C]{\thepage / \pageref{LastPage}}
\renewcommand{\footrulewidth}{0.5pt}
\renewcommand{\headrulewidth}{0.5pt}
\setlength{\parindent}{0pt} 
\setlength{\headheight}{15pt}

%
\newcommand{\N}{\mathbb{N}}
\newcommand{\NN}{\mathbb{N} \setminus \{0\}}
\newcommand{\Z}{\mathbb{Z}}
\newcommand{\ggT}{\text{ggT}}
\newcommand{\kgV}{\text{kgV}}
%

\author{Tutor: not known}
\date{}
\title{Max Wisniewski , Alexander Steen}

\begin{document}

\lstset{language=Java, basicstyle=\ttfamily\fontsize{10pt}{10pt}\selectfont\upshape, commentstyle=\rmfamily\slshape, keywordstyle=\rmfamily\bfseries, breaklines=true, frame=single, xleftmargin=3mm, xrightmargin=3mm, tabsize=2}

\maketitle
\thispagestyle{fancy}

%% -------------------------------------------
%%                 AUFGABE 1
%% -------------------------------------------

\section*{Aufgabe 1}

Es seien $a,b \in \N \setminus \{ 0,1 \}$ natürliche Zahlen, sowie
$$
a= p_1^{k_1}\cdot ...\cdot p_s^{k_s} \quad \text{und} \quad b = q_1^{l_1} \cdot ... \cdot q_s^{l_t}
$$
\begin{enumerate}[\bfseries a)]

\item Geben Sie die Primfaktorzerlegung von $\ggT (a,b)$ und $\kgV (a,b)$ an.\\

\item Beweisen Sie die Formel

$$
\ggT (a,b) \cdot \kgV (a,b) = a \cdot b
$$

\end{enumerate}
\section*{Aufgabe 2}

\section*{Aufgabe 3}

\label{LastPage}
\end{document}