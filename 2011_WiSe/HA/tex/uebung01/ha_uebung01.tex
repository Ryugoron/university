\documentclass[11pt,a4paper,ngerman]{article}
\usepackage[bottom=2.5cm,top=2.5cm]{geometry} 
\usepackage{babel}
\usepackage[utf8]{inputenc} 
\usepackage[T1]{fontenc} 
\usepackage{ae} 
\usepackage{amssymb} 
\usepackage{amsmath} 
\usepackage{graphicx}
\usepackage{fancyhdr}
\usepackage{fancyref}
\usepackage{listings}
\usepackage{xcolor}
\usepackage{paralist}
\usepackage[pdftex, bookmarks=false, pdfstartview={FitH}, linkbordercolor=white]{hyperref}
\usepackage{fancyhdr}
\pagestyle{fancy}
\fancyhead[C]{Höhere Algorithmik}
\fancyhead[L]{Übung Nr. 1}
\fancyhead[R]{WS 2011/12}
\fancyfoot{}
\fancyfoot[L]{}
\fancyfoot[C]{\thepage / \pageref{LastPage}}
\renewcommand{\footrulewidth}{0.5pt}
\renewcommand{\headrulewidth}{0.5pt}
\setlength{\parindent}{0pt} 
\setlength{\headheight}{15pt}

\author{Tutor: Tilmann}
\date{}
\title{Max Wisniewski , Alexander Steen}

\begin{document}

\lstset{language=Java, basicstyle=\ttfamily\fontsize{10pt}{10pt}\selectfont\upshape, commentstyle=\rmfamily\slshape, keywordstyle=\rmfamily\bfseries, breaklines=true, frame=single, xleftmargin=3mm, xrightmargin=3mm, tabsize=2}

\maketitle
\thispagestyle{fancy}

\section*{Aufgabe 1}

Die Funktionen der Aufgabe sollen derart geordnet werden, so dass $ g_i \in \Omega (g_{i+1}) $ gilt. Geben Sie auch an, wenn sogar $ g_i = \Theta (g_{i+1}) $ ist. 

\vspace{12px}

Die Folge erfüllt die Eigenschaft und enthält die Elemente, die geordent werden müssen:\\

$(g_i)_{1 \leq i \leq 10} = ( n^{\frac{1}{\log{n}}} , \ln{n} , \log^2{n}  , (\sqrt{2})^{\log{n}}  ,  n^2 ,  4^{log{n}} , (\lceil \log{n} \rceil)! ,   n^{log{(log{(n)})}} ,  2^n  , 2^{(2^n)})$

\begin{description}
\item{\textbf{Beweis}} 
\begin{enumerate}

\item
$  2^n  \in \Omega (2^{(2^n)})$ : \\
$\underset{n\rightarrow\infty}{lim} \frac{2^n}{2^{(2^n)}} = \underset{n\rightarrow\infty}{lim} \frac{2^n \cdot 1}{2^n \cdot 2^{(2^n) - n}} = $
$ \underset{n\rightarrow\infty}{lim} \frac{1}{2^{(2^n) - n}}$\\
Da $2^n$ stärker wächst als $n$, gilt:\\
$ \underset{n\rightarrow\infty}{lim} \frac{1}{2^{(2^n) - n}} = 0$\\
Damit gilt nach konvergenz Kriterium $2^n \in \Omega (2^{(2^n)})\Rightarrow g_9 \in \Omega (g_{10})$

\item
$ n^{\log{(\log{(n)})}} \in \Omega (2^n)$: \\
Es gilt: $n^{\log \log n} = 2^{\log n \cdot \log \log n} = 2^{\log (n + \log n)}$\\
$$
\underset{n\rightarrow\infty}{lim} \frac{ 2^{\log (n + \log n)}}{2^n} = \underset{n\rightarrow\infty}{lim} 2^{n - \log (n + \log n)} = 0 \Leftarrow \underset{n\rightarrow\infty}{lim} n - \log (n + \log n) = \infty
$$
Dies ist der Fall, wenn $n$ stärker wächst als $ \log (n + \log n) $.
\item
$(\lceil \log{n} \rceil)! \in \Omega (n^{log{(\log{(n)})}})$: \\
Später

\item
$4^{\log{(n)}} \in \Omega ((\lceil \log{n} \rceil)!)$: \\
Wie gehabt

\item
$n^2 \in \Theta (4^{\log{(n)}})$: \\
Wir zeigen an dieser Stelle, das gilt: $n^2 = 4^{\log{(n)}}$. Damit gilt die Beziehung für $\Theta$ sofort.\\
$4^{\log n} = (2^2)^{\log n} =  2^{2\cdot \log n} = 2^{\log n^2} = n^2$\\
$\Rightarrow g_5 \in \Theta (g_6)$

\item
$(\sqrt{2})^{\log{n}} \in \Theta (n^2)$:\\
Ersteinmal gilt : $(\sqrt{2})^{\log{n}} = (2^{\frac{1}{2}})^{\log n} = 2^{\frac{1}{2} \cdot \log n} = 2^{\log \sqrt{n}} = \sqrt{2}$\\
Nun wenden wir wieder das Konvergenzkreterium an: \\
$\underset{n\rightarrow\infty}{\lim} \frac{\sqrt{2}}{n^2} = \underset{n\rightarrow\infty}{\lim} \frac{1}{n^{1.5}} = 0$\\
$\Rightarrow (\sqrt{2})^{\log{n}} \in \Theta (n^2) \Rightarrow g_4 \in \Omega (g_5)$

\item
$\log^2 n \in \Omega (\sqrt{n})$: \\
$$\underset{n\rightarrow\infty}{\lim} \frac{\log^2 n}{\sqrt{n}} = \underset{n\rightarrow\infty}{\lim} \frac{\log^2 n}{\log (2^{\sqrt{n}})} = 0$$
Sagt Wolframalpha

\item
$\ln n \in \Omega (\log^2 n)$: \\
$$\underset{n\rightarrow\infty}{\lim} \frac{\ln n}{\log^2 n} = \underset{n\rightarrow\infty}{\lim} \frac{(\ln 2)(\log n)}{\log^2 n} = \underset{n\rightarrow\infty}{\lim} \frac{\ln 2}{\log n} = 0$$
Nach Konvergenzkriterium gilt: $g_2 \in \Omega(g_3)$
\item
$n^{\frac{1}{\log n}} \in \Omega (\ln n)$: \\
Zunächst gilt: $n^{\frac{1}{\log n}} = 2^{\frac{\log n}{\log n}} = 2$.\\
Daraus folgt offensichtlich:\\
$\underset{n\rightarrow\infty}{\lim} \frac{2}{\ln n} = 0 \Rightarrow g_1 \in \Omega (g_2)$

\end{enumerate}
\end{description}

\section*{Aufgabe 2}

Sei $n$ die Anzahl der verschiedenen Sammelbilder. Sei $X$ Zufallsvariable für die Anzahl der benötigten Packungen Müsli, bis wir alle $n$ Sammelbilder haben. Die Wahrscheinlichkeit für ein bestimmtes Bild ist gleichverteilt.

\subsection*{(a)}

tbd

\subsection*{(b)}

 tbd

\subsection*{(c)}

tbd


\section*{Aufgabe 3}

\subsection*{(a)}

\begin{description}

\item{\bfseries Selectionsort}

\begin{description}

\item{\bfseries Beschreibung:} Solange bis die zu sortierende Liste leer ist, sucht Selectionsort das kleinste Element, löscht dieses aus der Liste und fügt es hinten an eine am Anfang neu erzeugte Liste an.

\item{\bfseries Laufzeit:} Um das kleinste Element in einer unsortierten Liste zu finden, muss man sich jedes Element einmal ansehen. Bei einer Liste der Länge $n$ bedeutet dies $T_{smallest}(n) = n$.\\
Nach der obigen Beschreibung nehmen wir $n$ mal das kleinste Element aus der Liste, wobei die Liste in jedem Schritt um ein Element schrumpft. Um das Element hinten an die neue Liste zu hängen, können wir konstante laufzeit annehmen. (LinkedList rear oder Array auf das ende + 1).\\
$$
\begin{array}{rcl}
\Rightarrow T(n) &=& \sum_{i=0}^{n-1} i \\
&=& \frac{n\cdot (n-1)}{2} \\
&=& \frac{1}{2} \left( n^2 - n \right)
\end{array}
$$

\end{description}

\item{\bfseries Mergesort}

\begin{description}

\item{\bfseries Beschreibung:} Teile die Liste rekursiv in 2 gleichgroße Listen, bis die Teile trivial zu lösen sind, z.B. bei einem Element, und vereinige danach die sortierten Teillisten auf dem Weg den Rekursionsbaum hoch wieder, in der richtigen Reihenfolge.

\item{\bfseries Laufzeit:} In jedem Schritt Teilen wir die Liste in 2 Teile. Haben wir Listen der Größe 1 erreicht, können wir aufhören. Diese sind ohne weiteres sortiert. Nach jedem Teilen benötigen wir noch $n$ Schritte um beide Listen zu mergen (da wir im Worst-Case immer abwechselnd ein Element aus den jeweiligen Listen nehmen müssen)
$$
\begin{array}{rcl}
\Rightarrow T(1) &=& 0  \\
T(2) &=& 2 \text{ ,aus folgender Formel ausgerechnet}\\
T(n) &=& T(\left\lfloor \frac{n}{2} \right\rfloor)+ T(\left\lceil \frac{n}{2} \right\rceil) + n  \quad , n>1\\
\end{array}
$$
Da $n$ eine Zweierpotenz ist, können wir $n = 2^k$ substituieren.
$$\Rightarrow T(2^k) = T(\left\lfloor 2^{k-1} \right\rfloor)+ T(\left\lceil 2^{k-1} \right\rceil) + n$$
Nun gilt $\forall a > 0 : 2^{n-1} \in \mathbb{N} $ und da $n>1 \Rightarrow k>0$.
$$\Rightarrow T(2^k) = 2\cdot T(2^{k-1}) + n$$
Nun stellen wir die charakteristische Gleichung für den homogenen Teil auf:
$$x^k = 2\cdot x^{k-1} \Leftrightarrow x = 2$$
Als Lösung des inhomogenen Teils wählen wir $k \cdot 2^k$:
$$
\begin{array}{rcl}
T(2^k) &=& c_1 \cdot 2^k + c_2 \cdot k \cdot 2^k\\
\stackrel{\text{resub.}}{\Leftrightarrow} T(n) &=& c_1 \cdot n + c_2 \cdot n \cdot \log n\\
&T(1) = 0&\\
T(1) = 0 &=& c_1 \cdot 1 + c_2 \cdot 1 \cdot \log 1\\
0&=&c_1\\
&T(2) = 2&\\
T(2) = 2 &=& 0 \cdot 1 + c_2 \cdot 2 \cdot \log 2\\
2&=&c_2 \cdot 2\\
c_2 &=& 1 
\end{array}
$$
Setzen wir dies in unsere aufgelöste Form ein erhalten wir:
$$
\begin{array}{rcl}
T(n) &=& c_1 \cdot n + c_2 \cdot n \cdot \log n\\ &=& 0 \cdot n + 1 \cdot n \log n \\ &=& n \cdot \log n
\end{array}
$$

\end{description}

\end{description}

\subsection*{(b)}

tbd

\subsection*{(c)}

tbd

\label{LastPage}

\end{document}