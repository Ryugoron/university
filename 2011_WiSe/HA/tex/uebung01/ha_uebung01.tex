\documentclass[11pt,a4paper,ngerman]{article}
\usepackage[bottom=2.5cm,top=2.5cm]{geometry} 
\usepackage{babel}
\usepackage[utf8]{inputenc} 
\usepackage[T1]{fontenc} 
\usepackage{ae} 
\usepackage{amssymb} 
\usepackage{amsmath} 
\usepackage{graphicx}
\usepackage{fancyhdr}
\usepackage{fancyref}
\usepackage{listings}
\usepackage{xcolor}
\usepackage{paralist}
\usepackage[pdftex, bookmarks=false, pdfstartview={FitH}, linkbordercolor=white]{hyperref}
\usepackage{fancyhdr}
\pagestyle{fancy}
\fancyhead[C]{Höhere Algorithmik}
\fancyhead[L]{Übung Nr. 1}
\fancyhead[R]{WS 2011/12}
\fancyfoot{}
\fancyfoot[L]{}
\fancyfoot[C]{\thepage}
\renewcommand{\footrulewidth}{0.5pt}
\renewcommand{\headrulewidth}{0.5pt}
\setlength{\parindent}{0pt} 
\setlength{\headheight}{15pt}

\author{Tutor: Tilmann}
\date{}
\title{Max Wisniewski , Alexander Steen}

\begin{document}

\lstset{language=Java, basicstyle=\ttfamily\fontsize{10pt}{10pt}\selectfont\upshape, commentstyle=\rmfamily\slshape, keywordstyle=\rmfamily\bfseries, breaklines=true, frame=single, xleftmargin=3mm, xrightmargin=3mm, tabsize=2}

\maketitle
\thispagestyle{fancy}

\section*{Aufgabe 1}

Die Funktionen der Aufgabe sollen derart geordnet werden, so dass $ g_i \in \Omega (g_{i+1}) $ gilt. Geben Sie auch an, wenn sogar $ g_i = \Theta (g_{i+1}) $ ist. 

\vspace{12px}

Die Folge erfüllt die Eigenschaft und enthält die Elemente, die geordent werden müssen:\\

$(g_i)_{1 \leq i \leq 10} = ( n^{\frac{1}{\log{n}}} , \ln{n} , \log^2{n}  , (\sqrt{2})^{\log{n}}  ,  n^2 ,  4^{log{n}} , (\lceil \log{n} \rceil)! ,   n^{log{(log{(n)})}} ,  2^n  , 2^{(2^n)})$

\begin{description}
\item{\textbf{Beweis}} 
\begin{enumerate}

\item
$  2^n  \in \Omega (2^{(2^n)})$ : \\
$\underset{n\rightarrow\infty}{lim} \frac{2^n}{2^{(2^n)}} = \underset{n\rightarrow\infty}{lim} \frac{2^n \cdot 1}{2^n \cdot 2^{(2^n) - n}} = $
$ \underset{n\rightarrow\infty}{lim} \frac{1}{2^{(2^n) - n}}$\\
Da $2^n$ stärker wächst als $n$, gilt:\\
$ \underset{n\rightarrow\infty}{lim} \frac{1}{2^{(2^n) - n}} = 0$\\
Damit gilt nach konvergenz Kriterium $2^n \in \Omega (2^{(2^n)})\Rightarrow g_9 \in \Omega (g_{10})$

\item
$ n^{log{(\log{(n)})}} \in \Omega (2^n)$: \\
Kein Plan

\item
$(\lceil \log{n} \rceil)! \in \Omega (n^{log{(\log{(n)})}})$: \\
Später

\item
$4^{\log{(n)}} \in \Omega ((\lceil \log{n} \rceil)!)$: \\
Wie gehabt

\item
$n^2 \in \Theta (4^{\log{(n)}})$: \\
Wir zeigen an dieser Stelle, das gilt: $n^2 = 4^{\log{(n)}}$. Damit gilt die Beziehung für $\Theta$ sofort.\\
$4^{\log n} = (2^2)^{\log n} =  2^{2\cdot \log n} = 2^{\log n^2} = n^2$\\
$\Rightarrow g_5 \in \Theta (g_6)$

\item
$(\sqrt{2})^{\log{n}} \in \Theta (n^2)$:\\
Ersteinmal gilt : $(\sqrt{2})^{\log{n}} = (2^{\frac{1}{2}})^{\log n} = 2^{\frac{1}{2} \cdot \log n} = 2^{\log \sqrt{n}} = \sqrt{2}$\\
Nun wenden wir wieder das Konvergenzkreterium an: \\
$\underset{n\rightarrow\infty}{\lim} \frac{\sqrt{2}}{n^2} = \underset{n\rightarrow\infty}{\lim} \frac{1}{n^{1.5}} = 0$\\
$\Rightarrow (\sqrt{2})^{\log{n}} \in \Theta (n^2) \Rightarrow g_4 \in \Omega (g_5)$

\item
$\log^2 n \in \Omega (\sqrt{2})$: \\
tbd

\item
$\ln n \in \Omega (\log^2 n)$: \\
tbd

\item
$n^{\frac{1}{\log n}} \in \Omega (\ln n)$: \\
Zunächst gilt: $n^{\frac{1}{\log n}} = 2^{\frac{\log n}{\log n}} = 2$.\\
Daraus folgt offensichtlich:\\
$\underset{n\rightarrow\infty}{\lim} \frac{2}{\ln n} = 0 \Rightarrow g_1 \in \Omega (g_2)$

\end{enumerate}
\end{description}

\section*{Aufgabe 2}

Sei $n$ die Anzahl der verschiedenen Sammelbilder. Sei $X$ Zufallsvariable für die Anzahl der benötigten Packungen Müsli, bis wir alle $n$ Sammelbilder haben. Die Wahrscheinlichkeit für ein bestimmtes Bild ist gleichverteilt.

\subsection*{(a)}

tbd

\subsection*{(b)}

 tbd

\subsection*{(c)}

tbd


\section*{Aufgabe 3}

\subsection*{(a)}

tbd

\subsection*{(b)}

tbd

\subsection*{(c)}

tbd

\end{document}