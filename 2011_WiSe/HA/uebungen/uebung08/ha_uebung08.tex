\documentclass[11pt,a4paper,ngerman]{article}
\usepackage[bottom=2.5cm,top=2.5cm]{geometry} 
\usepackage{babel}
\usepackage[utf8]{inputenc} 
\usepackage[T1]{fontenc} 
\usepackage{ae} 
\usepackage{amssymb} 
\usepackage{amsmath} 
\usepackage{graphicx}
\usepackage{fancyhdr}
\usepackage{fancyref}
\usepackage{listings}
\usepackage{xcolor}
\usepackage{paralist}

%%\usepackage[pdftex, bookmarks=false, pdfstartview={FitH}, linkbordercolor=white]{hyperref}
\usepackage{fancyhdr}
\pagestyle{fancy}
\fancyhead[C]{Höhere Algorithmik}
\fancyhead[L]{Übung Nr. 7}
\fancyhead[R]{WS 2011/12}
\fancyfoot{}
\fancyfoot[L]{}
\fancyfoot[C]{\thepage$\,$ von \pageref{LastPage}}
\renewcommand{\footrulewidth}{0.5pt}
\renewcommand{\headrulewidth}{0.5pt}
\setlength{\parindent}{0pt} 
\setlength{\headheight}{15pt}

\author{Tutor: Lena Schlipf}
\date{}
\title{Max Wisniewski , Alexander Steen}

\begin{document}

\lstset{language=Java, basicstyle=\ttfamily\fontsize{10pt}{10pt}\selectfont\upshape, commentstyle=\rmfamily\slshape, keywordstyle=\rmfamily\bfseries, breaklines=true, frame=single, xleftmargin=3mm, xrightmargin=3mm, tabsize=2, mathescape=true}

\maketitle
\thispagestyle{fancy}

%% ----------------------------------
%%		AUFGABE 1
%% ----------------------------------
\subsection*{Aufgabe 1 \mdseries Das Offline-Minimum-Problem}

Wir verwalten eine Menge $T \subseteq \{ 1, ... , n\}$, welche durch die Operationen \emph{insert} und \emph{extract-min} verändert wird. Inizial gilt $T=\emptyset$. Wir bekommen dazu eine Folge $S$ von insert und extract Operationen, in der wir jedes Element in $\{1,...,n\}$ \emph{genau einmal} einfügen.

\begin{enumerate}[\bfseries a)]

\item Betrachten  Sie die folgende Operationsfolge
$$
4,8,E,3,E,9,2,6,E,E,E,1,7,E,5 \; ,
$$
wobei eine Zahl $i$ bedeutet, dass \emph{insert(i)} ausgeführt wird. Geben Sie die Ergebnisse der einzelnen extract-min Operationen an.

\textbf{Lösung:}\\
\begin{enumerate}[1 {. extract-min:} ]

\item 4

\item 3

\item 2

\item 6

\item 8

\item 1

\end{enumerate}

\item Gegeben ist eine Algorithmus und eine Folge von Operationen
$$
I_1,E,I_2,E,I_3,E ...,I_m,E,I_{m-1}
$$
wobei jedes $I_{j}$ eine Folge von insert Operationen darstellt und E wieder eine extract-min Operation. $K(j)$ ist die Menge, die die Zahlen aus $I_j$ enthält.\\

\textbf{Ausgeführt:}\\
Wir haben 7 Insertfolgen, dass heißt $m+1=7$.\\
$I_1 = 4,8 \; I_2 = 3 \; I_3 = 9,2,6 \; I_4 = I_5 = [] \; I_6 = 1,7 \; I_7 = 5$
\begin{enumerate}[{i=}1 $\rightarrow$]

\item $j=6$ : em[6] = 1, l=7 $\rightarrow K(7) = \{1,7,5\} $

\item $j=3$ : em[3] = 2, l=4 $\rightarrow K(4) = \{ 9,2,6 \}$

\item $j=2$ : em[2] = 3, l=4 $\rightarrow K(4) = \{ 9, 2, 6, 3 \}$

\item $j=1$ : em[1] = 4, l=4 $\rightarrow K(4) = \{ 9, 2, 6, 3, 4, 8\}$

\item $j=7$ : nichts, da $7 = m+1$

\item $j=4$ : em[4] = 6, l=5 $\rightarrow K(5) = \{ 9, 2, 6, 3, 4, 8 \}$

\item $j=7$ : nichts, da $7 = m+1$

\item $j = 5$ : em[5] = 8, l=7 $\rightarrow K(7) = \{ 1, 2, 3, 4,  5, 7, 9 \}$

\item $j=7$ 

Für das i-te Extract-min steht sein Ergebnis nun in em[i]. Wir erhalten die selbe Folge von Ergebnissen, wie wir sie in a) berechnet haben.

\end{enumerate}

\textbf{Korrektheit:}\\

\begin{description}

\item{\bfseries Beh.:} $\forall i \leq n \; : \; (i = E_k \Rightarrow em[k] = i) \lor i$ wird nie genommen, \\wobei $E_k$ für das Ergebnis der k-ten extract-min Operation steht. 

\item{\bfseries I.A.:} Wir haben noch keine Vereinigungen gemacht. D.h. jedes K(j) existiert noch und ist die Menge der Zahlen, die in $I_j$ eingefügt wurden. Wenn wir nun ein $E_k$ haben, dass die 1 sieht, (d.h. $k<m+1$), dann wird dieses $E_k$ es nehmen. Da 1 ein globales Minimum ist, wird ist das auch ein lokales Minimum, wenn danach noch ein extract ausgeführt wird.\\
Sollte kein extract mehr ausgeführt werden, so gilt $1 \in K(l), l=m+1$. Der Algorithmus trägt es nicht ein.\\

Damit gilt unsere Behauptung für $i=1$.

\item{\bfseries I.S.:} Nach Induktionsvorraussetzung wurden die Zahlen ${1,..,i-1}$ schon ihren Extract Operationen zugewiesen. Sie $R = \{ E_k | E_k = \{ 1, ..., i-1 \} \}$


\end{description}

\item Beschreiben Sie, wie man den Algorithmus aus b) effizient mit einer Union-Find Struktur implementieren kann und analysieren Sie die Laufzeit.\\

\textbf{Lösung:}\\

\end{enumerate}

%% ----------------------------------
%%		AUFGABE 2
%% ----------------------------------
\subsection*{Aufgabe 2 \mdseries Amortisierte Analyse}

\begin{enumerate}[\bfseries a)]

\item Gegeben sei ein elektirsches Binärzählwerk mit beliebig vielen Ziffern aus der Menge $\{ 0, 1\}$. Das Umschalten einer Ziffer kostet eine Stromheinheit. Wie viele Stromeinheiten kostet es insgesamt, wenn man das Zählwerk von 0 bis $n$ aufsteigend zählen lässt? Was sind die amortisierten Kosten pro Zählvorgang?\\

\textbf{Lösung:}\\
Gesamtkosten: Wenn einen binären Zähler hochzählen, dann müssen wir eine Ziffer an der Stelle $k$ genau dann ändern, wenn wir die Ziffer an der Stelle $k-1$ von 1 auf 0 ändern. Da wir jede Ziffer einmal auf 0 und einmal auf 1 setzen können, wird die Ziffer an der Stelle $k$ halb so oft umgesetzt werden, wie die Ziffer an der Stelle $k-1$ werden. Die erste Ziffer an der Stelle 0, wir bei jedem erhöhen verändert. Dies ergibt für uns, dass wir die 0te Stelle $n$ mal ändern müssen, die erste Stelle $\frac{n}{2}$, die zweite Stelle $\frac{n}{4}$ und an der kten Stelle wird das Bit $\frac{n}{2^k}$ geändert.\\

In der Summe werden wir also:
$$
\sum_{i=0}^{\log n} \frac{n}{2^i} = n \cdot \sum_{i=0}^{\log n} \frac{1}{2^i} \leq n \cdot \sum_{i=0}^{\infty} \left( \frac{1}{2} \right)^i = n\cdot 2
$$
Bits ändern müssen. Dies hat zur Folge, da jedes ändern 1ne Stromeinheit kostet, das wir beim Hochzählen höchstens $2n$ Stromeinheiten brauchen werden.\\

Amortisierte Kosten: Wir haben eine Folge der Länge $n$ und die Kosten dieser Folge betragen $2n$. Das ergibt amortisierte Laufzeit von $T_{amo} = \frac{2n}{n} = 2 = O(1)$.\\

Nach der Analyse könnten wir nun auch aussagen, dass wenn wir für jedes Zählen 2 Stromeinheiten zahlen, immer über 0 bleiben würden.

\item Entwicklen und analysieren Sie eine Methode um ein Array zu verwalten, welche seine Größe dynamisch ändern kann.

\textbf{Lösung:}\\

voll -> verdoppeln\\
1/3 -> halbieren

\end{enumerate}

%% ---------------------------------------------
%% 			AUFGABE 3
%% ---------------------------------------------
\subsection*{Aufgabe 3 \mdseries Wahrscheinlichkeitsrechnung und Hashing}

\begin{enumerate}[\bfseries a)]

\item Auf der ausgelassenen Weihanchtsfeier der Teilnehmer der HA-Vorlesung gibt es einen Julklapp. Die Teilnehmer bringen jeweils ein Geschenk mit und alle Geschenke werdenin einen großen Sack getan. Danach zieht jeder der Partygäste zufällig ein Geschenk aus dem Sack. Was ist die erwartete Anzahl der Leute, die ihr eigenes Geschenk ziehen?\\

\textbf{Lösung:}\\
Wir konstruieren uns eine Indikatorvariable, ob die k-te Person ihr eigenes Geschenk zieht. Die Wahrscheinlichkeit bestimmen wir unabhängig von den $k-1$ vorrigen Personen, so dass wir über alle Indikatorvariablen eine Summer bilden können und diese immer noch mit der Linearität des Erwartungswertes nach aussen ziehen können.\\

Sei
$$
X_k = \left\{ \begin{array}{lr} 1 &,\text{k-te Person zieht ihr geschenk}\\ 0 &, \text{sonst}\end{array} \right. .
$$
Die Wahrscheinlichkeit $Pr(X = X_k)$  muss nun mit einberechnen, dass die $k-1$ Gäste zuvor das Geschenk nicht gezogen haben. Da bei jedem Zug die Anzahl der Geschänke sich verringert haben wir die Wahrscheinlichkeit:
$$
Pr(X_k = 1) = \frac{n-1}{n} \cdot \frac{n-2}{n-1} \cdot ... \cdot \frac{n - k}{n-k +1} \cdot \frac{1}{n-k} = \frac{(n-1)! \cdot (n-k)! }{(n-k -1)! n! (n-k)} = \frac{1}{n}
$$
Wir sehen, dass  die Wahrscheinlichkeit nicht mehr von $k$ sondern nur noch von $n$ abhängt. Der Erwartungswert ist somit:\\
$$
E \left[ \sum_{i=1}^{n} X_i \right] = \sum_{i=1}^n E \left[ X_i \right] \stackrel{Indikator}{=} \sum_{i=1}^n Pr(X_i=1) = \sum_{i=1}^n \frac{1}{n} = \frac{n}{n} = 1
$$

Wir erwarten also, dass 1ne Person ihr eigenes Geschenk zieht.

\item Sei $N$ die Größe einer gegebenen Hashtabelle und $n$ beliebig. Zeigen Sie, dass für jede Schlüsselmenge $K$ mit $|K| \geq (n-1)N+1$ und jede Hashfunktion $h \; : \; K \rightarrow \{ 0 , ... , N-1\}$ eine Menge $S \subseteq K$ mit $|S| \geq n$ existiert, so dass alle Elemente von $S$ auf denselben Eintrag der Hashtabelle abgebildet werden.\\
Was bedeutet das für die worst-case Laufzeit von Hashing mit Verkettung.\\

\textbf{Lösung:}\\
Diese Aufgabe lösen wir dem Taubenschlagprinzip. Bei diesem gilt: \\
Sei 
$f \; : \; A \rightarrow B$, dann $$\exists y \in B \; : \; \#\{x \in A | f(x) = y \} = \left\lceil \frac{ \#A}{ \#f(A)} \right\rceil.$$
Nun haben wir $f=h$, $A = K$ $f(A) \subseteq \{0,...N\}$ mit $\#A = \# K \geq (n-1)N+1$ und $\# f(A) \leq \# \{0,...,N-1 \} = N$.\\

Nach dem Taubenschlagprinzip existiert nun $t \in \{0, ..., N-1 \}$, so dass für $S_t = \{ x \in K \; | \; h(x) = t \} \subseteq K$  gilt:
$$
\#S_t \geq \left\lceil \frac{\#K}{\#\{0,...,N-1\}} \right\rceil = \left\lceil \frac{(n-1)N+1}{N} \right\rceil = \left\lceil n-1 + \frac{1}{N} \right\rceil = n.
$$

Diese $S_t \subseteq K$ können wir nun als unser $S$ wählen, so dass wir die Vorraussetzung der Aussage erfüllen.\\

Für die Laufzeit bedeutet das, wenn wir gerade einen Wert $x \in K$ suchen, so dass $h(x) = t$ gilt, dann müssen wir im schlimmsten Fall $n$ Schritt machen, da wir die Position im Array in $O(1)$ finden, aber danach eine Liste der Länge $n$ durchsuchen müssen.\\

Dies gilt auch für Löschen, da wir ebenfalls das Element suchen müssen.\\

Einfügen geht weiterhin in $O(1)$, da man in $O(1)$ in verkette Listen einfügen kann.

\end{enumerate}

\label{LastPage}
\end{document}
