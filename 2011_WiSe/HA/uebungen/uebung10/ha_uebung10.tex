\documentclass[11pt,a4paper,ngerman]{article}
\usepackage[bottom=2.5cm,top=2.5cm]{geometry} 
\usepackage{babel}
\usepackage[utf8]{inputenc} 
\usepackage[T1]{fontenc} 
\usepackage{ae} 
\usepackage{amssymb} 
\usepackage{amsmath} 
\usepackage{graphicx}
\usepackage{fancyhdr}
\usepackage{fancyref}
\usepackage{listings}
\usepackage{xcolor}
\usepackage{paralist}

%%\usepackage[pdftex, bookmarks=false, pdfstartview={FitH}, linkbordercolor=white]{hyperref}
\usepackage{fancyhdr}
\pagestyle{fancy}
\fancyhead[C]{Höhere Algorithmik}
\fancyhead[L]{Übung Nr. 10}
\fancyhead[R]{WS 2011/12}
\fancyfoot{}
\fancyfoot[L]{}
\fancyfoot[C]{\thepage$\,$ von \pageref{LastPage}}
\renewcommand{\footrulewidth}{0.5pt}
\renewcommand{\headrulewidth}{0.5pt}
\setlength{\parindent}{0pt} 
\setlength{\headheight}{15pt}

\author{Tutor: Lena Schlipf}
\date{}
\title{Max Wisniewski , Alexander Steen}

\begin{document}

\lstset{language=Java, basicstyle=\ttfamily\fontsize{10pt}{10pt}\selectfont\upshape, commentstyle=\rmfamily\slshape, keywordstyle=\rmfamily\bfseries, breaklines=true, frame=single, xleftmargin=3mm, xrightmargin=3mm, tabsize=2, mathescape=true}

\maketitle
\thispagestyle{fancy}

%% ----------------------------------
%%		AUFGABE 1
%% ----------------------------------
\subsection*{Aufgabe 1 \mdseries Hashing mit Verkettung}


\begin{enumerate}[\bfseries a)]

\item Z.z: Es gilt für $i = 0, ..., n - 1$ und $r = 0, ...,n$,
$$ Pr[Q_i = r] = \left( \frac{1}{n} \right)^r \left(1-  \frac{1}{n} \right)^{n-r} \binom{n}{r}$$
\\
Da es sich hier um eine Binomialverteilung handelt, kann die Wahrscheinlichkeit mit der gegebenen Formel berechnet werden (wie aus der Schule bekannt).

\mbox{}\hfill $\square$

\item Z.z: $ Pr[\max_{i=0}^{n-1} Q_i = r] \leq n \cdot Pr[Q_0 = r] $ 



\mbox{}\hfill $\square$

\item Mit Hilfe der Abschätzung $\binom{n}{r} \leq \left( \frac{ne}{r} \right)^r$ ist zu zeigen: $ Pr[Q_0 = r] \leq \frac{e^r}{r^r} $

$$ 
\begin{array}{lcl}
Pr[Q_0 = r] & = & \left( \frac{1}{n} \right)^r \left(1-  \frac{1}{n} \right)^{n-r} \binom{n}{r} \\
 & \leq & \left( \frac{1}{n} \right)^r \left(1-  \frac{1}{n} \right)^{n-r} \left( \frac{ne}{r} \right)^r \\
 & = & \left( \frac{1}{n} \cdot \frac{ne}{r} \right)^r \left(1-  \frac{1}{n} \right)^{n-r} \\
 & = & \left( \frac{e}{r} \right)^r \left(1-  \frac{1}{n} \right)^{n-r} \\
 & \stackrel{*}{\leq} & \frac{e^r}{r^r}
\end{array}
$$
(*) gilt, da $1 - \frac{1}{n} < 1$ ist und damit auch insbesondere $\left(1-  \frac{1}{n} \right)^{n-r} < 1$ gilt.

\mbox{}\hfill $\square$

\item Sei $r_0 := c \log n / \log \log n $ für ein $c > 1$. Z.z:
$$ \exists c > 1 \forall r \geq r_0 : Pr[\max_{i=0}^{n-1} Q_i \geq r_0] \leq 1/n $$

\mbox{}\hfill $\square$

\item Z.z: 
$$ E[\max_{i=0}^{n-1} Q_i] \leq r_0 \cdot Pr[\max_{i=0}^{n-1} Q_i \leq r_0] + n \cdot Pr[\max_{i=0}^{n-1} Q_i \geq r_0] $$
Folgern Sie: $E[\max_{i=0}^{n-1} Q_i] = O(\log n / \log \log n)$. Ist das ein Widerspruch zur Vorlesung?

\mbox{}\hfill $\square$
\end{enumerate}


%% ----------------------------------
%%		AUFGABE 2
%% ----------------------------------
\subsection*{Aufgabe 2 \mdseries Page-Rank}

\begin{enumerate}[\bfseries a)]

\item Geben Sie die modifizierte Adjazenzmatrix $A'$ für den Graphen an.

$$ A' = 
\begin{pmatrix}
 0 & 1 & 0 & 0 \\
 0 & 0 & \frac{1}{2} & \frac{1}{2} \\
 1 & 0 & 0 & 0 \\
 0 & 0 & 1 & 0 \\
\end{pmatrix}
$$

\item Bestimmen Sie den Page-Rank-Score für jeden Knoten algebraisch durch Lösen des Gleichungssystems (verwenden Sie den Dämpfungsfaktor $0.25$).

Sei $A''$ die gedämpfte Matrix, für die sich ergibt:
$$ A'' = 0.75 \cdot A' + \frac{1}{16} 1_{4 \times 4} = \frac{1}{16} \cdot 
\begin{pmatrix}
 1 & 13 & 1 & 1 \\
 1 & 1 & 7 & 7 \\
 13 & 1 & 1 & 1 \\
 1 & 1 & 13 & 1 \\
\end{pmatrix}
$$

Für den Page-Rank-Score lösen wir folgendes Gleichungssystem:
$$ v^{*} A'' = v^{*}$$
Dabei ist $v^{*} = (v_1, v_2, v_3, v_4) $ der Page-Rank-Vektor und Eigenvektor der Matrix $A''$ zum Eigenwert 1, also berechnen wir $Ker(A'' - E_4)^{t}$, also:

$$ (A'' - E_4)^{t} v^{*} = 0 $$
Mit Hilfe des Gaußverfahren lösen wir dann:
$$
\frac{1}{16} \cdot
\begin{pmatrix}
 -15 & 1 & 13 & 1 \\
 13 & -15 & 1 & 1 \\
 1 & 7 & -15 & 13 \\
 1 & 7 & 1 & -15
\end{pmatrix}
  \rightsquigarrow
\begin{pmatrix}
 1 & 0 & 0 & - \frac{359}{212} \\
 0 & 1 & 0 & - \frac{175}{106} \\
 0 & 0 & 1 & - \frac{7}{4} \\
 0 & 0 & 0 & 0
\end{pmatrix}
$$

Also ergibt sich als Ergebnisvektor $\left( \frac{359}{212}, \frac{175}{106}, \frac{7}{4}, 1 \right)$ und damit nach Normalisierung $v^{*} = \left( \frac{359}{1292}, \frac{175}{646}, \frac{371}{1292}, \frac{53}{323} \right) $
\item Führen Sie den iterativen Page-Rank-Algorithmus für das Beispiel durch (wieder mit Dämpfungsfaktor $0.25$). Wie viele Iterationen sind notwendig, bis der absolute Fehler kleiner als $0.001$ ist?

$$
\begin{array}{crcl}
\quad &
\left( \frac{1}{4}, \frac{1}{4}, \frac{1}{4}, \frac{1}{4} \right) \cdot A'' & = &
\left( \frac{1}{4}, \frac{1}{4}, \frac{11}{32}, \frac{5}{32} \right) \\
\rightsquigarrow &
\left( \frac{1}{4}, \frac{1}{4}, \frac{11}{32}, \frac{5}{32} \right) \cdot A'' & = &
\left( \frac{41}{128}, \frac{1}{4}, \frac{35}{128}, \frac{5}{32} \right) \\
\rightsquigarrow &
\left( \frac{41}{128}, \frac{1}{4}, \frac{35}{128}, \frac{5}{32} \right) \cdot A'' & = &
\left( \frac{137}{512}, \frac{155}{512}, \frac{35}{128}, \frac{5}{32} \right) \\
\rightsquigarrow &
\left( \frac{137}{512}, \frac{155}{512}, \frac{35}{128}, \frac{5}{32} \right) \cdot A'' & = &
\left( \frac{137}{512}, \frac{539}{2048}, \frac{1201}{4096}, \frac{721}{4096} \right) \\
\rightsquigarrow &
\left( \frac{137}{512}, \frac{539}{2048}, \frac{1201}{4096}, \frac{721}{4096} \right) \cdot A'' & = &
\left( \frac{4627}{16384}, \frac{539}{2048}, \frac{1201}{4096}, \frac{2641}{16384} \right) \\
\rightsquigarrow &
\left( \frac{4627}{16384}, \frac{539}{2048}, \frac{1201}{4096}, \frac{2641}{16384} \right) \cdot A'' & = &
\left( \frac{4627}{16384}, \frac{17977}{65536}, \frac{18487}{65536}, \frac{2641}{16384} \right) \\
\rightsquigarrow &
\left( \frac{4627}{16384}, \frac{17977}{65536}, \frac{18487}{65536}, \frac{2641}{16384} \right) \cdot A'' & = &
\left( \frac{2235}{8192}, \frac{71883}{262144}, \frac{150033}{524288}, \frac{86649}{524288} \right) \\
\end{array}
$$
\end{enumerate}

%% ---------------------------------------------
%% 			AUFGABE 3
%% ---------------------------------------------
\subsection*{Aufgabe 3 \mdseries Prioritätswarteschlangen}

\begin{enumerate}[\bfseries a)]

\item Nennen Sie zwei Ihnen bekannte Implementierungen des abstrakten Datentyps Prioritätswarteschlange, und geben Sie die zugehörigen Laufzeiten an.

\begin{description}
\item[Binärheap] Laufzeiten: \\
Insert: $O(\log n)$, \\
Extract-min: $O(\log n)$, \\
Decrease-key: $O(\log n)$
\item[AVL-Baum] Laufzeiten: \\
Insert: $O(\log n)$, \\
Extract-min: $O(\log n)$, \\
Decrease-key: $O(\log n)$
\end{description}


\item Zeigen Sie, wie man mit Hilfe einer Prioritätswarteschlange eine Folge von $n$ Elementen aus einem total geordneten Universum sortieren kann.

Der folgende Algorithmus nutzt eine Prioritätswarteschlange $Q$ um eine Folge $a_1, ..., a_n$ von $n$ Elementen zu sortieren:

\begin{lstlisting}[language=Pascal]
for i from 1 to n do
	Q.insert(a[i])
end for
for i from 1 to n do
	a[i] <- Q.extract-min()
end while
return a
\end{lstlisting}

Am Ende steht in \texttt{a} die sortierte Liste der Elemente. Der Algorithmus ist korrekt, weil wir bei jedem Aufruf von \texttt{extract-min} das jeweils kleinste Element, das noch in der Prioritätswarteschlange enthalten ist, nacheinander in das Array hinzufügen.

\item Wie Sie wissen, benötigt jeder vergleichsbasierte Sortieralgorithmus mindestens $\Omega(n \log n)$ Operationen. In Anbetracht von (b), was besagt dies über die Laufzeit jeder vergleichsbasierten Implementierung einer Prioritätswarteschlange?
Kann amortisierte Analyse hier helfen?
\\ \\
Da wir zum Sortieren $n$ mal \texttt{insert} und $n$ mal \texttt{extract-min} ausführen, folgt daraus, dass mindestens eine der beiden Operationen $\Omega(\log n)$ Zeit benötigt.\\
Amortisierte Analyse bringt hier keine Laufzeitverbesserung, da $\Omega(\log n)$ untere Schranke für jede Ausführung einer der beiden Operationen ist. Somit kann man keine Kosten umverteilen.
\end{enumerate}

\label{LastPage}
\end{document}
