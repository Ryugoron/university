\documentclass[11pt,a4paper,ngerman]{article}
\usepackage[bottom=2.5cm,top=2.5cm]{geometry} 
\usepackage{babel}
\usepackage[utf8]{inputenc} 
\usepackage[T1]{fontenc} 
\usepackage{ae} 
\usepackage{amssymb} 
\usepackage{amsmath} 
\usepackage{graphicx}
\usepackage{fancyhdr}
\usepackage{fancyref}
\usepackage{listings}
\usepackage{xcolor}
\usepackage{paralist}

\usepackage[pdftex, bookmarks=false, pdfstartview={FitH}, linkbordercolor=white]{hyperref}
\usepackage{fancyhdr}
\pagestyle{fancy}
\fancyhead[C]{Höhere Algorithmik}
\fancyhead[L]{Übung Nr. 2}
\fancyhead[R]{WS 2011/12}
\fancyfoot{}
\fancyfoot[L]{}
\fancyfoot[C]{\thepage / \pageref{LastPage}}
\renewcommand{\footrulewidth}{0.5pt}
\renewcommand{\headrulewidth}{0.5pt}
\setlength{\parindent}{0pt} 
\setlength{\headheight}{15pt}

\author{Tutor: Lena Schlipf}
\date{}
\title{Max Wisniewski , Alexander Steen}

\begin{document}

\lstset{language=Java, basicstyle=\ttfamily\fontsize{10pt}{10pt}\selectfont\upshape, commentstyle=\rmfamily\slshape, keywordstyle=\rmfamily\bfseries, breaklines=true, frame=single, xleftmargin=3mm, xrightmargin=3mm, tabsize=2}

\maketitle
\thispagestyle{fancy}

%% -----------------------------------------------------
%%                         AUFGABE 1
%% -----------------------------------------------------

\section*{Aufgabe 1}

Sei $S = \{ s_1, s_2, ..., s_n \} $ eine Menge von $n$ paarweise verschiedenen Elementen aus einem totalgeordneten Universum. Seien $w_1, w_2, ..., w_n$ positive Gewichte, so dass das Element $s_i$ Gewicht $w_i$ hat. Es gelte $\overset{n}{\underset{i=1}{\sum}} w_i = 1$. Gesucht ist der \emph{gewichtete Median} von $S$.

\begin{enumerate}[\bfseries (a)]
%% -----------------------------------------------
%%                         a)
%% -----------------------------------------------

\item Angenommen, Sie haben eine Funktion, welche den gewichteten Median in Zeit $T(n)$ bestimmt. Zeigen Sie, wie man den (normalen) Median in Zeit $O(n) + T(n)$ berechnen kann.\\

\textbf{Beweis:} Als Eingabe für unseren Algorithmus (Median) bekommen wir eine Liste von Werten ($s_1, ..., s_n$). Diese erweitern wir nun auf die Form für den gewichteten Median. Dazu legen wir eine 2. Liste mit den Werten an (oder erweitern die andere Liste auf Tupel $(s_i, w_i)$, was beides auf das selbe hinaus läuft). Für die Gewichte wählen wir $\forall 1\leq i \leq n : w_i = \frac{1}{n}$. Auf die neue Strucktur können wir den Algorithmus für den \emph{gewichteten Median} anwenden.\\

\textbf{Laufzeit:} Wir konstruieren zunächst die neue Liste. Dafür gehen wir einmal drüber und erzeugen Tupel. Das Gewicht können wir in $O(1)$ erstellen und den Tupel auch. Dies tun wir für $n$ Elemente. Je nachdem, ob wir $n$ besitzen oder noch suchen müssen, haben wir als Laufzeit $n$ oder $2n$. Wir benötigen aber für das Erstellen der neuen Eingabe $O(n)$ Schritte. Danach führen wir den gegebenen Algorithmus aus.\\
$\Rightarrow T_{\text{Median}}(n) = T_{\text{gewichteter Median}}(n) + O(n)$\\

\textbf{Median:} Bleibt zu zeigen, dass das Ergebnis $s_g$ der Median ist:\\
Kleinere Elemente, sei $n_k$ die Anzahl der Elemente, die kleiner als $s_g$ sind:
$$
\sum_{s_i < s_g} w_i \stackrel{\text{Gewichte}}{=} \sum_{s_i< s_g} \frac{1}{n} \stackrel{\text{Def.: }n_k}{=} \frac{n_k}{n} \stackrel{\text{Def.}}{\leq} \frac{1}{2} 
$$
$$
\Leftrightarrow n_k <= \frac{n}{2}
$$
Kleinere Elemente, sei $n_g$ die Anzahl der Elemente, die größer sind als $s_g$ sind:
$$
\sum_{s_i > s_g} w_i \stackrel{\text{Gewichte}}{=} \sum_{s_i> s_g} \frac{1}{n} \stackrel{\text{Def.: }n_g}{=} \frac{n_k}{n} \stackrel{\text{Def.}}{<} \frac{1}{2}
$$
$$
\Leftrightarrow n_g < \frac{n}{2}
$$

Da gelten muss, das $n_g + n_k + 1 = n$ erfüllen unsere beiden Ergebnisse die Bedingungen für den normalen Median.
%% -----------------------------------------------
%%                         b)
%% -----------------------------------------------
\item  Zeigen Sie, wie man den gewichteten Median in $O(n \cdot \log n)$ Zeit berechnen kann.\\

\textbf{Beweis:} Wir wählne uns einen geeigneten Sortieralgorithmus. Hier bietet sich Mergesort an. Aus dem letzten Übungszettel wissen wir, dass $T_{\text{Mergesort}}(n) = n \cdot \log n - (n - 1)$ ist.\\
Nachdem wir unseren Liste sortiert haben tun wir folgendes:
\begin{enumerate}[{Schritt} 1:]
\item Erzeuge $sum = 0$ Laufsummenvariable mit 0 und indexvariable $i = 0$
\item Rechne $sum \leftarrow sum + w_i$.
\item Ist $sum <= \frac{1}{2}$ inkrementiere $i$ um eins und mache mit Schritt 2 weiter.
\item (Ist dies nicht der Fall) Gebe $s_i$ zurück.
\end{enumerate}

Was wir als erstes sehen, ist dass die Summer der Gewichte, die kleiner als der \emph{gewichtete Median} sind, $\leq \frac{1}{2}$ da wir beim ersten mal, wenn es größer ist aus der Schleife gehen und dort nur den Media drauf gerechnet haben. Da wir beim finden über $\frac{1}{2}$ gekommen sind, wissen wir, dass aus der Bedingung, dass die Gesammtgewichte $=1$ sind, dass die Gewichte der Elemente, die größer sind $\geq \frac{1}{2}$ sein müssen.\\

\textbf{Laufzeit:} Die erste Überlegung ist, ob die Schleife abbricht.\\
Eine einfache Antwort wäre, dass wir alle Gewichte aufsummieren, d.h. wir wissen im $n$-ten Schritt, dass $sum = 1$ ist. Die Abbruchbedingung wird also spätestens nach $n$ Schritten erfüllt. Da wir aber äußerste Beweisfetischisten sind, weden wir für die Terminierung noch einen formellen Beweis am Ende des Übungsblattes geben.\\
Für die Laufzeit ergibt sich nun : $T(n) = T_{\text{Mergesort}} + n \cdot (c) + d$, wobei $c$ die konstanten kosten sind, für das aufsummieren der Gewichte und inkrementieren des indexes sind und $d$ die Fixkosten für das anlegen der beiden Variablen sind.\\
$\Rightarrow T(n) = n \cdot \log n - (n-1) + n \cdot c + d \in O(n \log n)$.

%% -----------------------------------------------
%%                         c)
%% -----------------------------------------------
\item Zeigen Sie, wie man den gewichteten Median in $O(n)$ Zeit  finden kann, wenn ein Linearzeitalgorithmus zum Finden des normalen Medians zur Verfügung steht.

\end{enumerate}
%% -----------------------------------------------------
%%                         AUFGABE 2
%% -----------------------------------------------------

\section*{Aufgabe 2}

In der Vorlesung wurden zur Berechnung des BFPRT-Algorithmus die Menge in 5er Gruppen unterteilt. 

\begin{enumerate}[\bfseries (a)]

\item Was passiert bei 7er Blöcken.\\

\begin{description}

\item{\bfseries Splitter:} Als erstes prüfen wir, ob die Splittereigenschaft noch erfüllt ist (analog zur Untersuchung aus der VL).\\
Elemente, die kleiner sind:\\
Wir haben $n/7$ Gruppen aufgerunded. Davon wissen wir wieder, dass die Hälft Dieser Gruppen einen kleineren Median hat als unser gefundener Splitter. In jeder dieser Gruppen sind 4 Elemente kleiner als der jeweilige Median (und durch Transitivität kleiner als der Splitter). Dies gilt für alle Gruppen, bis auf die nicht voll besetzte (da können 3 Fehlen) und die Gruppe des Splitters (in der der Splitter fehlt)\\
$
\Rightarrow n_k =4 \cdot \left\lceil \frac{1}{2} \cdot \left\lceil \frac{n}{7}\right\rceil  \right\rceil - 4
$
Elemente, die größer sind:\\
Wir haben hier exakt die selbe Abschätzung, wie oben. $n/7$ Gruppen, von denen die Hälte größer ist, diese besitzen jeweils 4 Elemente, von denen wir die Größe relativ zum Splitter kennen. Davon gehen wieder 4 Elemente ab. Aus der letzten Gruppe und der Splitter selber.\\
$
\Rightarrow n_g = 4 \cdot \left\lceil \frac{1}{2} \cdot \left\lceil \frac{n}{7}\right\rceil  \right\rceil - 4
$
\end{description}

Nun können wir untersuchen, wie viele Elemente wir wegschmeißen können:\\
Wir schätzen das ganze erstmal nach unten ab:
$$
\begin{array}{crcl}
&4 \cdot \left\lceil \frac{1}{2} \cdot \left\lceil \frac{n}{7}\right\rceil  \right\rceil - 4 \geq 4 \cdot \frac{1}{2} \cdot \frac{n}{7} - 4 &\geq& \frac{1}{4}\\
\Leftrightarrow& \frac{2}{7} \cdot n & \geq& \frac{17}{4}\\
\Leftrightarrow &  n & \geq & \frac{17 \cdot 7}{4 \cdot 2}\\
\Rightarrow & n &\geq& 15
\end{array}
$$

\textbf{Beweis:} Nun bestimmen wir noch die Laufzeit.
Als erstes gilt $\Omega (n)$ als Komplexität des Problems (nicht des Algorithmus). Dies gilt, das wir jedes Element mindestens einmal anfassen müssen, um es zu prüfen. Sollten wir eine echt sublineare Laufzeit erreichen, würden wir einige Elemente nicht betrachten. Wir könnten also einen Fall konstruieren, in dem wir den Median nicht betrachten. Das Ergebnis dieses Algorithmus würde also nicht korrekt sein. $\Rightarrow \Omega (n)$ ist untere Schranke für die Komplexität des Problems.\\
Für die Beschrenkung nach oben, zeigen wir, dass immer noch gilt $T(n) \in O(n)$\\
Als Laufzeit können wir ersteinmal, da der Splitter immer noch seine Eigenschaften erfüllt, die Formel aus der VL nehmen:\\

$$
T(n) \leq \left\{
\begin{array}{lr}
O(1) & ,n < 50\\
O(n) + T\left( \left \lceil \frac{n}{7} \right \rceil \right) + T \left( \frac{3}{4} n \right) & , sonst
\end{array}
\right.
$$

Dies Beweisen wir mit einer Induktion über n.

\begin{description}

\item{\bfseries Behauptung:} $\exists \alpha > 0 \; : \;T(n) \leq \alpha \cdot n$

\item{\bfseries I.A.} Für $n< 50$

\item{\bfseries I.S.}

$$
\begin{array}{rcl}
T(n) &\leq& O(n) + T\left( \left \lceil \frac{n}{7} \right \rceil \right) + T \left( \frac{3}{4} n \right)\\
&\stackrel{I.V.}{\leq}& O(n) + \alpha \left \lceil \frac{n}{7} \right \rceil + \alpha \cdot \frac{3}{4} n\\
&\leq& c \cdot n + \alpha \frac{n}{7} + \alpha + \alpha \frac{3}{4} n \leq \alpha n
\end{array}
$$

Zeigen wir nun den letzten Schritt:

$$
\begin{array}{crcl}
& cn + \alpha \cdot \left( \frac{n}{7} + \frac{3}{4}n + 1 \right) &\leq& \alpha n\\
\Leftrightarrow & cn & \leq & \alpha \cdot \left( n - \frac{n}{7} - \frac{3}{4}n - 1 \right)\\
\Leftrightarrow& cn &\leq& \alpha \left( \frac{28n - 4n - 21n}{28} - 1 \right)
\end{array}
$$

\end{description}

\item Was passiert bei 3er Blöcken.\\

\end{enumerate}


%% -----------------------------------------------------
%%                         AUFGABE 3
%% -----------------------------------------------------

\section*{Aufgabe 3}

\begin{enumerate}[\bfseries (a)]
%% -----------------------------------------------
%%                         a)
%% -----------------------------------------------

\item Für zwei ganzzahlige Vektoren $x=(x_1; x_2; ...; x_n)$ und $y=(y_1; y_2; ...; y_n))$ mit $0\leq x_i, y_i \leq M$ und einen Wert $u > M$ betrachten wir die Zahlen
$$
a = x_1 u^n + x_2 u^{n-1}+ ... + x_n u^1
$$
und
$$
b=y_1u^n + y_2 u^{n-1} + ... + y_n u^1.
$$

Zeigen Sie: $a=b \Leftrightarrow x=y$ , $a<b \Leftrightarrow x < y$(lexikographisch).

%% -----------------------------------------------
%%                         b)
%% -----------------------------------------------

\item Entwerfen Sie einen Algorithmus, der für zwei gegebene Folgen nichtnegativer Zahlen $x=(x_1; ... ; x_m)$ und $y=(y_1; ...; y_n)$ in "linearer" Zeit entscheidet, ob $x$ als Teilfolge $(y_{i+1},...,y_{i+m})$ in $y$ vorkommt ($0 \leq i \leq n-m$). Berechnen Sie die Kosten des Algorithmus im EKM und im LKM.


\end{enumerate}
\label{LastPage}

\end{document}