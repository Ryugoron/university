\documentclass[11pt,a4paper,ngerman]{article}
\usepackage[bottom=2.5cm,top=2.5cm]{geometry} 
\usepackage{babel}
\usepackage[utf8]{inputenc} 
\usepackage[T1]{fontenc} 
\usepackage{ae} 
\usepackage{amssymb} 
\usepackage{amsmath} 
\usepackage{graphicx}
\usepackage{fancyhdr}
\usepackage{fancyref}
\usepackage{listings}
\usepackage{xcolor}
\usepackage{paralist}

\usepackage[pdftex, bookmarks=false, pdfstartview={FitH}, linkbordercolor=white]{hyperref}
\usepackage{fancyhdr}
\pagestyle{fancy}
\fancyhead[C]{Höhere Algorithmik}
\fancyhead[L]{Übung Nr. 2}
\fancyhead[R]{WS 2011/12}
\fancyfoot{}
\fancyfoot[L]{}
\fancyfoot[C]{\thepage / \pageref{LastPage}}
\renewcommand{\footrulewidth}{0.5pt}
\renewcommand{\headrulewidth}{0.5pt}
\setlength{\parindent}{0pt} 
\setlength{\headheight}{15pt}

\author{Tutor: Lena Schlipf}
\date{}
\title{Max Wisniewski , Alexander Steen}

\begin{document}

\lstset{language=Java, basicstyle=\ttfamily\fontsize{10pt}{10pt}\selectfont\upshape, commentstyle=\rmfamily\slshape, keywordstyle=\rmfamily\bfseries, breaklines=true, frame=single, xleftmargin=3mm, xrightmargin=3mm, tabsize=2}

\maketitle
\thispagestyle{fancy}

%% -----------------------------------------------------
%%                         AUFGABE 1
%% -----------------------------------------------------

\section*{Aufgabe 1}

Sei $S = \{ s_1, s_2, ..., s_n \} $ eine Menge von $n$ paarweise verschiedenen Elementen aus einem totalgeordneten Universum. Seien $w_1, w_2, ..., w_n$ positive Gewichte, so dass das Element $s_i$ Gewicht $w_i$ hat. Es gelte $\overset{n}{\underset{i=1}{\sum}} w_i = 1$. Gesucht ist der \emph{gewichtete Median} von $S$.

\begin{enumerate}[\bfseries (a)]
%% -----------------------------------------------
%%                         a)
%% -----------------------------------------------

\item Angenommen, Sie haben eine Funktion, welche den gewichteten Median in Zeit $T(n)$ bestimmt. Zeigen Sie, wie man den (normalen) Median in Zeit $O(n) + T(n)$ berechnen kann.

%% -----------------------------------------------
%%                         b)
%% -----------------------------------------------
\item  Zeigen Sie, wie man den gewichteten Median in $O(n \cdot \log n)$ Zeit berechnen kann.

%% -----------------------------------------------
%%                         c)
%% -----------------------------------------------
\item Zeigen Sie, wie man den gewichteten Median in $O(n)$ Zeit  finden kann, wenn ein Linearzeitalgorithmus zum Finden des normalen Medians zur Verfügung steht.

\end{enumerate}
%% -----------------------------------------------------
%%                         AUFGABE 2
%% -----------------------------------------------------

\section*{Aufgabe 2}

In der Vorlesung wurden zur Berechnung des BFPRT-Algorithmus die Menge in 5er Gruppen unterteilt. Untersuchen Sie, wie sich der Algorithmus für $k=2n+1,\; n \in \mathbb N $ verhält.


%% -----------------------------------------------------
%%                         AUFGABE 3
%% -----------------------------------------------------

\section*{Aufgabe 3}

\begin{enumerate}[\bfseries (a)]
%% -----------------------------------------------
%%                         a)
%% -----------------------------------------------

\item Für zwei ganzzahlige Vektoren $x=(x_1; x_2; ...; x_n)$ und $y=(y_1; y_2; ...; y_n))$ mit $0\leq x_i, y_i \leq M$ und einen Wert $u > M$ betrachten wir die Zahlen
$$
a = x_1 u^n + x_2 u^{n-1}+ ... + x_n u^1
$$
und
$$
b=y_1u^n + y_2 u^{n-1} + ... + y_n u^1.
$$

Zeigen Sie: $a=b \Leftrightarrow x=y$ , $a<b \Leftrightarrow x < y$(lexikographisch).

%% -----------------------------------------------
%%                         b)
%% -----------------------------------------------

\item Entwerfen Sie einen Algorithmus, der für zwei gegebene Folgen nichtnegativer Zahlen $x=(x_1; ... ; x_m)$ und $y=(y_1; ...; y_n)$ in "linearer" Zeit entscheidet, ob $x$ als Teilfolge $(y_{i+1},...,y_{i+m})$ in $y$ vorkommt ($0 \leq i \leq n-m$). Berechnen Sie die Kosten des Algorithmus im EKM und im LKM.


\end{enumerate}
\label{LastPage}

\end{document}