\documentclass[11pt,a4paper,ngerman]{article}
\usepackage[bottom=2.5cm,top=2.5cm]{geometry} 
\usepackage{babel}
\usepackage[utf8]{inputenc} 
\usepackage[T1]{fontenc} 
\usepackage{ae} 
\usepackage{amssymb} 
\usepackage{amsmath} 
\usepackage{graphicx}
\usepackage{fancyhdr}
\usepackage{fancyref}
\usepackage{listings}
\usepackage{xcolor}
\usepackage{paralist}

%%\usepackage[pdftex, bookmarks=false, pdfstartview={FitH}, linkbordercolor=white]{hyperref}
\usepackage{fancyhdr}
\pagestyle{fancy}
\fancyhead[C]{Höhere Algorithmik}
\fancyhead[L]{Übung Nr. 6}
\fancyhead[R]{WS 2011/12}
\fancyfoot{}
\fancyfoot[L]{}
\fancyfoot[C]{\thepage$\,$ von \pageref{LastPage}}
\renewcommand{\footrulewidth}{0.5pt}
\renewcommand{\headrulewidth}{0.5pt}
\setlength{\parindent}{0pt} 
\setlength{\headheight}{15pt}

\author{Tutor: Lena Schlipf}
\date{}
\title{Max Wisniewski , Alexander Steen}

\begin{document}

\lstset{language=Java, basicstyle=\ttfamily\fontsize{10pt}{10pt}\selectfont\upshape, commentstyle=\rmfamily\slshape, keywordstyle=\rmfamily\bfseries, breaklines=true, frame=single, xleftmargin=3mm, xrightmargin=3mm, tabsize=2, mathescape=true}

\maketitle
\thispagestyle{fancy}

%% ----------------------------------
%%		AUFGABE 1
%% ----------------------------------
\subsection*{Aufgabe 1 \mdseries Unabhängige Mengen in Pfaden}

Sei $G = \left( V ,\, E \right) $ ein \emph{Pfad}, mit den Knoten $v_1, v_2, ..., v_n$.\\
Dazu gehört eine Funktion $w \; : \; V \longrightarrow \mathbb{N}^*$, die einem Knoten ein positives ganzzahliges Gewicht zuordnet, im folgenden mit $w(v_i) = w_i$ bezeichnet.
\\

Zur Notation:\\
Wir werden im folgenden einen Pfad wie folgt darstellen:\\
$[w_0, w_1, w_2, w_3 , ... , w_{n-1}]$.\\
Dabei sind zwei Knoten benachbert, wenn diese nebeneinander in der Liste stehen. Die Knoten sind ihrer Nummerierung entsprechend aufsteigend angeordnet und in der Liste steht nur der Wert, die die Funktion $w$ ihm zuordnet.
\begin{enumerate}[\bfseries (a)]

%% -------------------------------
%%		a)
%% ------------------------------

\item Geben Sie ein Beispiel, dass der Algorithmus (siehe Aufgabenblatt) keine unabhängige Menge maximalen Gewichts liefert.\\

\textbf{Lösung:}\\
Betrachten wir den \emph{Pfad} $[1, 9 , 10 , 9 , 1]$.\\

Nach dem Algorithmus nehmen wir den Größten Wert, hier Knoten $2$ mit dem Wert $10$, und entfernen ihn und seine Nachbern (Knoten $1$ und $3$). Es verbleibt der Folgende Graph: $[1], [1]$. Der restliche Graph ist \emph{unabhängig}, bei weiterer Ausführung werden die beiden 1en genommen und nicht mehr gelöscht als diese beiden in zwei Schritten, da sie keine Nachbern mehr besitzen.\\

Der Algorithmus liefert uns also die Knoten $0,2,4$ mit dem gesammt Gewicht $12$.\\
Wie man an diesem Beispiel aber leicht sieht, wäre die maximale unabängige Menge allerdings $1,3$ mit dem Gewicht $18$. Diese beiden Knoten wurden aber als Nachbern eines wenig kleinere Knotens im 1 Schritt allerdings verworfen.\\

Da wir ein Gegenbeispiel gefunden haben, kann der Algorithmus nicht korrekt sein. \mbox{} \hfill $\square$

%% ------------------------------------
%%		b)
%% ------------------------------------

\item Geben Sie ein Beispiel, bei dem der Algorithmus (vgl. Aufgabenzettel) keine unabhängige Menge maximalen Gewichts bestimmt.\\

\textbf{Lösung:}\\
Der Algorithmus zerlegt die Menge in zwei unabängeige Menge, die eine hat gerade Indizes, die andere ungerade.\\
Wir wollen uns eine Menge konstruieren, in der die maximal unabhängige Menge sowohl gerade, als auch ungerade Indizes enthält.\\

Wir untersuchen $[1,9,1,1,9,1]$.\\
Die Menge der Werte mit geraden Indizes ist hier $\{ 9, 1 , 1 \}$ und die mit ungeraden ist $\{ 1 , 1 , 9 \}$. Beide ergeben ein Gesammtgewicht von $11$.\\
Auch hier sehen wir schnell, dass Element $1,4$ ein Gesammtgewicht von $18$ hat und somit mehr Gewicht hat, als die beiden Menge, deren Maximum den größten Wert unter den unabhängigen Menge haben soll.\\

Da dies wie an diesem Beispiel gezeigt nicht immer der Fall ist, bestimmt der Algorithmus lieder nicht die unabhängige Menge mit maximalen Gewicht. \mbox{} \hfill $\square$

%% -----------------------------------------
%%		c)
%% -----------------------------------------
\item Geben Sie einen effizienten Algorithmus an, der eine unabhängige Menge maximalen Gewichtes in einem Pfad bestimmt. Analysieren Sie Laufzeit und Platzbedarf.\\

\textbf{Lösung:}\\

\begin{description}

\item{\bfseries Teillösungen:}\\
Als Teillösung wählen wir $E[j]$ := Maximum der \emph{unabhängigen Teilmengen}, wenn wir nur die ersten $j$ Knoten betrachten.

\item{\bfseries Rekursion:}
$$
\begin{array}{lrcl}
&E[0] &=& w(0)\\
&E[1] &=& max \{ w(0) , w(1) \}\\
\forall 2 \leq i \leq n \; : & E[i] &=& \max \{ E[j-2] + w(j) , \; E[j-1] \}
\end{array}
$$

Sollten wir nur 1 Element haben nehmen wir dies einfach. Haben wir 2 Nehmen wir das Maximum von beiden. Nehmen wir das Maximum von erstens, wir nehmen das j - te Element müssen dafür aber das daneben verwerfen, oder zweitens wir nehmen es nicht, aber können das daneben wählen.\\

Ziel ist es $E[n-1]$ zu bestimmen, weil wir dort alle Knoten betrachtet haben.

\item{\bfseries Korrektheit} 



\item{\bfseries Algorithmus}

\begin{lstlisting}[language=Pascal]
if n = 1 then
	Erg := [0];	// Die optimale Menge enthaelt nur 0
	return w(0); // und hat deren Wert
Erg1 := [0];	// Sonst speichern wir den Wert
w1 := w(0);	//  in der ersten Laufvariable
if(w(0) > w(1) then
	Erg2 := [0];	// Un den andern Anker
	w2 := w(0);	// in der zweiten
else
	Erg2 := [1];
	w2 := w(1);

//
// Wir gehen wie bei Fibonacci mit
// Akkumulatoren durch, dabei schieben
// wir immer das neure Element (2) in das
// aeltere (1) und errechnen das neue.
//
for i := 2 to n-1 do
	if w1 + w(i) > w2 then
		save := w1 +w(i);
		w1 := w2;
		w2 := save;
		ergSave := Erg2;
		Erg2 := Erg1 ++ [i];
		Erg1 := Erg2;
	else
		save := w1;
		w1 := w2;
		w2 := save;
		ergSave := Erg2;
		Erg2 := Erg1;
		Erg1 := ergSave;
Erg := Erg2;
return w2;
\end{lstlisting}

Da wir immer nur den letzte Wert (w2) und den vorletzten Wert (w1) für den nächsten Schirtt benötigen, müssen wir auch nur diese beiden speichern. Danach gehen wir wie beim dynamischen Programmieren unserere Parameter von klein nach groß durch.

\item{\bfseries Laufzeit \& Platz}\\
\textbf{Platzbedarf:} $\Theta (1)$, da wir nur 2 Werte und swap speichern müssen. Dies wird durch die Speicherung der liste allerdings relativiert, da wir nun noch maximal $\frac{n}{2}$ Elemente in Erg1, Erg2 mitführen.\\

\textbf{Laufzeit:} Wir betrachten jeden Knoten von $V$ genau einmal $\Rightarrow \Theta (n)$.

\end{description}

\end{enumerate}

%% ------------------------------------------
%%		AUFGABE 2
%% -----------------------------------------
\subsection*{Aufgabe 2 \mdseries Vorlesungsplanung}

In der Vorlesung haben wir einen \emph{Greedy - Algorithmus} gesehen, mit der wir aus einer Menge von Vorlesungen, eine Maximale Menge von Vorlesungen finden können, die alle paarweise \emph{verträglich} sind. Untersuchen Sie, ob der Algorithmus mit einer der folgenden Ordnungen im Sortierschritt immer noch funktionert. Geben Sie Pseudocode an und begründen Sie seine Korrektheit und Laufzeit.

\begin{enumerate}[\bfseries (a)]


%% -----------------------------------------
%%		a)
%% -----------------------------------------
\item Sortiere aufsteigend nach den Anfangszeiten\\

Dies ist kein valide Lösung. Nehmen wir das Beispiel\\
$R = \{ I_1, I_2, I_3 \}$, wobei $I_1 = (1, 100), \; I_2 = (2, 10), \; I_3 = (12 , 20)$.\\

Wie wir schnell sehen, ist das beste Ergebnis $I_2, I_3$, diese beiden sind \emph{verträglich}, da sie nach einander liegen. Der Algorithmus würde uns allerdings $I_1$ liefern, weil es am frühesten anfängt.\\
Danach würden alle unverträglichen verworfen und damit auch $I_2, I_3$.\\

Da wir ein Gegenbeispiel gefunden haben, kann der Algorithmus mit dieser Ordnung nicht korrekt sein.

%% -----------------------------------------
%%		b)
%% -----------------------------------------
\item Sortiere absteigend nach den Anfangszeiten.

Diese Sortierung gibt zusammen mit dem Algorithmus
\begin{lstlisting}[language=Pascal]
A := $\emptyset$
while R $\not= \emptyset$ do
	i := max a(i)
	A := $A \cup \{I_i \}$
	R := R $\setminus \{ x \; | \; e(x) > a(i)\}$
\end{lstlisting}

\textbf{Beweis:}\\
Sei $S$ eine optimale Teilmenge von $R$. Wir wollen zeigen, dass $| A | = | S |$ gilt.\\

Seien $i_1 , i_2 , ... , i_k$ und $j_1 , j_2 , ... , j_m$ die Indizes der Intervalle von $A$ bzw. von $S$, zeitlich (absteigend) geordneter Folgen.\\

Wir induzieren über die Länge der beiden Folgen.

\begin{description}

\item{\bfseries Beh.:}  $\forall r \leq m \; : \; a(i_r) \geq a(j_r)$

\item{\bfseries I.A.} $a(i_1) \geq a(j_1)$ ist trivialer weise erfüllt, da der Algorithmus im ersten Schritt das $i_1$ wählt, mit dem größten $a(i_1)$.

\item{\bfseries I.S.} $r \longrightarrow r + 1$\\
Nach dem Algorithmus, liegt $a(i_{r+1})$ noch in $R$, da nach Ordnung gilt $e(i_r)\geq a(i_r) \stackrel{I.V.}{\geq} a(j_{r}) \geq e(j_{r+1}) \geq a(j_{r+1})$.\\

Der neue Anfang von $a(i_{r+1})$ muss also im Intervall $(a(j_{r+1}) , a(j_{r}))$ (Wir haben hier an dieser Stelle da ganze umgedreht, weil Intervalle sorum aufgeschrieben werden)  liegen. Damit muss $a(i_{r+1}) \geq a(j_{r+1})$ sein. \mbox{} \hfill $\square$

\item{\bfseries Beweis der Optimalität:}\\
Angenommen $k < m$ dann können wir aus unserer oberen Behauptung ziehen, dass noch min ein Element in $R$ sein muss, nämlich $I_{k+1}$. Der Algortihmus kann also nicht terminiert sein.\\

Angenommen $k > m$, dann wäre $S$ nicht optimal gewesen.\\

$\Rightarrow A$ ist eine optimale Lösung. \mbox{} \hfill $\square$ 

\end{description} 


%% -----------------------------------------
%%		c)
%% -----------------------------------------
\item Sortiere absteigend nach den Endzeiten.\\

Dieser Fall verhält sich analog zu \emph{a)}.  Nehmen wir die selben Intervalle $R = \{ I_1, I_2, I_3 \}$, wobei $I_1 = (1, 100), \; I_2 = (2, 10), \; I_3 = (12 , 20)$.\\

Da $I_1$ auch das letzte Ende hat, wird der Algorithmus dieses Interval zu erst nehme und alle anderen verwerfen, da diese Überlappen. Optimal hier wäre wie gehabt $I_2, I_3$.\\

Wir haben wieder ein Gegenbeispiel gefunden, damit kann der Algorithmus nicht korrekt sein.

%% -----------------------------------------
%%		d)
%% -----------------------------------------
\item Sortiere aufsteigend nach der Länge\\

Diese Ordnung bildet keine valide Lösung.\\
Nehmen wir als Beispiel $R = \{ I_1 , I_2 , I_3\}$ mit $I_1 = (1,10)$ , $I_2 = (9,12)$ , $I_3 = (11,20)$.\\

Der Algorithmus würde uns als kürzestes Interval $I_2$ liefern. Damit fallen $I_1, I_3$ aus der Menge, weil sie nicht veträglich sind. Die maximale Anzahl von Intervallen ist aber $I_1, I_2$, da diese beiden \emph{verträglich} sind.


%% -----------------------------------------
%%		e)
%% -----------------------------------------
\item Sortiere absteigend nach der Länge

Dieser Fall verhält sich analog zu \emph{a), b)}. Nehmen wieder die selben Intervalle $R = \{ I_1, I_2, I_3 \}$, wobei $I_1 = (1, 100), \; I_2 = (2, 10), \; I_3 = (12 , 20)$.\\

Das längeste Intervall ist $I_1$. Da dieses $I_2, I_3$ überdeckt, werden die beiden aus der Menge geschmissen. Damit liefert der Algorithmus $I_1$. Das maximale Menge von \emph{verträglichen} Intervallen wäre allerdings $I_2, I_3$.\\

Wir haben ein Gegenbeispiel gefunden und damit ist der Algorithmus nicht korrekt.

\end{enumerate}


%% -----------------------------------------
%%		AUFGABE 3
%% -----------------------------------------
\subsection*{Aufgabe 3 \mdseries Autobahnfahrt}

Gegeben ist eine Menge von Tankstellen entlang eines Weges und ein Auto, dass $n$ Kilometer fahren kann.\\
Wir suchen nun eine Minimale Anzahl von Tankstellen, mit der Wir unser Ziel erreichen können (ohne mit einem leeren Tank liegen zu bleiben).\\

\textbf{Lösung:}\\

\textbf{Notation und Idee:}\\
Wir setzten hier einmal vorraus, dass die Tankstellen uns in einer Liste gegeben werden, die sie nach Abstand zum Startpunkt sortiert. Zwischen 2 Tankstellen möchten wir den Abstand dieser beiden Tankstellen haben.\\

Als Darstellung wählen wir eine Liste des Treibstoffverbrauches $S = [\delta_1 , \delta_2 , ... \delta_{m+1}]$, wobei $\delta_1$ der verbrauchte Treibstoff vom Startpunkt zur ersten Tankstelle ist, $\delta_2$ der verbrauchte Treibstoff zwischen der ersten und der zweiten Tankstelle usw.\\
$\delta_{n+1}$ ist nun der verbrauchte Treibstoff von der letzten Tankstelle zum Zielpunkt.\\

Für den Beweis konstruieren wir uns noch eine Funktion $w \; : \; Tankstelle \longrightarrow \mathbb{N}$, die den Abstand einer Tankstelle zum Start angibt.\\


Um das Ziel zu finden gehen wir nun die Tankstellen durch, bis wir nicht mehr genügend Sprit bis zur nächsten Tankstelle haben.

\textbf{Algorithmus:}
\begin{lstlisting}[language=Pascal]
stops := 0;
tank := n;
for i := 0 to m do
	if S[i] > n then
		error	// Die Strecke kann nicht befahren werden
	if S[i] > tank then
		stops++;
		tank := n;
	tank := tank - S[i];		
\end{lstlisting}

\textbf{Korrektheit:}\\

Sei $k$ eine optimale Anzahl von Stopps und $l$ die Anzahl von Stopps, die uns der Algorithmus zurückgibt. Nun seien $i_1 , i_2 , ..., i_k$ und $j_1 , j_2 , ... , j_l$ die Indizes der Tankstellen bei der $k$ bzw. $l$ getankt haben.\\

\begin{description}

\item{\bfseries Beh.:} $\forall r \leq k \; : \; w(i_r) \geq w(j_r)$

\item{\bfseries Bew.:} Induktion über $k$

\item{\bfseries I.A.} $w(i_1) \geq w(j_1)$ gilt, da unser Algorithmus ersteinmal soweit fährt, bis die nächste Etappe nicht mehr zu schaffen ist. Da die nächste Tankstelle vom Start also nicht zu erreichen ist, muss der optimale stopp von $k$ vorher tanken.

\item{\bfseries I.S.} $r \longrightarrow r + 1$\\
Nach I.V. gilt $w(i_r) \geq w(j_r)$. Der Algorithmus fährt mit dem Auto nun so weit, dass für den Weg bis zur nächsten Tankstelle der Sprit nicht mehr genügt.\\
Da die optimale Tour höchstens genau so weit ist, wie die Greedy - Tour, kann der Sprit maximal genau so weit reichen, wie der Sprit der Greedy-Tour.\\

Formal ist: $w(j_r) \leq w(i_r) \leq w(i_{r+1}) \leq w(t)$, wobei $t$ die Tankstelle direkt nach $i_{r+1}$ ist. Da der Sprit bis nach $t$ von$i_r$ nicht reicht, werden wir von $j_r$ nicht bis $t$ kommen. Deshalb muss der nächste Stop der optimalen Tour zwischen $w(j_r)$ und $w(i_r)$ liegen. Damit gilt: $w(j_{r+1} \leq w(i_{r+1})$

\item{\bfseries Beweis der Optimalität}\\
Ist $l > k$, wissen wir, dass die Greedy-Tour nach $k$ Stops mindestens genau so viel Strecke zurückgelegt hat, wie die optimale Tour. Der Weg zum Ziel sollte von dieser Tankstelle also aus möglich sein, deswegen würde der Algorithmus keinen weiteren Stop mehr machen. Dieser Fall ist also nicht möglich.\\

Ist $k > l$ wäre $k$ keine optimale Tour.\\

Damit kann nur $k =l$ gelten. Unser Algorithmus liefert also eine minimale Anzahl von Stops.
\end{description}

\textbf{Laufzeit:}\\

Sollte die Darstellung nicht sortiert sein, so müssen wir alle Tankstellen nach Abstand zum Startpunkt sortieren. Dies ist mit $\Omega (m \cdot \log m)$ nach unten beschränkt. Kann aber je nach Darstellung der Tankstellen beliebig schlecht werden.\\

Nehmen wir in, dass die Tankstellen, wie angenommen, schon sortiert nach Abstand zum Start sortiert sind. Nun können wir in $O(m)$ die Abstände zwischen den einzelnen Tankstellen bestimmen, damit den Treibstoffverbraucht und diesen in die Liste $S$ eintragen.\\

Der angegebene Algorithmus läuft nun in $\Theta (m)$ durch, da wir für jede Tankstelle entscheiden, ob wir den nächsten Weg noch schaffen. Da wir 2 Vergleiche haben und maximal 2 Operationen in der Schleife.

\label{LastPage}
\end{document}
