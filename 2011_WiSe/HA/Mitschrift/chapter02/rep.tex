\chapter{Repräsentant einer Menge}

Gegeben sei folgendes statistisches Problem:\\
Es seien $n$ Zahlen / Datensätze gegeben, wobei $n \ll 0$ gilt.\\
Gesucht ist ein Repräsentativer Wert für diese Menge.

\section{Naiver Ansatz}

\begin{description}

\item{Idee} Wir verwenden den Durchschnitt / Mittelwert.\\
Die Laufzeit ist einfach, da wir nur einmal über alle Datensätze müssen. Setzen wir dabei eine konstante Zeit für Addition und Division vorraus, ist die Laufzeit $O(n)$.

\item{Problem:} Der Mittelwert ist Anfällig für Außreißer und daher nicht sehr aussagekräftig.\\
Sind beispielsweise $n-1$ Werte zwischen $0$ und $10$ und ein $n$ter liegt bei $10.000.000$ so wird das ganze Ergebnis zu diesem Wert hin verfälscht.

\end{description}

Dieser Repräsentant ist leicht zu berechnen, aber nicht sehr schön.\\
Betrachten wir daher einen anderen Ansatz.

\section{K - SELECT}

\begin{description}

\item{\bfseries Def.:} Ein Element $s$ einer total geordneten Menge $S$ hat den Rang $k$ \\$:\Leftrightarrow \;$ es gibt genau $(k-1)$ Elemente in $S$, die kleiner sind als $s$.\\

Man schreibt dafür $\; rg(s)$.

\item{\bfseries Def.:} Sei $S$ total geordnet mit $n = |\,S\,|$ und $s\in S$.\\
$$s \text{ heißt Median} :\Leftrightarrow \; rg(s) = \left\lceil \frac{n+1}{2} \right\rceil .$$

\end{description}

\subsection{Das Problem}

Gegeben Sei $S, \; |S| = n$ paarweise verschiedene Zahlen.\\
Nun wollen wir den Median $s$ von $S$ möglichst effizient finden.

\subsection{Algorithmus I in $\Theta(n \cdot \log n)$}

Was die Laufzeit schon nahe legt, bedienen wir uns hier eines Sortieralgorithmuses.

\begin{enumerate}

\item Soritere $S$. z.\,B. mit Heap - Sort .\\
Benötigt $\Theta (n\cdot \log n)$ Schritte.

\item Gib das Element an der Stelle $\left\lceil \frac{n+1}{2} \right\rceil$ aus.\\
Benötigt $\Theta (1)$ Schritte.

\end{enumerate}

\textbf{Laufzeit: } $T(n) = \Theta (n \cdot \log n) + \Theta(1) = \Theta(n \cdot \log n)$.

\vspace{\baselineskip}

Da für (vergleichsbasiertes) Sortieren jede Lösung mit $\Omega (n \cdot \log n)$ beschränkt ist, kann eine Lösung für das Medianproblem die Sortierung verwendet nicht schneller sein. Bleibt zu untersuchen, ob der Median ähnlich schwer ist, oder ob es einen Algorithmus gibt, der das Problem schneller lösen kann.

\subsection{SELECT in O (n)}

Angenommen es existiert eine Funktion SPLITTER(S), welche uns ein Element $q\in S$ liefert, so dass gilt:\\
$$rg(q) \geq \left\lfloor \frac{1}{4} \; n \right\rfloor \quad \land \quad rg(q) \leq \left\lceil \frac{3}{4} \; n\right\rceil.$$

\begin{description}

\item{\bfseries Lemma:} Angenommen wir können SPLITTER ohne weitere Kosten benutzen. Dann können wir den Median in $O (n)$ Zeit berechnen.

\item{\bfseries Beweis:} Um diese Aussage zu beweisen lösen wir das allgemeinere Problem 
$$\text{SELECT}(k,S)$$
finde Element mit Rang $k$.
\vspace{\baselineskip}
Dieses Problem wird "Auswahlproblem" genannt.

\item{\bfseries Idee:} Nehme SPLITTER als PIVOT Element und teile die Menge der Daten daran auf.

\pagebreak

\item{\bfseries Pseudocode:}


\begin{lstlisting}
SELECT( k , S )
	IF |S|  <  100 THEN
		RETURN BRUTFORCE( k , S )  // z. B. Algorithmus I
	q $\leftarrow$  SPLITTER( S )
	$S_< \; \leftarrow$ { s $\in$ S | s < q }
	$S_> \; \leftarrow$ { s $\in$ S | s > q}
	IF $| \;S_<\; | \geq  k$ THEN
		RETURN  SELECT( k , $S_<$ )
	ELSE IF $| S_< | = k - 1$ THEN
		RETURN q
	ELSE
		RETURN SELECT( $k - |S_<| - 1 \;,\;  S_>$)
\end{lstlisting}

\item{\bfseries Laufzeitanalyse:} \\ Da $rg(q) \in \left[\left\lfloor \frac{1}{4}\;n\;\right\rfloor\, , \, \left\lceil \frac{3}{4} \; n \right\rceil \right]$ gilt $\left| S_< \right| \, , \, \left| S_> \right| \leq \frac{3}{4} \; n$.\\
\vspace{\baselineskip}
Also gilt:
$$T(n) \leq \left\{ \begin{array}{lr} O(1) & , n < 100 \\ O(n) + T\left( \frac{3}{4} \; n\right) & ,\text{sonst} \end{array}\right. $$

\begin{description}

\item{\bfseries Behauptung:} $$T(n) \in O(n)$$

\item{\bfseries Beweis:}
$$
\begin{array}{rl}
T(n) & \leq c\cdot n + T\left( \frac{3}{4} \;n \right) \\
& \leq c \cdot n + c\left( \frac{3}{4} \;n \right) + T \left( \right(\frac{3}{4}\left)^2 \;n \right) \\
& \leq c\cdot n \sum_{i=0}^{\infty} \left( \frac{3}{4}^i \right) + O(1)\\
& \leq (4c)\cdot n + O(1)\\
&= O(n)
\end{array}
$$
\mbox{}\hfill $\square$
\end{description}

\subsection{Implementierung von SPLITTER}

Damit k-SELECT die versprochene lineare Laufzeit erreicht, müssen wir uns als nächstes die Implementierung von Splitter ansehen. Da wir uns in jedem Schritt einen neuen Splitter besorgen, muss die Laufzeit sehr gering gehalten werden.

\subsubsection{Randomisierte Lösung}

Die erste Idee ist es, statt dem SPLITTER mit den gewünschten Eigenschaften einfach einen zufällig Gewählten zu nehmen. Wenn man diese Laufzeit berechnen wird man auch auf eine lineare kommen.\\
\vspace{\baselineskip}\\
Um dieses Problem zu lösen werden wir später Randomisierte Algorithmen betrachten und wie man Laufzeiten aus Erwartungswerten bestimmt.

\subsubsection{BFRPT - Algorithm}

Der Algorithmus wurde nach seinen Entdeckern Blum\footnote{Turing Award 1995}, Floyd\footnote{Turing Award 1978}, Pratt, Rivest\footnote{Turing Award 2002}, Tarijan\footnote{Turing Award 1986} benannt.\\
\vspace{\baselineskip}\\
Grundlegend funktioniert der Algorithmus folgender Maßen:\\
Man wählt zufällig eine Stichprobe $S' \subseteq S$ mit $\left| S' \right| = \left| \frac{n}{5} \right|$, so dass der Median von $S'$ ein guter Splitter von $S$ ist. Bestimme rekursiv den Media von $S'$.

\subsubsection{Wählen von $S'$}

Die Idee ist $S$ in 5er Gruppen zu unterteilen. Innerhalb dieser Gruppen können wir den Median in konstanter Zeit findet. Baue aus den Medianen der 5er Gruppen die Menge $S'$ und nimm deren Median.

\begin{description}

\item{\bfseries Lemma:} Der Median von $S'$ ist ein guter SPLITTER von $S$, wenn $n$ groß genug ist.

\item{\bfseries Anschaung:} HIER WIRD NOCH EIN BILD UND ERKLÄRUNG EINGEFÜGT!!

\item{\bfseries Beweis:}

Wir wollen prüfen, ob der Median $g$, den wir finden, wirklich SPLITTER Eigenschaften besitzt. Das heißt wir wollen wissen, ob min. $\frac{1}{4}$ kleiner und $\frac{1}{4}$ größer ist.\\
\vspace{\baselineskip}\\
Größer:\\
Es sind $\left\lceil \frac{1}{2} \left\lceil \frac{n}{5} \right\rceil \right\rceil$ Elemente aus $S'$ größer als $g$. Da alle Elemente aus $S'$ Mediane ihrer 5er Gruppen sind, wissen wir, das in jeder dieser Gruppen 3 Elemente größer sind als$g$. Dies gilt für alle Gruppen, außer die Gruppevon $g$ selber und die mögliche letzte Gruppe.\\
Dies führt zu $3 \cdot \left\lceil \frac{1}{2} \left\lceil \frac{n}{5} \right\rceil \right\rceil -3$\\
\vspace{\baselineskip}\\ 
Kleiner:\\
Es gibt ebenso$ \left\lceil \frac{1}{2} \left\lceil \frac{n}{5} \right\rceil \right \rceil$ Gruppen, deren Meridiane kleiner sind als $g$. In jeder dieser Gruppe, wissen wir von 3 Elementen die kleiner sind, bis auf die Gruppe von $g$ und die letzte Gruppe, die in diese Klasse fallen könnte.\\
Dies führt zu mindestens $3 \left\lceil \frac{1}{2} \left\lceil \frac{n}{5} \right\rceil \right\rceil -3$ Elementen, die kleiner sind als $q$.\\
\vspace{\baselineskip}\\
Zusammensetzen:\\
Es gilt $3 \left\lceil \frac{1}{2} \left\lceil \frac{n}{5} \right\rceil \right\rceil -3 \geq 3 \cdot \frac{1}{2} \cdot \frac{1}{5} n - 3 = \frac{3}{10}n -3$.\\
Für einen guten SPLITTER, muss die Anzahl der außerhalb liegenden Elemente (sowohl größer als auch kleiner) größer als $\frac{1}{4} n$ sein.\\
$$
\begin{array}{rcl}
\Rightarrow \frac{3}{10}n - 3 \geq \frac{1}{4} &\Leftrightarrow& \left( \frac{3}{10} -  \frac{1}{4} \right)n \geq 3\\
&\Leftrightarrow& n\geq 60
\end{array}
$$

Wir sehen hier, dass wir mit dem Verfahren garantiert einen guten SPLITTER finden, wenn wir mehr als 60 Elemente haben. Mit diesem Problem haben wir uns aber schon im Algorithmus beschäftigt. Dort haben wir gesagt, dass wir bei Listen bestimmter Größe das ganze Problem mit BRUTEFORCE lösen wollen. Damit werden die Listen in denen wir einen SPLITTER suchen immer garantiert 60 Elemente besitzen.\vspace{\baselineskip}\\
Nachdem wir nun wissen, dass wir mit diesem Verfahren einen SPLITTER erhalten, müssen wir prüfen, ob dieses Verfahren den Algorithmus asymptotisch langsamer macht oder ob wir bei einer lineraten Laufzeit bleiben.

\end{description} 

\end{description}

\subsubsection{Laufzeit K-SELECT mit BFRPT Algorithmus}

Betrachten wir nocheinmal, wie unser Algorithmus nun nach dem Einsetzen des SPLITTER Codes aussieht.

\begin{lstlisting}
SELECT (S, K)
	IF $|S| < 100$ THEN
		RETURN BRUTEFORCE(S , K)
	Unterteile S in 5er Gruppen
	S'$\leftarrow\{\text{Median jeder 5er Gruppe}\}$
	q $\leftarrow$ SELECT( S', $\left\lceil \frac{|S'| + 1}{2} \right\rceil$)
	$S_< \leftarrow \{ s \in S \; | \; s < q \}$
	$S_> \leftarrow \{ s \in S \; | \; s > q\}$
	IF $|S_<| \geq k$ THEN
		RETURN SELECT($S_<$, K)
	ELSE IF $|S_<| = K - 1$ THEN
		RETURN q
	ELSE
		RETURN SELECT($S_>, \;K - |S_<| - 1$)
\end{lstlisting}

Nun können wir aus dem Programm die Anzahl der Vergleiche ablesen:

$$
T(n) \leq \left\{ \begin{array}{lr} O(1) &, n<100\\ O(n) + T\left( \left\lceil \frac{n}{5}\right\rceil \right) + T\left( \frac{3}{4}n \right) & , \text{sonst}\end{array}\right.
$$

\begin{description}

\item{\bfseries Behauptung:} $T(n) = O(n)$

\item{\bfseries Beweis:} Induktion über n mit $\exists \alpha > 0 \; : \; T(n) \leq \alpha \cdot n$

\begin{description}

\item{\bfseries I.A.:} $n<100$ klar, da $T(n)$ konstant.

\item{\bfseries I.S.:} $n \rightarrow n+1$\\

$$
\begin{array}{rcl}
T(n) &\leq& c \cdot n + T \left( \left\lceil \frac{n}{5} \right\rceil \right) + T \left( \frac{3}{4} n \right)\\
&\stackrel{\text{I.A.}}{\leq}& c\cdot n + \alpha \left\lceil \frac{n}{5} \right\rceil + \alpha \left( \frac{3}{4} n\right)\\
&\leq& c \cdot n + \alpha \left( \frac{n}{5} +1\right) \alpha \left( \frac{3}{4} n \right)\\
&=& n \left( c + \frac{19}{20} \right) \alpha + \alpha\\
&\leq& \alpha \cdot n
\end{array}
$$

Den letzten Schritt darf jeder für sich selbst nachvollziehen.\\
So sehen wir, dass die Laufzeit immer noch lineare ist.

\end{description}

\mbox{} \hfill $\square$

\end{description}

\section{Bemerkung}

Was wollten wir an diesem Beispiel sehen?\\
Was wir erreicht haben, ist ein Algorithmus, der kurz, elegant und optimal ist. Um diesen zu gewinne mussten wir nicht-triviale Struckturen benutzen, auf die man nicht mehr so leicht kommt, wie auf Quicksort oder Mergesort.\\
Die Laufzeit des Algorithmus war nicht sofort offensichtlich und brauchte eine Analyse samt Beweis.\vspace{\baselineskip}\\
Mit solchen Algorithmen werden wir uns im folgenden in der Vorlesung beschäftigen. Ein jeder hat mindestens einen kleinen Kniff dabei.\\
Zu Bemerken bleibt, dass der oben genannte und analysierte Algorithmus zwar linear läuft, die Konstanten sind allerdings so hoch, das bis zu einer Listengröße von $2^{25}$ Quicksort und anschließendes nehmen des k-ten Elements schneller ist.