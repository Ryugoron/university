\chapter{Grundlegende Techniken}

\section{Lösen von Rekursionsgleichungen}

\subsection{Raten und Induktion}

Diesen Ansatz ist immer dann zu wählen, wenn einem die Laufzeit des Algorithmus sofort klar ist.\\

Das ganze führt natürlich nur zum Ziel, wenn die Induktion aufgeht. Man kann also prinzipiell ziemlich lange raten, bis man auf ein vernünftiges Ergebnis kommt.

\subsection{Auflösen und Muster erkennen}

Beim Auflösen, führen wir die Rekursion Schritt für Schritt aus. Sobald wir ein Muster beim Auflösen erkennen, können wir es bis auf Ebene $k$ hinschreiben und dann analysieren, für welches $k$ der Anker erreicht wird. Diesen kann man dann in die Gleichung einsetzen und wird dadurch die endgülitge Laufzeit erhalten.

\textbf{Beispiel:} Mergesort\\
Annahme: $n = 2^k$.
$$
T(n) = \left\{
\begin{array}{lr}
0 & , n = 1\\
2 T\left( \frac{n}{2} \right) + n - 1, \text{sonst}
\end{array}
\right.
$$

$$
\begin{array}{rcl}
T(n) &=& 2 T\left( \frac{n}{2} \right) + n - 1\\
&=& 2 \cdot \left( 2 T\left( \frac{n}{2^2} \right) + \frac{n}{2} - 1\right) + n - 1\\
&=& 2^2 T\left( \frac{n}{2^2} \right) 2n -2^1 - 2^0\\
&=& 2^3 T \left( \frac{n}{2^3} \right) 3n - 2^2 - 2^ 1 - 2^0\\
&& \text{Nach k Schritten}\\
$=$ 2^k T \left( \frac{n}{2^k} \right) +  kn - \overset{k-1}{\underset{i=0}{\sum}} 2^i\\
&& k = \log n \text{ (Beim Anker)}\\
&=& 2^{\log n} \cdot T(1) + n \cdot \log n - \frac{2^{\log n} - 1}{2 - 1}\\
&=& n \log n - n + 1
\end{array}
$$

\subsection{Rekursionbaum analysieren}

tbd

\subsection{Master Theorem}

tbd

\section{Divide \& Conquer}

\textbf{Def.:} Unterteile ein großes Problem in kleinere Probleme. Löse diese einzeln und setzte die Teilergebnisse zusammen.\\

\textbf{Bsp.:} Mergesort, Quicksort\\

Um Divide\&Conquer Algorithmen zu Analysieren muss man häufig Rekursionsgleichungen analysieren.

\subsection{Beispiel: Multiplizieren von Zahlen}

Gegeben sind zwei Zahlen $a,b \in \N$\\
Gescuht ist $ a \cdot b$

\subsubsection{Schulmethode}

\textbf{Beispiel:} Sei $a=1234$ und $b = 512$

$$
\begin{array}{cccccccc}
1 & 2 & 3 & 4 & \cdot & 5 & 1 & 2\\
\hline
&&&&2&4&6&8\\
&&&1&2&3&4&\\
&&6&1&7&0&&\\
\hline
&&6&3&1&8&0&8
\end{array}
$$

Sei $n := $ Anzahl der Ziffern (ist eine Zahl kürzer als die andere, füllen wir sie am Anfang mit 0en auf).\\
$\Rightarrow n^2$ Multiplikationen und $n^2$ Additionen $\Rightarrow \Theta (n^2)$\\

Dies gilt für jedes Zahlensystem, da sich die Anzahl der Ziffern mit den Systemen immer weiter verändern.\\

\subsubsection{Divide\&Conquer beim Multiplizieren}

Wir teilen $a,b$ in 2 kleinere Zahlen (weniger Ziffern) und lösen das ganze rekursiv. Wir nehmen jetzt einmal an, dass wir uns im Binärsystem bewegen.

$$
a = a_h \cdot 2^{\left\lceil \frac{n}{2} \right\rceil} + a_l
$$
$$
b = b_h \cdot 2^{\left\lceil \frac{n}{2} \right\rceil} + b_l
$$

$$
\Rightarrow \begin{array}{rcl}
a \cdot b &=& \left( a_h \cdot 2^{\left\lceil \frac{n}{2} \right\rceil} + a_l \right) \left( b_h \cdot 2^{\left\lceil \frac{n}{2} \right\rceil} + b_l \right)\\
&=& a_hb_h \cdot 2^{2 \left\lceil \frac{n}{2} \right\rceil} + a_hb_l \cdot 2^{\left\lceil \frac{n}{2} \right\rceil}+ a_lb_h \cdot 2^{\left\lceil \frac{n}{2} \right\rceil} + a_lb_l
\end{array}
$$
Das führt zu:
\begin{itemize}
\item 2 Multiplikationen mit $\frac{n}{2}$ Bits
\item 2 x Bitshifting
\item 3 Additionen
\end{itemize}
Anker bei 1,2,4 Bits (egal so lange eine Konstante gewählt wird).\\

\textbf{Laufzeit:}
$$
T(n) \leq \left\{ \begin{array}{lr}
O(1) &, n = O(1)\\
4 T(\frac{n}{2}) + O(n) &, sonst
\end{array} \right.
$$

Diese Formel wird an dieser Stelle nicht aufgelöst, da das Ergbnis mit $\Theta (n^2)$ nicht besser geworden ist.\\

\textbf{Optimierung:}\\

Unser Problem ist, das wir 4 rekursive Aufrufe haben. Doch durch einen genialen Einfall, können wir das ganze auf 3 reduzieren.\\

Betrachten wir einmal die Form von eben, aber wir streichen die Verschiebung durch $2^k$.

$$
\begin{array}{crcl}
&\left( a_h + a_l \right) \left( b_h + b_l \right) &=& a_hb_h + a_hb_l + a_lb_j + a_lb_l\\
\Leftrightarrow& a_hb_l + a_lb_j &=& \left( a_h + a_l \right) \left( b_h + b_l \right) - a_hb_h - a_lb_l
\end{array}
$$

Hier sieht man, dass wir mit 3 Gleichungen auf unsere 4 Terme kommen können. Das bedeutet für unsere Rekursion:

$$
T(n) \leq \left\{ \begin{array}{lr}
O(1) &, n \leq 3\\
4 T(\frac{n}{2}) + O(n) &, sonst
\end{array} \right.
$$

\textbf{Lösen der Rekursionsgleichung}

$$
\begin{array}{rcl}
T(n) &\leq& 3 \cdot T\left( \frac{n}{2} \right) + cn\\
&\leq& 3 \cdot \left( 3 \cdot T \left( \frac{n}{2^2} \right) + c \cdot \frac{n}{2} \right) + cn\\
&& ... \\
&\leq& 3^k \cdot T\left( \frac{n}{2^k}\right) + cn \cdot \left( \overset{\log k}{\underset{i=0}{\sum}} \frac{3^i}{2^i}\right)\\
&& \text{Nach }k=\log n\text{ Schritten haben wir den Anker erreicht}\\
&=& 3^{\log n} \cdot O(1) + cn \cdot \left( \overset{\log k}{\underset{i=0}{\sum}} \frac{3^i}{2^i}\right)\\
&=& c\cdot \frac{\left( \frac{3}{2}\right)^{\log n + 1} - 1}{\frac{3}{2} - 1}\\
&=& 2cn\cdot \frac{3}{2} \cdot \left( \frac{3}{2} \right)^{\log n}\\
&=&3cn^{\log 3}
\end{array}
$$

Die Laufzeit ist also $T(n) \in O(n^{\log 3})$, wobei $n^{\log 3} \approx n^{1.58}$ ist, was weniger ist als $n^2$.