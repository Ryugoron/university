\section{Dynamisches Programmieren}

Das dynamische Programmieren wurde schon einmal ein ALP III eingeführt. Im weiteren Sinne handelt es sich um eine Rekursion, in der schon einmal berechnete Rekursionswerte gecached werden oder alle möglichen Ergebnisse von klein nach groß vorberechnet werden.

\subsection{Einführungsbeispiel: Einkaufsproblem}

Wir haben ein Guthaben von $1,60$ und wollen aus den folgenden Waren einkaufen, so dass der Wert dieser Waren sich für uns maximiert.\\

\begin{tabular}{cccc}
Anzahl & Name & Preis & Wert\\
\hline
1x & Apfel & $0,40$\euro{} & 3\\
1x & Brötchen & $0,60$\euro{} & 9\\
1x & Buttermilch & $1,00$\euro{} & 5\\
1x & Gummibärchenl & $0,80$\euro{} & 10\\
1x & Bifi & $0,60$\euro{} & 6\\
\end{tabular}