\documentclass[11pt,a4paper,ngerman]{article}
\usepackage[bottom=2.5cm,top=2.5cm]{geometry} 
\usepackage{babel}
\usepackage[utf8]{inputenc} 
\usepackage[T1]{fontenc} 
\usepackage{ae} 
\usepackage{amssymb} 
\usepackage{amsmath} 
\usepackage{graphicx}
\usepackage{fancyhdr}
\usepackage{fancyref}
\usepackage{listings}
\usepackage{xcolor}
\usepackage{paralist}
\usepackage{fancyhdr}
\usepackage{subfigure}
\pagestyle{fancy}
\fancyhead[C]{Mikroprozessorpraktikum}
\fancyhead[L]{Protokoll 5}
\fancyhead[R]{WS 2011/12}
\fancyfoot{}
\fancyfoot[L]{}
\fancyfoot[C]{\thepage / \pageref{LastPage}}
\renewcommand{\footrulewidth}{0.5pt}
\renewcommand{\headrulewidth}{0.5pt}
\setlength{\parindent}{0pt} 
\setlength{\headheight}{15pt}


\author{Teilnehmer:\\ \\Marco Träger, Matr. 4130515\\Alexander Steen, Matr. 4357549}
\date{Gruppe: Freitag, Arbeitsplatz: HWP 1}
\title{Mikroprozessorpraktikum WS 2011/12\\ Aufgabenkomplex: 5}

\begin{document}

\lstset{language=c, basicstyle=\ttfamily\fontsize{9pt}{9pt}\selectfont\upshape, commentstyle=\rmfamily\slshape, keywordstyle=\rmfamily\bfseries, breaklines=true, frame=single, xleftmargin=3mm, xrightmargin=3mm, tabsize=2}

\maketitle
\thispagestyle{fancy}
\newpage
\section*{A 5.1 LPM und Interrupt}

\begin{description}
	\item[A 5.1.1] Starten Sie den Controller und überprüfen Sie messtechnisch den Stromverbrauch und die MCLK-Taktfrequenz. \\
	
	7.36 $\cdot 10^6$ Hz, 5.83 mA
	
	\begin{lstlisting}[numbers=left]
	init511() {
		P5SEL |= (1 << 4);
		P5DIR |= (1 << 4);
	}
	\end{lstlisting}
	
	Fügen Sie in der while(1) Schleife der main() Funktion einen Befehl ein, der den Mikrocontroller in den LPM4 Mode setzt. Starten Sie das Programm. 
	Was bewirkt der Befehl? Wie verhalten sich die Taktfrequenz und der Stromverbrauch? \\
	
	\begin{lstlisting}[numbers=left]
	while(1) {
		LPM4
	}
	\end{lstlisting}
	
	0 Hz, 0.43 mA
	
	Programmieren Sie den Port1 in der Form, dass bei einem Druck auf die Taste (P1.0) ein Interrupt ausgelöst wird. Die while(1) Schleife bleibt unverändert, es 
	befindet sich nur die LPM4 Anweisung in der Schleife. Realisieren Sie in der ISR des PORT1 eine 10 Sekunden dauernde Warteschleife.
	Dokumentieren Sie die Beobachtungen (Stromverbrauch und Taktfrequenz) zum Zeitpunkt des Tastendruckes. 
	
	\begin{lstlisting}[numbers=left]
	#define waitingTillWeDontWegThatFlippsingThing 1000
	
	#pragma vector = PORT1_VECTOR
	__interrupt void PORT1 (void) {
		CLEAR(P1IFG, 0xFF);

		if((P1IN & 0x01)) {
			int i = 0;
			while(i < 20) {
				wait(50000);	// wait 0.5 seconds
				i++;
			}
		}
		wait(waitingTillWeDontWegThatFlippsingThing)
	}
	
	init511() {
		P5SEL |= (1 << 4);
		P5DIR |= (1 << 4);
		
		P1DIR &= ~(0x01);
		P1SEL &= ~(0x01);
		P1IE |= (0x01);
		P1IES &= ~(0x01);
	}
	\end{lstlisting}
	
	Leerlauf: 0 Hz, 0.44mA \\
	Taster: 7.38 $\cdot 10^6$ Hz, 4.08mA
\end{description}

\section*{A 5.1 Auto Shutdown mit einer ON/OFF Logik}
\begin{description}
	\item[A 5.1.2] Für die Umsetzung der oben beschriebenen Verhaltensweise, werden der Watchdog in einer Timer Anwendung und parallel dazu zwei Interruptquellen genutzt. Der Taster (P1.0) soll als Bedienelement genutzt werden. Die unten erwähnten Variablen Status und Tick sind globale Variablen. \\
	
	\begin{lstlisting}[numbers=left]
	
	int status = LPM4_bits;
	int tick = 0;
	
	\end{lstlisting}
\end{description}
	
\label{LastPage}
\end{document}