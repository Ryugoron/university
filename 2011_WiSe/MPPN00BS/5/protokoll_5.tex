\documentclass[11pt,a4paper,ngerman]{article}
\usepackage[bottom=2.5cm,top=2.5cm]{geometry} 
\usepackage{babel}
\usepackage[utf8]{inputenc} 
\usepackage[T1]{fontenc} 
\usepackage{ae} 
\usepackage{amssymb} 
\usepackage{amsmath} 
\usepackage{graphicx}
\usepackage{fancyhdr}
\usepackage{fancyref}
\usepackage{listings}
\usepackage{xcolor}
\usepackage{paralist}
\usepackage{fancyhdr}
\usepackage{subfigure}
\pagestyle{fancy}
\fancyhead[C]{Mikroprozessorpraktikum}
\fancyhead[L]{Protokoll 5}
\fancyhead[R]{WS 2011/12}
\fancyfoot{}
\fancyfoot[L]{}
\fancyfoot[C]{\thepage / \pageref{LastPage}}
\renewcommand{\footrulewidth}{0.5pt}
\renewcommand{\headrulewidth}{0.5pt}
\setlength{\parindent}{0pt} 
\setlength{\headheight}{15pt}


\author{Teilnehmer:\\ \\Marco Träger, Matr. 4130515\\Alexander Steen, Matr. 4357549}
\date{Gruppe: Freitag, Arbeitsplatz: HWP 1}
\title{Mikroprozessorpraktikum WS 2011/12\\ Aufgabenkomplex: 5}

\begin{document}

\lstset{language=c, basicstyle=\ttfamily\fontsize{9pt}{9pt}\selectfont\upshape, commentstyle=\rmfamily\slshape, keywordstyle=\rmfamily\bfseries, breaklines=true, frame=single, xleftmargin=3mm, xrightmargin=3mm, tabsize=2}

\maketitle
\thispagestyle{fancy}
\newpage
\section*{A 5.1 LPM und Interrupt}

\begin{description}
	\item[A 5.1.1] Starten Sie den Controller und überprüfen Sie messtechnisch den Stromverbrauch und die \texttt{MCLK}-Taktfrequenz. \\
	
	Ohne weitere Programme einzufügen, d.h. mit leerer \texttt{while(1)}-Schleife, wird nun der Stromverbrauch und die Taktfrequenz gemessen. Um letzteres zu messen, legen wir das Taktsignal mittels folgendem Code an den Pin \texttt{P5.4}:
	
	\begin{lstlisting}[numbers=left]
	init511() {
		P5SEL |= (1 << 4); //MCLK-Takt anlegen
		P5DIR |= (1 << 4); //Als Ausgang nutzen
	}
	\end{lstlisting}
	
	Diese Funktion wird vor der \texttt{while(1)}-Schleife aufgerufen und damit nur einmal ausgeführt (also wird damit die Messung nicht verfälscht). Eine genauere Erklärung des Codesegments findet sich in Protokoll 2, Sektion A 2.1. Der Stromverbrauch wird wie üblich mit dem Multimeter extern gemessen.\\
	Die Messungen ergeben für das oben erklärte Szenario einen Stromverbrauch von $5.83$ mA bei einer Taktfrequenz von ca. 7.36 $\cdot 10^6$ Hz $= 7.36$ MHz. \\
	
	Nun wird in die \texttt{while(1)}-Schleife der main-Funktion der Befehl \texttt{LPM4} eingefügt. Dieser Befehl bewirkt, dass der Mikrocontroller in den \texttt{LPM4}-Mode versetzt wird. Der Code sieht nun also folgendermaßen aus:
	
	\begin{lstlisting}[numbers=left]
	while(1) {
		LPM4
	}
	\end{lstlisting}
	
	Nach dem Start des Programmes werden nun, wie oben, Taktfrequenz und Stromverbrauch gemessen. Auffällig ist, dass die Messung eine Taktfrequenz von $0$ Hz ergibt, also im \texttt{LPM4}-Mode scheinbar kein Taktzyklus mehr ausgeführt wird. Der Stromverbrauch ist dabei um $5.4$ mA auf $0.43$ mA gesunken.\\

	
	Als nächstes fügen wir auf Port 1 eine ISR ein, die bei einem Tastendruck auf \texttt{P1.0} eine zehn Sekunden dauernde Warteschleife ausführt. Die \texttt{main}-Funktion bleibt dabei unverändert. Der Code, der diese Anforderungen realisiert, sieht folgendermaßen aus:
%%Habe hier wait(.) rausgenommen, wird wohl nicht benötigt.
	\begin{lstlisting}[numbers=left]
	#pragma vector = PORT1_VECTOR
	__interrupt void PORT1 (void) {
		CLEAR(P1IFG, 0xFF);

		if((P1IN & 0x01)) { //Taster gedrueckt
			int i = 0;
			while(i < 20) { // 20*0.5 Sekunden = 10 Sekunden
				wait(50000); // Warte 0.5 Sekunden
				++i;
			}
		}
	}
	
	init511() {
		// Initialisierung zur Taktmessung
		P5SEL |= (1 << 4);
		P5DIR |= (1 << 4);
		// Taster-Initialisierung
		P1DIR &= ~(0x01);
		P1SEL &= ~(0x01);
		P1IE |= (0x01); //Interrupts anschalten
		P1IES &= ~(0x01);
	}
	\end{lstlisting}
	
	Die ISR führt eine einfache Schleife aus, in der pro Durchlauf $0.5$ Sekunden gewartet wird. Da die Schleife 20 mal durchläuft, erhalten wir eine insgesamte Wartezeit von zehn Sekunden. Der Parameter von \texttt{wait} wurde nicht direkt als zehn Sekunden gewählt, da wir sonst die zulässigen Bereich des Parameters überschreiten. Die Funktion \texttt{init511} setzt alle benötigten Register. \\
	Die Messung ergibt nach dem Start des Programmes eine Taktfrequenz von $0$ Hz bei einem Stromverbrauch von ca. $0.44$ mA. Dies deckt sich mit unseren vorigen Messungen. Bei einem Tastendruck erhöhen sich die Messergebnisse für einen Zeitraum von knapp zehn Sekunden auf eine Taktfrequenz von $7.38$ MHz bei einem Stromverbrauch von $4.08$ mA.
\end{description}

\section*{A 5.2 Auto Shutdown mit einer ON/OFF Logik}
\begin{description}
	\item[A 5.2.1] Für die Umsetzung der in der Aufgabe beschriebenen Verhaltensweise, werden (1) Watchdog-Timer samt zugehöriger ISR und (2) Taster-ISR genutzt. \\

	Die \texttt{main}-Funktion ruft im Folgenden die Funktion \texttt{aufgabe512} in einer Endlosschleife auf. Diese ist wie in der Aufgabe gefordert implementiert.
	
	\begin{lstlisting}[numbers=left]
// Die Variable status bezeichnet den Modus, in dem wir
// uns befinden. Dabei ist status = 1, falls wir im LPM4-Modus
// sind. Sonst ungleich 1.
int status = 1;
// Zaehlen der Ticks
int tick = 0;

void init512() {
	// Initialisieren aller noetigen Register,
	// alle LEDs aus.
	LED1ON;LED2ON;LED3ON;LEDON;
	P5SEL |= (1 << 4);
	P5DIR |= (1 << 4);
	// Taster
	P1DIR &= ~(0x01);
    P1SEL &= ~(0x01);
    P1IE |= (0x01);
    P1IES &= ~(0x01);
	// LEDs
    P4DIR |= (0x07);
    P4SEL &= ~(0x07);
    P4OUT |= 0x07;
}

void aufgabe512() {
	if (status == 1) { // LPM4-Modus
		LED2OFF; LED3OFF;
		LPM4;
	} else { // aktiver Modus
		LED2ON;
	}
	// Solange die Taste abfragen, wie sie
	// gedrueckt wird.
	while ((P1IN & 0x01)) {
		if (tick > 2) {
			// LPM4-Modus anfordern,
			// Schleife verlassen
			status = 1;
			break;
		}
	}
}
	\end{lstlisting}
	
	Die Funktion \texttt{init512} wird dabei, wie immer, vor dem Aufruf von \texttt{aufgabe512} ausgeführt. Die Funktion \texttt{aufgabe512} tut exakt das, was in der Aufgabe gefordert ist: Falls wir im \texttt{LPM4}-Modus sein sollen (gdw. \texttt{status == 1}), werden die LEDs ausgeschaltet und der Mikrocontroller in den \texttt{LPM4}-Modus versetzt. Ist dies nicht der Fall, so wird die LED \texttt{P4.1} eingeschaltet. Danach wird der Taster abgefragt und überprüft, ob dieser länger als zwei Ticks gedrückt wurde. Ist dies der Fall, so wird dies als Abschaltvorgang interpretiert und die \texttt{status}-Variable auf eins gesetzt um in nächsten Schleifenzyklus (der \texttt{main}-Funktion) den \texttt{LPM4}-Modus zu aktivieren. \\
	Die \texttt{tick}-Variable wird dabei sowohl von der ISR des Tasters als auch der ISR vom Watchdog-Timer beeinflusst:
	\begin{lstlisting}[numbers=left]
#pragma vector = PORT1_VECTOR
__interrupt void PORT1 (void) {
  P1IFG &= ~0xFF;

  if((P1IN & 0x01)) { //Taste gedrueckt
  	if (status == 1) { // Waren/Sind wir gerade in LPM4-Mode?
  		status = 0; //Mode auf aktiv setzen: aufwachen
  		
  		//Watchdog-Timer aktivieren, Ein-Sekunden-Timeout
  		WDTCTL = (WDTPW + WDTSSEL + WDTCNTCL);
  		// Interrupt auf Watchdog-Zimer-Intervall schalten
  		IE1 = WDTIE;
  	} else { //sind im aktiven Mode
  		// Es wurde etwas gemacht: tick zuruecksetzen
  		tick = 0;
  	}
  }
}

#pragma vector = WDT_VECTOR
__interrupt void watchdog (void) {
  IFG1 &= ~WDTIFG;
  // Sekunde ist vergangen: tick inkrementieren
  tick++;
  // Blink-LED toggeln
  P4OUT ^= 0x04;
  
  // Sind wir im aktivem Modus und seit mehr als
  // 60 Sekunden wurde die Taste nicht gedrueckt?
  if ((status == 0) && (tick > 60)) {
    // Dann LPM4-Mode anfordern
  	status = 1;
  	LED3OFF;
  }
}

	\end{lstlisting}
	
	Die ISR des Watchdog-Timers hat zwei Aufgaben: Sie simuliert eine Uhr, in dem sie jede Sekunde die \texttt{tick}-Variable erhöht. Außerdem wird überprüft, ob seit 60 Sekunden (60 ticks) keine Aktion (Tastendruck) ausgeführt wurde. Ist dies der Fall so wird angenommen, dass das Modul gerade nicht aktiv benutzt wird und der Wechsel in den \texttt{LPM4}-Mode eingeleitet.\\
	Die ISR des Tasters verhält sich, je nach Modus des Moduls, unterschiedlich: Wurde der Taster während des aktiven Modus gedrückt, so wird \texttt{tick} auf Null zurückgesetzt (um Aktivität kenntlich zu machen). Wird der Taster gedrückt, während man in \texttt{LPM4}-Modus ist, so wird das Modul aufgeweckt, status entsprechend gesetzt und der Watchdog-Timer eingeschaltet. \\
	Damit ist ein Mechanismus zum automatischen ausschalten nach Inaktivität bzw. zum Ein- und Ausschalten via Tastendruck implementiert.\\
	
	Der Test des Programmes zeigt korrektes Verhalten und bestätigt damit die Erklärungen. Auch Messungen der Taktfrequenz und damit Beobachtungen über das Ein- und Austretens aus dem \texttt{LPM4}-Mode, belegen die Funktion des Programmes.
\end{description}

\section*{A 5.3 Wake Up binär gesteuert}
\begin{description}
	\item[A 5.3.1] Unter Benutzung des Tasters als Sensor, soll ein Wake-Up-Mechanismus umgesetzt werden, der auf Zustandsänderungen des Sensors entsprechend reagiert. \\
	
	Das nachfolgende Programm setzt das gewünschte Verhalten um. Die Initialisierung der Register und das Erlauben von Interrupt passiert in der Funktion \texttt{init513}, die wie immer vor der Funktion \texttt{aufgabe513} aufgerufen wird. Die \texttt{main}-Funktion enthält dann eine Endlos-Schleife, in der \texttt{aufgabe513} aufgerufen wird.
	
	\begin{lstlisting}[numbers=left]
// Diese Variable codiert den aktuellen Status:
// status == 1 bedeutet, dass wir in den aktiven Modus wechseln
// status == 0 bedeutet, dass wir in den LPM4-Modus wechseln 
unsigned char status = 1;

void init513() {
	// Alle LEDs aus, Register initialisieren
	LED1OFF;LED2OFF;LED3OFF;LEDOFF;
	P5SEL |= (1 << 4);
	P5DIR |= (1 << 4);
	//Taster interruptfaehig
	P1DIR &= ~(0x01);
    P1SEL &= ~(0x01);
    P1IE |= (0x01);
	// LED-Register einrichten
    P4DIR |= (0x07);
    P4SEL &= ~(0x07);
    P4OUT |= 0x07;
	// Interrupt-Flanke auf low-hi
    P1IES &= ~(0x01); //LH
}

#pragma vector = PORT1_VECTOR
__interrupt void PORT1 (void) {
	P1IFG &= ~0xFF;

	if ((P1IES & 0x01) == 1) {//HL ist Interruptquelle
  		status = 1; //aktiver Mode
  		P1IES &= ~(0x01);  // LH  ist Interruptquelle
  		LPM4_EXIT; // Aus LPM4-Mode austreten
  	} else {
  		status = 0; //LPM anfordern
  		P1IES |= (0x01); //HL ist nun Interruptquelle
  	}
}

void aufgabe513() {
	if (status == 0) { //LPM angefordert
		LED1OFF;LED2OFF;LED3OFF;
		LPM4; 
	} else {  //aktiver Mode
		// Ampelsequenz
		P4OUT = ROT;
		wait(waittime);
		P4OUT = ROTGELB;
		wait(waittime);
		P4OUT = GRUEN;
		wait(waittime);
		P4OUT = GELB;
		wait(waittime);
		P4OUT = ROT;
		wait(waittime);
	}
}
	\end{lstlisting}
	
	Die ISR des Tasters überprüft, ob der Interrupt durch eine LH- oder HL-Flanke ausgelöst wurde und entscheidet entsprechend, ob in den aktiven Modus oder in den \texttt{LPM4}-Modus gewechselt werden soll. Die Funktion \texttt{aufgabe513} setzt dann entsprechend der Belegung von \texttt{status} entweder den \texttt{LPM4}-Modus um oder schaltet eine Ampelsequenz. Die Variablen \texttt{waittime} und die Makros der Ampelsequenz wurden dabei aus Protokoll 1 übernommen.\\
	Das Programm wurde erfolgreich getestet, das Wechseln in den Low-Power-Mode wurde durch Taktfrequenzmessungen bestätigt, ebenso liefen die Ampelsequenzen erfolgreich durch.
\end{description}
\label{LastPage}
\end{document}