\documentclass[11pt,a4paper,ngerman]{article}
\usepackage[bottom=2.5cm,top=2.5cm]{geometry} 
\usepackage{babel}
\usepackage[utf8]{inputenc} 
\usepackage[T1]{fontenc} 
\usepackage{ae} 
\usepackage{amssymb} 
\usepackage{amsmath} 
\usepackage{graphicx}
\usepackage{fancyhdr}
\usepackage{fancyref}
\usepackage{listings}
\usepackage{xcolor}
\usepackage{paralist}
\usepackage{fancyhdr}
\usepackage{subfigure}
%\usepackage{hyperref} % @added (traeger 28.02.12) @removed (steen 29.02.12)
\pagestyle{fancy}
\fancyhead[C]{Mikroprozessorpraktikum}
\fancyhead[L]{Protokoll 4}
\fancyhead[R]{WS 2011/12}
\fancyfoot{}
\fancyfoot[L]{}
\fancyfoot[C]{\thepage / \pageref{LastPage}}
\renewcommand{\footrulewidth}{0.5pt}
\renewcommand{\headrulewidth}{0.5pt}
\setlength{\parindent}{0pt} 
\setlength{\headheight}{15pt}


\author{Teilnehmer:\\ \\Marco Träger, Matr. 4130515\\Alexander Steen, Matr. 4357549}
\date{Gruppe: Freitag, Arbeitsplatz: HWP 1}
\title{Mikroprozessorpraktikum WS 2011/12\\ Aufgabenkomplex: 4}

\begin{document}

\lstset{language=c, basicstyle=\ttfamily\fontsize{9pt}{9pt}\selectfont\upshape, commentstyle=\rmfamily\slshape, keywordstyle=\rmfamily\bfseries, breaklines=true, frame=single, xleftmargin=3mm, xrightmargin=3mm, tabsize=2}

\maketitle
\thispagestyle{fancy}
\newpage
\section*{A 4.1 Taster als Interruptquelle}

\begin{description}
	\item[A 4.1.1] Beide Taster an Port1 (P1.0 und P1.1) sollen interruptfähig sein und auf eine LH-Flanke den Interrupt auslösen. Zunächst soll die Toggle-Funktionalität implementiert werden. \\

Zunächst definieren wir uns einige Variablen: \texttt{counter} wird im folgenden Code (weiter unten) die Anzahl der Tastendrücke am Taster P1.0 zählen. \texttt{waittime} wird ebenfalls weiter unten im Code genutzt, um in dem Interrupt mittels \texttt{wait()} eine Abarbeitungszeit zu simulieren; damit minimieren wir das Prellen der Tasterkontakte und verhindern, dass mehrfach in den Interrupt gesprungen wird.
\begin{lstlisting}[numbers=left]
int counter = 0;
int waittime = 15000;

void init411() {
	// LEDs vorbereiten
	P4DIR |= (0x07);
	P4SEL &= ~(0x07);
	P4OUT |= 0x07;
	
	// Taster vorbereiten
	P1DIR &= ~(0x03);
	P1SEL &= ~(0x03);
	// Interrupt einschalten 
	P1IE |= (0x03);
	// Auf LH-Flanke stellen
	P1IES &= ~(0x03);
}

void aufgabe411() {
	wait(50000);
	P4OUT ^= 0x04; //LED4.2 toggeln
}
\end{lstlisting}
Von der \texttt{main.c}-Datei wird die \texttt{init411}-Funktion vor der Endlosschleife und dem darin enthaltenen Funktionsaufruf von \texttt{aufgabe411} ausgeführt. Es werden zunächst alle notwendigen Einstellungen für die LEDs und die Taster vorgenommen. Danach wird in einer Endlosscheife die LED 4.2 getoggelt.

Als nächstes wird ein Interrupt für den Port 1 definiert. Die Vorgehensweise wurde aus der Datei \texttt{interrupt.c} aus dem Startprojekt übernommen.
\begin{lstlisting}[numbers=left,firstnumber=23]
#pragma vector = PORT1_VECTOR
__interrupt void PORT1 (void) {
	// P1.0 wurde gedrueckt
	if((P1IN & 0x01)) {
		counter++;
		// Zehnte Mal gedrueckt
		if(counter == 10) {
			counter = 0;
			P4OUT ^= 0x01; //toggle
		}
		wait(waittime);
	}
	// P1.1 wurde gedrueckt
	if((P1IN & 0x02)) {
		P4OUT ^= 0x02; //toggle
		wait(waittime);
	}
	
	// Interrupt-Flag zuruecksetzen
	CLEAR(P1IFG, 0xFF);
}
\end{lstlisting}
\newpage
In dem Interrupt wird überprüft, welche der beiden Taster gedrückt wurde und danach entschieden, ob entweder die LED 4.1 sofort oder die LED 4.0, falls der zehnte Druck in Folge, getoggelt wird. Zusätzlich wird in beiden Fällen mittels \texttt{wait(waittime)} ein wenig Arbeitslast simuliert, sodass ein Prellen der Tastkontakte vermieden wird. Am Ende der ISR wird die Interrupt-Flag zurückgesetzt und somit der Interrupt als abgearbeitet markiert.
\\
Um nun nur bei leuchtender LED 4.1 den Taster P1.0 interruptfähig zu schalten, wird zwischen Zeile 37 und 38 die Anweisung \texttt{P1IE  \^{}= 0x01;} eingefügt. Damit wird bei jedem Druck auf die Taste P1.1 ebenfalls die Interruptfägifkeit von P1.0 getoggelt. Da wir mit ausgeschalteten LEDs starten, müssen wir nur noch die Initialisierung von P1IE in Zeile 14 von \texttt{P1IE |= (0x03);} auf \texttt{P1IE |= (0x02);} ändern. Damit startet das Programm ohne Interruptfähigkeit für P1.0.
\end{description}

\section*{A 4.2 Totmannschaltung}
\begin{description}
	\item[A 4.2.1] Der Watchdog und ein Tasten-Interrupt sollen in dieser Aufgabe gleichzeitig genutzt werden. In der Technik kommt so etwas in Form einer sogenannten Totmannschaltung vor. Betätigt der Fahrer eines Fahrzeugs innerhalb eines festgelegten Zeitraumes nicht eine Taste, erfolgt eine Notbremsung. \\

Die Initialisierung erfolgt analog zu \texttt{A 4.1.1}. Allerdings wird hier nur Taster P1.0 interruptfähig geschaltet und zusätzlich der Watchdog eingerichtet. Aus Aufgabe \texttt{A 3.1.1} wissen wir, dass bei einem Takt von 32768 Hz auf Basis der \texttt{ACLK}-Taktquelle der Watchdog nach einer Sekunde auslöst, falls \texttt{WDTIS1} = 0 und \texttt{WDTIS0} = 0. Da wir vier Sekunden Zeit vor dem Watchdog-Interrupt haben wollen, müssen wir den Takt-Vorteiler im \texttt{BCSCTL1}-Register auf 4 stellen. Zusätzlich wird der Watchdog mit dem \texttt{WDTTMSEL}-Bit gestartet, damit wir den Watchdog-Timer nutzen. Die Funktion \texttt{aufgabe421} wird in einer Endlosschleife ausgeführt und toggelt die LED 4.0.

\begin{lstlisting}[numbers=left]
void init421() {
	// LEDs vorbereiten
	P4DIR |= (0x07);
	P4SEL &= ~(0x07);
	P4OUT |= 0x07; LEDs aus
	LEDOFF;
	
	// Watchdog einrichten
	BCSCTL1 |= DIVA1;	// Divisor /4
	WDTCTL = WDTPW + WDTTMSEL + WDTSSEL; //PW, Timer Interval, ACLK
	IE1 = WDTIE; // Watchdog interrupt enable
	
	LED2ON;
	
	// taster vorbereiten
	P1DIR &= ~(0x01);
	P1SEL &= ~(0x01);
	P1IE |= (0x01);
	P1IES &= ~(0x01);
}

void aufgabe421() {
	wait(50000);
	P4OUT ^= 0x01;
}
\end{lstlisting}
	
Des Weiteren brauchen wir nicht nur eine ISR auf dem Taster, sondern auch für den Watchdog-Timer-Interrupt. Das Schlüsselwort \texttt{WDT\_VECTOR} wurde der Interrupttabelle in der Datei \texttt{interrupt.c} entnommen.
\newpage
\begin{lstlisting}
#pragma vector = PORT1_VECTOR
__interrupt void PORT1 (void) {
	// Pin gedrueckt
	if((P1IN & 0x01)) {
		// Watchdog zuruecksetzen
		WDTCTL = WDTPW + WDTSSEL + WDTCNTCL;
		LED2OFF;
		wait(1500);
	}
	CLEAR(P1IFG, 0xFF);
}


#pragma vector = WDT_VECTOR
__interrupt void WDT (void) {
	if(P4OUT & 0x02) {	// LED4.1 ist aus
		P4OUT &= ~(0x02); // LED anschalten
	} else {
		//Watchdog ausschalten
		WDTCTL = WDTPW + WDTHOLD;
		IE1 &= ~WDTIE;
		// Ampelsequenz
		while(1) {
			P4OUT = ROT;
			wait(WAITTIME);
			P4OUT = ROTGELB;
			wait(WAITTIME);
			P4OUT = GRUEN;
			wait(WAITTIME);
			P4OUT = GELB;
			wait(WAITTIME);
			P4OUT = ROT;
			wait(WAITTIME);
		}
	}
	//WDTIFG is reset automatically by servicing the interrupt
}
\end{lstlisting}
Die ISR für den Taster setzt den Watchdog-Zähler zurück und schaltet die LED 4.1 aus. In der Watchdog-Timer-ISR wird anhand der LED 4.1 entschieden, was gemacht wird: Ist die LED aus, so wird sie eingeschaltet. Ist sie allerdings bereits eingeschaltet, wird der Watchdog ausgeschaltet und in eine endlose Ampelsequenz übergegangen.\\
Die groß geschriebenen Konstanten beziehen sich auf die Defintionen aus Aufgabe A 1.1.6.
\end{description}

\section*{A 4.3 Touchscreen}
\begin{description}
\item[A 4.3.1] Der Touchscreen soll als Taste genutzt werden. Unter Nutzung eines Interrupts soll auf einen Druck auf das Touchscreen die Anzahl der bisherigen Betätigungen angezeigt werden. Genauer:
\begin{itemize}
\item Ist während des Drucks des TS der Taster P1.0 gedrückt, soll der Zähler heruntergezählt werden 
\item Ist während des Drucks des TS der Taster P1.1 gedrückt, soll der Zähler hochgezählt werden
\item Ist während des Drucks des TS kein Taster gedrückt, soll die LED P4.1 getoggelt werden.
\end{itemize} 
Außerdem soll für die Dauer des Drucks auf P1.0 bzw. P1.1 die LED P4.0 bzw. P4.2 eingeschaltet sein.

Um die Anzahl der Betätigungen zu speichern, legen wir uns eine Zählvariable an. Hierbei soll auch negativ gezählt werden können. Für den Ausgabetext auf dem LCD wird ein Stringbuffer \texttt{ausgabe431} angelegt. \\
Die \text{init}-Funktion ist ein wenig größer geworden, da wir vorbereitend für die Abfrage des Touchevents einige Register setzen müssen. Außerdem werden die Register für die Taster, die LEDs und den Interrupt für den TS gesetzt. Genaueres ist im Code in der jeweiligen Zuweisung erklärt:

\begin{lstlisting}
int counter431 = 0;
char ausgabe431[4];

void init431() {
	// Taster init:
	P1DIR &= ~(0x03);
	P1SEL &= ~(0x03);
	// LED init:
	P4DIR |= (0x07);
	P4SEL &= ~(0x07);
	P4OUT |= 0x07;
	// Touch init (siehe Beispielcode auf der Seite)
	TS_TIP_DIR_IN;
	TS_YP_DIR_IN; TS_YM_DIR_IN;
	TS_XP_DIR_IN; TS_XM_DIR_IN; 
	TS_TIP_1; // YP Y-Achse wird ueber einen PullUp Widerstand auf 1 gezogen
	TS_XM_0; // XM X-Achse wird auf 0 gesetzt
	// Die Ausgaenge jetzt freigeben
	TS_TIP_DIR_OUT; // YP auf 1
	TS_XM_DIR_OUT;  // XM auf 0
	// Interrupts setzen
	P1IE |= ((0x01) << 6); // Interrupts fuer P1.6
	P1IES |= ((0x01) << 6); // HL-Flanke fuer P1.6
	// LCD clearen
	lcd_clear(WHITE);
	lcd_paint();
	// initial P4.1 an
	P4OUT &= ~(0x02);
}
\end{lstlisting}

Um während eines Tasterdrucks auf P1.0 bzw. P1.1 die jeweiligen LEDs anzuschalten, wird folgender Code in der \texttt{main}-loop ausgeführt:

\begin{lstlisting}
void aufgabe431() {
	//Polling von P1.0 und P1.1
	// Links gedrueckt?
	if (P1IN & 0x01) {
		P4OUT &= ~((0x01)<<2);
	} else {
		P4OUT |= 0x01<<2;
	} 
	// Rechts gedrueckt?
	if (P1IN & 0x02) {
		P4OUT &= ~(0x01);
	} else {
		P4OUT |= 0x01;
	}
}
\end{lstlisting}

Hier wird, wie üblich, je nach Tastendruck die richtige LED eingeschaltet bzw. ausgeschaltet, falls der Taster nicht mehr gedrückt wird.
\newpage
Die ISR für den Touchscreen fragt entsprechend nach einer ausgelösten HI-LO-Flanke, ob der linke bzw. der rechte Taster gedrückt ist und entscheidet damit, ob der Zähler hoch- oder runtergezählt wird, oder die LED P4.1 getoggelt werden soll (falls keine Taste gedrückt).
\begin{lstlisting}
#pragma vector = PORT1_VECTOR
__interrupt void touch (void) {
	if (P1IFG & 0x40) {
		// Touchscreen gedrueckt
		if (P1IN & 0x01) {
			//P1.0 gedrueckt, counter runter
  		--counter431;
		} else if (P1IN & 0x02) {
			//P1.1 gedrueckt, counter hoch
			++counter431;
		} else {
			//P4.1 toggeln
			P4OUT ^= (0x02);
		}
		sprintf(ausgabe431, "%i", counter431);
		lcd_clear(WHITE);
		lcd_string(BLACK, 10, 10, ausgabe431);
		lcd_paint();
	}
	CLEAR(P1IFG, 0xFF);
}
\end{lstlisting}

Die Funktionen zum Schreiben auf den Touchscreen wurden dem Rahmenwerk entnommen und können nach Import von \texttt{HW\textunderscore LCD.h} genutzt werden.
Dabei löscht \texttt{lcd\textunderscore clear} den Displayinhalt, \texttt{lcd\textunderscore  string} schreibt eine Zeichenkette an eine bestimmte Position und \texttt{lcd\textunderscore paint} übernimmt die zuvor gemachten Änderungen.
\end{description}

\label{LastPage}
\end{document}
